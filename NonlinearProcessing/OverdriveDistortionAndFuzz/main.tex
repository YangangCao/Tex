\documentclass[10pt,a4paper,oneside]{article}
\usepackage[utf8]{inputenc}
\usepackage{amsmath}
\usepackage{amsfonts}
\usepackage{amssymb}
\usepackage{graphicx}
\usepackage{breqn}
\usepackage{pgfplots}
\usepackage{draftwatermark} % 设置水印
\SetWatermarkText{DNV Group} % 水印内容
\usepackage{tikz} % system block diagram
\usepackage{textcomp}
\usetikzlibrary{datavisualization}
\usetikzlibrary{shapes,arrows} % system block diagram
\usepackage{booktabs}
\usepackage[framed,numbered,autolinebreaks,useliterate]{mcode} % matlab code block
\author{Yangang Cao}

\date{March 1, 2019}
\newcommand{\degree}{^\circ}
\tikzset{
	delay/.style    = {draw, thick, rectangle, minimum height = 3em,
		minimum width = 3em},
	sum/.style      = {draw, circle, node distance = 2cm}, 
	prod/.style     = {draw, circle, node distance = 2cm},
	input/.style    = {coordinate}, % Input
	output/.style  = {coordinate} % Output
}
% Defining string as labels of certain blocks.
\newcommand{\product}{$\displaystyle \times$}
\newcommand{\delay}{\large$z^{-1}$}
\begin{document}

\title{Overdrive and Distortion}
\maketitle 
As pointed out in the section on valve simulation, the distorted electric guitar is a central part of rock music. In addition to the guitar amplifier as a major sound effect device, several stomp boxes (foot-operated pedals) have been used by guitar players for the creation of their typical guitar sound. Guitar heroes like Jimi Hendrix have made use of several small analog effect devices to achieve their unmistakable sound. Most of these effect devices have been used to create higher harmonics for the guitar sound in a faster way and at a much lower sound level compared to valve amplifiers. In this context terms like overdrive, distortion and fuzz are used. Several definitions of these terms for musical applications especially in the guitar player world are available. For our discussion we will define overdrive as a first state where a nearly linear audio effect device at low input levels is driven by higher input levels into the nonlinear region of its characteristic curve. The operating region is in the linear region as well as in the nonlinear region, with a smooth transition. The main sound characteristic is of course from the nonlinear part. Overdrive has a warm and smooth sound. The second state is termed distortion, where the effects device mainly operates in the nonlinear region of the characteristic curve and reaches the upper input level, where the output level is fixed to a maximum level. Distortion covers a wide tonal area starting beyond tube warmth to buzz saw effects. All metal and grunge sounds fall into this category. The operating status of fuzz is represented by a completely nonlinear behavior of the effect device with a sound characterized by the guitar player terms “harder” and “harsher” than distortion. The fuzz effect is generally used on single-note lead lines.

{\bfseries Overdrive.} For overdrive simulations a soft clipping of the input values has to be performed. One possible approach for a soft saturation nonlinearity is given by
\[
f(x)=\left\{\begin{array}{lll}{2 x} & {\text { for }} & {0 \leq x \leq 1 / 3,} \\ {\frac{3-(2-3 x)^{2}}{3}} & {\text { for }} & {1 / 3 \leq x \leq 2 / 3,} \\ {1} & {\text { for }} & {2 / 3 \leq x \leq 1.}\end{array}\right.
\]
The static input to output relation is shown in following figure.

% This file was created by matlab2tikz.
%
%The latest updates can be retrieved from
%  http://www.mathworks.com/matlabcentral/fileexchange/22022-matlab2tikz-matlab2tikz
%where you can also make suggestions and rate matlab2tikz.
%
\definecolor{mycolor1}{rgb}{0.00000,0.44700,0.74100}%
%
\begin{tikzpicture}

\begin{axis}[%
width=1.621in,
height=1.476in,
at={(0.703in,0.424in)},
scale only axis,
xmin=-1,
xmax=1,
xlabel style={font=\color{white!15!black}},
xlabel={x},
ymin=-1.1,
ymax=1.1,
ylabel style={font=\color{white!15!black}},
ylabel={y},
title style={font=\bfseries},
title={Static characteristic: y=f(x)},
xmajorgrids,
ymajorgrids
]
\addplot [color=mycolor1, forget plot]
  table[row sep=crcr]{%
-1	-1\\
-0.999	-1\\
-0.998	-1\\
-0.997	-1\\
-0.996	-1\\
-0.995	-1\\
-0.994	-1\\
-0.993	-1\\
-0.992	-1\\
-0.991	-1\\
-0.99	-1\\
-0.989	-1\\
-0.988	-1\\
-0.987	-1\\
-0.986	-1\\
-0.985	-1\\
-0.984	-1\\
-0.983	-1\\
-0.982	-1\\
-0.981	-1\\
-0.98	-1\\
-0.979	-1\\
-0.978	-1\\
-0.977	-1\\
-0.976	-1\\
-0.975	-1\\
-0.974	-1\\
-0.973	-1\\
-0.972	-1\\
-0.971	-1\\
-0.97	-1\\
-0.969	-1\\
-0.968	-1\\
-0.967	-1\\
-0.966	-1\\
-0.965	-1\\
-0.964	-1\\
-0.963	-1\\
-0.962	-1\\
-0.961	-1\\
-0.96	-1\\
-0.959	-1\\
-0.958	-1\\
-0.957	-1\\
-0.956	-1\\
-0.955	-1\\
-0.954	-1\\
-0.953	-1\\
-0.952	-1\\
-0.951	-1\\
-0.95	-1\\
-0.949	-1\\
-0.948	-1\\
-0.947	-1\\
-0.946	-1\\
-0.945	-1\\
-0.944	-1\\
-0.943	-1\\
-0.942	-1\\
-0.941	-1\\
-0.94	-1\\
-0.939	-1\\
-0.938	-1\\
-0.937	-1\\
-0.936	-1\\
-0.935	-1\\
-0.934	-1\\
-0.933	-1\\
-0.932	-1\\
-0.931	-1\\
-0.93	-1\\
-0.929	-1\\
-0.928	-1\\
-0.927	-1\\
-0.926	-1\\
-0.925	-1\\
-0.924	-1\\
-0.923	-1\\
-0.922	-1\\
-0.921	-1\\
-0.92	-1\\
-0.919	-1\\
-0.918	-1\\
-0.917	-1\\
-0.916	-1\\
-0.915	-1\\
-0.914	-1\\
-0.913	-1\\
-0.912	-1\\
-0.911	-1\\
-0.91	-1\\
-0.909	-1\\
-0.908	-1\\
-0.907	-1\\
-0.906	-1\\
-0.905	-1\\
-0.904	-1\\
-0.903	-1\\
-0.902	-1\\
-0.901	-1\\
-0.9	-1\\
-0.899	-1\\
-0.898	-1\\
-0.897	-1\\
-0.896	-1\\
-0.895	-1\\
-0.894	-1\\
-0.893	-1\\
-0.892	-1\\
-0.891	-1\\
-0.89	-1\\
-0.889	-1\\
-0.888	-1\\
-0.887	-1\\
-0.886	-1\\
-0.885	-1\\
-0.884	-1\\
-0.883	-1\\
-0.882	-1\\
-0.881	-1\\
-0.88	-1\\
-0.879	-1\\
-0.878	-1\\
-0.877	-1\\
-0.876	-1\\
-0.875	-1\\
-0.874	-1\\
-0.873	-1\\
-0.872	-1\\
-0.871	-1\\
-0.87	-1\\
-0.869	-1\\
-0.868	-1\\
-0.867	-1\\
-0.866	-1\\
-0.865	-1\\
-0.864	-1\\
-0.863	-1\\
-0.862	-1\\
-0.861	-1\\
-0.86	-1\\
-0.859	-1\\
-0.858	-1\\
-0.857	-1\\
-0.856	-1\\
-0.855	-1\\
-0.854	-1\\
-0.853	-1\\
-0.852	-1\\
-0.851	-1\\
-0.85	-1\\
-0.849	-1\\
-0.848	-1\\
-0.847	-1\\
-0.846	-1\\
-0.845	-1\\
-0.844	-1\\
-0.843	-1\\
-0.842	-1\\
-0.841	-1\\
-0.84	-1\\
-0.839	-1\\
-0.838	-1\\
-0.837	-1\\
-0.836	-1\\
-0.835	-1\\
-0.834	-1\\
-0.833	-1\\
-0.832	-1\\
-0.831	-1\\
-0.83	-1\\
-0.829	-1\\
-0.828	-1\\
-0.827	-1\\
-0.826	-1\\
-0.825	-1\\
-0.824	-1\\
-0.823	-1\\
-0.822	-1\\
-0.821	-1\\
-0.82	-1\\
-0.819	-1\\
-0.818	-1\\
-0.817	-1\\
-0.816	-1\\
-0.815	-1\\
-0.814	-1\\
-0.813	-1\\
-0.812	-1\\
-0.811	-1\\
-0.81	-1\\
-0.809	-1\\
-0.808	-1\\
-0.807	-1\\
-0.806	-1\\
-0.805	-1\\
-0.804	-1\\
-0.803	-1\\
-0.802	-1\\
-0.801	-1\\
-0.8	-1\\
-0.799	-1\\
-0.798	-1\\
-0.797	-1\\
-0.796	-1\\
-0.795	-1\\
-0.794	-1\\
-0.793	-1\\
-0.792	-1\\
-0.791	-1\\
-0.79	-1\\
-0.789	-1\\
-0.788	-1\\
-0.787	-1\\
-0.786	-1\\
-0.785	-1\\
-0.784	-1\\
-0.783	-1\\
-0.782	-1\\
-0.781	-1\\
-0.78	-1\\
-0.779	-1\\
-0.778	-1\\
-0.777	-1\\
-0.776	-1\\
-0.775	-1\\
-0.774	-1\\
-0.773	-1\\
-0.772	-1\\
-0.771	-1\\
-0.77	-1\\
-0.769	-1\\
-0.768	-1\\
-0.767	-1\\
-0.766	-1\\
-0.765	-1\\
-0.764	-1\\
-0.763	-1\\
-0.762	-1\\
-0.761	-1\\
-0.76	-1\\
-0.759	-1\\
-0.758	-1\\
-0.757	-1\\
-0.756	-1\\
-0.755	-1\\
-0.754	-1\\
-0.753	-1\\
-0.752	-1\\
-0.751	-1\\
-0.75	-1\\
-0.749	-1\\
-0.748	-1\\
-0.747	-1\\
-0.746	-1\\
-0.745	-1\\
-0.744	-1\\
-0.743	-1\\
-0.742	-1\\
-0.741	-1\\
-0.74	-1\\
-0.739	-1\\
-0.738	-1\\
-0.737	-1\\
-0.736	-1\\
-0.735	-1\\
-0.734	-1\\
-0.733	-1\\
-0.732	-1\\
-0.731	-1\\
-0.73	-1\\
-0.729	-1\\
-0.728	-1\\
-0.727	-1\\
-0.726	-1\\
-0.725	-1\\
-0.724	-1\\
-0.723	-1\\
-0.722	-1\\
-0.721	-1\\
-0.72	-1\\
-0.719	-1\\
-0.718	-1\\
-0.717	-1\\
-0.716	-1\\
-0.715	-1\\
-0.714	-1\\
-0.713	-1\\
-0.712	-1\\
-0.711	-1\\
-0.71	-1\\
-0.709	-1\\
-0.708	-1\\
-0.707	-1\\
-0.706	-1\\
-0.705	-1\\
-0.704	-1\\
-0.703	-1\\
-0.702	-1\\
-0.701	-1\\
-0.7	-1\\
-0.699	-1\\
-0.698	-1\\
-0.697	-1\\
-0.696	-1\\
-0.695	-1\\
-0.694	-1\\
-0.693	-1\\
-0.692	-1\\
-0.691	-1\\
-0.69	-1\\
-0.689	-1\\
-0.688	-1\\
-0.687	-1\\
-0.686	-1\\
-0.685	-1\\
-0.684	-1\\
-0.683	-1\\
-0.682	-1\\
-0.681	-1\\
-0.68	-1\\
-0.679	-1\\
-0.678	-1\\
-0.677	-1\\
-0.676	-1\\
-0.675	-1\\
-0.674	-1\\
-0.673	-1\\
-0.672	-1\\
-0.671	-1\\
-0.67	-1\\
-0.669	-1\\
-0.668	-1\\
-0.667	-1\\
-0.666	-0.999998666666667\\
-0.665	-0.999991666666667\\
-0.664	-0.999978666666667\\
-0.663	-0.999959666666667\\
-0.662	-0.999934666666667\\
-0.661	-0.999903666666667\\
-0.66	-0.999866666666667\\
-0.659	-0.999823666666667\\
-0.658	-0.999774666666667\\
-0.657	-0.999719666666667\\
-0.656	-0.999658666666667\\
-0.655	-0.999591666666667\\
-0.654	-0.999518666666667\\
-0.653	-0.999439666666667\\
-0.652	-0.999354666666667\\
-0.651	-0.999263666666667\\
-0.65	-0.999166666666667\\
-0.649	-0.999063666666667\\
-0.648	-0.998954666666667\\
-0.647	-0.998839666666667\\
-0.646	-0.998718666666667\\
-0.645	-0.998591666666667\\
-0.644	-0.998458666666667\\
-0.643	-0.998319666666667\\
-0.642	-0.998174666666667\\
-0.641	-0.998023666666667\\
-0.64	-0.997866666666667\\
-0.639	-0.997703666666667\\
-0.638	-0.997534666666667\\
-0.637	-0.997359666666667\\
-0.636	-0.997178666666667\\
-0.635	-0.996991666666667\\
-0.634	-0.996798666666667\\
-0.633	-0.996599666666667\\
-0.632	-0.996394666666667\\
-0.631	-0.996183666666667\\
-0.63	-0.995966666666667\\
-0.629	-0.995743666666667\\
-0.628	-0.995514666666667\\
-0.627	-0.995279666666667\\
-0.626	-0.995038666666667\\
-0.625	-0.994791666666667\\
-0.624	-0.994538666666667\\
-0.623	-0.994279666666667\\
-0.622	-0.994014666666667\\
-0.621	-0.993743666666667\\
-0.62	-0.993466666666667\\
-0.619	-0.993183666666667\\
-0.618	-0.992894666666667\\
-0.617	-0.992599666666667\\
-0.616	-0.992298666666667\\
-0.615	-0.991991666666667\\
-0.614	-0.991678666666667\\
-0.613	-0.991359666666667\\
-0.612	-0.991034666666667\\
-0.611	-0.990703666666667\\
-0.61	-0.990366666666667\\
-0.609	-0.990023666666667\\
-0.608	-0.989674666666667\\
-0.607	-0.989319666666667\\
-0.606	-0.988958666666667\\
-0.605	-0.988591666666667\\
-0.604	-0.988218666666667\\
-0.603	-0.987839666666667\\
-0.602	-0.987454666666667\\
-0.601	-0.987063666666667\\
-0.6	-0.986666666666667\\
-0.599	-0.986263666666667\\
-0.598	-0.985854666666667\\
-0.597	-0.985439666666667\\
-0.596	-0.985018666666667\\
-0.595	-0.984591666666667\\
-0.594	-0.984158666666667\\
-0.593	-0.983719666666667\\
-0.592	-0.983274666666667\\
-0.591	-0.982823666666667\\
-0.59	-0.982366666666667\\
-0.589	-0.981903666666667\\
-0.588	-0.981434666666667\\
-0.587	-0.980959666666667\\
-0.586	-0.980478666666667\\
-0.585	-0.979991666666667\\
-0.584	-0.979498666666667\\
-0.583	-0.978999666666667\\
-0.582	-0.978494666666667\\
-0.581	-0.977983666666667\\
-0.58	-0.977466666666667\\
-0.579	-0.976943666666667\\
-0.578	-0.976414666666667\\
-0.577	-0.975879666666667\\
-0.576	-0.975338666666667\\
-0.575	-0.974791666666667\\
-0.574	-0.974238666666667\\
-0.573	-0.973679666666667\\
-0.572	-0.973114666666667\\
-0.571	-0.972543666666667\\
-0.57	-0.971966666666667\\
-0.569	-0.971383666666667\\
-0.568	-0.970794666666667\\
-0.567	-0.970199666666667\\
-0.566	-0.969598666666667\\
-0.565	-0.968991666666667\\
-0.564	-0.968378666666667\\
-0.563	-0.967759666666667\\
-0.562	-0.967134666666667\\
-0.561	-0.966503666666667\\
-0.56	-0.965866666666667\\
-0.559	-0.965223666666667\\
-0.558	-0.964574666666667\\
-0.557	-0.963919666666667\\
-0.556	-0.963258666666667\\
-0.555	-0.962591666666667\\
-0.554	-0.961918666666667\\
-0.553	-0.961239666666667\\
-0.552	-0.960554666666667\\
-0.551	-0.959863666666667\\
-0.55	-0.959166666666667\\
-0.549	-0.958463666666667\\
-0.548	-0.957754666666667\\
-0.547	-0.957039666666667\\
-0.546	-0.956318666666667\\
-0.545	-0.955591666666667\\
-0.544	-0.954858666666667\\
-0.543	-0.954119666666667\\
-0.542	-0.953374666666667\\
-0.541	-0.952623666666667\\
-0.54	-0.951866666666667\\
-0.539	-0.951103666666667\\
-0.538	-0.950334666666667\\
-0.537	-0.949559666666667\\
-0.536	-0.948778666666667\\
-0.535	-0.947991666666667\\
-0.534	-0.947198666666667\\
-0.533	-0.946399666666667\\
-0.532	-0.945594666666667\\
-0.531	-0.944783666666667\\
-0.53	-0.943966666666667\\
-0.529	-0.943143666666666\\
-0.528	-0.942314666666667\\
-0.527	-0.941479666666667\\
-0.526	-0.940638666666667\\
-0.525	-0.939791666666667\\
-0.524	-0.938938666666667\\
-0.523	-0.938079666666667\\
-0.522	-0.937214666666667\\
-0.521	-0.936343666666667\\
-0.52	-0.935466666666667\\
-0.519	-0.934583666666667\\
-0.518	-0.933694666666667\\
-0.517	-0.932799666666667\\
-0.516	-0.931898666666667\\
-0.515	-0.930991666666667\\
-0.514	-0.930078666666667\\
-0.513	-0.929159666666667\\
-0.512	-0.928234666666667\\
-0.511	-0.927303666666667\\
-0.51	-0.926366666666667\\
-0.509	-0.925423666666667\\
-0.508	-0.924474666666667\\
-0.507	-0.923519666666667\\
-0.506	-0.922558666666667\\
-0.505	-0.921591666666667\\
-0.504	-0.920618666666667\\
-0.503	-0.919639666666667\\
-0.502	-0.918654666666667\\
-0.501	-0.917663666666667\\
-0.5	-0.916666666666667\\
-0.499	-0.915663666666667\\
-0.498	-0.914654666666667\\
-0.497	-0.913639666666667\\
-0.496	-0.912618666666667\\
-0.495	-0.911591666666667\\
-0.494	-0.910558666666667\\
-0.493	-0.909519666666667\\
-0.492	-0.908474666666667\\
-0.491	-0.907423666666667\\
-0.49	-0.906366666666667\\
-0.489	-0.905303666666667\\
-0.488	-0.904234666666667\\
-0.487	-0.903159666666667\\
-0.486	-0.902078666666667\\
-0.485	-0.900991666666667\\
-0.484	-0.899898666666667\\
-0.483	-0.898799666666667\\
-0.482	-0.897694666666667\\
-0.481	-0.896583666666667\\
-0.48	-0.895466666666667\\
-0.479	-0.894343666666667\\
-0.478	-0.893214666666667\\
-0.477	-0.892079666666667\\
-0.476	-0.890938666666667\\
-0.475	-0.889791666666667\\
-0.474	-0.888638666666667\\
-0.473	-0.887479666666667\\
-0.472	-0.886314666666667\\
-0.471	-0.885143666666667\\
-0.47	-0.883966666666667\\
-0.469	-0.882783666666667\\
-0.468	-0.881594666666667\\
-0.467	-0.880399666666667\\
-0.466	-0.879198666666667\\
-0.465	-0.877991666666667\\
-0.464	-0.876778666666667\\
-0.463	-0.875559666666667\\
-0.462	-0.874334666666667\\
-0.461	-0.873103666666667\\
-0.46	-0.871866666666667\\
-0.459	-0.870623666666667\\
-0.458	-0.869374666666667\\
-0.457	-0.868119666666667\\
-0.456	-0.866858666666667\\
-0.455	-0.865591666666667\\
-0.454	-0.864318666666667\\
-0.453	-0.863039666666667\\
-0.452	-0.861754666666667\\
-0.451	-0.860463666666667\\
-0.45	-0.859166666666667\\
-0.449	-0.857863666666667\\
-0.448	-0.856554666666667\\
-0.447	-0.855239666666667\\
-0.446	-0.853918666666667\\
-0.445	-0.852591666666667\\
-0.444	-0.851258666666667\\
-0.443	-0.849919666666667\\
-0.442	-0.848574666666667\\
-0.441	-0.847223666666667\\
-0.44	-0.845866666666667\\
-0.439	-0.844503666666667\\
-0.438	-0.843134666666666\\
-0.437	-0.841759666666667\\
-0.436	-0.840378666666667\\
-0.435	-0.838991666666666\\
-0.434	-0.837598666666667\\
-0.433	-0.836199666666667\\
-0.432	-0.834794666666667\\
-0.431	-0.833383666666667\\
-0.43	-0.831966666666667\\
-0.429	-0.830543666666667\\
-0.428	-0.829114666666667\\
-0.427	-0.827679666666667\\
-0.426	-0.826238666666666\\
-0.425	-0.824791666666667\\
-0.424	-0.823338666666667\\
-0.423	-0.821879666666667\\
-0.422	-0.820414666666667\\
-0.421	-0.818943666666667\\
-0.42	-0.817466666666667\\
-0.419	-0.815983666666667\\
-0.418	-0.814494666666667\\
-0.417	-0.812999666666667\\
-0.416	-0.811498666666667\\
-0.415	-0.809991666666667\\
-0.414	-0.808478666666667\\
-0.413	-0.806959666666667\\
-0.412	-0.805434666666667\\
-0.411	-0.803903666666667\\
-0.41	-0.802366666666667\\
-0.409	-0.800823666666667\\
-0.408	-0.799274666666667\\
-0.407	-0.797719666666667\\
-0.406	-0.796158666666667\\
-0.405	-0.794591666666667\\
-0.404	-0.793018666666667\\
-0.403	-0.791439666666667\\
-0.402	-0.789854666666667\\
-0.401	-0.788263666666667\\
-0.4	-0.786666666666667\\
-0.399	-0.785063666666667\\
-0.398	-0.783454666666667\\
-0.397	-0.781839666666667\\
-0.396	-0.780218666666667\\
-0.395	-0.778591666666667\\
-0.394	-0.776958666666667\\
-0.393	-0.775319666666667\\
-0.392	-0.773674666666667\\
-0.391	-0.772023666666667\\
-0.39	-0.770366666666667\\
-0.389	-0.768703666666667\\
-0.388	-0.767034666666667\\
-0.387	-0.765359666666667\\
-0.386	-0.763678666666667\\
-0.385	-0.761991666666667\\
-0.384	-0.760298666666667\\
-0.383	-0.758599666666667\\
-0.382	-0.756894666666667\\
-0.381	-0.755183666666667\\
-0.38	-0.753466666666667\\
-0.379	-0.751743666666667\\
-0.378	-0.750014666666667\\
-0.377	-0.748279666666667\\
-0.376	-0.746538666666667\\
-0.375	-0.744791666666667\\
-0.374	-0.743038666666667\\
-0.373	-0.741279666666667\\
-0.372	-0.739514666666667\\
-0.371	-0.737743666666667\\
-0.37	-0.735966666666667\\
-0.369	-0.734183666666667\\
-0.368	-0.732394666666667\\
-0.367	-0.730599666666667\\
-0.366	-0.728798666666667\\
-0.365	-0.726991666666667\\
-0.364	-0.725178666666667\\
-0.363	-0.723359666666667\\
-0.362	-0.721534666666667\\
-0.361	-0.719703666666667\\
-0.36	-0.717866666666667\\
-0.359	-0.716023666666667\\
-0.358	-0.714174666666667\\
-0.357	-0.712319666666667\\
-0.356	-0.710458666666667\\
-0.355	-0.708591666666667\\
-0.354	-0.706718666666667\\
-0.353	-0.704839666666667\\
-0.352	-0.702954666666667\\
-0.351	-0.701063666666667\\
-0.35	-0.699166666666667\\
-0.349	-0.697263666666667\\
-0.348	-0.695354666666667\\
-0.347	-0.693439666666667\\
-0.346	-0.691518666666667\\
-0.345	-0.689591666666666\\
-0.344	-0.687658666666667\\
-0.343	-0.685719666666667\\
-0.342	-0.683774666666667\\
-0.341	-0.681823666666667\\
-0.34	-0.679866666666667\\
-0.339	-0.677903666666667\\
-0.338	-0.675934666666667\\
-0.337	-0.673959666666667\\
-0.336	-0.671978666666667\\
-0.335	-0.669991666666667\\
-0.334	-0.667998666666667\\
-0.333	-0.666\\
-0.332	-0.664\\
-0.331	-0.662\\
-0.33	-0.66\\
-0.329	-0.658\\
-0.328	-0.656\\
-0.327	-0.654\\
-0.326	-0.652\\
-0.325	-0.65\\
-0.324	-0.648\\
-0.323	-0.646\\
-0.322	-0.644\\
-0.321	-0.642\\
-0.32	-0.64\\
-0.319	-0.638\\
-0.318	-0.636\\
-0.317	-0.634\\
-0.316	-0.632\\
-0.315	-0.63\\
-0.314	-0.628\\
-0.313	-0.626\\
-0.312	-0.624\\
-0.311	-0.622\\
-0.31	-0.62\\
-0.309	-0.618\\
-0.308	-0.616\\
-0.307	-0.614\\
-0.306	-0.612\\
-0.305	-0.61\\
-0.304	-0.608\\
-0.303	-0.606\\
-0.302	-0.604\\
-0.301	-0.602\\
-0.3	-0.6\\
-0.299	-0.598\\
-0.298	-0.596\\
-0.297	-0.594\\
-0.296	-0.592\\
-0.295	-0.59\\
-0.294	-0.588\\
-0.293	-0.586\\
-0.292	-0.584\\
-0.291	-0.582\\
-0.29	-0.58\\
-0.289	-0.578\\
-0.288	-0.576\\
-0.287	-0.574\\
-0.286	-0.572\\
-0.285	-0.57\\
-0.284	-0.568\\
-0.283	-0.566\\
-0.282	-0.564\\
-0.281	-0.562\\
-0.28	-0.56\\
-0.279	-0.558\\
-0.278	-0.556\\
-0.277	-0.554\\
-0.276	-0.552\\
-0.275	-0.55\\
-0.274	-0.548\\
-0.273	-0.546\\
-0.272	-0.544\\
-0.271	-0.542\\
-0.27	-0.54\\
-0.269	-0.538\\
-0.268	-0.536\\
-0.267	-0.534\\
-0.266	-0.532\\
-0.265	-0.53\\
-0.264	-0.528\\
-0.263	-0.526\\
-0.262	-0.524\\
-0.261	-0.522\\
-0.26	-0.52\\
-0.259	-0.518\\
-0.258	-0.516\\
-0.257	-0.514\\
-0.256	-0.512\\
-0.255	-0.51\\
-0.254	-0.508\\
-0.253	-0.506\\
-0.252	-0.504\\
-0.251	-0.502\\
-0.25	-0.5\\
-0.249	-0.498\\
-0.248	-0.496\\
-0.247	-0.494\\
-0.246	-0.492\\
-0.245	-0.49\\
-0.244	-0.488\\
-0.243	-0.486\\
-0.242	-0.484\\
-0.241	-0.482\\
-0.24	-0.48\\
-0.239	-0.478\\
-0.238	-0.476\\
-0.237	-0.474\\
-0.236	-0.472\\
-0.235	-0.47\\
-0.234	-0.468\\
-0.233	-0.466\\
-0.232	-0.464\\
-0.231	-0.462\\
-0.23	-0.46\\
-0.229	-0.458\\
-0.228	-0.456\\
-0.227	-0.454\\
-0.226	-0.452\\
-0.225	-0.45\\
-0.224	-0.448\\
-0.223	-0.446\\
-0.222	-0.444\\
-0.221	-0.442\\
-0.22	-0.44\\
-0.219	-0.438\\
-0.218	-0.436\\
-0.217	-0.434\\
-0.216	-0.432\\
-0.215	-0.43\\
-0.214	-0.428\\
-0.213	-0.426\\
-0.212	-0.424\\
-0.211	-0.422\\
-0.21	-0.42\\
-0.209	-0.418\\
-0.208	-0.416\\
-0.207	-0.414\\
-0.206	-0.412\\
-0.205	-0.41\\
-0.204	-0.408\\
-0.203	-0.406\\
-0.202	-0.404\\
-0.201	-0.402\\
-0.2	-0.4\\
-0.199	-0.398\\
-0.198	-0.396\\
-0.197	-0.394\\
-0.196	-0.392\\
-0.195	-0.39\\
-0.194	-0.388\\
-0.193	-0.386\\
-0.192	-0.384\\
-0.191	-0.382\\
-0.19	-0.38\\
-0.189	-0.378\\
-0.188	-0.376\\
-0.187	-0.374\\
-0.186	-0.372\\
-0.185	-0.37\\
-0.184	-0.368\\
-0.183	-0.366\\
-0.182	-0.364\\
-0.181	-0.362\\
-0.18	-0.36\\
-0.179	-0.358\\
-0.178	-0.356\\
-0.177	-0.354\\
-0.176	-0.352\\
-0.175	-0.35\\
-0.174	-0.348\\
-0.173	-0.346\\
-0.172	-0.344\\
-0.171	-0.342\\
-0.17	-0.34\\
-0.169	-0.338\\
-0.168	-0.336\\
-0.167	-0.334\\
-0.166	-0.332\\
-0.165	-0.33\\
-0.164	-0.328\\
-0.163	-0.326\\
-0.162	-0.324\\
-0.161	-0.322\\
-0.16	-0.32\\
-0.159	-0.318\\
-0.158	-0.316\\
-0.157	-0.314\\
-0.156	-0.312\\
-0.155	-0.31\\
-0.154	-0.308\\
-0.153	-0.306\\
-0.152	-0.304\\
-0.151	-0.302\\
-0.15	-0.3\\
-0.149	-0.298\\
-0.148	-0.296\\
-0.147	-0.294\\
-0.146	-0.292\\
-0.145	-0.29\\
-0.144	-0.288\\
-0.143	-0.286\\
-0.142	-0.284\\
-0.141	-0.282\\
-0.14	-0.28\\
-0.139	-0.278\\
-0.138	-0.276\\
-0.137	-0.274\\
-0.136	-0.272\\
-0.135	-0.27\\
-0.134	-0.268\\
-0.133	-0.266\\
-0.132	-0.264\\
-0.131	-0.262\\
-0.13	-0.26\\
-0.129	-0.258\\
-0.128	-0.256\\
-0.127	-0.254\\
-0.126	-0.252\\
-0.125	-0.25\\
-0.124	-0.248\\
-0.123	-0.246\\
-0.122	-0.244\\
-0.121	-0.242\\
-0.12	-0.24\\
-0.119	-0.238\\
-0.118	-0.236\\
-0.117	-0.234\\
-0.116	-0.232\\
-0.115	-0.23\\
-0.114	-0.228\\
-0.113	-0.226\\
-0.112	-0.224\\
-0.111	-0.222\\
-0.11	-0.22\\
-0.109	-0.218\\
-0.108	-0.216\\
-0.107	-0.214\\
-0.106	-0.212\\
-0.105	-0.21\\
-0.104	-0.208\\
-0.103	-0.206\\
-0.102	-0.204\\
-0.101	-0.202\\
-0.1	-0.2\\
-0.099	-0.198\\
-0.098	-0.196\\
-0.097	-0.194\\
-0.096	-0.192\\
-0.095	-0.19\\
-0.094	-0.188\\
-0.093	-0.186\\
-0.092	-0.184\\
-0.091	-0.182\\
-0.09	-0.18\\
-0.089	-0.178\\
-0.088	-0.176\\
-0.087	-0.174\\
-0.086	-0.172\\
-0.085	-0.17\\
-0.084	-0.168\\
-0.083	-0.166\\
-0.082	-0.164\\
-0.081	-0.162\\
-0.08	-0.16\\
-0.079	-0.158\\
-0.078	-0.156\\
-0.077	-0.154\\
-0.076	-0.152\\
-0.075	-0.15\\
-0.074	-0.148\\
-0.073	-0.146\\
-0.072	-0.144\\
-0.071	-0.142\\
-0.07	-0.14\\
-0.069	-0.138\\
-0.0679999999999999	-0.136\\
-0.0669999999999999	-0.134\\
-0.0659999999999999	-0.132\\
-0.0649999999999999	-0.13\\
-0.0639999999999999	-0.128\\
-0.0629999999999999	-0.126\\
-0.0619999999999999	-0.124\\
-0.0609999999999999	-0.122\\
-0.0599999999999999	-0.12\\
-0.0589999999999999	-0.118\\
-0.0579999999999999	-0.116\\
-0.0569999999999999	-0.114\\
-0.0559999999999999	-0.112\\
-0.0549999999999999	-0.11\\
-0.0539999999999999	-0.108\\
-0.0529999999999999	-0.106\\
-0.0519999999999999	-0.104\\
-0.0509999999999999	-0.102\\
-0.0499999999999999	-0.0999999999999999\\
-0.0489999999999999	-0.0979999999999999\\
-0.0479999999999999	-0.0959999999999999\\
-0.0469999999999999	-0.0939999999999999\\
-0.0459999999999999	-0.0919999999999999\\
-0.0449999999999999	-0.0899999999999999\\
-0.0439999999999999	-0.0879999999999999\\
-0.0429999999999999	-0.0859999999999999\\
-0.0419999999999999	-0.0839999999999999\\
-0.0409999999999999	-0.0819999999999999\\
-0.04	-0.0800000000000001\\
-0.039	-0.0780000000000001\\
-0.038	-0.0760000000000001\\
-0.037	-0.0740000000000001\\
-0.036	-0.0720000000000001\\
-0.035	-0.0700000000000001\\
-0.034	-0.0680000000000001\\
-0.033	-0.0660000000000001\\
-0.032	-0.0640000000000001\\
-0.031	-0.0620000000000001\\
-0.03	-0.0600000000000001\\
-0.029	-0.0580000000000001\\
-0.028	-0.056\\
-0.027	-0.054\\
-0.026	-0.052\\
-0.025	-0.05\\
-0.024	-0.048\\
-0.023	-0.046\\
-0.022	-0.044\\
-0.021	-0.042\\
-0.02	-0.04\\
-0.019	-0.038\\
-0.018	-0.036\\
-0.017	-0.034\\
-0.016	-0.032\\
-0.015	-0.03\\
-0.014	-0.028\\
-0.013	-0.026\\
-0.012	-0.024\\
-0.011	-0.022\\
-0.01	-0.02\\
-0.00900000000000001	-0.018\\
-0.00800000000000001	-0.016\\
-0.00700000000000001	-0.014\\
-0.00600000000000001	-0.012\\
-0.005	-0.01\\
-0.004	-0.00800000000000001\\
-0.003	-0.00600000000000001\\
-0.002	-0.004\\
-0.001	-0.002\\
0	0\\
0.001	0.002\\
0.002	0.004\\
0.003	0.00600000000000001\\
0.004	0.00800000000000001\\
0.005	0.01\\
0.00600000000000001	0.012\\
0.00700000000000001	0.014\\
0.00800000000000001	0.016\\
0.00900000000000001	0.018\\
0.01	0.02\\
0.011	0.022\\
0.012	0.024\\
0.013	0.026\\
0.014	0.028\\
0.015	0.03\\
0.016	0.032\\
0.017	0.034\\
0.018	0.036\\
0.019	0.038\\
0.02	0.04\\
0.021	0.042\\
0.022	0.044\\
0.023	0.046\\
0.024	0.048\\
0.025	0.05\\
0.026	0.052\\
0.027	0.054\\
0.028	0.056\\
0.029	0.0580000000000001\\
0.03	0.0600000000000001\\
0.031	0.0620000000000001\\
0.032	0.0640000000000001\\
0.033	0.0660000000000001\\
0.034	0.0680000000000001\\
0.035	0.0700000000000001\\
0.036	0.0720000000000001\\
0.037	0.0740000000000001\\
0.038	0.0760000000000001\\
0.039	0.0780000000000001\\
0.04	0.0800000000000001\\
0.0409999999999999	0.0819999999999999\\
0.0419999999999999	0.0839999999999999\\
0.0429999999999999	0.0859999999999999\\
0.0439999999999999	0.0879999999999999\\
0.0449999999999999	0.0899999999999999\\
0.0459999999999999	0.0919999999999999\\
0.0469999999999999	0.0939999999999999\\
0.0479999999999999	0.0959999999999999\\
0.0489999999999999	0.0979999999999999\\
0.0499999999999999	0.0999999999999999\\
0.0509999999999999	0.102\\
0.0519999999999999	0.104\\
0.0529999999999999	0.106\\
0.0539999999999999	0.108\\
0.0549999999999999	0.11\\
0.0559999999999999	0.112\\
0.0569999999999999	0.114\\
0.0579999999999999	0.116\\
0.0589999999999999	0.118\\
0.0599999999999999	0.12\\
0.0609999999999999	0.122\\
0.0619999999999999	0.124\\
0.0629999999999999	0.126\\
0.0639999999999999	0.128\\
0.0649999999999999	0.13\\
0.0659999999999999	0.132\\
0.0669999999999999	0.134\\
0.0679999999999999	0.136\\
0.069	0.138\\
0.07	0.14\\
0.071	0.142\\
0.072	0.144\\
0.073	0.146\\
0.074	0.148\\
0.075	0.15\\
0.076	0.152\\
0.077	0.154\\
0.078	0.156\\
0.079	0.158\\
0.08	0.16\\
0.081	0.162\\
0.082	0.164\\
0.083	0.166\\
0.084	0.168\\
0.085	0.17\\
0.086	0.172\\
0.087	0.174\\
0.088	0.176\\
0.089	0.178\\
0.09	0.18\\
0.091	0.182\\
0.092	0.184\\
0.093	0.186\\
0.094	0.188\\
0.095	0.19\\
0.096	0.192\\
0.097	0.194\\
0.098	0.196\\
0.099	0.198\\
0.1	0.2\\
0.101	0.202\\
0.102	0.204\\
0.103	0.206\\
0.104	0.208\\
0.105	0.21\\
0.106	0.212\\
0.107	0.214\\
0.108	0.216\\
0.109	0.218\\
0.11	0.22\\
0.111	0.222\\
0.112	0.224\\
0.113	0.226\\
0.114	0.228\\
0.115	0.23\\
0.116	0.232\\
0.117	0.234\\
0.118	0.236\\
0.119	0.238\\
0.12	0.24\\
0.121	0.242\\
0.122	0.244\\
0.123	0.246\\
0.124	0.248\\
0.125	0.25\\
0.126	0.252\\
0.127	0.254\\
0.128	0.256\\
0.129	0.258\\
0.13	0.26\\
0.131	0.262\\
0.132	0.264\\
0.133	0.266\\
0.134	0.268\\
0.135	0.27\\
0.136	0.272\\
0.137	0.274\\
0.138	0.276\\
0.139	0.278\\
0.14	0.28\\
0.141	0.282\\
0.142	0.284\\
0.143	0.286\\
0.144	0.288\\
0.145	0.29\\
0.146	0.292\\
0.147	0.294\\
0.148	0.296\\
0.149	0.298\\
0.15	0.3\\
0.151	0.302\\
0.152	0.304\\
0.153	0.306\\
0.154	0.308\\
0.155	0.31\\
0.156	0.312\\
0.157	0.314\\
0.158	0.316\\
0.159	0.318\\
0.16	0.32\\
0.161	0.322\\
0.162	0.324\\
0.163	0.326\\
0.164	0.328\\
0.165	0.33\\
0.166	0.332\\
0.167	0.334\\
0.168	0.336\\
0.169	0.338\\
0.17	0.34\\
0.171	0.342\\
0.172	0.344\\
0.173	0.346\\
0.174	0.348\\
0.175	0.35\\
0.176	0.352\\
0.177	0.354\\
0.178	0.356\\
0.179	0.358\\
0.18	0.36\\
0.181	0.362\\
0.182	0.364\\
0.183	0.366\\
0.184	0.368\\
0.185	0.37\\
0.186	0.372\\
0.187	0.374\\
0.188	0.376\\
0.189	0.378\\
0.19	0.38\\
0.191	0.382\\
0.192	0.384\\
0.193	0.386\\
0.194	0.388\\
0.195	0.39\\
0.196	0.392\\
0.197	0.394\\
0.198	0.396\\
0.199	0.398\\
0.2	0.4\\
0.201	0.402\\
0.202	0.404\\
0.203	0.406\\
0.204	0.408\\
0.205	0.41\\
0.206	0.412\\
0.207	0.414\\
0.208	0.416\\
0.209	0.418\\
0.21	0.42\\
0.211	0.422\\
0.212	0.424\\
0.213	0.426\\
0.214	0.428\\
0.215	0.43\\
0.216	0.432\\
0.217	0.434\\
0.218	0.436\\
0.219	0.438\\
0.22	0.44\\
0.221	0.442\\
0.222	0.444\\
0.223	0.446\\
0.224	0.448\\
0.225	0.45\\
0.226	0.452\\
0.227	0.454\\
0.228	0.456\\
0.229	0.458\\
0.23	0.46\\
0.231	0.462\\
0.232	0.464\\
0.233	0.466\\
0.234	0.468\\
0.235	0.47\\
0.236	0.472\\
0.237	0.474\\
0.238	0.476\\
0.239	0.478\\
0.24	0.48\\
0.241	0.482\\
0.242	0.484\\
0.243	0.486\\
0.244	0.488\\
0.245	0.49\\
0.246	0.492\\
0.247	0.494\\
0.248	0.496\\
0.249	0.498\\
0.25	0.5\\
0.251	0.502\\
0.252	0.504\\
0.253	0.506\\
0.254	0.508\\
0.255	0.51\\
0.256	0.512\\
0.257	0.514\\
0.258	0.516\\
0.259	0.518\\
0.26	0.52\\
0.261	0.522\\
0.262	0.524\\
0.263	0.526\\
0.264	0.528\\
0.265	0.53\\
0.266	0.532\\
0.267	0.534\\
0.268	0.536\\
0.269	0.538\\
0.27	0.54\\
0.271	0.542\\
0.272	0.544\\
0.273	0.546\\
0.274	0.548\\
0.275	0.55\\
0.276	0.552\\
0.277	0.554\\
0.278	0.556\\
0.279	0.558\\
0.28	0.56\\
0.281	0.562\\
0.282	0.564\\
0.283	0.566\\
0.284	0.568\\
0.285	0.57\\
0.286	0.572\\
0.287	0.574\\
0.288	0.576\\
0.289	0.578\\
0.29	0.58\\
0.291	0.582\\
0.292	0.584\\
0.293	0.586\\
0.294	0.588\\
0.295	0.59\\
0.296	0.592\\
0.297	0.594\\
0.298	0.596\\
0.299	0.598\\
0.3	0.6\\
0.301	0.602\\
0.302	0.604\\
0.303	0.606\\
0.304	0.608\\
0.305	0.61\\
0.306	0.612\\
0.307	0.614\\
0.308	0.616\\
0.309	0.618\\
0.31	0.62\\
0.311	0.622\\
0.312	0.624\\
0.313	0.626\\
0.314	0.628\\
0.315	0.63\\
0.316	0.632\\
0.317	0.634\\
0.318	0.636\\
0.319	0.638\\
0.32	0.64\\
0.321	0.642\\
0.322	0.644\\
0.323	0.646\\
0.324	0.648\\
0.325	0.65\\
0.326	0.652\\
0.327	0.654\\
0.328	0.656\\
0.329	0.658\\
0.33	0.66\\
0.331	0.662\\
0.332	0.664\\
0.333	0.666\\
0.334	0.667998666666667\\
0.335	0.669991666666667\\
0.336	0.671978666666667\\
0.337	0.673959666666667\\
0.338	0.675934666666667\\
0.339	0.677903666666667\\
0.34	0.679866666666667\\
0.341	0.681823666666667\\
0.342	0.683774666666667\\
0.343	0.685719666666667\\
0.344	0.687658666666667\\
0.345	0.689591666666666\\
0.346	0.691518666666667\\
0.347	0.693439666666667\\
0.348	0.695354666666667\\
0.349	0.697263666666667\\
0.35	0.699166666666667\\
0.351	0.701063666666667\\
0.352	0.702954666666667\\
0.353	0.704839666666667\\
0.354	0.706718666666667\\
0.355	0.708591666666667\\
0.356	0.710458666666667\\
0.357	0.712319666666667\\
0.358	0.714174666666667\\
0.359	0.716023666666667\\
0.36	0.717866666666667\\
0.361	0.719703666666667\\
0.362	0.721534666666667\\
0.363	0.723359666666667\\
0.364	0.725178666666667\\
0.365	0.726991666666667\\
0.366	0.728798666666667\\
0.367	0.730599666666667\\
0.368	0.732394666666667\\
0.369	0.734183666666667\\
0.37	0.735966666666667\\
0.371	0.737743666666667\\
0.372	0.739514666666667\\
0.373	0.741279666666667\\
0.374	0.743038666666667\\
0.375	0.744791666666667\\
0.376	0.746538666666667\\
0.377	0.748279666666667\\
0.378	0.750014666666667\\
0.379	0.751743666666667\\
0.38	0.753466666666667\\
0.381	0.755183666666667\\
0.382	0.756894666666667\\
0.383	0.758599666666667\\
0.384	0.760298666666667\\
0.385	0.761991666666667\\
0.386	0.763678666666667\\
0.387	0.765359666666667\\
0.388	0.767034666666667\\
0.389	0.768703666666667\\
0.39	0.770366666666667\\
0.391	0.772023666666667\\
0.392	0.773674666666667\\
0.393	0.775319666666667\\
0.394	0.776958666666667\\
0.395	0.778591666666667\\
0.396	0.780218666666667\\
0.397	0.781839666666667\\
0.398	0.783454666666667\\
0.399	0.785063666666667\\
0.4	0.786666666666667\\
0.401	0.788263666666667\\
0.402	0.789854666666667\\
0.403	0.791439666666667\\
0.404	0.793018666666667\\
0.405	0.794591666666667\\
0.406	0.796158666666667\\
0.407	0.797719666666667\\
0.408	0.799274666666667\\
0.409	0.800823666666667\\
0.41	0.802366666666667\\
0.411	0.803903666666667\\
0.412	0.805434666666667\\
0.413	0.806959666666667\\
0.414	0.808478666666667\\
0.415	0.809991666666667\\
0.416	0.811498666666667\\
0.417	0.812999666666667\\
0.418	0.814494666666667\\
0.419	0.815983666666667\\
0.42	0.817466666666667\\
0.421	0.818943666666667\\
0.422	0.820414666666667\\
0.423	0.821879666666667\\
0.424	0.823338666666667\\
0.425	0.824791666666667\\
0.426	0.826238666666666\\
0.427	0.827679666666667\\
0.428	0.829114666666667\\
0.429	0.830543666666667\\
0.43	0.831966666666667\\
0.431	0.833383666666667\\
0.432	0.834794666666667\\
0.433	0.836199666666667\\
0.434	0.837598666666667\\
0.435	0.838991666666666\\
0.436	0.840378666666667\\
0.437	0.841759666666667\\
0.438	0.843134666666666\\
0.439	0.844503666666667\\
0.44	0.845866666666667\\
0.441	0.847223666666667\\
0.442	0.848574666666667\\
0.443	0.849919666666667\\
0.444	0.851258666666667\\
0.445	0.852591666666667\\
0.446	0.853918666666667\\
0.447	0.855239666666667\\
0.448	0.856554666666667\\
0.449	0.857863666666667\\
0.45	0.859166666666667\\
0.451	0.860463666666667\\
0.452	0.861754666666667\\
0.453	0.863039666666667\\
0.454	0.864318666666667\\
0.455	0.865591666666667\\
0.456	0.866858666666667\\
0.457	0.868119666666667\\
0.458	0.869374666666667\\
0.459	0.870623666666667\\
0.46	0.871866666666667\\
0.461	0.873103666666667\\
0.462	0.874334666666667\\
0.463	0.875559666666667\\
0.464	0.876778666666667\\
0.465	0.877991666666667\\
0.466	0.879198666666667\\
0.467	0.880399666666667\\
0.468	0.881594666666667\\
0.469	0.882783666666667\\
0.47	0.883966666666667\\
0.471	0.885143666666667\\
0.472	0.886314666666667\\
0.473	0.887479666666667\\
0.474	0.888638666666667\\
0.475	0.889791666666667\\
0.476	0.890938666666667\\
0.477	0.892079666666667\\
0.478	0.893214666666667\\
0.479	0.894343666666667\\
0.48	0.895466666666667\\
0.481	0.896583666666667\\
0.482	0.897694666666667\\
0.483	0.898799666666667\\
0.484	0.899898666666667\\
0.485	0.900991666666667\\
0.486	0.902078666666667\\
0.487	0.903159666666667\\
0.488	0.904234666666667\\
0.489	0.905303666666667\\
0.49	0.906366666666667\\
0.491	0.907423666666667\\
0.492	0.908474666666667\\
0.493	0.909519666666667\\
0.494	0.910558666666667\\
0.495	0.911591666666667\\
0.496	0.912618666666667\\
0.497	0.913639666666667\\
0.498	0.914654666666667\\
0.499	0.915663666666667\\
0.5	0.916666666666667\\
0.501	0.917663666666667\\
0.502	0.918654666666667\\
0.503	0.919639666666667\\
0.504	0.920618666666667\\
0.505	0.921591666666667\\
0.506	0.922558666666667\\
0.507	0.923519666666667\\
0.508	0.924474666666667\\
0.509	0.925423666666667\\
0.51	0.926366666666667\\
0.511	0.927303666666667\\
0.512	0.928234666666667\\
0.513	0.929159666666667\\
0.514	0.930078666666667\\
0.515	0.930991666666667\\
0.516	0.931898666666667\\
0.517	0.932799666666667\\
0.518	0.933694666666667\\
0.519	0.934583666666667\\
0.52	0.935466666666667\\
0.521	0.936343666666667\\
0.522	0.937214666666667\\
0.523	0.938079666666667\\
0.524	0.938938666666667\\
0.525	0.939791666666667\\
0.526	0.940638666666667\\
0.527	0.941479666666667\\
0.528	0.942314666666667\\
0.529	0.943143666666666\\
0.53	0.943966666666667\\
0.531	0.944783666666667\\
0.532	0.945594666666667\\
0.533	0.946399666666667\\
0.534	0.947198666666667\\
0.535	0.947991666666667\\
0.536	0.948778666666667\\
0.537	0.949559666666667\\
0.538	0.950334666666667\\
0.539	0.951103666666667\\
0.54	0.951866666666667\\
0.541	0.952623666666667\\
0.542	0.953374666666667\\
0.543	0.954119666666667\\
0.544	0.954858666666667\\
0.545	0.955591666666667\\
0.546	0.956318666666667\\
0.547	0.957039666666667\\
0.548	0.957754666666667\\
0.549	0.958463666666667\\
0.55	0.959166666666667\\
0.551	0.959863666666667\\
0.552	0.960554666666667\\
0.553	0.961239666666667\\
0.554	0.961918666666667\\
0.555	0.962591666666667\\
0.556	0.963258666666667\\
0.557	0.963919666666667\\
0.558	0.964574666666667\\
0.559	0.965223666666667\\
0.56	0.965866666666667\\
0.561	0.966503666666667\\
0.562	0.967134666666667\\
0.563	0.967759666666667\\
0.564	0.968378666666667\\
0.565	0.968991666666667\\
0.566	0.969598666666667\\
0.567	0.970199666666667\\
0.568	0.970794666666667\\
0.569	0.971383666666667\\
0.57	0.971966666666667\\
0.571	0.972543666666667\\
0.572	0.973114666666667\\
0.573	0.973679666666667\\
0.574	0.974238666666667\\
0.575	0.974791666666667\\
0.576	0.975338666666667\\
0.577	0.975879666666667\\
0.578	0.976414666666667\\
0.579	0.976943666666667\\
0.58	0.977466666666667\\
0.581	0.977983666666667\\
0.582	0.978494666666667\\
0.583	0.978999666666667\\
0.584	0.979498666666667\\
0.585	0.979991666666667\\
0.586	0.980478666666667\\
0.587	0.980959666666667\\
0.588	0.981434666666667\\
0.589	0.981903666666667\\
0.59	0.982366666666667\\
0.591	0.982823666666667\\
0.592	0.983274666666667\\
0.593	0.983719666666667\\
0.594	0.984158666666667\\
0.595	0.984591666666667\\
0.596	0.985018666666667\\
0.597	0.985439666666667\\
0.598	0.985854666666667\\
0.599	0.986263666666667\\
0.6	0.986666666666667\\
0.601	0.987063666666667\\
0.602	0.987454666666667\\
0.603	0.987839666666667\\
0.604	0.988218666666667\\
0.605	0.988591666666667\\
0.606	0.988958666666667\\
0.607	0.989319666666667\\
0.608	0.989674666666667\\
0.609	0.990023666666667\\
0.61	0.990366666666667\\
0.611	0.990703666666667\\
0.612	0.991034666666667\\
0.613	0.991359666666667\\
0.614	0.991678666666667\\
0.615	0.991991666666667\\
0.616	0.992298666666667\\
0.617	0.992599666666667\\
0.618	0.992894666666667\\
0.619	0.993183666666667\\
0.62	0.993466666666667\\
0.621	0.993743666666667\\
0.622	0.994014666666667\\
0.623	0.994279666666667\\
0.624	0.994538666666667\\
0.625	0.994791666666667\\
0.626	0.995038666666667\\
0.627	0.995279666666667\\
0.628	0.995514666666667\\
0.629	0.995743666666667\\
0.63	0.995966666666667\\
0.631	0.996183666666667\\
0.632	0.996394666666667\\
0.633	0.996599666666667\\
0.634	0.996798666666667\\
0.635	0.996991666666667\\
0.636	0.997178666666667\\
0.637	0.997359666666667\\
0.638	0.997534666666667\\
0.639	0.997703666666667\\
0.64	0.997866666666667\\
0.641	0.998023666666667\\
0.642	0.998174666666667\\
0.643	0.998319666666667\\
0.644	0.998458666666667\\
0.645	0.998591666666667\\
0.646	0.998718666666667\\
0.647	0.998839666666667\\
0.648	0.998954666666667\\
0.649	0.999063666666667\\
0.65	0.999166666666667\\
0.651	0.999263666666667\\
0.652	0.999354666666667\\
0.653	0.999439666666667\\
0.654	0.999518666666667\\
0.655	0.999591666666667\\
0.656	0.999658666666667\\
0.657	0.999719666666667\\
0.658	0.999774666666667\\
0.659	0.999823666666667\\
0.66	0.999866666666667\\
0.661	0.999903666666667\\
0.662	0.999934666666667\\
0.663	0.999959666666667\\
0.664	0.999978666666667\\
0.665	0.999991666666667\\
0.666	0.999998666666667\\
0.667	1\\
0.668	1\\
0.669	1\\
0.67	1\\
0.671	1\\
0.672	1\\
0.673	1\\
0.674	1\\
0.675	1\\
0.676	1\\
0.677	1\\
0.678	1\\
0.679	1\\
0.68	1\\
0.681	1\\
0.682	1\\
0.683	1\\
0.684	1\\
0.685	1\\
0.686	1\\
0.687	1\\
0.688	1\\
0.689	1\\
0.69	1\\
0.691	1\\
0.692	1\\
0.693	1\\
0.694	1\\
0.695	1\\
0.696	1\\
0.697	1\\
0.698	1\\
0.699	1\\
0.7	1\\
0.701	1\\
0.702	1\\
0.703	1\\
0.704	1\\
0.705	1\\
0.706	1\\
0.707	1\\
0.708	1\\
0.709	1\\
0.71	1\\
0.711	1\\
0.712	1\\
0.713	1\\
0.714	1\\
0.715	1\\
0.716	1\\
0.717	1\\
0.718	1\\
0.719	1\\
0.72	1\\
0.721	1\\
0.722	1\\
0.723	1\\
0.724	1\\
0.725	1\\
0.726	1\\
0.727	1\\
0.728	1\\
0.729	1\\
0.73	1\\
0.731	1\\
0.732	1\\
0.733	1\\
0.734	1\\
0.735	1\\
0.736	1\\
0.737	1\\
0.738	1\\
0.739	1\\
0.74	1\\
0.741	1\\
0.742	1\\
0.743	1\\
0.744	1\\
0.745	1\\
0.746	1\\
0.747	1\\
0.748	1\\
0.749	1\\
0.75	1\\
0.751	1\\
0.752	1\\
0.753	1\\
0.754	1\\
0.755	1\\
0.756	1\\
0.757	1\\
0.758	1\\
0.759	1\\
0.76	1\\
0.761	1\\
0.762	1\\
0.763	1\\
0.764	1\\
0.765	1\\
0.766	1\\
0.767	1\\
0.768	1\\
0.769	1\\
0.77	1\\
0.771	1\\
0.772	1\\
0.773	1\\
0.774	1\\
0.775	1\\
0.776	1\\
0.777	1\\
0.778	1\\
0.779	1\\
0.78	1\\
0.781	1\\
0.782	1\\
0.783	1\\
0.784	1\\
0.785	1\\
0.786	1\\
0.787	1\\
0.788	1\\
0.789	1\\
0.79	1\\
0.791	1\\
0.792	1\\
0.793	1\\
0.794	1\\
0.795	1\\
0.796	1\\
0.797	1\\
0.798	1\\
0.799	1\\
0.8	1\\
0.801	1\\
0.802	1\\
0.803	1\\
0.804	1\\
0.805	1\\
0.806	1\\
0.807	1\\
0.808	1\\
0.809	1\\
0.81	1\\
0.811	1\\
0.812	1\\
0.813	1\\
0.814	1\\
0.815	1\\
0.816	1\\
0.817	1\\
0.818	1\\
0.819	1\\
0.82	1\\
0.821	1\\
0.822	1\\
0.823	1\\
0.824	1\\
0.825	1\\
0.826	1\\
0.827	1\\
0.828	1\\
0.829	1\\
0.83	1\\
0.831	1\\
0.832	1\\
0.833	1\\
0.834	1\\
0.835	1\\
0.836	1\\
0.837	1\\
0.838	1\\
0.839	1\\
0.84	1\\
0.841	1\\
0.842	1\\
0.843	1\\
0.844	1\\
0.845	1\\
0.846	1\\
0.847	1\\
0.848	1\\
0.849	1\\
0.85	1\\
0.851	1\\
0.852	1\\
0.853	1\\
0.854	1\\
0.855	1\\
0.856	1\\
0.857	1\\
0.858	1\\
0.859	1\\
0.86	1\\
0.861	1\\
0.862	1\\
0.863	1\\
0.864	1\\
0.865	1\\
0.866	1\\
0.867	1\\
0.868	1\\
0.869	1\\
0.87	1\\
0.871	1\\
0.872	1\\
0.873	1\\
0.874	1\\
0.875	1\\
0.876	1\\
0.877	1\\
0.878	1\\
0.879	1\\
0.88	1\\
0.881	1\\
0.882	1\\
0.883	1\\
0.884	1\\
0.885	1\\
0.886	1\\
0.887	1\\
0.888	1\\
0.889	1\\
0.89	1\\
0.891	1\\
0.892	1\\
0.893	1\\
0.894	1\\
0.895	1\\
0.896	1\\
0.897	1\\
0.898	1\\
0.899	1\\
0.9	1\\
0.901	1\\
0.902	1\\
0.903	1\\
0.904	1\\
0.905	1\\
0.906	1\\
0.907	1\\
0.908	1\\
0.909	1\\
0.91	1\\
0.911	1\\
0.912	1\\
0.913	1\\
0.914	1\\
0.915	1\\
0.916	1\\
0.917	1\\
0.918	1\\
0.919	1\\
0.92	1\\
0.921	1\\
0.922	1\\
0.923	1\\
0.924	1\\
0.925	1\\
0.926	1\\
0.927	1\\
0.928	1\\
0.929	1\\
0.93	1\\
0.931	1\\
0.932	1\\
0.933	1\\
0.934	1\\
0.935	1\\
0.936	1\\
0.937	1\\
0.938	1\\
0.939	1\\
0.94	1\\
0.941	1\\
0.942	1\\
0.943	1\\
0.944	1\\
0.945	1\\
0.946	1\\
0.947	1\\
0.948	1\\
0.949	1\\
0.95	1\\
0.951	1\\
0.952	1\\
0.953	1\\
0.954	1\\
0.955	1\\
0.956	1\\
0.957	1\\
0.958	1\\
0.959	1\\
0.96	1\\
0.961	1\\
0.962	1\\
0.963	1\\
0.964	1\\
0.965	1\\
0.966	1\\
0.967	1\\
0.968	1\\
0.969	1\\
0.97	1\\
0.971	1\\
0.972	1\\
0.973	1\\
0.974	1\\
0.975	1\\
0.976	1\\
0.977	1\\
0.978	1\\
0.979	1\\
0.98	1\\
0.981	1\\
0.982	1\\
0.983	1\\
0.984	1\\
0.985	1\\
0.986	1\\
0.987	1\\
0.988	1\\
0.989	1\\
0.99	1\\
0.991	1\\
0.992	1\\
0.993	1\\
0.994	1\\
0.995	1\\
0.996	1\\
0.997	1\\
0.998	1\\
0.999	1\\
1	1\\
};
\end{axis}

\begin{axis}[%
width=1.621in,
height=1.476in,
at={(2.904in,0.424in)},
scale only axis,
unbounded coords=jump,
xmin=-30,
xmax=0,
xlabel style={font=\color{white!15!black}},
xlabel={X in dB},
ymin=-30,
ymax=1,
ylabel style={font=\color{white!15!black}},
ylabel={Y in dB},
title style={font=\bfseries},
title={Log. output over input level},
xmajorgrids,
ymajorgrids
]
\addplot [color=mycolor1, forget plot]
  table[row sep=crcr]{%
29.5424250943932	0\\
29.542135559913	0\\
29.5418460157812	0\\
29.5415564619971	0\\
29.5412668985601	0\\
29.5409773254695	0\\
29.5406877427247	0\\
29.540398150325	0\\
29.5401085482699	0\\
29.5398189365586	0\\
29.5395293151905	0\\
29.5392396841651	0\\
29.5389500434815	0\\
29.5386603931392	0\\
29.5383707331376	0\\
29.538081063476	0\\
29.5377913841537	0\\
29.5375016951701	0\\
29.5372119965246	0\\
29.5369222882165	0\\
29.5366325702452	0\\
29.53634284261	0\\
29.5360531053103	0\\
29.5357633583455	0\\
29.5354736017148	0\\
29.5351838354177	0\\
29.5348940594535	0\\
29.5346042738216	0\\
29.5343144785213	0\\
29.5340246735519	0\\
29.5337348589129	0\\
29.5334450346036	0\\
29.5331552006232	0\\
29.5328653569713	0\\
29.5325755036471	0\\
29.5322856406501	0\\
29.5319957679795	0\\
29.5317058856346	0\\
29.531415993615	0\\
29.5311260919198	0\\
29.5308361805486	0\\
29.5305462595005	0\\
29.5302563287751	0\\
29.5299663883715	0\\
29.5296764382893	0\\
29.5293864785277	0\\
29.5290965090861	0\\
29.5288065299638	0\\
29.5285165411602	0\\
29.5282265426747	0\\
29.5279365345066	0\\
29.5276465166553	0\\
29.52735648912	0\\
29.5270664519003	0\\
29.5267764049953	0\\
29.5264863484046	0\\
29.5261962821273	0\\
29.525906206163	0\\
29.5256161205108	0\\
29.5253260251703	0\\
29.5250359201407	0\\
29.5247458054214	0\\
29.5244556810117	0\\
29.524165546911	0\\
29.5238754031187	0\\
29.5235852496341	0\\
29.5232950864565	0\\
29.5230049135853	0\\
29.5227147310199	0\\
29.5224245387597	0\\
29.5221343368038	0\\
29.5218441251518	0\\
29.521553903803	0\\
29.5212636727566	0\\
29.5209734320122	0\\
29.520683181569	0\\
29.5203929214263	0\\
29.5201026515836	0\\
29.5198123720401	0\\
29.5195220827953	0\\
29.5192317838485	0\\
29.518941475199	0\\
29.5186511568462	0\\
29.5183608287894	0\\
29.518070491028	0\\
29.5177801435613	0\\
29.5174897863887	0\\
29.5171994195096	0\\
29.5169090429233	0\\
29.5166186566291	0\\
29.5163282606264	0\\
29.5160378549145	0\\
29.5157474394928	0\\
29.5154570143607	0\\
29.5151665795175	0\\
29.5148761349625	0\\
29.5145856806951	0\\
29.5142952167147	0\\
29.5140047430206	0\\
29.5137142596121	0\\
29.5134237664886	0\\
29.5131332636495	0\\
29.512842751094	0\\
29.5125522288216	0\\
29.5122616968316	0\\
29.5119711551234	0\\
29.5116806036962	0\\
29.5113900425495	0\\
29.5110994716826	0\\
29.5108088910949	0\\
29.5105183007856	0\\
29.5102277007542	0\\
29.509937091	0\\
29.5096464715223	0\\
29.5093558423205	0\\
29.509065203394	0\\
29.5087745547421	0\\
29.5084838963641	0\\
29.5081932282594	0\\
29.5079025504273	0\\
29.5076118628672	0\\
29.5073211655785	0\\
29.5070304585604	0\\
29.5067397418124	0\\
29.5064490153338	0\\
29.5061582791239	0\\
29.5058675331821	0\\
29.5055767775077	0\\
29.5052860121001	0\\
29.5049952369586	0\\
29.5047044520826	0\\
29.5044136574714	0\\
29.5041228531244	0\\
29.5038320390409	0\\
29.5035412152202	0\\
29.5032503816618	0\\
29.502959538365	0\\
29.502668685329	0\\
29.5023778225533	0\\
29.5020869500373	0\\
29.5017960677801	0\\
29.5015051757813	0\\
29.5012142740401	0\\
29.500923362556	0\\
29.5006324413281	0\\
29.500341510356	0\\
29.5000505696389	0\\
29.4997596191761	0\\
29.4994686589672	0\\
29.4991776890112	0\\
29.4988867093078	0\\
29.498595719856	0\\
29.4983047206555	0\\
29.4980137117053	0\\
29.497722693005	0\\
29.4974316645539	0\\
29.4971406263512	0\\
29.4968495783964	0\\
29.4965585206888	0\\
29.4962674532278	0\\
29.4959763760126	0\\
29.4956852890427	0\\
29.4953941923174	0\\
29.495103085836	0\\
29.4948119695978	0\\
29.4945208436023	0\\
29.4942297078488	0\\
29.4939385623366	0\\
29.493647407065	0\\
29.4933562420335	0\\
29.4930650672413	0\\
29.4927738826878	0\\
29.4924826883723	0\\
29.4921914842942	0\\
29.4919002704529	0\\
29.4916090468476	0\\
29.4913178134777	0\\
29.4910265703427	0\\
29.4907353174417	0\\
29.4904440547742	0\\
29.4901527823395	0\\
29.489861500137	0\\
29.4895702081659	0\\
29.4892789064257	0\\
29.4889875949157	0\\
29.4886962736352	0\\
29.4884049425835	0\\
29.4881136017601	0\\
29.4878222511643	0\\
29.4875308907953	0\\
29.4872395206526	0\\
29.4869481407355	0\\
29.4866567510433	0\\
29.4863653515755	0\\
29.4860739423312	0\\
29.4857825233099	0\\
29.485491094511	0\\
29.4851996559336	0\\
29.4849082075773	0\\
29.4846167494414	0\\
29.4843252815251	0\\
29.4840338038279	0\\
29.483742316349	0\\
29.4834508190879	0\\
29.4831593120438	0\\
29.4828677952161	0\\
29.4825762686042	0\\
29.4822847322074	0\\
29.481993186025	0\\
29.4817016300564	0\\
29.4814100643009	0\\
29.4811184887578	0\\
29.4808269034266	0\\
29.4805353083066	0\\
29.480243703397	0\\
29.4799520886973	0\\
29.4796604642067	0\\
29.4793688299247	0\\
29.4790771858505	0\\
29.4787855319836	0\\
29.4784938683231	0\\
29.4782021948686	0\\
29.4779105116193	0\\
29.4776188185746	0\\
29.4773271157338	0\\
29.4770354030963	0\\
29.4767436806614	0\\
29.4764519484284	0\\
29.4761602063967	0\\
29.4758684545656	0\\
29.4755766929345	0\\
29.4752849215027	0\\
29.4749931402696	0\\
29.4747013492344	0\\
29.4744095483966	0\\
29.4741177377554	0\\
29.4738259173103	0\\
29.4735340870606	0\\
29.4732422470055	0\\
29.4729503971445	0\\
29.4726585374768	0\\
29.4723666680019	0\\
29.472074788719	0\\
29.4717828996276	0\\
29.4714910007269	0\\
29.4711990920163	0\\
29.4709071734951	0\\
29.4706152451626	0\\
29.4703233070183	0\\
29.4700313590615	0\\
29.4697394012914	0\\
29.4694474337074	0\\
29.4691554563089	0\\
29.4688634690952	0\\
29.4685714720657	0\\
29.4682794652197	0\\
29.4679874485565	0\\
29.4676954220754	0\\
29.4674033857759	0\\
29.4671113396572	0\\
29.4668192837187	0\\
29.4665272179597	0\\
29.4662351423796	0\\
29.4659430569777	0\\
29.4656509617534	0\\
29.4653588567059	0\\
29.4650667418347	0\\
29.464774617139	0\\
29.4644824826182	0\\
29.4641903382717	0\\
29.4638981840988	0\\
29.4636060200987	0\\
29.463313846271	0\\
29.4630216626148	0\\
29.4627294691295	0\\
29.4624372658146	0\\
29.4621450526692	0\\
29.4618528296928	0\\
29.4615605968847	0\\
29.4612683542443	0\\
29.4609761017708	0\\
29.4606838394635	0\\
29.460391567322	0\\
29.4600992853454	0\\
29.4598069935332	0\\
29.4595146918846	0\\
29.459222380399	0\\
29.4589300590757	0\\
29.4586377279141	0\\
29.4583453869135	0\\
29.4580530360733	0\\
29.4577606753927	0\\
29.4574683048712	0\\
29.457175924508	0\\
29.4568835343026	0\\
29.4565911342541	0\\
29.456298724362	0\\
29.4560063046257	0\\
29.4557138750444	0\\
29.4554214356175	0\\
29.4551289863442	0\\
29.4548365272241	0\\
29.4545440582563	0\\
29.4542515794403	0\\
29.4539590907754	0\\
29.4536665922608	0\\
29.453374083896	0\\
29.4530815656803	0\\
29.4527890376129	0\\
29.4524964996934	0\\
29.4522039519209	0\\
29.4519113942948	0\\
29.4516188268145	0\\
29.4513262494793	0\\
29.4510336622886	0\\
29.4507410652415	0\\
29.4504484583377	0\\
29.4501558415762	0\\
29.4498632149565	0\\
29.4495705784779	0\\
29.4492779321398	0\\
29.4489852759414	0\\
29.4486926098822	0\\
29.4483999339614	0\\
29.4481072481784	0\\
29.4478145525326	0\\
29.4475218470232	0\\
29.4472291316496	0\\
29.4469364064111	0\\
29.4466436713071	0\\
29.4463509263368	0\\
29.4460581714998	0\\
29.4457654067951	0\\
29.4454726322223	0\\
29.4451798477807	0\\
29.4448870534695	0\\
29.4445942492881	0\\
29.4443014352358	0\\
29.444008611312	0\\
29.4437157775161	0\\
29.4434229338473	0\\
29.4431300803049	0\\
29.4428372168884	0\\
29.442544343597	0\\
29.4422514604301	0\\
29.4419585673871	0\\
29.4416656644671	0\\
29.4413727516697	0\\
29.4410798289941	0\\
29.4407868964396	0\\
29.4404939540056	0\\
29.4402010016915	0\\
29.4399080394964	0\\
29.4396150674199	0\\
29.4393220854612	0\\
29.4390290936196	0\\
29.4387360918946	0\\
29.4384430802853	0\\
29.4381500587912	0\\
29.4378570274116	0\\
29.4375639861458	0\\
29.4372709349932	0\\
29.436977873953	0\\
29.4366848030247	0\\
29.4363917222075	0\\
29.4360986315007	0\\
29.4358055309038	0\\
29.4355124204161	0\\
29.4352193000368	0\\
29.4349261697653	0\\
29.434633029601	0\\
29.4343398795432	0\\
29.4340467195911	0\\
29.4337535497442	0\\
29.4334603700018	0\\
29.4331671803632	0\\
29.4328739808277	0\\
29.4325807713947	0\\
29.4322875520635	0\\
29.4319943228334	0\\
29.4317010837038	0\\
29.431407834674	0\\
29.4311145757433	0\\
29.430821306911	0\\
29.4305280281766	0\\
29.4302347395392	0\\
29.4299414409984	0\\
29.4296481325533	0\\
29.4293548142033	0\\
29.4290614859477	0\\
29.428768147786	0\\
29.4284747997173	0\\
29.4281814417411	0\\
29.4278880738567	0\\
29.4275946960633	0\\
29.4273013083604	0\\
29.4270079107472	0\\
29.4267145032231	0\\
29.4264210857875	0\\
29.4261276584396	0\\
29.4258342211788	0\\
29.4255407740044	0\\
29.4252473169157	0\\
29.4249538499121	0\\
29.4246603729929	0\\
29.4243668861574	0\\
29.4240733894051	0\\
29.423779882735	0\\
29.4234863661468	0\\
29.4231928396395	0\\
29.4228993032127	0\\
29.4226057568655	0\\
29.4223122005974	0\\
29.4220186344077	0\\
29.4217250582956	0\\
29.4214314722606	0\\
29.421137876302	0\\
29.420844270419	0\\
29.420550654611	0\\
29.4202570288774	0\\
29.4199633932175	0\\
29.4196697476305	0\\
29.4193760921159	0\\
29.419082426673	0\\
29.418788751301	0\\
29.4184950659994	0\\
29.4182013707674	0\\
29.4179076656043	0\\
29.4176139505096	0\\
29.4173202254825	0\\
29.4170264905224	0\\
29.4167327456285	0\\
29.4164389908003	0\\
29.416145226037	0\\
29.4158514513379	0\\
29.4155576667025	0\\
29.41526387213	0\\
29.4149700676197	0\\
29.4146762531711	0\\
29.4143824287833	0\\
29.4140885944558	0\\
29.4137947501878	0\\
29.4135008959787	0\\
29.4132070318279	0\\
29.4129131577346	0\\
29.4126192736981	0\\
29.4123253797179	0\\
29.4120314757932	0\\
29.4117375619233	0\\
29.4114436381076	0\\
29.4111497043455	0\\
29.4108557606362	0\\
29.410561806979	0\\
29.4102678433733	0\\
29.4099738698184	0\\
29.4096798863137	0\\
29.4093858928584	0\\
29.409091889452	0\\
29.4087978760936	0\\
29.4085038527827	0\\
29.4082098195186	0\\
29.4079157763006	0\\
29.407621723128	0\\
29.4073276600001	0\\
29.4070335869164	0\\
29.406739503876	0\\
29.4064454108783	0\\
29.4061513079228	0\\
29.4058571950086	0\\
29.405563072135	0\\
29.4052689393016	0\\
29.4049747965074	0\\
29.404680643752	0\\
29.4043864810346	0\\
29.4040923083545	0\\
29.403798125711	0\\
29.4035039331036	0\\
29.4032097305314	0\\
29.4029155179939	0\\
29.4026212954904	0\\
29.4023270630201	0\\
29.4020328205824	0\\
29.4017385681767	0\\
29.4014443058022	0\\
29.4011500334583	0\\
29.4008557511443	0\\
29.4005614588595	0\\
29.4002671566033	0\\
29.3999728443749	0\\
29.3996785221738	0\\
29.3993841899992	0\\
29.3990898478504	0\\
29.3987954957268	0\\
29.3985011336278	0\\
29.3982067615525	0\\
29.3979123795004	0\\
29.3976179874707	0\\
29.3973235854628	0\\
29.3970291734761	0\\
29.3967347515098	0\\
29.3964403195633	0\\
29.3961458776358	0\\
29.3958514257268	0\\
29.3955569638355	0\\
29.3952624919612	0\\
29.3949680101033	0\\
29.3946735182612	0\\
29.394379016434	0\\
29.3940845046212	0\\
29.3937899828221	0\\
29.393495451036	0\\
29.3932009092621	0\\
29.3929063575	0\\
29.3926117957488	0\\
29.3923172240078	0\\
29.3920226422765	0\\
29.3917280505541	0\\
29.39143344884	0\\
29.3911388371334	0\\
29.3908442154337	0\\
29.3905495837403	0\\
29.3902549420524	0\\
29.3899602903693	0\\
29.3896656286904	0\\
29.3893709570151	0\\
29.3890762753425	0\\
29.3887815836721	0\\
29.3884868820032	0\\
29.3881921703351	0\\
29.3878974486671	0\\
29.3876027169985	0\\
29.3873079753287	0\\
29.3870132236569	0\\
29.3867184619826	0\\
29.3864236903049	0\\
29.3861289086234	0\\
29.3858341169371	0\\
29.3855393152456	0\\
29.385244503548	0\\
29.3849496818438	0\\
29.3846548501322	0\\
29.3843600084126	0\\
29.3840651566843	0\\
29.3837702949466	0\\
29.3834754231988	0\\
29.3831805414402	0\\
29.3828856496703	0\\
29.3825907478882	0\\
29.3822958360933	0\\
29.3820009142849	0\\
29.3817059824624	0\\
29.3814110406251	0\\
29.3811160887722	0\\
29.3808211269032	0\\
29.3805261550172	0\\
29.3802311731137	0\\
29.379936181192	0\\
29.3796411792514	0\\
29.3793461672912	0\\
29.3790511453107	0\\
29.3787561133092	0\\
29.3784610712861	0\\
29.3781660192407	0\\
29.3778709571723	0\\
29.3775758850802	0\\
29.3772808029638	0\\
29.3769857108223	0\\
29.376690608655	0\\
29.3763954964614	0\\
29.3761003742407	0\\
29.3758052419922	0\\
29.3755100997153	0\\
29.3752149474092	0\\
29.3749197850734	0\\
29.374624612707	0\\
29.3743294303094	0\\
29.37403423788	0\\
29.3737390354181	0\\
29.3734438229229	0\\
29.3731486003939	0\\
29.3728533678302	0\\
29.3725581252313	0\\
29.3722628725964	0\\
29.3719676099249	0\\
29.3716723372161	0\\
29.3713770544693	0\\
29.3710817616838	0\\
29.370786458859	0\\
29.3704911459941	0\\
29.3701958230884	0\\
29.3699004901414	0\\
29.3696051471523	0\\
29.3693097941203	0\\
29.369014431045	0\\
29.3687190579255	0\\
29.3684236747611	0\\
29.3681282815513	0\\
29.3678328782952	0\\
29.3675374649923	0\\
29.3672420416419	0\\
29.3669466082431	0\\
29.3666511647955	0\\
29.3663557112983	0\\
29.3660602477507	0\\
29.3657647741522	0\\
29.365469290502	0\\
29.3651737967995	0\\
29.3648782930439	0\\
29.3645827792347	0\\
29.364287255371	0\\
29.3639917214523	0\\
29.3636961774777	0\\
29.3634006234468	0\\
29.3631050593587	0\\
29.3628094852128	0\\
29.3625139010084	0\\
29.3622183067447	0\\
29.3619227024213	0\\
29.3616270880372	0\\
29.361331463592	0\\
29.3610358290848	0\\
29.3607401845149	0\\
29.3604445298818	0\\
29.3601488651847	0\\
29.359853190423	0\\
29.3595575055959	0\\
29.3592618107027	0\\
29.3589661057428	0\\
29.3586703907155	0\\
29.3583746656202	0\\
29.358078930456	0\\
29.3577831852224	0\\
29.3574874299186	0\\
29.357191664544	0\\
29.3568958890979	0\\
29.3566001035795	0\\
29.3563043079883	0\\
29.3560085023235	0\\
29.3557126865844	0\\
29.3554168607703	0\\
29.3551210248807	0\\
29.3548251789147	0\\
29.3545293228716	0\\
29.3542334567509	0\\
29.3539375805518	0\\
29.3536416942736	0\\
29.3533457979157	0\\
29.3530498914773	0\\
29.3527539749578	0\\
29.3524580483565	0\\
29.3521621116727	0\\
29.3518661649056	0\\
29.3515702080547	0\\
29.3512742411193	0\\
29.3509782640985	0\\
29.3506822769918	0\\
29.3503862797985	0\\
29.3500902725179	0\\
29.3497942551493	0\\
29.349498227692	0\\
29.3492021901453	0\\
29.3489061425085	0\\
29.348610084781	0\\
29.348314016962	0\\
29.3480179390509	0\\
29.347721851047	0\\
29.3474257529496	0\\
29.347129644758	0\\
29.3468335264715	0\\
29.3465373980894	0\\
29.346241259611	0\\
29.3459451110358	0\\
29.3456489523628	0\\
29.3453527835916	0\\
29.3450566047213	0\\
29.3447604157513	0\\
29.344464216681	0\\
29.3441680075095	0\\
29.3438717882363	0\\
29.3435755588606	0\\
29.3432793193818	0\\
29.3429830697991	0\\
29.342686810112	0\\
29.3423905403195	0\\
29.3420942604212	0\\
29.3417979704163	0\\
29.3415016703041	0\\
29.341205360084	0\\
29.3409090397551	0\\
29.3406127093169	0\\
29.3403163687687	0\\
29.3400200181097	0\\
29.3397236573394	0\\
29.3394272864569	0\\
29.3391309054616	0\\
29.3388345143528	0\\
29.3385381131298	0\\
29.3382417017919	0\\
29.3379452803385	0\\
29.3376488487688	0\\
29.3373524070822	0\\
29.3370559552779	0\\
29.3367594933553	0\\
29.3364630213137	0\\
29.3361665391524	0\\
29.3358700468707	0\\
29.3355735444679	0\\
29.3352770319433	0\\
29.3349805092963	0\\
29.334683976526	0\\
29.334387433632	0\\
29.3340908806134	0\\
29.3337943174695	0\\
29.3334977441997	0\\
29.3332011608033	0\\
29.3329045672796	0\\
29.3326079636279	0\\
29.3323113498474	0\\
29.3320147259376	0\\
29.3317180918977	0\\
29.3314214477271	0\\
29.331124793425	0\\
29.3308281289907	0\\
29.3305314544236	0\\
29.3302347697229	0\\
29.329938074888	0\\
29.3296413699182	0\\
29.3293446548128	0\\
29.329047929571	0\\
29.3287511941923	0\\
29.3284544486758	0\\
29.328157693021	0\\
29.3278609272271	0\\
29.3275641512934	0\\
29.3272673652193	0\\
29.326970569004	0\\
29.3266737626468	0\\
29.3263769461471	0\\
29.3260801195042	0\\
29.3257832827173	0\\
29.3254864357858	0\\
29.325189578709	0\\
29.3248927114862	0\\
29.3245958341167	0\\
29.3242989465998	0\\
29.3240020489347	0\\
29.3237051411209	0\\
29.3234082231577	0\\
29.3231112950442	0\\
29.3228143567799	0\\
29.322517408364	0\\
29.3222204497958	0\\
29.3219234810747	0\\
29.3216265022	0\\
29.3213295131709	0\\
29.3210325139868	0\\
29.320735504647	0\\
29.3204384851507	0\\
29.3201414554973	0\\
29.3198444156862	0\\
29.3195473657165	0\\
29.3192503055876	0\\
29.3189532352988	0\\
29.3186561548494	0\\
29.3183590642387	0\\
29.3180619634661	0\\
29.3177648525307	0\\
29.317467731432	0\\
29.3171706001693	0\\
29.3168734587417	0\\
29.3165763071487	0\\
29.3162791453896	0\\
29.3159819734636	0\\
29.3156847913701	0\\
29.3153875991083	0\\
29.3150903966776	0\\
29.3147931840772	0\\
29.3144959613065	0\\
29.3141987283648	0\\
29.3139014852514	0\\
29.3136042319656	0\\
29.3133069685066	0\\
29.3130096948738	0\\
29.3127124110666	0\\
29.3124151170841	0\\
29.3121178129257	0\\
29.3118204985908	0\\
29.3115231740785	0\\
29.3112258393882	0\\
29.3109284945193	0\\
29.310631139471	0\\
29.3103337742426	0\\
29.3100363988334	0\\
29.3097390132428	0\\
29.3094416174699	0\\
29.3091442115143	0\\
29.308846795375	0\\
29.3085493690515	0\\
29.308251932543	0\\
29.3079544858489	0\\
29.3076570289684	0\\
29.3073595619008	0\\
29.3070620846455	0\\
29.3067645972018	0\\
29.3064670995689	0\\
29.3061695917461	0\\
29.3058720737328	0\\
29.3055745455283	0\\
29.3052770071318	0\\
29.3049794585427	0\\
29.3046818997603	0\\
29.3043843307838	0\\
29.3040867516126	0\\
29.303789162246	0\\
29.3034915626832	0\\
29.3031939529236	0\\
29.3028963329664	0\\
29.3025987028111	0\\
29.3023010624568	0\\
29.3020034119029	0\\
29.3017057511487	0\\
29.3014080801934	0\\
29.3011103990364	0\\
29.300812707677	0\\
29.3005150061145	0\\
29.3002172943482	0\\
29.2999195723773	0\\
29.2996218402012	0\\
29.2993240978192	0\\
29.2990263452306	0\\
29.2987285824347	0\\
29.2984308094307	0\\
29.298133026218	0\\
29.2978352327959	0\\
29.2975374291637	0\\
29.2972396153207	0\\
29.2969417912661	0\\
29.2966439569994	0\\
29.2963461125197	0\\
29.2960482578264	0\\
29.2957503929187	0\\
29.2954525177961	0\\
29.2951546324577	0\\
29.2948567369029	0\\
29.294558831131	0\\
29.2942609151413	0\\
29.293962988933	0\\
29.2936650525055	0\\
29.2933671058581	0\\
29.29306914899	0\\
29.2927711819007	0\\
29.2924732045892	0\\
29.2921752170551	0\\
29.2918772192975	0\\
29.2915792113158	0\\
29.2912811931093	0\\
29.2909831646772	0\\
29.2906851260189	0\\
29.2903870771336	0\\
29.2900890180207	0\\
29.2897909486794	0\\
29.2894928691091	0\\
29.2891947793091	0\\
29.2888966792786	0\\
29.2885985690169	0\\
29.2883004485234	0\\
29.2880023177973	0\\
29.287704176838	0\\
29.2874060256447	0\\
29.2871078642167	0\\
29.2868096925533	0\\
29.2865115106539	0\\
29.2862133185177	0\\
29.2859151161441	0\\
29.2856169035322	0\\
29.2853186806815	0\\
29.2850204475912	0\\
29.2847222042606	0\\
29.284423950689	0\\
29.2841256868757	0\\
29.28382741282	0\\
29.2835291285212	0\\
29.2832308339786	0\\
29.2829325291915	0\\
29.2826342141591	0\\
29.2823358888809	0\\
29.282037553356	0\\
29.2817392075838	0\\
29.2814408515636	0\\
29.2811424852946	0\\
29.2808441087762	0\\
29.2805457220077	0\\
29.2802473249883	0\\
29.2799489177173	0\\
29.2796505001941	0\\
29.279352072418	0\\
29.2790536343882	0\\
29.278755186104	0\\
29.2784567275647	0\\
29.2781582587697	0\\
29.2778597797181	0\\
29.2775612904094	0\\
29.2772627908428	0\\
29.2769642810176	0\\
29.2766657609331	0\\
29.2763672305886	0\\
29.2760686899834	0\\
29.2757701391167	0\\
29.2754715779879	0\\
29.2751730065963	0\\
29.2748744249412	0\\
29.2745758330218	0\\
29.2742772308375	0\\
29.2739786183875	0\\
29.2736799956711	0\\
29.2733813626877	0\\
29.2730827194366	0\\
29.2727840659169	0\\
29.2724854021281	0\\
29.2721867280693	0\\
29.27188804374	0\\
29.2715893491394	0\\
29.2712906442667	0\\
29.2709919291214	0\\
29.2706932037026	0\\
29.2703944680097	0\\
29.270095722042	0\\
29.2697969657987	0\\
29.2694981992792	0\\
29.2691994224827	0\\
29.2689006354086	0\\
29.268601838056	0\\
29.2683030304245	0\\
29.2680042125131	0\\
29.2677053843212	0\\
29.2674065458481	0\\
29.2671076970932	0\\
29.2668088380556	0\\
29.2665099687347	0\\
29.2662110891297	0\\
29.2659121992401	0\\
29.2656132990649	0\\
29.2653143886037	0\\
29.2650154678555	0\\
29.2647165368198	0\\
29.2644175954958	0\\
29.2641186438829	0\\
29.2638196819802	0\\
29.2635207097872	0\\
29.263221727303	0\\
29.262922734527	0\\
29.2626237314585	0\\
29.2623247180967	0\\
29.262025694441	0\\
29.2617266604907	0\\
29.261427616245	0\\
29.2611285617032	0\\
29.2608294968646	0\\
29.2605304217285	0\\
29.2602313362943	0\\
29.2599322405611	0\\
29.2596331345283	0\\
29.2593340181952	0\\
29.2590348915611	0\\
29.2587357546252	0\\
29.2584366073869	0\\
29.2581374498453	0\\
29.257838282	0\\
29.25753910385	0\\
29.2572399153947	0\\
29.2569407166335	0\\
29.2566415075655	0\\
29.2563422881901	0\\
29.2560430585065	0\\
29.2557438185142	0\\
29.2554445682122	0\\
29.2551453076	0\\
29.2548460366768	0\\
29.254546755442	0\\
29.2542474638947	0\\
29.2539481620343	0\\
29.2536488498602	0\\
29.2533495273714	0\\
29.2530501945675	0\\
29.2527508514476	0\\
29.252451498011	0\\
29.252152134257	0\\
29.251852760185	0\\
29.2515533757942	0\\
29.2512539810838	0\\
29.2509545760533	0\\
29.2506551607018	0\\
29.2503557350287	0\\
29.2500562990332	0\\
29.2497568527146	0\\
29.2494573960723	0\\
29.2491579291055	0\\
29.2488584518135	0\\
29.2485589641956	0\\
29.248259466251	0\\
29.2479599579791	0\\
29.2476604393792	0\\
29.2473609104505	0\\
29.2470613711923	0\\
29.2467618216039	0\\
29.2464622616847	0\\
29.2461626914338	0\\
29.2458631108506	0\\
29.2455635199344	0\\
29.2452639186844	0\\
29.2449643070999	0\\
29.2446646851803	0\\
29.2443650529248	0\\
29.2440654103327	0\\
29.2437657574033	0\\
29.2434660941358	0\\
29.2431664205296	0\\
29.242866736584	0\\
29.2425670422981	0\\
29.2422673376714	0\\
29.2419676227031	0\\
29.2416678973925	0\\
29.2413681617389	0\\
29.2410684157415	0\\
29.2407686593997	0\\
29.2404688927127	0\\
29.2401691156798	0\\
29.2398693283003	0\\
29.2395695305736	0\\
29.2392697224988	0\\
29.2389699040752	0\\
29.2386700753023	0\\
29.2383702361791	0\\
29.2380703867051	0\\
29.2377705268795	0\\
29.2374706567015	0\\
29.2371707761706	0\\
29.2368708852859	0\\
29.2365709840467	0\\
29.2362710724524	0\\
29.2359711505022	0\\
29.2356712181954	0\\
29.2353712755313	0\\
29.2350713225091	0\\
29.2347713591282	0\\
29.2344713853879	0\\
29.2341714012873	0\\
29.2338714068259	0\\
29.2335714020029	0\\
29.2332713868175	0\\
29.2329713612691	0\\
29.2326713253569	0\\
29.2323712790803	0\\
29.2320712224385	0\\
29.2317711554308	0\\
29.2314710780564	0\\
29.2311709903148	0\\
29.230870892205	0\\
29.2305707837265	0\\
29.2302706648786	0\\
29.2299705356604	0\\
29.2296703960713	0\\
29.2293702461105	0\\
29.2290700857774	0\\
29.2287699150713	0\\
29.2284697339913	0\\
29.2281695425369	0\\
29.2278693407072	0\\
29.2275691285016	0\\
29.2272689059193	0\\
29.2269686729597	0\\
29.2266684296219	0\\
29.2263681759054	0\\
29.2260679118093	0\\
29.225767637333	0\\
29.2254673524757	0\\
29.2251670572368	0\\
29.2248667516155	0\\
29.224566435611	0\\
29.2242661092227	0\\
29.2239657724499	0\\
29.2236654252918	0\\
29.2233650677477	0\\
29.2230646998169	0\\
29.2227643214986	0\\
29.2224639327923	0\\
29.222163533697	0\\
29.2218631242122	0\\
29.2215627043371	0\\
29.221262274071	0\\
29.2209618334132	0\\
29.2206613823628	0\\
29.2203609209194	0\\
29.220060449082	0\\
29.21975996685	0\\
29.2194594742227	0\\
29.2191589711993	0\\
29.2188584577792	0\\
29.2185579339616	0\\
29.2182573997457	0\\
29.217956855131	0\\
29.2176563001166	0\\
29.2173557347018	0\\
29.2170551588859	0\\
29.2167545726682	0\\
29.216453976048	0\\
29.2161533690246	0\\
29.2158527515972	0\\
29.2155521237651	0\\
29.2152514855276	0\\
29.2149508368839	0\\
29.2146501778335	0\\
29.2143495083754	0\\
29.2140488285091	0\\
29.2137481382337	0\\
29.2134474375486	0\\
29.2131467264531	0\\
29.2128460049464	0\\
29.2125452730278	0\\
29.2122445306966	0\\
29.211943777952	0\\
29.2116430147934	0\\
29.21134224122	0\\
29.2110414572312	0\\
29.2107406628261	0\\
29.210439858004	0\\
29.2101390427643	0\\
29.2098382171062	0\\
29.209537381029	0\\
29.209236534532	0\\
29.2089356776144	0\\
29.2086348102756	0\\
29.2083339325147	0\\
29.2080330443312	0\\
29.2077321457242	0\\
29.207431236693	0\\
29.207130317237	0\\
29.2068293873553	0\\
29.2065284470473	0\\
29.2062274963123	0\\
29.2059265351495	0\\
29.2056255635582	0\\
29.2053245815377	0\\
29.2050235890872	0\\
29.2047225862061	0\\
29.2044215728936	0\\
29.204120549149	0\\
29.2038195149716	0\\
29.2035184703606	0\\
29.2032174153153	0\\
29.202916349835	0\\
29.202615273919	0\\
29.2023141875665	0\\
29.2020130907769	0\\
29.2017119835494	0\\
29.2014108658832	0\\
29.2011097377777	0\\
29.2008085992322	0\\
29.2005074502458	0\\
29.2002062908179	0\\
29.1999051209478	0\\
29.1996039406348	0\\
29.199302749878	0\\
29.1990015486768	0\\
29.1987003370305	0\\
29.1983991149383	0\\
29.1980978823996	0\\
29.1977966394135	0\\
29.1974953859794	0\\
29.1971941220965	0\\
29.1968928477642	0\\
29.1965915629816	0\\
29.1962902677481	0\\
29.1959889620629	0\\
29.1956876459254	0\\
29.1953863193347	0\\
29.1950849822902	0\\
29.1947836347912	0\\
29.1944822768369	0\\
29.1941809084265	0\\
29.1938795295594	0\\
29.1935781402349	0\\
29.1932767404521	0\\
29.1929753302105	0\\
29.1926739095092	0\\
29.1923724783475	0\\
29.1920710367247	0\\
29.1917695846402	0\\
29.191468122093	0\\
29.1911666490827	0\\
29.1908651656083	0\\
29.1905636716691	0\\
29.1902621672646	0\\
29.1899606523938	0\\
29.1896591270562	0\\
29.1893575912509	0\\
29.1890560449773	0\\
29.1887544882346	0\\
29.188452921022	0\\
29.188151343339	0\\
29.1878497551846	0\\
29.1875481565583	0\\
29.1872465474592	0\\
29.1869449278867	0\\
29.1866432978401	0\\
29.1863416573185	0\\
29.1860400063213	0\\
29.1857383448477	0\\
29.1854366728971	0\\
29.1851349904686	0\\
29.1848332975616	0\\
29.1845315941754	0\\
29.1842298803091	0\\
29.1839281559621	0\\
29.1836264211337	0\\
29.1833246758231	0\\
29.1830229200295	0\\
29.1827211537524	0\\
29.1824193769908	0\\
29.1821175897442	0\\
29.1818157920117	0\\
29.1815139837927	0\\
29.1812121650864	0\\
29.1809103358921	0\\
29.1806084962091	0\\
29.1803066460366	0\\
29.1800047853739	0\\
29.1797029142203	0\\
29.179401032575	0\\
29.1790991404373	0\\
29.1787972378065	0\\
29.1784953246819	0\\
29.1781934010627	0\\
29.1778914669482	0\\
29.1775895223377	0\\
29.1772875672304	0\\
29.1769856016256	0\\
29.1766836255226	0\\
29.1763816389206	0\\
29.176079641819	0\\
29.1757776342169	0\\
29.1754756161137	0\\
29.1751735875086	0\\
29.174871548401	0\\
29.1745694987899	0\\
29.1742674386749	0\\
29.173965368055	0\\
29.1736632869296	0\\
29.173361195298	0\\
29.1730590931593	0\\
29.172756980513	0\\
29.1724548573582	0\\
29.1721527236942	0\\
29.1718505795203	0\\
29.1715484248358	0\\
29.1712462596399	0\\
29.1709440839319	0\\
29.170641897711	0\\
29.1703397009766	0\\
29.1700374937279	0\\
29.1697352759641	0\\
29.1694330476846	0\\
29.1691308088886	0\\
29.1688285595754	0\\
29.1685262997442	0\\
29.1682240293943	0\\
29.167921748525	0\\
29.1676194571356	0\\
29.1673171552252	0\\
29.1670148427933	0\\
29.166712519839	0\\
29.1664101863616	0\\
29.1661078423603	0\\
29.1658054878346	0\\
29.1655031227835	0\\
29.1652007472065	0\\
29.1648983611026	0\\
29.1645959644713	0\\
29.1642935573118	0\\
29.1639911396234	0\\
29.1636887114053	0\\
29.1633862726567	0\\
29.163083823377	0\\
29.1627813635654	0\\
29.1624788932212	0\\
29.1621764123437	0\\
29.1618739209321	0\\
29.1615714189857	0\\
29.1612689065037	0\\
29.1609663834854	0\\
29.1606638499301	0\\
29.1603613058371	0\\
29.1600587512056	0\\
29.1597561860348	0\\
29.1594536103241	0\\
29.1591510240728	0\\
29.15884842728	0\\
29.158545819945	0\\
29.1582432020672	0\\
29.1579405736457	0\\
29.1576379346798	0\\
29.1573352851689	0\\
29.1570326251122	0\\
29.1567299545088	0\\
29.1564272733582	0\\
29.1561245816595	0\\
29.1558218794121	0\\
29.1555191666152	0\\
29.155216443268	0\\
29.1549137093698	0\\
29.15461096492	0\\
29.1543082099177	0\\
29.1540054443622	0\\
29.1537026682527	0\\
29.1533998815887	0\\
29.1530970843692	0\\
29.1527942765936	0\\
29.1524914582611	0\\
29.1521886293711	0\\
29.1518857899227	0\\
29.1515829399153	0\\
29.151280079348	0\\
29.1509772082202	0\\
29.1506743265311	0\\
29.15037143428	0\\
29.1500685314661	0\\
29.1497656180888	0\\
29.1494626941472	0\\
29.1491597596406	0\\
29.1488568145684	0\\
29.1485538589297	0\\
29.1482508927238	0\\
29.1479479159501	0\\
29.1476449286076	0\\
29.1473419306958	0\\
29.1470389222139	0\\
29.1467359031611	0\\
29.1464328735367	0\\
29.1461298333399	0\\
29.1458267825701	0\\
29.1455237212265	0\\
29.1452206493083	0\\
29.1449175668149	0\\
29.1446144737454	0\\
29.1443113700991	0\\
29.1440082558754	0\\
29.1437051310734	0\\
29.1434019956924	0\\
29.1430988497317	0\\
29.1427956931905	0\\
29.1424925260682	0\\
29.1421893483639	0\\
29.1418861600769	0\\
29.1415829612066	0\\
29.141279751752	0\\
29.1409765317126	0\\
29.1406733010876	0\\
29.1403700598762	0\\
29.1400668080777	0\\
29.1397635456913	0\\
29.1394602727164	0\\
29.1391569891521	0\\
29.1388536949978	0\\
29.1385503902526	0\\
29.1382470749159	0\\
29.137943748987	0\\
29.137640412465	0\\
29.1373370653492	0\\
29.137033707639	0\\
29.1367303393335	0\\
29.136426960432	0\\
29.1361235709338	0\\
29.1358201708381	0\\
29.1355167601442	0\\
29.1352133388514	0\\
29.1349099069589	0\\
29.134606464466	0\\
29.1343030113719	0\\
29.1339995476758	0\\
29.1336960733772	0\\
29.1333925884752	0\\
29.133089092969	0\\
29.1327855868579	0\\
29.1324820701413	0\\
29.1321785428183	0\\
29.1318750048882	0\\
29.1315714563502	0\\
29.1312678972037	0\\
29.1309643274479	0\\
29.130660747082	0\\
29.1303571561053	0\\
29.130053554517	0\\
29.1297499423165	0\\
29.1294463195029	0\\
29.1291426860756	0\\
29.1288390420337	0\\
29.1285353873766	0\\
29.1282317221035	0\\
29.1279280462137	0\\
29.1276243597064	0\\
29.1273206625809	0\\
29.1270169548364	0\\
29.1267132364721	0\\
29.1264095074875	0\\
29.1261057678816	0\\
29.1258020176538	0\\
29.1254982568033	0\\
29.1251944853294	0\\
29.1248907032314	0\\
29.1245869105084	0\\
29.1242831071598	0\\
29.1239792931847	0\\
29.1236754685826	0\\
29.1233716333525	0\\
29.1230677874939	0\\
29.1227639310058	0\\
29.1224600638877	0\\
29.1221561861387	0\\
29.121852297758	0\\
29.1215483987451	0\\
29.121244489099	0\\
29.1209405688191	0\\
29.1206366379047	0\\
29.1203326963549	0\\
29.120028744169	0\\
29.1197247813464	0\\
29.1194208078862	0\\
29.1191168237877	0\\
29.1188128290501	0\\
29.1185088236727	0\\
29.1182048076549	0\\
29.1179007809957	0\\
29.1175967436945	0\\
29.1172926957506	0\\
29.1169886371631	0\\
29.1166845679314	0\\
29.1163804880546	0\\
29.1160763975321	0\\
29.1157722963632	0\\
29.1154681845469	0\\
29.1151640620827	0\\
29.1148599289698	0\\
29.1145557852074	0\\
29.1142516307947	0\\
29.1139474657311	0\\
29.1136432900158	0\\
29.113339103648	0\\
29.113034906627	0\\
29.1127306989521	0\\
29.1124264806224	0\\
29.1121222516373	0\\
29.111818011996	0\\
29.1115137616978	0\\
29.1112095007419	0\\
29.1109052291276	0\\
29.110600946854	0\\
29.1102966539206	0\\
29.1099923503265	0\\
29.1096880360709	0\\
29.1093837111532	0\\
29.1090793755726	0\\
29.1087750293283	0\\
29.1084706724196	0\\
29.1081663048457	0\\
29.107861926606	0\\
29.1075575376995	0\\
29.1072531381257	0\\
29.1069487278838	0\\
29.1066443069729	0\\
29.1063398753924	0\\
29.1060354331415	0\\
29.1057309802195	0\\
29.1054265166256	0\\
29.105122042359	0\\
29.1048175574191	0\\
29.104513061805	0\\
29.1042085555161	0\\
29.1039040385515	0\\
29.1035995109106	0\\
29.1032949725925	0\\
29.1029904235966	0\\
29.102685863922	0\\
29.102381293568	0\\
29.102076712534	0\\
29.101772120819	0\\
29.1014675184225	0\\
29.1011629053436	0\\
29.1008582815815	0\\
29.1005536471356	0\\
29.1002490020051	0\\
29.0999443461892	0\\
29.0996396796872	0\\
29.0993350024983	0\\
29.0990303146218	0\\
29.098725616057	0\\
29.098420906803	0\\
29.0981161868592	0\\
29.0978114562247	0\\
29.0975067148989	0\\
29.097201962881	0\\
29.0968972001702	0\\
29.0965924267658	0\\
29.096287642667	0\\
29.0959828478731	0\\
29.0956780423834	0\\
29.095373226197	0\\
29.0950683993132	0\\
29.0947635617314	0\\
29.0944587134506	0\\
29.0941538544703	0\\
29.0938489847895	0\\
29.0935441044077	0\\
29.0932392133239	0\\
29.0929343115376	0\\
29.0926293990478	0\\
29.092324475854	0\\
29.0920195419552	0\\
29.0917145973508	0\\
29.0914096420401	0\\
29.0911046760222	0\\
29.0907996992964	0\\
29.0904947118619	0\\
29.0901897137181	0\\
29.0898847048641	0\\
29.0895796852993	0\\
29.0892746550228	0\\
29.0889696140338	0\\
29.0886645623318	0\\
29.0883594999158	0\\
29.0880544267852	0\\
29.0877493429391	0\\
29.0874442483769	0\\
29.0871391430978	0\\
29.086834027101	0\\
29.0865289003857	0\\
29.0862237629513	0\\
29.085918614797	0\\
29.085613455922	0\\
29.0853082863255	0\\
29.0850031060069	0\\
29.0846979149653	0\\
29.0843927132	0\\
29.0840875007103	0\\
29.0837822774954	0\\
29.0834770435545	0\\
29.0831717988869	0\\
29.0828665434918	0\\
29.0825612773685	0\\
29.0822560005163	0\\
29.0819507129343	0\\
29.0816454146218	0\\
29.0813401055781	0\\
29.0810347858024	0\\
29.0807294552939	0\\
29.080424114052	0\\
29.0801187620758	0\\
29.0798133993646	0\\
29.0795080259176	0\\
29.0792026417341	0\\
29.0788972468134	0\\
29.0785918411546	0\\
29.078286424757	0\\
29.0779809976199	0\\
29.0776755597425	0\\
29.077370111124	0\\
29.0770646517637	0\\
29.0767591816609	0\\
29.0764537008147	0\\
29.0761482092245	0\\
29.0758427068895	0\\
29.0755371938088	0\\
29.0752316699819	0\\
29.0749261354078	0\\
29.0746205900859	0\\
29.0743150340153	0\\
29.0740094671954	0\\
29.0737038896254	0\\
29.0733983013045	0\\
29.073092702232	0\\
29.0727870924071	0\\
29.072481471829	0\\
29.072175840497	0\\
29.0718701984104	0\\
29.0715645455684	0\\
29.0712588819701	0\\
29.070953207615	0\\
29.0706475225021	0\\
29.0703418266308	0\\
29.0700361200004	0\\
29.0697304026099	0\\
29.0694246744587	0\\
29.0691189355461	0\\
29.0688131858712	0\\
29.0685074254333	0\\
29.0682016542317	0\\
29.0678958722655	0\\
29.0675900795341	0\\
29.0672842760367	0\\
29.0669784617725	0\\
29.0666726367408	0\\
29.0663668009408	0\\
29.0660609543717	0\\
29.0657550970328	0\\
29.0654492289233	0\\
29.0651433500425	0\\
29.0648374603896	0\\
29.0645315599638	0\\
29.0642256487645	0\\
29.0639197267907	0\\
29.0636137940419	0\\
29.0633078505171	0\\
29.0630018962158	0\\
29.062695931137	0\\
29.06238995528	0\\
29.0620839686442	0\\
29.0617779712286	0\\
29.0614719630327	0\\
29.0611659440555	0\\
29.0608599142963	0\\
29.0605538737545	0\\
29.0602478224291	0\\
29.0599417603195	0\\
29.0596356874249	0\\
29.0593296037446	0\\
29.0590235092777	0\\
29.0587174040236	0\\
29.0584112879814	0\\
29.0581051611504	0\\
29.0577990235299	0\\
29.057492875119	0\\
29.057186715917	0\\
29.0568805459232	0\\
29.0565743651369	0\\
29.0562681735571	0\\
29.0559619711832	0\\
29.0556557580144	0\\
29.05534953405	0\\
29.0550432992892	0\\
29.0547370537312	0\\
29.0544307973752	0\\
29.0541245302206	0\\
29.0538182522665	0\\
29.0535119635122	0\\
29.0532056639569	0\\
29.0528993535999	0\\
29.0525930324403	0\\
29.0522867004776	0\\
29.0519803577107	0\\
29.0516740041391	0\\
29.051367639762	0\\
29.0510612645785	0\\
29.050754878588	0\\
29.0504484817896	0\\
29.0501420741826	0\\
29.0498356557663	0\\
29.0495292265398	0\\
29.0492227865024	0\\
29.0489163356534	0\\
29.048609873992	0\\
29.0483034015174	0\\
29.0479969182288	0\\
29.0476904241256	0\\
29.0473839192069	0\\
29.0470774034719	0\\
29.04677087692	0\\
29.0464643395503	0\\
29.0461577913621	0\\
29.0458512323546	0\\
29.045544662527	0\\
29.0452380818787	0\\
29.0449314904087	0\\
29.0446248881165	0\\
29.0443182750011	0\\
29.0440116510618	0\\
29.043705016298	0\\
29.0433983707087	0\\
29.0430917142933	0\\
29.0427850470509	0\\
29.0424783689809	0\\
29.0421716800823	0\\
29.0418649803546	0\\
29.0415582697969	0\\
29.0412515484085	0\\
29.0409448161885	0\\
29.0406380731363	0\\
29.040331319251	0\\
29.0400245545319	0\\
29.0397177789782	0\\
29.0394109925892	0\\
29.0391041953641	0\\
29.0387973873021	0\\
29.0384905684024	0\\
29.0381837386644	0\\
29.0378768980872	0\\
29.0375700466701	0\\
29.0372631844123	0\\
29.036956311313	0\\
29.0366494273714	0\\
29.0363425325869	0\\
29.0360356269586	0\\
29.0357287104858	0\\
29.0354217831677	0\\
29.0351148450035	0\\
29.0348078959925	0\\
29.0345009361339	0\\
29.0341939654269	0\\
29.0338869838708	0\\
29.0335799914648	0\\
29.0332729882082	0\\
29.0329659741001	0\\
29.0326589491398	0\\
29.0323519133266	0\\
29.0320448666596	0\\
29.0317378091381	0\\
29.0314307407613	0\\
29.0311236615286	0\\
29.030816571439	0\\
29.0305094704918	0\\
29.0302023586863	0\\
29.0298952360217	0\\
29.0295881024972	0\\
29.0292809581121	0\\
29.0289738028656	0\\
29.0286666367569	0\\
29.0283594597852	0\\
29.0280522719499	0\\
29.02774507325	0\\
29.0274378636849	0\\
29.0271306432538	0\\
29.0268234119559	0\\
29.0265161697904	0\\
29.0262089167566	0\\
29.0259016528537	0\\
29.0255943780809	0\\
29.0252870924375	0\\
29.0249797959228	0\\
29.0246724885358	0\\
29.0243651702759	0\\
29.0240578411423	0\\
29.0237505011341	0\\
29.0234431502508	0\\
29.0231357884914	0\\
29.0228284158553	0\\
29.0225210323415	0\\
29.0222136379495	0\\
29.0219062326783	0\\
29.0215988165273	0\\
29.0212913894957	0\\
29.0209839515826	0\\
29.0206765027873	0\\
29.0203690431091	0\\
29.0200615725472	0\\
29.0197540911008	0\\
29.0194465987691	0\\
29.0191390955514	0\\
29.0188315814469	0\\
29.0185240564548	0\\
29.0182165205744	0\\
29.0179089738049	0\\
29.0176014161454	0\\
29.0172938475953	0\\
29.0169862681538	0\\
29.0166786778201	0\\
29.0163710765934	0\\
29.0160634644729	0\\
29.015755841458	0\\
29.0154482075477	0\\
29.0151405627414	0\\
29.0148329070383	0\\
29.0145252404375	0\\
29.0142175629384	0\\
29.0139098745401	0\\
29.0136021752419	0\\
29.013294465043	0\\
29.0129867439427	0\\
29.0126790119401	0\\
29.0123712690345	0\\
29.0120635152251	0\\
29.0117557505112	0\\
29.0114479748919	0\\
29.0111401883666	0\\
29.0108323909343	0\\
29.0105245825945	0\\
29.0102167633462	0\\
29.0099089331887	0\\
29.0096010921212	0\\
29.009293240143	0\\
29.0089853772533	0\\
29.0086775034513	0\\
29.0083696187362	0\\
29.0080617231073	0\\
29.0077538165638	0\\
29.0074458991049	0\\
29.0071379707299	0\\
29.0068300314379	0\\
29.0065220812282	0\\
29.0062141201001	0\\
29.0059061480527	0\\
29.0055981650853	0\\
29.005290171197	0\\
29.0049821663872	0\\
29.0046741506551	0\\
29.0043661239998	0\\
29.0040580864206	0\\
29.0037500379167	0\\
29.0034419784874	0\\
29.0031339081319	0\\
29.0028258268494	0\\
29.0025177346391	0\\
29.0022096315002	0\\
29.001901517432	0\\
29.0015933924338	0\\
29.0012852565046	0\\
29.0009771096438	0\\
29.0006689518506	0\\
29.0003607831241	0\\
29.0000526034637	0\\
28.9997444128686	0\\
28.9994362113379	0\\
28.9991279988708	0\\
28.9988197754668	0\\
28.9985115411248	0\\
28.9982032958442	0\\
28.9978950396242	0\\
28.997586772464	0\\
28.9972784943629	0\\
28.99697020532	0\\
28.9966619053346	0\\
28.9963535944059	0\\
28.9960452725331	0\\
28.9957369397155	0\\
28.9954285959522	0\\
28.9951202412426	0\\
28.9948118755857	0\\
28.9945034989809	0\\
28.9941951114274	0\\
28.9938867129244	0\\
28.993578303471	0\\
28.9932698830666	0\\
28.9929614517104	0\\
28.9926530094015	0\\
28.9923445561392	0\\
28.9920360919228	0\\
28.9917276167514	0\\
28.9914191306242	0\\
28.9911106335406	0\\
28.9908021254996	0\\
28.9904936065006	0\\
28.9901850765427	0\\
28.9898765356252	0\\
28.9895679837473	0\\
28.9892594209082	0\\
28.9889508471072	0\\
28.9886422623434	0\\
28.988333666616	0\\
28.9880250599244	0\\
28.9877164422677	0\\
28.9874078136452	0\\
28.987099174056	0\\
28.9867905234994	0\\
28.9864818619745	0\\
28.9861731894807	0\\
28.9858645060172	0\\
28.9855558115831	0\\
28.9852471061777	0\\
28.9849383898002	0\\
28.9846296624499	0\\
28.9843209241258	0\\
28.9840121748274	0\\
28.9837034145537	0\\
28.983394643304	0\\
28.9830858610776	0\\
28.9827770678736	0\\
28.9824682636912	0\\
28.9821594485297	0\\
28.9818506223884	0\\
28.9815417852663	0\\
28.9812329371628	0\\
28.9809240780771	0\\
28.9806152080083	0\\
28.9803063269557	0\\
28.9799974349186	0\\
28.979688531896	0\\
28.9793796178874	0\\
28.9790706928918	0\\
28.9787617569084	0\\
28.9784528099366	0\\
28.9781438519755	0\\
28.9778348830244	0\\
28.9775259030824	0\\
28.9772169121488	0\\
28.9769079102228	0\\
28.9765988973036	0\\
28.9762898733905	0\\
28.9759808384826	0\\
28.9756717925791	0\\
28.9753627356794	0\\
28.9750536677825	0\\
28.9747445888878	0\\
28.9744354989944	0\\
28.9741263981016	0\\
28.9738172862085	0\\
28.9735081633145	0\\
28.9731990294186	0\\
28.9728898845202	0\\
28.9725807286184	0\\
28.9722715617124	0\\
28.9719623838016	0\\
28.971653194885	0\\
28.9713439949619	0\\
28.9710347840316	0\\
28.9707255620932	0\\
28.9704163291459	0\\
28.970107085189	0\\
28.9697978302217	0\\
28.9694885642432	0\\
28.9691792872527	0\\
28.9688699992495	0\\
28.9685607002327	0\\
28.9682513902016	0\\
28.9679420691554	0\\
28.9676327370932	0\\
28.9673233940144	0\\
28.9670140399181	0\\
28.9667046748035	0\\
28.9663952986699	0\\
28.9660859115165	0\\
28.9657765133424	0\\
28.965467104147	0\\
28.9651576839294	0\\
28.9648482526888	0\\
28.9645388104244	0\\
28.9642293571356	0\\
28.9639198928214	0\\
28.963610417481	0\\
28.9633009311138	0\\
28.9629914337189	0\\
28.9626819252956	0\\
28.962372405843	0\\
28.9620628753603	0\\
28.9617533338468	0\\
28.9614437813017	0\\
28.9611342177242	0\\
28.9608246431136	0\\
28.9605150574689	0\\
28.9602054607895	0\\
28.9598958530746	0\\
28.9595862343233	0\\
28.9592766045349	0\\
28.9589669637086	0\\
28.9586573118436	0\\
28.9583476489391	0\\
28.9580379749944	0\\
28.9577282900086	0\\
28.957418593981	0\\
28.9571088869107	0\\
28.956799168797	0\\
28.9564894396392	0\\
28.9561796994363	0\\
28.9558699481876	0\\
28.9555601858924	0\\
28.9552504125499	0\\
28.9549406281591	0\\
28.9546308327195	0\\
28.9543210262301	0\\
28.9540112086903	0\\
28.9537013800991	0\\
28.9533915404559	0\\
28.9530816897597	0\\
28.95277182801	0\\
28.9524619552057	0\\
28.9521520713463	0\\
28.9518421764308	0\\
28.9515322704584	0\\
28.9512223534285	0\\
28.9509124253402	0\\
28.9506024861927	0\\
28.9502925359853	0\\
28.949982574717	0\\
28.9496726023872	0\\
28.9493626189951	0\\
28.9490526245399	0\\
28.9487426190207	0\\
28.9484326024368	0\\
28.9481225747874	0\\
28.9478125360717	0\\
28.947502486289	0\\
28.9471924254384	0\\
28.9468823535191	0\\
28.9465722705303	0\\
28.9462621764714	0\\
28.9459520713414	0\\
28.9456419551395	0\\
28.9453318278651	0\\
28.9450216895172	0\\
28.9447115400952	0\\
28.9444013795982	0\\
28.9440912080254	0\\
28.943781025376	0\\
28.9434708316493	0\\
28.9431606268444	0\\
28.9428504109606	0\\
28.942540183997	0\\
28.9422299459529	0\\
28.9419196968275	0\\
28.9416094366201	0\\
28.9412991653297	0\\
28.9409888829556	0\\
28.940678589497	0\\
28.9403682849532	0\\
28.9400579693232	0\\
28.9397476426065	0\\
28.9394373048021	0\\
28.9391269559092	0\\
28.9388165959271	0\\
28.938506224855	0\\
28.938195842692	0\\
28.9378854494375	0\\
28.9375750450905	0\\
28.9372646296503	0\\
28.9369542031162	0\\
28.9366437654872	0\\
28.9363333167627	0\\
28.9360228569418	0\\
28.9357123860238	0\\
28.9354019040078	0\\
28.935091410893	0\\
28.9347809066787	0\\
28.934470391364	0\\
28.9341598649482	0\\
28.9338493274305	0\\
28.9335387788101	0\\
28.9332282190862	0\\
28.9329176482579	0\\
28.9326070663246	0\\
28.9322964732853	0\\
28.9319858691394	0\\
28.931675253886	0\\
28.9313646275243	0\\
28.9310539900535	0\\
28.9307433414729	0\\
28.9304326817816	0\\
28.9301220109788	0\\
28.9298113290638	0\\
28.9295006360357	0\\
28.9291899318938	0\\
28.9288792166373	0\\
28.9285684902653	0\\
28.9282577527771	0\\
28.9279470041719	0\\
28.9276362444488	0\\
28.9273254736072	0\\
28.9270146916461	0\\
28.9267038985648	0\\
28.9263930943626	0\\
28.9260822790385	0\\
28.9257714525918	0\\
28.9254606150217	0\\
28.9251497663275	0\\
28.9248389065083	0\\
28.9245280355633	0\\
28.9242171534917	0\\
28.9239062602927	0\\
28.9235953559656	0\\
28.9232844405095	0\\
28.9229735139237	0\\
28.9226625762073	0\\
28.9223516273595	0\\
28.9220406673796	0\\
28.9217296962667	0\\
28.9214187140201	0\\
28.9211077206389	0\\
28.9207967161224	0\\
28.9204857004698	0\\
28.9201746736802	0\\
28.9198636357529	0\\
28.9195525866871	0\\
28.9192415264819	0\\
28.9189304551366	0\\
28.9186193726504	0\\
28.9183082790225	0\\
28.917997174252	0\\
28.9176860583382	0\\
28.9173749312803	0\\
28.9170637930775	0\\
28.916752643729	0\\
28.9164414832339	0\\
28.9161303115916	0\\
28.9158191288011	0\\
28.9155079348617	0\\
28.9151967297726	0\\
28.914885513533	0\\
28.9145742861421	0\\
28.9142630475991	0\\
28.9139517979032	0\\
28.9136405370536	0\\
28.9133292650495	0\\
28.91301798189	0\\
28.9127066875745	0\\
28.9123953821021	0\\
28.912084065472	0\\
28.9117727376833	0\\
28.9114613987354	0\\
28.9111500486273	0\\
28.9108386873584	0\\
28.9105273149278	0\\
28.9102159313346	0\\
28.9099045365782	0\\
28.9095931306576	0\\
28.9092817135722	0\\
28.908970285321	0\\
28.9086588459033	0\\
28.9083473953184	0\\
28.9080359335653	0\\
28.9077244606433	0\\
28.9074129765516	0\\
28.9071014812894	0\\
28.9067899748559	0\\
28.9064784572502	0\\
28.9061669284717	0\\
28.9058553885194	0\\
28.9055438373927	0\\
28.9052322750906	0\\
28.9049207016123	0\\
28.9046091169572	0\\
28.9042975211243	0\\
28.903985914113	0\\
28.9036742959222	0\\
28.9033626665514	0\\
28.9030510259996	0\\
28.9027393742661	0\\
28.90242771135	0\\
28.9021160372506	0\\
28.9018043519671	0\\
28.9014926554986	0\\
28.9011809478444	0\\
28.9008692290036	0\\
28.9005574989755	0\\
28.9002457577592	0\\
28.8999340053539	0\\
28.8996222417589	0\\
28.8993104669733	0\\
28.8989986809964	0\\
28.8986868838272	0\\
28.8983750754651	0\\
28.8980632559092	0\\
28.8977514251588	0\\
28.8974395832129	0\\
28.8971277300709	0\\
28.8968158657318	0\\
28.896503990195	0\\
28.8961921034595	0\\
28.8958802055246	0\\
28.8955682963896	0\\
28.8952563760535	0\\
28.8949444445156	0\\
28.894632501775	0\\
28.894320547831	0\\
28.8940085826828	0\\
28.8936966063296	0\\
28.8933846187705	0\\
28.8930726200047	0\\
28.8927606100315	0\\
28.8924485888501	0\\
28.8921365564595	0\\
28.8918245128591	0\\
28.8915124580481	0\\
28.8912003920255	0\\
28.8908883147907	0\\
28.8905762263427	0\\
28.8902641266809	0\\
28.8899520158043	0\\
28.8896398937122	0\\
28.8893277604039	0\\
28.8890156158784	0\\
28.8887034601349	0\\
28.8883912931727	0\\
28.888079114991	0\\
28.887766925589	0\\
28.8874547249657	0\\
28.8871425131206	0\\
28.8868302900526	0\\
28.8865180557611	0\\
28.8862058102451	0\\
28.885893553504	0\\
28.8855812855369	0\\
28.885269006343	0\\
28.8849567159215	0\\
28.8846444142716	0\\
28.8843321013924	0\\
28.8840197772832	0\\
28.8837074419431	0\\
28.8833950953715	0\\
28.8830827375673	0\\
28.8827703685299	0\\
28.8824579882584	0\\
28.882145596752	0\\
28.8818331940099	0\\
28.8815207800314	0\\
28.8812083548155	0\\
28.8808959183615	0\\
28.8805834706686	0\\
28.880271011736	0\\
28.8799585415628	0\\
28.8796460601483	0\\
28.8793335674916	0\\
28.8790210635919	0\\
28.8787085484485	0\\
28.8783960220605	0\\
28.8780834844271	0\\
28.8777709355474	0\\
28.8774583754208	0\\
28.8771458040464	0\\
28.8768332214233	0\\
28.8765206275507	0\\
28.876208022428	0\\
28.8758954060541	0\\
28.8755827784284	0\\
28.87527013955	0\\
28.8749574894181	0\\
28.8746448280319	0\\
28.8743321553906	0\\
28.8740194714934	0\\
28.8737067763395	0\\
28.873394069928	0\\
28.8730813522581	0\\
28.8727686233291	0\\
28.8724558831401	0\\
28.8721431316903	0\\
28.8718303689789	0\\
28.8715175950052	0\\
28.8712048097681	0\\
28.8708920132671	0\\
28.8705792055012	0\\
28.8702663864697	0\\
28.8699535561717	0\\
28.8696407146064	0\\
28.8693278617731	0\\
28.8690149976708	0\\
28.8687021222988	0\\
28.8683892356563	0\\
28.8680763377425	0\\
28.8677634285565	0\\
28.8674505080976	0\\
28.8671375763649	0\\
28.8668246333576	0\\
28.8665116790749	0\\
28.866198713516	0\\
28.8658857366801	0\\
28.8655727485663	0\\
28.8652597491739	0\\
28.864946738502	0\\
28.8646337165499	0\\
28.8643206833167	0\\
28.8640076388015	0\\
28.8636945830037	0\\
28.8633815159223	0\\
28.8630684375566	0\\
28.8627553479058	0\\
28.8624422469689	0\\
28.8621291347453	0\\
28.8618160112341	0\\
28.8615028764345	0\\
28.8611897303457	0\\
28.8608765729668	0\\
28.8605634042971	0\\
28.8602502243357	0\\
28.8599370330818	0\\
28.8596238305347	0\\
28.8593106166934	0\\
28.8589973915572	0\\
28.8586841551253	0\\
28.8583709073968	0\\
28.858057648371	0\\
28.857744378047	0\\
28.857431096424	0\\
28.8571178035011	0\\
28.8568044992777	0\\
28.8564911837528	0\\
28.8561778569256	0\\
28.8558645187954	0\\
28.8555511693613	0\\
28.8552378086224	0\\
28.8549244365781	0\\
28.8546110532274	0\\
28.8542976585696	0\\
28.8539842526038	0\\
28.8536708353292	0\\
28.8533574067451	0\\
28.8530439668505	0\\
28.8527305156446	0\\
28.8524170531268	0\\
28.852103579296	0\\
28.8517900941516	0\\
28.8514765976927	0\\
28.8511630899185	0\\
28.8508495708281	0\\
28.8505360404208	0\\
28.8502224986957	0\\
28.849908945652	0\\
28.849595381289	0\\
28.8492818056057	0\\
28.8489682186014	0\\
28.8486546202752	0\\
28.8483410106264	0\\
28.848027389654	0\\
28.8477137573574	0\\
28.8474001137357	0\\
28.847086458788	0\\
28.8467727925135	0\\
28.8464591149115	0\\
28.8461454259811	0\\
28.8458317257214	0\\
28.8455180141318	0\\
28.8452042912112	0\\
28.844890556959	0\\
28.8445768113744	0\\
28.8442630544564	0\\
28.8439492862043	0\\
28.8436355066172	0\\
28.8433217156944	0\\
28.843007913435	0\\
28.8426940998382	0\\
28.8423802749032	0\\
28.8420664386291	0\\
28.8417525910152	0\\
28.8414387320606	0\\
28.8411248617645	0\\
28.8408109801261	0\\
28.8404970871445	0\\
28.840183182819	0\\
28.8398692671488	0\\
28.8395553401329	0\\
28.8392414017706	0\\
28.838927452061	0\\
28.8386134910034	0\\
28.8382995185969	0\\
28.8379855348408	0\\
28.8376715397341	0\\
28.837357533276	0\\
28.8370435154658	0\\
28.8367294863027	0\\
28.8364154457857	0\\
28.8361013939141	0\\
28.835787330687	0\\
28.8354732561037	0\\
28.8351591701633	0\\
28.834845072865	0\\
28.8345309642079	0\\
28.8342168441913	0\\
28.8339027128143	0\\
28.8335885700762	0\\
28.833274415976	0\\
28.832960250513	0\\
28.8326460736863	0\\
28.8323318854951	0\\
28.8320176859386	0\\
28.831703475016	0\\
28.8313892527265	0\\
28.8310750190691	0\\
28.8307607740432	0\\
28.8304465176479	0\\
28.8301322498823	0\\
28.8298179707457	0\\
28.8295036802371	0\\
28.8291893783559	0\\
28.8288750651011	0\\
28.828560740472	0\\
28.8282464044677	0\\
28.8279320570874	0\\
28.8276176983302	0\\
28.8273033281954	0\\
28.8269889466822	0\\
28.8266745537896	0\\
28.826360149517	0\\
28.8260457338633	0\\
28.825731306828	0\\
28.82541686841	0\\
28.8251024186086	0\\
28.8247879574229	0\\
28.8244734848523	0\\
28.8241590008957	0\\
28.8238445055524	0\\
28.8235299988215	0\\
28.8232154807023	0\\
28.8229009511939	0\\
28.8225864102955	0\\
28.8222718580063	0\\
28.8219572943254	0\\
28.821642719252	0\\
28.8213281327853	0\\
28.8210135349244	0\\
28.8206989256686	0\\
28.820384305017	0\\
28.8200696729688	0\\
28.8197550295231	0\\
28.8194403746792	0\\
28.8191257084362	0\\
28.8188110307933	0\\
28.8184963417496	0\\
28.8181816413044	0\\
28.8178669294567	0\\
28.8175522062058	0\\
28.8172374715509	0\\
28.8169227254911	0\\
28.8166079680256	0\\
28.8162931991536	0\\
28.8159784188742	0\\
28.8156636271867	0\\
28.8153488240901	0\\
28.8150340095837	0\\
28.8147191836666	0\\
28.8144043463381	0\\
28.8140894975972	0\\
28.8137746374432	0\\
28.8134597658752	0\\
28.8131448828924	0\\
28.8128299884939	0\\
28.8125150826791	0\\
28.8122001654469	0\\
28.8118852367966	0\\
28.8115702967274	0\\
28.8112553452384	0\\
28.8109403823288	0\\
28.8106254079978	0\\
28.8103104222446	0\\
28.8099954250682	0\\
28.809680416468	0\\
28.809365396443	0\\
28.8090503649924	0\\
28.8087353221155	0\\
28.8084202678113	0\\
28.8081052020791	0\\
28.807790124918	0\\
28.8074750363271	0\\
28.8071599363058	0\\
28.806844824853	0\\
28.8065297019681	0\\
28.8062145676501	0\\
28.8058994218983	0\\
28.8055842647118	0\\
28.8052690960897	0\\
28.8049539160313	0\\
28.8046387245358	0\\
28.8043235216022	0\\
28.8040083072298	0\\
28.8036930814177	0\\
28.8033778441651	0\\
28.8030625954712	0\\
28.8027473353351	0\\
28.8024320637561	0\\
28.8021167807332	0\\
28.8018014862656	0\\
28.8014861803526	0\\
28.8011708629933	0\\
28.8008555341868	0\\
28.8005401939323	0\\
28.8002248422291	0\\
28.7999094790762	0\\
28.7995941044728	0\\
28.7992787184181	0\\
28.7989633209113	0\\
28.7986479119515	0\\
28.7983324915379	0\\
28.7980170596697	0\\
28.797701616346	0\\
28.7973861615661	0\\
28.797070695329	0\\
28.7967552176339	0\\
28.7964397284801	0\\
28.7961242278666	0\\
28.7958087157927	0\\
28.7954931922575	0\\
28.7951776572601	0\\
28.7948621107999	0\\
28.7945465528758	0\\
28.7942309834871	0\\
28.793915402633	0\\
28.7935998103125	0\\
28.793284206525	0\\
28.7929685912695	0\\
28.7926529645452	0\\
28.7923373263513	0\\
28.7920216766869	0\\
28.7917060155513	0\\
28.7913903429435	0\\
28.7910746588628	0\\
28.7907589633083	0\\
28.7904432562791	0\\
28.7901275377746	0\\
28.7898118077937	0\\
28.7894960663357	0\\
28.7891803133997	0\\
28.7888645489849	0\\
28.7885487730905	0\\
28.7882329857157	0\\
28.7879171868595	0\\
28.7876013765213	0\\
28.7872855547	0\\
28.7869697213949	0\\
28.7866538766053	0\\
28.7863380203301	0\\
28.7860221525686	0\\
28.78570627332	0\\
28.7853903825834	0\\
28.785074480358	0\\
28.7847585666429	0\\
28.7844426414373	0\\
28.7841267047404	0\\
28.7838107565514	0\\
28.7834947968694	0\\
28.7831788256935	0\\
28.782862843023	0\\
28.7825468488569	0\\
28.7822308431945	0\\
28.781914826035	0\\
28.7815987973774	0\\
28.781282757221	0\\
28.7809667055649	0\\
28.7806506424083	0\\
28.7803345677503	0\\
28.7800184815901	0\\
28.7797023839268	0\\
28.7793862747597	0\\
28.7790701540879	0\\
28.7787540219106	0\\
28.7784378782268	0\\
28.7781217230359	0\\
28.7778055563368	0\\
28.7774893781289	0\\
28.7771731884112	0\\
28.776856987183	0\\
28.7765407744434	0\\
28.7762245501915	0\\
28.7759083144265	0\\
28.7755920671476	0\\
28.7752758083539	0\\
28.7749595380446	0\\
28.7746432562189	0\\
28.7743269628759	0\\
28.7740106580147	0\\
28.7736943416346	0\\
28.7733780137347	0\\
28.7730616743142	0\\
28.7727453233722	0\\
28.7724289609079	0\\
28.7721125869204	0\\
28.771796201409	0\\
28.7714798043727	0\\
28.7711633958107	0\\
28.7708469757222	0\\
28.7705305441064	0\\
28.7702141009623	0\\
28.7698976462893	0\\
28.7695811800863	0\\
28.7692647023527	0\\
28.7689482130875	0\\
28.7686317122899	0\\
28.768315199959	0\\
28.7679986760941	0\\
28.7676821406943	0\\
28.7673655937587	0\\
28.7670490352865	0\\
28.7667324652769	0\\
28.766415883729	0\\
28.766099290642	0\\
28.765782686015	0\\
28.7654660698472	0\\
28.7651494421378	0\\
28.7648328028858	0\\
28.7645161520906	0\\
28.7641994897512	0\\
28.7638828158667	0\\
28.7635661304365	0\\
28.7632494334595	0\\
28.762932724935	0\\
28.7626160048621	0\\
28.76229927324	0\\
28.7619825300678	0\\
28.7616657753447	0\\
28.7613490090699	0\\
28.7610322312425	0\\
28.7607154418616	0\\
28.7603986409265	0\\
28.7600818284362	0\\
28.75976500439	0\\
28.759448168787	0\\
28.7591313216263	0\\
28.7588144629071	0\\
28.7584975926286	0\\
28.75818071079	0\\
28.7578638173903	0\\
28.7575469124287	0\\
28.7572299959044	0\\
28.7569130678166	0\\
28.7565961281644	0\\
28.7562791769469	0\\
28.7559622141634	0\\
28.7556452398129	0\\
28.7553282538946	0\\
28.7550112564078	0\\
28.7546942473514	0\\
28.7543772267248	0\\
28.754060194527	0\\
28.7537431507572	0\\
28.7534260954146	0\\
28.7531090284983	0\\
28.7527919500074	0\\
28.7524748599412	0\\
28.7521577582987	0\\
28.7518406450792	0\\
28.7515235202818	0\\
28.7512063839056	0\\
28.7508892359498	0\\
28.7505720764136	0\\
28.7502549052961	0\\
28.7499377225964	0\\
28.7496205283138	0\\
28.7493033224473	0\\
28.7489861049961	0\\
28.7486688759594	0\\
28.7483516353364	0\\
28.7480343831261	0\\
28.7477171193277	0\\
28.7473998439405	0\\
28.7470825569635	0\\
28.7467652583959	0\\
28.7464479482368	0\\
28.7461306264855	0\\
28.745813293141	0\\
28.7454959482025	0\\
28.7451785916691	0\\
28.7448612235401	0\\
28.7445438438146	0\\
28.7442264524916	0\\
28.7439090495705	0\\
28.7435916350502	0\\
28.7432742089301	0\\
28.7429567712091	0\\
28.7426393218866	0\\
28.7423218609616	0\\
28.7420043884332	0\\
28.7416869043007	0\\
28.7413694085632	0\\
28.7410519012198	0\\
28.7407343822698	0\\
28.7404168517121	0\\
28.7400993095461	0\\
28.7397817557708	0\\
28.7394641903854	0\\
28.739146613389	0\\
28.7388290247808	0\\
28.73851142456	0\\
28.7381938127257	0\\
28.737876189277	0\\
28.7375585542132	0\\
28.7372409075332	0\\
28.7369232492364	0\\
28.7366055793219	0\\
28.7362878977887	0\\
28.7359702046361	0\\
28.7356524998632	0\\
28.7353347834691	0\\
28.735017055453	0\\
28.7346993158141	0\\
28.7343815645515	0\\
28.7340638016644	0\\
28.7337460271518	0\\
28.733428241013	0\\
28.7331104432471	0\\
28.7327926338532	0\\
28.7324748128305	0\\
28.7321569801782	0\\
28.7318391358954	0\\
28.7315212799812	0\\
28.7312034124348	0\\
28.7308855332554	0\\
28.730567642442	0\\
28.7302497399939	0\\
28.7299318259101	0\\
28.7296139001899	0\\
28.7292959628324	0\\
28.7289780138367	0\\
28.7286600532019	0\\
28.7283420809274	0\\
28.728024097012	0\\
28.7277061014551	0\\
28.7273880942558	0\\
28.7270700754132	0\\
28.7267520449264	0\\
28.7264340027947	0\\
28.7261159490171	0\\
28.7257978835928	0\\
28.725479806521	0\\
28.7251617178007	0\\
28.7248436174312	0\\
28.7245255054116	0\\
28.7242073817411	0\\
28.7238892464187	0\\
28.7235710994437	0\\
28.7232529408151	0\\
28.7229347705322	0\\
28.722616588594	0\\
28.7222983949998	0\\
28.7219801897486	0\\
28.7216619728396	0\\
28.721343744272	0\\
28.7210255040448	0\\
28.7207072521574	0\\
28.7203889886087	0\\
28.7200707133979	0\\
28.7197524265243	0\\
28.7194341279868	0\\
28.7191158177847	0\\
28.7187974959172	0\\
28.7184791623833	0\\
28.7181608171822	0\\
28.7178424603131	0\\
28.717524091775	0\\
28.7172057115672	0\\
28.7168873196888	0\\
28.7165689161389	0\\
28.7162505009167	0\\
28.7159320740213	0\\
28.7156136354519	0\\
28.7152951852076	0\\
28.7149767232875	0\\
28.7146582496908	0\\
28.7143397644167	0\\
28.7140212674642	0\\
28.7137027588326	0\\
28.7133842385209	0\\
28.7130657065284	0\\
28.7127471628541	0\\
28.7124286074972	0\\
28.7121100404568	0\\
28.7117914617321	0\\
28.7114728713223	0\\
28.7111542692264	0\\
28.7108356554436	0\\
28.7105170299731	0\\
28.710198392814	0\\
28.7098797439654	0\\
28.7095610834265	0\\
28.7092424111964	0\\
28.7089237272743	0\\
28.7086050316593	0\\
28.7082863243506	0\\
28.7079676053472	0\\
28.7076488746484	0\\
28.7073301322532	0\\
28.7070113781609	0\\
28.7066926123705	0\\
28.7063738348812	0\\
28.7060550456922	0\\
28.7057362448026	0\\
28.7054174322114	0\\
28.705098607918	0\\
28.7047797719213	0\\
28.7044609242206	0\\
28.704142064815	0\\
28.7038231937035	0\\
28.7035043108855	0\\
28.70318541636	0\\
28.7028665101261	0\\
28.702547592183	0\\
28.7022286625299	0\\
28.7019097211658	0\\
28.7015907680899	0\\
28.7012718033013	0\\
28.7009528267993	0\\
28.7006338385829	0\\
28.7003148386512	0\\
28.6999958270035	0\\
28.6996768036388	0\\
28.6993577685563	0\\
28.6990387217551	0\\
28.6987196632343	0\\
28.6984005929932	0\\
28.6980815110308	0\\
28.6977624173463	0\\
28.6974433119388	0\\
28.6971241948075	0\\
28.6968050659515	0\\
28.6964859253699	0\\
28.6961667730619	0\\
28.6958476090265	0\\
28.6955284332631	0\\
28.6952092457706	0\\
28.6948900465482	0\\
28.6945708355952	0\\
28.6942516129105	0\\
28.6939323784933	0\\
28.6936131323428	0\\
28.6932938744582	0\\
28.6929746048385	0\\
28.6926553234828	0\\
28.6923360303905	0\\
28.6920167255604	0\\
28.6916974089919	0\\
28.691378080684	0\\
28.6910587406358	0\\
28.6907393888466	0\\
28.6904200253155	0\\
28.6901006500415	0\\
28.6897812630238	0\\
28.6894618642616	0\\
28.689142453754	0\\
28.6888230315001	0\\
28.688503597499	0\\
28.68818415175	0\\
28.6878646942521	0\\
28.6875452250045	0\\
28.6872257440063	0\\
28.6869062512566	0\\
28.6865867467546	0\\
28.6862672304994	0\\
28.6859477024902	0\\
28.685628162726	0\\
28.6853086112061	0\\
28.6849890479295	0\\
28.6846694728954	0\\
28.684349886103	0\\
28.6840302875512	0\\
28.6837106772394	0\\
28.6833910551666	0\\
28.683071421332	0\\
28.6827517757347	0\\
28.6824321183738	0\\
28.6821124492485	0\\
28.6817927683578	0\\
28.681473075701	0\\
28.6811533712772	0\\
28.6808336550854	0\\
28.6805139271249	0\\
28.6801941873948	0\\
28.6798744358942	0\\
28.6795546726222	0\\
28.6792348975779	0\\
28.6789151107606	0\\
28.6785953121693	0\\
28.6782755018031	0\\
28.6779556796613	0\\
28.6776358457429	0\\
28.6773160000471	0\\
28.6769961425729	0\\
28.6766762733196	0\\
28.6763563922863	0\\
28.676036499472	0\\
28.675716594876	0\\
28.6753966784973	0\\
28.6750767503351	0\\
28.6747568103886	0\\
28.6744368586568	0\\
28.6741168951389	0\\
28.673796919834	0\\
28.6734769327412	0\\
28.6731569338598	0\\
28.6728369231887	0\\
28.6725169007272	0\\
28.6721968664744	0\\
28.6718768204293	0\\
28.6715567625912	0\\
28.6712366929592	0\\
28.6709166115324	0\\
28.6705965183099	0\\
28.6702764132909	0\\
28.6699562964744	0\\
28.6696361678597	0\\
28.6693160274458	0\\
28.6689958752319	0\\
28.6686757112171	0\\
28.6683555354006	0\\
28.6680353477814	0\\
28.6677151483588	0\\
28.6673949371317	0\\
28.6670747140994	0\\
28.6667544792611	0\\
28.6664342326157	0\\
28.6661139741625	0\\
28.6657937039005	0\\
28.665473421829	0\\
28.665153127947	0\\
28.6648328222536	0\\
28.6645125047481	0\\
28.6641921754295	0\\
28.6638718342969	0\\
28.6635514813495	0\\
28.6632311165864	0\\
28.6629107400068	0\\
28.6625903516097	0\\
28.6622699513943	0\\
28.6619495393598	0\\
28.6616291155051	0\\
28.6613086798296	0\\
28.6609882323323	0\\
28.6606677730123	0\\
28.6603473018687	0\\
28.6600268189008	0\\
28.6597063241075	0\\
28.6593858174881	0\\
28.6590652990417	0\\
28.6587447687673	0\\
28.6584242266642	0\\
28.6581036727315	0\\
28.6577831069682	0\\
28.6574625293735	0\\
28.6571419399466	0\\
28.6568213386865	0\\
28.6565007255924	0\\
28.6561801006634	0\\
28.6558594638986	0\\
28.6555388152972	0\\
28.6552181548583	0\\
28.654897482581	0\\
28.6545767984644	0\\
28.6542561025077	0\\
28.65393539471	0\\
28.6536146750705	0\\
28.6532939435881	0\\
28.6529732002621	0\\
28.6526524450917	0\\
28.6523316780758	0\\
28.6520108992137	0\\
28.6516901085044	0\\
28.6513693059472	0\\
28.651048491541	0\\
28.6507276652851	0\\
28.6504068271786	0\\
28.6500859772205	0\\
28.6497651154101	0\\
28.6494442417464	0\\
28.6491233562286	0\\
28.6488024588558	0\\
28.6484815496271	0\\
28.6481606285416	0\\
28.6478396955985	0\\
28.6475187507968	0\\
28.6471977941358	0\\
28.6468768256145	0\\
28.6465558452321	0\\
28.6462348529876	0\\
28.6459138488803	0\\
28.6455928329091	0\\
28.6452718050733	0\\
28.644950765372	0\\
28.6446297138043	0\\
28.6443086503693	0\\
28.6439875750661	0\\
28.6436664878939	0\\
28.6433453888518	0\\
28.6430242779388	0\\
28.6427031551543	0\\
28.6423820204971	0\\
28.6420608739665	0\\
28.6417397155617	0\\
28.6414185452816	0\\
28.6410973631255	0\\
28.6407761690924	0\\
28.6404549631815	0\\
28.640133745392	0\\
28.6398125157228	0\\
28.6394912741732	0\\
28.6391700207423	0\\
28.6388487554291	0\\
28.6385274782329	0\\
28.6382061891527	0\\
28.6378848881876	0\\
28.6375635753368	0\\
28.6372422505994	0\\
28.6369209139745	0\\
28.6365995654613	0\\
28.6362782050588	0\\
28.6359568327661	0\\
28.6356354485825	0\\
28.635314052507	0\\
28.6349926445387	0\\
28.6346712246767	0\\
28.6343497929203	0\\
28.6340283492684	0\\
28.6337068937202	0\\
28.6333854262749	0\\
28.6330639469315	0\\
28.6327424556892	0\\
28.6324209525471	0\\
28.6320994375043	0\\
28.6317779105599	0\\
28.6314563717131	0\\
28.6311348209629	0\\
28.6308132583085	0\\
28.630491683749	0\\
28.6301700972836	0\\
28.6298484989112	0\\
28.6295268886311	0\\
28.6292052664425	0\\
28.6288836323442	0\\
28.6285619863357	0\\
28.6282403284158	0\\
28.6279186585838	0\\
28.6275969768387	0\\
28.6272752831797	0\\
28.626953577606	0\\
28.6266318601165	0\\
28.6263101307105	0\\
28.625988389387	0\\
28.6256666361452	0\\
28.6253448709842	0\\
28.6250230939031	0\\
28.624701304901	0\\
28.624379503977	0\\
28.6240576911303	0\\
28.62373586636	0\\
28.6234140296651	0\\
28.6230921810449	0\\
28.6227703204984	0\\
28.6224484480247	0\\
28.6221265636229	0\\
28.6218046672922	0\\
28.6214827590317	0\\
28.6211608388405	0\\
28.6208389067177	0\\
28.6205169626624	0\\
28.6201950066738	0\\
28.619873038751	0\\
28.619551058893	0\\
28.619229067099	0\\
28.6189070633681	0\\
28.6185850476994	0\\
28.6182630200921	0\\
28.6179409805452	0\\
28.6176189290578	0\\
28.6172968656292	0\\
28.6169747902583	0\\
28.6166527029444	0\\
28.6163306036864	0\\
28.6160084924836	0\\
28.6156863693351	0\\
28.6153642342399	0\\
28.6150420871972	0\\
28.6147199282061	0\\
28.6143977572656	0\\
28.6140755743751	0\\
28.6137533795334	0\\
28.6134311727398	0\\
28.6131089539933	0\\
28.6127867232931	0\\
28.6124644806383	0\\
28.612142226028	0\\
28.6118199594613	0\\
28.6114976809373	0\\
28.6111753904552	0\\
28.610853088014	0\\
28.6105307736128	0\\
28.6102084472508	0\\
28.6098861089272	0\\
28.6095637586409	0\\
28.6092413963911	0\\
28.608919022177	0\\
28.6085966359975	0\\
28.608274237852	0\\
28.6079518277393	0\\
28.6076294056588	0\\
28.6073069716094	0\\
28.6069845255903	0\\
28.6066620676006	0\\
28.6063395976395	0\\
28.6060171157059	0\\
28.6056946217991	0\\
28.6053721159182	0\\
28.6050495980622	0\\
28.6047270682302	0\\
28.6044045264214	0\\
28.604081972635	0\\
28.6037594068699	0\\
28.6034368291253	0\\
28.6031142394004	0\\
28.6027916376942	0\\
28.6024690240058	0\\
28.6021463983344	0\\
28.601823760679	0\\
28.6015011110388	0\\
28.6011784494129	0\\
28.6008557758003	0\\
28.6005330902003	0\\
28.6002103926118	0\\
28.5998876830341	0\\
28.5995649614662	0\\
28.5992422279073	0\\
28.5989194823564	0\\
28.5985967248126	0\\
28.5982739552751	0\\
28.597951173743	0\\
28.5976283802153	0\\
28.5973055746912	0\\
28.5969827571698	0\\
28.5966599276503	0\\
28.5963370861316	0\\
28.5960142326129	0\\
28.5956913670934	0\\
28.5953684895721	0\\
28.5950456000482	0\\
28.5947226985207	0\\
28.5943997849887	0\\
28.5940768594514	0\\
28.5937539219079	0\\
28.5934309723573	0\\
28.5931080107986	0\\
28.5927850372311	0\\
28.5924620516537	0\\
28.5921390540657	0\\
28.591816044466	0\\
28.5914930228539	0\\
28.5911699892284	0\\
28.5908469435887	0\\
28.5905238859337	0\\
28.5902008162628	0\\
28.5898777345748	0\\
28.5895546408691	0\\
28.5892315351446	0\\
28.5889084174004	0\\
28.5885852876358	0\\
28.5882621458497	0\\
28.5879389920413	0\\
28.5876158262097	0\\
28.5872926483539	0\\
28.5869694584732	0\\
28.5866462565666	0\\
28.5863230426332	0\\
28.5859998166722	0\\
28.5856765786825	0\\
28.5853533286634	0\\
28.5850300666139	0\\
28.5847067925331	0\\
28.5843835064202	0\\
28.5840602082742	0\\
28.5837368980943	0\\
28.5834135758795	0\\
28.5830902416289	0\\
28.5827668953418	0\\
28.5824435370171	0\\
28.5821201666539	0\\
28.5817967842515	0\\
28.5814733898088	0\\
28.581149983325	0\\
28.5808265647991	0\\
28.5805031342304	0\\
28.5801796916178	0\\
28.5798562369605	0\\
28.5795327702577	0\\
28.5792092915083	0\\
28.5788858007115	0\\
28.5785622978664	0\\
28.5782387829721	0\\
28.5779152560278	0\\
28.5775917170324	0\\
28.5772681659851	0\\
28.5769446028851	0\\
28.5766210277314	0\\
28.5762974405231	0\\
28.5759738412593	0\\
28.5756502299391	0\\
28.5753266065616	0\\
28.575002971126	0\\
28.5746793236313	0\\
28.5743556640766	0\\
28.5740319924611	0\\
28.5737083087838	0\\
28.5733846130438	0\\
28.5730609052402	0\\
28.5727371853722	0\\
28.5724134534388	0\\
28.5720897094391	0\\
28.5717659533723	0\\
28.5714421852374	0\\
28.5711184050335	0\\
28.5707946127598	0\\
28.5704708084153	0\\
28.5701469919991	0\\
28.5698231635104	0\\
28.5694993229482	0\\
28.5691754703116	0\\
28.5688516055998	0\\
28.5685277288118	0\\
28.5682038399467	0\\
28.5678799390036	0\\
28.5675560259817	0\\
28.5672321008801	0\\
28.5669081636977	0\\
28.5665842144338	0\\
28.5662602530874	0\\
28.5659362796576	0\\
28.5656122941435	0\\
28.5652882965443	0\\
28.564964286859	0\\
28.5646402650867	0\\
28.5643162312265	0\\
28.5639921852775	0\\
28.5636681272389	0\\
28.5633440571096	0\\
28.5630199748889	0\\
28.5626958805758	0\\
28.5623717741693	0\\
28.5620476556687	0\\
28.561723525073	0\\
28.5613993823813	0\\
28.5610752275926	0\\
28.5607510607062	0\\
28.560426881721	0\\
28.5601026906362	0\\
28.5597784874509	0\\
28.5594542721642	0\\
28.5591300447751	0\\
28.5588058052828	0\\
28.5584815536864	0\\
28.558157289985	0\\
28.5578330141776	0\\
28.5575087262633	0\\
28.5571844262413	0\\
28.5568601141107	0\\
28.5565357898705	0\\
28.5562114535198	0\\
28.5558871050578	0\\
28.5555627444835	0\\
28.555238371796	0\\
28.5549139869944	0\\
28.5545895900779	0\\
28.5542651810454	0\\
28.5539407598962	0\\
28.5536163266292	0\\
28.5532918812437	0\\
28.5529674237387	0\\
28.5526429541132	0\\
28.5523184723664	0\\
28.5519939784974	0\\
28.5516694725052	0\\
28.551344954389	0\\
28.5510204241479	0\\
28.5506958817809	0\\
28.5503713272871	0\\
28.5500467606657	0\\
28.5497221819157	0\\
28.5493975910362	0\\
28.5490729880264	0\\
28.5487483728853	0\\
28.5484237456119	0\\
28.5480991062055	0\\
28.547774454665	0\\
28.5474497909896	0\\
28.5471251151784	0\\
28.5468004272305	0\\
28.5464757271449	0\\
28.5461510149208	0\\
28.5458262905573	0\\
28.5455015540533	0\\
28.5451768054081	0\\
28.5448520446207	0\\
28.5445272716902	0\\
28.5442024866158	0\\
28.5438776893964	0\\
28.5435528800312	0\\
28.5432280585193	0\\
28.5429032248598	0\\
28.5425783790517	0\\
28.5422535210942	0\\
28.5419286509863	0\\
28.5416037687272	0\\
28.5412788743159	0\\
28.5409539677515	0\\
28.5406290490331	0\\
28.5403041181598	0\\
28.5399791751307	0\\
28.5396542199449	0\\
28.5393292526015	0\\
28.5390042730995	0\\
28.5386792814381	0\\
28.5383542776163	0\\
28.5380292616333	0\\
28.537704233488	0\\
28.5373791931797	0\\
28.5370541407074	0\\
28.5367290760702	0\\
28.5364039992671	0\\
28.5360789102973	0\\
28.5357538091599	0\\
28.5354286958539	0\\
28.5351035703785	0\\
28.5347784327327	0\\
28.5344532829156	0\\
28.5341281209263	0\\
28.5338029467639	0\\
28.5334777604275	0\\
28.5331525619161	0\\
28.5328273512289	0\\
28.5325021283649	0\\
28.5321768933233	0\\
28.5318516461031	0\\
28.5315263867034	0\\
28.5312011151233	0\\
28.5308758313619	0\\
28.5305505354183	0\\
28.5302252272915	0\\
28.5298999069807	0\\
28.5295745744849	0\\
28.5292492298032	0\\
28.5289238729347	0\\
28.5285985038786	0\\
28.5282731226338	0\\
28.5279477291994	0\\
28.5276223235747	0\\
28.5272969057585	0\\
28.5269714757502	0\\
28.5266460335486	0\\
28.5263205791529	0\\
28.5259951125622	0\\
28.5256696337756	0\\
28.5253441427921	0\\
28.5250186396109	0\\
28.524693124231	0\\
28.5243675966516	0\\
28.5240420568716	0\\
28.5237165048902	0\\
28.5233909407065	0\\
28.5230653643196	0\\
28.5227397757285	0\\
28.5224141749323	0\\
28.5220885619302	0\\
28.5217629367211	0\\
28.5214372993042	0\\
28.5211116496786	0\\
28.5207859878433	0\\
28.5204603137975	0\\
28.5201346275402	0\\
28.5198089290705	0\\
28.5194832183875	0\\
28.5191574954903	0\\
28.5188317603779	0\\
28.5185060130495	0\\
28.5181802535041	0\\
28.5178544817408	0\\
28.5175286977586	0\\
28.5172029015568	0\\
28.5168770931343	0\\
28.5165512724903	0\\
28.5162254396238	0\\
28.5158995945339	0\\
28.5155737372197	0\\
28.5152478676802	0\\
28.5149219859146	0\\
28.514596091922	0\\
28.5142701857013	0\\
28.5139442672518	0\\
28.5136183365725	0\\
28.5132923936624	0\\
28.5129664385207	0\\
28.5126404711464	0\\
28.5123144915386	0\\
28.5119884996964	0\\
28.5116624956189	0\\
28.5113364793052	0\\
28.5110104507543	0\\
28.5106844099653	0\\
28.5103583569373	0\\
28.5100322916694	0\\
28.5097062141606	0\\
28.5093801244101	0\\
28.509054022417	0\\
28.5087279081802	0\\
28.5084017816989	0\\
28.5080756429722	0\\
28.5077494919992	0\\
28.5074233287788	0\\
28.5070971533103	0\\
28.5067709655927	0\\
28.506444765625	0\\
28.5061185534064	0\\
28.5057923289359	0\\
28.5054660922126	0\\
28.5051398432356	0\\
28.504813582004	0\\
28.5044873085168	0\\
28.5041610227731	0\\
28.5038347247721	0\\
28.5035084145127	0\\
28.5031820919941	0\\
28.5028557572153	0\\
28.5025294101755	0\\
28.5022030508736	0\\
28.5018766793089	0\\
28.5015502954802	0\\
28.5012238993868	0\\
28.5008974910278	0\\
28.5005710704021	0\\
28.5002446375089	0\\
28.4999181923472	0\\
28.4995917349162	0\\
28.4992652652148	0\\
28.4989387832423	0\\
28.4986122889976	0\\
28.4982857824798	0\\
28.497959263688	0\\
28.4976327326213	0\\
28.4973061892788	0\\
28.4969796336596	0\\
28.4966530657626	0\\
28.4963264855871	0\\
28.495999893132	0\\
28.4956732883964	0\\
28.4953466713795	0\\
28.4950200420803	0\\
28.4946934004979	0\\
28.4943667466313	0\\
28.4940400804797	0\\
28.4937134020421	0\\
28.4933867113175	0\\
28.4930600083051	0\\
28.4927332930039	0\\
28.4924065654131	0\\
28.4920798255316	0\\
28.4917530733586	0\\
28.4914263088931	0\\
28.4910995321343	0\\
28.4907727430811	0\\
28.4904459417327	0\\
28.4901191280881	0\\
28.4897923021464	0\\
28.4894654639067	0\\
28.489138613368	0\\
28.4888117505295	0\\
28.4884848753902	0\\
28.4881579879492	0\\
28.4878310882056	0\\
28.4875041761583	0\\
28.4871772518066	0\\
28.4868503151494	0\\
28.4865233661859	0\\
28.4861964049151	0\\
28.4858694313361	0\\
28.485542445448	0\\
28.4852154472498	0\\
28.4848884367407	0\\
28.4845614139196	0\\
28.4842343787857	0\\
28.483907331338	0\\
28.4835802715756	0\\
28.4832531994977	0\\
28.4829261151031	0\\
28.4825990183911	0\\
28.4822719093607	0\\
28.4819447880109	0\\
28.481617654341	0\\
28.4812905083498	0\\
28.4809633500365	0\\
28.4806361794001	0\\
28.4803089964398	0\\
28.4799818011546	0\\
28.4796545935435	0\\
28.4793273736057	0\\
28.4790001413402	0\\
28.4786728967461	0\\
28.4783456398225	0\\
28.4780183705683	0\\
28.4776910889828	0\\
28.4773637950649	0\\
28.4770364888138	0\\
28.4767091702285	0\\
28.4763818393081	0\\
28.4760544960516	0\\
28.4757271404581	0\\
28.4753997725268	0\\
28.4750723922566	0\\
28.4747449996466	0\\
28.4744175946959	0\\
28.4740901774036	0\\
28.4737627477687	0\\
28.4734353057904	0\\
28.4731078514676	0\\
28.4727803847994	0\\
28.472452905785	0\\
28.4721254144234	0\\
28.4717979107136	0\\
28.4714703946547	0\\
28.4711428662458	0\\
28.470815325486	0\\
28.4704877723743	0\\
28.4701602069099	0\\
28.4698326290916	0\\
28.4695050389187	0\\
28.4691774363903	0\\
28.4688498215052	0\\
28.4685221942627	0\\
28.4681945546619	0\\
28.4678669027017	0\\
28.4675392383812	0\\
28.4672115616995	0\\
28.4668838726557	0\\
28.4665561712489	0\\
28.4662284574781	0\\
28.4659007313423	0\\
28.4655729928407	0\\
28.4652452419723	0\\
28.4649174787362	0\\
28.4645897031314	0\\
28.464261915157	0\\
28.4639341148121	0\\
28.4636063020957	0\\
28.463278477007	0\\
28.4629506395449	0\\
28.4626227897085	0\\
28.462294927497	0\\
28.4619670529094	0\\
28.4616391659446	0\\
28.4613112666019	0\\
28.4609833548802	0\\
28.4606554307787	0\\
28.4603274942964	0\\
28.4599995454323	0\\
28.4596715841856	0\\
28.4593436105553	0\\
28.4590156245404	0\\
28.45868762614	0\\
28.4583596153532	0\\
28.4580315921791	0\\
28.4577035566167	0\\
28.4573755086651	0\\
28.4570474483233	0\\
28.4567193755905	0\\
28.4563912904656	0\\
28.4560631929477	0\\
28.4557350830359	0\\
28.4554069607293	0\\
28.455078826027	0\\
28.4547506789279	0\\
28.4544225194312	0\\
28.4540943475358	0\\
28.453766163241	0\\
28.4534379665457	0\\
28.4531097574491	0\\
28.45278153595	0\\
28.4524533020478	0\\
28.4521250557413	0\\
28.4517967970296	0\\
28.4514685259119	0\\
28.4511402423872	0\\
28.4508119464545	0\\
28.4504836381129	0\\
28.4501553173615	0\\
28.4498269841993	0\\
28.4494986386254	0\\
28.4491702806389	0\\
28.4488419102387	0\\
28.4485135274241	0\\
28.448185132194	0\\
28.4478567245474	0\\
28.4475283044836	0\\
28.4471998720014	0\\
28.4468714271001	0\\
28.4465429697785	0\\
28.4462145000359	0\\
28.4458860178712	0\\
28.4455575232836	0\\
28.4452290162721	0\\
28.4449004968356	0\\
28.4445719649734	0\\
28.4442434206845	0\\
28.4439148639678	0\\
28.4435862948226	0\\
28.4432577132478	0\\
28.4429291192425	0\\
28.4426005128058	0\\
28.4422718939366	0\\
28.4419432626342	0\\
28.4416146188975	0\\
28.4412859627256	0\\
28.4409572941176	0\\
28.4406286130724	0\\
28.4402999195893	0\\
28.4399712136672	0\\
28.4396424953051	0\\
28.4393137645022	0\\
28.4389850212576	0\\
28.4386562655702	0\\
28.4383274974391	0\\
28.4379987168634	0\\
28.4376699238421	0\\
28.4373411183744	0\\
28.4370123004592	0\\
28.4366834700956	0\\
28.4363546272827	0\\
28.4360257720195	0\\
28.435696904305	0\\
28.4353680241385	0\\
28.4350391315188	0\\
28.4347102264451	0\\
28.4343813089164	0\\
28.4340523789318	0\\
28.4337234364903	0\\
28.433394481591	0\\
28.4330655142329	0\\
28.4327365344151	0\\
28.4324075421367	0\\
28.4320785373966	0\\
28.4317495201941	0\\
28.431420490528	0\\
28.4310914483975	0\\
28.4307623938017	0\\
28.4304333267395	0\\
28.4301042472101	0\\
28.4297751552125	0\\
28.4294460507457	0\\
28.4291169338089	0\\
28.428787804401	0\\
28.4284586625211	0\\
28.4281295081683	0\\
28.4278003413416	0\\
28.4274711620401	0\\
28.4271419702629	0\\
28.4268127660089	0\\
28.4264835492772	0\\
28.426154320067	0\\
28.4258250783772	0\\
28.4254958242069	0\\
28.4251665575552	0\\
28.4248372784211	0\\
28.4245079868037	0\\
28.4241786827019	0\\
28.423849366115	0\\
28.4235200370419	0\\
28.4231906954816	0\\
28.4228613414333	0\\
28.422531974896	0\\
28.4222025958687	0\\
28.4218732043505	0\\
28.4215438003404	0\\
28.4212143838376	0\\
28.420884954841	0\\
28.4205555133497	0\\
28.4202260593627	0\\
28.4198965928792	0\\
28.4195671138981	0\\
28.4192376224185	0\\
28.4189081184394	0\\
28.41857860196	0\\
28.4182490729793	0\\
28.4179195314962	0\\
28.41758997751	0\\
28.4172604110195	0\\
28.4169308320239	0\\
28.4166012405223	0\\
28.4162716365136	0\\
28.4159420199969	0\\
28.4156123909713	0\\
28.4152827494358	0\\
28.4149530953895	0\\
28.4146234288314	0\\
28.4142937497606	0\\
28.4139640581761	0\\
28.413634354077	0\\
28.4133046374623	0\\
28.4129749083311	0\\
28.4126451666824	0\\
28.4123154125153	0\\
28.4119856458288	0\\
28.411655866622	0\\
28.4113260748939	0\\
28.4109962706435	0\\
28.41066645387	0\\
28.4103366245723	0\\
28.4100067827496	0\\
28.4096769284008	0\\
28.4093470615251	0\\
28.4090171821214	0\\
28.4086872901888	0\\
28.4083573857263	0\\
28.4080274687331	0\\
28.4076975392082	0\\
28.4073675971505	0\\
28.4070376425592	0\\
28.4067076754332	0\\
28.4063776957718	0\\
28.4060477035738	0\\
28.4057176988384	0\\
28.4053876815645	0\\
28.4050576517513	0\\
28.4047276093978	0\\
28.404397554503	0\\
28.4040674870659	0\\
28.4037374070857	0\\
28.4034073145614	0\\
28.403077209492	0\\
28.4027470918765	0\\
28.4024169617141	0\\
28.4020868190037	0\\
28.4017566637444	0\\
28.4014264959352	0\\
28.4010963155753	0\\
28.4007661226636	0\\
28.4004359171991	0\\
28.400105699181	0\\
28.3997754686083	0\\
28.39944522548	0\\
28.3991149697952	0\\
28.3987847015528	0\\
28.3984544207521	0\\
28.3981241273919	0\\
28.3977938214714	0\\
28.3974635029895	0\\
28.3971331719454	0\\
28.3968028283381	0\\
28.3964724721666	0\\
28.39614210343	0\\
28.3958117221273	0\\
28.3954813282575	0\\
28.3951509218197	0\\
28.394820502813	0\\
28.3944900712364	0\\
28.3941596270889	0\\
28.3938291703696	0\\
28.3934987010774	0\\
28.3931682192116	0\\
28.3928377247711	0\\
28.3925072177549	0\\
28.3921766981621	0\\
28.3918461659917	0\\
28.3915156212428	0\\
28.3911850639144	0\\
28.3908544940056	0\\
28.3905239115154	0\\
28.3901933164428	0\\
28.3898627087869	0\\
28.3895320885468	0\\
28.3892014557214	0\\
28.3888708103098	0\\
28.3885401523111	0\\
28.3882094817243	0\\
28.3878787985484	0\\
28.3875481027825	0\\
28.3872173944257	0\\
28.3868866734769	0\\
28.3865559399352	0\\
28.3862251937996	0\\
28.3858944350692	0\\
28.3855636637431	0\\
28.3852328798202	0\\
28.3849020832996	0\\
28.3845712741804	0\\
28.3842404524615	0\\
28.3839096181421	0\\
28.3835787712212	0\\
28.3832479116977	0\\
28.3829170395708	0\\
28.3825861548395	0\\
28.3822552575028	0\\
28.3819243475598	0\\
28.3815934250095	0\\
28.381262489851	0\\
28.3809315420832	0\\
28.3806005817052	0\\
28.3802696087161	0\\
28.379938623115	0\\
28.3796076249007	0\\
28.3792766140725	0\\
28.3789455906292	0\\
28.37861455457	0\\
28.3782835058939	0\\
28.3779524446	0\\
28.3776213706872	0\\
28.3772902841546	0\\
28.3769591850012	0\\
28.3766280732262	0\\
28.3762969488284	0\\
28.3759658118071	0\\
28.3756346621611	0\\
28.3753034998896	0\\
28.3749723249915	0\\
28.3746411374659	0\\
28.3743099373119	0\\
28.3739787245285	0\\
28.3736474991147	0\\
28.3733162610695	0\\
28.372985010392	0\\
28.3726537470813	0\\
28.3723224711363	0\\
28.3719911825562	0\\
28.3716598813398	0\\
28.3713285674864	0\\
28.3709972409948	0\\
28.3706659018642	0\\
28.3703345500936	0\\
28.370003185682	0\\
28.3696718086285	0\\
28.369340418932	0\\
28.3690090165917	0\\
28.3686776016065	0\\
28.3683461739755	0\\
28.3680147336977	0\\
28.3676832807722	0\\
28.367351815198	0\\
28.3670203369742	0\\
28.3666888460997	0\\
28.3663573425736	0\\
28.3660258263949	0\\
28.3656942975627	0\\
28.365362756076	0\\
28.3650312019339	0\\
28.3646996351353	0\\
28.3643680556793	0\\
28.364036463565	0\\
28.3637048587913	0\\
28.3633732413573	0\\
28.3630416112621	0\\
28.3627099685046	0\\
28.362378313084	0\\
28.3620466449992	0\\
28.3617149642492	0\\
28.3613832708332	0\\
28.3610515647501	0\\
28.360719845999	0\\
28.3603881145788	0\\
28.3600563704887	0\\
28.3597246137277	0\\
28.3593928442947	0\\
28.3590610621889	0\\
28.3587292674093	0\\
28.3583974599548	0\\
28.3580656398246	0\\
28.3577338070176	0\\
28.3574019615329	0\\
28.3570701033695	0\\
28.3567382325265	0\\
28.3564063490029	0\\
28.3560744527976	0\\
28.3557425439098	0\\
28.3554106223385	0\\
28.3550786880827	0\\
28.3547467411414	0\\
28.3544147815137	0\\
28.3540828091986	0\\
28.3537508241951	0\\
28.3534188265022	0\\
28.353086816119	0\\
28.3527547930446	0\\
28.3524227572779	0\\
28.352090708818	0\\
28.3517586476638	0\\
28.3514265738145	0\\
28.3510944872691	0\\
28.3507623880266	0\\
28.3504302760859	0\\
28.3500981514463	0\\
28.3497660141066	0\\
28.3494338640659	0\\
28.3491017013232	0\\
28.3487695258776	0\\
28.3484373377281	0\\
28.3481051368737	0\\
28.3477729233135	0\\
28.3474406970464	0\\
28.3471084580716	0\\
28.346776206388	0\\
28.3464439419946	0\\
28.3461116648905	0\\
28.3457793750747	0\\
28.3454470725463	0\\
28.3451147573043	0\\
28.3447824293476	0\\
28.3444500886754	0\\
28.3441177352866	0\\
28.3437853691803	0\\
28.3434529903555	0\\
28.3431205988112	0\\
28.3427881945465	0\\
28.3424557775604	0\\
28.3421233478519	0\\
28.34179090542	0\\
28.3414584502637	0\\
28.3411259823822	0\\
28.3407935017744	0\\
28.3404610084393	0\\
28.340128502376	0\\
28.3397959835834	0\\
28.3394634520607	0\\
28.3391309078069	0\\
28.3387983508208	0\\
28.3384657811017	0\\
28.3381331986485	0\\
28.3378006034603	0\\
28.3374679955359	0\\
28.3371353748746	0\\
28.3368027414753	0\\
28.3364700953371	0\\
28.3361374364589	0\\
28.3358047648398	0\\
28.3354720804788	0\\
28.3351393833749	0\\
28.3348066735272	0\\
28.3344739509347	0\\
28.3341412155964	0\\
28.3338084675113	0\\
28.3334757066784	0\\
28.3331429330969	0\\
28.3328101467656	0\\
28.3324773476837	0\\
28.3321445358501	0\\
28.3318117112639	0\\
28.331478873924	0\\
28.3311460238296	0\\
28.3308131609796	0\\
28.3304802853731	0\\
28.330147397009	0\\
28.3298144958864	0\\
28.3294815820044	0\\
28.3291486553619	0\\
28.328815715958	0\\
28.3284827637917	0\\
28.328149798862	0\\
28.3278168211679	0\\
28.3274838307084	0\\
28.3271508274827	0\\
28.3268178114896	0\\
28.3264847827283	0\\
28.3261517411977	0\\
28.3258186868968	0\\
28.3254856198247	0\\
28.3251525399804	0\\
28.324819447363	0\\
28.3244863419714	0\\
28.3241532238046	0\\
28.3238200928617	0\\
28.3234869491418	0\\
28.3231537926437	0\\
28.3228206233666	0\\
28.3224874413094	0\\
28.3221542464712	0\\
28.3218210388511	0\\
28.3214878184479	0\\
28.3211545852608	0\\
28.3208213392887	0\\
28.3204880805307	0\\
28.3201548089858	0\\
28.319821524653	0\\
28.3194882275313	0\\
28.3191549176198	0\\
28.3188215949175	0\\
28.3184882594233	0\\
28.3181549111363	0\\
28.3178215500556	0\\
28.3174881761801	0\\
28.3171547895089	0\\
28.3168213900409	0\\
28.3164879777752	0\\
28.3161545527109	0\\
28.3158211148468	0\\
28.3154876641821	0\\
28.3151542007158	0\\
28.3148207244469	0\\
28.3144872353743	0\\
28.3141537334972	0\\
28.3138202188145	0\\
28.3134866913252	0\\
28.3131531510284	0\\
28.3128195979231	0\\
28.3124860320083	0\\
28.3121524532829	0\\
28.3118188617462	0\\
28.3114852573969	0\\
28.3111516402342	0\\
28.3108180102571	0\\
28.3104843674646	0\\
28.3101507118557	0\\
28.3098170434294	0\\
28.3094833621847	0\\
28.3091496681207	0\\
28.3088159612364	0\\
28.3084822415307	0\\
28.3081485090027	0\\
28.3078147636515	0\\
28.3074810054759	0\\
28.3071472344752	0\\
28.3068134506481	0\\
28.3064796539938	0\\
28.3061458445113	0\\
28.3058120221997	0\\
28.3054781870578	0\\
28.3051443390847	0\\
28.3048104782795	0\\
28.3044766046411	0\\
28.3041427181686	0\\
28.303808818861	0\\
28.3034749067172	0\\
28.3031409817364	0\\
28.3028070439175	0\\
28.3024730932595	0\\
28.3021391297614	0\\
28.3018051534223	0\\
28.3014711642412	0\\
28.301137162217	0\\
28.3008031473489	0\\
28.3004691196357	0\\
28.3001350790766	0\\
28.2998010256704	0\\
28.2994669594164	0\\
};
\addplot [color=mycolor1, forget plot]
  table[row sep=crcr]{%
28.2994669594164	0\\
28.2991328803133	0\\
28.2987987883604	0\\
28.2984646835565	0\\
28.2981305659007	0\\
28.297796435392	0\\
28.2974622920294	0\\
28.2971281358119	0\\
28.2967939667385	0\\
28.2964597848083	0\\
28.2961255900203	0\\
28.2957913823734	0\\
28.2954571618666	0\\
28.2951229284991	0\\
28.2947886822697	0\\
28.2944544231776	0\\
28.2941201512216	0\\
28.2937858664009	0\\
28.2934515687144	0\\
28.2931172581612	0\\
28.2927829347402	0\\
28.2924485984504	0\\
28.292114249291	0\\
28.2917798872608	0\\
28.2914455123589	0\\
28.2911111245843	0\\
28.290776723936	0\\
28.290442310413	0\\
28.2901078840144	0\\
28.289773444739	0\\
28.2894389925861	0\\
28.2891045275544	0\\
28.2887700496431	0\\
28.2884355588512	0\\
28.2881010551777	0\\
28.2877665386215	0\\
28.2874320091817	0\\
28.2870974668573	0\\
28.2867629116474	0\\
28.2864283435508	0\\
28.2860937625666	0\\
28.2857591686939	0\\
28.2854245619316	0\\
28.2850899422787	0\\
28.2847553097343	0\\
28.2844206642974	0\\
28.2840860059669	0\\
28.2837513347418	0\\
28.2834166506212	0\\
28.2830819536041	0\\
28.2827472436895	0\\
28.2824125208764	0\\
28.2820777851638	0\\
28.2817430365507	0\\
28.281408275036	0\\
28.2810735006189	0\\
28.2807387132983	0\\
28.2804039130733	0\\
28.2800690999427	0\\
28.2797342739057	0\\
28.2793994349612	0\\
28.2790645831083	0\\
28.2787297183459	0\\
28.2783948406731	0\\
28.2780599500888	0\\
28.2777250465921	0\\
28.2773901301819	0\\
28.2770552008573	0\\
28.2767202586173	0\\
28.2763853034609	0\\
28.276050335387	0\\
28.2757153543948	0\\
28.2753803604831	0\\
28.275045353651	0\\
28.2747103338975	0\\
28.2743753012216	0\\
28.2740402556223	0\\
28.2737051970986	0\\
28.2733701256495	0\\
28.273035041274	0\\
28.2726999439711	0\\
28.2723648337399	0\\
28.2720297105792	0\\
28.2716945744882	0\\
28.2713594254658	0\\
28.271024263511	0\\
28.2706890886228	0\\
28.2703539008003	0\\
28.2700187000424	0\\
28.2696834863481	0\\
28.2693482597165	0\\
28.2690130201465	0\\
28.2686777676371	0\\
28.2683425021874	0\\
28.2680072237963	0\\
28.2676719324628	0\\
28.267336628186	0\\
28.2670013109648	0\\
28.2666659807982	0\\
28.2663306376853	0\\
28.265995281625	0\\
28.2656599126164	0\\
28.2653245306584	0\\
28.26498913575	0\\
28.2646537278903	0\\
28.2643183070783	0\\
28.2639828733128	0\\
28.263647426593	0\\
28.2633119669179	0\\
28.2629764942863	0\\
28.2626410086974	0\\
28.2623055101502	0\\
28.2619699986436	0\\
28.2616344741766	0\\
28.2612989367482	0\\
28.2609633863575	0\\
28.2606278230034	0\\
28.2602922466849	0\\
28.2599566574011	0\\
28.2596210551509	0\\
28.2592854399333	0\\
28.2589498117473	0\\
28.2586141705919	0\\
28.2582785164662	0\\
28.257942849369	0\\
28.2576071692995	0\\
28.2572714762566	0\\
28.2569357702392	0\\
28.2566000512465	0\\
28.2562643192774	0\\
28.2559285743309	0\\
28.2555928164059	0\\
28.2552570455016	0\\
28.2549212616168	0\\
28.2545854647506	0\\
28.254249654902	0\\
28.25391383207	0\\
28.2535779962535	0\\
28.2532421474516	0\\
28.2529062856633	0\\
28.2525704108875	0\\
28.2522345231233	0\\
28.2518986223696	0\\
28.2515627086254	0\\
28.2512267818898	0\\
28.2508908421618	0\\
28.2505548894402	0\\
28.2502189237242	0\\
28.2498829450127	0\\
28.2495469533047	0\\
28.2492109485992	0\\
28.2488749308952	0\\
28.2485389001918	0\\
28.2482028564878	0\\
28.2478667997822	0\\
28.2475307300742	0\\
28.2471946473626	0\\
28.2468585516465	0\\
28.2465224429249	0\\
28.2461863211967	0\\
28.2458501864609	0\\
28.2455140387166	0\\
28.2451778779627	0\\
28.2448417041983	0\\
28.2445055174222	0\\
28.2441693176336	0\\
28.2438331048314	0\\
28.2434968790146	0\\
28.2431606401821	0\\
28.242824388333	0\\
28.2424881234663	0\\
28.242151845581	0\\
28.241815554676	0\\
28.2414792507504	0\\
28.2411429338031	0\\
28.2408066038332	0\\
28.2404702608395	0\\
28.2401339048212	0\\
28.2397975357772	0\\
28.2394611537065	0\\
28.239124758608	0\\
28.2387883504809	0\\
28.238451929324	0\\
28.2381154951364	0\\
28.237779047917	0\\
28.2374425876648	0\\
28.2371061143789	0\\
28.2367696280582	0\\
28.2364331287017	0\\
28.2360966163085	0\\
28.2357600908774	0\\
28.2354235524075	0\\
28.2350870008977	0\\
28.2347504363471	0\\
28.2344138587547	0\\
28.2340772681194	0\\
28.2337406644402	0\\
28.2334040477162	0\\
28.2330674179462	0\\
28.2327307751294	0\\
28.2323941192646	0\\
28.2320574503509	0\\
28.2317207683873	0\\
28.2313840733726	0\\
28.2310473653061	0\\
28.2307106441865	0\\
28.230373910013	0\\
28.2300371627845	0\\
28.2297004024999	0\\
28.2293636291584	0\\
28.2290268427587	0\\
28.2286900433001	0\\
28.2283532307814	0\\
28.2280164052016	0\\
28.2276795665597	0\\
28.2273427148547	0\\
28.2270058500856	0\\
28.2266689722513	0\\
28.2263320813509	0\\
28.2259951773834	0\\
28.2256582603477	0\\
28.2253213302428	0\\
28.2249843870677	0\\
28.2246474308214	0\\
28.2243104615029	0\\
28.2239734791111	0\\
28.2236364836451	0\\
28.2232994751038	0\\
28.2229624534862	0\\
28.2226254187913	0\\
28.2222883710181	0\\
28.2219513101656	0\\
28.2216142362327	0\\
28.2212771492185	0\\
28.2209400491219	0\\
28.2206029359419	0\\
28.2202658096775	0\\
28.2199286703276	0\\
28.2195915178914	0\\
28.2192543523677	0\\
28.2189171737555	0\\
28.2185799820538	0\\
28.2182427772617	0\\
28.217905559378	0\\
28.2175683284017	0\\
28.217231084332	0\\
28.2168938271676	0\\
28.2165565569077	0\\
28.2162192735511	0\\
28.215881977097	0\\
28.2155446675442	0\\
28.2152073448917	0\\
28.2148700091386	0\\
28.2145326602838	0\\
28.2141952983263	0\\
28.2138579232651	0\\
28.2135205350991	0\\
28.2131831338273	0\\
28.2128457194488	0\\
28.2125082919625	0\\
28.2121708513674	0\\
28.2118333976624	0\\
28.2114959308466	0\\
28.2111584509189	0\\
28.2108209578783	0\\
28.2104834517238	0\\
28.2101459324543	0\\
28.209808400069	0\\
28.2094708545666	0\\
28.2091332959463	0\\
28.2087957242069	0\\
28.2084581393476	0\\
28.2081205413671	0\\
28.2077829302647	0\\
28.2074453060391	0\\
28.2071076686894	0\\
28.2067700182146	0\\
28.2064323546136	0\\
28.2060946778855	0\\
28.2057569880292	0\\
28.2054192850437	0\\
28.2050815689279	0\\
28.2047438396809	0\\
28.2044060973016	0\\
28.2040683417891	0\\
28.2037305731422	0\\
28.2033927913599	0\\
28.2030549964414	0\\
28.2027171883854	0\\
28.202379367191	0\\
28.2020415328572	0\\
28.201703685383	0\\
28.2013658247672	0\\
28.201027951009	0\\
28.2006900641073	0\\
28.2003521640611	0\\
28.2000142508692	0\\
28.1996763245308	0\\
28.1993383850448	0\\
28.1990004324102	0\\
28.1986624666259	0\\
28.1983244876909	0\\
28.1979864956043	0\\
28.1976484903649	0\\
28.1973104719717	0\\
28.1969724404238	0\\
28.1966343957201	0\\
28.1962963378596	0\\
28.1959582668413	0\\
28.195620182664	0\\
28.1952820853269	0\\
28.1949439748289	0\\
28.1946058511689	0\\
28.194267714346	0\\
28.1939295643591	0\\
28.1935914012071	0\\
28.1932532248891	0\\
28.1929150354041	0\\
28.192576832751	0\\
28.1922386169287	0\\
28.1919003879363	0\\
28.1915621457728	0\\
28.191223890437	0\\
28.190885621928	0\\
28.1905473402448	0\\
28.1902090453863	0\\
28.1898707373515	0\\
28.1895324161394	0\\
28.189194081749	0\\
28.1888557341791	0\\
28.1885173734289	0\\
28.1881789994972	0\\
28.1878406123831	0\\
28.1875022120854	0\\
28.1871637986033	0\\
28.1868253719356	0\\
28.1864869320814	0\\
28.1861484790396	0\\
28.1858100128091	0\\
28.185471533389	0\\
28.1851330407782	0\\
28.1847945349757	0\\
28.1844560159805	0\\
28.1841174837915	0\\
28.1837789384077	0\\
28.1834403798281	0\\
28.1831018080516	0\\
28.1827632230773	0\\
28.182424624904	0\\
28.1820860135308	0\\
28.1817473889567	0\\
28.1814087511806	0\\
28.1810701002014	0\\
28.1807314360182	0\\
28.1803927586299	0\\
28.1800540680354	0\\
28.1797153642339	0\\
28.1793766472241	0\\
28.1790379170052	0\\
28.178699173576	0\\
28.1783604169356	0\\
28.1780216470829	0\\
28.1776828640168	0\\
28.1773440677364	0\\
28.1770052582406	0\\
28.1766664355284	0\\
28.1763275995987	0\\
28.1759887504505	0\\
28.1756498880829	0\\
28.1753110124947	0\\
28.1749721236849	0\\
28.1746332216525	0\\
28.1742943063965	0\\
28.1739553779158	0\\
28.1736164362094	0\\
28.1732774812762	0\\
28.1729385131153	0\\
28.1725995317256	0\\
28.1722605371061	0\\
28.1719215292557	0\\
28.1715825081734	0\\
28.1712434738581	0\\
28.1709044263089	0\\
28.1705653655247	0\\
28.1702262915045	0\\
28.1698872042472	0\\
28.1695481037518	0\\
28.1692089900173	0\\
28.1688698630426	0\\
28.1685307228267	0\\
28.1681915693686	0\\
28.1678524026672	0\\
28.1675132227215	0\\
28.1671740295305	0\\
28.1668348230931	0\\
28.1664956034083	0\\
28.1661563704751	0\\
28.1658171242923	0\\
28.1654778648591	0\\
28.1651385921743	0\\
28.164799306237	0\\
28.164460007046	0\\
28.1641206946004	0\\
28.1637813688991	0\\
28.1634420299411	0\\
28.1631026777253	0\\
28.1627633122507	0\\
28.1624239335163	0\\
28.162084541521	0\\
28.1617451362638	0\\
28.1614057177437	0\\
28.1610662859596	0\\
28.1607268409105	0\\
28.1603873825953	0\\
28.1600479110131	0\\
28.1597084261627	0\\
28.1593689280432	0\\
28.1590294166535	0\\
28.1586898919925	0\\
28.1583503540593	0\\
28.1580108028527	0\\
28.1576712383718	0\\
28.1573316606155	0\\
28.1569920695828	0\\
28.1566524652726	0\\
28.156312847684	0\\
28.1559732168157	0\\
28.1556335726669	0\\
28.1552939152365	0\\
28.1549542445235	0\\
28.1546145605267	0\\
28.1542748632452	0\\
28.1539351526779	0\\
28.1535954288239	0\\
28.1532556916819	0\\
28.1529159412511	0\\
28.1525761775304	0\\
28.1522364005186	0\\
28.1518966102149	0\\
28.1515568066181	0\\
28.1512169897272	0\\
28.1508771595412	0\\
28.1505373160591	0\\
28.1501974592797	0\\
28.149857589202	0\\
28.1495177058251	0\\
28.1491778091479	0\\
28.1488378991692	0\\
28.1484979758882	0\\
28.1481580393036	0\\
28.1478180894146	0\\
28.1474781262201	0\\
28.1471381497189	0\\
28.1467981599102	0\\
28.1464581567928	0\\
28.1461181403656	0\\
28.1457781106278	0\\
28.1454380675781	0\\
28.1450980112156	0\\
28.1447579415392	0\\
28.1444178585479	0\\
28.1440777622407	0\\
28.1437376526164	0\\
28.1433975296741	0\\
28.1430573934127	0\\
28.1427172438312	0\\
28.1423770809285	0\\
28.1420369047035	0\\
28.1416967151554	0\\
28.1413565122829	0\\
28.141016296085	0\\
28.1406760665608	0\\
28.1403358237091	0\\
28.1399955675289	0\\
28.1396552980192	0\\
28.139315015179	0\\
28.1389747190071	0\\
28.1386344095025	0\\
28.1382940866643	0\\
28.1379537504913	0\\
28.1376134009825	0\\
28.1372730381369	0\\
28.1369326619533	0\\
28.1365922724309	0\\
28.1362518695685	0\\
28.135911453365	0\\
28.1355710238195	0\\
28.1352305809309	0\\
28.1348901246981	0\\
28.1345496551201	0\\
28.1342091721958	0\\
28.1338686759242	0\\
28.1335281663043	0\\
28.1331876433351	0\\
28.1328471070153	0\\
28.1325065573441	0\\
28.1321659943204	0\\
28.131825417943	0\\
28.1314848282111	0\\
28.1311442251235	0\\
28.1308036086791	0\\
28.130462978877	0\\
28.130122335716	0\\
28.1297816791952	0\\
28.1294410093135	0\\
28.1291003260698	0\\
28.1287596294632	0\\
28.1284189194924	0\\
28.1280781961566	0\\
28.1277374594546	0\\
28.1273967093854	0\\
28.1270559459479	0\\
28.1267151691411	0\\
28.126374378964	0\\
28.1260335754155	0\\
28.1256927584945	0\\
28.1253519282001	0\\
28.1250110845311	0\\
28.1246702274865	0\\
28.1243293570652	0\\
28.1239884732663	0\\
28.1236475760886	0\\
28.1233066655311	0\\
28.1229657415927	0\\
28.1226248042725	0\\
28.1222838535693	0\\
28.1219428894821	0\\
28.1216019120098	0\\
28.1212609211515	0\\
28.120919916906	0\\
28.1205788992723	0\\
28.1202378682493	0\\
28.1198968238361	0\\
28.1195557660315	0\\
28.1192146948344	0\\
28.1188736102439	0\\
28.1185325122589	0\\
28.1181914008784	0\\
28.1178502761012	0\\
28.1175091379263	0\\
28.1171679863527	0\\
28.1168268213794	0\\
28.1164856430052	0\\
28.1161444512291	0\\
28.1158032460501	0\\
28.1154620274671	0\\
28.1151207954791	0\\
28.114779550085	0\\
28.1144382912837	0\\
28.1140970190743	0\\
28.1137557334555	0\\
28.1134144344265	0\\
28.1130731219861	0\\
28.1127317961333	0\\
28.1123904568671	0\\
28.1120491041863	0\\
28.1117077380899	0\\
28.1113663585769	0\\
28.1110249656462	0\\
28.1106835592968	0\\
28.1103421395275	0\\
28.1100007063374	0\\
28.1096592597254	0\\
28.1093177996905	0\\
28.1089763262315	0\\
28.1086348393475	0\\
28.1082933390373	0\\
28.1079518252999	0\\
28.1076102981343	0\\
28.1072687575394	0\\
28.1069272035142	0\\
28.1065856360575	0\\
28.1062440551684	0\\
28.1059024608457	0\\
28.1055608530885	0\\
28.1052192318956	0\\
28.1048775972661	0\\
28.1045359491988	0\\
28.1041942876927	0\\
28.1038526127467	0\\
28.1035109243598	0\\
28.1031692225309	0\\
28.102827507259	0\\
28.102485778543	0\\
28.1021440363818	0\\
28.1018022807745	0\\
28.1014605117198	0\\
28.1011187292168	0\\
28.1007769332645	0\\
28.1004351238617	0\\
28.1000933010074	0\\
28.0997514647005	0\\
28.09940961494	0\\
28.0990677517249	0\\
28.098725875054	0\\
28.0983839849263	0\\
28.0980420813407	0\\
28.0977001642963	0\\
28.0973582337918	0\\
28.0970162898263	0\\
28.0966743323988	0\\
28.096332361508	0\\
28.0959903771531	0\\
28.0956483793329	0\\
28.0953063680463	0\\
28.0949643432923	0\\
28.0946223050699	0\\
28.094280253378	0\\
28.0939381882155	0\\
28.0935961095813	0\\
28.0932540174744	0\\
28.0929119118938	0\\
28.0925697928384	0\\
28.092227660307	0\\
28.0918855142988	0\\
28.0915433548125	0\\
28.0912011818471	0\\
28.0908589954016	0\\
28.0905167954749	0\\
28.090174582066	0\\
28.0898323551737	0\\
28.0894901147971	0\\
28.089147860935	0\\
28.0888055935864	0\\
28.0884633127502	0\\
28.0881210184254	0\\
28.0877787106109	0\\
28.0874363893056	0\\
28.0870940545085	0\\
28.0867517062185	0\\
28.0864093444346	0\\
28.0860669691557	0\\
28.0857245803806	0\\
28.0853821781085	0\\
28.0850397623381	0\\
28.0846973330685	0\\
28.0843548902985	0\\
28.0840124340271	0\\
28.0836699642533	0\\
28.0833274809759	0\\
28.0829849841939	0\\
28.0826424739063	0\\
28.0822999501119	0\\
28.0819574128098	0\\
28.0816148619988	0\\
28.0812722976778	0\\
28.0809297198459	0\\
28.080587128502	0\\
28.0802445236449	0\\
28.0799019052736	0\\
28.0795592733871	0\\
28.0792166279843	0\\
28.078873969064	0\\
28.0785312966254	0\\
28.0781886106672	0\\
28.0778459111884	0\\
28.077503198188	0\\
28.0771604716649	0\\
28.0768177316179	0\\
28.0764749780461	0\\
28.0761322109485	0\\
28.0757894303238	0\\
28.075446636171	0\\
28.0751038284892	0\\
28.0747610072771	0\\
28.0744181725338	0\\
28.0740753242582	0\\
28.0737324624491	0\\
28.0733895871056	0\\
28.0730466982266	0\\
28.072703795811	0\\
28.0723608798577	0\\
28.0720179503656	0\\
28.0716750073338	0\\
28.071332050761	0\\
28.0709890806464	0\\
28.0706460969887	0\\
28.0703030997869	0\\
28.06996008904	0\\
28.0696170647468	0\\
28.0692740269063	0\\
28.0689309755175	0\\
28.0685879105793	0\\
28.0682448320905	0\\
28.0679017400502	0\\
28.0675586344572	0\\
28.0672155153105	0\\
28.066872382609	0\\
28.0665292363516	0\\
28.0661860765374	0\\
28.0658429031651	0\\
28.0654997162337	0\\
28.0651565157422	0\\
28.0648133016895	0\\
28.0644700740745	0\\
28.0641268328961	0\\
28.0637835781533	0\\
28.063440309845	0\\
28.0630970279701	0\\
28.0627537325276	0\\
28.0624104235164	0\\
28.0620671009353	0\\
28.0617237647834	0\\
28.0613804150595	0\\
28.0610370517627	0\\
28.0606936748917	0\\
28.0603502844456	0\\
28.0600068804233	0\\
28.0596634628236	0\\
28.0593200316456	0\\
28.0589765868881	0\\
28.0586331285501	0\\
28.0582896566305	0\\
28.0579461711282	0\\
28.0576026720421	0\\
28.0572591593713	0\\
28.0569156331145	0\\
28.0565720932708	0\\
28.056228539839	0\\
28.0558849728181	0\\
28.0555413922069	0\\
28.0551977980046	0\\
28.0548541902098	0\\
28.0545105688217	0\\
28.054166933839	0\\
28.0538232852608	0\\
28.0534796230859	0\\
28.0531359473133	0\\
28.0527922579419	0\\
28.0524485549705	0\\
28.0521048383983	0\\
28.051761108224	0\\
28.0514173644466	0\\
28.051073607065	0\\
28.0507298360782	0\\
28.050386051485	0\\
28.0500422532844	0\\
28.0496984414753	0\\
28.0493546160566	0\\
28.0490107770273	0\\
28.0486669243862	0\\
28.0483230581324	0\\
28.0479791782647	0\\
28.047635284782	0\\
28.0472913776832	0\\
28.0469474569674	0\\
28.0466035226333	0\\
28.04625957468	0\\
28.0459156131063	0\\
28.0455716379112	0\\
28.0452276490936	0\\
28.0448836466524	0\\
28.0445396305865	0\\
28.0441956008949	0\\
28.0438515575765	0\\
28.0435075006301	0\\
28.0431634300548	0\\
28.0428193458494	0\\
28.0424752480128	0\\
28.042131136544	0\\
28.0417870114419	0\\
28.0414428727055	0\\
28.0410987203335	0\\
28.040754554325	0\\
28.0404103746789	0\\
28.0400661813941	0\\
28.0397219744695	0\\
28.039377753904	0\\
28.0390335196966	0\\
28.0386892718461	0\\
28.0383450103515	0\\
28.0380007352117	0\\
28.0376564464256	0\\
28.0373121439922	0\\
28.0369678279104	0\\
28.036623498179	0\\
28.036279154797	0\\
28.0359347977633	0\\
28.0355904270768	0\\
28.0352460427365	0\\
28.0349016447413	0\\
28.03455723309	0\\
28.0342128077816	0\\
28.0338683688151	0\\
28.0335239161892	0\\
28.033179449903	0\\
28.0328349699554	0\\
28.0324904763453	0\\
28.0321459690715	0\\
28.0318014481331	0\\
28.0314569135289	0\\
28.0311123652579	0\\
28.0307678033189	0\\
28.0304232277108	0\\
28.0300786384327	0\\
28.0297340354834	0\\
28.0293894188618	0\\
28.0290447885668	0\\
28.0287001445974	0\\
28.0283554869525	0\\
28.0280108156309	0\\
28.0276661306316	0\\
28.0273214319535	0\\
28.0269767195956	0\\
28.0266319935567	0\\
28.0262872538357	0\\
28.0259425004317	0\\
28.0255977333434	0\\
28.0252529525698	0\\
28.0249081581097	0\\
28.0245633499623	0\\
28.0242185281262	0\\
28.0238736926005	0\\
28.0235288433841	0\\
28.0231839804758	0\\
28.0228391038746	0\\
28.0224942135794	0\\
28.0221493095891	0\\
28.0218043919027	0\\
28.021459460519	0\\
28.0211145154369	0\\
28.0207695566554	0\\
28.0204245841733	0\\
28.0200795979896	0\\
28.0197345981033	0\\
28.0193895845131	0\\
28.0190445572181	0\\
28.018699516217	0\\
28.018354461509	0\\
28.0180093930927	0\\
28.0176643109673	0\\
28.0173192151315	0\\
28.0169741055842	0\\
28.0166289823245	0\\
28.0162838453512	0\\
28.0159386946632	0\\
28.0155935302594	0\\
28.0152483521387	0\\
28.0149031603001	0\\
28.0145579547424	0\\
28.0142127354646	0\\
28.0138675024656	0\\
28.0135222557442	0\\
28.0131769952995	0\\
28.0128317211302	0\\
28.0124864332353	0\\
28.0121411316138	0\\
28.0117958162645	0\\
28.0114504871863	0\\
28.0111051443781	0\\
28.0107597878389	0\\
28.0104144175676	0\\
28.010069033563	0\\
28.0097236358241	0\\
28.0093782243498	0\\
28.0090327991389	0\\
28.0086873601905	0\\
28.0083419075033	0\\
28.0079964410764	0\\
28.0076509609086	0\\
28.0073054669988	0\\
28.0069599593459	0\\
28.0066144379489	0\\
28.0062689028066	0\\
28.005923353918	0\\
28.0055777912819	0\\
28.0052322148973	0\\
28.004886624763	0\\
28.004541020878	0\\
28.0041954032412	0\\
28.0038497718515	0\\
28.0035041267078	0\\
28.003158467809	0\\
28.0028127951539	0\\
28.0024671087416	0\\
28.0021214085709	0\\
28.0017756946407	0\\
28.0014299669499	0\\
28.0010842254975	0\\
28.0007384702823	0\\
28.0003927013032	0\\
28.0000469185591	0\\
27.999701122049	0\\
27.9993553117717	0\\
27.9990094877262	0\\
27.9986636499114	0\\
27.9983177983261	0\\
27.9979719329692	0\\
27.9976260538397	0\\
27.9972801609365	0\\
27.9969342542584	0\\
27.9965883338045	0\\
27.9962423995735	0\\
27.9958964515643	0\\
27.995550489776	0\\
27.9952045142073	0\\
27.9948585248572	0\\
27.9945125217246	0\\
27.9941665048084	0\\
27.9938204741075	0\\
27.9934744296208	0\\
27.9931283713471	0\\
27.9927822992855	0\\
27.9924362134347	0\\
27.9920901137938	0\\
27.9917440003615	0\\
27.9913978731369	0\\
27.9910517321187	0\\
27.9907055773059	0\\
27.9903594086975	0\\
27.9900132262922	0\\
27.989667030089	0\\
27.9893208200869	0\\
27.9889745962846	0\\
27.9886283586812	0\\
27.9882821072754	0\\
27.9879358420662	0\\
27.9875895630526	0\\
27.9872432702333	0\\
27.9868969636074	0\\
27.9865506431736	0\\
27.9862043089309	0\\
27.9858579608782	0\\
27.9855115990144	0\\
27.9851652233384	0\\
27.9848188338491	0\\
27.9844724305454	0\\
27.9841260134262	0\\
27.9837795824903	0\\
27.9834331377367	0\\
27.9830866791643	0\\
27.982740206772	0\\
27.9823937205587	0\\
27.9820472205232	0\\
27.9817007066645	0\\
27.9813541789815	0\\
27.981007637473	0\\
27.980661082138	0\\
27.9803145129753	0\\
27.9799679299839	0\\
27.9796213331626	0\\
27.9792747225104	0\\
27.9789280980261	0\\
27.9785814597087	0\\
27.978234807557	0\\
27.9778881415699	0\\
27.9775414617464	0\\
27.9771947680852	0\\
27.9768480605854	0\\
27.9765013392458	0\\
27.9761546040653	0\\
27.9758078550428	0\\
27.9754610921772	0\\
27.9751143154674	0\\
27.9747675249122	0\\
27.9744207205107	0\\
27.9740739022616	0\\
27.9737270701639	0\\
27.9733802242164	0\\
27.9730333644181	0\\
27.9726864907678	0\\
27.9723396032645	0\\
27.971992701907	0\\
27.9716457866943	0\\
27.9712988576252	0\\
27.9709519146986	0\\
27.9706049579134	0\\
27.9702579872685	0\\
27.9699110027627	0\\
27.9695640043951	0\\
27.9692169921645	0\\
27.9688699660697	0\\
27.9685229261097	0\\
27.9681758722833	0\\
27.9678288045895	0\\
27.9674817230271	0\\
27.9671346275951	0\\
27.9667875182923	0\\
27.9664403951175	0\\
27.9660932580698	0\\
27.965746107148	0\\
27.965398942351	0\\
27.9650517636776	0\\
27.9647045711269	0\\
27.9643573646975	0\\
27.9640101443886	0\\
27.9636629101988	0\\
27.9633156621272	0\\
27.9629684001726	0\\
27.9626211243339	0\\
27.9622738346101	0\\
27.9619265309999	0\\
27.9615792135022	0\\
27.9612318821161	0\\
27.9608845368403	0\\
27.9605371776737	0\\
27.9601898046153	0\\
27.9598424176639	0\\
27.9594950168184	0\\
27.9591476020777	0\\
27.9588001734408	0\\
27.9584527309063	0\\
27.9581052744734	0\\
27.9577578041408	0\\
27.9574103199074	0\\
27.9570628217722	0\\
27.956715309734	0\\
27.9563677837917	0\\
27.9560202439441	0\\
27.9556726901903	0\\
27.955325122529	0\\
27.9549775409592	0\\
27.9546299454797	0\\
27.9542823360894	0\\
27.9539347127873	0\\
27.9535870755721	0\\
27.9532394244428	0\\
27.9528917593983	0\\
27.9525440804375	0\\
27.9521963875592	0\\
27.9518486807623	0\\
27.9515009600458	0\\
27.9511532254084	0\\
27.9508054768491	0\\
27.9504577143668	0\\
27.9501099379604	0\\
27.9497621476287	0\\
27.9494143433706	0\\
27.949066525185	0\\
27.9487186930709	0\\
27.948370847027	0\\
27.9480229870522	0\\
27.9476751131455	0\\
27.9473272253058	0\\
27.9469793235318	0\\
27.9466314078226	0\\
27.9462834781769	0\\
27.9459355345937	0\\
27.9455875770719	0\\
27.9452396056102	0\\
27.9448916202077	0\\
27.9445436208632	0\\
27.9441956075756	0\\
27.9438475803437	0\\
27.9434995391665	0\\
27.9431514840428	0\\
27.9428034149715	0\\
27.9424553319516	0\\
27.9421072349817	0\\
27.941759124061	0\\
27.9414109991882	0\\
27.9410628603622	0\\
27.9407147075819	0\\
27.9403665408462	0\\
27.9400183601539	0\\
27.939670165504	0\\
27.9393219568953	0\\
27.9389737343268	0\\
27.9386254977972	0\\
27.9382772473054	0\\
27.9379289828505	0\\
27.9375807044311	0\\
27.9372324120463	0\\
27.9368841056949	0\\
27.9365357853757	0\\
27.9361874510876	0\\
27.9358391028296	0\\
27.9354907406005	0\\
27.9351423643992	0\\
27.9347939742246	0\\
27.9344455700755	0\\
27.9340971519508	0\\
27.9337487198494	0\\
27.9334002737702	0\\
27.9330518137121	0\\
27.9327033396739	0\\
27.9323548516545	0\\
27.9320063496528	0\\
27.9316578336677	0\\
27.931309303698	0\\
27.9309607597426	0\\
27.9306122018005	0\\
27.9302636298704	0\\
27.9299150439513	0\\
27.9295664440421	0\\
27.9292178301415	0\\
27.9288692022486	0\\
27.9285205603621	0\\
27.9281719044809	0\\
27.927823234604	0\\
27.9274745507301	0\\
27.9271258528583	0\\
27.9267771409872	0\\
27.926428415116	0\\
27.9260796752433	0\\
27.925730921368	0\\
27.9253821534892	0\\
27.9250333716055	0\\
27.924684575716	0\\
27.9243357658194	0\\
27.9239869419147	0\\
27.9236381040007	0\\
27.9232892520764	0\\
27.9229403861405	0\\
27.9225915061919	0\\
27.9222426122296	0\\
27.9218937042523	0\\
27.9215447822591	0\\
27.9211958462487	0\\
27.92084689622	0\\
27.9204979321719	0\\
27.9201489541032	0\\
27.9197999620129	0\\
27.9194509558999	0\\
27.9191019357629	0\\
27.9187529016008	0\\
27.9184038534126	0\\
27.9180547911971	0\\
27.9177057149532	0\\
27.9173566246798	0\\
27.9170075203756	0\\
27.9166584020397	0\\
27.9163092696708	0\\
27.9159601232678	0\\
27.9156109628297	0\\
27.9152617883553	0\\
27.9149125998434	0\\
27.9145633972929	0\\
27.9142141807027	0\\
27.9138649500717	0\\
27.9135157053987	0\\
27.9131664466827	0\\
27.9128171739224	0\\
27.9124678871167	0\\
27.9121185862646	0\\
27.9117692713649	0\\
27.9114199424164	0\\
27.9110705994181	0\\
27.9107212423687	0\\
27.9103718712673	0\\
27.9100224861125	0\\
27.9096730869034	0\\
27.9093236736387	0\\
27.9089742463174	0\\
27.9086248049384	0\\
27.9082753495004	0\\
27.9079258800023	0\\
27.9075763964431	0\\
27.9072268988215	0\\
27.9068773871366	0\\
27.906527861387	0\\
27.9061783215718	0\\
27.9058287676897	0\\
27.9054791997396	0\\
27.9051296177204	0\\
27.904780021631	0\\
27.9044304114703	0\\
27.904080787237	0\\
27.9037311489301	0\\
27.9033814965484	0\\
27.9030318300908	0\\
27.9026821495562	0\\
27.9023324549435	0\\
27.9019827462514	0\\
27.9016330234789	0\\
27.9012832866248	0\\
27.9009335356881	0\\
27.9005837706675	0\\
27.9002339915619	0\\
27.8998841983702	0\\
27.8995343910913	0\\
27.899184569724	0\\
27.8988347342672	0\\
27.8984848847197	0\\
27.8981350210805	0\\
27.8977851433484	0\\
27.8974352515222	0\\
27.8970853456008	0\\
27.8967354255831	0\\
27.896385491468	0\\
27.8960355432542	0\\
27.8956855809407	0\\
27.8953356045264	0\\
27.8949856140101	0\\
27.8946356093906	0\\
27.8942855906668	0\\
27.8939355578377	0\\
27.893585510902	0\\
27.8932354498586	0\\
27.8928853747064	0\\
27.8925352854442	0\\
27.8921851820709	0\\
27.8918350645854	0\\
27.8914849329865	0\\
27.8911347872731	0\\
27.8907846274441	0\\
27.8904344534983	0\\
27.8900842654345	0\\
27.8897340632517	0\\
27.8893838469487	0\\
27.8890336165243	0\\
27.8886833719775	0\\
27.888333113307	0\\
27.8879828405118	0\\
27.8876325535907	0\\
27.8872822525426	0\\
27.8869319373663	0\\
27.8865816080606	0\\
27.8862312646245	0\\
27.8858809070568	0\\
27.8855305353564	0\\
27.8851801495221	0\\
27.8848297495528	0\\
27.8844793354474	0\\
27.8841289072046	0\\
27.8837784648234	0\\
27.8834280083026	0\\
27.8830775376411	0\\
27.8827270528378	0\\
27.8823765538914	0\\
27.8820260408009	0\\
27.8816755135651	0\\
27.8813249721829	0\\
27.8809744166531	0\\
27.8806238469746	0\\
27.8802732631463	0\\
27.8799226651669	0\\
27.8795720530354	0\\
27.8792214267507	0\\
27.8788707863115	0\\
27.8785201317167	0\\
27.8781694629653	0\\
27.877818780056	0\\
27.8774680829877	0\\
27.8771173717592	0\\
27.8767666463695	0\\
27.8764159068173	0\\
27.8760651531016	0\\
27.8757143852212	0\\
27.8753636031749	0\\
27.8750128069616	0\\
27.8746619965802	0\\
27.8743111720294	0\\
27.8739603333083	0\\
27.8736094804155	0\\
27.8732586133501	0\\
27.8729077321107	0\\
27.8725568366964	0\\
27.8722059271058	0\\
27.871855003338	0\\
27.8715040653918	0\\
27.8711531132659	0\\
27.8708021469593	0\\
27.8704511664708	0\\
27.8701001717992	0\\
27.8697491629435	0\\
27.8693981399025	0\\
27.8690471026749	0\\
27.8686960512598	0\\
27.8683449856559	0\\
27.867993905862	0\\
27.8676428118771	0\\
27.8672917037	0\\
27.8669405813296	0\\
27.8665894447646	0\\
27.866238294004	0\\
27.8658871290466	0\\
27.8655359498912	0\\
27.8651847565368	0\\
27.864833548982	0\\
27.8644823272259	0\\
27.8641310912673	0\\
27.863779841105	0\\
27.8634285767378	0\\
27.8630772981646	0\\
27.8627260053843	0\\
27.8623746983958	0\\
27.8620233771977	0\\
27.8616720417891	0\\
27.8613206921688	0\\
27.8609693283356	0\\
27.8606179502883	0\\
27.8602665580259	0\\
27.8599151515471	0\\
27.8595637308508	0\\
27.8592122959359	0\\
27.8588608468012	0\\
27.8585093834456	0\\
27.8581579058679	0\\
27.8578064140669	0\\
27.8574549080416	0\\
27.8571033877907	0\\
27.8567518533131	0\\
27.8564003046077	0\\
27.8560487416733	0\\
27.8556971645087	0\\
27.8553455731128	0\\
27.8549939674845	0\\
27.8546423476226	0\\
27.8542907135259	0\\
27.8539390651933	0\\
27.8535874026237	0\\
27.8532357258158	0\\
27.8528840347685	0\\
27.8525323294808	0\\
27.8521806099513	0\\
27.8518288761791	0\\
27.8514771281628	0\\
27.8511253659014	0\\
27.8507735893938	0\\
27.8504217986386	0\\
27.8500699936349	0\\
27.8497181743815	0\\
27.8493663408771	0\\
27.8490144931206	0\\
27.848662631111	0\\
27.848310754847	0\\
27.8479588643274	0\\
27.8476069595512	0\\
27.8472550405171	0\\
27.8469031072241	0\\
27.8465511596709	0\\
27.8461991978564	0\\
27.8458472217794	0\\
27.8454952314388	0\\
27.8451432268335	0\\
27.8447912079622	0\\
27.8444391748239	0\\
27.8440871274173	0\\
27.8437350657413	0\\
27.8433829897947	0\\
27.8430308995765	0\\
27.8426787950853	0\\
27.8423266763202	0\\
27.8419745432799	0\\
27.8416223959632	0\\
27.841270234369	0\\
27.8409180584962	0\\
27.8405658683436	0\\
27.84021366391	0\\
27.8398614451943	0\\
27.8395092121952	0\\
27.8391569649118	0\\
27.8388047033427	0\\
27.8384524274869	0\\
27.8381001373431	0\\
27.8377478329103	0\\
27.8373955141872	0\\
27.8370431811728	0\\
27.8366908338657	0\\
27.836338472265	0\\
27.8359860963694	0\\
27.8356337061777	0\\
27.8352813016888	0\\
27.8349288829016	0\\
27.8345764498149	0\\
27.8342240024274	0\\
27.8338715407382	0\\
27.8335190647459	0\\
27.8331665744495	0\\
27.8328140698478	0\\
27.8324615509395	0\\
27.8321090177237	0\\
27.831756470199	0\\
27.8314039083643	0\\
27.8310513322186	0\\
27.8306987417605	0\\
27.830346136989	0\\
27.8299935179028	0\\
27.8296408845009	0\\
27.8292882367821	0\\
27.8289355747451	0\\
27.8285828983889	0\\
27.8282302077122	0\\
27.827877502714	0\\
27.827524783393	0\\
27.8271720497481	0\\
27.8268193017781	0\\
27.8264665394818	0\\
27.8261137628582	0\\
27.825760971906	0\\
27.825408166624	0\\
27.8250553470111	0\\
27.8247025130662	0\\
27.824349664788	0\\
27.8239968021755	0\\
27.8236439252274	0\\
27.8232910339425	0\\
27.8229381283198	0\\
27.822585208358	0\\
27.8222322740561	0\\
27.8218793254127	0\\
27.8215263624267	0\\
27.8211733850971	0\\
27.8208203934226	0\\
27.820467387402	0\\
27.8201143670342	0\\
27.819761332318	0\\
27.8194082832523	0\\
27.8190552198359	0\\
27.8187021420676	0\\
27.8183490499462	0\\
27.8179959434707	0\\
27.8176428226397	0\\
27.8172896874522	0\\
27.816936537907	0\\
27.8165833740029	0\\
27.8162301957388	0\\
27.8158770031134	0\\
27.8155237961256	0\\
27.8151705747743	0\\
27.8148173390583	0\\
27.8144640889764	0\\
27.8141108245274	0\\
27.8137575457102	0\\
27.8134042525236	0\\
27.8130509449664	0\\
27.8126976230375	0\\
27.8123442867357	0\\
27.8119909360598	0\\
27.8116375710087	0\\
27.8112841915812	0\\
27.810930797776	0\\
27.8105773895922	0\\
27.8102239670284	0\\
27.8098705300835	0\\
27.8095170787563	0\\
27.8091636130457	0\\
27.8088101329505	0\\
27.8084566384695	0\\
27.8081031296016	0\\
27.8077496063456	0\\
27.8073960687002	0\\
27.8070425166645	0\\
27.806688950237	0\\
27.8063353694168	0\\
27.8059817742026	0\\
27.8056281645933	0\\
27.8052745405876	0\\
27.8049209021844	0\\
27.8045672493826	0\\
27.8042135821809	0\\
27.8038599005783	0\\
27.8035062045734	0\\
27.8031524941652	0\\
27.8027987693525	0\\
27.802445030134	0\\
27.8020912765087	0\\
27.8017375084753	0\\
27.8013837260327	0\\
27.8010299291797	0\\
27.8006761179152	0\\
27.8003222922379	0\\
27.7999684521467	0\\
27.7996145976403	0\\
27.7992607287178	0\\
27.7989068453777	0\\
27.7985529476191	0\\
27.7981990354407	0\\
27.7978451088413	0\\
27.7974911678197	0\\
27.7971372123748	0\\
27.7967832425055	0\\
27.7964292582105	0\\
27.7960752594886	0\\
27.7957212463387	0\\
27.7953672187596	0\\
27.7950131767501	0\\
27.7946591203091	0\\
27.7943050494354	0\\
27.7939509641277	0\\
27.793596864385	0\\
27.793242750206	0\\
27.7928886215896	0\\
27.7925344785345	0\\
27.7921803210397	0\\
27.7918261491039	0\\
27.791471962726	0\\
27.7911177619047	0\\
27.790763546639	0\\
27.7904093169275	0\\
27.7900550727693	0\\
27.7897008141629	0\\
27.7893465411074	0\\
27.7889922536014	0\\
27.7886379516439	0\\
27.7882836352337	0\\
27.7879293043695	0\\
27.7875749590502	0\\
27.7872205992746	0\\
27.7868662250416	0\\
27.7865118363499	0\\
27.7861574331983	0\\
27.7858030155858	0\\
27.7854485835111	0\\
27.785094136973	0\\
27.7847396759703	0\\
27.784385200502	0\\
27.7840307105667	0\\
27.7836762061633	0\\
27.7833216872906	0\\
27.7829671539476	0\\
27.7826126061328	0\\
27.7822580438453	0\\
27.7819034670837	0\\
27.781548875847	0\\
27.7811942701339	0\\
27.7808396499433	0\\
27.7804850152739	0\\
27.7801303661246	0\\
27.7797757024943	0\\
27.7794210243816	0\\
27.7790663317856	0\\
27.7787116247048	0\\
27.7783569031383	0\\
27.7780021670847	0\\
27.777647416543	0\\
27.7772926515119	0\\
27.7769378719902	0\\
27.7765830779768	0\\
27.7762282694705	0\\
27.77587344647	0\\
27.7755186089743	0\\
27.7751637569821	0\\
27.7748088904922	0\\
27.7744540095035	0\\
27.7740991140148	0\\
27.7737442040248	0\\
27.7733892795324	0\\
27.7730343405365	0\\
27.7726793870358	0\\
27.7723244190291	0\\
27.7719694365153	0\\
27.7716144394932	0\\
27.7712594279615	0\\
27.7709044019191	0\\
27.7705493613649	0\\
27.7701943062976	0\\
27.769839236716	0\\
27.769484152619	0\\
27.7691290540053	0\\
27.7687739408739	0\\
27.7684188132234	0\\
27.7680636710527	0\\
27.7677085143606	0\\
27.767353343146	0\\
27.7669981574076	0\\
27.7666429571443	0\\
27.7662877423549	0\\
27.7659325130381	0\\
27.7655772691928	0\\
27.7652220108178	0\\
27.7648667379119	0\\
27.764511450474	0\\
27.7641561485028	0\\
27.7638008319971	0\\
27.7634455009558	0\\
27.7630901553776	0\\
27.7627347952615	0\\
27.7623794206061	0\\
27.7620240314103	0\\
27.761668627673	0\\
27.7613132093928	0\\
27.7609577765687	0\\
27.7606023291994	0\\
27.7602468672838	0\\
27.7598913908206	0\\
27.7595358998087	0\\
27.7591803942469	0\\
27.758824874134	0\\
27.7584693394687	0\\
27.75811379025	0\\
27.7577582264766	0\\
27.7574026481472	0\\
27.7570470552609	0\\
27.7566914478162	0\\
27.7563358258121	0\\
27.7559801892474	0\\
27.7556245381208	0\\
27.7552688724311	0\\
27.7549131921773	0\\
27.754557497358	0\\
27.7542017879721	0\\
27.7538460640184	0\\
27.7534903254957	0\\
27.7531345724028	0\\
27.7527788047386	0\\
27.7524230225017	0\\
27.7520672256911	0\\
27.7517114143055	0\\
27.7513555883438	0\\
27.7509997478047	0\\
27.750643892687	0\\
27.7502880229896	0\\
27.7499321387113	0\\
27.7495762398508	0\\
27.749220326407	0\\
27.7488643983787	0\\
27.7485084557646	0\\
27.7481524985637	0\\
27.7477965267746	0\\
27.7474405403962	0\\
27.7470845394273	0\\
27.7467285238667	0\\
27.7463724937133	0\\
27.7460164489657	0\\
27.7456603896228	0\\
27.7453043156835	0\\
27.7449482271465	0\\
27.7445921240106	0\\
27.7442360062746	0\\
27.7438798739374	0\\
27.7435237269976	0\\
27.7431675654543	0\\
27.742811389306	0\\
27.7424551985517	0\\
27.7420989931901	0\\
27.7417427732201	0\\
27.7413865386404	0\\
27.7410302894499	0\\
27.7406740256473	0\\
27.7403177472314	0\\
27.7399614542011	0\\
27.7396051465551	0\\
27.7392488242923	0\\
27.7388924874115	0\\
27.7385361359114	0\\
27.7381797697908	0\\
27.7378233890486	0\\
27.7374669936836	0\\
27.7371105836945	0\\
27.7367541590801	0\\
27.7363977198393	0\\
27.7360412659709	0\\
27.7356847974736	0\\
27.7353283143463	0\\
27.7349718165877	0\\
27.7346153041966	0\\
27.7342587771719	0\\
27.7339022355124	0\\
27.7335456792168	0\\
27.7331891082839	0\\
27.7328325227126	0\\
27.7324759225016	0\\
27.7321193076497	0\\
27.7317626781558	0\\
27.7314060340186	0\\
27.7310493752369	0\\
27.7306927018096	0\\
27.7303360137354	0\\
27.7299793110131	0\\
27.7296225936415	0\\
27.7292658616194	0\\
27.7289091149457	0\\
27.728552353619	0\\
27.7281955776383	0\\
27.7278387870022	0\\
27.7274819817097	0\\
27.7271251617594	0\\
27.7267683271502	0\\
27.7264114778809	0\\
27.7260546139503	0\\
27.7256977353572	0\\
27.7253408421003	0\\
27.7249839341785	0\\
27.7246270115905	0\\
27.7242700743352	0\\
27.7239131224113	0\\
27.7235561558177	0\\
27.7231991745531	0\\
27.7228421786164	0\\
27.7224851680062	0\\
27.7221281427215	0\\
27.721771102761	0\\
27.7214140481235	0\\
27.7210569788078	0\\
27.7206998948127	0\\
27.7203427961369	0\\
27.7199856827794	0\\
27.7196285547388	0\\
27.7192714120139	0\\
27.7189142546037	0\\
27.7185570825067	0\\
27.7181998957219	0\\
27.7178426942481	0\\
27.7174854780839	0\\
27.7171282472283	0\\
27.71677100168	0\\
27.7164137414377	0\\
27.7160564665004	0\\
27.7156991768667	0\\
27.7153418725355	0\\
27.7149845535056	0\\
27.7146272197757	0\\
27.7142698713446	0\\
27.7139125082112	0\\
27.7135551303742	0\\
27.7131977378324	0\\
27.7128403305847	0\\
27.7124829086297	0\\
27.7121254719662	0\\
27.7117680205932	0\\
27.7114105545093	0\\
27.7110530737134	0\\
27.7106955782042	0\\
27.7103380679805	0\\
27.7099805430411	0\\
27.7096230033849	0\\
27.7092654490105	0\\
27.7089078799168	0\\
27.7085502961026	0\\
27.7081926975666	0\\
27.7078350843077	0\\
27.7074774563246	0\\
27.7071198136162	0\\
27.7067621561811	0\\
27.7064044840182	0\\
27.7060467971263	0\\
27.7056890955042	0\\
27.7053313791506	0\\
27.7049736480644	0\\
27.7046159022443	0\\
27.7042581416891	0\\
27.7039003663976	0\\
27.7035425763686	0\\
27.7031847716008	0\\
27.7028269520932	0\\
27.7024691178443	0\\
27.7021112688531	0\\
27.7017534051183	0\\
27.7013955266387	0\\
27.7010376334131	0\\
27.7006797254402	0\\
27.7003218027189	0\\
27.6999638652479	0\\
27.6996059130261	0\\
27.6992479460521	0\\
27.6988899643249	0\\
27.6985319678431	0\\
27.6981739566056	0\\
27.6978159306111	0\\
27.6974578898584	0\\
27.6970998343464	0\\
27.6967417640737	0\\
27.6963836790392	0\\
27.6960255792416	0\\
27.6956674646798	0\\
27.6953093353526	0\\
27.6949511912586	0\\
27.6945930323967	0\\
27.6942348587657	0\\
27.6938766703643	0\\
27.6935184671913	0\\
27.6931602492456	0\\
27.6928020165258	0\\
27.6924437690309	0\\
27.6920855067594	0\\
27.6917272297104	0\\
27.6913689378824	0\\
27.6910106312744	0\\
27.690652309885	0\\
27.6902939737131	0\\
27.6899356227574	0\\
27.6895772570167	0\\
27.6892188764898	0\\
27.6888604811756	0\\
27.6885020710726	0\\
27.6881436461799	0\\
27.687785206496	0\\
27.6874267520198	0\\
27.6870682827501	0\\
27.6867097986857	0\\
27.6863512998253	0\\
27.6859927861677	0\\
27.6856342577117	0\\
27.6852757144561	0\\
27.6849171563996	0\\
27.684558583541	0\\
27.6841999958792	0\\
27.6838413934128	0\\
27.6834827761407	0\\
27.6831241440616	0\\
27.6827654971743	0\\
27.6824068354776	0\\
27.6820481589703	0\\
27.6816894676512	0\\
27.6813307615189	0\\
27.6809720405724	0\\
27.6806133048103	0\\
27.6802545542315	0\\
27.6798957888347	0\\
27.6795370086187	0\\
27.6791782135822	0\\
27.6788194037242	0\\
27.6784605790432	0\\
27.6781017395382	0\\
27.6777428852078	0\\
27.6773840160509	0\\
27.6770251320662	0\\
27.6766662332525	0\\
27.6763073196086	0\\
27.6759483911333	0\\
27.6755894478252	0\\
27.6752304896833	0\\
27.6748715167063	0\\
27.6745125288929	0\\
27.6741535262419	0\\
27.6737945087521	0\\
27.6734354764223	0\\
27.6730764292512	0\\
27.6727173672376	0\\
27.6723582903803	0\\
27.6719991986781	0\\
27.6716400921297	0\\
27.6712809707339	0\\
27.6709218344895	0\\
27.6705626833952	0\\
27.6702035174499	0\\
27.6698443366522	0\\
27.669485141001	0\\
27.6691259304951	0\\
27.6687667051331	0\\
27.6684074649139	0\\
27.6680482098363	0\\
27.667688939899	0\\
27.6673296551008	0\\
27.6669703554404	0\\
27.6666110409167	0\\
27.6662517115284	0\\
27.6658923672742	0\\
27.665533008153	0\\
27.6651736341635	0\\
27.6648142453045	0\\
27.6644548415747	0\\
27.664095422973	0\\
27.663735989498	0\\
27.6633765411486	0\\
27.6630170779235	0\\
27.6626575998215	0\\
27.6622981068414	0\\
27.6619385989819	0\\
27.6615790762418	0\\
27.6612195386199	0\\
27.6608599861149	0\\
27.6605004187256	0\\
27.6601408364508	0\\
27.6597812392892	0\\
27.6594216272396	0\\
27.6590620003008	0\\
27.6587023584716	0\\
27.6583427017506	0\\
27.6579830301367	0\\
27.6576233436287	0\\
27.6572636422253	0\\
27.6569039259252	0\\
27.6565441947273	0\\
27.6561844486303	0\\
27.655824687633	0\\
27.655464911734	0\\
27.6551051209323	0\\
27.6547453152266	0\\
27.6543854946156	0\\
27.6540256590981	0\\
27.6536658086729	0\\
27.6533059433386	0\\
27.6529460630942	0\\
27.6525861679383	0\\
27.6522262578698	0\\
27.6518663328873	0\\
27.6515063929897	0\\
27.6511464381757	0\\
27.6507864684441	0\\
27.6504264837936	0\\
27.650066484223	0\\
27.6497064697311	0\\
27.6493464403166	0\\
27.6489863959783	0\\
27.648626336715	0\\
27.6482662625253	0\\
27.6479061734082	0\\
27.6475460693623	0\\
27.6471859503864	0\\
27.6468258164792	0\\
27.6464656676396	0\\
27.6461055038663	0\\
27.645745325158	0\\
27.6453851315136	0\\
27.6450249229317	0\\
27.6446646994111	0\\
27.6443044609507	0\\
27.6439442075491	0\\
27.6435839392051	0\\
27.6432236559175	0\\
27.642863357685	0\\
27.6425030445064	0\\
27.6421427163805	0\\
27.641782373306	0\\
27.6414220152817	0\\
27.6410616423063	0\\
27.6407012543786	0\\
27.6403408514974	0\\
27.6399804336613	0\\
27.6396200008693	0\\
27.63925955312	0\\
27.6388990904121	0\\
27.6385386127445	0\\
27.638178120116	0\\
27.6378176125252	0\\
27.6374570899709	0\\
27.6370965524518	0\\
27.6367359999669	0\\
27.6363754325147	0\\
27.6360148500941	0\\
27.6356542527037	0\\
27.6352936403425	0\\
27.6349330130091	0\\
27.6345723707022	0\\
27.6342117134207	0\\
27.6338510411633	0\\
27.6334903539287	0\\
27.6331296517157	0\\
27.6327689345231	0\\
27.6324082023496	0\\
27.632047455194	0\\
27.631686693055	0\\
27.6313259159314	0\\
27.630965123822	0\\
27.6306043167254	0\\
27.6302434946405	0\\
27.629882657566	0\\
27.6295218055006	0\\
27.6291609384431	0\\
27.6288000563924	0\\
27.628439159347	0\\
27.6280782473058	0\\
27.6277173202675	0\\
27.627356378231	0\\
27.6269954211948	0\\
27.6266344491579	0\\
27.6262734621188	0\\
27.6259124600765	0\\
27.6255514430296	0\\
27.625190410977	0\\
27.6248293639172	0\\
27.6244683018492	0\\
27.6241072247717	0\\
27.6237461326833	0\\
27.6233850255829	0\\
27.6230239034692	0\\
27.622662766341	0\\
27.622301614197	0\\
27.621940447036	0\\
27.6215792648567	0\\
27.6212180676578	0\\
27.6208568554382	0\\
27.6204956281965	0\\
27.6201343859316	0\\
27.6197731286421	0\\
27.6194118563269	0\\
27.6190505689846	0\\
27.618689266614	0\\
27.618327949214	0\\
27.6179666167831	0\\
27.6176052693202	0\\
27.617243906824	0\\
27.6168825292933	0\\
27.6165211367268	0\\
27.6161597291233	0\\
27.6157983064815	0\\
27.6154368688001	0\\
27.615075416078	0\\
27.6147139483138	0\\
27.6143524655064	0\\
27.6139909676543	0\\
27.6136294547565	0\\
27.6132679268117	0\\
27.6129063838185	0\\
27.6125448257758	0\\
27.6121832526822	0\\
27.6118216645366	0\\
27.6114600613377	0\\
27.6110984430843	0\\
27.610736809775	0\\
27.6103751614086	0\\
27.6100134979839	0\\
27.6096518194996	0\\
27.6092901259545	0\\
27.6089284173473	0\\
27.6085666936767	0\\
27.6082049549416	0\\
27.6078432011405	0\\
27.6074814322724	0\\
27.6071196483359	0\\
27.6067578493298	0\\
27.6063960352528	0\\
27.6060342061037	0\\
27.6056723618812	0\\
27.605310502584	0\\
27.604948628211	0\\
27.6045867387607	0\\
27.6042248342321	0\\
27.6038629146238	0\\
27.6035009799346	0\\
27.6031390301632	0\\
27.6027770653083	0\\
27.6024150853688	0\\
27.6020530903433	0\\
27.6016910802306	0\\
27.6013290550294	0\\
27.6009670147385	0\\
27.6006049593566	0\\
27.6002428888825	0\\
27.5998808033148	0\\
27.5995187026524	0\\
27.599156586894	0\\
27.5987944560383	0\\
27.5984323100841	0\\
27.5980701490301	0\\
27.597707972875	0\\
27.5973457816176	0\\
27.5969835752566	0\\
27.5966213537908	0\\
27.5962591172189	0\\
27.5958968655396	0\\
27.5955345987518	0\\
27.595172316854	0\\
27.5948100198451	0\\
27.5944477077239	0\\
27.5940853804889	0\\
27.5937230381391	0\\
27.5933606806731	0\\
27.5929983080896	0\\
27.5926359203875	0\\
27.5922735175653	0\\
27.591911099622	0\\
27.5915486665561	0\\
27.5911862183666	0\\
27.590823755052	0\\
27.5904612766111	0\\
27.5900987830427	0\\
27.5897362743455	0\\
27.5893737505182	0\\
27.5890112115596	0\\
27.5886486574684	0\\
27.5882860882434	0\\
27.5879235038833	0\\
27.5875609043868	0\\
27.5871982897526	0\\
27.5868356599795	0\\
27.5864730150663	0\\
27.5861103550116	0\\
27.5857476798143	0\\
27.5853849894729	0\\
27.5850222839864	0\\
27.5846595633533	0\\
27.5842968275725	0\\
27.5839340766427	0\\
27.5835713105626	0\\
27.5832085293309	0\\
27.5828457329464	0\\
27.5824829214078	0\\
27.5821200947139	0\\
27.5817572528634	0\\
27.5813943958549	0\\
27.5810315236873	0\\
27.5806686363593	0\\
27.5803057338697	0\\
27.579942816217	0\\
27.5795798834002	0\\
27.5792169354178	0\\
27.5788539722687	0\\
27.5784909939516	0\\
27.5781280004652	0\\
27.5777649918083	0\\
27.5774019679795	0\\
27.5770389289776	0\\
27.5766758748014	0\\
27.5763128054495	0\\
27.5759497209208	0\\
27.5755866212138	0\\
27.5752235063275	0\\
27.5748603762604	0\\
27.5744972310113	0\\
27.5741340705791	0\\
27.5737708949623	0\\
27.5734077041597	0\\
27.57304449817	0\\
27.572681276992	0\\
27.5723180406245	0\\
27.5719547890661	0\\
27.5715915223155	0\\
27.5712282403715	0\\
27.5708649432329	0\\
27.5705016308983	0\\
27.5701383033665	0\\
27.5697749606362	0\\
27.5694116027061	0\\
27.569048229575	0\\
27.5686848412416	0\\
27.5683214377046	0\\
27.5679580189628	0\\
27.5675945850148	0\\
27.5672311358594	0\\
27.5668676714954	0\\
27.5665041919214	0\\
27.5661406971363	0\\
27.5657771871386	0\\
27.5654136619272	0\\
27.5650501215007	0\\
27.5646865658579	0\\
27.5643229949976	0\\
27.5639594089183	0\\
27.563595807619	0\\
27.5632321910983	0\\
27.5628685593548	0\\
27.5625049123875	0\\
27.5621412501948	0\\
27.5617775727757	0\\
27.5614138801288	0\\
27.5610501722529	0\\
27.5606864491466	0\\
27.5603227108088	0\\
27.559958957238	0\\
27.5595951884331	0\\
27.5592314043927	0\\
27.5588676051157	0\\
27.5585037906006	0\\
27.5581399608463	0\\
27.5577761158515	0\\
27.5574122556148	0\\
27.5570483801351	0\\
27.556684489411	0\\
27.5563205834412	0\\
27.5559566622245	0\\
27.5555927257596	0\\
27.5552287740453	0\\
27.5548648070801	0\\
27.554500824863	0\\
27.5541368273925	0\\
27.5537728146674	0\\
27.5534087866865	0\\
27.5530447434484	0\\
27.5526806849518	0\\
27.5523166111956	0\\
27.5519525221784	0\\
27.5515884178989	0\\
27.5512242983559	0\\
27.550860163548	0\\
27.550496013474	0\\
27.5501318481327	0\\
27.5497676675227	0\\
27.5494034716427	0\\
27.5490392604915	0\\
27.5486750340678	0\\
27.5483107923703	0\\
27.5479465353977	0\\
27.5475822631488	0\\
27.5472179756222	0\\
27.5468536728168	0\\
27.5464893547311	0\\
27.546125021364	0\\
27.5457606727141	0\\
27.5453963087801	0\\
27.5450319295609	0\\
27.544667535055	0\\
27.5443031252612	0\\
27.5439387001783	0\\
27.5435742598049	0\\
27.5432098041397	0\\
27.5428453331816	0\\
27.5424808469291	0\\
27.5421163453811	0\\
27.5417518285361	0\\
27.5413872963931	0\\
27.5410227489505	0\\
27.5406581862073	0\\
27.540293608162	0\\
27.5399290148135	0\\
27.5395644061603	0\\
27.5391997822013	0\\
27.5388351429352	0\\
27.5384704883606	0\\
27.5381058184763	0\\
27.537741133281	0\\
27.5373764327734	0\\
27.5370117169522	0\\
27.5366469858161	0\\
27.5362822393639	0\\
27.5359174775943	0\\
27.535552700506	0\\
27.5351879080976	0\\
27.5348231003679	0\\
27.5344582773157	0\\
27.5340934389396	0\\
27.5337285852383	0\\
27.5333637162106	0\\
27.5329988318551	0\\
27.5326339321707	0\\
27.5322690171559	0\\
27.5319040868095	0\\
27.5315391411302	0\\
27.5311741801168	0\\
27.5308092037679	0\\
27.5304442120822	0\\
27.5300792050585	0\\
27.5297141826955	0\\
27.5293491449919	0\\
27.5289840919463	0\\
27.5286190235576	0\\
27.5282539398243	0\\
27.5278888407453	0\\
27.5275237263192	0\\
27.5271585965448	0\\
27.5267934514207	0\\
27.5264282909457	0\\
27.5260631151184	0\\
27.5256979239376	0\\
27.525332717402	0\\
27.5249674955103	0\\
27.5246022582612	0\\
27.5242370056535	0\\
27.5238717376857	0\\
27.5235064543567	0\\
27.5231411556651	0\\
27.5227758416096	0\\
27.522410512189	0\\
27.522045167402	0\\
27.5216798072472	0\\
27.5213144317234	0\\
27.5209490408293	0\\
27.5205836345636	0\\
27.520218212925	0\\
27.5198527759122	0\\
27.5194873235238	0\\
27.5191218557587	0\\
27.5187563726155	0\\
27.518390874093	0\\
27.5180253601898	0\\
27.5176598309046	0\\
27.5172942862361	0\\
27.5169287261831	0\\
27.5165631507443	0\\
27.5161975599183	0\\
27.5158319537038	0\\
27.5154663320997	0\\
27.5151006951045	0\\
27.514735042717	0\\
27.5143693749358	0\\
27.5140036917598	0\\
27.5136379931875	0\\
27.5132722792177	0\\
27.5129065498491	0\\
27.5125408050804	0\\
27.5121750449103	0\\
27.5118092693375	0\\
27.5114434783607	0\\
27.5110776719787	0\\
27.51071185019	0\\
27.5103460129934	0\\
27.5099801603877	0\\
27.5096142923714	0\\
27.5092484089434	0\\
27.5088825101023	0\\
27.5085165958469	0\\
27.5081506661757	0\\
27.5077847210876	0\\
27.5074187605812	0\\
27.5070527846552	0\\
27.5066867933083	0\\
27.5063207865393	0\\
27.5059547643468	0\\
27.5055887267295	0\\
27.5052226736861	0\\
27.5048566052154	0\\
27.504490521316	0\\
27.5041244219866	0\\
27.5037583072259	0\\
27.5033921770327	0\\
27.5030260314056	0\\
27.5026598703433	0\\
27.5022936938445	0\\
27.5019275019079	0\\
27.5015612945323	0\\
27.5011950717162	0\\
27.5008288334585	0\\
27.5004625797578	0\\
27.5000963106128	0\\
27.4997300260222	0\\
27.4993637259847	0\\
27.498997410499	0\\
27.4986310795638	0\\
27.4982647331777	0\\
27.4978983713396	0\\
27.4975319940481	0\\
27.4971656013018	0\\
27.4967991930995	0\\
27.4964327694399	0\\
27.4960663303216	0\\
27.4956998757435	0\\
27.495333405704	0\\
27.4949669202021	0\\
27.4946004192363	0\\
27.4942339028053	0\\
27.4938673709079	0\\
27.4935008235427	0\\
27.4931342607084	0\\
27.4927676824038	0\\
27.4924010886274	0\\
27.4920344793781	0\\
27.4916678546545	0\\
27.4913012144553	0\\
27.4909345587792	0\\
27.4905678876249	0\\
27.490201200991	0\\
27.4898344988763	0\\
27.4894677812795	0\\
27.4891010481993	0\\
27.4887342996342	0\\
27.4883675355832	0\\
27.4880007560447	0\\
27.4876339610176	0\\
27.4872671505006	0\\
27.4869003244922	0\\
27.4865334829912	0\\
27.4861666259963	0\\
27.4857997535062	0\\
27.4854328655196	0\\
27.4850659620351	0\\
27.4846990430515	0\\
27.4843321085675	0\\
27.4839651585817	0\\
27.4835981930928	0\\
27.4832312120995	0\\
27.4828642156005	0\\
27.4824972035945	0\\
27.4821301760803	0\\
27.4817631330563	0\\
27.4813960745215	0\\
27.4810290004744	0\\
27.4806619109138	0\\
27.4802948058382	0\\
27.4799276852465	0\\
27.4795605491373	0\\
27.4791933975093	0\\
27.4788262303612	0\\
27.4784590476917	0\\
27.4780918494994	0\\
27.4777246357831	0\\
27.4773574065414	0\\
27.476990161773	0\\
27.4766229014766	0\\
27.4762556256509	0\\
27.4758883342946	0\\
27.4755210274064	0\\
27.4751537049849	0\\
27.4747863670289	0\\
27.474419013537	0\\
27.4740516445079	0\\
27.4736842599403	0\\
27.4733168598329	0\\
27.4729494441844	0\\
27.4725820129934	0\\
27.4722145662586	0\\
27.4718471039788	0\\
27.4714796261526	0\\
27.4711121327787	0\\
27.4707446238557	0\\
27.4703770993824	0\\
27.4700095593575	0\\
27.4696420037796	0\\
27.4692744326474	0\\
27.4689068459596	0\\
27.4685392437149	0\\
27.4681716259119	0\\
27.4678039925494	0\\
27.467436343626	0\\
27.4670686791404	0\\
27.4667009990914	0\\
27.4663333034775	0\\
27.4659655922974	0\\
27.4655978655499	0\\
27.4652301232336	0\\
27.4648623653473	0\\
27.4644945918895	0\\
27.4641268028589	0\\
27.4637589982544	0\\
27.4633911780744	0\\
27.4630233423178	0\\
27.4626554909831	0\\
27.4622876240692	0\\
27.4619197415745	0\\
27.4615518434979	0\\
27.4611839298381	0\\
27.4608160005936	0\\
27.4604480557631	0\\
27.4600800953455	0\\
27.4597121193392	0\\
27.4593441277431	0\\
27.4589761205558	0\\
27.4586080977759	0\\
27.4582400594021	0\\
27.4578720054332	0\\
27.4575039358678	0\\
27.4571358507046	0\\
27.4567677499422	0\\
27.4563996335794	0\\
27.4560315016147	0\\
27.455663354047	0\\
27.4552951908748	0\\
27.4549270120969	0\\
27.4545588177119	0\\
27.4541906077185	0\\
27.4538223821154	0\\
27.4534541409012	0\\
27.4530858840747	0\\
27.4527176116344	0\\
27.4523493235791	0\\
27.4519810199075	0\\
27.4516127006183	0\\
27.45124436571	0\\
27.4508760151814	0\\
27.4505076490312	0\\
27.450139267258	0\\
27.4497708698605	0\\
27.4494024568374	0\\
27.4490340281873	0\\
27.448665583909	0\\
27.4482971240011	0\\
27.4479286484622	0\\
27.4475601572911	0\\
27.4471916504865	0\\
27.4468231280469	0\\
27.4464545899711	0\\
27.4460860362578	0\\
27.4457174669055	0\\
27.4453488819131	0\\
27.4449802812791	0\\
27.4446116650023	0\\
27.4442430330812	0\\
27.4438743855147	0\\
27.4435057223013	0\\
27.4431370434397	0\\
27.4427683489286	0\\
27.4423996387667	0\\
27.4420309129526	0\\
27.441662171485	0\\
27.4412934143627	0\\
27.4409246415841	0\\
27.4405558531481	0\\
27.4401870490533	0\\
27.4398182292983	0\\
27.4394493938819	0\\
27.4390805428026	0\\
27.4387116760593	0\\
27.4383427936504	0\\
27.4379738955748	0\\
27.4376049818311	0\\
27.4372360524179	0\\
27.4368671073339	0\\
27.4364981465778	0\\
27.4361291701483	0\\
27.435760178044	0\\
27.4353911702636	0\\
27.4350221468057	0\\
27.4346531076691	0\\
27.4342840528524	0\\
27.4339149823542	0\\
27.4335458961732	0\\
27.4331767943082	0\\
27.4328076767577	0\\
27.4324385435204	0\\
27.4320693945951	0\\
27.4317002299803	0\\
27.4313310496747	0\\
27.430961853677	0\\
27.4305926419859	0\\
27.4302234146	0\\
27.429854171518	0\\
27.4294849127386	0\\
27.4291156382603	0\\
27.428746348082	0\\
27.4283770422022	0\\
27.4280077206197	0\\
27.427638383333	0\\
27.4272690303409	0\\
27.426899661642	0\\
27.4265302772349	0\\
27.4261608771184	0\\
27.4257914612911	0\\
27.4254220297517	0\\
27.4250525824988	0\\
27.4246831195311	0\\
27.4243136408472	0\\
27.4239441464459	0\\
27.4235746363257	0\\
27.4232051104854	0\\
27.4228355689236	0\\
27.422466011639	0\\
27.4220964386302	0\\
27.4217268498959	0\\
27.4213572454347	0\\
27.4209876252454	0\\
27.4206179893265	0\\
27.4202483376768	0\\
27.4198786702949	0\\
27.4195089871794	0\\
27.4191392883291	0\\
27.4187695737425	0\\
27.4183998434184	0\\
27.4180300973554	0\\
27.4176603355521	0\\
27.4172905580073	0\\
27.4169207647195	0\\
27.4165509556875	0\\
27.4161811309099	0\\
27.4158112903854	0\\
27.4154414341125	0\\
27.4150715620901	0\\
27.4147016743166	0\\
27.4143317707909	0\\
27.4139618515115	0\\
27.4135919164772	0\\
27.4132219656865	0\\
27.4128519991381	0\\
27.4124820168307	0\\
27.412112018763	0\\
27.4117420049335	0\\
27.411371975341	0\\
27.4110019299841	0\\
27.4106318688615	0\\
27.4102617919719	0\\
27.4098916993137	0\\
27.4095215908859	0\\
27.4091514666869	0\\
27.4087813267155	0\\
27.4084111709703	0\\
27.4080409994499	0\\
27.4076708121531	0\\
27.4073006090784	0\\
27.4069303902246	0\\
27.4065601555902	0\\
27.406189905174	0\\
27.4058196389746	0\\
27.4054493569906	0\\
27.4050790592207	0\\
27.4047087456635	0\\
27.4043384163178	0\\
27.4039680711821	0\\
27.4035977102552	0\\
27.4032273335356	0\\
27.402856941022	0\\
27.4024865327131	0\\
27.4021161086076	0\\
27.401745668704	0\\
27.401375213001	0\\
27.4010047414974	0\\
27.4006342541916	0\\
27.4002637510825	0\\
27.3998932321686	0\\
27.3995226974485	0\\
27.3991521469211	0\\
27.3987815805848	0\\
27.3984109984383	0\\
27.3980404004804	0\\
27.3976697867096	0\\
27.3972991571246	0\\
27.396928511724	0\\
27.3965578505066	0\\
27.3961871734708	0\\
27.3958164806155	0\\
27.3954457719393	0\\
27.3950750474407	0\\
27.3947043071184	0\\
27.3943335509712	0\\
27.3939627789976	0\\
27.3935919911963	0\\
27.393221187566	0\\
27.3928503681052	0\\
27.3924795328127	0\\
27.392108681687	0\\
27.3917378147269	0\\
27.391366931931	0\\
27.3909960332978	0\\
27.3906251188262	0\\
27.3902541885147	0\\
27.389883242362	0\\
27.3895122803667	0\\
27.3891413025274	0\\
27.3887703088429	0\\
27.3883992993117	0\\
27.3880282739325	0\\
27.387657232704	0\\
27.3872861756247	0\\
27.3869151026934	0\\
27.3865440139087	0\\
27.3861729092692	0\\
27.3858017887736	0\\
27.3854306524205	0\\
27.3850595002086	0\\
27.3846883321365	0\\
27.3843171482029	0\\
27.3839459484063	0\\
27.3835747327455	0\\
27.3832035012191	0\\
27.3828322538257	0\\
27.382460990564	0\\
27.3820897114326	0\\
27.3817184164302	0\\
27.3813471055553	0\\
27.3809757788067	0\\
27.3806044361831	0\\
27.3802330776829	0\\
27.3798617033049	0\\
27.3794903130478	0\\
27.3791189069101	0\\
27.3787474848905	0\\
27.3783760469876	0\\
27.3780045932001	0\\
27.3776331235267	0\\
27.3772616379659	0\\
27.3768901365164	0\\
27.3765186191769	0\\
27.376147085946	0\\
27.3757755368223	0\\
27.3754039718044	0\\
27.3750323908911	0\\
27.3746607940809	0\\
27.3742891813725	0\\
27.3739175527646	0\\
27.3735459082557	0\\
27.3731742478445	0\\
27.3728025715297	0\\
27.3724308793099	0\\
27.3720591711836	0\\
27.3716874471497	0\\
27.3713157072066	0\\
27.3709439513531	0\\
27.3705721795878	0\\
27.3702003919093	0\\
27.3698285883162	0\\
27.3694567688072	0\\
27.369084933381	0\\
27.3687130820361	0\\
27.3683412147712	0\\
27.367969331585	0\\
27.367597432476	0\\
27.367225517443	0\\
27.3668535864845	0\\
27.3664816395992	0\\
27.3661096767857	0\\
27.3657376980426	0\\
27.3653657033687	0\\
27.3649936927624	0\\
27.3646216662226	0\\
27.3642496237477	0\\
27.3638775653365	0\\
27.3635054909875	0\\
27.3631334006995	0\\
27.3627612944709	0\\
27.3623891723006	0\\
27.362017034187	0\\
27.3616448801289	0\\
27.3612727101249	0\\
27.3609005241735	0\\
27.3605283222735	0\\
27.3601561044235	0\\
27.3597838706221	0\\
27.3594116208679	0\\
27.3590393551596	0\\
27.3586670734957	0\\
27.3582947758751	0\\
27.3579224622961	0\\
27.3575501327576	0\\
27.3571777872581	0\\
27.3568054257963	0\\
27.3564330483708	0\\
27.3560606549801	0\\
27.3556882456231	0\\
27.3553158202982	0\\
27.3549433790041	0\\
27.3545709217395	0\\
27.354198448503	0\\
27.3538259592932	0\\
27.3534534541087	0\\
27.3530809329481	0\\
27.3527083958102	0\\
27.3523358426935	0\\
27.3519632735967	0\\
27.3515906885183	0\\
27.351218087457	0\\
27.3508454704115	0\\
27.3504728373804	0\\
27.3501001883622	0\\
27.3497275233557	0\\
27.3493548423595	0\\
27.3489821453721	0\\
27.3486094323922	0\\
27.3482367034185	0\\
27.3478639584495	0\\
27.3474911974839	0\\
27.3471184205204	0\\
27.3467456275575	0\\
27.3463728185938	0\\
27.3459999936281	0\\
27.3456271526589	0\\
27.3452542956848	0\\
27.3448814227045	0\\
27.3445085337166	0\\
27.3441356287198	0\\
27.3437627077125	0\\
27.3433897706936	0\\
27.3430168176616	0\\
27.3426438486151	0\\
27.3422708635527	0\\
27.3418978624732	0\\
27.341524845375	0\\
27.3411518122569	0\\
27.3407787631174	0\\
27.3404056979551	0\\
27.3400326167688	0\\
27.339659519557	0\\
27.3392864063184	0\\
27.3389132770515	0\\
27.338540131755	0\\
27.3381669704275	0\\
27.3377937930676	0\\
27.337420599674	0\\
27.3370473902453	0\\
27.3366741647801	0\\
27.336300923277	0\\
27.3359276657346	0\\
27.3355543921516	0\\
27.3351811025266	0\\
27.3348077968582	0\\
27.334434475145	0\\
27.3340611373857	0\\
27.3336877835789	0\\
27.3333144137231	0\\
27.3329410278171	0\\
27.3325676258594	0\\
27.3321942078486	0\\
27.3318207737834	0\\
27.3314473236624	0\\
27.3310738574842	0\\
27.3307003752474	0\\
27.3303268769507	0\\
27.3299533625926	0\\
27.3295798321719	0\\
27.329206285687	0\\
27.3288327231367	0\\
27.3284591445195	0\\
27.328085549834	0\\
27.3277119390789	0\\
27.3273383122528	0\\
27.3269646693544	0\\
27.3265910103821	0\\
27.3262173353347	0\\
27.3258436442108	0\\
27.3254699370089	0\\
27.3250962137277	0\\
27.3247224743659	0\\
27.3243487189219	0\\
27.3239749473945	0\\
27.3236011597823	0\\
27.3232273560838	0\\
27.3228535362977	0\\
27.3224797004227	0\\
27.3221058484572	0\\
27.3217319804	0\\
27.3213580962496	0\\
27.3209841960047	0\\
27.3206102796639	0\\
27.3202363472257	0\\
27.3198623986889	0\\
27.319488434052	0\\
27.3191144533136	0\\
27.3187404564723	0\\
27.3183664435269	0\\
27.3179924144758	0\\
27.3176183693176	0\\
27.3172443080511	0\\
27.3168702306748	0\\
27.3164961371873	0\\
27.3161220275872	0\\
27.3157479018732	0\\
27.3153737600439	0\\
27.3149996020978	0\\
27.3146254280336	0\\
27.3142512378499	0\\
27.3138770315452	0\\
27.3135028091184	0\\
27.3131285705678	0\\
27.3127543158921	0\\
27.3123800450901	0\\
27.3120057581601	0\\
27.311631455101	0\\
27.3112571359112	0\\
27.3108828005894	0\\
27.3105084491342	0\\
27.3101340815442	0\\
27.309759697818	0\\
27.3093852979542	0\\
27.3090108819515	0\\
27.3086364498084	0\\
27.3082620015236	0\\
27.3078875370956	0\\
27.307513056523	0\\
27.3071385598046	0\\
27.3067640469388	0\\
27.3063895179243	0\\
27.3060149727597	0\\
27.3056404114437	0\\
27.3052658339747	0\\
27.3048912403514	0\\
27.3045166305725	0\\
27.3041420046365	0\\
27.303767362542	0\\
27.3033927042876	0\\
27.3030180298721	0\\
27.3026433392938	0\\
27.3022686325515	0\\
27.3018939096438	0\\
27.3015191705693	0\\
27.3011444153265	0\\
27.3007696439141	0\\
27.3003948563307	0\\
27.3000200525749	0\\
27.2996452326452	0\\
27.2992703965404	0\\
27.298895544259	0\\
27.2985206757995	0\\
27.2981457911607	0\\
27.2977708903411	0\\
27.2973959733392	0\\
27.2970210401539	0\\
27.2966460907835	0\\
27.2962711252267	0\\
27.2958961434822	0\\
27.2955211455485	0\\
27.2951461314242	0\\
27.294771101108	0\\
27.2943960545984	0\\
27.294020991894	0\\
27.2936459129934	0\\
27.2932708178953	0\\
27.2928957065982	0\\
27.2925205791008	0\\
27.2921454354016	0\\
27.2917702754992	0\\
27.2913950993923	0\\
27.2910199070794	0\\
27.2906446985592	0\\
27.2902694738302	0\\
27.289894232891	0\\
27.2895189757403	0\\
27.2891437023766	0\\
27.2887684127985	0\\
27.2883931070047	0\\
27.2880177849937	0\\
27.2876424467641	0\\
27.2872670923146	0\\
27.2868917216437	0\\
27.28651633475	0\\
27.2861409316322	0\\
27.2857655122887	0\\
27.2853900767183	0\\
27.2850146249195	0\\
27.2846391568909	0\\
27.2842636726311	0\\
27.2838881721388	0\\
27.2835126554124	0\\
27.2831371224506	0\\
27.282761573252	0\\
27.2823860078152	0\\
27.2820104261388	0\\
27.2816348282214	0\\
27.2812592140616	0\\
27.2808835836579	0\\
27.280507937009	0\\
27.2801322741134	0\\
27.2797565949698	0\\
27.2793808995768	0\\
27.2790051879329	0\\
27.2786294600367	0\\
27.2782537158869	0\\
27.277877955482	0\\
27.2775021788206	0\\
27.2771263859014	0\\
27.2767505767228	0\\
27.2763747512836	0\\
27.2759989095822	0\\
27.2756230516173	0\\
27.2752471773875	0\\
27.2748712868914	0\\
27.2744953801275	0\\
27.2741194570945	0\\
27.2737435177909	0\\
27.2733675622154	0\\
27.2729915903665	0\\
27.2726156022428	0\\
27.2722395978429	0\\
27.2718635771654	0\\
27.2714875402089	0\\
27.271111486972	0\\
27.2707354174533	0\\
27.2703593316513	0\\
27.2699832295647	0\\
27.269607111192	0\\
27.2692309765319	0\\
27.2688548255828	0\\
27.2684786583435	0\\
27.2681024748125	0\\
27.2677262749884	0\\
27.2673500588697	0\\
27.2669738264551	0\\
27.2665975777432	0\\
27.2662213127325	0\\
27.2658450314216	0\\
27.2654687338092	0\\
27.2650924198937	0\\
27.2647160896739	0\\
27.2643397431482	0\\
27.2639633803153	0\\
27.2635870011737	0\\
27.2632106057221	0\\
27.262834193959	0\\
27.262457765883	0\\
27.2620813214927	0\\
27.2617048607867	0\\
27.2613283837636	0\\
27.2609518904219	0\\
27.2605753807602	0\\
27.2601988547772	0\\
27.2598223124714	0\\
27.2594457538414	0\\
27.2590691788857	0\\
27.258692587603	0\\
27.2583159799919	0\\
27.2579393560509	0\\
27.2575627157786	0\\
27.2571860591736	0\\
27.2568093862345	0\\
27.2564326969598	0\\
27.2560559913482	0\\
27.2556792693983	0\\
27.2553025311085	0\\
27.2549257764776	0\\
27.254549005504	0\\
27.2541722181864	0\\
27.2537954145233	0\\
27.2534185945133	0\\
27.2530417581551	0\\
27.2526649054471	0\\
27.252288036388	0\\
27.2519111509764	0\\
27.2515342492108	0\\
27.2511573310898	0\\
27.250780396612	0\\
27.2504034457759	0\\
27.2500264785802	0\\
27.2496494950235	0\\
27.2492724951042	0\\
27.2488954788211	0\\
27.2485184461726	0\\
27.2481413971574	0\\
27.247764331774	0\\
27.247387250021	0\\
27.247010151897	0\\
27.2466330374005	0\\
27.2462559065302	0\\
27.2458787592846	0\\
27.2455015956623	0\\
27.2451244156619	0\\
27.244747219282	0\\
27.244370006521	0\\
27.2439927773777	0\\
27.2436155318506	0\\
27.2432382699382	0\\
27.2428609916392	0\\
27.2424836969521	0\\
27.2421063858755	0\\
27.2417290584079	0\\
27.241351714548	0\\
27.2409743542943	0\\
27.2405969776454	0\\
27.2402195845999	0\\
27.2398421751563	0\\
27.2394647493132	0\\
27.2390873070692	0\\
27.2387098484229	0\\
27.2383323733729	0\\
27.2379548819176	0\\
27.2375773740558	0\\
27.2371998497859	0\\
27.2368223091065	0\\
27.2364447520163	0\\
27.2360671785137	0\\
27.2356895885975	0\\
27.235311982266	0\\
27.2349343595179	0\\
27.2345567203519	0\\
27.2341790647663	0\\
27.2338013927599	0\\
27.2334237043312	0\\
27.2330459994787	0\\
27.2326682782011	0\\
27.2322905404969	0\\
27.2319127863647	0\\
27.231535015803	0\\
27.2311572288104	0\\
27.2307794253856	0\\
27.230401605527	0\\
27.2300237692332	0\\
27.2296459165028	0\\
27.2292680473344	0\\
27.2288901617266	0\\
27.2285122596778	0\\
27.2281343411868	0\\
27.2277564062519	0\\
27.2273784548719	0\\
27.2270004870453	0\\
27.2266225027707	0\\
27.2262445020465	0\\
27.2258664848715	0\\
27.2254884512441	0\\
27.225110401163	0\\
27.2247323346266	0\\
27.2243542516336	0\\
27.2239761521826	0\\
27.223598036272	0\\
27.2232199039005	0\\
27.2228417550667	0\\
27.222463589769	0\\
27.2220854080061	0\\
27.2217072097765	0\\
27.2213289950788	0\\
27.2209507639116	0\\
27.2205725162734	0\\
27.2201942521629	0\\
27.2198159715784	0\\
27.2194376745187	0\\
27.2190593609823	0\\
27.2186810309677	0\\
27.2183026844736	0\\
27.2179243214984	0\\
27.2175459420408	0\\
27.2171675460993	0\\
27.2167891336724	0\\
27.2164107047588	0\\
27.216032259357	0\\
27.2156537974656	0\\
27.2152753190831	0\\
27.2148968242081	0\\
27.2145183128391	0\\
27.2141397849747	0\\
27.2137612406136	0\\
27.2133826797541	0\\
27.213004102395	0\\
27.2126255085347	0\\
27.2122468981718	0\\
27.211868271305	0\\
27.2114896279327	0\\
27.2111109680534	0\\
27.2107322916659	0\\
27.2103535987686	0\\
27.20997488936	0\\
27.2095961634388	0\\
27.2092174210036	0\\
27.2088386620528	0\\
27.208459886585	0\\
27.2080810945988	0\\
27.2077022860927	0\\
27.2073234610654	0\\
27.2069446195153	0\\
27.206565761441	0\\
27.2061868868412	0\\
27.2058079957142	0\\
27.2054290880588	0\\
27.2050501638734	0\\
27.2046712231566	0\\
27.204292265907	0\\
27.2039132921232	0\\
27.2035343018036	0\\
27.2031552949468	0\\
27.2027762715515	0\\
27.2023972316161	0\\
27.2020181751392	0\\
27.2016391021194	0\\
27.2012600125552	0\\
27.2008809064452	0\\
27.2005017837879	0\\
27.200122644582	0\\
27.1997434888258	0\\
27.1993643165181	0\\
27.1989851276573	0\\
27.198605922242	0\\
27.1982267002708	0\\
27.1978474617423	0\\
27.1974682066548	0\\
27.1970889350072	0\\
27.1967096467978	0\\
27.1963303420252	0\\
27.195951020688	0\\
27.1955716827848	0\\
27.195192328314	0\\
27.1948129572743	0\\
27.1944335696642	0\\
27.1940541654823	0\\
27.193674744727	0\\
27.193295307397	0\\
27.1929158534909	0\\
27.192536383007	0\\
27.1921568959441	0\\
27.1917773923007	0\\
27.1913978720752	0\\
27.1910183352664	0\\
27.1906387818726	0\\
27.1902592118925	0\\
27.1898796253246	0\\
27.1895000221675	0\\
27.1891204024197	0\\
27.1887407660798	0\\
27.1883611131463	0\\
27.1879814436177	0\\
27.1876017574927	0\\
27.1872220547697	0\\
27.1868423354474	0\\
27.1864625995241	0\\
27.1860828469986	0\\
27.1857030778694	0\\
27.185323292135	0\\
27.1849434897939	0\\
27.1845636708447	0\\
27.184183835286	0\\
27.1838039831162	0\\
27.183424114334	0\\
27.1830442289379	0\\
27.1826643269264	0\\
27.1822844082981	0\\
27.1819044730516	0\\
27.1815245211853	0\\
27.1811445526978	0\\
27.1807645675877	0\\
27.1803845658535	0\\
27.1800045474937	0\\
27.179624512507	0\\
27.1792444608918	0\\
27.1788643926467	0\\
27.1784843077702	0\\
27.1781042062609	0\\
27.1777240881174	0\\
27.1773439533381	0\\
27.1769638019216	0\\
27.1765836338665	0\\
27.1762034491713	0\\
27.1758232478345	0\\
27.1754430298547	0\\
27.1750627952304	0\\
27.1746825439602	0\\
27.1743022760426	0\\
27.1739219914762	0\\
27.1735416902595	0\\
27.173161372391	0\\
27.1727810378693	0\\
27.1724006866929	0\\
27.1720203188604	0\\
27.1716399343703	0\\
27.1712595332211	0\\
27.1708791154114	0\\
27.1704986809398	0\\
27.1701182298047	0\\
27.1697377620047	0\\
27.1693572775384	0\\
27.1689767764043	0\\
27.1685962586009	0\\
27.1682157241267	0\\
27.1678351729804	0\\
27.1674546051604	0\\
27.1670740206653	0\\
27.1666934194936	0\\
27.1663128016439	0\\
27.1659321671147	0\\
27.1655515159045	0\\
27.1651708480119	0\\
27.1647901634354	0\\
27.1644094621736	0\\
27.1640287442249	0\\
27.163648009588	0\\
27.1632672582614	0\\
27.1628864902435	0\\
27.162505705533	0\\
27.1621249041283	0\\
27.1617440860281	0\\
27.1613632512308	0\\
27.1609823997349	0\\
27.1606015315391	0\\
27.1602206466419	0\\
27.1598397450417	0\\
27.1594588267372	0\\
27.1590778917268	0\\
27.1586969400091	0\\
27.1583159715826	0\\
27.1579349864459	0\\
27.1575539845975	0\\
27.1571729660359	0\\
27.1567919307596	0\\
27.1564108787673	0\\
27.1560298100574	0\\
27.1556487246284	0\\
27.1552676224789	0\\
27.1548865036075	0\\
27.1545053680126	0\\
27.1541242156928	0\\
27.1537430466467	0\\
27.1533618608727	0\\
27.1529806583694	0\\
27.1525994391352	0\\
27.1522182031689	0\\
27.1518369504688	0\\
27.1514556810336	0\\
27.1510743948616	0\\
27.1506930919516	0\\
27.1503117723019	0\\
27.1499304359112	0\\
27.149549082778	0\\
27.1491677129007	0\\
27.148786326278	0\\
27.1484049229083	0\\
27.1480235027902	0\\
27.1476420659222	0\\
27.1472606123029	0\\
27.1468791419307	0\\
27.1464976548042	0\\
27.1461161509219	0\\
27.1457346302824	0\\
27.1453530928842	0\\
27.1449715387257	0\\
27.1445899678057	0\\
27.1442083801224	0\\
27.1438267756746	0\\
27.1434451544607	0\\
27.1430635164792	0\\
27.1426818617287	0\\
27.1423001902077	0\\
27.1419185019147	0\\
27.1415367968482	0\\
27.1411550750069	0\\
27.1407733363891	0\\
27.1403915809934	0\\
27.1400098088184	0\\
27.1396280198626	0\\
27.1392462141245	0\\
27.1388643916026	0\\
27.1384825522954	0\\
27.1381006962016	0\\
27.1377188233195	0\\
27.1373369336477	0\\
27.1369550271848	0\\
27.1365731039292	0\\
27.1361911638796	0\\
27.1358092070343	0\\
27.135427233392	0\\
27.1350452429511	0\\
27.1346632357103	0\\
27.1342812116679	0\\
27.1338991708225	0\\
27.1335171131727	0\\
27.133135038717	0\\
27.1327529474539	0\\
27.1323708393818	0\\
27.1319887144994	0\\
27.1316065728052	0\\
27.1312244142976	0\\
27.1308422389752	0\\
27.1304600468365	0\\
27.1300778378801	0\\
27.1296956121044	0\\
27.129313369508	0\\
27.1289311100894	0\\
27.1285488338471	0\\
27.1281665407796	0\\
27.1277842308855	0\\
27.1274019041633	0\\
27.1270195606114	0\\
27.1266372002284	0\\
27.1262548230129	0\\
27.1258724289633	0\\
27.1254900180782	0\\
27.125107590356	0\\
27.1247251457954	0\\
27.1243426843947	0\\
27.1239602061526	0\\
27.1235777110675	0\\
27.123195199138	0\\
27.1228126703626	0\\
27.1224301247397	0\\
27.122047562268	0\\
27.1216649829459	0\\
27.1212823867719	0\\
27.1208997737446	0\\
27.1205171438625	0\\
27.120134497124	0\\
27.1197518335278	0\\
27.1193691530722	0\\
27.1189864557559	0\\
27.1186037415774	0\\
27.1182210105351	0\\
27.1178382626275	0\\
27.1174554978533	0\\
27.1170727162108	0\\
27.1166899176987	0\\
27.1163071023154	0\\
27.1159242700594	0\\
27.1155414209293	0\\
27.1151585549236	0\\
27.1147756720407	0\\
27.1143927722792	0\\
27.1140098556376	0\\
27.1136269221145	0\\
27.1132439717082	0\\
27.1128610044174	0\\
27.1124780202405	0\\
27.1120950191761	0\\
27.1117120012226	0\\
27.1113289663787	0\\
27.1109459146427	0\\
27.1105628460132	0\\
27.1101797604887	0\\
27.1097966580677	0\\
27.1094135387488	0\\
27.1090304025303	0\\
27.108647249411	0\\
27.1082640793892	0\\
27.1078808924634	0\\
27.1074976886322	0\\
27.1071144678941	0\\
27.1067312302476	0\\
27.1063479756912	0\\
27.1059647042234	0\\
27.1055814158427	0\\
27.1051981105476	0\\
27.1048147883366	0\\
27.1044314492083	0\\
27.1040480931611	0\\
27.1036647201935	0\\
27.1032813303041	0\\
27.1028979234913	0\\
27.1025144997538	0\\
27.1021310590898	0\\
27.1017476014981	0\\
27.101364126977	0\\
27.1009806355251	0\\
27.1005971271409	0\\
27.1002136018229	0\\
27.0998300595696	0\\
27.0994465003795	0\\
27.0990629242511	0\\
27.0986793311829	0\\
27.0982957211735	0\\
27.0979120942212	0\\
27.0975284503247	0\\
27.0971447894824	0\\
27.0967611116928	0\\
27.0963774169544	0\\
27.0959937052658	0\\
27.0956099766254	0\\
27.0952262310318	0\\
27.0948424684834	0\\
27.0944586889787	0\\
27.0940748925163	0\\
27.0936910790946	0\\
27.0933072487121	0\\
27.0929234013674	0\\
27.092539537059	0\\
27.0921556557853	0\\
27.0917717575448	0\\
27.0913878423361	0\\
27.0910039101576	0\\
27.0906199610079	0\\
27.0902359948855	0\\
27.0898520117887	0\\
27.0894680117162	0\\
27.0890839946665	0\\
27.088699960638	0\\
27.0883159096293	0\\
27.0879318416387	0\\
27.087547756665	0\\
27.0871636547064	0\\
27.0867795357616	0\\
27.086395399829	0\\
27.0860112469072	0\\
27.0856270769946	0\\
27.0852428900897	0\\
27.084858686191	0\\
27.084474465297	0\\
27.0840902274063	0\\
27.0837059725172	0\\
27.0833217006284	0\\
27.0829374117382	0\\
27.0825531058453	0\\
27.082168782948	0\\
27.0817844430449	0\\
27.0814000861346	0\\
27.0810157122153	0\\
27.0806313212858	0\\
27.0802469133444	0\\
27.0798624883897	0\\
27.0794780464201	0\\
27.0790935874342	0\\
27.0787091114304	0\\
27.0783246184073	0\\
27.0779401083632	0\\
27.0775555812968	0\\
27.0771710372065	0\\
27.0767864760909	0\\
27.0764018979483	0\\
27.0760173027773	0\\
27.0756326905763	0\\
27.075248061344	0\\
27.0748634150787	0\\
27.074478751779	0\\
27.0740940714433	0\\
27.0737093740702	0\\
27.073324659658	0\\
27.0729399282055	0\\
27.0725551797109	0\\
27.0721704141728	0\\
27.0717856315897	0\\
27.0714008319601	0\\
27.0710160152824	0\\
27.0706311815552	0\\
27.070246330777	0\\
27.0698614629462	0\\
27.0694765780613	0\\
27.0690916761209	0\\
27.0687067571233	0\\
27.0683218210671	0\\
27.0679368679508	0\\
27.0675518977729	0\\
27.0671669105318	0\\
27.0667819062261	0\\
27.0663968848542	0\\
27.0660118464146	0\\
27.0656267909058	0\\
27.0652417183263	0\\
27.0648566286746	0\\
27.0644715219491	0\\
27.0640863981484	0\\
27.0637012572709	0\\
27.0633160993152	0\\
27.0629309242796	0\\
27.0625457321627	0\\
27.062160522963	0\\
27.061775296679	0\\
27.0613900533091	0\\
27.0610047928518	0\\
27.0606195153056	0\\
27.0602342206691	0\\
27.0598489089406	0\\
27.0594635801186	0\\
27.0590782342018	0\\
27.0586928711884	0\\
27.0583074910771	0\\
27.0579220938662	0\\
27.0575366795543	0\\
27.0571512481399	0\\
27.0567657996215	0\\
27.0563803339974	0\\
27.0559948512663	0\\
27.0556093514265	0\\
27.0552238344766	0\\
27.0548383004151	0\\
27.0544527492404	0\\
27.054067180951	0\\
27.0536815955454	0\\
27.053295993022	0\\
27.0529103733794	0\\
27.0525247366161	0\\
27.0521390827304	0\\
27.051753411721	0\\
27.0513677235862	0\\
27.0509820183245	0\\
27.0505962959345	0\\
27.0502105564146	0\\
27.0498247997633	0\\
27.049439025979	0\\
27.0490532350603	0\\
27.0486674270056	0\\
27.0482816018134	0\\
27.0478957594822	0\\
27.0475099000104	0\\
27.0471240233965	0\\
27.0467381296391	0\\
27.0463522187365	0\\
27.0459662906873	0\\
27.04558034549	0\\
27.0451943831429	0\\
27.0448084036446	0\\
27.0444224069936	0\\
27.0440363931883	0\\
27.0436503622273	0\\
27.0432643141089	0\\
27.0428782488317	0\\
27.0424921663941	0\\
27.0421060667946	0\\
27.0417199500317	0\\
27.0413338161039	0\\
27.0409476650096	0\\
27.0405614967473	0\\
27.0401753113155	0\\
27.0397891087126	0\\
27.0394028889372	0\\
27.0390166519877	0\\
27.0386303978625	0\\
27.0382441265602	0\\
27.0378578380792	0\\
27.037471532418	0\\
27.0370852095751	0\\
27.0366988695488	0\\
27.0363125123378	0\\
27.0359261379405	0\\
27.0355397463553	0\\
27.0351533375807	0\\
27.0347669116152	0\\
27.0343804684572	0\\
27.0339940081053	0\\
27.0336075305579	0\\
27.0332210358135	0\\
27.0328345238704	0\\
27.0324479947273	0\\
27.0320614483826	0\\
27.0316748848347	0\\
27.0312883040821	0\\
27.0309017061233	0\\
27.0305150909567	0\\
27.0301284585809	0\\
27.0297418089942	0\\
27.0293551421952	0\\
27.0289684581823	0\\
27.028581756954	0\\
27.0281950385088	0\\
27.0278083028451	0\\
27.0274215499613	0\\
27.027034779856	0\\
27.0266479925277	0\\
27.0262611879747	0\\
27.0258743661955	0\\
27.0254875271887	0\\
27.0251006709527	0\\
27.0247137974859	0\\
27.0243269067868	0\\
27.0239399988539	0\\
27.0235530736857	0\\
27.0231661312805	0\\
27.022779171637	0\\
27.0223921947534	0\\
27.0220052006284	0\\
27.0216181892603	0\\
27.0212311606477	0\\
27.0208441147889	0\\
27.0204570516825	0\\
27.0200699713269	0\\
27.0196828737205	0\\
27.019295758862	0\\
27.0189086267496	0\\
27.0185214773819	0\\
27.0181343107573	0\\
27.0177471268743	0\\
27.0173599257313	0\\
27.0169727073268	0\\
27.0165854716594	0\\
27.0161982187273	0\\
27.0158109485291	0\\
27.0154236610633	0\\
27.0150363563283	0\\
27.0146490343226	0\\
27.0142616950446	0\\
27.0138743384928	0\\
27.0134869646656	0\\
27.0130995735616	0\\
27.0127121651791	0\\
27.0123247395166	0\\
27.0119372965727	0\\
27.0115498363456	0\\
27.011162358834	0\\
27.0107748640362	0\\
27.0103873519508	0\\
27.0099998225761	0\\
27.0096122759106	0\\
27.0092247119528	0\\
27.0088371307012	0\\
27.0084495321542	0\\
27.0080619163102	0\\
27.0076742831678	0\\
27.0072866327253	0\\
27.0068989649812	0\\
27.0065112799341	0\\
27.0061235775822	0\\
27.0057358579242	0\\
27.0053481209584	0\\
27.0049603666833	0\\
27.0045725950973	0\\
27.004184806199	0\\
27.0037969999867	0\\
27.0034091764589	0\\
27.0030213356141	0\\
27.0026334774508	0\\
27.0022456019673	0\\
27.0018577091621	0\\
27.0014697990337	0\\
27.0010818715806	0\\
27.0006939268012	0\\
27.0003059646938	0\\
26.9999179852571	0\\
26.9995299884895	0\\
26.9991419743893	0\\
26.998753942955	0\\
26.9983658941852	0\\
26.9979778280782	0\\
26.9975897446326	0\\
26.9972016438466	0\\
26.9968135257189	0\\
26.9964253902478	0\\
26.9960372374318	0\\
26.9956490672694	0\\
26.995260879759	0\\
26.994872674899	0\\
26.9944844526879	0\\
26.9940962131242	0\\
26.9937079562063	0\\
26.9933196819326	0\\
26.9929313903016	0\\
26.9925430813117	0\\
26.9921547549615	0\\
26.9917664112493	0\\
26.9913780501735	0\\
26.9909896717327	0\\
26.9906012759253	0\\
26.9902128627497	0\\
26.9898244322043	0\\
26.9894359842877	0\\
26.9890475189983	0\\
26.9886590363344	0\\
26.9882705362946	0\\
26.9878820188774	0\\
26.987493484081	0\\
26.9871049319041	0\\
26.986716362345	0\\
26.9863277754021	0\\
26.985939171074	0\\
26.9855505493591	0\\
26.9851619102558	0\\
26.9847732537625	0\\
26.9843845798778	0\\
26.9839958886	0\\
26.9836071799276	0\\
26.983218453859	0\\
26.9828297103927	0\\
26.9824409495271	0\\
26.9820521712606	0\\
26.9816633755918	0\\
26.981274562519	0\\
26.9808857320407	0\\
26.9804968841553	0\\
26.9801080188613	0\\
26.9797191361572	0\\
26.9793302360412	0\\
26.978941318512	0\\
26.9785523835678	0\\
26.9781634312073	0\\
26.9777744614288	0\\
26.9773854742307	0\\
26.9769964696115	0\\
26.9766074475696	0\\
26.9762184081036	0\\
26.9758293512117	0\\
26.9754402768925	0\\
26.9750511851443	0\\
26.9746620759657	0\\
26.9742729493551	0\\
26.9738838053108	0\\
26.9734946438314	0\\
26.9731054649153	0\\
26.9727162685609	0\\
26.9723270547666	0\\
26.971937823531	0\\
26.9715485748523	0\\
26.9711593087291	0\\
26.9707700251599	0\\
26.9703807241429	0\\
26.9699914056768	0\\
26.9696020697598	0\\
26.9692127163905	0\\
26.9688233455672	0\\
26.9684339572885	0\\
26.9680445515527	0\\
26.9676551283583	0\\
26.9672656877037	0\\
26.9668762295874	0\\
26.9664867540077	0\\
26.9660972609632	0\\
26.9657077504522	0\\
26.9653182224732	0\\
26.9649286770247	0\\
26.964539114105	0\\
26.9641495337125	0\\
26.9637599358458	0\\
26.9633703205032	0\\
26.9629806876833	0\\
26.9625910373843	0\\
26.9622013696048	0\\
26.9618116843431	0\\
26.9614219815978	0\\
26.9610322613672	0\\
26.9606425236497	0\\
26.9602527684439	0\\
26.9598629957481	0\\
26.9594732055608	0\\
26.9590833978803	0\\
26.9586935727052	0\\
26.9583037300338	0\\
26.9579138698646	0\\
26.9575239921961	0\\
26.9571340970265	0\\
26.9567441843544	0\\
26.9563542541782	0\\
26.9559643064964	0\\
26.9555743413073	0\\
26.9551843586094	0\\
26.954794358401	0\\
26.9544043406808	0\\
26.9540143054469	0\\
26.953624252698	0\\
26.9532341824324	0\\
26.9528440946485	0\\
26.9524539893448	0\\
26.9520638665197	0\\
26.9516737261716	0\\
26.951283568299	0\\
26.9508933929003	0\\
26.9505031999738	0\\
26.9501129895181	0\\
26.9497227615315	0\\
26.9493325160124	0\\
26.9489422529594	0\\
26.9485519723708	0\\
26.9481616742451	0\\
26.9477713585806	0\\
26.9473810253758	0\\
26.9469906746291	0\\
26.946600306339	0\\
26.9462099205038	0\\
26.9458195171221	0\\
26.9454290961921	0\\
26.9450386577124	0\\
26.9446482016813	0\\
26.9442577280973	0\\
26.9438672369587	0\\
26.9434767282641	0\\
26.9430862020119	0\\
26.9426956582004	0\\
26.9423050968281	0\\
26.9419145178934	0\\
26.9415239213947	0\\
26.9411333073304	0\\
26.9407426756991	0\\
26.940352026499	0\\
26.9399613597286	0\\
26.9395706753863	0\\
26.9391799734706	0\\
26.9387892539798	0\\
26.9383985169124	0\\
26.9380077622668	0\\
26.9376169900415	0\\
26.9372262002347	0\\
26.936835392845	0\\
26.9364445678708	0\\
26.9360537253105	0\\
26.9356628651624	0\\
26.9352719874251	0\\
26.934881092097	0\\
26.9344901791764	0\\
26.9340992486617	0\\
26.9337083005515	0\\
26.933317334844	0\\
26.9329263515378	0\\
26.9325353506312	0\\
26.9321443321227	0\\
26.9317532960106	0\\
26.9313622422934	0\\
26.9309711709695	0\\
26.9305800820373	0\\
26.9301889754952	0\\
26.9297978513417	0\\
26.9294067095751	0\\
26.9290155501939	0\\
26.9286243731965	0\\
26.9282331785812	0\\
26.9278419663466	0\\
26.927450736491	0\\
26.9270594890128	0\\
26.9266682239104	0\\
26.9262769411823	0\\
26.9258856408269	0\\
26.9254943228426	0\\
26.9251029872277	0\\
26.9247116339808	0\\
26.9243202631001	0\\
26.9239288745842	0\\
26.9235374684314	0\\
26.9231460446402	0\\
26.9227546032089	0\\
26.922363144136	0\\
26.9219716674198	0\\
26.9215801730588	0\\
26.9211886610515	0\\
26.9207971313961	0\\
26.9204055840911	0\\
26.920014019135	0\\
26.9196224365261	0\\
26.9192308362628	0\\
26.9188392183436	0\\
26.9184475827668	0\\
26.9180559295309	0\\
26.9176642586343	0\\
26.9172725700753	0\\
26.9168808638524	0\\
26.916489139964	0\\
26.9160973984085	0\\
26.9157056391843	0\\
26.9153138622898	0\\
26.9149220677234	0\\
26.9145302554835	0\\
26.9141384255686	0\\
26.9137465779769	0\\
26.9133547127071	0\\
26.9129628297573	0\\
26.9125709291261	0\\
26.9121790108119	0\\
26.911787074813	0\\
26.9113951211278	0\\
26.9110031497549	0\\
26.9106111606924	0\\
26.910219153939	0\\
26.9098271294929	0\\
26.9094350873526	0\\
26.9090430275165	0\\
26.9086509499829	0\\
26.9082588547504	0\\
26.9078667418172	0\\
26.9074746111818	0\\
26.9070824628426	0\\
26.9066902967979	0\\
26.9062981130463	0\\
26.905905911586	0\\
26.9055136924155	0\\
26.9051214555333	0\\
26.9047292009376	0\\
26.9043369286269	0\\
26.9039446385996	0\\
26.9035523308541	0\\
26.9031600053888	0\\
26.902767662202	0\\
26.9023753012923	0\\
26.9019829226579	0\\
26.9015905262973	0\\
26.9011981122089	0\\
26.9008056803911	0\\
26.9004132308423	0\\
26.9000207635608	0\\
26.8996282785452	0\\
26.8992357757936	0\\
26.8988432553047	0\\
26.8984507170767	0\\
26.8980581611081	0\\
26.8976655873973	0\\
26.8972729959426	0\\
26.8968803867424	0\\
26.8964877597952	0\\
26.8960951150993	0\\
26.8957024526532	0\\
26.8953097724552	0\\
26.8949170745038	0\\
26.8945243587972	0\\
26.894131625334	0\\
26.8937388741125	0\\
26.893346105131	0\\
26.8929533183881	0\\
26.8925605138821	0\\
26.8921676916113	0\\
26.8917748515743	0\\
26.8913819937693	0\\
26.8909891181947	0\\
26.8905962248491	0\\
26.8902033137306	0\\
26.8898103848379	0\\
26.8894174381691	0\\
26.8890244737228	0\\
26.8886314914972	0\\
26.8882384914909	0\\
26.8878454737022	0\\
26.8874524381295	0\\
26.8870593847711	0\\
26.8866663136255	0\\
26.8862732246911	0\\
26.8858801179662	0\\
26.8854869934493	0\\
26.8850938511387	0\\
26.8847006910328	0\\
26.88430751313	0\\
26.8839143174287	0\\
26.8835211039273	0\\
26.8831278726241	0\\
26.8827346235176	0\\
26.8823413566062	0\\
26.8819480718882	0\\
26.881554769362	0\\
26.881161449026	0\\
26.8807681108787	0\\
26.8803747549183	0\\
26.8799813811432	0\\
26.879587989552	0\\
26.8791945801428	0\\
26.8788011529142	0\\
26.8784077078645	0\\
26.8780142449921	0\\
26.8776207642954	0\\
26.8772272657728	0\\
26.8768337494226	0\\
26.8764402152432	0\\
26.8760466632331	0\\
26.8756530933906	0\\
26.875259505714	0\\
26.8748659002019	0\\
26.8744722768525	0\\
26.8740786356642	0\\
26.8736849766355	0\\
26.8732912997647	0\\
26.8728976050501	0\\
26.8725038924903	0\\
26.8721101620834	0\\
26.8717164138281	0\\
26.8713226477225	0\\
26.8709288637651	0\\
26.8705350619543	0\\
26.8701412422885	0\\
26.869747404766	0\\
26.8693535493853	0\\
26.8689596761446	0\\
26.8685657850425	0\\
26.8681718760771	0\\
26.8677779492471	0\\
26.8673840045506	0\\
26.8669900419862	0\\
26.8665960615521	0\\
26.8662020632468	0\\
26.8658080470687	0\\
26.865414013016	0\\
26.8650199610873	0\\
26.8646258912808	0\\
26.8642318035949	0\\
26.8638376980281	0\\
26.8634435745787	0\\
26.8630494332451	0\\
26.8626552740257	0\\
26.8622610969188	0\\
26.8618669019227	0\\
26.861472689036	0\\
26.861078458257	0\\
26.860684209584	0\\
26.8602899430154	0\\
26.8598956585495	0\\
26.8595013561849	0\\
26.8591070359198	0\\
26.8587126977526	0\\
26.8583183416817	0\\
26.8579239677055	0\\
26.8575295758223	0\\
26.8571351660305	0\\
26.8567407383285	0\\
26.8563462927147	0\\
26.8559518291874	0\\
26.8555573477449	0\\
26.8551628483858	0\\
26.8547683311083	0\\
26.8543737959109	0\\
26.8539792427918	0\\
26.8535846717495	0\\
26.8531900827824	0\\
26.8527954758887	0\\
26.852400851067	0\\
26.8520062083154	0\\
26.8516115476325	0\\
26.8512168690166	0\\
26.8508221724661	0\\
26.8504274579793	0\\
26.8500327255546	0\\
26.8496379751903	0\\
26.8492432068849	0\\
26.8488484206367	0\\
26.8484536164441	0\\
};
\addplot [color=mycolor1, forget plot]
  table[row sep=crcr]{%
26.8484536164441	0\\
26.8480587943055	0\\
26.8476639542191	0\\
26.8472690961834	0\\
26.8468742201968	0\\
26.8464793262576	0\\
26.8460844143642	0\\
26.845689484515	0\\
26.8452945367082	0\\
26.8448995709424	0\\
26.8445045872158	0\\
26.8441095855268	0\\
26.8437145658739	0\\
26.8433195282552	0\\
26.8429244726693	0\\
26.8425293991145	0\\
26.8421343075891	0\\
26.8417391980916	0\\
26.8413440706202	0\\
26.8409489251734	0\\
26.8405537617494	0\\
26.8401585803468	0\\
26.8397633809638	0\\
26.8393681635988	0\\
26.8389729282501	0\\
26.8385776749162	0\\
26.8381824035954	0\\
26.837787114286	0\\
26.8373918069864	0\\
26.836996481695	0\\
26.8366011384102	0\\
26.8362057771303	0\\
26.8358103978536	0\\
26.8354150005785	0\\
26.8350195853035	0\\
26.8346241520268	0\\
26.8342287007468	0\\
26.8338332314619	0\\
26.8334377441704	0\\
26.8330422388706	0\\
26.8326467155611	0\\
26.83225117424	0\\
26.8318556149059	0\\
26.8314600375569	0\\
26.8310644421915	0\\
26.8306688288081	0\\
26.830273197405	0\\
26.8298775479806	0\\
26.8294818805332	0\\
26.8290861950611	0\\
26.8286904915628	0\\
26.8282947700366	0\\
26.8278990304808	0\\
26.8275032728939	0\\
26.8271074972741	0\\
26.8267117036198	0\\
26.8263158919295	0\\
26.8259200622013	0\\
26.8255242144338	0\\
26.8251283486251	0\\
26.8247324647738	0\\
26.8243365628782	0\\
26.8239406429366	0\\
26.8235447049473	0\\
26.8231487489087	0\\
26.8227527748193	0\\
26.8223567826772	0\\
26.821960772481	0\\
26.8215647442289	0\\
26.8211686979192	0\\
26.8207726335505	0\\
26.8203765511209	0\\
26.8199804506288	0\\
26.8195843320727	0\\
26.8191881954508	0\\
26.8187920407616	0\\
26.8183958680033	0\\
26.8179996771742	0\\
26.8176034682729	0\\
26.8172072412976	0\\
26.8168109962466	0\\
26.8164147331184	0\\
26.8160184519112	0\\
26.8156221526234	0\\
26.8152258352534	0\\
26.8148294997996	0\\
26.8144331462601	0\\
26.8140367746335	0\\
26.8136403849181	0\\
26.8132439771122	0\\
26.8128475512141	0\\
26.8124511072222	0\\
26.8120546451349	0\\
26.8116581649506	0\\
26.8112616666674	0\\
26.8108651502839	0\\
26.8104686157983	0\\
26.810072063209	0\\
26.8096754925144	0\\
26.8092789037127	0\\
26.8088822968024	0\\
26.8084856717817	0\\
26.8080890286491	0\\
26.8076923674029	0\\
26.8072956880413	0\\
26.8068989905629	0\\
26.8065022749658	0\\
26.8061055412485	0\\
26.8057087894094	0\\
26.8053120194466	0\\
26.8049152313586	0\\
26.8045184251438	0\\
26.8041216008004	0\\
26.8037247583269	0\\
26.8033278977215	0\\
26.8029310189827	0\\
26.8025341221086	0\\
26.8021372070978	0\\
26.8017402739485	0\\
26.8013433226591	0\\
26.8009463532279	0\\
26.8005493656532	0\\
26.8001523599335	0\\
26.799755336067	0\\
26.799358294052	0\\
26.798961233887	0\\
26.7985641555703	0\\
26.7981670591001	0\\
26.7977699444749	0\\
26.797372811693	0\\
26.7969756607527	0\\
26.7965784916524	0\\
26.7961813043904	0\\
26.795784098965	0\\
26.7953868753746	0\\
26.7949896336175	0\\
26.7945923736921	0\\
26.7941950955966	0\\
26.7937977993295	0\\
26.7934004848891	0\\
26.7930031522737	0\\
26.7926058014816	0\\
26.7922084325112	0\\
26.7918110453608	0\\
26.7914136400288	0\\
26.7910162165134	0\\
26.7906187748131	0\\
26.7902213149262	0\\
26.7898238368509	0\\
26.7894263405857	0\\
26.7890288261288	0\\
26.7886312934787	0\\
26.7882337426335	0\\
26.7878361735918	0\\
26.7874385863517	0\\
26.7870409809117	0\\
26.7866433572701	0\\
26.7862457154252	0\\
26.7858480553753	0\\
26.7854503771188	0\\
26.785052680654	0\\
26.7846549659792	0\\
26.7842572330928	0\\
26.7838594819931	0\\
26.7834617126785	0\\
26.7830639251472	0\\
26.7826661193976	0\\
26.782268295428	0\\
26.7818704532368	0\\
26.7814725928223	0\\
26.7810747141828	0\\
26.7806768173166	0\\
26.7802789022222	0\\
26.7798809688977	0\\
26.7794830173416	0\\
26.7790850475522	0\\
26.7786870595277	0\\
26.7782890532666	0\\
26.7778910287671	0\\
26.7774929860276	0\\
26.7770949250465	0\\
26.7766968458219	0\\
26.7762987483524	0\\
26.7759006326361	0\\
26.7755024986715	0\\
26.7751043464568	0\\
26.7747061759904	0\\
26.7743079872706	0\\
26.7739097802957	0\\
26.773511555064	0\\
26.773113311574	0\\
26.7727150498239	0\\
26.772316769812	0\\
26.7719184715366	0\\
26.7715201549962	0\\
26.7711218201889	0\\
26.7707234671132	0\\
26.7703250957673	0\\
26.7699267061496	0\\
26.7695282982585	0\\
26.7691298720921	0\\
26.7687314276489	0\\
26.7683329649272	0\\
26.7679344839252	0\\
26.7675359846414	0\\
26.7671374670741	0\\
26.7667389312215	0\\
26.7663403770819	0\\
26.7659418046538	0\\
26.7655432139354	0\\
26.7651446049251	0\\
26.7647459776212	0\\
26.7643473320219	0\\
26.7639486681256	0\\
26.7635499859307	0\\
26.7631512854355	0\\
26.7627525666382	0\\
26.7623538295372	0\\
26.7619550741308	0\\
26.7615563004173	0\\
26.7611575083951	0\\
26.7607586980625	0\\
26.7603598694177	0\\
26.7599610224592	0\\
26.7595621571852	0\\
26.759163273594	0\\
26.758764371684	0\\
26.7583654514535	0\\
26.7579665129007	0\\
26.7575675560241	0\\
26.7571685808219	0\\
26.7567695872924	0\\
26.756370575434	0\\
26.755971545245	0\\
26.7555724967237	0\\
26.7551734298683	0\\
26.7547743446773	0\\
26.754375241149	0\\
26.7539761192816	0\\
26.7535769790734	0\\
26.7531778205228	0\\
26.7527786436282	0\\
26.7523794483877	0\\
26.7519802347998	0\\
26.7515810028627	0\\
26.7511817525747	0\\
26.7507824839342	0\\
26.7503831969395	0\\
26.7499838915889	0\\
26.7495845678806	0\\
26.7491852258131	0\\
26.7487858653846	0\\
26.7483864865935	0\\
26.7479870894379	0\\
26.7475876739164	0\\
26.7471882400271	0\\
26.7467887877684	0\\
26.7463893171386	0\\
26.7459898281359	0\\
26.7455903207588	0\\
26.7451907950055	0\\
26.7447912508743	0\\
26.7443916883636	0\\
26.7439921074716	0\\
26.7435925081967	0\\
26.7431928905371	0\\
26.7427932544912	0\\
26.7423936000573	0\\
26.7419939272337	0\\
26.7415942360186	0\\
26.7411945264105	0\\
26.7407947984076	0\\
26.7403950520082	0\\
26.7399952872106	0\\
26.7395955040131	0\\
26.7391957024141	0\\
26.7387958824118	0\\
26.7383960440046	0\\
26.7379961871907	0\\
26.7375963119684	0\\
26.7371964183362	0\\
26.7367965062922	0\\
26.7363965758348	0\\
26.7359966269622	0\\
26.7355966596728	0\\
26.735196673965	0\\
26.7347966698369	0\\
26.7343966472869	0\\
26.7339966063132	0\\
26.7335965469143	0\\
26.7331964690884	0\\
26.7327963728338	0\\
26.7323962581488	0\\
26.7319961250317	0\\
26.7315959734808	0\\
26.7311958034944	0\\
26.7307956150708	0\\
26.7303954082083	0\\
26.7299951829053	0\\
26.7295949391599	0\\
26.7291946769706	0\\
26.7287943963356	0\\
26.7283940972532	0\\
26.7279937797217	0\\
26.7275934437394	0\\
26.7271930893046	0\\
26.7267927164157	0\\
26.7263923250708	0\\
26.7259919152683	0\\
26.7255914870066	0\\
26.7251910402839	0\\
26.7247905750984	0\\
26.7243900914486	0\\
26.7239895893326	0\\
26.7235890687489	0\\
26.7231885296956	0\\
26.7227879721711	0\\
26.7223873961737	0\\
26.7219868017017	0\\
26.7215861887534	0\\
26.721185557327	0\\
26.7207849074209	0\\
26.7203842390333	0\\
26.7199835521626	0\\
26.7195828468071	0\\
26.719182122965	0\\
26.7187813806346	0\\
26.7183806198143	0\\
26.7179798405022	0\\
26.7175790426968	0\\
26.7171782263964	0\\
26.7167773915991	0\\
26.7163765383033	0\\
26.7159756665073	0\\
26.7155747762094	0\\
26.7151738674079	0\\
26.714772940101	0\\
26.7143719942871	0\\
26.7139710299644	0\\
26.7135700471313	0\\
26.713169045786	0\\
26.7127680259269	0\\
26.7123669875521	0\\
26.7119659306601	0\\
26.711564855249	0\\
26.7111637613172	0\\
26.710762648863	0\\
26.7103615178847	0\\
26.7099603683805	0\\
26.7095592003487	0\\
26.7091580137877	0\\
26.7087568086957	0\\
26.708355585071	0\\
26.7079543429118	0\\
26.7075530822166	0\\
26.7071518029835	0\\
26.7067505052109	0\\
26.706349188897	0\\
26.7059478540401	0\\
26.7055465006386	0\\
26.7051451286906	0\\
26.7047437381946	0\\
26.7043423291487	0\\
26.7039409015512	0\\
26.7035394554005	0\\
26.7031379906949	0\\
26.7027365074325	0\\
26.7023350056117	0\\
26.7019334852308	0\\
26.7015319462881	0\\
26.7011303887818	0\\
26.7007288127103	0\\
26.7003272180717	0\\
26.6999256048645	0\\
26.6995239730868	0\\
26.699122322737	0\\
26.6987206538134	0\\
26.6983189663141	0\\
26.6979172602376	0\\
26.6975155355821	0\\
26.6971137923458	0\\
26.6967120305271	0\\
26.6963102501242	0\\
26.6959084511355	0\\
26.6955066335591	0\\
26.6951047973934	0\\
26.6947029426367	0\\
26.6943010692872	0\\
26.6938991773432	0\\
26.693497266803	0\\
26.6930953376648	0\\
26.692693389927	0\\
26.6922914235879	0\\
26.6918894386456	0\\
26.6914874350986	0\\
26.691085412945	0\\
26.6906833721831	0\\
26.6902813128113	0\\
26.6898792348277	0\\
26.6894771382307	0\\
26.6890750230186	0\\
26.6886728891896	0\\
26.688270736742	0\\
26.6878685656741	0\\
26.6874663759841	0\\
26.6870641676703	0\\
26.6866619407311	0\\
26.6862596951646	0\\
26.6858574309692	0\\
26.6854551481431	0\\
26.6850528466846	0\\
26.684650526592	0\\
26.6842481878635	0\\
26.6838458304975	0\\
26.6834434544921	0\\
26.6830410598457	0\\
26.6826386465566	0\\
26.6822362146229	0\\
26.6818337640431	0\\
26.6814312948153	0\\
26.6810288069378	0\\
26.680626300409	0\\
26.680223775227	0\\
26.6798212313901	0\\
26.6794186688967	0\\
26.6790160877449	0\\
26.6786134879331	0\\
26.6782108694596	0\\
26.6778082323225	0\\
26.6774055765202	0\\
26.6770029020509	0\\
26.6766002089129	0\\
26.6761974971045	0\\
26.6757947666239	0\\
26.6753920174695	0\\
26.6749892496394	0\\
26.674586463132	0\\
26.6741836579454	0\\
26.6737808340781	0\\
26.6733779915282	0\\
26.672975130294	0\\
26.6725722503738	0\\
26.6721693517659	0\\
26.6717664344684	0\\
26.6713634984798	0\\
26.6709605437981	0\\
26.6705575704218	0\\
26.6701545783491	0\\
26.6697515675782	0\\
26.6693485381074	0\\
26.668945489935	0\\
26.6685424230592	0\\
26.6681393374784	0\\
26.6677362331906	0\\
26.6673331101943	0\\
26.6669299684877	0\\
26.6665268080691	0\\
26.6661236289366	0\\
26.6657204310887	0\\
26.6653172145235	0\\
26.6649139792393	0\\
26.6645107252343	0\\
26.6641074525069	0\\
26.6637041610552	0\\
26.6633008508777	0\\
26.6628975219724	0\\
26.6624941743377	0\\
26.6620908079718	0\\
26.661687422873	0\\
26.6612840190396	0\\
26.6608805964697	0\\
26.6604771551618	0\\
26.660073695114	0\\
26.6596702163245	0\\
26.6592667187917	0\\
26.6588632025138	0\\
26.6584596674891	0\\
26.6580561137158	0\\
26.6576525411922	0\\
26.6572489499165	0\\
26.656845339887	0\\
26.656441711102	0\\
26.6560380635597	0\\
26.6556343972583	0\\
26.6552307121962	0\\
26.6548270083715	0\\
26.6544232857826	0\\
26.6540195444276	0\\
26.6536157843049	0\\
26.6532120054127	0\\
26.6528082077492	0\\
26.6524043913128	0\\
26.6520005561016	0\\
26.651596702114	0\\
26.6511928293481	0\\
26.6507889378022	0\\
26.6503850274746	0\\
26.6499810983636	0\\
26.6495771504673	0\\
26.6491731837841	0\\
26.6487691983121	0\\
26.6483651940497	0\\
26.6479611709951	0\\
26.6475571291465	0\\
26.6471530685022	0\\
26.6467489890605	0\\
26.6463448908196	0\\
26.6459407737777	0\\
26.6455366379331	0\\
26.6451324832841	0\\
26.6447283098289	0\\
26.6443241175657	0\\
26.6439199064928	0\\
26.6435156766085	0\\
26.6431114279109	0\\
26.6427071603984	0\\
26.6423028740692	0\\
26.6418985689216	0\\
26.6414942449537	0\\
26.6410899021639	0\\
26.6406855405504	0\\
26.6402811601114	0\\
26.6398767608451	0\\
26.6394723427499	0\\
26.639067905824	0\\
26.6386634500656	0\\
26.638258975473	0\\
26.6378544820443	0\\
26.637449969778	0\\
26.6370454386721	0\\
26.636640888725	0\\
26.6362363199348	0\\
26.6358317323	0\\
26.6354271258185	0\\
26.6350225004889	0\\
26.6346178563092	0\\
26.6342131932777	0\\
26.6338085113926	0\\
26.6334038106523	0\\
26.6329990910549	0\\
26.6325943525986	0\\
26.6321895952819	0\\
26.6317848191027	0\\
26.6313800240595	0\\
26.6309752101505	0\\
26.6305703773738	0\\
26.6301655257278	0\\
26.6297606552106	0\\
26.6293557658206	0\\
26.6289508575559	0\\
26.6285459304149	0\\
26.6281409843956	0\\
26.6277360194965	0\\
26.6273310357156	0\\
26.6269260330513	0\\
26.6265210115018	0\\
26.6261159710654	0\\
26.6257109117402	0\\
26.6253058335245	0\\
26.6249007364166	0\\
26.6244956204146	0\\
26.6240904855169	0\\
26.6236853317217	0\\
26.6232801590271	0\\
26.6228749674315	0\\
26.622469756933	0\\
26.62206452753	0\\
26.6216592792206	0\\
26.621254012003	0\\
26.6208487258756	0\\
26.6204434208366	0\\
26.6200380968841	0\\
26.6196327540164	0\\
26.6192273922318	0\\
26.6188220115285	0\\
26.6184166119047	0\\
26.6180111933587	0\\
26.6176057558886	0\\
26.6172002994928	0\\
26.6167948241694	0\\
26.6163893299167	0\\
26.615983816733	0\\
26.6155782846164	0\\
26.6151727335651	0\\
26.6147671635775	0\\
26.6143615746518	0\\
26.6139559667861	0\\
26.6135503399787	0\\
26.6131446942279	0\\
26.6127390295318	0\\
26.6123333458888	0\\
26.6119276432969	0\\
26.6115219217546	0\\
26.6111161812599	0\\
26.6107104218112	0\\
26.6103046434066	0\\
26.6098988460443	0\\
26.6094930297227	0\\
26.60908719444	0\\
26.6086813401942	0\\
26.6082754669838	0\\
26.6078695748069	0\\
26.6074636636618	0\\
26.6070577335466	0\\
26.6066517844596	0\\
26.606245816399	0\\
26.6058398293631	0\\
26.6054338233501	0\\
26.6050277983582	0\\
26.6046217543856	0\\
26.6042156914306	0\\
26.6038096094913	0\\
26.6034035085661	0\\
26.602997388653	0\\
26.6025912497505	0\\
26.6021850918566	0\\
26.6017789149696	0\\
26.6013727190878	0\\
26.6009665042093	0\\
26.6005602703323	0\\
26.6001540174552	0\\
26.5997477455761	0\\
26.5993414546932	0\\
26.5989351448048	0\\
26.598528815909	0\\
26.5981224680042	0\\
26.5977161010885	0\\
26.5973097151602	0\\
26.5969033102174	0\\
26.5964968862584	0\\
26.5960904432814	0\\
26.5956839812846	0\\
26.5952775002663	0\\
26.5948710002247	0\\
26.5944644811579	0\\
26.5940579430643	0\\
26.593651385942	0\\
26.5932448097892	0\\
26.5928382146042	0\\
26.5924316003852	0\\
26.5920249671304	0\\
26.591618314838	0\\
26.5912116435062	0\\
26.5908049531333	0\\
26.5903982437174	0\\
26.5899915152569	0\\
26.5895847677498	0\\
26.5891780011944	0\\
26.588771215589	0\\
26.5883644109318	0\\
26.5879575872209	0\\
26.5875507444545	0\\
26.587143882631	0\\
26.5867370017485	0\\
26.5863301018052	0\\
26.5859231827993	0\\
26.5855162447291	0\\
26.5851092875928	0\\
26.5847023113885	0\\
26.5842953161145	0\\
26.583888301769	0\\
26.5834812683503	0\\
26.5830742158564	0\\
26.5826671442857	0\\
26.5822600536364	0\\
26.5818529439066	0\\
26.5814458150946	0\\
26.5810386671986	0\\
26.5806315002168	0\\
26.5802243141474	0\\
26.5798171089886	0\\
26.5794098847386	0\\
26.5790026413957	0\\
26.578595378958	0\\
26.5781880974238	0\\
26.5777807967912	0\\
26.5773734770585	0\\
26.5769661382239	0\\
26.5765587802856	0\\
26.5761514032418	0\\
26.5757440070907	0\\
26.5753365918305	0\\
26.5749291574594	0\\
26.5745217039757	0\\
26.5741142313775	0\\
26.573706739663	0\\
26.5732992288305	0\\
26.5728916988781	0\\
26.5724841498041	0\\
26.5720765816067	0\\
26.571668994284	0\\
26.5712613878343	0\\
26.5708537622558	0\\
26.5704461175467	0\\
26.5700384537052	0\\
26.5696307707294	0\\
26.5692230686177	0\\
26.5688153473682	0\\
26.568407606979	0\\
26.5679998474485	0\\
26.5675920687748	0\\
26.567184270956	0\\
26.5667764539905	0\\
26.5663686178765	0\\
26.565960762612	0\\
26.5655528881954	0\\
26.5651449946247	0\\
26.5647370818983	0\\
26.5643291500143	0\\
26.563921198971	0\\
26.5635132287664	0\\
26.5631052393989	0\\
26.5626972308666	0\\
26.5622892031678	0\\
26.5618811563005	0\\
26.5614730902631	0\\
26.5610650050537	0\\
26.5606569006705	0\\
26.5602487771117	0\\
26.5598406343756	0\\
26.5594324724602	0\\
26.5590242913639	0\\
26.5586160910847	0\\
26.558207871621	0\\
26.5577996329709	0\\
26.5573913751325	0\\
26.5569830981042	0\\
26.556574801884	0\\
26.5561664864702	0\\
26.555758151861	0\\
26.5553497980546	0\\
26.5549414250491	0\\
26.5545330328428	0\\
26.5541246214339	0\\
26.5537161908205	0\\
26.5533077410008	0\\
26.5528992719731	0\\
26.5524907837356	0\\
26.5520822762863	0\\
26.5516737496236	0\\
26.5512652037456	0\\
26.5508566386504	0\\
26.5504480543364	0\\
26.5500394508017	0\\
26.5496308280444	0\\
26.5492221860628	0\\
26.5488135248551	0\\
26.5484048444194	0\\
26.547996144754	0\\
26.547587425857	0\\
26.5471786877266	0\\
26.546769930361	0\\
26.5463611537585	0\\
26.5459523579171	0\\
26.5455435428351	0\\
26.5451347085107	0\\
26.544725854942	0\\
26.5443169821273	0\\
26.5439080900647	0\\
26.5434991787524	0\\
26.5430902481886	0\\
26.5426812983716	0\\
26.5422723292994	0\\
26.5418633409703	0\\
26.5414543333824	0\\
26.541045306534	0\\
26.5406362604232	0\\
26.5402271950482	0\\
26.5398181104072	0\\
26.5394090064984	0\\
26.53899988332	0\\
26.5385907408701	0\\
26.538181579147	0\\
26.5377723981487	0\\
26.5373631978736	0\\
26.5369539783198	0\\
26.5365447394855	0\\
26.5361354813688	0\\
26.5357262039679	0\\
26.5353169072811	0\\
26.5349075913064	0\\
26.5344982560422	0\\
26.5340889014865	0\\
26.5336795276376	0\\
26.5332701344936	0\\
26.5328607220527	0\\
26.5324512903131	0\\
26.532041839273	0\\
26.5316323689305	0\\
26.5312228792839	0\\
26.5308133703312	0\\
26.5304038420708	0\\
26.5299942945007	0\\
26.5295847276192	0\\
26.5291751414244	0\\
26.5287655359146	0\\
26.5283559110878	0\\
26.5279462669423	0\\
26.5275366034762	0\\
26.5271269206877	0\\
26.526717218575	0\\
26.5263074971363	0\\
26.5258977563698	0\\
26.5254879962736	0\\
26.5250782168458	0\\
26.5246684180848	0\\
26.5242585999886	0\\
26.5238487625554	0\\
26.5234389057835	0\\
26.5230290296709	0\\
26.5226191342159	0\\
26.5222092194166	0\\
26.5217992852712	0\\
26.5213893317779	0\\
26.5209793589348	0\\
26.5205693667402	0\\
26.5201593551921	0\\
26.5197493242888	0\\
26.5193392740285	0\\
26.5189292044093	0\\
26.5185191154293	0\\
26.5181090070868	0\\
26.51769887938	0\\
26.5172887323069	0\\
26.5168785658658	0\\
26.5164683800549	0\\
26.5160581748723	0\\
26.5156479503161	0\\
26.5152377063846	0\\
26.5148274430759	0\\
26.5144171603882	0\\
26.5140068583197	0\\
26.5135965368685	0\\
26.5131861960328	0\\
26.5127758358108	0\\
26.5123654562006	0\\
26.5119550572004	0\\
26.5115446388084	0\\
26.5111342010227	0\\
26.5107237438415	0\\
26.510313267263	0\\
26.5099027712853	0\\
26.5094922559066	0\\
26.5090817211251	0\\
26.5086711669389	0\\
26.5082605933463	0\\
26.5078500003453	0\\
26.5074393879341	0\\
26.5070287561109	0\\
26.5066181048739	0\\
26.5062074342212	0\\
26.505796744151	0\\
26.5053860346615	0\\
26.5049753057507	0\\
26.504564557417	0\\
26.5041537896584	0\\
26.5037430024731	0\\
26.5033321958592	0\\
26.502921369815	0\\
26.5025105243386	0\\
26.5020996594281	0\\
26.5016887750818	0\\
26.5012778712977	0\\
26.5008669480741	0\\
26.500456005409	0\\
26.5000450433008	0\\
26.4996340617474	0\\
26.4992230607471	0\\
26.498812040298	0\\
26.4984010003984	0\\
26.4979899410463	0\\
26.4975788622399	0\\
26.4971677639774	0\\
26.4967566462569	0\\
26.4963455090766	0\\
26.4959343524346	0\\
26.4955231763291	0\\
26.4951119807583	0\\
26.4947007657204	0\\
26.4942895312134	0\\
26.4938782772355	0\\
26.4934670037849	0\\
26.4930557108598	0\\
26.4926443984582	0\\
26.4922330665784	0\\
26.4918217152185	0\\
26.4914103443767	0\\
26.4909989540511	0\\
26.4905875442399	0\\
26.4901761149412	0\\
26.4897646661531	0\\
26.4893531978739	0\\
26.4889417101017	0\\
26.4885302028346	0\\
26.4881186760709	0\\
26.4877071298085	0\\
26.4872955640458	0\\
26.4868839787808	0\\
26.4864723740117	0\\
26.4860607497367	0\\
26.4856491059539	0\\
26.4852374426614	0\\
26.4848257598574	0\\
26.4844140575401	0\\
26.4840023357076	0\\
26.483590594358	0\\
26.4831788334895	0\\
26.4827670531003	0\\
26.4823552531885	0\\
26.4819434337523	0\\
26.4815315947897	0\\
26.481119736299	0\\
26.4807078582783	0\\
26.4802959607258	0\\
26.4798840436395	0\\
26.4794721070176	0\\
26.4790601508584	0\\
26.4786481751599	0\\
26.4782361799202	0\\
26.4778241651376	0\\
26.4774121308102	0\\
26.4770000769361	0\\
26.4765880035134	0\\
26.4761759105403	0\\
26.475763798015	0\\
26.4753516659356	0\\
26.4749395143002	0\\
26.474527343107	0\\
26.4741151523542	0\\
26.4737029420398	0\\
26.473290712162	0\\
26.472878462719	0\\
26.4724661937089	0\\
26.4720539051298	0\\
26.4716415969799	0\\
26.4712292692573	0\\
26.4708169219602	0\\
26.4704045550867	0\\
26.469992168635	0\\
26.4695797626032	0\\
26.4691673369894	0\\
26.4687548917917	0\\
26.4683424270084	0\\
26.4679299426375	0\\
26.4675174386772	0\\
26.4671049151257	0\\
26.466692371981	0\\
26.4662798092413	0\\
26.4658672269048	0\\
26.4654546249696	0\\
26.4650420034337	0\\
26.4646293622955	0\\
26.4642167015529	0\\
26.4638040212042	0\\
26.4633913212475	0\\
26.4629786016808	0\\
26.4625658625024	0\\
26.4621531037104	0\\
26.4617403253029	0\\
26.4613275272781	0\\
26.460914709634	0\\
26.4605018723689	0\\
26.4600890154808	0\\
26.459676138968	0\\
26.4592632428284	0\\
26.4588503270603	0\\
26.4584373916618	0\\
26.458024436631	0\\
26.4576114619661	0\\
26.4571984676652	0\\
26.4567854537264	0\\
26.4563724201479	0\\
26.4559593669278	0\\
26.4555462940641	0\\
26.4551332015552	0\\
26.454720089399	0\\
26.4543069575937	0\\
26.4538938061375	0\\
26.4534806350285	0\\
26.4530674442648	0\\
26.4526542338445	0\\
26.4522410037657	0\\
26.4518277540267	0\\
26.4514144846255	0\\
26.4510011955603	0\\
26.4505878868291	0\\
26.4501745584302	0\\
26.4497612103616	0\\
26.4493478426214	0\\
26.4489344552079	0\\
26.4485210481191	0\\
26.4481076213531	0\\
26.4476941749081	0\\
26.4472807087822	0\\
26.4468672229735	0\\
26.4464537174802	0\\
26.4460401923004	0\\
26.4456266474322	0\\
26.4452130828737	0\\
26.444799498623	0\\
26.4443858946784	0\\
26.4439722710378	0\\
26.4435586276995	0\\
26.4431449646615	0\\
26.4427312819221	0\\
26.4423175794792	0\\
26.441903857331	0\\
26.4414901154757	0\\
26.4410763539113	0\\
26.4406625726361	0\\
26.440248771648	0\\
26.4398349509453	0\\
26.439421110526	0\\
26.4390072503883	0\\
26.4385933705304	0\\
26.4381794709502	0\\
26.4377655516459	0\\
26.4373516126158	0\\
26.4369376538578	0\\
26.4365236753701	0\\
26.4361096771508	0\\
26.435695659198	0\\
26.4352816215099	0\\
26.4348675640846	0\\
26.4344534869202	0\\
26.4340393900148	0\\
26.4336252733665	0\\
26.4332111369734	0\\
26.4327969808337	0\\
26.4323828049455	0\\
26.4319686093069	0\\
26.431554393916	0\\
26.4311401587709	0\\
26.4307259038697	0\\
26.4303116292107	0\\
26.4298973347917	0\\
26.4294830206111	0\\
26.4290686866669	0\\
26.4286543329572	0\\
26.4282399594801	0\\
26.4278255662338	0\\
26.4274111532163	0\\
26.4269967204258	0\\
26.4265822678604	0\\
26.4261677955182	0\\
26.4257533033973	0\\
26.4253387914958	0\\
26.4249242598118	0\\
26.4245097083435	0\\
26.424095137089	0\\
26.4236805460463	0\\
26.4232659352136	0\\
26.422851304589	0\\
26.4224366541706	0\\
26.4220219839564	0\\
26.4216072939448	0\\
26.4211925841336	0\\
26.4207778545211	0\\
26.4203631051053	0\\
26.4199483358844	0\\
26.4195335468565	0\\
26.4191187380196	0\\
26.4187039093719	0\\
26.4182890609115	0\\
26.4178741926365	0\\
26.417459304545	0\\
26.4170443966351	0\\
26.4166294689049	0\\
26.4162145213526	0\\
26.4157995539761	0\\
26.4153845667737	0\\
26.4149695597435	0\\
26.4145545328835	0\\
26.4141394861918	0\\
26.4137244196666	0\\
26.4133093333059	0\\
26.412894227108	0\\
26.4124791010708	0\\
26.4120639551924	0\\
26.411648789471	0\\
26.4112336039047	0\\
26.4108183984916	0\\
26.4104031732298	0\\
26.4099879281173	0\\
26.4095726631524	0\\
26.409157378333	0\\
26.4087420736573	0\\
26.4083267491234	0\\
26.4079114047294	0\\
26.4074960404734	0\\
26.4070806563534	0\\
26.4066652523677	0\\
26.4062498285143	0\\
26.4058343847912	0\\
26.4054189211966	0\\
26.4050034377287	0\\
26.4045879343854	0\\
26.4041724111649	0\\
26.4037568680653	0\\
26.4033413050846	0\\
26.4029257222211	0\\
26.4025101194727	0\\
26.4020944968376	0\\
26.4016788543139	0\\
26.4012631918996	0\\
26.4008475095929	0\\
26.4004318073918	0\\
26.4000160852946	0\\
26.3996003432991	0\\
26.3991845814036	0\\
26.3987687996062	0\\
26.3983529979049	0\\
26.3979371762978	0\\
26.397521334783	0\\
26.3971054733586	0\\
26.3966895920228	0\\
26.3962736907736	0\\
26.395857769609	0\\
26.3954418285273	0\\
26.3950258675264	0\\
26.3946098866045	0\\
26.3941938857597	0\\
26.39377786499	0\\
26.3933618242935	0\\
26.3929457636685	0\\
26.3925296831128	0\\
26.3921135826247	0\\
26.3916974622021	0\\
26.3912813218433	0\\
26.3908651615462	0\\
26.3904489813091	0\\
26.3900327811299	0\\
26.3896165610067	0\\
26.3892003209377	0\\
26.388784060921	0\\
26.3883677809546	0\\
26.3879514810365	0\\
26.387535161165	0\\
26.387118821338	0\\
26.3867024615538	0\\
26.3862860818102	0\\
26.3858696821056	0\\
26.3854532624378	0\\
26.3850368228051	0\\
26.3846203632054	0\\
26.384203883637	0\\
26.3837873840978	0\\
26.383370864586	0\\
26.3829543250996	0\\
26.3825377656368	0\\
26.3821211861955	0\\
26.381704586774	0\\
26.3812879673702	0\\
26.3808713279823	0\\
26.3804546686083	0\\
26.3800379892463	0\\
26.3796212898945	0\\
26.3792045705508	0\\
26.3787878312134	0\\
26.3783710718804	0\\
26.3779542925497	0\\
26.3775374932196	0\\
26.3771206738881	0\\
26.3767038345533	0\\
26.3762869752132	0\\
26.3758700958659	0\\
26.3754531965096	0\\
26.3750362771422	0\\
26.374619337762	0\\
26.3742023783668	0\\
26.3737853989549	0\\
26.3733683995243	0\\
26.3729513800731	0\\
26.3725343405993	0\\
26.3721172811011	0\\
26.3717002015765	0\\
26.3712831020236	0\\
26.3708659824404	0\\
26.3704488428251	0\\
26.3700316831757	0\\
26.3696145034903	0\\
26.369197303767	0\\
26.3687800840038	0\\
26.3683628441989	0\\
26.3679455843502	0\\
26.3675283044559	0\\
26.3671110045141	0\\
26.3666936845227	0\\
26.36627634448	0\\
26.3658589843839	0\\
26.3654416042325	0\\
26.365024204024	0\\
26.3646067837563	0\\
26.3641893434276	0\\
26.3637718830359	0\\
26.3633544025793	0\\
26.3629369020559	0\\
26.3625193814637	0\\
26.3621018408008	0\\
26.3616842800653	0\\
26.3612666992552	0\\
26.3608490983687	0\\
26.3604314774037	0\\
26.3600138363584	0\\
26.3595961752308	0\\
26.359178494019	0\\
26.3587607927211	0\\
26.3583430713351	0\\
26.3579253298591	0\\
26.3575075682912	0\\
26.3570897866294	0\\
26.3566719848718	0\\
26.3562541630164	0\\
26.3558363210614	0\\
26.3554184590048	0\\
26.3550005768447	0\\
26.3545826745791	0\\
26.354164752206	0\\
26.3537468097237	0\\
26.35332884713	0\\
26.3529108644232	0\\
26.3524928616012	0\\
26.3520748386621	0\\
26.351656795604	0\\
26.3512387324249	0\\
26.350820649123	0\\
26.3504025456962	0\\
26.3499844221426	0\\
26.3495662784604	0\\
26.3491481146475	0\\
26.348729930702	0\\
26.348311726622	0\\
26.3478935024055	0\\
26.3474752580506	0\\
26.3470569935554	0\\
26.3466387089179	0\\
26.3462204041362	0\\
26.3458020792084	0\\
26.3453837341324	0\\
26.3449653689064	0\\
26.3445469835284	0\\
26.3441285779965	0\\
26.3437101523087	0\\
26.3432917064631	0\\
26.3428732404578	0\\
26.3424547542908	0\\
26.3420362479601	0\\
26.3416177214639	0\\
26.3411991748001	0\\
26.3407806079669	0\\
26.3403620209622	0\\
26.3399434137842	0\\
26.3395247864309	0\\
26.3391061389004	0\\
26.3386874711907	0\\
26.3382687832998	0\\
26.3378500752259	0\\
26.3374313469669	0\\
26.337012598521	0\\
26.3365938298862	0\\
26.3361750410604	0\\
26.3357562320419	0\\
26.3353374028286	0\\
26.3349185534186	0\\
26.33449968381	0\\
26.3340807940007	0\\
26.3336618839889	0\\
26.3332429537726	0\\
26.3328240033499	0\\
26.3324050327187	0\\
26.3319860418772	0\\
26.3315670308234	0\\
26.3311479995554	0\\
26.3307289480711	0\\
26.3303098763687	0\\
26.3298907844462	0\\
26.3294716723017	0\\
26.3290525399331	0\\
26.3286333873386	0\\
26.3282142145162	0\\
26.3277950214639	0\\
26.3273758081798	0\\
26.326956574662	0\\
26.3265373209084	0\\
26.3261180469171	0\\
26.3256987526863	0\\
26.3252794382138	0\\
26.3248601034978	0\\
26.3244407485363	0\\
26.3240213733274	0\\
26.3236019778691	0\\
26.3231825621594	0\\
26.3227631261964	0\\
26.3223436699781	0\\
26.3219241935026	0\\
26.3215046967679	0\\
26.3210851797721	0\\
26.3206656425132	0\\
26.3202460849893	0\\
26.3198265071983	0\\
26.3194069091384	0\\
26.3189872908075	0\\
26.3185676522037	0\\
26.3181479933251	0\\
26.3177283141697	0\\
26.3173086147356	0\\
26.3168888950207	0\\
26.3164691550231	0\\
26.3160493947409	0\\
26.3156296141721	0\\
26.3152098133147	0\\
26.3147899921668	0\\
26.3143701507264	0\\
26.3139502889915	0\\
26.3135304069603	0\\
26.3131105046306	0\\
26.3126905820006	0\\
26.3122706390684	0\\
26.3118506758318	0\\
26.3114306922891	0\\
26.3110106884381	0\\
26.310590664277	0\\
26.3101706198037	0\\
26.3097505550164	0\\
26.3093304699131	0\\
26.3089103644917	0\\
26.3084902387503	0\\
26.308070092687	0\\
26.3076499262998	0\\
26.3072297395867	0\\
26.3068095325458	0\\
26.306389305175	0\\
26.3059690574725	0\\
26.3055487894362	0\\
26.3051285010642	0\\
26.3047081923545	0\\
26.3042878633052	0\\
26.3038675139143	0\\
26.3034471441797	0\\
26.3030267540997	0\\
26.302606343672	0\\
26.3021859128949	0\\
26.3017654617663	0\\
26.3013449902843	0\\
26.3009244984469	0\\
26.3005039862521	0\\
26.3000834536979	0\\
26.2996629007825	0\\
26.2992423275037	0\\
26.2988217338597	0\\
26.2984011198484	0\\
26.2979804854679	0\\
26.2975598307162	0\\
26.2971391555914	0\\
26.2967184600915	0\\
26.2962977442144	0\\
26.2958770079583	0\\
26.2954562513211	0\\
26.2950354743009	0\\
26.2946146768957	0\\
26.2941938591035	0\\
26.2937730209223	0\\
26.2933521623503	0\\
26.2929312833853	0\\
26.2925103840254	0\\
26.2920894642687	0\\
26.2916685241131	0\\
26.2912475635568	0\\
26.2908265825976	0\\
26.2904055812337	0\\
26.289984559463	0\\
26.2895635172836	0\\
26.2891424546935	0\\
26.2887213716908	0\\
26.2883002682733	0\\
26.2878791444393	0\\
26.2874580001866	0\\
26.2870368355133	0\\
26.2866156504174	0\\
26.2861944448969	0\\
26.28577321895	0\\
26.2853519725744	0\\
26.2849307057684	0\\
26.2845094185299	0\\
26.284088110857	0\\
26.2836667827475	0\\
26.2832454341997	0\\
26.2828240652114	0\\
26.2824026757807	0\\
26.2819812659057	0\\
26.2815598355843	0\\
26.2811383848145	0\\
26.2807169135944	0\\
26.2802954219219	0\\
26.2798739097952	0\\
26.2794523772121	0\\
26.2790308241708	0\\
26.2786092506692	0\\
26.2781876567054	0\\
26.2777660422774	0\\
26.2773444073831	0\\
26.2769227520206	0\\
26.2765010761879	0\\
26.276079379883	0\\
26.2756576631039	0\\
26.2752359258487	0\\
26.2748141681154	0\\
26.2743923899019	0\\
26.2739705912062	0\\
26.2735487720265	0\\
26.2731269323606	0\\
26.2727050722067	0\\
26.2722831915626	0\\
26.2718612904265	0\\
26.2714393687963	0\\
26.2710174266701	0\\
26.2705954640458	0\\
26.2701734809215	0\\
26.2697514772951	0\\
26.2693294531647	0\\
26.2689074085283	0\\
26.2684853433839	0\\
26.2680632577294	0\\
26.267641151563	0\\
26.2672190248826	0\\
26.2667968776861	0\\
26.2663747099718	0\\
26.2659525217374	0\\
26.265530312981	0\\
26.2651080837007	0\\
26.2646858338945	0\\
26.2642635635603	0\\
26.2638412726961	0\\
26.2634189613	0\\
26.2629966293699	0\\
26.2625742769039	0\\
26.2621519038999	0\\
26.261729510356	0\\
26.2613070962702	0\\
26.2608846616405	0\\
26.2604622064648	0\\
26.2600397307411	0\\
26.2596172344676	0\\
26.2591947176421	0\\
26.2587721802627	0\\
26.2583496223274	0\\
26.2579270438341	0\\
26.2575044447809	0\\
26.2570818251658	0\\
26.2566591849867	0\\
26.2562365242418	0\\
26.2558138429288	0\\
26.255391141046	0\\
26.2549684185912	0\\
26.2545456755625	0\\
26.2541229119578	0\\
26.2537001277752	0\\
26.2532773230126	0\\
26.2528544976681	0\\
26.2524316517396	0\\
26.2520087852252	0\\
26.2515858981228	0\\
26.2511629904304	0\\
26.2507400621461	0\\
26.2503171132677	0\\
26.2498941437934	0\\
26.2494711537211	0\\
26.2490481430488	0\\
26.2486251117745	0\\
26.2482020598962	0\\
26.2477789874118	0\\
26.2473558943195	0\\
26.2469327806171	0\\
26.2465096463026	0\\
26.2460864913741	0\\
26.2456633158296	0\\
26.2452401196669	0\\
26.2448169028842	0\\
26.2443936654794	0\\
26.2439704074506	0\\
26.2435471287956	0\\
26.2431238295125	0\\
26.2427005095992	0\\
26.2422771690538	0\\
26.2418538078743	0\\
26.2414304260586	0\\
26.2410070236048	0\\
26.2405836005107	0\\
26.2401601567745	0\\
26.239736692394	0\\
26.2393132073673	0\\
26.2388897016924	0\\
26.2384661753672	0\\
26.2380426283898	0\\
26.2376190607581	0\\
26.2371954724701	0\\
26.2367718635237	0\\
26.2363482339171	0\\
26.2359245836481	0\\
26.2355009127148	0\\
26.2350772211151	0\\
26.234653508847	0\\
26.2342297759085	0\\
26.2338060222976	0\\
26.2333822480122	0\\
26.2329584530504	0\\
26.2325346374101	0\\
26.2321108010894	0\\
26.2316869440861	0\\
26.2312630663983	0\\
26.2308391680239	0\\
26.230415248961	0\\
26.2299913092075	0\\
26.2295673487613	0\\
26.2291433676206	0\\
26.2287193657832	0\\
26.2282953432472	0\\
26.2278713000104	0\\
26.227447236071	0\\
26.2270231514268	0\\
26.2265990460759	0\\
26.2261749200162	0\\
26.2257507732457	0\\
26.2253266057623	0\\
26.2249024175642	0\\
26.2244782086491	0\\
26.2240539790152	0\\
26.2236297286604	0\\
26.2232054575826	0\\
26.2227811657798	0\\
26.2223568532501	0\\
26.2219325199914	0\\
26.2215081660016	0\\
26.2210837912787	0\\
26.2206593958208	0\\
26.2202349796258	0\\
26.2198105426916	0\\
26.2193860850162	0\\
26.2189616065977	0\\
26.2185371074339	0\\
26.2181125875228	0\\
26.2176880468625	0\\
26.2172634854509	0\\
26.2168389032859	0\\
26.2164143003656	0\\
26.2159896766879	0\\
26.2155650322507	0\\
26.2151403670521	0\\
26.21471568109	0\\
26.2142909743624	0\\
26.2138662468672	0\\
26.2134414986025	0\\
26.2130167295661	0\\
26.2125919397561	0\\
26.2121671291705	0\\
26.2117422978071	0\\
26.211317445664	0\\
26.2108925727391	0\\
26.2104676790304	0\\
26.2100427645359	0\\
26.2096178292535	0\\
26.2091928731812	0\\
26.208767896317	0\\
26.2083428986587	0\\
26.2079178802045	0\\
26.2074928409522	0\\
26.2070677808999	0\\
26.2066427000454	0\\
26.2062175983868	0\\
26.205792475922	0\\
26.205367332649	0\\
26.2049421685657	0\\
26.20451698367	0\\
26.2040917779601	0\\
26.2036665514338	0\\
26.2032413040891	0\\
26.2028160359239	0\\
26.2023907469362	0\\
26.201965437124	0\\
26.2015401064852	0\\
26.2011147550178	0\\
26.2006893827198	0\\
26.2002639895891	0\\
26.1998385756236	0\\
26.1994131408214	0\\
26.1989876851803	0\\
26.1985622086984	0\\
26.1981367113736	0\\
26.1977111932039	0\\
26.1972856541872	0\\
26.1968600943214	0\\
26.1964345136046	0\\
26.1960089120347	0\\
26.1955832896096	0\\
26.1951576463273	0\\
26.1947319821858	0\\
26.194306297183	0\\
26.1938805913168	0\\
26.1934548645853	0\\
26.1930291169864	0\\
26.192603348518	0\\
26.1921775591781	0\\
26.1917517489646	0\\
26.1913259178755	0\\
26.1909000659087	0\\
26.1904741930623	0\\
26.1900482993341	0\\
26.1896223847221	0\\
26.1891964492242	0\\
26.1887704928385	0\\
26.1883445155628	0\\
26.1879185173951	0\\
26.1874924983334	0\\
26.1870664583756	0\\
26.1866403975197	0\\
26.1862143157635	0\\
26.1857882131051	0\\
26.1853620895425	0\\
26.1849359450734	0\\
26.184509779696	0\\
26.1840835934082	0\\
26.1836573862078	0\\
26.1832311580929	0\\
26.1828049090614	0\\
26.1823786391112	0\\
26.1819523482403	0\\
26.1815260364466	0\\
26.1810997037282	0\\
26.1806733500828	0\\
26.1802469755085	0\\
26.1798205800033	0\\
26.179394163565	0\\
26.1789677261916	0\\
26.1785412678811	0\\
26.1781147886313	0\\
26.1776882884403	0\\
26.177261767306	0\\
26.1768352252263	0\\
26.1764086621992	0\\
26.1759820782226	0\\
26.1755554732944	0\\
26.1751288474127	0\\
26.1747022005752	0\\
26.1742755327801	0\\
26.1738488440251	0\\
26.1734221343084	0\\
26.1729954036277	0\\
26.1725686519811	0\\
26.1721418793664	0\\
26.1717150857817	0\\
26.1712882712248	0\\
26.1708614356937	0\\
26.1704345791863	0\\
26.1700077017007	0\\
26.1695808032346	0\\
26.1691538837861	0\\
26.1687269433531	0\\
26.1682999819334	0\\
26.1678729995252	0\\
26.1674459961262	0\\
26.1670189717345	0\\
26.166591926348	0\\
26.1661648599645	0\\
26.1657377725821	0\\
26.1653106641987	0\\
26.1648835348121	0\\
26.1644563844204	0\\
26.1640292130215	0\\
26.1636020206133	0\\
26.1631748071938	0\\
26.1627475727608	0\\
26.1623203173123	0\\
26.1618930408463	0\\
26.1614657433606	0\\
26.1610384248532	0\\
26.1606110853221	0\\
26.1601837247652	0\\
26.1597563431803	0\\
26.1593289405655	0\\
26.1589015169186	0\\
26.1584740722376	0\\
26.1580466065205	0\\
26.1576191197651	0\\
26.1571916119693	0\\
26.1567640831312	0\\
26.1563365332486	0\\
26.1559089623195	0\\
26.1554813703417	0\\
26.1550537573133	0\\
26.1546261232321	0\\
26.1541984680961	0\\
26.1537707919032	0\\
26.1533430946513	0\\
26.1529153763384	0\\
26.1524876369624	0\\
26.1520598765211	0\\
26.1516320950126	0\\
26.1512042924347	0\\
26.1507764687854	0\\
26.1503486240626	0\\
26.1499207582643	0\\
26.1494928713882	0\\
26.1490649634325	0\\
26.1486370343949	0\\
26.1482090842735	0\\
26.1477811130661	0\\
26.1473531207706	0\\
26.1469251073851	0\\
26.1464970729073	0\\
26.1460690173353	0\\
26.1456409406669	0\\
26.1452128429001	0\\
26.1447847240328	0\\
26.1443565840628	0\\
26.1439284229883	0\\
26.1435002408069	0\\
26.1430720375167	0\\
26.1426438131156	0\\
26.1422155676015	0\\
26.1417873009723	0\\
26.141359013226	0\\
26.1409307043604	0\\
26.1405023743734	0\\
26.1400740232631	0\\
26.1396456510273	0\\
26.1392172576639	0\\
26.1387888431708	0\\
26.138360407546	0\\
26.1379319507873	0\\
26.1375034728928	0\\
26.1370749738602	0\\
26.1366464536875	0\\
26.1362179123727	0\\
26.1357893499136	0\\
26.1353607663081	0\\
26.1349321615542	0\\
26.1345035356498	0\\
26.1340748885928	0\\
26.1336462203811	0\\
26.1332175310126	0\\
26.1327888204852	0\\
26.1323600887969	0\\
26.1319313359455	0\\
26.131502561929	0\\
26.1310737667452	0\\
26.1306449503921	0\\
26.1302161128676	0\\
26.1297872541697	0\\
26.1293583742961	0\\
26.1289294732448	0\\
26.1285005510138	0\\
26.1280716076008	0\\
26.127642643004	0\\
26.127213657221	0\\
26.12678465025	0\\
26.1263556220886	0\\
26.125926572735	0\\
26.1254975021869	0\\
26.1250684104423	0\\
26.1246392974991	0\\
26.1242101633552	0\\
26.1237810080085	0\\
26.1233518314569	0\\
26.1229226336983	0\\
26.1224934147306	0\\
26.1220641745517	0\\
26.1216349131595	0\\
26.121205630552	0\\
26.120776326727	0\\
26.1203470016824	0\\
26.1199176554161	0\\
26.1194882879261	0\\
26.1190588992101	0\\
26.1186294892663	0\\
26.1182000580923	0\\
26.1177706056862	0\\
26.1173411320458	0\\
26.1169116371691	0\\
26.1164821210539	0\\
26.1160525836981	0\\
26.1156230250996	0\\
26.1151934452564	0\\
26.1147638441664	0\\
26.1143342218273	0\\
26.1139045782372	0\\
26.1134749133939	0\\
26.1130452272953	0\\
26.1126155199393	0\\
26.1121857913239	0\\
26.1117560414468	0\\
26.1113262703061	0\\
26.1108964778995	0\\
26.1104666642251	0\\
26.1100368292807	0\\
26.1096069730641	0\\
26.1091770955733	0\\
26.1087471968062	0\\
26.1083172767607	0\\
26.1078873354346	0\\
26.1074573728259	0\\
26.1070273889325	0\\
26.1065973837522	0\\
26.1061673572829	0\\
26.1057373095225	0\\
26.105307240469	0\\
26.1048771501201	0\\
26.1044470384739	0\\
26.1040169055282	0\\
26.1035867512808	0\\
26.1031565757297	0\\
26.1027263788728	0\\
26.1022961607079	0\\
26.101865921233	0\\
26.1014356604459	0\\
26.1010053783445	0\\
26.1005750749267	0\\
26.1001447501904	0\\
26.0997144041334	0\\
26.0992840367538	0\\
26.0988536480493	0\\
26.0984232380178	0\\
26.0979928066573	0\\
26.0975623539656	0\\
26.0971318799406	0\\
26.0967013845801	0\\
26.0962708678822	0\\
26.0958403298445	0\\
26.0954097704652	0\\
26.0949791897419	0\\
26.0945485876727	0\\
26.0941179642553	0\\
26.0936873194877	0\\
26.0932566533678	0\\
26.0928259658934	0\\
26.0923952570624	0\\
26.0919645268727	0\\
26.0915337753223	0\\
26.0911030024088	0\\
26.0906722081304	0\\
26.0902413924847	0\\
26.0898105554698	0\\
26.0893796970834	0\\
26.0889488173235	0\\
26.0885179161879	0\\
26.0880869936746	0\\
26.0876560497814	0\\
26.0872250845062	0\\
26.0867940978468	0\\
26.0863630898011	0\\
26.0859320603671	0\\
26.0855010095426	0\\
26.0850699373254	0\\
26.0846388437135	0\\
26.0842077287047	0\\
26.0837765922969	0\\
26.0833454344879	0\\
26.0829142552757	0\\
26.0824830546582	0\\
26.0820518326331	0\\
26.0816205891984	0\\
26.081189324352	0\\
26.0807580380917	0\\
26.0803267304153	0\\
26.0798954013209	0\\
26.0794640508061	0\\
26.079032678869	0\\
26.0786012855074	0\\
26.0781698707191	0\\
26.077738434502	0\\
26.0773069768541	0\\
26.0768754977731	0\\
26.0764439972569	0\\
26.0760124753035	0\\
26.0755809319106	0\\
26.0751493670762	0\\
26.0747177807981	0\\
26.0742861730742	0\\
26.0738545439024	0\\
26.0734228932804	0\\
26.0729912212063	0\\
26.0725595276778	0\\
26.0721278126928	0\\
26.0716960762493	0\\
26.071264318345	0\\
26.0708325389778	0\\
26.0704007381456	0\\
26.0699689158463	0\\
26.0695370720777	0\\
26.0691052068377	0\\
26.0686733201241	0\\
26.0682414119348	0\\
26.0678094822678	0\\
26.0673775311208	0\\
26.0669455584916	0\\
26.0665135643783	0\\
26.0660815487786	0\\
26.0656495116904	0\\
26.0652174531115	0\\
26.0647853730399	0\\
26.0643532714734	0\\
26.0639211484098	0\\
26.063489003847	0\\
26.0630568377829	0\\
26.0626246502153	0\\
26.062192441142	0\\
26.0617602105611	0\\
26.0613279584702	0\\
26.0608956848673	0\\
26.0604633897502	0\\
26.0600310731168	0\\
26.059598734965	0\\
26.0591663752925	0\\
26.0587339940973	0\\
26.0583015913772	0\\
26.0578691671301	0\\
26.0574367213538	0\\
26.0570042540462	0\\
26.0565717652051	0\\
26.0561392548284	0\\
26.055706722914	0\\
26.0552741694596	0\\
26.0548415944633	0\\
26.0544089979227	0\\
26.0539763798358	0\\
26.0535437402004	0\\
26.0531110790144	0\\
26.0526783962756	0\\
26.0522456919819	0\\
26.0518129661311	0\\
26.0513802187211	0\\
26.0509474497497	0\\
26.0505146592148	0\\
26.0500818471143	0\\
26.0496490134459	0\\
26.0492161582076	0\\
26.0487832813971	0\\
26.0483503830124	0\\
26.0479174630513	0\\
26.0474845215116	0\\
26.0470515583913	0\\
26.046618573688	0\\
26.0461855673997	0\\
26.0457525395242	0\\
26.0453194900595	0\\
26.0448864190032	0\\
26.0444533263533	0\\
26.0440202121076	0\\
26.043587076264	0\\
26.0431539188203	0\\
26.0427207397744	0\\
26.042287539124	0\\
26.0418543168671	0\\
26.0414210730015	0\\
26.040987807525	0\\
26.0405545204355	0\\
26.0401212117308	0\\
26.0396878814088	0\\
26.0392545294673	0\\
26.0388211559041	0\\
26.0383877607171	0\\
26.0379543439042	0\\
26.0375209054631	0\\
26.0370874453917	0\\
26.0366539636879	0\\
26.0362204603495	0\\
26.0357869353744	0\\
26.0353533887603	0\\
26.0349198205051	0\\
26.0344862306067	0\\
26.0340526190629	0\\
26.0336189858715	0\\
26.0331853310304	0\\
26.0327516545374	0\\
26.0323179563904	0\\
26.0318842365871	0\\
26.0314504951255	0\\
26.0310167320033	0\\
26.0305829472185	0\\
26.0301491407687	0\\
26.029715312652	0\\
26.029281462866	0\\
26.0288475914087	0\\
26.0284136982778	0\\
26.0279797834713	0\\
26.0275458469869	0\\
26.0271118888225	0\\
26.0266779089759	0\\
26.0262439074449	0\\
26.0258098842274	0\\
26.0253758393213	0\\
26.0249417727242	0\\
26.0245076844342	0\\
26.0240735744489	0\\
26.0236394427663	0\\
26.0232052893842	0\\
26.0227711143003	0\\
26.0223369175126	0\\
26.0219026990188	0\\
26.0214684588169	0\\
26.0210341969045	0\\
26.0205999132796	0\\
26.02016560794	0\\
26.0197312808835	0\\
26.0192969321079	0\\
26.0188625616111	0\\
26.0184281693908	0\\
26.017993755445	0\\
26.0175593197714	0\\
26.0171248623679	0\\
26.0166903832322	0\\
26.0162558823623	0\\
26.015821359756	0\\
26.015386815411	0\\
26.0149522493252	0\\
26.0145176614964	0\\
26.0140830519225	0\\
26.0136484206012	0\\
26.0132137675304	0\\
26.012779092708	0\\
26.0123443961316	0\\
26.0119096777993	0\\
26.0114749377087	0\\
26.0110401758577	0\\
26.0106053922441	0\\
26.0101705868658	0\\
26.0097357597206	0\\
26.0093009108062	0\\
26.0088660401206	0\\
26.0084311476614	0\\
26.0079962334267	0\\
26.0075612974141	0\\
26.0071263396214	0\\
26.0066913600466	0\\
26.0062563586874	0\\
26.0058213355416	0\\
26.0053862906071	0\\
26.0049512238817	0\\
26.0045161353632	0\\
26.0040810250493	0\\
26.003645892938	0\\
26.003210739027	0\\
26.0027755633142	0\\
26.0023403657974	0\\
26.0019051464743	0\\
26.0014699053429	0\\
26.0010346424008	0\\
26.000599357646	0\\
26.0001640510763	0\\
25.9997287226893	0\\
25.9992933724831	0\\
25.9988580004553	0\\
25.9984226066039	0\\
25.9979871909265	0\\
25.9975517534211	0\\
25.9971162940853	0\\
25.9966808129172	0\\
25.9962453099143	0\\
25.9958097850747	0\\
25.995374238396	0\\
25.9949386698761	0\\
25.9945030795127	0\\
25.9940674673038	0\\
25.9936318332471	0\\
25.9931961773404	0\\
25.9927604995815	0\\
25.9923247999683	0\\
25.9918890784985	0\\
25.9914533351699	0\\
25.9910175699804	0\\
25.9905817829277	0\\
25.9901459740098	0\\
25.9897101432242	0\\
25.989274290569	0\\
25.9888384160418	0\\
25.9884025196406	0\\
25.987966601363	0\\
25.9875306612069	0\\
25.9870946991701	0\\
25.9866587152504	0\\
25.9862227094457	0\\
25.9857866817536	0\\
25.985350632172	0\\
25.9849145606988	0\\
25.9844784673317	0\\
25.9840423520684	0\\
25.9836062149069	0\\
25.983170055845	0\\
25.9827338748803	0\\
25.9822976720107	0\\
25.9818614472341	0\\
25.9814252005482	0\\
25.9809889319508	0\\
25.9805526414397	0\\
25.9801163290127	0\\
25.9796799946677	0\\
25.9792436384023	0\\
25.9788072602145	0\\
25.978370860102	0\\
25.9779344380625	0\\
25.977497994094	0\\
25.9770615281941	0\\
25.9766250403608	0\\
25.9761885305917	0\\
25.9757519988847	0\\
25.9753154452376	0\\
25.9748788696481	0\\
25.9744422721142	0\\
25.9740056526334	0\\
25.9735690112037	0\\
25.9731323478229	0\\
25.9726956624887	0\\
25.9722589551989	0\\
25.9718222259514	0\\
25.9713854747439	0\\
25.9709487015741	0\\
25.97051190644	0\\
25.9700750893392	0\\
25.9696382502697	0\\
25.969201389229	0\\
25.9687645062152	0\\
25.9683276012259	0\\
25.9678906742589	0\\
25.9674537253121	0\\
25.9670167543831	0\\
25.9665797614699	0\\
25.9661427465702	0\\
25.9657057096817	0\\
25.9652686508023	0\\
25.9648315699297	0\\
25.9643944670618	0\\
25.9639573421963	0\\
25.963520195331	0\\
25.9630830264637	0\\
25.9626458355922	0\\
25.9622086227142	0\\
25.9617713878276	0\\
25.9613341309302	0\\
25.9608968520196	0\\
25.9604595510937	0\\
25.9600222281504	0\\
25.9595848831872	0\\
25.9591475162022	0\\
25.9587101271929	0\\
25.9582727161573	0\\
25.9578352830931	0\\
25.957397827998	0\\
25.9569603508699	0\\
25.9565228517066	0\\
25.9560853305057	0\\
25.9556477872652	0\\
25.9552102219827	0\\
25.954772634656	0\\
25.9543350252831	0\\
25.9538973938615	0\\
25.9534597403891	0\\
25.9530220648636	0\\
25.952584367283	0\\
25.9521466476448	0\\
25.9517089059469	0\\
25.9512711421871	0\\
25.9508333563632	0\\
25.9503955484729	0\\
25.949957718514	0\\
25.9495198664842	0\\
25.9490819923814	0\\
25.9486440962034	0\\
25.9482061779478	0\\
25.9477682376125	0\\
25.9473302751953	0\\
25.9468922906939	0\\
25.946454284106	0\\
25.9460162554296	0\\
25.9455782046623	0\\
25.9451401318018	0\\
25.9447020368461	0\\
25.9442639197928	0\\
25.9438257806398	0\\
25.9433876193847	0\\
25.9429494360254	0\\
25.9425112305596	0\\
25.9420730029851	0\\
25.9416347532997	0\\
25.9411964815012	0\\
25.9407581875872	0\\
25.9403198715556	0\\
25.9398815334042	0\\
25.9394431731306	0\\
25.9390047907328	0\\
25.9385663862084	0\\
25.9381279595552	0\\
25.9376895107709	0\\
25.9372510398535	0\\
25.9368125468005	0\\
25.9363740316098	0\\
25.9359354942791	0\\
25.9354969348062	0\\
25.9350583531889	0\\
25.934619749425	0\\
25.9341811235121	0\\
25.933742475448	0\\
25.9333038052306	0\\
25.9328651128576	0\\
25.9324263983267	0\\
25.9319876616357	0\\
25.9315489027824	0\\
25.9311101217645	0\\
25.9306713185798	0\\
25.930232493226	0\\
25.9297936457009	0\\
25.9293547760023	0\\
25.9289158841279	0\\
25.9284769700755	0\\
25.9280380338429	0\\
25.9275990754277	0\\
25.9271600948278	0\\
25.9267210920409	0\\
25.9262820670648	0\\
25.9258430198972	0\\
25.9254039505359	0\\
25.9249648589787	0\\
25.9245257452232	0\\
25.9240866092673	0\\
25.9236474511087	0\\
25.9232082707452	0\\
25.9227690681744	0\\
25.9223298433943	0\\
25.9218905964025	0\\
25.9214513271967	0\\
25.9210120357748	0\\
25.9205727221344	0\\
25.9201333862734	0\\
25.9196940281895	0\\
25.9192546478805	0\\
25.918815245344	0\\
25.9183758205778	0\\
25.9179363735798	0\\
25.9174969043476	0\\
25.917057412879	0\\
25.9166178991717	0\\
25.9161783632235	0\\
25.9157388050322	0\\
25.9152992245954	0\\
25.914859621911	0\\
25.9144199969767	0\\
25.9139803497902	0\\
25.9135406803493	0\\
25.9131009886517	0\\
25.9126612746952	0\\
25.9122215384775	0\\
25.9117817799964	0\\
25.9113419992496	0\\
25.9109021962348	0\\
25.9104623709499	0\\
25.9100225233925	0\\
25.9095826535603	0\\
25.9091427614513	0\\
25.908702847063	0\\
25.9082629103932	0\\
25.9078229514396	0\\
25.9073829702001	0\\
25.9069429666724	0\\
25.9065029408541	0\\
25.906062892743	0\\
25.905622822337	0\\
25.9051827296336	0\\
25.9047426146307	0\\
25.904302477326	0\\
25.9038623177172	0\\
25.9034221358021	0\\
25.9029819315784	0\\
25.9025417050438	0\\
25.9021014561962	0\\
25.9016611850331	0\\
25.9012208915525	0\\
25.9007805757519	0\\
25.9003402376292	0\\
25.899899877182	0\\
25.8994594944082	0\\
25.8990190893054	0\\
25.8985786618713	0\\
25.8981382121038	0\\
25.8976977400006	0\\
25.8972572455593	0\\
25.8968167287778	0\\
25.8963761896537	0\\
25.8959356281848	0\\
25.8954950443689	0\\
25.8950544382036	0\\
25.8946138096866	0\\
25.8941731588158	0\\
25.8937324855889	0\\
25.8932917900035	0\\
25.8928510720574	0\\
25.8924103317484	0\\
25.8919695690742	0\\
25.8915287840324	0\\
25.8910879766209	0\\
25.8906471468374	0\\
25.8902062946796	0\\
25.8897654201451	0\\
25.8893245232319	0\\
25.8888836039375	0\\
25.8884426622597	0\\
25.8880016981962	0\\
25.8875607117448	0\\
25.8871197029032	0\\
25.8866786716691	0\\
25.8862376180403	0\\
25.8857965420144	0\\
25.8853554435892	0\\
25.8849143227624	0\\
25.8844731795317	0\\
25.8840320138949	0\\
25.8835908258497	0\\
25.8831496153938	0\\
25.882708382525	0\\
25.8822671272409	0\\
25.8818258495392	0\\
25.8813845494178	0\\
25.8809432268743	0\\
25.8805018819065	0\\
25.880060514512	0\\
25.8796191246886	0\\
25.879177712434	0\\
25.8787362777459	0\\
25.878294820622	0\\
25.8778533410602	0\\
25.877411839058	0\\
25.8769703146132	0\\
25.8765287677236	0\\
25.8760871983867	0\\
25.8756456066005	0\\
25.8752039923625	0\\
25.8747623556705	0\\
25.8743206965222	0\\
25.8738790149153	0\\
25.8734373108476	0\\
25.8729955843167	0\\
25.8725538353204	0\\
25.8721120638564	0\\
25.8716702699223	0\\
25.871228453516	0\\
25.8707866146351	0\\
25.8703447532774	0\\
25.8699028694405	0\\
25.8694609631222	0\\
25.8690190343201	0\\
25.868577083032	0\\
25.8681351092557	0\\
25.8676931129887	0\\
25.8672510942289	0\\
25.8668090529739	0\\
25.8663669892215	0\\
25.8659249029693	0\\
25.8654827942151	0\\
25.8650406629565	0\\
25.8645985091913	0\\
25.8641563329172	0\\
25.863714134132	0\\
25.8632719128332	0\\
25.8628296690186	0\\
25.862387402686	0\\
25.861945113833	0\\
25.8615028024573	0\\
25.8610604685566	0\\
25.8606181121287	0\\
25.8601757331713	0\\
25.859733331682	0\\
25.8592909076586	0\\
25.8588484610987	0\\
25.8584059920001	0\\
25.8579635003605	0\\
25.8575209861776	0\\
25.857078449449	0\\
25.8566358901725	0\\
25.8561933083458	0\\
25.8557507039666	0\\
25.8553080770326	0\\
25.8548654275414	0\\
25.8544227554909	0\\
25.8539800608786	0\\
25.8535373437023	0\\
25.8530946039597	0\\
25.8526518416485	0\\
25.8522090567664	0\\
25.8517662493111	0\\
25.8513234192803	0\\
25.8508805666716	0\\
25.8504376914828	0\\
25.8499947937116	0\\
25.8495518733557	0\\
25.8491089304127	0\\
25.8486659648804	0\\
25.8482229767565	0\\
25.8477799660386	0\\
25.8473369327245	0\\
25.8468938768118	0\\
25.8464507982983	0\\
25.8460076971816	0\\
25.8455645734595	0\\
25.8451214271295	0\\
25.8446782581895	0\\
25.8442350666371	0\\
25.84379185247	0\\
25.8433486156859	0\\
25.8429053562824	0\\
25.8424620742574	0\\
25.8420187696084	0\\
25.8415754423332	0\\
25.8411320924294	0\\
25.8406887198947	0\\
25.8402453247269	0\\
25.8398019069236	0\\
25.8393584664825	0\\
25.8389150034013	0\\
25.8384715176777	0\\
25.8380280093093	0\\
25.8375844782939	0\\
25.8371409246292	0\\
25.8366973483128	0\\
25.8362537493424	0\\
25.8358101277157	0\\
25.8353664834304	0\\
25.8349228164841	0\\
25.8344791268747	0\\
25.8340354145996	0\\
25.8335916796567	0\\
25.8331479220436	0\\
25.832704141758	0\\
25.8322603387976	0\\
25.83181651316	0\\
25.831372664843	0\\
25.8309287938442	0\\
25.8304849001613	0\\
25.830040983792	0\\
25.829597044734	0\\
25.8291530829849	0\\
25.8287090985424	0\\
25.8282650914043	0\\
25.8278210615681	0\\
25.8273770090317	0\\
25.8269329337925	0\\
25.8264888358484	0\\
25.826044715197	0\\
25.825600571836	0\\
25.825156405763	0\\
25.8247122169758	0\\
25.8242680054719	0\\
25.8238237712492	0\\
25.8233795143052	0\\
25.8229352346377	0\\
25.8224909322443	0\\
25.8220466071227	0\\
25.8216022592705	0\\
25.8211578886855	0\\
25.8207134953653	0\\
25.8202690793076	0\\
25.8198246405101	0\\
25.8193801789703	0\\
25.8189356946861	0\\
25.8184911876551	0\\
25.8180466578749	0\\
25.8176021053432	0\\
25.8171575300578	0\\
25.8167129320161	0\\
25.8162683112161	0\\
25.8158236676552	0\\
25.8153790013312	0\\
25.8149343122417	0\\
25.8144896003845	0\\
25.8140448657571	0\\
25.8136001083573	0\\
25.8131553281826	0\\
25.8127105252309	0\\
25.8122656994997	0\\
25.8118208509868	0\\
25.8113759796897	0\\
25.8109310856061	0\\
25.8104861687338	0\\
25.8100412290704	0\\
25.8095962666135	0\\
25.8091512813608	0\\
25.8087062733099	0\\
25.8082612424587	0\\
25.8078161888046	0\\
25.8073711123454	0\\
25.8069260130787	0\\
25.8064808910021	0\\
25.8060357461135	0\\
25.8055905784103	0\\
25.8051453878904	0\\
25.8047001745512	0\\
25.8042549383906	0\\
25.8038096794061	0\\
25.8033643975954	0\\
25.8029190929562	0\\
25.8024737654861	0\\
25.8020284151829	0\\
25.801583042044	0\\
25.8011376460673	0\\
25.8006922272504	0\\
25.8002467855908	0\\
25.7998013210864	0\\
25.7993558337347	0\\
25.7989103235334	0\\
25.7984647904801	0\\
25.7980192345725	0\\
25.7975736558083	0\\
25.7971280541851	0\\
25.7966824297006	0\\
25.7962367823524	0\\
25.7957911121382	0\\
25.7953454190556	0\\
25.7948997031023	0\\
25.794453964276	0\\
25.7940082025742	0\\
25.7935624179946	0\\
25.793116610535	0\\
25.7926707801929	0\\
25.792224926966	0\\
25.7917790508519	0\\
25.7913331518484	0\\
25.7908872299529	0\\
25.7904412851633	0\\
25.7899953174771	0\\
25.789549326892	0\\
25.7891033134057	0\\
25.7886572770157	0\\
25.7882112177198	0\\
25.7877651355155	0\\
25.7873190304006	0\\
25.7868729023727	0\\
25.7864267514294	0\\
25.7859805775683	0\\
25.7855343807872	0\\
25.7850881610836	0\\
25.7846419184552	0\\
25.7841956528997	0\\
25.7837493644147	0\\
25.7833030529978	0\\
25.7828567186467	0\\
25.782410361359	0\\
25.7819639811323	0\\
25.7815175779644	0\\
25.7810711518528	0\\
25.7806247027952	0\\
25.7801782307893	0\\
25.7797317358326	0\\
25.7792852179228	0\\
25.7788386770576	0\\
25.7783921132345	0\\
25.7779455264513	0\\
25.7774989167056	0\\
25.777052283995	0\\
25.7766056283171	0\\
25.7761589496696	0\\
25.7757122480501	0\\
25.7752655234563	0\\
25.7748187758857	0\\
25.7743720053361	0\\
25.7739252118051	0\\
25.7734783952903	0\\
25.7730315557893	0\\
25.7725846932998	0\\
25.7721378078194	0\\
25.7716908993457	0\\
25.7712439678764	0\\
25.7707970134091	0\\
25.7703500359415	0\\
25.7699030354711	0\\
25.7694560119957	0\\
25.7690089655127	0\\
25.76856189602	0\\
25.768114803515	0\\
25.7676676879955	0\\
25.767220549459	0\\
25.7667733879033	0\\
25.7663262033258	0\\
25.7658789957243	0\\
25.7654317650964	0\\
25.7649845114397	0\\
25.7645372347519	0\\
25.7640899350305	0\\
25.7636426122732	0\\
25.7631952664776	0\\
25.7627478976413	0\\
25.7623005057621	0\\
25.7618530908374	0\\
25.761405652865	0\\
25.7609581918424	0\\
25.7605107077673	0\\
25.7600632006372	0\\
25.75961567045	0\\
25.759168117203	0\\
25.758720540894	0\\
25.7582729415207	0\\
25.7578253190805	0\\
25.7573776735712	0\\
25.7569300049904	0\\
25.7564823133356	0\\
25.7560345986045	0\\
25.7555868607948	0\\
25.755139099904	0\\
25.7546913159298	0\\
25.7542435088697	0\\
25.7537956787215	0\\
25.7533478254827	0\\
25.7528999491509	0\\
25.7524520497238	0\\
25.752004127199	0\\
25.7515561815741	0\\
25.7511082128467	0\\
25.7506602210144	0\\
25.7502122060749	0\\
25.7497641680257	0\\
25.7493161068646	0\\
25.748868022589	0\\
25.7484199151966	0\\
25.7479717846851	0\\
25.747523631052	0\\
25.7470754542949	0\\
25.7466272544115	0\\
25.7461790313994	0\\
25.7457307852562	0\\
25.7452825159795	0\\
25.744834223567	0\\
25.7443859080161	0\\
25.7439375693246	0\\
25.74348920749	0\\
25.74304082251	0\\
25.7425924143822	0\\
25.7421439831042	0\\
25.7416955286735	0\\
25.7412470510879	0\\
25.7407985503448	0\\
25.740350026442	0\\
25.739901479377	0\\
25.7394529091475	0\\
25.739004315751	0\\
25.7385556991851	0\\
25.7381070594475	0\\
25.7376583965357	0\\
25.7372097104475	0\\
25.7367610011803	0\\
25.7363122687317	0\\
25.7358635130995	0\\
25.7354147342811	0\\
25.7349659322742	0\\
25.7345171070765	0\\
25.7340682586854	0\\
25.7336193870986	0\\
25.7331704923137	0\\
25.7327215743284	0\\
25.7322726331401	0\\
25.7318236687465	0\\
25.7313746811453	0\\
25.7309256703339	0\\
25.7304766363101	0\\
25.7300275790714	0\\
25.7295784986154	0\\
25.7291293949397	0\\
25.7286802680419	0\\
25.7282311179196	0\\
25.7277819445704	0\\
25.7273327479919	0\\
25.7268835281817	0\\
25.7264342851374	0\\
25.7259850188566	0\\
25.7255357293369	0\\
25.7250864165758	0\\
25.7246370805711	0\\
25.7241877213202	0\\
25.7237383388208	0\\
25.7232889330704	0\\
25.7228395040667	0\\
25.7223900518072	0\\
25.7219405762896	0\\
25.7214910775115	0\\
25.7210415554703	0\\
25.7205920101638	0\\
25.7201424415895	0\\
25.719692849745	0\\
25.7192432346279	0\\
25.7187935962357	0\\
25.7183439345662	0\\
25.7178942496168	0\\
25.7174445413851	0\\
25.7169948098688	0\\
25.7165450550655	0\\
25.7160952769726	0\\
25.7156454755879	0\\
25.7151956509088	0\\
25.7147458029331	0\\
25.7142959316582	0\\
25.7138460370818	0\\
25.7133961192014	0\\
25.7129461780146	0\\
25.712496213519	0\\
25.7120462257123	0\\
25.7115962145919	0\\
25.7111461801555	0\\
25.7106961224006	0\\
25.7102460413249	0\\
25.7097959369259	0\\
25.7093458092011	0\\
25.7088956581483	0\\
25.7084454837649	0\\
25.7079952860486	0\\
25.7075450649969	0\\
25.7070948206074	0\\
25.7066445528777	0\\
25.7061942618053	0\\
25.7057439473879	0\\
25.7052936096231	0\\
25.7048432485083	0\\
25.7043928640412	0\\
25.7039424562194	0\\
25.7034920250404	0\\
25.7030415705018	0\\
25.7025910926012	0\\
25.7021405913362	0\\
25.7016900667044	0\\
25.7012395187032	0\\
25.7007889473304	0\\
25.7003383525834	0\\
25.6998877344599	0\\
25.6994370929574	0\\
25.6989864280735	0\\
25.6985357398057	0\\
25.6980850281517	0\\
25.6976342931091	0\\
25.6971835346753	0\\
25.6967327528479	0\\
25.6962819476246	0\\
25.6958311190029	0\\
25.6953802669804	0\\
25.6949293915546	0\\
25.6944784927231	0\\
25.6940275704836	0\\
25.6935766248334	0\\
25.6931256557703	0\\
25.6926746632918	0\\
25.6922236473954	0\\
25.6917726080788	0\\
25.6913215453395	0\\
25.690870459175	0\\
25.690419349583	0\\
25.6899682165609	0\\
25.6895170601064	0\\
25.6890658802171	0\\
25.6886146768904	0\\
25.688163450124	0\\
25.6877121999154	0\\
25.6872609262622	0\\
25.6868096291619	0\\
25.6863583086122	0\\
25.6859069646105	0\\
25.6854555971545	0\\
25.6850042062417	0\\
25.6845527918696	0\\
25.6841013540359	0\\
25.683649892738	0\\
25.6831984079736	0\\
25.6827468997402	0\\
25.6822953680354	0\\
25.6818438128567	0\\
25.6813922342017	0\\
25.6809406320679	0\\
25.6804890064529	0\\
25.6800373573543	0\\
25.6795856847696	0\\
25.6791339886964	0\\
25.6786822691322	0\\
25.6782305260746	0\\
25.6777787595211	0\\
25.6773269694694	0\\
25.6768751559169	0\\
25.6764233188612	0\\
25.6759714582999	0\\
25.6755195742305	0\\
25.6750676666505	0\\
25.6746157355576	0\\
25.6741637809493	0\\
25.6737118028231	0\\
25.6732598011766	0\\
25.6728077760074	0\\
25.6723557273129	0\\
25.6719036550907	0\\
25.6714515593385	0\\
25.6709994400537	0\\
25.6705472972339	0\\
25.6700951308766	0\\
25.6696429409794	0\\
25.6691907275399	0\\
25.6687384905555	0\\
25.6682862300239	0\\
25.6678339459426	0\\
25.6673816383091	0\\
25.6669293071209	0\\
25.6664769523757	0\\
25.666024574071	0\\
25.6655721722043	0\\
25.6651197467731	0\\
25.6646672977751	0\\
25.6642148252077	0\\
25.6637623290685	0\\
25.663309809355	0\\
25.6628572660648	0\\
25.6624046991955	0\\
25.6619521087445	0\\
25.6614994947094	0\\
25.6610468570878	0\\
25.6605941958772	0\\
25.6601415110751	0\\
25.6596888026791	0\\
25.6592360706867	0\\
25.6587833150955	0\\
25.6583305359029	0\\
25.6578777331066	0\\
25.6574249067041	0\\
25.6569720566929	0\\
25.6565191830705	0\\
25.6560662858346	0\\
25.6556133649825	0\\
25.655160420512	0\\
25.6547074524204	0\\
25.6542544607053	0\\
25.6538014453644	0\\
25.653348406395	0\\
25.6528953437948	0\\
25.6524422575613	0\\
25.6519891476919	0\\
25.6515360141843	0\\
25.651082857036	0\\
25.6506296762446	0\\
25.6501764718074	0\\
25.6497232437222	0\\
25.6492699919863	0\\
25.6488167165974	0\\
25.648363417553	0\\
25.6479100948505	0\\
25.6474567484876	0\\
25.6470033784618	0\\
25.6465499847705	0\\
25.6460965674114	0\\
25.6456431263819	0\\
25.6451896616796	0\\
25.6447361733021	0\\
25.6442826612467	0\\
25.6438291255111	0\\
25.6433755660928	0\\
25.6429219829894	0\\
25.6424683761982	0\\
25.642014745717	0\\
25.6415610915431	0\\
25.6411074136741	0\\
25.6406537121076	0\\
25.6401999868411	0\\
25.639746237872	0\\
25.639292465198	0\\
25.6388386688165	0\\
25.6383848487251	0\\
25.6379310049212	0\\
25.6374771374025	0\\
25.6370232461663	0\\
25.6365693312104	0\\
25.6361153925321	0\\
25.6356614301289	0\\
25.6352074439985	0\\
25.6347534341384	0\\
25.6342994005459	0\\
25.6338453432187	0\\
25.6333912621544	0\\
25.6329371573503	0\\
25.632483028804	0\\
25.6320288765131	0\\
25.631574700475	0\\
25.6311205006873	0\\
25.6306662771475	0\\
25.6302120298531	0\\
25.6297577588016	0\\
25.6293034639906	0\\
25.6288491454175	0\\
25.6283948030798	0\\
25.6279404369751	0\\
25.6274860471009	0\\
25.6270316334547	0\\
25.6265771960341	0\\
25.6261227348364	0\\
25.6256682498593	0\\
25.6252137411003	0\\
25.6247592085568	0\\
25.6243046522263	0\\
25.6238500721065	0\\
25.6233954681947	0\\
25.6229408404886	0\\
25.6224861889855	0\\
25.6220315136831	0\\
25.6215768145788	0\\
25.6211220916701	0\\
25.6206673449546	0\\
25.6202125744297	0\\
25.6197577800929	0\\
25.6193029619419	0\\
25.618848119974	0\\
25.6183932541867	0\\
25.6179383645777	0\\
25.6174834511443	0\\
25.6170285138841	0\\
25.6165735527946	0\\
25.6161185678733	0\\
25.6156635591177	0\\
25.6152085265253	0\\
25.6147534700936	0\\
25.6142983898201	0\\
25.6138432857024	0\\
25.6133881577378	0\\
25.6129330059239	0\\
25.6124778302582	0\\
25.6120226307383	0\\
25.6115674073615	0\\
25.6111121601255	0\\
25.6106568890276	0\\
25.6102015940655	0\\
25.6097462752366	0\\
25.6092909325384	0\\
25.6088355659683	0\\
25.608380175524	0\\
25.6079247612028	0\\
25.6074693230024	0\\
25.6070138609201	0\\
25.6065583749535	0\\
25.6061028651001	0\\
25.6056473313573	0\\
25.6051917737227	0\\
25.6047361921938	0\\
25.604280586768	0\\
25.6038249574428	0\\
25.6033693042158	0\\
25.6029136270844	0\\
25.6024579260462	0\\
25.6020022010985	0\\
25.6015464522389	0\\
25.6010906794649	0\\
25.600634882774	0\\
25.6001790621637	0\\
25.5997232176315	0\\
25.5992673491747	0\\
25.5988114567911	0\\
25.5983555404779	0\\
25.5978996002328	0\\
25.5974436360531	0\\
25.5969876479365	0\\
25.5965316358803	0\\
25.5960755998821	0\\
25.5956195399393	0\\
25.5951634560495	0\\
25.59470734821	0\\
25.5942512164185	0\\
25.5937950606724	0\\
25.5933388809691	0\\
25.5928826773062	0\\
25.5924264496811	0\\
25.5919701980914	0\\
25.5915139225344	0\\
25.5910576230077	0\\
25.5906012995088	0\\
25.5901449520351	0\\
25.5896885805842	0\\
25.5892321851535	0\\
25.5887757657404	0\\
25.5883193223425	0\\
25.5878628549573	0\\
25.5874063635822	0\\
25.5869498482146	0\\
25.5864933088522	0\\
25.5860367454923	0\\
25.5855801581325	0\\
25.5851235467701	0\\
25.5846669114028	0\\
25.5842102520279	0\\
25.583753568643	0\\
25.5832968612454	0\\
25.5828401298328	0\\
25.5823833744025	0\\
25.581926594952	0\\
25.5814697914789	0\\
25.5810129639805	0\\
25.5805561124544	0\\
25.580099236898	0\\
25.5796423373089	0\\
25.5791854136844	0\\
25.578728466022	0\\
25.5782714943192	0\\
25.5778144985736	0\\
25.5773574787825	0\\
25.5769004349434	0\\
25.5764433670538	0\\
25.5759862751111	0\\
25.5755291591129	0\\
25.5750720190566	0\\
25.5746148549396	0\\
25.5741576667595	0\\
25.5737004545137	0\\
25.5732432181996	0\\
25.5727859578148	0\\
25.5723286733566	0\\
25.5718713648227	0\\
25.5714140322103	0\\
25.5709566755171	0\\
25.5704992947404	0\\
25.5700418898777	0\\
25.5695844609264	0\\
25.5691270078842	0\\
25.5686695307483	0\\
25.5682120295163	0\\
25.5677545041857	0\\
25.5672969547538	0\\
25.5668393812181	0\\
25.5663817835762	0\\
25.5659241618255	0\\
25.5654665159634	0\\
25.5650088459873	0\\
25.5645511518948	0\\
25.5640934336834	0\\
25.5636356913504	0\\
25.5631779248933	0\\
25.5627201343096	0\\
25.5622623195967	0\\
25.5618044807521	0\\
25.5613466177732	0\\
25.5608887306576	0\\
25.5604308194026	0\\
25.5599728840058	0\\
25.5595149244645	0\\
25.5590569407762	0\\
25.5585989329384	0\\
25.5581409009485	0\\
25.557682844804	0\\
25.5572247645023	0\\
25.5567666600409	0\\
25.5563085314173	0\\
25.5558503786288	0\\
25.555392201673	0\\
25.5549340005473	0\\
25.5544757752491	0\\
25.5540175257758	0\\
25.5535592521251	0\\
25.5531009542942	0\\
25.5526426322806	0\\
25.5521842860818	0\\
25.5517259156953	0\\
25.5512675211184	0\\
25.5508091023486	0\\
25.5503506593834	0\\
25.5498921922202	0\\
25.5494337008565	0\\
25.5489751852897	0\\
25.5485166455172	0\\
25.5480580815365	0\\
25.5475994933451	0\\
25.5471408809403	0\\
25.5466822443197	0\\
25.5462235834806	0\\
25.5457648984205	0\\
25.5453061891369	0\\
25.5448474556272	0\\
25.5443886978888	0\\
25.5439299159191	0\\
25.5434711097157	0\\
25.5430122792759	0\\
25.5425534245973	0\\
25.5420945456771	0\\
25.5416356425129	0\\
25.5411767151021	0\\
25.5407177634422	0\\
25.5402587875306	0\\
25.5397997873646	0\\
25.5393407629418	0\\
25.5388817142596	0\\
25.5384226413155	0\\
25.5379635441068	0\\
25.537504422631	0\\
25.5370452768855	0\\
25.5365861068678	0\\
25.5361269125753	0\\
25.5356676940054	0\\
25.5352084511556	0\\
25.5347491840232	0\\
25.5342898926058	0\\
25.5338305769008	0\\
25.5333712369055	0\\
25.5329118726175	0\\
25.5324524840342	0\\
25.5319930711529	0\\
25.5315336339711	0\\
25.5310741724863	0\\
25.5306146866958	0\\
25.5301551765971	0\\
25.5296956421877	0\\
25.5292360834649	0\\
25.5287765004262	0\\
25.528316893069	0\\
25.5278572613907	0\\
25.5273976053888	0\\
25.5269379250607	0\\
25.5264782204037	0\\
25.5260184914155	0\\
25.5255587380932	0\\
25.5250989604345	0\\
25.5246391584367	0\\
25.5241793320972	0\\
25.5237194814134	0\\
25.5232596063828	0\\
25.5227997070028	0\\
25.5223397832709	0\\
25.5218798351843	0\\
25.5214198627407	0\\
25.5209598659373	0\\
25.5204998447716	0\\
25.520039799241	0\\
25.519579729343	0\\
25.5191196350749	0\\
25.5186595164342	0\\
25.5181993734182	0\\
25.5177392060245	0\\
25.5172790142504	0\\
25.5168187980934	0\\
25.5163585575508	0\\
25.5158982926201	0\\
25.5154380032986	0\\
25.5149776895839	0\\
25.5145173514733	0\\
25.5140569889642	0\\
25.5135966020541	0\\
25.5131361907403	0\\
25.5126757550203	0\\
25.5122152948915	0\\
25.5117548103512	0\\
25.511294301397	0\\
25.5108337680262	0\\
25.5103732102362	0\\
25.5099126280245	0\\
25.5094520213884	0\\
25.5089913903253	0\\
25.5085307348328	0\\
25.5080700549081	0\\
25.5076093505487	0\\
25.507148621752	0\\
25.5066878685153	0\\
25.5062270908362	0\\
25.505766288712	0\\
25.5053054621402	0\\
25.504844611118	0\\
25.504383735643	0\\
25.5039228357125	0\\
25.5034619113239	0\\
25.5030009624747	0\\
25.5025399891622	0\\
25.5020789913839	0\\
25.5016179691372	0\\
25.5011569224194	0\\
25.5006958512279	0\\
25.5002347555602	0\\
25.4997736354137	0\\
25.4993124907857	0\\
25.4988513216737	0\\
25.4983901280751	0\\
25.4979289099872	0\\
25.4974676674075	0\\
25.4970064003333	0\\
25.4965451087621	0\\
25.4960837926912	0\\
25.4956224521181	0\\
25.4951610870402	0\\
25.4946996974548	0\\
25.4942382833593	0\\
25.4937768447512	0\\
25.4933153816278	0\\
25.4928538939865	0\\
25.4923923818248	0\\
25.4919308451399	0\\
25.4914692839294	0\\
25.4910076981906	0\\
25.4905460879209	0\\
25.4900844531177	0\\
25.4896227937783	0\\
25.4891611099002	0\\
25.4886994014808	0\\
25.4882376685175	0\\
25.4877759110076	0\\
25.4873141289485	0\\
25.4868523223377	0\\
25.4863904911724	0\\
25.4859286354502	0\\
25.4854667551684	0\\
25.4850048503243	0\\
25.4845429209154	0\\
25.4840809669391	0\\
25.4836189883927	0\\
25.4831569852736	0\\
25.4826949575792	0\\
25.482232905307	0\\
25.4817708284542	0\\
25.4813087270183	0\\
25.4808466009966	0\\
25.4803844503866	0\\
25.4799222751856	0\\
25.479460075391	0\\
25.4789978510002	0\\
25.4785356020105	0\\
25.4780733284194	0\\
25.4776110302242	0\\
25.4771487074224	0\\
25.4766863600112	0\\
25.4762239879881	0\\
25.4757615913504	0\\
25.4752991700956	0\\
25.474836724221	0\\
25.4743742537239	0\\
25.4739117586018	0\\
25.4734492388521	0\\
25.4729866944721	0\\
25.4725241254592	0\\
25.4720615318107	0\\
25.4715989135241	0\\
25.4711362705967	0\\
25.470673603026	0\\
25.4702109108091	0\\
25.4697481939437	0\\
25.4692854524269	0\\
25.4688226862563	0\\
25.4683598954291	0\\
25.4678970799427	0\\
25.4674342397946	0\\
25.466971374982	0\\
25.4665084855024	0\\
25.4660455713531	0\\
25.4655826325315	0\\
25.465119669035	0\\
25.4646566808609	0\\
25.4641936680066	0\\
25.4637306304695	0\\
25.4632675682469	0\\
25.4628044813362	0\\
25.4623413697348	0\\
25.4618782334401	0\\
25.4614150724493	0\\
25.4609518867599	0\\
25.4604886763693	0\\
25.4600254412748	0\\
25.4595621814737	0\\
25.4590988969635	0\\
25.4586355877414	0\\
25.4581722538049	0\\
25.4577088951514	0\\
25.4572455117781	0\\
25.4567821036825	0\\
25.4563186708619	0\\
25.4558552133137	0\\
25.4553917310352	0\\
25.4549282240238	0\\
25.4544646922768	0\\
25.4540011357917	0\\
25.4535375545658	0\\
25.4530739485963	0\\
25.4526103178808	0\\
25.4521466624165	0\\
25.4516829822008	0\\
25.4512192772311	0\\
25.4507555475047	0\\
25.450291793019	0\\
25.4498280137714	0\\
25.4493642097591	0\\
25.4489003809795	0\\
25.4484365274301	0\\
25.4479726491081	0\\
25.4475087460109	0\\
25.4470448181359	0\\
25.4465808654804	0\\
25.4461168880417	0\\
25.4456528858173	0\\
25.4451888588044	0\\
25.4447248070005	0\\
25.4442607304029	0\\
25.4437966290088	0\\
25.4433325028157	0\\
25.442868351821	0\\
25.4424041760219	0\\
25.4419399754159	0\\
25.4414757500002	0\\
25.4410114997722	0\\
25.4405472247293	0\\
25.4400829248688	0\\
25.4396186001881	0\\
25.4391542506845	0\\
25.4386898763553	0\\
25.4382254771979	0\\
25.4377610532097	0\\
25.4372966043879	0\\
25.43683213073	0\\
25.4363676322332	0\\
25.435903108895	0\\
25.4354385607126	0\\
25.4349739876834	0\\
25.4345093898048	0\\
25.434044767074	0\\
25.4335801194885	0\\
25.4331154470455	0\\
25.4326507497425	0\\
25.4321860275766	0\\
25.4317212805454	0\\
25.4312565086461	0\\
25.4307917118761	0\\
25.4303268902326	0\\
25.4298620437131	0\\
25.4293971723149	0\\
25.4289322760353	0\\
25.4284673548716	0\\
25.4280024088212	0\\
25.4275374378815	0\\
25.4270724420497	0\\
25.4266074213232	0\\
25.4261423756993	0\\
25.4256773051754	0\\
25.4252122097487	0\\
25.4247470894167	0\\
25.4242819441767	0\\
25.4238167740259	0\\
25.4233515789618	0\\
25.4228863589816	0\\
25.4224211140827	0\\
25.4219558442624	0\\
25.421490549518	0\\
25.4210252298469	0\\
25.4205598852465	0\\
25.4200945157139	0\\
25.4196291212466	0\\
25.4191637018419	0\\
25.4186982574972	0\\
25.4182327882096	0\\
25.4177672939766	0\\
25.4173017747956	0\\
25.4168362306637	0\\
25.4163706615784	0\\
25.415905067537	0\\
25.4154394485367	0\\
25.414973804575	0\\
25.4145081356491	0\\
25.4140424417564	0\\
25.4135767228941	0\\
25.4131109790597	0\\
25.4126452102504	0\\
25.4121794164636	0\\
25.4117135976965	0\\
25.4112477539465	0\\
25.410781885211	0\\
25.4103159914871	0\\
25.4098500727724	0\\
25.409384129064	0\\
25.4089181603593	0\\
25.4084521666556	0\\
25.4079861479502	0\\
25.4075201042404	0\\
25.4070540355237	0\\
25.4065879417972	0\\
25.4061218230583	0\\
25.4056556793043	0\\
25.4051895105325	0\\
25.4047233167402	0\\
25.4042570979249	0\\
25.4037908540836	0\\
25.4033245852139	0\\
25.4028582913129	0\\
25.4023919723781	0\\
25.4019256284066	0\\
25.4014592593959	0\\
25.4009928653432	0\\
25.4005264462459	0\\
25.4000600021012	0\\
25.3995935329065	0\\
25.399127038659	0\\
25.3986605193562	0\\
25.3981939749952	0\\
25.3977274055735	0\\
25.3972608110882	0\\
25.3967941915368	0\\
25.3963275469165	0\\
25.3958608772246	0\\
25.3953941824584	0\\
25.3949274626153	0\\
25.3944607176926	0\\
25.3939939476875	0\\
25.3935271525973	0\\
25.3930603324194	0\\
25.3925934871511	0\\
25.3921266167896	0\\
25.3916597213323	0\\
25.3911928007764	0\\
25.3907258551193	0\\
25.3902588843583	0\\
25.3897918884907	0\\
25.3893248675137	0\\
25.3888578214247	0\\
25.388390750221	0\\
25.3879236538998	0\\
25.3874565324585	0\\
25.3869893858944	0\\
25.3865222142047	0\\
25.3860550173868	0\\
25.385587795438	0\\
25.3851205483555	0\\
25.3846532761366	0\\
25.3841859787787	0\\
25.383718656279	0\\
25.3832513086349	0\\
25.3827839358436	0\\
25.3823165379024	0\\
25.3818491148086	0\\
25.3813816665595	0\\
25.3809141931525	0\\
25.3804466945847	0\\
25.3799791708535	0\\
25.3795116219561	0\\
25.37904404789	0\\
25.3785764486523	0\\
25.3781088242403	0\\
25.3776411746513	0\\
25.3771734998827	0\\
25.3767057999317	0\\
25.3762380747956	0\\
25.3757703244717	0\\
25.3753025489572	0\\
25.3748347482495	0\\
25.3743669223459	0\\
25.3738990712436	0\\
25.3734311949398	0\\
25.372963293432	0\\
25.3724953667174	0\\
25.3720274147933	0\\
25.3715594376569	0\\
25.3710914353055	0\\
25.3706234077365	0\\
25.370155354947	0\\
25.3696872769344	0\\
25.369219173696	0\\
25.3687510452291	0\\
25.3682828915308	0\\
25.3678147125986	0\\
25.3673465084297	0\\
25.3668782790213	0\\
25.3664100243708	0\\
25.3659417444753	0\\
25.3654734393323	0\\
25.365005108939	0\\
25.3645367532926	0\\
25.3640683723904	0\\
25.3635999662298	0\\
25.3631315348079	0\\
25.3626630781221	0\\
25.3621945961696	0\\
25.3617260889477	0\\
25.3612575564537	0\\
25.3607889986848	0\\
25.3603204156384	0\\
25.3598518073116	0\\
25.3593831737019	0\\
25.3589145148064	0\\
25.3584458306224	0\\
25.3579771211471	0\\
25.357508386378	0\\
25.3570396263121	0\\
25.3565708409468	0\\
25.3561020302794	0\\
25.3556331943072	0\\
25.3551643330273	0\\
25.3546954464371	0\\
25.3542265345338	0\\
25.3537575973148	0\\
25.3532886347772	0\\
25.3528196469183	0\\
25.3523506337354	0\\
25.3518815952258	0\\
25.3514125313868	0\\
25.3509434422155	0\\
25.3504743277092	0\\
25.3500051878653	0\\
25.349536022681	0\\
25.3490668321535	0\\
25.3485976162801	0\\
25.3481283750581	0\\
25.3476591084847	0\\
25.3471898165572	0\\
25.3467204992728	0\\
25.3462511566289	0\\
25.3457817886226	0\\
25.3453123952513	0\\
25.3448429765121	0\\
25.3443735324024	0\\
25.3439040629193	0\\
25.3434345680603	0\\
25.3429650478224	0\\
25.342495502203	0\\
25.3420259311994	0\\
25.3415563348087	0\\
25.3410867130283	0\\
25.3406170658553	0\\
25.3401473932871	0\\
25.3396776953209	0\\
25.3392079719539	0\\
25.3387382231835	0\\
25.3382684490068	0\\
25.3377986494211	0\\
25.3373288244236	0\\
25.3368589740117	0\\
25.3363890981825	0\\
25.3359191969333	0\\
25.3354492702614	0\\
25.334979318164	0\\
25.3345093406384	0\\
25.3340393376818	0\\
25.3335693092914	0\\
25.3330992554645	0\\
25.3326291761983	0\\
25.3321590714902	0\\
25.3316889413373	0\\
25.3312187857368	0\\
25.3307486046861	0\\
25.3302783981824	0\\
25.3298081662229	0\\
25.3293379088048	0\\
25.3288676259255	0\\
25.3283973175821	0\\
25.3279269837719	0\\
25.3274566244921	0\\
25.32698623974	0\\
25.3265158295127	0\\
25.3260453938077	0\\
25.325574932622	0\\
25.325104445953	0\\
25.3246339337979	0\\
25.3241633961538	0\\
25.3236928330181	0\\
25.323222244388	0\\
25.3227516302607	0\\
25.3222809906335	0\\
25.3218103255036	0\\
25.3213396348682	0\\
25.3208689187246	0\\
25.32039817707	0\\
25.3199274099016	0\\
25.3194566172167	0\\
25.3189857990125	0\\
25.3185149552862	0\\
25.3180440860351	0\\
25.3175731912564	0\\
25.3171022709474	0\\
25.3166313251052	0\\
25.3161603537271	0\\
25.3156893568104	0\\
25.3152183343522	0\\
25.3147472863498	0\\
25.3142762128004	0\\
25.3138051137013	0\\
25.3133339890497	0\\
25.3128628388427	0\\
25.3123916630777	0\\
25.3119204617518	0\\
25.3114492348624	0\\
25.3109779824065	0\\
25.3105067043815	0\\
25.3100354007845	0\\
25.3095640716128	0\\
25.3090927168637	0\\
25.3086213365343	0\\
25.3081499306218	0\\
25.3076784991235	0\\
25.3072070420366	0\\
25.3067355593584	0\\
25.3062640510859	0\\
25.3057925172166	0\\
25.3053209577476	0\\
25.304849372676	0\\
25.3043777619992	0\\
25.3039061257143	0\\
25.3034344638186	0\\
25.3029627763093	0\\
25.3024910631836	0\\
25.3020193244387	0\\
25.3015475600719	0\\
25.3010757700803	0\\
25.3006039544612	0\\
25.3001321132117	0\\
25.2996602463292	0\\
25.2991883538108	0\\
25.2987164356537	0\\
25.2982444918552	0\\
25.2977725224124	0\\
25.2973005273226	0\\
25.296828506583	0\\
25.2963564601907	0\\
25.2958843881431	0\\
25.2954122904373	0\\
25.2949401670705	0\\
25.29446801804	0\\
25.2939958433429	0\\
25.2935236429765	0\\
25.2930514169379	0\\
25.2925791652245	0\\
25.2921068878333	0\\
25.2916345847615	0\\
25.2911622560065	0\\
25.2906899015654	0\\
25.2902175214355	0\\
25.2897451156138	0\\
25.2892726840976	0\\
25.2888002268842	0\\
25.2883277439707	0\\
25.2878552353543	0\\
25.2873827010323	0\\
25.2869101410018	0\\
25.2864375552601	0\\
25.2859649438043	0\\
25.2854923066317	0\\
25.2850196437394	0\\
25.2845469551247	0\\
25.2840742407847	0\\
25.2836015007166	0\\
25.2831287349177	0\\
25.2826559433852	0\\
25.2821831261162	0\\
25.2817102831079	0\\
25.2812374143576	0\\
25.2807645198624	0\\
25.2802915996196	0\\
25.2798186536263	0\\
25.2793456818797	0\\
25.278872684377	0\\
25.2783996611154	0\\
25.2779266120922	0\\
25.2774535373045	0\\
25.2769804367494	0\\
25.2765073104243	0\\
25.2760341583262	0\\
25.2755609804524	0\\
25.2750877768001	0\\
25.2746145473664	0\\
25.2741412921486	0\\
25.2736680111438	0\\
25.2731947043493	0\\
25.2727213717622	0\\
25.2722480133796	0\\
25.2717746291989	0\\
25.2713012192171	0\\
25.2708277834316	0\\
25.2703543218393	0\\
25.2698808344377	0\\
25.2694073212237	0\\
25.2689337821947	0\\
25.2684602173477	0\\
25.26798662668	0\\
25.2675130101889	0\\
25.2670393678713	0\\
25.2665656997246	0\\
25.2660920057459	0\\
25.2656182859324	0\\
25.2651445402813	0\\
25.2646707687898	0\\
25.264196971455	0\\
25.2637231482741	0\\
25.2632492992443	0\\
25.2627754243628	0\\
25.2623015236268	0\\
25.2618275970334	0\\
25.2613536445799	0\\
25.2608796662633	0\\
25.2604056620809	0\\
25.2599316320299	0\\
25.2594575761074	0\\
25.2589834943106	0\\
25.2585093866366	0\\
25.2580352530828	0\\
25.2575610936461	0\\
25.2570869083239	0\\
25.2566126971132	0\\
25.2561384600113	0\\
25.2556641970153	0\\
25.2551899081224	0\\
25.2547155933297	0\\
25.2542412526345	0\\
25.2537668860339	0\\
25.2532924935251	0\\
25.2528180751052	0\\
25.2523436307715	0\\
25.251869160521	0\\
25.251394664351	0\\
25.2509201422586	0\\
25.250445594241	0\\
25.2499710202953	0\\
25.2494964204188	0\\
25.2490217946086	0\\
25.2485471428618	0\\
25.2480724651756	0\\
25.2475977615473	0\\
25.2471230319738	0\\
25.2466482764525	0\\
25.2461734949805	0\\
25.2456986875549	0\\
25.2452238541729	0\\
25.2447489948316	0\\
25.2442741095283	0\\
25.2437991982601	0\\
25.2433242610241	0\\
25.2428492978176	0\\
25.2423743086375	0\\
25.2418992934813	0\\
25.2414242523459	0\\
25.2409491852285	0\\
25.2404740921263	0\\
25.2399989730365	0\\
25.2395238279563	0\\
25.2390486568826	0\\
25.2385734598128	0\\
25.238098236744	0\\
25.2376229876733	0\\
25.237147712598	0\\
25.236672411515	0\\
25.2361970844217	0\\
25.2357217313151	0\\
25.2352463521924	0\\
25.2347709470508	0\\
25.2342955158873	0\\
25.2338200586993	0\\
25.2333445754837	0\\
25.2328690662378	0\\
25.2323935309587	0\\
25.2319179696436	0\\
25.2314423822896	0\\
25.2309667688938	0\\
25.2304911294534	0\\
25.2300154639656	0\\
25.2295397724275	0\\
25.2290640548362	0\\
25.2285883111889	0\\
25.2281125414828	0\\
25.2276367457149	0\\
25.2271609238825	0\\
25.2266850759826	0\\
25.2262092020125	0\\
25.2257333019692	0\\
25.2252573758499	0\\
25.2247814236517	0\\
25.2243054453718	0\\
25.2238294410074	0\\
25.2233534105555	0\\
25.2228773540133	0\\
25.222401271378	0\\
25.2219251626467	0\\
25.2214490278165	0\\
25.2209728668845	0\\
25.220496679848	0\\
25.2200204667039	0\\
25.2195442274496	0\\
25.2190679620821	0\\
25.2185916705985	0\\
25.218115352996	0\\
25.2176390092717	0\\
25.2171626394228	0\\
25.2166862434464	0\\
25.2162098213396	0\\
25.2157333730995	0\\
25.2152568987234	0\\
25.2147803982082	0\\
25.2143038715512	0\\
25.2138273187495	0\\
25.2133507398002	0\\
25.2128741347005	0\\
25.2123975034474	0\\
25.2119208460382	0\\
25.2114441624698	0\\
25.2109674527396	0\\
25.2104907168445	0\\
25.2100139547818	0\\
25.2095371665485	0\\
25.2090603521417	0\\
25.2085835115587	0\\
25.2081066447965	0\\
25.2076297518522	0\\
25.207152832723	0\\
25.2066758874061	0\\
25.2061989158984	0\\
25.2057219181972	0\\
25.2052448942995	0\\
25.2047678442026	0\\
25.2042907679035	0\\
25.2038136653993	0\\
25.2033365366871	0\\
25.2028593817641	0\\
25.2023822006275	0\\
25.2019049932742	0\\
25.2014277597015	0\\
25.2009504999064	0\\
25.2004732138861	0\\
25.1999959016377	0\\
25.1995185631583	0\\
25.1990411984451	0\\
25.1985638074951	0\\
25.1980863903054	0\\
25.1976089468732	0\\
25.1971314771956	0\\
25.1966539812697	0\\
25.1961764590926	0\\
25.1956989106614	0\\
25.1952213359733	0\\
25.1947437350253	0\\
25.1942661078146	0\\
25.1937884543383	0\\
25.1933107745934	0\\
25.1928330685772	0\\
25.1923553362867	0\\
25.191877577719	0\\
25.1913997928712	0\\
25.1909219817404	0\\
25.1904441443238	0\\
25.1899662806185	0\\
25.1894883906215	0\\
25.18901047433	0\\
25.188532531741	0\\
25.1880545628518	0\\
25.1875765676593	0\\
25.1870985461607	0\\
25.1866204983531	0\\
25.1861424242336	0\\
25.1856643237993	0\\
25.1851861970473	0\\
25.1847080439747	0\\
25.1842298645786	0\\
25.1837516588561	0\\
25.1832734268043	0\\
25.1827951684204	0\\
25.1823168837013	0\\
25.1818385726443	0\\
25.1813602352464	0\\
25.1808818715046	0\\
25.1804034814162	0\\
25.1799250649782	0\\
25.1794466221877	0\\
25.1789681530418	0\\
25.1784896575376	0\\
25.1780111356722	0\\
25.1775325874426	0\\
25.1770540128461	0\\
25.1765754118796	0\\
25.1760967845403	0\\
25.1756181308252	0\\
25.1751394507315	0\\
25.1746607442563	0\\
25.1741820113965	0\\
25.1737032521494	0\\
25.1732244665121	0\\
25.1727456544815	0\\
25.1722668160549	0\\
25.1717879512292	0\\
25.1713090600016	0\\
25.1708301423692	0\\
25.1703511983291	0\\
25.1698722278783	0\\
25.1693932310139	0\\
25.168914207733	0\\
25.1684351580327	0\\
25.1679560819102	0\\
25.1674769793624	0\\
25.1669978503864	0\\
25.1665186949794	0\\
25.1660395131385	0\\
25.1655603048606	0\\
25.165081070143	0\\
25.1646018089826	0\\
25.1641225213765	0\\
25.163643207322	0\\
25.1631638668159	0\\
25.1626844998554	0\\
25.1622051064376	0\\
25.1617256865596	0\\
25.1612462402184	0\\
25.1607667674111	0\\
25.1602872681348	0\\
25.1598077423866	0\\
25.1593281901635	0\\
25.1588486114627	0\\
25.1583690062812	0\\
25.157889374616	0\\
25.1574097164643	0\\
25.1569300318231	0\\
25.1564503206895	0\\
25.1559705830606	0\\
25.1554908189334	0\\
25.1550110283051	0\\
25.1545312111726	0\\
25.1540513675331	0\\
25.1535714973837	0\\
25.1530916007213	0\\
25.1526116775432	0\\
25.1521317278462	0\\
25.1516517516276	0\\
25.1511717488844	0\\
25.1506917196136	0\\
25.1502116638123	0\\
25.1497315814776	0\\
25.1492514726066	0\\
25.1487713371963	0\\
25.1482911752437	0\\
25.1478109867461	0\\
25.1473307717003	0\\
25.1468505301035	0\\
25.1463702619528	0\\
25.1458899672451	0\\
25.1454096459777	0\\
25.1449292981474	0\\
25.1444489237515	0\\
25.1439685227869	0\\
25.1434880952507	0\\
25.14300764114	0\\
25.1425271604518	0\\
25.1420466531833	0\\
25.1415661193314	0\\
25.1410855588932	0\\
25.1406049718657	0\\
25.1401243582461	0\\
25.1396437180314	0\\
25.1391630512186	0\\
25.1386823578049	0\\
25.1382016377871	0\\
25.1377208911625	0\\
25.1372401179281	0\\
25.1367593180808	0\\
25.1362784916179	0\\
25.1357976385362	0\\
25.1353167588329	0\\
25.1348358525051	0\\
25.1343549195497	0\\
25.1338739599639	0\\
25.1333929737447	0\\
25.132911960889	0\\
25.1324309213941	0\\
25.1319498552569	0\\
25.1314687624745	0\\
25.1309876430439	0\\
25.1305064969622	0\\
25.1300253242264	0\\
25.1295441248335	0\\
25.1290628987807	0\\
25.128581646065	0\\
25.1281003666833	0\\
25.1276190606328	0\\
25.1271377279105	0\\
25.1266563685135	0\\
25.1261749824387	0\\
25.1256935696832	0\\
25.1252121302441	0\\
25.1247306641185	0\\
25.1242491713032	0\\
25.1237676517955	0\\
25.1232861055923	0\\
25.1228045326907	0\\
25.1223229330876	0\\
25.1218413067803	0\\
25.1213596537656	0\\
25.1208779740406	0\\
25.1203962676024	0\\
25.119914534448	0\\
25.1194327745745	0\\
25.1189509879788	0\\
25.118469174658	0\\
25.1179873346092	0\\
25.1175054678293	0\\
25.1170235743155	0\\
25.1165416540647	0\\
25.116059707074	0\\
25.1155777333404	0\\
25.1150957328609	0\\
25.1146137056326	0\\
25.1141316516525	0\\
25.1136495709176	0\\
25.113167463425	0\\
25.1126853291717	0\\
25.1122031681548	0\\
25.1117209803711	0\\
25.1112387658179	0\\
25.1107565244921	0\\
25.1102742563907	0\\
25.1097919615107	0\\
25.1093096398493	0\\
25.1088272914033	0\\
25.1083449161699	0\\
25.1078625141461	0\\
25.1073800853288	0\\
25.1068976297152	0\\
25.1064151473022	0\\
25.1059326380868	0\\
25.1054501020661	0\\
};
\addplot [color=mycolor1, forget plot]
  table[row sep=crcr]{%
25.1054501020661	0\\
25.1049675392371	0\\
25.1044849495969	0\\
25.1040023331423	0\\
25.1035196898706	0\\
25.1030370197786	0\\
25.1025543228634	0\\
25.102071599122	0\\
25.1015888485515	0\\
25.1011060711488	0\\
25.100623266911	0\\
25.1001404358351	0\\
25.0996575779181	0\\
25.099174693157	0\\
25.0986917815488	0\\
25.0982088430906	0\\
25.0977258777793	0\\
25.0972428856121	0\\
25.0967598665858	0\\
25.0962768206975	0\\
25.0957937479442	0\\
25.0953106483229	0\\
25.0948275218307	0\\
25.0943443684646	0\\
25.0938611882215	0\\
25.0933779810984	0\\
25.0928947470924	0\\
25.0924114862005	0\\
25.0919281984198	0\\
25.0914448837471	0\\
25.0909615421795	0\\
25.090478173714	0\\
25.0899947783477	0\\
25.0895113560774	0\\
25.0890279069003	0\\
25.0885444308134	0\\
25.0880609278135	0\\
25.0875773978979	0\\
25.0870938410633	0\\
25.0866102573069	0\\
25.0861266466257	0\\
25.0856430090166	0\\
25.0851593444767	0\\
25.0846756530029	0\\
25.0841919345923	0\\
25.0837081892418	0\\
25.0832244169485	0\\
25.0827406177094	0\\
25.0822567915213	0\\
25.0817729383815	0\\
25.0812890582868	0\\
25.0808051512342	0\\
25.0803212172207	0\\
25.0798372562434	0\\
25.0793532682992	0\\
25.0788692533852	0\\
25.0783852114982	0\\
25.0779011426354	0\\
25.0774170467937	0\\
25.07693292397	0\\
25.0764487741615	0\\
25.075964597365	0\\
25.0754803935776	0\\
25.0749961627962	0\\
25.074511905018	0\\
25.0740276202397	0\\
25.0735433084585	0\\
25.0730589696713	0\\
25.0725746038751	0\\
25.0720902110669	0\\
25.0716057912437	0\\
25.0711213444024	0\\
25.0706368705401	0\\
25.0701523696537	0\\
25.0696678417403	0\\
25.0691832867968	0\\
25.0686987048201	0\\
25.0682140958073	0\\
25.0677294597554	0\\
25.0672447966614	0\\
25.0667601065221	0\\
25.0662753893347	0\\
25.065790645096	0\\
25.0653058738031	0\\
25.064821075453	0\\
25.0643362500426	0\\
25.0638513975689	0\\
25.0633665180288	0\\
25.0628816114195	0\\
25.0623966777377	0\\
25.0619117169806	0\\
25.0614267291451	0\\
25.0609417142282	0\\
25.0604566722267	0\\
25.0599716031378	0\\
25.0594865069584	0\\
25.0590013836855	0\\
25.058516233316	0\\
25.0580310558469	0\\
25.0575458512752	0\\
25.0570606195979	0\\
25.0565753608118	0\\
25.0560900749141	0\\
25.0556047619017	0\\
25.0551194217715	0\\
25.0546340545205	0\\
25.0541486601456	0\\
25.053663238644	0\\
25.0531777900124	0\\
25.0526923142479	0\\
25.0522068113475	0\\
25.051721281308	0\\
25.0512357241266	0\\
25.0507501398001	0\\
25.0502645283255	0\\
25.0497788896998	0\\
25.0492932239199	0\\
25.0488075309828	0\\
25.0483218108855	0\\
25.0478360636249	0\\
25.047350289198	0\\
25.0468644876017	0\\
25.0463786588331	0\\
25.045892802889	0\\
25.0454069197664	0\\
25.0449210094624	0\\
25.0444350719737	0\\
25.0439491072975	0\\
25.0434631154307	0\\
25.0429770963701	0\\
25.0424910501129	0\\
25.0420049766559	0\\
25.041518875996	0\\
25.0410327481303	0\\
25.0405465930557	0\\
25.0400604107692	0\\
25.0395742012676	0\\
25.039087964548	0\\
25.0386017006074	0\\
25.0381154094426	0\\
25.0376290910506	0\\
25.0371427454283	0\\
25.0366563725728	0\\
25.0361699724809	0\\
25.0356835451497	0\\
25.035197090576	0\\
25.0347106087568	0\\
25.0342240996891	0\\
25.0337375633698	0\\
25.0332509997959	0\\
25.0327644089642	0\\
25.0322777908718	0\\
25.0317911455156	0\\
25.0313044728926	0\\
25.0308177729996	0\\
25.0303310458336	0\\
25.0298442913916	0\\
25.0293575096705	0\\
25.0288707006672	0\\
25.0283838643788	0\\
25.0278970008021	0\\
25.027410109934	0\\
25.0269231917716	0\\
25.0264362463118	0\\
25.0259492735514	0\\
25.0254622734875	0\\
25.0249752461169	0\\
25.0244881914366	0\\
25.0240011094436	0\\
25.0235140001348	0\\
25.0230268635071	0\\
25.0225396995574	0\\
25.0220525082828	0\\
25.02156528968	0\\
25.0210780437462	0\\
25.0205907704781	0\\
25.0201034698727	0\\
25.019616141927	0\\
25.0191287866379	0\\
25.0186414040023	0\\
25.0181539940171	0\\
25.0176665566794	0\\
25.0171790919859	0\\
25.0166915999337	0\\
25.0162040805196	0\\
25.0157165337407	0\\
25.0152289595938	0\\
25.0147413580758	0\\
25.0142537291837	0\\
25.0137660729144	0\\
25.0132783892649	0\\
25.012790678232	0\\
25.0123029398127	0\\
25.0118151740039	0\\
25.0113273808025	0\\
25.0108395602054	0\\
25.0103517122097	0\\
25.0098638368121	0\\
25.0093759340097	0\\
25.0088880037993	0\\
25.0084000461779	0\\
25.0079120611423	0\\
25.0074240486896	0\\
25.0069360088165	0\\
25.0064479415201	0\\
25.0059598467973	0\\
25.0054717246449	0\\
25.0049835750599	0\\
25.0044953980393	0\\
25.0040071935798	0\\
25.0035189616785	0\\
25.0030307023322	0\\
25.0025424155379	0\\
25.0020541012925	0\\
25.0015657595929	0\\
25.001077390436	0\\
25.0005889938187	0\\
25.0001005697379	0\\
24.9996121181906	0\\
24.9991236391736	0\\
24.9986351326839	0\\
24.9981465987184	0\\
24.9976580372739	0\\
24.9971694483474	0\\
24.9966808319358	0\\
24.9961921880361	0\\
24.995703516645	0\\
24.9952148177596	0\\
24.9947260913767	0\\
24.9942373374932	0\\
24.993748556106	0\\
24.9932597472121	0\\
24.9927709108083	0\\
24.9922820468916	0\\
24.9917931554589	0\\
24.9913042365069	0\\
24.9908152900328	0\\
24.9903263160333	0\\
24.9898373145053	0\\
24.9893482854458	0\\
24.9888592288516	0\\
24.9883701447198	0\\
24.987881033047	0\\
24.9873918938303	0\\
24.9869027270666	0\\
24.9864135327527	0\\
24.9859243108855	0\\
24.985435061462	0\\
24.984945784479	0\\
24.9844564799335	0\\
24.9839671478223	0\\
24.9834777881422	0\\
24.9829884008903	0\\
24.9824989860634	0\\
24.9820095436584	0\\
24.9815200736722	0\\
24.9810305761017	0\\
24.9805410509437	0\\
24.9800514981952	0\\
24.9795619178531	0\\
24.9790723099142	0\\
24.9785826743754	0\\
24.9780930112336	0\\
24.9776033204857	0\\
24.9771136021286	0\\
24.9766238561592	0\\
24.9761340825744	0\\
24.975644281371	0\\
24.9751544525459	0\\
24.9746645960961	0\\
24.9741747120184	0\\
24.9736848003096	0\\
24.9731948609667	0\\
24.9727048939866	0\\
24.9722148993661	0\\
24.9717248771021	0\\
24.9712348271915	0\\
24.9707447496312	0\\
24.970254644418	0\\
24.9697645115489	0\\
24.9692743510206	0\\
24.9687841628302	0\\
24.9682939469744	0\\
24.9678037034502	0\\
24.9673134322544	0\\
24.9668231333839	0\\
24.9663328068356	0\\
24.9658424526063	0\\
24.965352070693	0\\
24.9648616610924	0\\
24.9643712238015	0\\
24.9638807588171	0\\
24.9633902661362	0\\
24.9628997457556	0\\
24.9624091976721	0\\
24.9619186218826	0\\
24.961428018384	0\\
24.9609373871732	0\\
24.9604467282471	0\\
24.9599560416024	0\\
24.9594653272361	0\\
24.9589745851451	0\\
24.9584838153262	0\\
24.9579930177762	0\\
24.9575021924921	0\\
24.9570113394707	0\\
24.9565204587088	0\\
24.9560295502034	0\\
24.9555386139513	0\\
24.9550476499494	0\\
24.9545566581945	0\\
24.9540656386834	0\\
24.9535745914132	0\\
24.9530835163805	0\\
24.9525924135823	0\\
24.9521012830154	0\\
24.9516101246767	0\\
24.9511189385631	0\\
24.9506277246714	0\\
24.9501364829984	0\\
24.9496452135411	0\\
24.9491539162962	0\\
24.9486625912607	0\\
24.9481712384314	0\\
24.9476798578051	0\\
24.9471884493787	0\\
24.9466970131491	0\\
24.9462055491131	0\\
24.9457140572676	0\\
24.9452225376093	0\\
24.9447309901353	0\\
24.9442394148422	0\\
24.9437478117271	0\\
24.9432561807866	0\\
24.9427645220177	0\\
24.9422728354173	0\\
24.9417811209821	0\\
24.941289378709	0\\
24.9407976085949	0\\
24.9403058106366	0\\
24.939813984831	0\\
24.9393221311749	0\\
24.9388302496651	0\\
24.9383383402985	0\\
24.937846403072	0\\
24.9373544379823	0\\
24.9368624450264	0\\
24.936370424201	0\\
24.9358783755031	0\\
24.9353862989294	0\\
24.9348941944768	0\\
24.9344020621422	0\\
24.9339099019223	0\\
24.9334177138141	0\\
24.9329254978143	0\\
24.9324332539198	0\\
24.9319409821274	0\\
24.931448682434	0\\
24.9309563548364	0\\
24.9304639993315	0\\
24.929971615916	0\\
24.9294792045869	0\\
24.9289867653409	0\\
24.9284942981749	0\\
24.9280018030857	0\\
24.9275092800702	0\\
24.9270167291251	0\\
24.9265241502474	0\\
24.9260315434338	0\\
24.9255389086812	0\\
24.9250462459864	0\\
24.9245535553463	0\\
24.9240608367576	0\\
24.9235680902173	0\\
24.923075315722	0\\
24.9225825132687	0\\
24.9220896828542	0\\
24.9215968244753	0\\
24.9211039381289	0\\
24.9206110238117	0\\
24.9201180815206	0\\
24.9196251112524	0\\
24.9191321130039	0\\
24.918639086772	0\\
24.9181460325535	0\\
24.9176529503452	0\\
24.9171598401439	0\\
24.9166667019465	0\\
24.9161735357498	0\\
24.9156803415505	0\\
24.9151871193455	0\\
24.9146938691317	0\\
24.9142005909058	0\\
24.9137072846647	0\\
24.9132139504052	0\\
24.9127205881241	0\\
24.9122271978182	0\\
24.9117337794843	0\\
24.9112403331192	0\\
24.9107468587199	0\\
24.910253356283	0\\
24.9097598258054	0\\
24.9092662672839	0\\
24.9087726807153	0\\
24.9082790660965	0\\
24.9077854234242	0\\
24.9072917526953	0\\
24.9067980539065	0\\
24.9063043270547	0\\
24.9058105721367	0\\
24.9053167891492	0\\
24.9048229780892	0\\
24.9043291389534	0\\
24.9038352717386	0\\
24.9033413764416	0\\
24.9028474530592	0\\
24.9023535015883	0\\
24.9018595220256	0\\
24.901365514368	0\\
24.9008714786122	0\\
24.9003774147551	0\\
24.8998833227934	0\\
24.899389202724	0\\
24.8988950545436	0\\
24.8984008782491	0\\
24.8979066738372	0\\
24.8974124413048	0\\
24.8969181806487	0\\
24.8964238918657	0\\
24.8959295749524	0\\
24.8954352299059	0\\
24.8949408567228	0\\
24.8944464554	0\\
24.8939520259342	0\\
24.8934575683222	0\\
24.8929630825609	0\\
24.892468568647	0\\
24.8919740265774	0\\
24.8914794563487	0\\
24.8909848579579	0\\
24.8904902314017	0\\
24.8899955766769	0\\
24.8895008937802	0\\
24.8890061827086	0\\
24.8885114434587	0\\
24.8880166760274	0\\
24.8875218804114	0\\
24.8870270566076	0\\
24.8865322046127	0\\
24.8860373244235	0\\
24.8855424160369	0\\
24.8850474794495	0\\
24.8845525146582	0\\
24.8840575216597	0\\
24.8835625004509	0\\
24.8830674510285	0\\
24.8825723733893	0\\
24.8820772675302	0\\
24.8815821334478	0\\
24.8810869711389	0\\
24.8805917806004	0\\
24.8800965618291	0\\
24.8796013148216	0\\
24.8791060395748	0\\
24.8786107360855	0\\
24.8781154043504	0\\
24.8776200443664	0\\
24.8771246561301	0\\
24.8766292396384	0\\
24.8761337948881	0\\
24.8756383218759	0\\
24.8751428205986	0\\
24.874647291053	0\\
24.8741517332358	0\\
24.8736561471439	0\\
24.8731605327739	0\\
24.8726648901228	0\\
24.8721692191871	0\\
24.8716735199638	0\\
24.8711777924496	0\\
24.8706820366412	0\\
24.8701862525355	0\\
24.8696904401292	0\\
24.869194599419	0\\
24.8686987304018	0\\
24.8682028330742	0\\
24.8677069074332	0\\
24.8672109534753	0\\
24.8667149711975	0\\
24.8662189605964	0\\
24.8657229216689	0\\
24.8652268544117	0\\
24.8647307588215	0\\
24.8642346348952	0\\
24.8637384826294	0\\
24.863242302021	0\\
24.8627460930667	0\\
24.8622498557633	0\\
24.8617535901075	0\\
24.8612572960961	0\\
24.8607609737259	0\\
24.8602646229936	0\\
24.8597682438959	0\\
24.8592718364297	0\\
24.8587754005916	0\\
24.8582789363785	0\\
24.8577824437871	0\\
24.8572859228141	0\\
24.8567893734564	0\\
24.8562927957106	0\\
24.8557961895735	0\\
24.8552995550419	0\\
24.8548028921126	0\\
24.8543062007822	0\\
24.8538094810475	0\\
24.8533127329053	0\\
24.8528159563523	0\\
24.8523191513853	0\\
24.851822318001	0\\
24.8513254561962	0\\
24.8508285659677	0\\
24.8503316473121	0\\
24.8498347002262	0\\
24.8493377247068	0\\
24.8488407207506	0\\
24.8483436883544	0\\
24.8478466275148	0\\
24.8473495382288	0\\
24.8468524204929	0\\
24.8463552743039	0\\
24.8458580996586	0\\
24.8453608965538	0\\
24.8448636649861	0\\
24.8443664049522	0\\
24.8438691164491	0\\
24.8433717994733	0\\
24.8428744540217	0\\
24.8423770800909	0\\
24.8418796776777	0\\
24.8413822467788	0\\
24.840884787391	0\\
24.840387299511	0\\
24.8398897831356	0\\
24.8393922382615	0\\
24.8388946648853	0\\
24.838397063004	0\\
24.8378994326141	0\\
24.8374017737124	0\\
24.8369040862957	0\\
24.8364063703606	0\\
24.835908625904	0\\
24.8354108529225	0\\
24.8349130514129	0\\
24.8344152213719	0\\
24.8339173627962	0\\
24.8334194756826	0\\
24.8329215600278	0\\
24.8324236158285	0\\
24.8319256430814	0\\
24.8314276417833	0\\
24.830929611931	0\\
24.830431553521	0\\
24.8299334665502	0\\
24.8294353510153	0\\
24.8289372069129	0\\
24.8284390342399	0\\
24.8279408329929	0\\
24.8274426031687	0\\
24.826944344764	0\\
24.8264460577755	0\\
24.8259477421999	0\\
24.8254493980339	0\\
24.8249510252743	0\\
24.8244526239178	0\\
24.8239541939611	0\\
24.8234557354009	0\\
24.822957248234	0\\
24.822458732457	0\\
24.8219601880667	0\\
24.8214616150597	0\\
24.8209630134329	0\\
24.8204643831829	0\\
24.8199657243063	0\\
24.8194670368001	0\\
24.8189683206608	0\\
24.8184695758851	0\\
24.8179708024698	0\\
24.8174720004116	0\\
24.8169731697072	0\\
24.8164743103533	0\\
24.8159754223466	0\\
24.8154765056839	0\\
24.8149775603617	0\\
24.8144785863769	0\\
24.8139795837262	0\\
24.8134805524061	0\\
24.8129814924136	0\\
24.8124824037452	0\\
24.8119832863977	0\\
24.8114841403677	0\\
24.810984965652	0\\
24.8104857622473	0\\
24.8099865301502	0\\
24.8094872693576	0\\
24.808987979866	0\\
24.8084886616722	0\\
24.8079893147728	0\\
24.8074899391647	0\\
24.8069905348444	0\\
24.8064911018087	0\\
24.8059916400543	0\\
24.8054921495778	0\\
24.804992630376	0\\
24.8044930824456	0\\
24.8039935057832	0\\
24.8034939003855	0\\
24.8029942662494	0\\
24.8024946033713	0\\
24.8019949117481	0\\
24.8014951913764	0\\
24.800995442253	0\\
24.8004956643744	0\\
24.7999958577374	0\\
24.7994960223388	0\\
24.7989961581751	0\\
24.798496265243	0\\
24.7979963435394	0\\
24.7974963930607	0\\
24.7969964138038	0\\
24.7964964057654	0\\
24.795996368942	0\\
24.7954963033304	0\\
24.7949962089273	0\\
24.7944960857293	0\\
24.7939959337332	0\\
24.7934957529356	0\\
24.7929955433332	0\\
24.7924953049228	0\\
24.7919950377008	0\\
24.7914947416642	0\\
24.7909944168095	0\\
24.7904940631334	0\\
24.7899936806325	0\\
24.7894932693037	0\\
24.7889928291435	0\\
24.7884923601486	0\\
24.7879918623157	0\\
24.7874913356416	0\\
24.7869907801227	0\\
24.7864901957559	0\\
24.7859895825379	0\\
24.7854889404651	0\\
24.7849882695345	0\\
24.7844875697426	0\\
24.783986841086	0\\
24.7834860835616	0\\
24.7829852971659	0\\
24.7824844818956	0\\
24.7819836377473	0\\
24.7814827647179	0\\
24.7809818628038	0\\
24.7804809320019	0\\
24.7799799723087	0\\
24.7794789837209	0\\
24.7789779662352	0\\
24.7784769198483	0\\
24.7779758445568	0\\
24.7774747403574	0\\
24.7769736072467	0\\
24.7764724452215	0\\
24.7759712542783	0\\
24.7754700344139	0\\
24.7749687856249	0\\
24.7744675079079	0\\
24.7739662012597	0\\
24.7734648656769	0\\
24.7729635011561	0\\
24.772462107694	0\\
24.7719606852873	0\\
24.7714592339326	0\\
24.7709577536266	0\\
24.7704562443659	0\\
24.7699547061472	0\\
24.7694531389671	0\\
24.7689515428224	0\\
24.7684499177096	0\\
24.7679482636254	0\\
24.7674465805665	0\\
24.7669448685295	0\\
24.7664431275111	0\\
24.7659413575079	0\\
24.7654395585165	0\\
24.7649377305337	0\\
24.7644358735561	0\\
24.7639339875803	0\\
24.7634320726029	0\\
24.7629301286207	0\\
24.7624281556302	0\\
24.7619261536282	0\\
24.7614241226112	0\\
24.7609220625759	0\\
24.760419973519	0\\
24.759917855437	0\\
24.7594157083267	0\\
24.7589135321847	0\\
24.7584113270076	0\\
24.7579090927921	0\\
24.7574068295348	0\\
24.7569045372323	0\\
24.7564022158813	0\\
24.7558998654785	0\\
24.7553974860204	0\\
24.7548950775037	0\\
24.7543926399251	0\\
24.7538901732811	0\\
24.7533876775685	0\\
24.7528851527838	0\\
24.7523825989238	0\\
24.7518800159849	0\\
24.751377403964	0\\
24.7508747628575	0\\
24.7503720926621	0\\
24.7498693933746	0\\
24.7493666649914	0\\
24.7488639075092	0\\
24.7483611209247	0\\
24.7478583052345	0\\
24.7473554604352	0\\
24.7468525865235	0\\
24.7463496834959	0\\
24.7458467513492	0\\
24.7453437900799	0\\
24.7448407996846	0\\
24.74433778016	0\\
24.7438347315028	0\\
24.7433316537095	0\\
24.7428285467767	0\\
24.7423254107011	0\\
24.7418222454794	0\\
24.7413190511081	0\\
24.7408158275838	0\\
24.7403125749032	0\\
24.739809293063	0\\
24.7393059820596	0\\
24.7388026418898	0\\
24.7382992725501	0\\
24.7377958740372	0\\
24.7372924463477	0\\
24.7367889894783	0\\
24.7362855034254	0\\
24.7357819881859	0\\
24.7352784437561	0\\
24.7347748701329	0\\
24.7342712673127	0\\
24.7337676352923	0\\
24.7332639740682	0\\
24.732760283637	0\\
24.7322565639954	0\\
24.7317528151399	0\\
24.7312490370672	0\\
24.7307452297739	0\\
24.7302413932565	0\\
24.7297375275118	0\\
24.7292336325363	0\\
24.7287297083266	0\\
24.7282257548793	0\\
24.7277217721911	0\\
24.7272177602585	0\\
24.7267137190781	0\\
24.7262096486466	0\\
24.7257055489606	0\\
24.7252014200166	0\\
24.7246972618112	0\\
24.7241930743412	0\\
24.723688857603	0\\
24.7231846115933	0\\
24.7226803363086	0\\
24.7221760317456	0\\
24.7216716979009	0\\
24.7211673347711	0\\
24.7206629423527	0\\
24.7201585206424	0\\
24.7196540696368	0\\
24.7191495893324	0\\
24.7186450797259	0\\
24.7181405408138	0\\
24.7176359725928	0\\
24.7171313750595	0\\
24.7166267482104	0\\
24.7161220920421	0\\
24.7156174065512	0\\
24.7151126917344	0\\
24.7146079475881	0\\
24.7141031741091	0\\
24.7135983712938	0\\
24.713093539139	0\\
24.7125886776411	0\\
24.7120837867968	0\\
24.7115788666026	0\\
24.7110739170551	0\\
24.710568938151	0\\
24.7100639298868	0\\
24.709558892259	0\\
24.7090538252644	0\\
24.7085487288994	0\\
24.7080436031606	0\\
24.7075384480447	0\\
24.7070332635482	0\\
24.7065280496677	0\\
24.7060228063998	0\\
24.705517533741	0\\
24.705012231688	0\\
24.7045069002373	0\\
24.7040015393855	0\\
24.7034961491292	0\\
24.7029907294649	0\\
24.7024852803892	0\\
24.7019798018988	0\\
24.7014742939901	0\\
24.7009687566598	0\\
24.7004631899045	0\\
24.6999575937206	0\\
24.6994519681048	0\\
24.6989463130537	0\\
24.6984406285638	0\\
24.6979349146318	0\\
24.6974291712541	0\\
24.6969233984273	0\\
24.696417596148	0\\
24.6959117644128	0\\
24.6954059032183	0\\
24.694900012561	0\\
24.6943940924375	0\\
24.6938881428444	0\\
24.6933821637781	0\\
24.6928761552354	0\\
24.6923701172127	0\\
24.6918640497066	0\\
24.6913579527137	0\\
24.6908518262306	0\\
24.6903456702537	0\\
24.6898394847798	0\\
24.6893332698052	0\\
24.6888270253267	0\\
24.6883207513407	0\\
24.6878144478438	0\\
24.6873081148326	0\\
24.6868017523037	0\\
24.6862953602535	0\\
24.6857889386787	0\\
24.6852824875758	0\\
24.6847760069413	0\\
24.6842694967719	0\\
24.683762957064	0\\
24.6832563878143	0\\
24.6827497890193	0\\
24.6822431606754	0\\
24.6817365027794	0\\
24.6812298153277	0\\
24.6807230983169	0\\
24.6802163517436	0\\
24.6797095756042	0\\
24.6792027698954	0\\
24.6786959346137	0\\
24.6781890697556	0\\
24.6776821753177	0\\
24.6771752512966	0\\
24.6766682976887	0\\
24.6761613144906	0\\
24.675654301699	0\\
24.6751472593102	0\\
24.6746401873209	0\\
24.6741330857276	0\\
24.6736259545269	0\\
24.6731187937153	0\\
24.6726116032893	0\\
24.6721043832454	0\\
24.6715971335803	0\\
24.6710898542905	0\\
24.6705825453724	0\\
24.6700752068227	0\\
24.6695678386378	0\\
24.6690604408144	0\\
24.6685530133489	0\\
24.6680455562379	0\\
24.6675380694779	0\\
24.6670305530655	0\\
24.6665230069971	0\\
24.6660154312694	0\\
24.6655078258789	0\\
24.665000190822	0\\
24.6644925260954	0\\
24.6639848316955	0\\
24.6634771076189	0\\
24.6629693538621	0\\
24.6624615704216	0\\
24.6619537572941	0\\
24.6614459144759	0\\
24.6609380419637	0\\
24.6604301397539	0\\
24.6599222078431	0\\
24.6594142462278	0\\
24.6589062549045	0\\
24.6583982338699	0\\
24.6578901831203	0\\
24.6573821026523	0\\
24.6568739924624	0\\
24.6563658525472	0\\
24.6558576829032	0\\
24.6553494835268	0\\
24.6548412544147	0\\
24.6543329955634	0\\
24.6538247069692	0\\
24.6533163886289	0\\
24.6528080405389	0\\
24.6522996626956	0\\
24.6517912550957	0\\
24.6512828177357	0\\
24.650774350612	0\\
24.6502658537212	0\\
24.6497573270597	0\\
24.6492487706242	0\\
24.6487401844111	0\\
24.6482315684169	0\\
24.6477229226382	0\\
24.6472142470714	0\\
24.6467055417131	0\\
24.6461968065598	0\\
24.6456880416079	0\\
24.645179246854	0\\
24.6446704222947	0\\
24.6441615679263	0\\
24.6436526837455	0\\
24.6431437697487	0\\
24.6426348259324	0\\
24.6421258522932	0\\
24.6416168488274	0\\
24.6411078155318	0\\
24.6405987524027	0\\
24.6400896594366	0\\
24.6395805366301	0\\
24.6390713839796	0\\
24.6385622014817	0\\
24.6380529891329	0\\
24.6375437469296	0\\
24.6370344748683	0\\
24.6365251729456	0\\
24.636015841158	0\\
24.6355064795019	0\\
24.6349970879738	0\\
24.6344876665703	0\\
24.6339782152879	0\\
24.6334687341229	0\\
24.632959223072	0\\
24.6324496821316	0\\
24.6319401112982	0\\
24.6314305105683	0\\
24.6309208799384	0\\
24.630411219405	0\\
24.6299015289646	0\\
24.6293918086136	0\\
24.6288820583486	0\\
24.6283722781661	0\\
24.6278624680624	0\\
24.6273526280342	0\\
24.6268427580779	0\\
24.62633285819	0\\
24.625822928367	0\\
24.6253129686053	0\\
24.6248029789015	0\\
24.624292959252	0\\
24.6237829096534	0\\
24.623272830102	0\\
24.6227627205944	0\\
24.622252581127	0\\
24.6217424116965	0\\
24.6212322122991	0\\
24.6207219829314	0\\
24.6202117235899	0\\
24.6197014342711	0\\
24.6191911149714	0\\
24.6186807656873	0\\
24.6181703864153	0\\
24.6176599771518	0\\
24.6171495378935	0\\
24.6166390686366	0\\
24.6161285693777	0\\
24.6156180401132	0\\
24.6151074808397	0\\
24.6145968915536	0\\
24.6140862722514	0\\
24.6135756229295	0\\
24.6130649435844	0\\
24.6125542342126	0\\
24.6120434948106	0\\
24.6115327253748	0\\
24.6110219259017	0\\
24.6105110963877	0\\
24.6100002368294	0\\
24.6094893472232	0\\
24.6089784275655	0\\
24.6084674778528	0\\
24.6079564980816	0\\
24.6074454882484	0\\
24.6069344483495	0\\
24.6064233783816	0\\
24.6059122783409	0\\
24.6054011482241	0\\
24.6048899880275	0\\
24.6043787977476	0\\
24.6038675773809	0\\
24.6033563269238	0\\
24.6028450463728	0\\
24.6023337357244	0\\
24.6018223949749	0\\
24.6013110241209	0\\
24.6007996231588	0\\
24.6002881920851	0\\
24.5997767308962	0\\
24.5992652395886	0\\
24.5987537181587	0\\
24.5982421666029	0\\
24.5977305849178	0\\
24.5972189730998	0\\
24.5967073311453	0\\
24.5961956590508	0\\
24.5956839568127	0\\
24.5951722244275	0\\
24.5946604618916	0\\
24.5941486692014	0\\
24.5936368463535	0\\
24.5931249933443	0\\
24.5926131101701	0\\
24.5921011968275	0\\
24.5915892533129	0\\
24.5910772796227	0\\
24.5905652757534	0\\
24.5900532417015	0\\
24.5895411774633	0\\
24.5890290830353	0\\
24.5885169584139	0\\
24.5880048035956	0\\
24.5874926185769	0\\
24.5869804033541	0\\
24.5864681579237	0\\
24.5859558822821	0\\
24.5854435764258	0\\
24.5849312403512	0\\
24.5844188740548	0\\
24.5839064775329	0\\
24.583394050782	0\\
24.5828815937986	0\\
24.582369106579	0\\
24.5818565891198	0\\
24.5813440414173	0\\
24.5808314634679	0\\
24.5803188552682	0\\
24.5798062168145	0\\
24.5792935481032	0\\
24.5787808491308	0\\
24.5782681198938	0\\
24.5777553603884	0\\
24.5772425706113	0\\
24.5767297505587	0\\
24.5762169002271	0\\
24.575704019613	0\\
24.5751911087127	0\\
24.5746781675227	0\\
24.5741651960395	0\\
24.5736521942593	0\\
24.5731391621787	0\\
24.5726260997941	0\\
24.5721130071018	0\\
24.5715998840984	0\\
24.5710867307802	0\\
24.5705735471436	0\\
24.5700603331851	0\\
24.5695470889011	0\\
24.569033814288	0\\
24.5685205093422	0\\
24.5680071740601	0\\
24.5674938084381	0\\
24.5669804124727	0\\
24.5664669861603	0\\
24.5659535294973	0\\
24.56544004248	0\\
24.5649265251049	0\\
24.5644129773685	0\\
24.563899399267	0\\
24.563385790797	0\\
24.5628721519548	0\\
24.5623584827369	0\\
24.5618447831396	0\\
24.5613310531594	0\\
24.5608172927926	0\\
24.5603035020358	0\\
24.5597896808851	0\\
24.5592758293372	0\\
24.5587619473883	0\\
24.558248035035	0\\
24.5577340922735	0\\
24.5572201191003	0\\
24.5567061155117	0\\
24.5561920815043	0\\
24.5556780170744	0\\
24.5551639222183	0\\
24.5546497969325	0\\
24.5541356412134	0\\
24.5536214550574	0\\
24.5531072384609	0\\
24.5525929914202	0\\
24.5520787139318	0\\
24.551564405992	0\\
24.5510500675973	0\\
24.550535698744	0\\
24.5500212994286	0\\
24.5495068696474	0\\
24.5489924093968	0\\
24.5484779186733	0\\
24.5479633974731	0\\
24.5474488457927	0\\
24.5469342636285	0\\
24.5464196509769	0\\
24.5459050078342	0\\
24.5453903341969	0\\
24.5448756300613	0\\
24.5443608954237	0\\
24.5438461302807	0\\
24.5433313346286	0\\
24.5428165084637	0\\
24.5423016517825	0\\
24.5417867645813	0\\
24.5412718468565	0\\
24.5407568986045	0\\
24.5402419198217	0\\
24.5397269105044	0\\
24.5392118706491	0\\
24.538696800252	0\\
24.5381816993096	0\\
24.5376665678183	0\\
24.5371514057745	0\\
24.5366362131744	0\\
24.5361209900146	0\\
24.5356057362912	0\\
24.5350904520009	0\\
24.5345751371398	0\\
24.5340597917044	0\\
24.5335444156911	0\\
24.5330290090962	0\\
24.5325135719161	0\\
24.5319981041471	0\\
24.5314826057857	0\\
24.5309670768282	0\\
24.530451517271	0\\
24.5299359271104	0\\
24.5294203063428	0\\
24.5289046549646	0\\
24.5283889729721	0\\
24.5278732603617	0\\
24.5273575171297	0\\
24.5268417432726	0\\
24.5263259387867	0\\
24.5258101036683	0\\
24.5252942379139	0\\
24.5247783415197	0\\
24.5242624144821	0\\
24.5237464567976	0\\
24.5232304684624	0\\
24.5227144494729	0\\
24.5221983998255	0\\
24.5216823195165	0\\
24.5211662085423	0\\
24.5206500668992	0\\
24.5201338945836	0\\
24.5196176915918	0\\
24.5191014579203	0\\
24.5185851935653	0\\
24.5180688985232	0\\
24.5175525727904	0\\
24.5170362163631	0\\
24.5165198292379	0\\
24.5160034114109	0\\
24.5154869628786	0\\
24.5149704836373	0\\
24.5144539736833	0\\
24.5139374330131	0\\
24.5134208616229	0\\
24.5129042595091	0\\
24.512387626668	0\\
24.511870963096	0\\
24.5113542687894	0\\
24.5108375437446	0\\
24.5103207879579	0\\
24.5098040014257	0\\
24.5092871841442	0\\
24.5087703361099	0\\
24.5082534573191	0\\
24.5077365477681	0\\
24.5072196074532	0\\
24.5067026363708	0\\
24.5061856345173	0\\
24.5056686018889	0\\
24.505151538482	0\\
24.5046344442929	0\\
24.504117319318	0\\
24.5036001635537	0\\
24.5030829769961	0\\
24.5025657596418	0\\
24.5020485114869	0\\
24.5015312325278	0\\
24.501013922761	0\\
24.5004965821826	0\\
24.499979210789	0\\
24.4994618085767	0\\
24.4989443755417	0\\
24.4984269116806	0\\
24.4979094169896	0\\
24.4973918914651	0\\
24.4968743351034	0\\
24.4963567479008	0\\
24.4958391298536	0\\
24.4953214809582	0\\
24.4948038012109	0\\
24.494286090608	0\\
24.4937683491457	0\\
24.4932505768206	0\\
24.4927327736288	0\\
24.4922149395667	0\\
24.4916970746306	0\\
24.4911791788168	0\\
24.4906612521217	0\\
24.4901432945415	0\\
24.4896253060727	0\\
24.4891072867114	0\\
24.488589236454	0\\
24.4880711552968	0\\
24.4875530432361	0\\
24.4870349002683	0\\
24.4865167263897	0\\
24.4859985215965	0\\
24.4854802858852	0\\
24.4849620192519	0\\
24.484443721693	0\\
24.4839253932048	0\\
24.4834070337836	0\\
24.4828886434258	0\\
24.4823702221276	0\\
24.4818517698854	0\\
24.4813332866953	0\\
24.4808147725539	0\\
24.4802962274573	0\\
24.4797776514018	0\\
24.4792590443839	0\\
24.4787404063996	0\\
24.4782217374455	0\\
24.4777030375177	0\\
24.4771843066126	0\\
24.4766655447265	0\\
24.4761467518556	0\\
24.4756279279963	0\\
24.4751090731448	0\\
24.4745901872975	0\\
24.4740712704507	0\\
24.4735523226006	0\\
24.4730333437436	0\\
24.4725143338759	0\\
24.4719952929939	0\\
24.4714762210937	0\\
24.4709571181719	0\\
24.4704379842245	0\\
24.4699188192479	0\\
24.4693996232384	0\\
24.4688803961923	0\\
24.4683611381059	0\\
24.4678418489754	0\\
24.4673225287972	0\\
24.4668031775675	0\\
24.4662837952827	0\\
24.465764381939	0\\
24.4652449375326	0\\
24.46472546206	0\\
24.4642059555173	0\\
24.4636864179008	0\\
24.4631668492069	0\\
24.4626472494318	0\\
24.4621276185717	0\\
24.4616079566231	0\\
24.4610882635821	0\\
24.460568539445	0\\
24.4600487842082	0\\
24.4595289978678	0\\
24.4590091804202	0\\
24.4584893318617	0\\
24.4579694521884	0\\
24.4574495413968	0\\
24.456929599483	0\\
24.4564096264434	0\\
24.4558896222741	0\\
24.4553695869716	0\\
24.454849520532	0\\
24.4543294229517	0\\
24.4538092942268	0\\
24.4532891343537	0\\
24.4527689433287	0\\
24.452248721148	0\\
24.4517284678078	0\\
24.4512081833045	0\\
24.4506878676343	0\\
24.4501675207934	0\\
24.4496471427782	0\\
24.4491267335849	0\\
24.4486062932098	0\\
24.4480858216491	0\\
24.4475653188991	0\\
24.4470447849561	0\\
24.4465242198162	0\\
24.4460036234759	0\\
24.4454829959313	0\\
24.4449623371786	0\\
24.4444416472143	0\\
24.4439209260344	0\\
24.4434001736353	0\\
24.4428793900132	0\\
24.4423585751644	0\\
24.4418377290852	0\\
24.4413168517717	0\\
24.4407959432203	0\\
24.4402750034272	0\\
24.4397540323886	0\\
24.4392330301008	0\\
24.4387119965601	0\\
24.4381909317627	0\\
24.4376698357048	0\\
24.4371487083827	0\\
24.4366275497927	0\\
24.436106359931	0\\
24.4355851387938	0\\
24.4350638863774	0\\
24.434542602678	0\\
24.4340212876919	0\\
24.4334999414154	0\\
24.4329785638446	0\\
24.4324571549758	0\\
24.4319357148053	0\\
24.4314142433292	0\\
24.4308927405439	0\\
24.4303712064456	0\\
24.4298496410305	0\\
24.4293280442948	0\\
24.4288064162348	0\\
24.4282847568468	0\\
24.4277630661269	0\\
24.4272413440714	0\\
24.4267195906766	0\\
24.4261978059386	0\\
24.4256759898537	0\\
24.4251541424182	0\\
24.4246322636282	0\\
24.4241103534801	0\\
24.42358841197	0\\
24.4230664390941	0\\
24.4225444348487	0\\
24.4220223992301	0\\
24.4215003322344	0\\
24.4209782338579	0\\
24.4204561040968	0\\
24.4199339429474	0\\
24.4194117504058	0\\
24.4188895264683	0\\
24.4183672711311	0\\
24.4178449843904	0\\
24.4173226662425	0\\
24.4168003166835	0\\
24.4162779357098	0\\
24.4157555233175	0\\
24.4152330795028	0\\
24.414710604262	0\\
24.4141880975913	0\\
24.4136655594869	0\\
24.413142989945	0\\
24.4126203889618	0\\
24.4120977565336	0\\
24.4115750926566	0\\
24.4110523973269	0\\
24.4105296705409	0\\
24.4100069122946	0\\
24.4094841225844	0\\
24.4089613014064	0\\
24.4084384487568	0\\
24.4079155646319	0\\
24.4073926490279	0\\
24.4068697019409	0\\
24.4063467233673	0\\
24.4058237133031	0\\
24.4053006717446	0\\
24.4047775986881	0\\
24.4042544941296	0\\
24.4037313580655	0\\
24.4032081904919	0\\
24.402684991405	0\\
24.4021617608011	0\\
24.4016384986763	0\\
24.4011152050268	0\\
24.4005918798489	0\\
24.4000685231387	0\\
24.3995451348925	0\\
24.3990217151063	0\\
24.3984982637766	0\\
24.3979747808993	0\\
24.3974512664708	0\\
24.3969277204872	0\\
24.3964041429447	0\\
24.3958805338396	0\\
24.395356893168	0\\
24.394833220926	0\\
24.39430951711	0\\
24.3937857817161	0\\
24.3932620147404	0\\
24.3927382161793	0\\
24.3922143860288	0\\
24.3916905242851	0\\
24.3911666309445	0\\
24.3906427060031	0\\
24.3901187494571	0\\
24.3895947613028	0\\
24.3890707415362	0\\
24.3885466901536	0\\
24.3880226071512	0\\
24.3874984925251	0\\
24.3869743462716	0\\
24.3864501683867	0\\
24.3859259588668	0\\
24.3854017177079	0\\
24.3848774449063	0\\
24.3843531404581	0\\
24.3838288043596	0\\
24.3833044366068	0\\
24.382780037196	0\\
24.3822556061234	0\\
24.381731143385	0\\
24.3812066489772	0\\
24.3806821228961	0\\
24.3801575651378	0\\
24.3796329756985	0\\
24.3791083545744	0\\
24.3785837017617	0\\
24.3780590172566	0\\
24.3775343010551	0\\
24.3770095531536	0\\
24.3764847735481	0\\
24.3759599622347	0\\
24.3754351192098	0\\
24.3749102444695	0\\
24.3743853380099	0\\
24.3738603998271	0\\
24.3733354299174	0\\
24.372810428277	0\\
24.3722853949019	0\\
24.3717603297883	0\\
24.3712352329325	0\\
24.3707101043306	0\\
24.3701849439786	0\\
24.3696597518729	0\\
24.3691345280095	0\\
24.3686092723846	0\\
24.3680839849944	0\\
24.3675586658351	0\\
24.3670333149027	0\\
24.3665079321934	0\\
24.3659825177035	0\\
24.3654570714289	0\\
24.364931593366	0\\
24.3644060835109	0\\
24.3638805418597	0\\
24.3633549684085	0\\
24.3628293631536	0\\
24.362303726091	0\\
24.3617780572169	0\\
24.3612523565275	0\\
24.3607266240189	0\\
24.3602008596873	0\\
24.3596750635287	0\\
24.3591492355395	0\\
24.3586233757156	0\\
24.3580974840533	0\\
24.3575715605487	0\\
24.3570456051979	0\\
24.356519617997	0\\
24.3559935989423	0\\
24.3554675480299	0\\
24.3549414652559	0\\
24.3544153506164	0\\
24.3538892041076	0\\
24.3533630257256	0\\
24.3528368154665	0\\
24.3523105733266	0\\
24.3517842993019	0\\
24.3512579933886	0\\
24.3507316555827	0\\
24.3502052858805	0\\
24.3496788842781	0\\
24.3491524507716	0\\
24.3486259853571	0\\
24.3480994880308	0\\
24.3475729587888	0\\
24.3470463976273	0\\
24.3465198045422	0\\
24.3459931795299	0\\
24.3454665225864	0\\
24.3449398337079	0\\
24.3444131128904	0\\
24.3438863601301	0\\
24.3433595754231	0\\
24.3428327587656	0\\
24.3423059101537	0\\
24.3417790295834	0\\
24.341252117051	0\\
24.3407251725525	0\\
24.3401981960841	0\\
24.3396711876419	0\\
24.3391441472219	0\\
24.3386170748204	0\\
24.3380899704335	0\\
24.3375628340571	0\\
24.3370356656876	0\\
24.336508465321	0\\
24.3359812329533	0\\
24.3354539685808	0\\
24.3349266721995	0\\
24.3343993438056	0\\
24.3338719833951	0\\
24.3333445909642	0\\
24.332817166509	0\\
24.3322897100255	0\\
24.33176222151	0\\
24.3312347009585	0\\
24.3307071483672	0\\
24.330179563732	0\\
24.3296519470492	0\\
24.3291242983148	0\\
24.328596617525	0\\
24.3280689046759	0\\
24.3275411597635	0\\
24.3270133827839	0\\
24.3264855737334	0\\
24.3259577326078	0\\
24.3254298594035	0\\
24.3249019541164	0\\
24.3243740167427	0\\
24.3238460472785	0\\
24.3233180457199	0\\
24.3227900120629	0\\
24.3222619463036	0\\
24.3217338484383	0\\
24.3212057184628	0\\
24.3206775563735	0\\
24.3201493621662	0\\
24.3196211358373	0\\
24.3190928773826	0\\
24.3185645867984	0\\
24.3180362640806	0\\
24.3175079092255	0\\
24.3169795222291	0\\
24.3164511030874	0\\
24.3159226517967	0\\
24.3153941683529	0\\
24.3148656527521	0\\
24.3143371049905	0\\
24.3138085250641	0\\
24.313279912969	0\\
24.3127512687012	0\\
24.312222592257	0\\
24.3116938836323	0\\
24.3111651428232	0\\
24.3106363698259	0\\
24.3101075646364	0\\
24.3095787272507	0\\
24.309049857665	0\\
24.3085209558753	0\\
24.3079920218778	0\\
24.3074630556684	0\\
24.3069340572434	0\\
24.3064050265986	0\\
24.3058759637303	0\\
24.3053468686344	0\\
24.3048177413072	0\\
24.3042885817446	0\\
24.3037593899426	0\\
24.3032301658975	0\\
24.3027009096052	0\\
24.3021716210619	0\\
24.3016423002635	0\\
24.3011129472062	0\\
24.300583561886	0\\
24.300054144299	0\\
24.2995246944413	0\\
24.2989952123089	0\\
24.2984656978979	0\\
24.2979361512044	0\\
24.2974065722244	0\\
24.296876960954	0\\
24.2963473173892	0\\
24.2958176415261	0\\
24.2952879333608	0\\
24.2947581928893	0\\
24.2942284201077	0\\
24.293698615012	0\\
24.2931687775984	0\\
24.2926389078628	0\\
24.2921090058013	0\\
24.29157907141	0\\
24.2910491046849	0\\
24.2905191056221	0\\
24.2899890742176	0\\
24.2894590104675	0\\
24.2889289143679	0\\
24.2883987859147	0\\
24.2878686251041	0\\
24.2873384319321	0\\
24.2868082063947	0\\
24.286277948488	0\\
24.285747658208	0\\
24.2852173355508	0\\
24.2846869805125	0\\
24.284156593089	0\\
24.2836261732764	0\\
24.2830957210708	0\\
24.2825652364682	0\\
24.2820347194646	0\\
24.2815041700562	0\\
24.2809735882388	0\\
24.2804429740087	0\\
24.2799123273617	0\\
24.279381648294	0\\
24.2788509368016	0\\
24.2783201928805	0\\
24.2777894165267	0\\
24.2772586077364	0\\
24.2767277665055	0\\
24.27619689283	0\\
24.2756659867061	0\\
24.2751350481297	0\\
24.2746040770968	0\\
24.2740730736036	0\\
24.273542037646	0\\
24.27301096922	0\\
24.2724798683217	0\\
24.2719487349472	0\\
24.2714175690924	0\\
24.2708863707533	0\\
24.2703551399261	0\\
24.2698238766067	0\\
24.2692925807911	0\\
24.2687612524754	0\\
24.2682298916556	0\\
24.2676984983278	0\\
24.2671670724878	0\\
24.2666356141319	0\\
24.2661041232559	0\\
24.2655725998559	0\\
24.2650410439279	0\\
24.264509455468	0\\
24.2639778344722	0\\
24.2634461809364	0\\
24.2629144948567	0\\
24.2623827762291	0\\
24.2618510250497	0\\
24.2613192413144	0\\
24.2607874250192	0\\
24.2602555761602	0\\
24.2597236947334	0\\
24.2591917807347	0\\
24.2586598341603	0\\
24.258127855006	0\\
24.257595843268	0\\
24.2570637989422	0\\
24.2565317220247	0\\
24.2559996125113	0\\
24.2554674703983	0\\
24.2549352956814	0\\
24.2544030883568	0\\
24.2538708484205	0\\
24.2533385758685	0\\
24.2528062706967	0\\
24.2522739329011	0\\
24.2517415624779	0\\
24.2512091594229	0\\
24.2506767237321	0\\
24.2501442554017	0\\
24.2496117544275	0\\
24.2490792208055	0\\
24.2485466545318	0\\
24.2480140556024	0\\
24.2474814240132	0\\
24.2469487597602	0\\
24.2464160628395	0\\
24.245883333247	0\\
24.2453505709788	0\\
24.2448177760307	0\\
24.2442849483988	0\\
24.2437520880792	0\\
24.2432191950677	0\\
24.2426862693603	0\\
24.2421533109532	0\\
24.2416203198422	0\\
24.2410872960233	0\\
24.2405542394925	0\\
24.2400211502458	0\\
24.2394880282792	0\\
24.2389548735887	0\\
24.2384216861702	0\\
24.2378884660197	0\\
24.2373552131333	0\\
24.2368219275068	0\\
24.2362886091364	0\\
24.2357552580178	0\\
24.2352218741472	0\\
24.2346884575205	0\\
24.2341550081337	0\\
24.2336215259828	0\\
24.2330880110636	0\\
24.2325544633723	0\\
24.2320208829048	0\\
24.231487269657	0\\
24.230953623625	0\\
24.2304199448046	0\\
24.2298862331919	0\\
24.2293524887828	0\\
24.2288187115734	0\\
24.2282849015595	0\\
24.2277510587372	0\\
24.2272171831024	0\\
24.226683274651	0\\
24.2261493333791	0\\
24.2256153592826	0\\
24.2250813523574	0\\
24.2245473125996	0\\
24.2240132400051	0\\
24.2234791345699	0\\
24.2229449962898	0\\
24.222410825161	0\\
24.2218766211793	0\\
24.2213423843406	0\\
24.2208081146411	0\\
24.2202738120765	0\\
24.2197394766429	0\\
24.2192051083362	0\\
24.2186707071524	0\\
24.2181362730875	0\\
24.2176018061373	0\\
24.2170673062979	0\\
24.2165327735651	0\\
24.215998207935	0\\
24.2154636094035	0\\
24.2149289779665	0\\
24.21439431362	0\\
24.21385961636	0\\
24.2133248861823	0\\
24.212790123083	0\\
24.212255327058	0\\
24.2117204981031	0\\
24.2111856362145	0\\
24.2106507413879	0\\
24.2101158136194	0\\
24.209580852905	0\\
24.2090458592404	0\\
24.2085108326217	0\\
24.2079757730449	0\\
24.2074406805058	0\\
24.2069055550004	0\\
24.2063703965246	0\\
24.2058352050745	0\\
24.2052999806458	0\\
24.2047647232345	0\\
24.2042294328367	0\\
24.2036941094481	0\\
24.2031587530648	0\\
24.2026233636827	0\\
24.2020879412977	0\\
24.2015524859057	0\\
24.2010169975027	0\\
24.2004814760847	0\\
24.1999459216474	0\\
24.1994103341869	0\\
24.198874713699	0\\
24.1983390601798	0\\
24.1978033736252	0\\
24.1972676540309	0\\
24.1967319013931	0\\
24.1961961157076	0\\
24.1956602969703	0\\
24.1951244451772	0\\
24.1945885603241	0\\
24.194052642407	0\\
24.1935166914219	0\\
24.1929807073646	0\\
24.192444690231	0\\
24.1919086400171	0\\
24.1913725567188	0\\
24.190836440332	0\\
24.1903002908526	0\\
24.1897641082766	0\\
24.1892278925998	0\\
24.1886916438181	0\\
24.1881553619275	0\\
24.1876190469239	0\\
24.1870826988032	0\\
24.1865463175613	0\\
24.1860099031941	0\\
24.1854734556975	0\\
24.1849369750675	0\\
24.1844004612999	0\\
24.1838639143906	0\\
24.1833273343355	0\\
24.1827907211307	0\\
24.1822540747718	0\\
24.181717395255	0\\
24.1811806825759	0\\
24.1806439367307	0\\
24.1801071577151	0\\
24.1795703455251	0\\
24.1790335001565	0\\
24.1784966216053	0\\
24.1779597098673	0\\
24.1774227649385	0\\
24.1768857868148	0\\
24.1763487754919	0\\
24.175811730966	0\\
24.1752746532327	0\\
24.1747375422881	0\\
24.174200398128	0\\
24.1736632207483	0\\
24.1731260101449	0\\
24.1725887663138	0\\
24.1720514892506	0\\
24.1715141789515	0\\
24.1709768354122	0\\
24.1704394586287	0\\
24.1699020485967	0\\
24.1693646053123	0\\
24.1688271287713	0\\
24.1682896189696	0\\
24.1677520759031	0\\
24.1672144995676	0\\
24.166676889959	0\\
24.1661392470733	0\\
24.1656015709062	0\\
24.1650638614537	0\\
24.1645261187117	0\\
24.1639883426759	0\\
24.1634505333424	0\\
24.162912690707	0\\
24.1623748147655	0\\
24.1618369055139	0\\
24.161298962948	0\\
24.1607609870636	0\\
24.1602229778567	0\\
24.1596849353231	0\\
24.1591468594587	0\\
24.1586087502594	0\\
24.158070607721	0\\
24.1575324318394	0\\
24.1569942226105	0\\
24.1564559800302	0\\
24.1559177040942	0\\
24.1553793947985	0\\
24.1548410521389	0\\
24.1543026761113	0\\
24.1537642667116	0\\
24.1532258239356	0\\
24.1526873477792	0\\
24.1521488382383	0\\
24.1516102953086	0\\
24.1510717189861	0\\
24.1505331092667	0\\
24.1499944661461	0\\
24.1494557896203	0\\
24.148917079685	0\\
24.1483783363362	0\\
24.1478395595698	0\\
24.1473007493814	0\\
24.1467619057671	0\\
24.1462230287227	0\\
24.145684118244	0\\
24.1451451743268	0\\
24.1446061969671	0\\
24.1440671861606	0\\
24.1435281419032	0\\
24.1429890641909	0\\
24.1424499530193	0\\
24.1419108083844	0\\
24.1413716302819	0\\
24.1408324187079	0\\
24.140293173658	0\\
24.1397538951282	0\\
24.1392145831142	0\\
24.138675237612	0\\
24.1381358586173	0\\
24.137596446126	0\\
24.137057000134	0\\
24.136517520637	0\\
24.1359780076309	0\\
24.1354384611117	0\\
24.1348988810749	0\\
24.1343592675167	0\\
24.1338196204326	0\\
24.1332799398187	0\\
24.1327402256707	0\\
24.1322004779845	0\\
24.1316606967558	0\\
24.1311208819806	0\\
24.1305810336546	0\\
24.1300411517737	0\\
24.1295012363337	0\\
24.1289612873305	0\\
24.1284213047598	0\\
24.1278812886175	0\\
24.1273412388994	0\\
24.1268011556013	0\\
24.1262610387192	0\\
24.1257208882486	0\\
24.1251807041857	0\\
24.124640486526	0\\
24.1241002352655	0\\
24.1235599503999	0\\
24.1230196319252	0\\
24.122479279837	0\\
24.1219388941313	0\\
24.1213984748039	0\\
24.1208580218504	0\\
24.1203175352669	0\\
24.119777015049	0\\
24.1192364611927	0\\
24.1186958736936	0\\
24.1181552525477	0\\
24.1176145977507	0\\
24.1170739092985	0\\
24.1165331871868	0\\
24.1159924314115	0\\
24.1154516419684	0\\
24.1149108188532	0\\
24.1143699620619	0\\
24.1138290715901	0\\
24.1132881474337	0\\
24.1127471895886	0\\
24.1122061980504	0\\
24.1116651728151	0\\
24.1111241138783	0\\
24.110583021236	0\\
24.1100418948839	0\\
24.1095007348178	0\\
24.1089595410335	0\\
24.1084183135269	0\\
24.1078770522936	0\\
24.1073357573295	0\\
24.1067944286305	0\\
24.1062530661922	0\\
24.1057116700105	0\\
24.1051702400812	0\\
24.1046287764001	0\\
24.1040872789629	0\\
24.1035457477655	0\\
24.1030041828036	0\\
24.1024625840731	0\\
24.1019209515697	0\\
24.1013792852892	0\\
24.1008375852274	0\\
24.1002958513801	0\\
24.099754083743	0\\
24.099212282312	0\\
24.0986704470829	0\\
24.0981285780514	0\\
24.0975866752132	0\\
24.0970447385643	0\\
24.0965027681003	0\\
24.0959607638171	0\\
24.0954187257104	0\\
24.094876653776	0\\
24.0943345480097	0\\
24.0937924084072	0\\
24.0932502349644	0\\
24.092708027677	0\\
24.0921657865407	0\\
24.0916235115514	0\\
24.0910812027049	0\\
24.0905388599968	0\\
24.089996483423	0\\
24.0894540729793	0\\
24.0889116286614	0\\
24.088369150465	0\\
24.087826638386	0\\
24.0872840924201	0\\
24.0867415125631	0\\
24.0861988988107	0\\
24.0856562511588	0\\
24.085113569603	0\\
24.0845708541392	0\\
24.0840281047631	0\\
24.0834853214704	0\\
24.0829425042569	0\\
24.0823996531185	0\\
24.0818567680508	0\\
24.0813138490496	0\\
24.0807708961106	0\\
24.0802279092297	0\\
24.0796848884025	0\\
24.0791418336249	0\\
24.0785987448925	0\\
24.0780556222012	0\\
24.0775124655467	0\\
24.0769692749247	0\\
24.076426050331	0\\
24.0758827917613	0\\
24.0753394992115	0\\
24.0747961726772	0\\
24.0742528121541	0\\
24.0737094176381	0\\
24.0731659891249	0\\
24.0726225266102	0\\
24.0720790300898	0\\
24.0715354995595	0\\
24.0709919350148	0\\
24.0704483364517	0\\
24.0699047038658	0\\
24.0693610372529	0\\
24.0688173366088	0\\
24.0682736019291	0\\
24.0677298332096	0\\
24.067186030446	0\\
24.0666421936341	0\\
24.0660983227697	0\\
24.0655544178483	0\\
24.0650104788659	0\\
24.0644665058181	0\\
24.0639224987006	0\\
24.0633784575093	0\\
24.0628343822397	0\\
24.0622902728877	0\\
24.061746129449	0\\
24.0612019519192	0\\
24.0606577402942	0\\
24.0601134945697	0\\
24.0595692147413	0\\
24.0590249008049	0\\
24.058480552756	0\\
24.0579361705906	0\\
24.0573917543042	0\\
24.0568473038926	0\\
24.0563028193516	0\\
24.0557583006768	0\\
24.055213747864	0\\
24.0546691609089	0\\
24.0541245398071	0\\
24.0535798845545	0\\
24.0530351951468	0\\
24.0524904715796	0\\
24.0519457138486	0\\
24.0514009219497	0\\
24.0508560958784	0\\
24.0503112356306	0\\
24.0497663412019	0\\
24.049221412588	0\\
24.0486764497847	0\\
24.0481314527876	0\\
24.0475864215925	0\\
24.047041356195	0\\
24.046496256591	0\\
24.045951122776	0\\
24.0454059547458	0\\
24.0448607524961	0\\
24.0443155160226	0\\
24.043770245321	0\\
24.0432249403871	0\\
24.0426796012164	0\\
24.0421342278047	0\\
24.0415888201478	0\\
24.0410433782412	0\\
24.0404979020808	0\\
24.0399523916621	0\\
24.039406846981	0\\
24.038861268033	0\\
24.0383156548139	0\\
24.0377700073195	0\\
24.0372243255452	0\\
24.036678609487	0\\
24.0361328591404	0\\
24.0355870745012	0\\
24.035041255565	0\\
24.0344954023276	0\\
24.0339495147846	0\\
24.0334035929316	0\\
24.0328576367645	0\\
24.0323116462789	0\\
24.0317656214704	0\\
24.0312195623347	0\\
24.0306734688676	0\\
24.0301273410648	0\\
24.0295811789218	0\\
24.0290349824344	0\\
24.0284887515982	0\\
24.027942486409	0\\
24.0273961868624	0\\
24.0268498529542	0\\
24.0263034846798	0\\
24.0257570820352	0\\
24.0252106450158	0\\
24.0246641736175	0\\
24.0241176678358	0\\
24.0235711276665	0\\
24.0230245531052	0\\
24.0224779441476	0\\
24.0219313007893	0\\
24.0213846230261	0\\
24.0208379108536	0\\
24.0202911642674	0\\
24.0197443832633	0\\
24.0191975678369	0\\
24.0186507179839	0\\
24.0181038336998	0\\
24.0175569149805	0\\
24.0170099618216	0\\
24.0164629742186	0\\
24.0159159521673	0\\
24.0153688956634	0\\
24.0148218047025	0\\
24.0142746792803	0\\
24.0137275193923	0\\
24.0131803250344	0\\
24.0126330962021	0\\
24.012085832891	0\\
24.011538535097	0\\
24.0109912028155	0\\
24.0104438360423	0\\
24.0098964347729	0\\
24.0093489990032	0\\
24.0088015287286	0\\
24.0082540239449	0\\
24.0077064846477	0\\
24.0071589108327	0\\
24.0066113024955	0\\
24.0060636596317	0\\
24.005515982237	0\\
24.0049682703071	0\\
24.0044205238375	0\\
24.003872742824	0\\
24.0033249272621	0\\
24.0027770771476	0\\
24.002229192476	0\\
24.001681273243	0\\
24.0011333194443	0\\
24.0005853310754	0\\
24.000037308132	0\\
23.9994892506098	0\\
23.9989411585044	0\\
23.9983930318114	0\\
23.9978448705264	0\\
23.9972966746451	0\\
23.9967484441631	0\\
23.9962001790761	0\\
23.9956518793797	0\\
23.9951035450695	0\\
23.9945551761411	0\\
23.9940067725902	0\\
23.9934583344124	0\\
23.9929098616033	0\\
23.9923613541586	0\\
23.9918128120739	0\\
23.9912642353447	0\\
23.9907156239668	0\\
23.9901669779357	0\\
23.9896182972471	0\\
23.9890695818966	0\\
23.9885208318798	0\\
23.9879720471923	0\\
23.9874232278298	0\\
23.9868743737879	0\\
23.9863254850621	0\\
23.9857765616481	0\\
23.9852276035416	0\\
23.984678610738	0\\
23.9841295832332	0\\
23.9835805210225	0\\
23.9830314241018	0\\
23.9824822924665	0\\
23.9819331261123	0\\
23.9813839250348	0\\
23.9808346892297	0\\
23.9802854186924	0\\
23.9797361134187	0\\
23.9791867734041	0\\
23.9786373986442	0\\
23.9780879891346	0\\
23.9775385448711	0\\
23.976989065849	0\\
23.9764395520641	0\\
23.975890003512	0\\
23.9753404201882	0\\
23.9747908020883	0\\
23.974241149208	0\\
23.9736914615429	0\\
23.9731417390885	0\\
23.9725919818404	0\\
23.9720421897942	0\\
23.9714923629456	0\\
23.9709425012901	0\\
23.9703926048234	0\\
23.9698426735409	0\\
23.9692927074383	0\\
23.9687427065112	0\\
23.9681926707552	0\\
23.9676426001659	0\\
23.9670924947388	0\\
23.9665423544696	0\\
23.9659921793537	0\\
23.9654419693869	0\\
23.9648917245647	0\\
23.9643414448827	0\\
23.9637911303364	0\\
23.9632407809215	0\\
23.9626903966335	0\\
23.962139977468	0\\
23.9615895234206	0\\
23.9610390344869	0\\
23.9604885106624	0\\
23.9599379519427	0\\
23.9593873583234	0\\
23.9588367298001	0\\
23.9582860663684	0\\
23.9577353680237	0\\
23.9571846347618	0\\
23.9566338665781	0\\
23.9560830634682	0\\
23.9555322254277	0\\
23.9549813524522	0\\
23.9544304445373	0\\
23.9538795016785	0\\
23.9533285238713	0\\
23.9527775111114	0\\
23.9522264633943	0\\
23.9516753807155	0\\
23.9511242630707	0\\
23.9505731104554	0\\
23.9500219228652	0\\
23.9494707002955	0\\
23.9489194427421	0\\
23.9483681502004	0\\
23.947816822666	0\\
23.9472654601344	0\\
23.9467140626012	0\\
23.946162630062	0\\
23.9456111625124	0\\
23.9450596599478	0\\
23.9445081223638	0\\
23.9439565497561	0\\
23.94340494212	0\\
23.9428532994513	0\\
23.9423016217453	0\\
23.9417499089978	0\\
23.9411981612042	0\\
23.94064637836	0\\
23.9400945604609	0\\
23.9395427075024	0\\
23.9389908194799	0\\
23.9384388963892	0\\
23.9378869382256	0\\
23.9373349449848	0\\
23.9367829166622	0\\
23.9362308532535	0\\
23.9356787547542	0\\
23.9351266211597	0\\
23.9345744524657	0\\
23.9340222486677	0\\
23.9334700097612	0\\
23.9329177357417	0\\
23.9323654266049	0\\
23.9318130823461	0\\
23.931260702961	0\\
23.9307082884451	0\\
23.9301558387939	0\\
23.929603354003	0\\
23.9290508340678	0\\
23.9284982789839	0\\
23.9279456887469	0\\
23.9273930633522	0\\
23.9268404027954	0\\
23.926287707072	0\\
23.9257349761775	0\\
23.9251822101075	0\\
23.9246294088575	0\\
23.924076572423	0\\
23.9235237007995	0\\
23.9229707939825	0\\
23.9224178519676	0\\
23.9218648747503	0\\
23.9213118623261	0\\
23.9207588146905	0\\
23.920205731839	0\\
23.9196526137672	0\\
23.9190994604705	0\\
23.9185462719445	0\\
23.9179930481847	0\\
23.9174397891866	0\\
23.9168864949456	0\\
23.9163331654574	0\\
23.9157798007175	0\\
23.9152264007212	0\\
23.9146729654642	0\\
23.914119494942	0\\
23.91356598915	0\\
23.9130124480837	0\\
23.9124588717387	0\\
23.9119052601105	0\\
23.9113516131946	0\\
23.9107979309864	0\\
23.9102442134814	0\\
23.9096904606753	0\\
23.9091366725634	0\\
23.9085828491412	0\\
23.9080289904044	0\\
23.9074750963483	0\\
23.9069211669684	0\\
23.9063672022603	0\\
23.9058132022194	0\\
23.9052591668412	0\\
23.9047050961213	0\\
23.9041509900551	0\\
23.9035968486381	0\\
23.9030426718657	0\\
23.9024884597336	0\\
23.9019342122371	0\\
23.9013799293718	0\\
23.9008256111331	0\\
23.9002712575166	0\\
23.8997168685176	0\\
23.8991624441318	0\\
23.8986079843545	0\\
23.8980534891813	0\\
23.8974989586076	0\\
23.8969443926289	0\\
23.8963897912407	0\\
23.8958351544385	0\\
23.8952804822177	0\\
23.8947257745739	0\\
23.8941710315025	0\\
23.8936162529989	0\\
23.8930614390587	0\\
23.8925065896773	0\\
23.8919517048502	0\\
23.8913967845729	0\\
23.8908418288407	0\\
23.8902868376493	0\\
23.8897318109941	0\\
23.8891767488705	0\\
23.888621651274	0\\
23.8880665182001	0\\
23.8875113496442	0\\
23.8869561456019	0\\
23.8864009060684	0\\
23.8858456310395	0\\
23.8852903205103	0\\
23.8847349744766	0\\
23.8841795929336	0\\
23.8836241758769	0\\
23.883068723302	0\\
23.8825132352042	0\\
23.881957711579	0\\
23.881402152422	0\\
23.8808465577285	0\\
23.880290927494	0\\
23.8797352617139	0\\
23.8791795603837	0\\
23.8786238234989	0\\
23.8780680510549	0\\
23.8775122430472	0\\
23.8769563994711	0\\
23.8764005203223	0\\
23.875844605596	0\\
23.8752886552877	0\\
23.8747326693929	0\\
23.8741766479071	0\\
23.8736205908256	0\\
23.873064498144	0\\
23.8725083698576	0\\
23.8719522059619	0\\
23.8713960064523	0\\
23.8708397713243	0\\
23.8702835005734	0\\
23.8697271941949	0\\
23.8691708521842	0\\
23.8686144745369	0\\
23.8680580612484	0\\
23.867501612314	0\\
23.8669451277292	0\\
23.8663886074895	0\\
23.8658320515903	0\\
23.865275460027	0\\
23.8647188327951	0\\
23.8641621698899	0\\
23.8636054713069	0\\
23.8630487370415	0\\
23.8624919670892	0\\
23.8619351614454	0\\
23.8613783201055	0\\
23.8608214430648	0\\
23.860264530319	0\\
23.8597075818633	0\\
23.8591505976931	0\\
23.858593577804	0\\
23.8580365221913	0\\
23.8574794308504	0\\
23.8569223037768	0\\
23.8563651409659	0\\
23.8558079424131	0\\
23.8552507081137	0\\
23.8546934380633	0\\
23.8541361322572	0\\
23.8535787906908	0\\
23.8530214133596	0\\
23.8524640002589	0\\
23.8519065513842	0\\
23.8513490667309	0\\
23.8507915462944	0\\
23.85023399007	0\\
23.8496763980533	0\\
23.8491187702395	0\\
23.8485611066241	0\\
23.8480034072026	0\\
23.8474456719702	0\\
23.8468879009225	0\\
23.8463300940547	0\\
23.8457722513624	0\\
23.8452143728409	0\\
23.8446564584855	0\\
23.8440985082918	0\\
23.843540522255	0\\
23.8429825003707	0\\
23.8424244426341	0\\
23.8418663490407	0\\
23.8413082195858	0\\
23.840750054265	0\\
23.8401918530734	0\\
23.8396336160066	0\\
23.8390753430599	0\\
23.8385170342287	0\\
23.8379586895084	0\\
23.8374003088944	0\\
23.8368418923821	0\\
23.8362834399668	0\\
23.835724951644	0\\
23.835166427409	0\\
23.8346078672571	0\\
23.8340492711839	0\\
23.8334906391846	0\\
23.8329319712546	0\\
23.8323732673894	0\\
23.8318145275842	0\\
23.8312557518345	0\\
23.8306969401357	0\\
23.830138092483	0\\
23.829579208872	0\\
23.8290202892979	0\\
23.8284613337562	0\\
23.8279023422421	0\\
23.8273433147511	0\\
23.8267842512786	0\\
23.8262251518199	0\\
23.8256660163703	0\\
23.8251068449253	0\\
23.8245476374802	0\\
23.8239883940304	0\\
23.8234291145712	0\\
23.822869799098	0\\
23.8223104476061	0\\
23.821751060091	0\\
23.821191636548	0\\
23.8206321769724	0\\
23.8200726813595	0\\
23.8195131497049	0\\
23.8189535820037	0\\
23.8183939782515	0\\
23.8178343384434	0\\
23.8172746625749	0\\
23.8167149506413	0\\
23.816155202638	0\\
23.8155954185604	0\\
23.8150355984037	0\\
23.8144757421633	0\\
23.8139158498346	0\\
23.813355921413	0\\
23.8127959568937	0\\
23.8122359562721	0\\
23.8116759195436	0\\
23.8111158467034	0\\
23.8105557377471	0\\
23.8099955926698	0\\
23.8094354114669	0\\
23.8088751941338	0\\
23.8083149406658	0\\
23.8077546510583	0\\
23.8071943253065	0\\
23.8066339634058	0\\
23.8060735653516	0\\
23.8055131311392	0\\
23.8049526607639	0\\
23.8043921542211	0\\
23.803831611506	0\\
23.8032710326141	0\\
23.8027104175406	0\\
23.8021497662809	0\\
23.8015890788303	0\\
23.8010283551841	0\\
23.8004675953377	0\\
23.7999067992864	0\\
23.7993459670254	0\\
23.7987850985503	0\\
23.7982241938561	0\\
23.7976632529383	0\\
23.7971022757923	0\\
23.7965412624132	0\\
23.7959802127965	0\\
23.7954191269375	0\\
23.7948580048314	0\\
23.7942968464736	0\\
23.7937356518594	0\\
23.7931744209841	0\\
23.7926131538431	0\\
23.7920518504316	0\\
23.791490510745	0\\
23.7909291347785	0\\
23.7903677225275	0\\
23.7898062739874	0\\
23.7892447891533	0\\
23.7886832680206	0\\
23.7881217105846	0\\
23.7875601168406	0\\
23.786998486784	0\\
23.7864368204099	0\\
23.7858751177138	0\\
23.785313378691	0\\
23.7847516033366	0\\
23.7841897916461	0\\
23.7836279436147	0\\
23.7830660592378	0\\
23.7825041385106	0\\
23.7819421814284	0\\
23.7813801879865	0\\
23.7808181581802	0\\
23.7802560920048	0\\
23.7796939894557	0\\
23.779131850528	0\\
23.7785696752171	0\\
23.7780074635182	0\\
23.7774452154268	0\\
23.7768829309379	0\\
23.7763206100471	0\\
23.7757582527494	0\\
23.7751958590402	0\\
23.7746334289149	0\\
23.7740709623686	0\\
23.7735084593966	0\\
23.7729459199943	0\\
23.772383344157	0\\
23.7718207318798	0\\
23.7712580831581	0\\
23.7706953979872	0\\
23.7701326763623	0\\
23.7695699182787	0\\
23.7690071237317	0\\
23.7684442927166	0\\
23.7678814252286	0\\
23.767318521263	0\\
23.7667555808151	0\\
23.7661926038801	0\\
23.7656295904533	0\\
23.7650665405301	0\\
23.7645034541056	0\\
23.7639403311751	0\\
23.7633771717339	0\\
23.7628139757773	0\\
23.7622507433005	0\\
23.7616874742988	0\\
23.7611241687674	0\\
23.7605608267016	0\\
23.7599974480967	0\\
23.7594340329479	0\\
23.7588705812505	0\\
23.7583070929998	0\\
23.7577435681909	0\\
23.7571800068193	0\\
23.75661640888	0\\
23.7560527743684	0\\
23.7554891032797	0\\
23.7549253956092	0\\
23.7543616513521	0\\
23.7537978705037	0\\
23.7532340530592	0\\
23.7526701990139	0\\
23.752106308363	0\\
23.7515423811017	0\\
23.7509784172254	0\\
23.7504144167293	0\\
23.7498503796085	0\\
23.7492863058584	0\\
23.7487221954742	0\\
23.7481580484511	0\\
23.7475938647843	0\\
23.7470296444692	0\\
23.7464653875009	0\\
23.7459010938748	0\\
23.7453367635859	0\\
23.7447723966296	0\\
23.7442079930011	0\\
23.7436435526956	0\\
23.7430790757083	0\\
23.7425145620346	0\\
23.7419500116695	0\\
23.7413854246085	0\\
23.7408208008466	0\\
23.7402561403791	0\\
23.7396914432012	0\\
23.7391267093082	0\\
23.7385619386953	0\\
23.7379971313578	0\\
23.7374322872907	0\\
23.7368674064894	0\\
23.7363024889491	0\\
23.735737534665	0\\
23.7351725436323	0\\
23.7346075158462	0\\
23.734042451302	0\\
23.7334773499949	0\\
23.7329122119201	0\\
23.7323470370727	0\\
23.7317818254481	0\\
23.7312165770415	0\\
23.7306512918479	0\\
23.7300859698628	0\\
23.7295206110812	0\\
23.7289552154983	0\\
23.7283897831095	0\\
23.7278243139099	0\\
23.7272588078946	0\\
23.726693265059	0\\
23.7261276853982	0\\
23.7255620689073	0\\
23.7249964155818	0\\
23.7244307254166	0\\
23.723864998407	0\\
23.7232992345483	0\\
23.7227334338356	0\\
23.7221675962641	0\\
23.721601721829	0\\
23.7210358105256	0\\
23.7204698623489	0\\
23.7199038772943	0\\
23.7193378553568	0\\
23.7187717965318	0\\
23.7182057008143	0\\
23.7176395681996	0\\
23.7170733986828	0\\
23.7165071922592	0\\
23.715940948924	0\\
23.7153746686723	0\\
23.7148083514992	0\\
23.7142419974001	0\\
23.7136756063701	0\\
23.7131091784043	0\\
23.712542713498	0\\
23.7119762116463	0\\
23.7114096728444	0\\
23.7108430970875	0\\
23.7102764843708	0\\
23.7097098346894	0\\
23.7091431480385	0\\
23.7085764244133	0\\
23.7080096638091	0\\
23.7074428662208	0\\
23.7068760316438	0\\
23.7063091600731	0\\
23.705742251504	0\\
23.7051753059317	0\\
23.7046083233513	0\\
23.7040413037579	0\\
23.7034742471467	0\\
23.702907153513	0\\
23.7023400228518	0\\
23.7017728551584	0\\
23.7012056504279	0\\
23.7006384086554	0\\
23.7000711298361	0\\
23.6995038139652	0\\
23.6989364610379	0\\
23.6983690710492	0\\
23.6978016439944	0\\
23.6972341798687	0\\
23.6966666786671	0\\
23.6960991403848	0\\
23.695531565017	0\\
23.6949639525588	0\\
23.6943963030054	0\\
23.693828616352	0\\
23.6932608925936	0\\
23.6926931317255	0\\
23.6921253337427	0\\
23.6915574986405	0\\
23.690989626414	0\\
23.6904217170582	0\\
23.6898537705685	0\\
23.6892857869398	0\\
23.6887177661674	0\\
23.6881497082464	0\\
23.6875816131719	0\\
23.6870134809391	0\\
23.6864453115431	0\\
23.6858771049791	0\\
23.6853088612422	0\\
23.6847405803274	0\\
23.6841722622301	0\\
23.6836039069452	0\\
23.6830355144679	0\\
23.6824670847934	0\\
23.6818986179168	0\\
23.6813301138332	0\\
23.6807615725378	0\\
23.6801929940256	0\\
23.6796243782918	0\\
23.6790557253316	0\\
23.67848703514	0\\
23.6779183077122	0\\
23.6773495430433	0\\
23.6767807411284	0\\
23.6762119019627	0\\
23.6756430255412	0\\
23.6750741118591	0\\
23.6745051609116	0\\
23.6739361726936	0\\
23.6733671472004	0\\
23.672798084427	0\\
23.6722289843686	0\\
23.6716598470203	0\\
23.6710906723772	0\\
23.6705214604344	0\\
23.669952211187	0\\
23.6693829246302	0\\
23.668813600759	0\\
23.6682442395685	0\\
23.6676748410539	0\\
23.6671054052102	0\\
23.6665359320327	0\\
23.6659664215162	0\\
23.6653968736561	0\\
23.6648272884473	0\\
23.664257665885	0\\
23.6636880059643	0\\
23.6631183086803	0\\
23.662548574028	0\\
23.6619788020026	0\\
23.6614089925992	0\\
23.6608391458128	0\\
23.6602692616386	0\\
23.6596993400716	0\\
23.659129381107	0\\
23.6585593847398	0\\
23.6579893509652	0\\
23.6574192797782	0\\
23.6568491711738	0\\
23.6562790251473	0\\
23.6557088416937	0\\
23.655138620808	0\\
23.6545683624853	0\\
23.6539980667209	0\\
23.6534277335096	0\\
23.6528573628466	0\\
23.652286954727	0\\
23.6517165091459	0\\
23.6511460260983	0\\
23.6505755055793	0\\
23.650004947584	0\\
23.6494343521075	0\\
23.6488637191448	0\\
23.6482930486911	0\\
23.6477223407413	0\\
23.6471515952907	0\\
23.6465808123341	0\\
23.6460099918668	0\\
23.6454391338837	0\\
23.64486823838	0\\
23.6442973053507	0\\
23.6437263347909	0\\
23.6431553266956	0\\
23.64258428106	0\\
23.642013197879	0\\
23.6414420771477	0\\
23.6408709188613	0\\
23.6402997230147	0\\
23.639728489603	0\\
23.6391572186213	0\\
23.6385859100647	0\\
23.6380145639281	0\\
23.6374431802067	0\\
23.6368717588955	0\\
23.6363002999895	0\\
23.6357288034838	0\\
23.6351572693735	0\\
23.6345856976537	0\\
23.6340140883192	0\\
23.6334424413653	0\\
23.632870756787	0\\
23.6322990345792	0\\
23.6317272747371	0\\
23.6311554772557	0\\
23.6305836421301	0\\
23.6300117693552	0\\
23.6294398589261	0\\
23.628867910838	0\\
23.6282959250857	0\\
23.6277239016643	0\\
23.627151840569	0\\
23.6265797417947	0\\
23.6260076053364	0\\
23.6254354311892	0\\
23.6248632193482	0\\
23.6242909698083	0\\
23.6237186825646	0\\
23.6231463576121	0\\
23.6225739949459	0\\
23.622001594561	0\\
23.6214291564523	0\\
23.6208566806151	0\\
23.6202841670441	0\\
23.6197116157346	0\\
23.6191390266815	0\\
23.6185663998798	0\\
23.6179937353246	0\\
23.6174210330109	0\\
23.6168482929336	0\\
23.6162755150879	0\\
23.6157026994688	0\\
23.6151298460712	0\\
23.6145569548901	0\\
23.6139840259207	0\\
23.6134110591579	0\\
23.6128380545966	0\\
23.6122650122321	0\\
23.6116919320591	0\\
23.6111188140728	0\\
23.6105456582682	0\\
23.6099724646402	0\\
23.609399233184	0\\
23.6088259638944	0\\
23.6082526567665	0\\
23.6076793117953	0\\
23.6071059289758	0\\
23.606532508303	0\\
23.6059590497719	0\\
23.6053855533775	0\\
23.6048120191148	0\\
23.6042384469788	0\\
23.6036648369645	0\\
23.603091189067	0\\
23.6025175032811	0\\
23.6019437796019	0\\
23.6013700180244	0\\
23.6007962185435	0\\
23.6002223811543	0\\
23.5996485058518	0\\
23.599074592631	0\\
23.5985006414867	0\\
23.5979266524141	0\\
23.5973526254081	0\\
23.5967785604637	0\\
23.5962044575759	0\\
23.5956303167397	0\\
23.59505613795	0\\
23.5944819212018	0\\
23.5939076664901	0\\
23.59333337381	0\\
23.5927590431563	0\\
23.592184674524	0\\
23.5916102679082	0\\
23.5910358233038	0\\
23.5904613407057	0\\
23.589886820109	0\\
23.5893122615086	0\\
23.5887376648995	0\\
23.5881630302767	0\\
23.5875883576351	0\\
23.5870136469698	0\\
23.5864388982755	0\\
23.5858641115475	0\\
23.5852892867805	0\\
23.5847144239696	0\\
23.5841395231098	0\\
23.5835645841959	0\\
23.582989607223	0\\
23.582414592186	0\\
23.5818395390799	0\\
23.5812644478996	0\\
23.5806893186401	0\\
23.5801141512964	0\\
23.5795389458634	0\\
23.578963702336	0\\
23.5783884207093	0\\
23.5778131009781	0\\
23.5772377431375	0\\
23.5766623471823	0\\
23.5760869131076	0\\
23.5755114409082	0\\
23.5749359305792	0\\
23.5743603821154	0\\
23.5737847955118	0\\
23.5732091707634	0\\
23.5726335078651	0\\
23.5720578068118	0\\
23.5714820675985	0\\
23.5709062902201	0\\
23.5703304746717	0\\
23.569754620948	0\\
23.569178729044	0\\
23.5686027989548	0\\
23.5680268306751	0\\
23.5674508242	0\\
23.5668747795244	0\\
23.5662986966432	0\\
23.5657225755513	0\\
23.5651464162438	0\\
23.5645702187154	0\\
23.5639939829612	0\\
23.563417708976	0\\
23.5628413967549	0\\
23.5622650462926	0\\
23.5616886575842	0\\
23.5611122306246	0\\
23.5605357654087	0\\
23.5599592619313	0\\
23.5593827201875	0\\
23.5588061401722	0\\
23.5582295218802	0\\
23.5576528653065	0\\
23.557076170446	0\\
23.5564994372936	0\\
23.5559226658443	0\\
23.5553458560929	0\\
23.5547690080344	0\\
23.5541921216636	0\\
23.5536151969756	0\\
23.5530382339651	0\\
23.5524612326271	0\\
23.5518841929565	0\\
23.5513071149483	0\\
23.5507299985972	0\\
23.5501528438983	0\\
23.5495756508464	0\\
23.5489984194365	0\\
23.5484211496634	0\\
23.547843841522	0\\
23.5472664950072	0\\
23.546689110114	0\\
23.5461116868372	0\\
23.5455342251717	0\\
23.5449567251125	0\\
23.5443791866543	0\\
23.5438016097922	0\\
23.5432239945209	0\\
23.5426463408355	0\\
23.5420686487307	0\\
23.5414909182015	0\\
23.5409131492428	0\\
23.5403353418494	0\\
23.5397574960162	0\\
23.5391796117382	0\\
23.5386016890101	0\\
23.5380237278269	0\\
23.5374457281835	0\\
23.5368676900748	0\\
23.5362896134955	0\\
23.5357114984407	0\\
23.5351333449052	0\\
23.5345551528838	0\\
23.5339769223714	0\\
23.533398653363	0\\
23.5328203458533	0\\
23.5322419998373	0\\
23.5316636153099	0\\
23.5310851922658	0\\
23.5305067307	0\\
23.5299282306073	0\\
23.5293496919827	0\\
23.5287711148209	0\\
23.5281924991168	0\\
23.5276138448654	0\\
23.5270351520614	0\\
23.5264564206998	0\\
23.5258776507753	0\\
23.5252988422829	0\\
23.5247199952174	0\\
23.5241411095737	0\\
23.5235621853466	0\\
23.522983222531	0\\
23.5224042211217	0\\
23.5218251811136	0\\
23.5212461025016	0\\
23.5206669852804	0\\
23.5200878294451	0\\
23.5195086349903	0\\
23.5189294019109	0\\
23.5183501302019	0\\
23.517770819858	0\\
23.5171914708741	0\\
23.516612083245	0\\
23.5160326569656	0\\
23.5154531920307	0\\
23.5148736884352	0\\
23.5142941461739	0\\
23.5137145652416	0\\
23.5131349456333	0\\
23.5125552873436	0\\
23.5119755903675	0\\
23.5113958546998	0\\
23.5108160803354	0\\
23.510236267269	0\\
23.5096564154954	0\\
23.5090765250097	0\\
23.5084965958064	0\\
23.5079166278806	0\\
23.507336621227	0\\
23.5067565758404	0\\
23.5061764917157	0\\
23.5055963688477	0\\
23.5050162072312	0\\
23.504436006861	0\\
23.5038557677321	0\\
23.5032754898391	0\\
23.5026951731769	0\\
23.5021148177403	0\\
23.5015344235242	0\\
23.5009539905233	0\\
23.5003735187326	0\\
23.4997930081467	0\\
23.4992124587605	0\\
23.4986318705689	0\\
23.4980512435665	0\\
23.4974705777483	0\\
23.4968898731091	0\\
23.4963091296436	0\\
23.4957283473467	0\\
23.4951475262132	0\\
23.4945666662379	0\\
23.4939857674155	0\\
23.4934048297409	0\\
23.492823853209	0\\
23.4922428378144	0\\
23.491661783552	0\\
23.4910806904166	0\\
23.490499558403	0\\
23.489918387506	0\\
23.4893371777204	0\\
23.4887559290409	0\\
23.4881746414624	0\\
23.4875933149797	0\\
23.4870119495876	0\\
23.4864305452808	0\\
23.4858491020542	0\\
23.4852676199024	0\\
23.4846860988204	0\\
23.4841045388029	0\\
23.4835229398447	0\\
23.4829413019406	0\\
23.4823596250853	0\\
23.4817779092737	0\\
23.4811961545005	0\\
23.4806143607605	0\\
23.4800325280485	0\\
23.4794506563592	0\\
23.4788687456875	0\\
23.4782867960281	0\\
23.4777048073758	0\\
23.4771227797254	0\\
23.4765407130716	0\\
23.4759586074092	0\\
23.475376462733	0\\
23.4747942790377	0\\
23.4742120563182	0\\
23.4736297945692	0\\
23.4730474937854	0\\
23.4724651539616	0\\
23.4718827750927	0\\
23.4713003571732	0\\
23.4707179001981	0\\
23.4701354041621	0\\
23.4695528690599	0\\
23.4689702948863	0\\
23.468387681636	0\\
23.4678050293039	0\\
23.4672223378846	0\\
23.466639607373	0\\
23.4660568377637	0\\
23.4654740290516	0\\
23.4648911812313	0\\
23.4643082942977	0\\
23.4637253682455	0\\
23.4631424030694	0\\
23.4625593987642	0\\
23.4619763553246	0\\
23.4613932727453	0\\
23.4608101510212	0\\
23.460226990147	0\\
23.4596437901174	0\\
23.4590605509271	0\\
23.4584772725709	0\\
23.4578939550435	0\\
23.4573105983397	0\\
23.4567272024542	0\\
23.4561437673817	0\\
23.4555602931171	0\\
23.4549767796549	0\\
23.4543932269899	0\\
23.453809635117	0\\
23.4532260040307	0\\
23.4526423337259	0\\
23.4520586241972	0\\
23.4514748754394	0\\
23.4508910874473	0\\
23.4503072602154	0\\
23.4497233937387	0\\
23.4491394880117	0\\
23.4485555430293	0\\
23.4479715587861	0\\
23.4473875352768	0\\
23.4468034724963	0\\
23.4462193704391	0\\
23.4456352291	0\\
23.4450510484738	0\\
23.4444668285551	0\\
23.4438825693386	0\\
23.4432982708191	0\\
23.4427139329913	0\\
23.4421295558499	0\\
23.4415451393896	0\\
23.4409606836051	0\\
23.4403761884911	0\\
23.4397916540424	0\\
23.4392070802535	0\\
23.4386224671193	0\\
23.4380378146345	0\\
23.4374531227937	0\\
23.4368683915916	0\\
23.4362836210229	0\\
23.4356988110824	0\\
23.4351139617648	0\\
23.4345290730646	0\\
23.4339441449767	0\\
23.4333591774957	0\\
23.4327741706164	0\\
23.4321891243333	0\\
23.4316040386413	0\\
23.4310189135349	0\\
23.4304337490089	0\\
23.429848545058	0\\
23.4292633016769	0\\
23.4286780188602	0\\
23.4280926966026	0\\
23.4275073348988	0\\
23.4269219337435	0\\
23.4263364931315	0\\
23.4257510130572	0\\
23.4251654935155	0\\
23.424579934501	0\\
23.4239943360084	0\\
23.4234086980324	0\\
23.4228230205676	0\\
23.4222373036088	0\\
23.4216515471505	0\\
23.4210657511875	0\\
23.4204799157145	0\\
23.419894040726	0\\
23.4193081262168	0\\
23.4187221721816	0\\
23.418136178615	0\\
23.4175501455116	0\\
23.4169640728662	0\\
23.4163779606734	0\\
23.4157918089278	0\\
23.4152056176242	0\\
23.4146193867572	0\\
23.4140331163214	0\\
23.4134468063115	0\\
23.4128604567222	0\\
23.4122740675481	0\\
23.4116876387839	0\\
23.4111011704242	0\\
23.4105146624636	0\\
23.409928114897	0\\
23.4093415277187	0\\
23.4087549009237	0\\
23.4081682345064	0\\
23.4075815284615	0\\
23.4069947827837	0\\
23.4064079974676	0\\
23.4058211725079	0\\
23.4052343078991	0\\
23.4046474036361	0\\
23.4040604597133	0\\
23.4034734761254	0\\
23.4028864528671	0\\
23.402299389933	0\\
23.4017122873178	0\\
23.401125145016	0\\
23.4005379630223	0\\
23.3999507413314	0\\
23.3993634799378	0\\
23.3987761788363	0\\
23.3981888380214	0\\
23.3976014574877	0\\
23.39701403723	0\\
23.3964265772427	0\\
23.3958390775206	0\\
23.3952515380583	0\\
23.3946639588504	0\\
23.3940763398914	0\\
23.3934886811761	0\\
23.3929009826991	0\\
23.3923132444549	0\\
23.3917254664383	0\\
23.3911376486437	0\\
23.3905497910659	0\\
23.3899618936994	0\\
23.3893739565388	0\\
23.3887859795788	0\\
23.388197962814	0\\
23.387609906239	0\\
23.3870218098484	0\\
23.3864336736367	0\\
23.3858454975987	0\\
23.3852572817289	0\\
23.3846690260219	0\\
23.3840807304724	0\\
23.3834923950749	0\\
23.382904019824	0\\
23.3823156047143	0\\
23.3817271497405	0\\
23.381138654897	0\\
23.3805501201786	0\\
23.3799615455799	0\\
23.3793729310953	0\\
23.3787842767196	0\\
23.3781955824472	0\\
23.3776068482729	0\\
23.3770180741911	0\\
23.3764292601965	0\\
23.3758404062836	0\\
23.3752515124471	0\\
23.3746625786816	0\\
23.3740736049815	0\\
23.3734845913416	0\\
23.3728955377563	0\\
23.3723064442203	0\\
23.3717173107282	0\\
23.3711281372745	0\\
23.3705389238538	0\\
23.3699496704607	0\\
23.3693603770897	0\\
23.3687710437354	0\\
23.3681816703925	0\\
23.3675922570555	0\\
23.3670028037189	0\\
23.3664133103773	0\\
23.3658237770253	0\\
23.3652342036575	0\\
23.3646445902684	0\\
23.3640549368526	0\\
23.3634652434047	0\\
23.3628755099192	0\\
23.3622857363906	0\\
23.3616959228137	0\\
23.3611060691828	0\\
23.3605161754926	0\\
23.3599262417376	0\\
23.3593362679124	0\\
23.3587462540116	0\\
23.3581562000296	0\\
23.3575661059611	0\\
23.3569759718006	0\\
23.3563857975426	0\\
23.3557955831818	0\\
23.3552053287126	0\\
23.3546150341296	0\\
23.3540246994274	0\\
23.3534343246004	0\\
23.3528439096433	0\\
23.3522534545506	0\\
23.3516629593168	0\\
23.3510724239365	0\\
23.3504818484042	0\\
23.3498912327144	0\\
23.3493005768618	0\\
23.3487098808407	0\\
23.3481191446458	0\\
23.3475283682717	0\\
23.3469375517127	0\\
23.3463466949635	0\\
23.3457557980186	0\\
23.3451648608726	0\\
23.3445738835199	0\\
23.343982865955	0\\
23.3433918081726	0\\
23.3428007101672	0\\
23.3422095719331	0\\
23.3416183934651	0\\
23.3410271747576	0\\
23.3404359158051	0\\
23.3398446166022	0\\
23.3392532771433	0\\
23.3386618974231	0\\
23.3380704774359	0\\
23.3374790171764	0\\
23.336887516639	0\\
23.3362959758182	0\\
23.3357043947087	0\\
23.3351127733048	0\\
23.334521111601	0\\
23.333929409592	0\\
23.3333376672722	0\\
23.3327458846361	0\\
23.3321540616782	0\\
23.331562198393	0\\
23.3309702947751	0\\
23.3303783508189	0\\
23.3297863665189	0\\
23.3291943418697	0\\
23.3286022768657	0\\
23.3280101715014	0\\
23.3274180257713	0\\
23.32682583967	0\\
23.3262336131918	0\\
23.3256413463314	0\\
23.3250490390832	0\\
23.3244566914417	0\\
23.3238643034013	0\\
23.3232718749567	0\\
23.3226794061022	0\\
23.3220868968323	0\\
23.3214943471416	0\\
23.3209017570245	0\\
23.3203091264755	0\\
23.3197164554891	0\\
23.3191237440597	0\\
23.3185309921819	0\\
23.3179381998501	0\\
23.3173453670589	0\\
23.3167524938026	0\\
23.3161595800757	0\\
23.3155666258728	0\\
23.3149736311883	0\\
23.3143805960166	0\\
23.3137875203523	0\\
23.3131944041899	0\\
23.3126012475237	0\\
23.3120080503482	0\\
23.311414812658	0\\
23.3108215344475	0\\
23.3102282157111	0\\
23.3096348564433	0\\
23.3090414566386	0\\
23.3084480162914	0\\
23.3078545353962	0\\
23.3072610139475	0\\
23.3066674519397	0\\
23.3060738493672	0\\
23.3054802062246	0\\
23.3048865225062	0\\
23.3042927982066	0\\
23.3036990333201	0\\
23.3031052278413	0\\
23.3025113817646	0\\
23.3019174950844	0\\
23.3013235677951	0\\
23.3007295998913	0\\
23.3001355913674	0\\
23.2995415422177	0\\
23.2989474524368	0\\
23.2983533220191	0\\
23.2977591509591	0\\
23.2971649392511	0\\
23.2965706868896	0\\
23.2959763938691	0\\
23.295382060184	0\\
23.2947876858286	0\\
23.2941932707976	0\\
23.2935988150852	0\\
23.2930043186859	0\\
23.2924097815942	0\\
23.2918152038045	0\\
23.2912205853111	0\\
23.2906259261086	0\\
23.2900312261914	0\\
23.2894364855538	0\\
23.2888417041903	0\\
23.2882468820954	0\\
23.2876520192634	0\\
23.2870571156887	0\\
23.2864621713659	0\\
23.2858671862892	0\\
23.2852721604532	0\\
23.2846770938522	0\\
23.2840819864807	0\\
23.283486838333	0\\
23.2828916494036	0\\
23.2822964196868	0\\
23.2817011491772	0\\
23.281105837869	0\\
23.2805104857568	0\\
23.2799150928349	0\\
23.2793196590977	0\\
23.2787241845396	0\\
23.278128669155	0\\
23.2775331129384	0\\
23.2769375158841	0\\
23.2763418779865	0\\
23.2757461992401	0\\
23.2751504796391	0\\
23.2745547191781	0\\
23.2739589178514	0\\
23.2733630756534	0\\
23.2727671925785	0\\
23.272171268621	0\\
23.2715753037755	0\\
23.2709792980362	0\\
23.2703832513976	0\\
23.269787163854	0\\
23.2691910353998	0\\
23.2685948660294	0\\
23.2679986557373	0\\
23.2674024045177	0\\
23.266806112365	0\\
23.2662097792737	0\\
23.2656134052381	0\\
23.2650169902526	0\\
23.2644205343116	0\\
23.2638240374094	0\\
23.2632274995404	0\\
23.262630920699	0\\
23.2620343008795	0\\
23.2614376400764	0\\
23.260840938284	0\\
23.2602441954966	0\\
23.2596474117087	0\\
23.2590505869145	0\\
23.2584537211085	0\\
23.2578568142851	0\\
23.2572598664385	0\\
23.2566628775632	0\\
23.2560658476535	0\\
23.2554687767037	0\\
23.2548716647083	0\\
23.2542745116616	0\\
23.2536773175578	0\\
23.2530800823915	0\\
23.2524828061569	0\\
23.2518854888484	0\\
23.2512881304604	0\\
23.2506907309871	0\\
23.250093290423	0\\
23.2494958087624	0\\
23.2488982859996	0\\
23.2483007221289	0\\
23.2477031171448	0\\
23.2471054710416	0\\
23.2465077838135	0\\
23.245910055455	0\\
23.2453122859604	0\\
23.244714475324	0\\
23.2441166235402	0\\
23.2435187306032	0\\
23.2429207965075	0\\
23.2423228212474	0\\
23.2417248048171	0\\
23.241126747211	0\\
23.2405286484235	0\\
23.2399305084489	0\\
23.2393323272815	0\\
23.2387341049156	0\\
23.2381358413456	0\\
23.2375375365657	0\\
23.2369391905704	0\\
23.2363408033538	0\\
23.2357423749105	0\\
23.2351439052345	0\\
23.2345453943204	0\\
23.2339468421624	0\\
23.2333482487547	0\\
23.2327496140918	0\\
23.2321509381679	0\\
23.2315522209774	0\\
23.2309534625145	0\\
23.2303546627737	0\\
23.229755821749	0\\
23.229156939435	0\\
23.2285580158259	0\\
23.2279590509159	0\\
23.2273600446995	0\\
23.2267609971709	0\\
23.2261619083243	0\\
23.2255627781542	0\\
23.2249636066548	0\\
23.2243643938203	0\\
23.2237651396452	0\\
23.2231658441236	0\\
23.22256650725	0\\
23.2219671290185	0\\
23.2213677094235	0\\
23.2207682484592	0\\
23.22016874612	0\\
23.2195692024002	0\\
23.2189696172939	0\\
23.2183699907956	0\\
23.2177703228995	0\\
23.2171706135998	0\\
23.2165708628909	0\\
23.2159710707671	0\\
23.2153712372226	0\\
23.2147713622516	0\\
23.2141714458486	0\\
23.2135714880077	0\\
23.2129714887232	0\\
23.2123714479895	0\\
23.2117713658007	0\\
23.2111712421512	0\\
23.2105710770351	0\\
23.2099708704469	0\\
23.2093706223807	0\\
23.2087703328309	0\\
23.2081700017916	0\\
23.2075696292572	0\\
23.2069692152219	0\\
23.20636875968	0\\
23.2057682626258	0\\
23.2051677240534	0\\
23.2045671439572	0\\
23.2039665223314	0\\
23.2033658591702	0\\
23.202765154468	0\\
23.202164408219	0\\
23.2015636204174	0\\
23.2009627910575	0\\
23.2003619201335	0\\
23.1997610076397	0\\
23.1991600535704	0\\
23.1985590579197	0\\
23.1979580206819	0\\
23.1973569418513	0\\
23.1967558214221	0\\
23.1961546593886	0\\
23.195553455745	0\\
23.1949522104855	0\\
23.1943509236043	0\\
23.1937495950958	0\\
23.1931482249541	0\\
23.1925468131735	0\\
23.1919453597482	0\\
23.1913438646724	0\\
23.1907423279404	0\\
23.1901407495464	0\\
23.1895391294847	0\\
23.1889374677494	0\\
23.1883357643348	0\\
23.1877340192351	0\\
23.1871322324445	0\\
23.1865304039573	0\\
23.1859285337677	0\\
23.1853266218699	0\\
23.1847246682581	0\\
23.1841226729265	0\\
23.1835206358694	0\\
23.1829185570809	0\\
23.1823164365554	0\\
23.1817142742869	0\\
23.1811120702697	0\\
23.1805098244981	0\\
23.1799075369662	0\\
23.1793052076682	0\\
23.1787028365984	0\\
23.1781004237509	0\\
23.1774979691199	0\\
23.1768954726998	0\\
23.1762929344845	0\\
23.1756903544685	0\\
23.1750877326458	0\\
23.1744850690106	0\\
23.1738823635572	0\\
23.1732796162798	0\\
23.1726768271725	0\\
23.1720739962295	0\\
23.1714711234451	0\\
23.1708682088134	0\\
23.1702652523286	0\\
23.169662253985	0\\
23.1690592137766	0\\
23.1684561316976	0\\
23.1678530077424	0\\
23.167249841905	0\\
23.1666466341796	0\\
23.1660433845605	0\\
23.1654400930417	0\\
23.1648367596175	0\\
23.164233384282	0\\
23.1636299670295	0\\
23.1630265078541	0\\
23.1624230067499	0\\
23.1618194637112	0\\
23.1612158787321	0\\
23.1606122518068	0\\
23.1600085829295	0\\
23.1594048720943	0\\
23.1588011192953	0\\
23.1581973245269	0\\
23.1575934877831	0\\
23.156989609058	0\\
23.1563856883459	0\\
23.155781725641	0\\
23.1551777209373	0\\
23.154573674229	0\\
23.1539695855103	0\\
23.1533654547754	0\\
23.1527612820183	0\\
23.1521570672334	0\\
23.1515528104146	0\\
23.1509485115562	0\\
23.1503441706522	0\\
23.149739787697	0\\
23.1491353626845	0\\
23.148530895609	0\\
23.1479263864646	0\\
23.1473218352454	0\\
23.1467172419456	0\\
23.1461126065593	0\\
23.1455079290807	0\\
23.1449032095039	0\\
23.144298447823	0\\
23.1436936440322	0\\
23.1430887981256	0\\
23.1424839100974	0\\
23.1418789799416	0\\
23.1412740076525	0\\
23.140668993224	0\\
23.1400639366505	0\\
23.1394588379259	0\\
23.1388536970445	0\\
23.1382485140003	0\\
23.1376432887875	0\\
23.1370380214002	0\\
23.1364327118325	0\\
23.1358273600786	0\\
23.1352219661325	0\\
23.1346165299884	0\\
23.1340110516403	0\\
23.1334055310825	0\\
23.132799968309	0\\
23.132194363314	0\\
23.1315887160915	0\\
23.1309830266356	0\\
23.1303772949406	0\\
23.1297715210004	0\\
23.1291657048091	0\\
23.128559846361	0\\
23.1279539456501	0\\
23.1273480026704	0\\
23.1267420174162	0\\
23.1261359898815	0\\
23.1255299200603	0\\
23.1249238079469	0\\
23.1243176535353	0\\
23.1237114568195	0\\
23.1231052177937	0\\
23.1224989364521	0\\
23.1218926127886	0\\
23.1212862467973	0\\
23.1206798384724	0\\
23.120073387808	0\\
23.119466894798	0\\
23.1188603594367	0\\
23.1182537817181	0\\
23.1176471616363	0\\
23.1170404991854	0\\
23.1164337943593	0\\
23.1158270471524	0\\
23.1152202575585	0\\
23.1146134255718	0\\
23.1140065511863	0\\
23.1133996343962	0\\
23.1127926751955	0\\
23.1121856735783	0\\
23.1115786295386	0\\
23.1109715430706	0\\
23.1103644141682	0\\
23.1097572428256	0\\
23.1091500290368	0\\
23.108542772796	0\\
23.107935474097	0\\
23.1073281329341	0\\
23.1067207493012	0\\
23.1061133231925	0\\
23.105505854602	0\\
23.1048983435237	0\\
23.1042907899518	0\\
23.1036831938802	0\\
23.103075555303	0\\
23.1024678742142	0\\
23.101860150608	0\\
23.1012523844784	0\\
23.1006445758194	0\\
23.1000367246251	0\\
23.0994288308894	0\\
23.0988208946065	0\\
23.0982129157705	0\\
23.0976048943752	0\\
23.0969968304149	0\\
23.0963887238835	0\\
23.095780574775	0\\
23.0951723830835	0\\
23.0945641488031	0\\
23.0939558719278	0\\
23.0933475524515	0\\
23.0927391903684	0\\
23.0921307856725	0\\
23.0915223383577	0\\
23.0909138484182	0\\
23.0903053158479	0\\
23.089696740641	0\\
23.0890881227913	0\\
23.0884794622929	0\\
23.0878707591399	0\\
23.0872620133263	0\\
23.0866532248461	0\\
23.0860443936933	0\\
23.0854355198619	0\\
23.084826603346	0\\
23.0842176441395	0\\
23.0836086422365	0\\
23.082999597631	0\\
23.0823905103169	0\\
23.0817813802884	0\\
23.0811722075394	0\\
23.0805629920639	0\\
23.079953733856	0\\
23.0793444329096	0\\
23.0787350892187	0\\
23.0781257027774	0\\
23.0775162735796	0\\
23.0769068016193	0\\
23.0762972868906	0\\
23.0756877293874	0\\
23.0750781291037	0\\
23.0744684860336	0\\
23.073858800171	0\\
23.0732490715099	0\\
23.0726393000443	0\\
23.0720294857682	0\\
23.0714196286756	0\\
23.0708097287604	0\\
23.0701997860168	0\\
23.0695898004385	0\\
23.0689797720197	0\\
23.0683697007542	0\\
23.0677595866362	0\\
23.0671494296595	0\\
23.0665392298181	0\\
23.0659289871061	0\\
23.0653187015174	0\\
23.0647083730459	0\\
23.0640980016857	0\\
23.0634875874307	0\\
23.0628771302749	0\\
23.0622666302122	0\\
23.0616560872366	0\\
23.0610455013422	0\\
23.0604348725228	0\\
23.0598242007724	0\\
23.059213486085	0\\
23.0586027284545	0\\
23.0579919278749	0\\
23.0573810843402	0\\
23.0567701978444	0\\
23.0561592683813	0\\
23.0555482959449	0\\
23.0549372805292	0\\
23.0543262221282	0\\
23.0537151207357	0\\
23.0531039763458	0\\
23.0524927889524	0\\
23.0518815585494	0\\
23.0512702851308	0\\
23.0506589686905	0\\
23.0500476092225	0\\
23.0494362067207	0\\
23.0488247611791	0\\
23.0482132725916	0\\
23.0476017409521	0\\
23.0469901662545	0\\
23.0463785484929	0\\
23.0457668876611	0\\
23.0451551837531	0\\
23.0445434367628	0\\
23.0439316466842	0\\
23.0433198135111	0\\
23.0427079372375	0\\
23.0420960178574	0\\
23.0414840553646	0\\
23.040872049753	0\\
23.0402600010167	0\\
23.0396479091495	0\\
23.0390357741453	0\\
23.0384235959981	0\\
23.0378113747018	0\\
23.0371991102502	0\\
23.0365868026374	0\\
23.0359744518572	0\\
23.0353620579036	0\\
23.0347496207704	0\\
23.0341371404515	0\\
23.033524616941	0\\
23.0329120502326	0\\
23.0322994403203	0\\
23.0316867871979	0\\
23.0310740908595	0\\
23.0304613512989	0\\
23.02984856851	0\\
23.0292357424866	0\\
23.0286228732228	0\\
23.0280099607124	0\\
23.0273970049492	0\\
23.0267840059272	0\\
23.0261709636404	0\\
23.0255578780825	0\\
23.0249447492474	0\\
23.0243315771291	0\\
23.0237183617215	0\\
23.0231051030183	0\\
23.0224918010136	0\\
23.0218784557012	0\\
23.021265067075	0\\
23.0206516351288	0\\
23.0200381598566	0\\
23.0194246412522	0\\
23.0188110793095	0\\
23.0181974740224	0\\
23.0175838253848	0\\
23.0169701333905	0\\
23.0163563980333	0\\
23.0157426193073	0\\
23.0151287972062	0\\
23.0145149317239	0\\
23.0139010228543	0\\
23.0132870705912	0\\
23.0126730749286	0\\
23.0120590358602	0\\
23.01144495338	0\\
23.0108308274817	0\\
23.0102166581593	0\\
23.0096024454067	0\\
23.0089881892176	0\\
23.008373889586	0\\
23.0077595465056	0\\
23.0071451599704	0\\
23.0065307299742	0\\
23.0059162565108	0\\
23.0053017395741	0\\
23.0046871791579	0\\
23.0040725752562	0\\
23.0034579278626	0\\
23.0028432369712	0\\
23.0022285025756	0\\
23.0016137246699	0\\
23.0009989032477	0\\
23.000384038303	0\\
22.9997691298295	0\\
22.9991541778212	0\\
22.9985391822718	0\\
22.9979241431752	0\\
22.9973090605252	0\\
22.9966939343157	0\\
22.9960787645404	0\\
22.9954635511933	0\\
22.9948482942681	0\\
22.9942329937586	0\\
22.9936176496588	0\\
22.9930022619623	0\\
22.9923868306631	0\\
22.9917713557549	0\\
22.9911558372316	0\\
22.990540275087	0\\
22.9899246693148	0\\
22.989309019909	0\\
22.9886933268634	0\\
22.9880775901717	0\\
22.9874618098277	0\\
22.9868459858253	0\\
22.9862301181583	0\\
22.9856142068205	0\\
22.9849982518056	0\\
22.9843822531076	0\\
22.9837662107202	0\\
22.9831501246371	0\\
22.9825339948523	0\\
22.9819178213595	0\\
22.9813016041524	0\\
22.980685343225	0\\
22.9800690385709	0\\
22.9794526901841	0\\
22.9788362980582	0\\
22.9782198621871	0\\
22.9776033825646	0\\
22.9769868591844	0\\
22.9763702920404	0\\
22.9757536811262	0\\
22.9751370264358	0\\
22.9745203279629	0\\
22.9739035857013	0\\
22.9732867996447	0\\
22.972669969787	0\\
22.9720530961219	0\\
22.9714361786431	0\\
22.9708192173446	0\\
22.97020221222	0\\
22.9695851632631	0\\
22.9689680704677	0\\
22.9683509338275	0\\
22.9677337533364	0\\
22.9671165289881	0\\
22.9664992607763	0\\
22.9658819486949	0\\
22.9652645927376	0\\
22.9646471928981	0\\
22.9640297491702	0\\
22.9634122615477	0\\
22.9627947300244	0\\
22.9621771545939	0\\
22.9615595352501	0\\
22.9609418719867	0\\
22.9603241647975	0\\
22.9597064136761	0\\
22.9590886186164	0\\
22.9584707796121	0\\
22.957852896657	0\\
22.9572349697448	0\\
22.9566169988692	0\\
22.955998984024	0\\
22.9553809252029	0\\
22.9547628223998	0\\
22.9541446756082	0\\
22.953526484822	0\\
22.9529082500348	0\\
22.9522899712405	0\\
22.9516716484328	0\\
22.9510532816054	0\\
22.9504348707519	0\\
22.9498164158663	0\\
22.9491979169421	0\\
22.9485793739731	0\\
22.9479607869531	0\\
22.9473421558757	0\\
22.9467234807348	0\\
22.9461047615239	0\\
22.9454859982369	0\\
22.9448671908674	0\\
22.9442483394092	0\\
22.9436294438559	0\\
22.9430105042014	0\\
22.9423915204393	0\\
22.9417724925633	0\\
22.9411534205672	0\\
22.9405343044446	0\\
22.9399151441893	0\\
22.9392959397949	0\\
22.9386766912553	0\\
22.938057398564	0\\
22.9374380617148	0\\
22.9368186807013	0\\
22.9361992555174	0\\
22.9355797861567	0\\
22.9349602726128	0\\
22.9343407148795	0\\
22.9337211129505	0\\
22.9331014668195	0\\
22.9324817764801	0\\
22.9318620419261	0\\
22.9312422631511	0\\
22.9306224401489	0\\
22.9300025729131	0\\
22.9293826614374	0\\
22.9287627057155	0\\
22.9281427057411	0\\
22.9275226615078	0\\
22.9269025730094	0\\
22.9262824402395	0\\
22.9256622631917	0\\
22.9250420418599	0\\
22.9244217762376	0\\
22.9238014663185	0\\
22.9231811120964	0\\
22.9225607135648	0\\
};
\addplot [color=mycolor1, forget plot]
  table[row sep=crcr]{%
22.9225607135648	0\\
22.9219402707174	0\\
22.9213197835479	0\\
22.92069925205	0\\
22.9200786762174	0\\
22.9194580560436	0\\
22.9188373915224	0\\
22.9182166826475	0\\
22.9175959294124	0\\
22.9169751318109	0\\
22.9163542898366	0\\
22.9157334034831	0\\
22.9151124727441	0\\
22.9144914976134	0\\
22.9138704780844	0\\
22.9132494141509	0\\
22.9126283058066	0\\
22.912007153045	0\\
22.9113859558598	0\\
22.9107647142447	0\\
22.9101434281933	0\\
22.9095220976992	0\\
22.9089007227561	0\\
22.9082793033577	0\\
22.9076578394975	0\\
22.9070363311692	0\\
22.9064147783665	0\\
22.9057931810829	0\\
22.9051715393122	0\\
22.9045498530479	0\\
22.9039281222836	0\\
22.9033063470131	0\\
22.9026845272299	0\\
22.9020626629276	0\\
22.9014407541	0\\
22.9008188007405	0\\
22.9001968028428	0\\
22.8995747604006	0\\
22.8989526734075	0\\
22.898330541857	0\\
22.8977083657428	0\\
22.8970861450586	0\\
22.8964638797979	0\\
22.8958415699543	0\\
22.8952192155215	0\\
22.894596816493	0\\
22.8939743728625	0\\
22.8933518846236	0\\
22.8927293517698	0\\
22.8921067742949	0\\
22.8914841521923	0\\
22.8908614854558	0\\
22.8902387740788	0\\
22.889616018055	0\\
22.888993217378	0\\
22.8883703720414	0\\
22.8877474820387	0\\
22.8871245473637	0\\
22.8865015680098	0\\
22.8858785439706	0\\
22.8852554752398	0\\
22.8846323618109	0\\
22.8840092036776	0\\
22.8833860008333	0\\
22.8827627532718	0\\
22.8821394609865	0\\
22.881516123971	0\\
22.880892742219	0\\
22.880269315724	0\\
22.8796458444795	0\\
22.8790223284793	0\\
22.8783987677167	0\\
22.8777751621855	0\\
22.8771515118791	0\\
22.8765278167912	0\\
22.8759040769153	0\\
22.875280292245	0\\
22.8746564627738	0\\
22.8740325884954	0\\
22.8734086694032	0\\
22.8727847054909	0\\
22.8721606967519	0\\
22.8715366431799	0\\
22.8709125447684	0\\
22.870288401511	0\\
22.8696642134012	0\\
22.8690399804326	0\\
22.8684157025987	0\\
22.8677913798931	0\\
22.8671670123093	0\\
22.8665425998409	0\\
22.8659181424814	0\\
22.8652936402244	0\\
22.8646690930634	0\\
22.8640445009919	0\\
22.8634198640036	0\\
22.8627951820918	0\\
22.8621704552503	0\\
22.8615456834724	0\\
22.8609208667518	0\\
22.8602960050819	0\\
22.8596710984564	0\\
22.8590461468686	0\\
22.8584211503123	0\\
22.8577961087808	0\\
22.8571710222678	0\\
22.8565458907667	0\\
22.8559207142711	0\\
22.8552954927745	0\\
22.8546702262704	0\\
22.8540449147523	0\\
22.8534195582138	0\\
22.8527941566483	0\\
22.8521687100495	0\\
22.8515432184107	0\\
22.8509176817255	0\\
22.8502920999875	0\\
22.8496664731901	0\\
22.8490408013268	0\\
22.8484150843912	0\\
22.8477893223767	0\\
22.8471635152769	0\\
22.8465376630853	0\\
22.8459117657953	0\\
22.8452858234004	0\\
22.8446598358943	0\\
22.8440338032703	0\\
22.8434077255219	0\\
22.8427816026427	0\\
22.8421554346261	0\\
22.8415292214657	0\\
22.8409029631549	0\\
22.8402766596872	0\\
22.8396503110561	0\\
22.8390239172551	0\\
22.8383974782776	0\\
22.8377709941172	0\\
22.8371444647673	0\\
22.8365178902215	0\\
22.8358912704731	0\\
22.8352646055158	0\\
22.8346378953428	0\\
22.8340111399478	0\\
22.8333843393242	0\\
22.8327574934654	0\\
22.832130602365	0\\
22.8315036660164	0\\
22.8308766844131	0\\
22.8302496575485	0\\
22.8296225854161	0\\
22.8289954680093	0\\
22.8283683053218	0\\
22.8277410973467	0\\
22.8271138440778	0\\
22.8264865455083	0\\
22.8258592016319	0\\
22.8252318124418	0\\
22.8246043779316	0\\
22.8239768980947	0\\
22.8233493729246	0\\
22.8227218024148	0\\
22.8220941865586	0\\
22.8214665253495	0\\
22.820838818781	0\\
22.8202110668465	0\\
22.8195832695394	0\\
22.8189554268532	0\\
22.8183275387814	0\\
22.8176996053173	0\\
22.8170716264544	0\\
22.8164436021862	0\\
22.815815532506	0\\
22.8151874174074	0\\
22.8145592568837	0\\
22.8139310509283	0\\
22.8133027995347	0\\
22.8126745026964	0\\
22.8120461604066	0\\
22.811417772659	0\\
22.8107893394468	0\\
22.8101608607636	0\\
22.8095323366027	0\\
22.8089037669575	0\\
22.8082751518215	0\\
22.807646491188	0\\
22.8070177850506	0\\
22.8063890334026	0\\
22.8057602362373	0\\
22.8051313935483	0\\
22.804502505329	0\\
22.8038735715726	0\\
22.8032445922727	0\\
22.8026155674227	0\\
22.8019864970159	0\\
22.8013573810457	0\\
22.8007282195056	0\\
22.800099012389	0\\
22.7994697596892	0\\
22.7988404613996	0\\
22.7982111175137	0\\
22.7975817280247	0\\
22.7969522929262	0\\
22.7963228122115	0\\
22.795693285874	0\\
22.7950637139071	0\\
22.7944340963041	0\\
22.7938044330585	0\\
22.7931747241635	0\\
22.7925449696128	0\\
22.7919151693994	0\\
22.791285323517	0\\
22.7906554319588	0\\
22.7900254947182	0\\
22.7893955117886	0\\
22.7887654831634	0\\
22.7881354088358	0\\
22.7875052887994	0\\
22.7868751230475	0\\
22.7862449115734	0\\
22.7856146543705	0\\
22.7849843514321	0\\
22.7843540027517	0\\
22.7837236083226	0\\
22.7830931681381	0\\
22.7824626821916	0\\
22.7818321504765	0\\
22.781201572986	0\\
22.7805709497137	0\\
22.7799402806527	0\\
22.7793095657965	0\\
22.7786788051385	0\\
22.7780479986719	0\\
22.7774171463901	0\\
22.7767862482864	0\\
22.7761553043543	0\\
22.775524314587	0\\
22.7748932789778	0\\
22.7742621975202	0\\
22.7736310702075	0\\
22.7729998970329	0\\
22.7723686779898	0\\
22.7717374130717	0\\
22.7711061022717	0\\
22.7704747455832	0\\
22.7698433429995	0\\
22.769211894514	0\\
22.7685804001201	0\\
22.7679488598109	0\\
22.7673172735799	0\\
22.7666856414204	0\\
22.7660539633256	0\\
22.765422239289	0\\
22.7647904693038	0\\
22.7641586533633	0\\
22.7635267914608	0\\
22.7628948835897	0\\
22.7622629297433	0\\
22.7616309299149	0\\
22.7609988840978	0\\
22.7603667922853	0\\
22.7597346544706	0\\
22.7591024706472	0\\
22.7584702408083	0\\
22.7578379649472	0\\
22.7572056430572	0\\
22.7565732751316	0\\
22.7559408611637	0\\
22.7553084011468	0\\
22.7546758950743	0\\
22.7540433429392	0\\
22.7534107447351	0\\
22.7527781004551	0\\
22.7521454100926	0\\
22.7515126736408	0\\
22.7508798910931	0\\
22.7502470624426	0\\
22.7496141876827	0\\
22.7489812668067	0\\
22.7483482998078	0\\
22.7477152866794	0\\
22.7470822274147	0\\
22.7464491220069	0\\
22.7458159704494	0\\
22.7451827727354	0\\
22.7445495288581	0\\
22.743916238811	0\\
22.7432829025871	0\\
22.7426495201799	0\\
22.7420160915825	0\\
22.7413826167882	0\\
22.7407490957902	0\\
22.740115528582	0\\
22.7394819151566	0\\
22.7388482555073	0\\
22.7382145496275	0\\
22.7375807975103	0\\
22.7369469991491	0\\
22.736313154537	0\\
22.7356792636673	0\\
22.7350453265333	0\\
22.7344113431281	0\\
22.7337773134452	0\\
22.7331432374776	0\\
22.7325091152186	0\\
22.7318749466616	0\\
22.7312407317996	0\\
22.730606470626	0\\
22.729972163134	0\\
22.7293378093168	0\\
22.7287034091677	0\\
22.7280689626798	0\\
22.7274344698465	0\\
22.7267999306609	0\\
22.7261653451163	0\\
22.7255307132058	0\\
22.7248960349228	0\\
22.7242613102605	0\\
22.723626539212	0\\
22.7229917217706	0\\
22.7223568579295	0\\
22.7217219476819	0\\
22.7210869910211	0\\
22.7204519879402	0\\
22.7198169384325	0\\
22.7191818424911	0\\
22.7185467001094	0\\
22.7179115112804	0\\
22.7172762759974	0\\
22.7166409942536	0\\
22.7160056660422	0\\
22.7153702913564	0\\
22.7147348701895	0\\
22.7140994025345	0\\
22.7134638883847	0\\
22.7128283277334	0\\
22.7121927205736	0\\
22.7115570668986	0\\
22.7109213667016	0\\
22.7102856199757	0\\
22.7096498267142	0\\
22.7090139869103	0\\
22.708378100557	0\\
22.7077421676477	0\\
22.7071061881755	0\\
22.7064701621335	0\\
22.705834089515	0\\
22.7051979703132	0\\
22.7045618045211	0\\
22.703925592132	0\\
22.7032893331391	0\\
22.7026530275355	0\\
22.7020166753144	0\\
22.701380276469	0\\
22.7007438309924	0\\
22.7001073388777	0\\
22.6994708001183	0\\
22.6988342147072	0\\
22.6981975826376	0\\
22.6975609039026	0\\
22.6969241784954	0\\
22.6962874064092	0\\
22.6956505876371	0\\
22.6950137221723	0\\
22.6943768100079	0\\
22.6937398511371	0\\
22.693102845553	0\\
22.6924657932488	0\\
22.6918286942175	0\\
22.6911915484525	0\\
22.6905543559468	0\\
22.6899171166935	0\\
22.6892798306858	0\\
22.6886424979168	0\\
22.6880051183797	0\\
22.6873676920676	0\\
22.6867302189736	0\\
22.6860926990909	0\\
22.6854551324126	0\\
22.6848175189318	0\\
22.6841798586417	0\\
22.6835421515353	0\\
22.6829043976059	0\\
22.6822665968465	0\\
22.6816287492502	0\\
22.6809908548103	0\\
22.6803529135197	0\\
22.6797149253716	0\\
22.6790768903591	0\\
22.6784388084754	0\\
22.6778006797136	0\\
22.6771625040667	0\\
22.6765242815279	0\\
22.6758860120903	0\\
22.6752476957469	0\\
22.674609332491	0\\
22.6739709223155	0\\
22.6733324652137	0\\
22.6726939611786	0\\
22.6720554102032	0\\
22.6714168122808	0\\
22.6707781674044	0\\
22.670139475567	0\\
22.6695007367618	0\\
22.668861950982	0\\
22.6682231182204	0\\
22.6675842384704	0\\
22.6669453117249	0\\
22.666306337977	0\\
22.6656673172198	0\\
22.6650282494464	0\\
22.6643891346499	0\\
22.6637499728233	0\\
22.6631107639598	0\\
22.6624715080524	0\\
22.6618322050942	0\\
22.6611928550782	0\\
22.6605534579975	0\\
22.6599140138453	0\\
22.6592745226145	0\\
22.6586349842983	0\\
22.6579953988897	0\\
22.6573557663817	0\\
22.6567160867674	0\\
22.65607636004	0\\
22.6554365861924	0\\
22.6547967652177	0\\
22.654156897109	0\\
22.6535169818592	0\\
22.6528770194616	0\\
22.6522370099091	0\\
22.6515969531947	0\\
22.6509568493116	0\\
22.6503166982528	0\\
22.6496765000112	0\\
22.6490362545801	0\\
22.6483959619523	0\\
22.647755622121	0\\
22.6471152350791	0\\
22.6464748008198	0\\
22.6458343193361	0\\
22.6451937906209	0\\
22.6445532146674	0\\
22.6439125914685	0\\
22.6432719210173	0\\
22.6426312033068	0\\
22.6419904383301	0\\
22.6413496260801	0\\
22.64070876655	0\\
22.6400678597326	0\\
22.6394269056211	0\\
22.6387859042085	0\\
22.6381448554877	0\\
22.6375037594519	0\\
22.6368626160939	0\\
22.6362214254068	0\\
22.6355801873837	0\\
22.6349389020176	0\\
22.6342975693014	0\\
22.6336561892281	0\\
22.6330147617908	0\\
22.6323732869825	0\\
22.6317317647962	0\\
22.6310901952248	0\\
22.6304485782614	0\\
22.629806913899	0\\
22.6291652021305	0\\
22.628523442949	0\\
22.6278816363475	0\\
22.6272397823189	0\\
22.6265978808562	0\\
22.6259559319525	0\\
22.6253139356006	0\\
22.6246718917937	0\\
22.6240298005247	0\\
22.6233876617865	0\\
22.6227454755721	0\\
22.6221032418746	0\\
22.6214609606869	0\\
22.620818632002	0\\
22.6201762558128	0\\
22.6195338321123	0\\
22.6188913608936	0\\
22.6182488421495	0\\
22.617606275873	0\\
22.6169636620572	0\\
22.6163210006949	0\\
22.6156782917791	0\\
22.6150355353029	0\\
22.614392731259	0\\
22.6137498796406	0\\
22.6131069804406	0\\
22.6124640336519	0\\
22.6118210392675	0\\
22.6111779972803	0\\
22.6105349076833	0\\
22.6098917704694	0\\
22.6092485856316	0\\
22.6086053531628	0\\
22.607962073056	0\\
22.6073187453041	0\\
22.6066753699001	0\\
22.6060319468369	0\\
22.6053884761074	0\\
22.6047449577046	0\\
22.6041013916214	0\\
22.6034577778507	0\\
22.6028141163855	0\\
22.6021704072187	0\\
22.6015266503433	0\\
22.6008828457521	0\\
22.6002389934381	0\\
22.5995950933942	0\\
22.5989511456133	0\\
22.5983071500884	0\\
22.5976631068124	0\\
22.5970190157782	0\\
22.5963748769787	0\\
22.5957306904068	0\\
22.5950864560555	0\\
22.5944421739176	0\\
22.593797843986	0\\
22.5931534662538	0\\
22.5925090407137	0\\
22.5918645673587	0\\
22.5912200461817	0\\
22.5905754771755	0\\
22.5899308603332	0\\
22.5892861956475	0\\
22.5886414831115	0\\
22.5879967227179	0\\
22.5873519144597	0\\
22.5867070583298	0\\
22.586062154321	0\\
22.5854172024263	0\\
22.5847722026385	0\\
22.5841271549506	0\\
22.5834820593553	0\\
22.5828369158457	0\\
22.5821917244145	0\\
22.5815464850547	0\\
22.5809011977592	0\\
22.5802558625207	0\\
22.5796104793322	0\\
22.5789650481866	0\\
22.5783195690767	0\\
22.5776740419955	0\\
22.5770284669357	0\\
22.5763828438902	0\\
22.575737172852	0\\
22.5750914538138	0\\
22.5744456867685	0\\
22.5737998717091	0\\
22.5731540086283	0\\
22.572508097519	0\\
22.5718621383741	0\\
22.5712161311864	0\\
22.5705700759488	0\\
22.5699239726541	0\\
22.5692778212952	0\\
22.5686316218649	0\\
22.5679853743561	0\\
22.5673390787616	0\\
22.5666927350743	0\\
22.566046343287	0\\
22.5653999033925	0\\
22.5647534153837	0\\
22.5641068792535	0\\
22.5634602949946	0\\
22.5628136625998	0\\
22.5621669820622	0\\
22.5615202533743	0\\
22.5608734765291	0\\
22.5602266515195	0\\
22.5595797783382	0\\
22.558932856978	0\\
22.5582858874319	0\\
22.5576388696925	0\\
22.5569918037528	0\\
22.5563446896055	0\\
22.5556975272434	0\\
22.5550503166595	0\\
22.5544030578464	0\\
22.553755750797	0\\
22.5531083955041	0\\
22.5524609919606	0\\
22.5518135401592	0\\
22.5511660400927	0\\
22.5505184917539	0\\
22.5498708951356	0\\
22.5492232502307	0\\
22.548575557032	0\\
22.5479278155321	0\\
22.547280025724	0\\
22.5466321876004	0\\
22.5459843011541	0\\
22.545336366378	0\\
22.5446883832647	0\\
22.5440403518071	0\\
22.5433922719979	0\\
22.54274414383	0\\
22.5420959672962	0\\
22.5414477423891	0\\
22.5407994691016	0\\
22.5401511474265	0\\
22.5395027773566	0\\
22.5388543588846	0\\
22.5382058920032	0\\
22.5375573767053	0\\
22.5369088129837	0\\
22.5362602008311	0\\
22.5356115402402	0\\
22.5349628312038	0\\
22.5343140737148	0\\
22.5336652677658	0\\
22.5330164133496	0\\
22.532367510459	0\\
22.5317185590868	0\\
22.5310695592256	0\\
22.5304205108682	0\\
22.5297714140075	0\\
22.5291222686361	0\\
22.5284730747468	0\\
22.5278238323323	0\\
22.5271745413854	0\\
22.5265252018988	0\\
22.5258758138653	0\\
22.5252263772776	0\\
22.5245768921285	0\\
22.5239273584106	0\\
22.5232777761168	0\\
22.5226281452397	0\\
22.5219784657721	0\\
22.5213287377067	0\\
22.5206789610363	0\\
22.5200291357535	0\\
22.5193792618511	0\\
22.5187293393219	0\\
22.5180793681585	0\\
22.5174293483536	0\\
22.5167792799001	0\\
22.5161291627905	0\\
22.5154789970177	0\\
22.5148287825743	0\\
22.5141785194531	0\\
22.5135282076467	0\\
22.5128778471478	0\\
22.5122274379493	0\\
22.5115769800437	0\\
22.5109264734238	0\\
22.5102759180823	0\\
22.5096253140119	0\\
22.5089746612053	0\\
22.5083239596551	0\\
22.5076732093541	0\\
22.507022410295	0\\
22.5063715624705	0\\
22.5057206658733	0\\
22.505069720496	0\\
22.5044187263313	0\\
22.5037676833719	0\\
22.5031165916106	0\\
22.50246545104	0\\
22.5018142616527	0\\
22.5011630234415	0\\
22.500511736399	0\\
22.4998604005179	0\\
22.4992090157909	0\\
22.4985575822107	0\\
22.4979060997698	0\\
22.4972545684611	0\\
22.4966029882772	0\\
22.4959513592107	0\\
22.4952996812542	0\\
22.4946479544006	0\\
22.4939961786424	0\\
22.4933443539722	0\\
22.4926924803828	0\\
22.4920405578668	0\\
22.4913885864168	0\\
22.4907365660255	0\\
22.4900844966856	0\\
22.4894323783898	0\\
22.4887802111305	0\\
22.4881279949006	0\\
22.4874757296926	0\\
22.4868234154992	0\\
22.486171052313	0\\
22.4855186401268	0\\
22.484866178933	0\\
22.4842136687243	0\\
22.4835611094935	0\\
22.4829085012331	0\\
22.4822558439357	0\\
22.481603137594	0\\
22.4809503822006	0\\
22.4802975777482	0\\
22.4796447242293	0\\
22.4789918216365	0\\
22.4783388699626	0\\
22.4776858692002	0\\
22.4770328193417	0\\
22.4763797203799	0\\
22.4757265723074	0\\
22.4750733751168	0\\
22.4744201288007	0\\
22.4737668333517	0\\
22.4731134887624	0\\
22.4724600950255	0\\
22.4718066521335	0\\
22.471153160079	0\\
22.4704996188546	0\\
22.469846028453	0\\
22.4691923888668	0\\
22.4685387000884	0\\
22.4678849621106	0\\
22.4672311749259	0\\
22.466577338527	0\\
22.4659234529063	0\\
22.4652695180565	0\\
22.4646155339702	0\\
22.46396150064	0\\
22.4633074180584	0\\
22.462653286218	0\\
22.4619991051115	0\\
22.4613448747313	0\\
22.4606905950701	0\\
22.4600362661205	0\\
22.4593818878749	0\\
22.4587274603261	0\\
22.4580729834665	0\\
22.4574184572887	0\\
22.4567638817853	0\\
22.4561092569489	0\\
22.455454582772	0\\
22.4547998592472	0\\
22.454145086367	0\\
22.453490264124	0\\
22.4528353925107	0\\
22.4521804715198	0\\
22.4515255011437	0\\
22.4508704813751	0\\
22.4502154122064	0\\
22.4495602936302	0\\
22.4489051256391	0\\
22.4482499082256	0\\
22.4475946413822	0\\
22.4469393251016	0\\
22.4462839593761	0\\
22.4456285441984	0\\
22.4449730795611	0\\
22.4443175654565	0\\
22.4436620018774	0\\
22.4430063888161	0\\
22.4423507262653	0\\
22.4416950142174	0\\
22.441039252665	0\\
22.4403834416006	0\\
22.4397275810168	0\\
22.439071670906	0\\
22.4384157112608	0\\
22.4377597020736	0\\
22.4371036433371	0\\
22.4364475350437	0\\
22.4357913771859	0\\
22.4351351697563	0\\
22.4344789127473	0\\
22.4338226061515	0\\
22.4331662499614	0\\
22.4325098441694	0\\
22.4318533887681	0\\
22.43119688375	0\\
22.4305403291076	0\\
22.4298837248333	0\\
22.4292270709197	0\\
22.4285703673593	0\\
22.4279136141445	0\\
22.4272568112678	0\\
22.4265999587218	0\\
22.4259430564989	0\\
22.4252861045916	0\\
22.4246291029924	0\\
22.4239720516938	0\\
22.4233149506882	0\\
22.4226577999682	0\\
22.4220005995261	0\\
22.4213433493546	0\\
22.420686049446	0\\
22.4200286997928	0\\
22.4193713003875	0\\
22.4187138512226	0\\
22.4180563522905	0\\
22.4173988035838	0\\
22.4167412050947	0\\
22.4160835568159	0\\
22.4154258587398	0\\
22.4147681108589	0\\
22.4141103131655	0\\
22.4134524656522	0\\
22.4127945683113	0\\
22.4121366211355	0\\
22.411478624117	0\\
22.4108205772484	0\\
22.4101624805221	0\\
22.4095043339305	0\\
22.4088461374661	0\\
22.4081878911214	0\\
22.4075295948887	0\\
22.4068712487605	0\\
22.4062128527292	0\\
22.4055544067874	0\\
22.4048959109273	0\\
22.4042373651415	0\\
22.4035787694223	0\\
22.4029201237622	0\\
22.4022614281537	0\\
22.4016026825891	0\\
22.4009438870608	0\\
22.4002850415613	0\\
22.3996261460831	0\\
22.3989672006184	0\\
22.3983082051598	0\\
22.3976491596997	0\\
22.3969900642304	0\\
22.3963309187443	0\\
22.395671723234	0\\
22.3950124776917	0\\
22.3943531821099	0\\
22.393693836481	0\\
22.3930344407974	0\\
22.3923749950515	0\\
22.3917154992357	0\\
22.3910559533423	0\\
22.3903963573639	0\\
22.3897367112927	0\\
22.3890770151212	0\\
22.3884172688417	0\\
22.3877574724467	0\\
22.3870976259285	0\\
22.3864377292795	0\\
22.3857777824922	0\\
22.3851177855587	0\\
22.3844577384717	0\\
22.3837976412233	0\\
22.3831374938061	0\\
22.3824772962123	0\\
22.3818170484344	0\\
22.3811567504647	0\\
22.3804964022957	0\\
22.3798360039195	0\\
22.3791755553287	0\\
22.3785150565155	0\\
22.3778545074724	0\\
22.3771939081917	0\\
22.3765332586658	0\\
22.375872558887	0\\
22.3752118088476	0\\
22.3745510085401	0\\
22.3738901579568	0\\
22.3732292570899	0\\
22.372568305932	0\\
22.3719073044752	0\\
22.3712462527121	0\\
22.3705851506348	0\\
22.3699239982358	0\\
22.3692627955073	0\\
22.3686015424418	0\\
22.3679402390315	0\\
22.3672788852688	0\\
22.3666174811461	0\\
22.3659560266555	0\\
22.3652945217896	0\\
22.3646329665405	0\\
22.3639713609007	0\\
22.3633097048625	0\\
22.3626479984181	0\\
22.3619862415599	0\\
22.3613244342802	0\\
22.3606625765713	0\\
22.3600006684256	0\\
22.3593387098353	0\\
22.3586767007928	0\\
22.3580146412904	0\\
22.3573525313203	0\\
22.3566903708749	0\\
22.3560281599466	0\\
22.3553658985274	0\\
22.3547035866099	0\\
22.3540412241863	0\\
22.3533788112489	0\\
22.3527163477899	0\\
22.3520538338017	0\\
22.3513912692765	0\\
22.3507286542068	0\\
22.3500659885846	0\\
22.3494032724024	0\\
22.3487405056524	0\\
22.3480776883269	0\\
22.3474148204181	0\\
22.3467519019184	0\\
22.3460889328201	0\\
22.3454259131153	0\\
22.3447628427964	0\\
22.3440997218557	0\\
22.3434365502853	0\\
22.3427733280777	0\\
22.342110055225	0\\
22.3414467317196	0\\
22.3407833575536	0\\
22.3401199327193	0\\
22.3394564572091	0\\
22.3387929310151	0\\
22.3381293541297	0\\
22.337465726545	0\\
22.3368020482533	0\\
22.3361383192469	0\\
22.335474539518	0\\
22.3348107090589	0\\
22.3341468278618	0\\
22.3334828959189	0\\
22.3328189132226	0\\
22.332154879765	0\\
22.3314907955383	0\\
22.3308266605349	0\\
22.3301624747469	0\\
22.3294982381666	0\\
22.3288339507862	0\\
22.328169612598	0\\
22.3275052235941	0\\
22.3268407837668	0\\
22.3261762931083	0\\
22.3255117516109	0\\
22.3248471592667	0\\
22.324182516068	0\\
22.323517822007	0\\
22.3228530770759	0\\
22.3221882812669	0\\
22.3215234345722	0\\
22.3208585369841	0\\
22.3201935884948	0\\
22.3195285890964	0\\
22.3188635387811	0\\
22.3181984375412	0\\
22.3175332853689	0\\
22.3168680822563	0\\
22.3162028281957	0\\
22.3155375231793	0\\
22.3148721671992	0\\
22.3142067602476	0\\
22.3135413023167	0\\
22.3128757933988	0\\
22.312210233486	0\\
22.3115446225705	0\\
22.3108789606444	0\\
22.3102132477	0\\
22.3095474837294	0\\
22.3088816687248	0\\
22.3082158026784	0\\
22.3075498855823	0\\
22.3068839174288	0\\
22.30621789821	0\\
22.305551827918	0\\
22.3048857065451	0\\
22.3042195340834	0\\
22.303553310525	0\\
22.3028870358621	0\\
22.302220710087	0\\
22.3015543331916	0\\
22.3008879051683	0\\
22.3002214260091	0\\
22.2995548957062	0\\
22.2988883142517	0\\
22.2982216816378	0\\
22.2975549978567	0\\
22.2968882629005	0\\
22.2962214767613	0\\
22.2955546394312	0\\
22.2948877509025	0\\
22.2942208111672	0\\
22.2935538202176	0\\
22.2928867780456	0\\
22.2922196846435	0\\
22.2915525400033	0\\
22.2908853441173	0\\
22.2902180969775	0\\
22.2895507985761	0\\
22.2888834489051	0\\
22.2882160479567	0\\
22.2875485957231	0\\
22.2868810921963	0\\
22.2862135373685	0\\
22.2855459312317	0\\
22.2848782737781	0\\
22.2842105649999	0\\
22.283542804889	0\\
22.2828749934376	0\\
22.2822071306378	0\\
22.2815392164818	0\\
22.2808712509615	0\\
22.2802032340692	0\\
22.2795351657969	0\\
22.2788670461367	0\\
22.2781988750807	0\\
22.277530652621	0\\
22.2768623787497	0\\
22.2761940534589	0\\
22.2755256767406	0\\
22.274857248587	0\\
22.2741887689901	0\\
22.273520237942	0\\
22.2728516554348	0\\
22.2721830214606	0\\
22.2715143360114	0\\
22.2708455990793	0\\
22.2701768106564	0\\
22.2695079707347	0\\
22.2688390793064	0\\
22.2681701363635	0\\
22.267501141898	0\\
22.2668320959021	0\\
22.2661629983677	0\\
22.265493849287	0\\
22.264824648652	0\\
22.2641553964547	0\\
22.2634860926872	0\\
22.2628167373416	0\\
22.2621473304099	0\\
22.2614778718842	0\\
22.2608083617564	0\\
22.2601388000187	0\\
22.2594691866631	0\\
22.2587995216816	0\\
22.2581298050663	0\\
22.2574600368092	0\\
22.2567902169023	0\\
22.2561203453377	0\\
22.2554504221074	0\\
22.2547804472034	0\\
22.2541104206179	0\\
22.2534403423426	0\\
22.2527702123698	0\\
22.2521000306915	0\\
22.2514297972996	0\\
22.2507595121862	0\\
22.2500891753432	0\\
22.2494187867628	0\\
22.2487483464369	0\\
22.2480778543575	0\\
22.2474073105166	0\\
22.2467367149063	0\\
22.2460660675186	0\\
22.2453953683454	0\\
22.2447246173788	0\\
22.2440538146107	0\\
22.2433829600332	0\\
22.2427120536382	0\\
22.2420410954178	0\\
22.2413700853639	0\\
22.2406990234686	0\\
22.2400279097238	0\\
22.2393567441215	0\\
22.2386855266536	0\\
22.2380142573123	0\\
22.2373429360894	0\\
22.2366715629769	0\\
22.2360001379669	0\\
22.2353286610512	0\\
22.234657132222	0\\
22.233985551471	0\\
22.2333139187904	0\\
22.232642234172	0\\
22.2319704976079	0\\
22.23129870909	0\\
22.2306268686102	0\\
22.2299549761606	0\\
22.2292830317331	0\\
22.2286110353196	0\\
22.2279389869121	0\\
22.2272668865026	0\\
22.226594734083	0\\
22.2259225296452	0\\
22.2252502731813	0\\
22.2245779646831	0\\
22.2239056041426	0\\
22.2232331915518	0\\
22.2225607269025	0\\
22.2218882101867	0\\
22.2212156413964	0\\
22.2205430205235	0\\
22.2198703475599	0\\
22.2191976224976	0\\
22.2185248453284	0\\
22.2178520160444	0\\
22.2171791346373	0\\
22.2165062010993	0\\
22.2158332154221	0\\
22.2151601775978	0\\
22.2144870876181	0\\
22.2138139454751	0\\
22.2131407511606	0\\
22.2124675046666	0\\
22.211794205985	0\\
22.2111208551076	0\\
22.2104474520265	0\\
22.2097739967334	0\\
22.2091004892203	0\\
22.2084269294791	0\\
22.2077533175017	0\\
22.20707965328	0\\
22.2064059368059	0\\
22.2057321680713	0\\
22.2050583470681	0\\
22.2043844737881	0\\
22.2037105482232	0\\
22.2030365703654	0\\
22.2023625402066	0\\
22.2016884577385	0\\
22.2010143229531	0\\
22.2003401358422	0\\
22.1996658963978	0\\
22.1989916046118	0\\
22.1983172604759	0\\
22.197642863982	0\\
22.1969684151221	0\\
22.196293913888	0\\
22.1956193602715	0\\
22.1949447542646	0\\
22.194270095859	0\\
22.1935953850467	0\\
22.1929206218195	0\\
22.1922458061692	0\\
22.1915709380877	0\\
22.1908960175669	0\\
22.1902210445986	0\\
22.1895460191747	0\\
22.188870941287	0\\
22.1881958109273	0\\
22.1875206280875	0\\
22.1868453927595	0\\
22.186170104935	0\\
22.1854947646059	0\\
22.1848193717641	0\\
22.1841439264013	0\\
22.1834684285094	0\\
22.1827928780803	0\\
22.1821172751058	0\\
22.1814416195776	0\\
22.1807659114876	0\\
22.1800901508277	0\\
22.1794143375896	0\\
22.1787384717652	0\\
22.1780625533463	0\\
22.1773865823246	0\\
22.1767105586921	0\\
22.1760344824405	0\\
22.1753583535616	0\\
22.1746821720473	0\\
22.1740059378893	0\\
22.1733296510794	0\\
22.1726533116095	0\\
22.1719769194713	0\\
22.1713004746567	0\\
22.1706239771574	0\\
22.1699474269652	0\\
22.1692708240719	0\\
22.1685941684693	0\\
22.1679174601492	0\\
22.1672406991034	0\\
22.1665638853237	0\\
22.1658870188018	0\\
22.1652100995295	0\\
22.1645331274986	0\\
22.1638561027008	0\\
22.1631790251281	0\\
22.162501894772	0\\
22.1618247116244	0\\
22.1611474756771	0\\
22.1604701869218	0\\
22.1597928453502	0\\
22.1591154509543	0\\
22.1584380037256	0\\
22.157760503656	0\\
22.1570829507372	0\\
22.156405344961	0\\
22.1557276863191	0\\
22.1550499748033	0\\
22.1543722104053	0\\
22.1536943931168	0\\
22.1530165229297	0\\
22.1523385998357	0\\
22.1516606238264	0\\
22.1509825948937	0\\
22.1503045130293	0\\
22.1496263782249	0\\
22.1489481904722	0\\
22.148269949763	0\\
22.147591656089	0\\
22.1469133094419	0\\
22.1462349098135	0\\
22.1455564571954	0\\
22.1448779515795	0\\
22.1441993929574	0\\
22.1435207813208	0\\
22.1428421166614	0\\
22.1421633989711	0\\
22.1414846282414	0\\
22.1408058044641	0\\
22.1401269276309	0\\
22.1394479977335	0\\
22.1387690147636	0\\
22.1380899787129	0\\
22.1374108895731	0\\
22.1367317473359	0\\
22.136052551993	0\\
22.1353733035361	0\\
22.1346940019569	0\\
22.1340146472471	0\\
22.1333352393983	0\\
22.1326557784023	0\\
22.1319762642507	0\\
22.1312966969353	0\\
22.1306170764476	0\\
22.1299374027794	0\\
22.1292576759224	0\\
22.1285778958682	0\\
22.1278980626085	0\\
22.127218176135	0\\
22.1265382364393	0\\
22.1258582435132	0\\
22.1251781973481	0\\
22.124498097936	0\\
22.1238179452683	0\\
22.1231377393368	0\\
22.1224574801331	0\\
22.1217771676488	0\\
22.1210968018757	0\\
22.1204163828054	0\\
22.1197359104295	0\\
22.1190553847396	0\\
22.1183748057275	0\\
22.1176941733847	0\\
22.1170134877029	0\\
22.1163327486737	0\\
22.1156519562888	0\\
22.1149711105399	0\\
22.1142902114184	0\\
22.1136092589162	0\\
22.1129282530247	0\\
22.1122471937357	0\\
22.1115660810407	0\\
22.1108849149315	0\\
22.1102036953995	0\\
22.1095224224364	0\\
22.1088410960339	0\\
22.1081597161835	0\\
22.1074782828769	0\\
22.1067967961057	0\\
22.1061152558615	0\\
22.1054336621359	0\\
22.1047520149205	0\\
22.1040703142068	0\\
22.1033885599866	0\\
22.1027067522514	0\\
22.1020248909928	0\\
22.1013429762025	0\\
22.1006610078719	0\\
22.0999789859927	0\\
22.0992969105565	0\\
22.0986147815548	0\\
22.0979325989793	0\\
22.0972503628215	0\\
22.0965680730731	0\\
22.0958857297256	0\\
22.0952033327705	0\\
22.0945208821995	0\\
22.0938383780041	0\\
22.0931558201759	0\\
22.0924732087065	0\\
22.0917905435874	0\\
22.0911078248103	0\\
22.0904250523666	0\\
22.0897422262479	0\\
22.0890593464458	0\\
22.0883764129519	0\\
22.0876934257577	0\\
22.0870103848547	0\\
22.0863272902346	0\\
22.0856441418888	0\\
22.0849609398089	0\\
22.0842776839865	0\\
22.083594374413	0\\
22.0829110110802	0\\
22.0822275939794	0\\
22.0815441231022	0\\
22.0808605984402	0\\
22.0801770199849	0\\
22.0794933877278	0\\
22.0788097016604	0\\
22.0781259617744	0\\
22.0774421680611	0\\
22.0767583205122	0\\
22.0760744191191	0\\
22.0753904638735	0\\
22.0747064547667	0\\
22.0740223917903	0\\
22.0733382749358	0\\
22.0726541041948	0\\
22.0719698795588	0\\
22.0712856010191	0\\
22.0706012685675	0\\
22.0699168821953	0\\
22.0692324418941	0\\
22.0685479476554	0\\
22.0678633994706	0\\
22.0671787973313	0\\
22.0664941412289	0\\
22.065809431155	0\\
22.065124667101	0\\
22.0644398490585	0\\
22.0637549770189	0\\
22.0630700509736	0\\
22.0623850709143	0\\
22.0617000368323	0\\
22.0610149487191	0\\
22.0603298065663	0\\
22.0596446103653	0\\
22.0589593601075	0\\
22.0582740557844	0\\
22.0575886973876	0\\
22.0569032849084	0\\
22.0562178183383	0\\
22.0555322976688	0\\
22.0548467228914	0\\
22.0541610939975	0\\
22.0534754109785	0\\
22.052789673826	0\\
22.0521038825313	0\\
22.051418037086	0\\
22.0507321374814	0\\
22.050046183709	0\\
22.0493601757603	0\\
22.0486741136267	0\\
22.0479879972996	0\\
22.0473018267706	0\\
22.0466156020309	0\\
22.045929323072	0\\
22.0452429898855	0\\
22.0445566024626	0\\
22.0438701607949	0\\
22.0431836648737	0\\
22.0424971146905	0\\
22.0418105102367	0\\
22.0411238515038	0\\
22.0404371384831	0\\
22.039750371166	0\\
22.039063549544	0\\
22.0383766736085	0\\
22.0376897433508	0\\
22.0370027587625	0\\
22.0363157198349	0\\
22.0356286265594	0\\
22.0349414789273	0\\
22.0342542769302	0\\
22.0335670205594	0\\
22.0328797098063	0\\
22.0321923446622	0\\
22.0315049251187	0\\
22.030817451167	0\\
22.0301299227986	0\\
22.0294423400048	0\\
22.028754702777	0\\
22.0280670111066	0\\
22.027379264985	0\\
22.0266914644036	0\\
22.0260036093537	0\\
22.0253156998267	0\\
22.024627735814	0\\
22.0239397173069	0\\
22.0232516442968	0\\
22.0225635167751	0\\
22.0218753347332	0\\
22.0211870981623	0\\
22.0204988070539	0\\
22.0198104613993	0\\
22.0191220611899	0\\
22.0184336064169	0\\
22.0177450970719	0\\
22.017056533146	0\\
22.0163679146307	0\\
22.0156792415173	0\\
22.0149905137972	0\\
22.0143017314616	0\\
22.013612894502	0\\
22.0129240029096	0\\
22.0122350566758	0\\
22.0115460557919	0\\
22.0108570002493	0\\
22.0101678900392	0\\
22.0094787251531	0\\
22.0087895055822	0\\
22.0081002313178	0\\
22.0074109023513	0\\
22.0067215186739	0\\
22.0060320802771	0\\
22.005342587152	0\\
22.0046530392901	0\\
22.0039634366826	0\\
22.0032737793209	0\\
22.0025840671962	0\\
22.0018943002998	0\\
22.0012044786231	0\\
22.0005146021573	0\\
21.9998246708937	0\\
21.9991346848237	0\\
21.9984446439384	0\\
21.9977545482293	0\\
21.9970643976876	0\\
21.9963741923046	0\\
21.9956839320716	0\\
21.9949936169797	0\\
21.9943032470205	0\\
21.993612822185	0\\
21.9929223424646	0\\
21.9922318078506	0\\
21.9915412183342	0\\
21.9908505739067	0\\
21.9901598745593	0\\
21.9894691202834	0\\
21.9887783110701	0\\
21.9880874469108	0\\
21.9873965277968	0\\
21.9867055537192	0\\
21.9860145246693	0\\
21.9853234406383	0\\
21.9846323016176	0\\
21.9839411075984	0\\
21.9832498585719	0\\
21.9825585545293	0\\
21.981867195462	0\\
21.9811757813611	0\\
21.9804843122179	0\\
21.9797927880235	0\\
21.9791012087694	0\\
21.9784095744466	0\\
21.9777178850464	0\\
21.9770261405601	0\\
21.9763343409788	0\\
21.9756424862938	0\\
21.9749505764964	0\\
21.9742586115776	0\\
21.9735665915288	0\\
21.9728745163411	0\\
21.9721823860059	0\\
21.9714902005141	0\\
21.9707979598572	0\\
21.9701056640263	0\\
21.9694133130126	0\\
21.9687209068073	0\\
21.9680284454015	0\\
21.9673359287866	0\\
21.9666433569537	0\\
21.965950729894	0\\
21.9652580475986	0\\
21.9645653100588	0\\
21.9638725172658	0\\
21.9631796692107	0\\
21.9624867658847	0\\
21.961793807279	0\\
21.9611007933849	0\\
21.9604077241934	0\\
21.9597145996957	0\\
21.959021419883	0\\
21.9583281847465	0\\
21.9576348942774	0\\
21.9569415484667	0\\
21.9562481473058	0\\
21.9555546907857	0\\
21.9548611788976	0\\
21.9541676116326	0\\
21.953473988982	0\\
21.9527803109368	0\\
21.9520865774882	0\\
21.9513927886274	0\\
21.9506989443456	0\\
21.9500050446337	0\\
21.9493110894831	0\\
21.9486170788848	0\\
21.9479230128301	0\\
21.9472288913099	0\\
21.9465347143155	0\\
21.9458404818379	0\\
21.9451461938684	0\\
21.944451850398	0\\
21.9437574514179	0\\
21.9430629969192	0\\
21.942368486893	0\\
21.9416739213304	0\\
21.9409793002226	0\\
21.9402846235607	0\\
21.9395898913357	0\\
21.9388951035388	0\\
21.9382002601611	0\\
21.9375053611938	0\\
21.9368104066278	0\\
21.9361153964543	0\\
21.9354203306645	0\\
21.9347252092494	0\\
21.9340300322001	0\\
21.9333347995076	0\\
21.9326395111632	0\\
21.9319441671579	0\\
21.9312487674827	0\\
21.9305533121288	0\\
21.9298578010872	0\\
21.9291622343491	0\\
21.9284666119054	0\\
21.9277709337473	0\\
21.9270751998659	0\\
21.9263794102522	0\\
21.9256835648973	0\\
21.9249876637922	0\\
21.9242917069281	0\\
21.923595694296	0\\
21.9228996258869	0\\
21.9222035016919	0\\
21.9215073217021	0\\
21.9208110859086	0\\
21.9201147943023	0\\
21.9194184468743	0\\
21.9187220436157	0\\
21.9180255845175	0\\
21.9173290695709	0\\
21.9166324987667	0\\
21.915935872096	0\\
21.91523918955	0\\
21.9145424511196	0\\
21.9138456567958	0\\
21.9131488065698	0\\
21.9124519004324	0\\
21.9117549383749	0\\
21.911057920388	0\\
21.910360846463	0\\
21.9096637165908	0\\
21.9089665307624	0\\
21.9082692889689	0\\
21.9075719912013	0\\
21.9068746374505	0\\
21.9061772277076	0\\
21.9054797619636	0\\
21.9047822402096	0\\
21.9040846624364	0\\
21.9033870286351	0\\
21.9026893387967	0\\
21.9019915929123	0\\
21.9012937909727	0\\
21.900595932969	0\\
21.8998980188922	0\\
21.8992000487333	0\\
21.8985020224832	0\\
21.897803940133	0\\
21.8971058016736	0\\
21.896407607096	0\\
21.8957093563912	0\\
21.8950110495501	0\\
21.8943126865637	0\\
21.8936142674231	0\\
21.8929157921191	0\\
21.8922172606427	0\\
21.891518672985	0\\
21.8908200291368	0\\
21.8901213290891	0\\
21.8894225728329	0\\
21.8887237603591	0\\
21.8880248916587	0\\
21.8873259667227	0\\
21.8866269855419	0\\
21.8859279481074	0\\
21.88522885441	0\\
21.8845297044408	0\\
21.8838304981906	0\\
21.8831312356505	0\\
21.8824319168112	0\\
21.8817325416639	0\\
21.8810331101993	0\\
21.8803336224084	0\\
21.8796340782823	0\\
21.8789344778117	0\\
21.8782348209876	0\\
21.8775351078009	0\\
21.8768353382426	0\\
21.8761355123035	0\\
21.8754356299746	0\\
21.8747356912468	0\\
21.874035696111	0\\
21.8733356445581	0\\
21.872635536579	0\\
21.8719353721646	0\\
21.8712351513058	0\\
21.8705348739935	0\\
21.8698345402186	0\\
21.8691341499721	0\\
21.8684337032447	0\\
21.8677332000274	0\\
21.8670326403111	0\\
21.8663320240866	0\\
21.8656313513449	0\\
21.8649306220768	0\\
21.8642298362732	0\\
21.863528993925	0\\
21.862828095023	0\\
21.8621271395581	0\\
21.8614261275213	0\\
21.8607250589033	0\\
21.860023933695	0\\
21.8593227518873	0\\
21.858621513471	0\\
21.8579202184371	0\\
21.8572188667763	0\\
21.8565174584796	0\\
21.8558159935377	0\\
21.8551144719416	0\\
21.854412893682	0\\
21.8537112587498	0\\
21.8530095671359	0\\
21.8523078188311	0\\
21.8516060138262	0\\
21.8509041521121	0\\
21.8502022336796	0\\
21.8495002585196	0\\
21.8487982266228	0\\
21.8480961379802	0\\
21.8473939925824	0\\
21.8466917904204	0\\
21.845989531485	0\\
21.845287215767	0\\
21.8445848432571	0\\
21.8438824139463	0\\
21.8431799278254	0\\
21.842477384885	0\\
21.8417747851161	0\\
21.8410721285095	0\\
21.8403694150559	0\\
21.8396666447462	0\\
21.8389638175712	0\\
21.8382609335216	0\\
21.8375579925883	0\\
21.836854994762	0\\
21.8361519400335	0\\
21.8354488283937	0\\
21.8347456598332	0\\
21.834042434343	0\\
21.8333391519137	0\\
21.8326358125362	0\\
21.8319324162012	0\\
21.8312289628994	0\\
21.8305254526218	0\\
21.829821885359	0\\
21.8291182611018	0\\
21.828414579841	0\\
21.8277108415674	0\\
21.8270070462716	0\\
21.8263031939445	0\\
21.8255992845768	0\\
21.8248953181592	0\\
21.8241912946826	0\\
21.8234872141376	0\\
21.8227830765151	0\\
21.8220788818057	0\\
21.8213746300002	0\\
21.8206703210894	0\\
21.819965955064	0\\
21.8192615319146	0\\
21.8185570516322	0\\
21.8178525142073	0\\
21.8171479196307	0\\
21.8164432678931	0\\
21.8157385589854	0\\
21.8150337928981	0\\
21.814328969622	0\\
21.8136240891478	0\\
21.8129191514663	0\\
21.8122141565681	0\\
21.811509104444	0\\
21.8108039950847	0\\
21.8100988284809	0\\
21.8093936046232	0\\
21.8086883235024	0\\
21.8079829851093	0\\
21.8072775894344	0\\
21.8065721364684	0\\
21.8058666262022	0\\
21.8051610586263	0\\
21.8044554337315	0\\
21.8037497515084	0\\
21.8030440119478	0\\
21.8023382150402	0\\
21.8016323607764	0\\
21.8009264491471	0\\
21.8002204801429	0\\
21.7995144537546	0\\
21.7988083699727	0\\
21.798102228788	0\\
21.797396030191	0\\
21.7966897741726	0\\
21.7959834607233	0\\
21.7952770898338	0\\
21.7945706614947	0\\
21.7938641756968	0\\
21.7931576324306	0\\
21.7924510316868	0\\
21.7917443734561	0\\
21.7910376577291	0\\
21.7903308844964	0\\
21.7896240537487	0\\
21.7889171654767	0\\
21.7882102196709	0\\
21.787503216322	0\\
21.7867961554206	0\\
21.7860890369574	0\\
21.785381860923	0\\
21.784674627308	0\\
21.783967336103	0\\
21.7832599872986	0\\
21.7825525808856	0\\
21.7818451168544	0\\
21.7811375951957	0\\
21.7804300159001	0\\
21.7797223789583	0\\
21.7790146843607	0\\
21.7783069320981	0\\
21.777599122161	0\\
21.7768912545401	0\\
21.7761833292259	0\\
21.775475346209	0\\
21.77476730548	0\\
21.7740592070295	0\\
21.7733510508481	0\\
21.7726428369264	0\\
21.7719345652549	0\\
21.7712262358243	0\\
21.7705178486251	0\\
21.7698094036479	0\\
21.7691009008833	0\\
21.7683923403218	0\\
21.7676837219541	0\\
21.7669750457706	0\\
21.7662663117619	0\\
21.7655575199187	0\\
21.7648486702314	0\\
21.7641397626907	0\\
21.763430797287	0\\
21.762721774011	0\\
21.7620126928532	0\\
21.7613035538041	0\\
21.7605943568543	0\\
21.7598851019943	0\\
21.7591757892147	0\\
21.7584664185059	0\\
21.7577569898587	0\\
21.7570475032634	0\\
21.7563379587106	0\\
21.7556283561908	0\\
21.7549186956947	0\\
21.7542089772126	0\\
21.7534992007351	0\\
21.7527893662528	0\\
21.7520794737562	0\\
21.7513695232357	0\\
21.7506595146819	0\\
21.7499494480853	0\\
21.7492393234364	0\\
21.7485291407257	0\\
21.7478188999438	0\\
21.747108601081	0\\
21.746398244128	0\\
21.7456878290752	0\\
21.7449773559132	0\\
21.7442668246323	0\\
21.7435562352231	0\\
21.7428455876762	0\\
21.7421348819819	0\\
21.7414241181307	0\\
21.7407132961132	0\\
21.7400024159198	0\\
21.739291477541	0\\
21.7385804809673	0\\
21.7378694261891	0\\
21.7371583131969	0\\
21.7364471419812	0\\
21.7357359125325	0\\
21.7350246248411	0\\
21.7343132788977	0\\
21.7336018746925	0\\
21.7328904122161	0\\
21.732178891459	0\\
21.7314673124115	0\\
21.7307556750642	0\\
21.7300439794074	0\\
21.7293322254316	0\\
21.7286204131274	0\\
21.727908542485	0\\
21.727196613495	0\\
21.7264846261477	0\\
21.7257725804336	0\\
21.7250604763432	0\\
21.7243483138669	0\\
21.723636092995	0\\
21.722923813718	0\\
21.7222114760264	0\\
21.7214990799105	0\\
21.7207866253608	0\\
21.7200741123676	0\\
21.7193615409215	0\\
21.7186489110127	0\\
21.7179362226318	0\\
21.717223475769	0\\
21.7165106704149	0\\
21.7157978065597	0\\
21.715084884194	0\\
21.714371903308	0\\
21.7136588638923	0\\
21.7129457659371	0\\
21.7122326094329	0\\
21.7115193943701	0\\
21.710806120739	0\\
21.71009278853	0\\
21.7093793977334	0\\
21.7086659483398	0\\
21.7079524403394	0\\
21.7072388737226	0\\
21.7065252484798	0\\
21.7058115646013	0\\
21.7050978220775	0\\
21.7043840208988	0\\
21.7036701610556	0\\
21.7029562425381	0\\
21.7022422653367	0\\
21.7015282294419	0\\
21.7008141348439	0\\
21.700099981533	0\\
21.6993857694997	0\\
21.6986714987343	0\\
21.6979571692271	0\\
21.6972427809684	0\\
21.6965283339487	0\\
21.6958138281581	0\\
21.6950992635871	0\\
21.6943846402259	0\\
21.693669958065	0\\
21.6929552170946	0\\
21.692240417305	0\\
21.6915255586866	0\\
21.6908106412297	0\\
21.6900956649246	0\\
21.6893806297615	0\\
21.6886655357309	0\\
21.687950382823	0\\
21.6872351710281	0\\
21.6865199003365	0\\
21.6858045707386	0\\
21.6850891822246	0\\
21.6843737347848	0\\
21.6836582284095	0\\
21.682942663089	0\\
21.6822270388135	0\\
21.6815113555735	0\\
21.6807956133591	0\\
21.6800798121606	0\\
21.6793639519683	0\\
21.6786480327725	0\\
21.6779320545635	0\\
21.6772160173315	0\\
21.6764999210667	0\\
21.6757837657596	0\\
21.6750675514002	0\\
21.6743512779789	0\\
21.673634945486	0\\
21.6729185539117	0\\
21.6722021032462	0\\
21.6714855934798	0\\
21.6707690246028	0\\
21.6700523966053	0\\
21.6693357094777	0\\
21.6686189632102	0\\
21.667902157793	0\\
21.6671852932164	0\\
21.6664683694705	0\\
21.6657513865457	0\\
21.6650343444321	0\\
21.6643172431199	0\\
21.6636000825995	0\\
21.662882862861	0\\
21.6621655838947	0\\
21.6614482456907	0\\
21.6607308482393	0\\
21.6600133915307	0\\
21.659295875555	0\\
21.6585783003026	0\\
21.6578606657636	0\\
21.6571429719282	0\\
21.6564252187866	0\\
21.655707406329	0\\
21.6549895345456	0\\
21.6542716034266	0\\
21.6535536129622	0\\
21.6528355631426	0\\
21.652117453958	0\\
21.6513992853985	0\\
21.6506810574543	0\\
21.6499627701156	0\\
21.6492444233726	0\\
21.6485260172154	0\\
21.6478075516343	0\\
21.6470890266194	0\\
21.6463704421608	0\\
21.6456517982487	0\\
21.6449330948734	0\\
21.6442143320249	0\\
21.6434955096933	0\\
21.642776627869	0\\
21.6420576865419	0\\
21.6413386857023	0\\
21.6406196253402	0\\
21.639900505446	0\\
21.6391813260096	0\\
21.6384620870212	0\\
21.637742788471	0\\
21.6370234303491	0\\
21.6363040126456	0\\
21.6355845353507	0\\
21.6348649984544	0\\
21.634145401947	0\\
21.6334257458185	0\\
21.632706030059	0\\
21.6319862546587	0\\
21.6312664196077	0\\
21.6305465248961	0\\
21.629826570514	0\\
21.6291065564515	0\\
21.6283864826987	0\\
21.6276663492457	0\\
21.6269461560826	0\\
21.6262259031996	0\\
21.6255055905867	0\\
21.6247852182339	0\\
21.6240647861315	0\\
21.6233442942695	0\\
21.6226237426379	0\\
21.6219031312268	0\\
21.6211824600264	0\\
21.6204617290267	0\\
21.6197409382177	0\\
21.6190200875897	0\\
21.6182991771325	0\\
21.6175782068363	0\\
21.6168571766912	0\\
21.6161360866872	0\\
21.6154149368144	0\\
21.6146937270629	0\\
21.6139724574226	0\\
21.6132511278837	0\\
21.6125297384361	0\\
21.61180828907	0\\
21.6110867797754	0\\
21.6103652105424	0\\
21.6096435813609	0\\
21.608921892221	0\\
21.6082001431127	0\\
21.6074783340262	0\\
21.6067564649513	0\\
21.6060345358782	0\\
21.6053125467969	0\\
21.6045904976973	0\\
21.6038683885696	0\\
21.6031462194037	0\\
21.6024239901896	0\\
21.6017017009174	0\\
21.600979351577	0\\
21.6002569421586	0\\
21.599534472652	0\\
21.5988119430472	0\\
21.5980893533344	0\\
21.5973667035035	0\\
21.5966439935444	0\\
21.5959212234472	0\\
21.5951983932019	0\\
21.5944755027984	0\\
21.5937525522267	0\\
21.5930295414769	0\\
21.5923064705389	0\\
21.5915833394026	0\\
21.5908601480581	0\\
21.5901368964954	0\\
21.5894135847043	0\\
21.5886902126749	0\\
21.5879667803971	0\\
21.5872432878609	0\\
21.5865197350563	0\\
21.5857961219732	0\\
21.5850724486016	0\\
21.5843487149313	0\\
21.5836249209525	0\\
21.582901066655	0\\
21.5821771520287	0\\
21.5814531770637	0\\
21.5807291417498	0\\
21.580005046077	0\\
21.5792808900352	0\\
21.5785566736144	0\\
21.5778323968045	0\\
21.5771080595953	0\\
21.576383661977	0\\
21.5756592039393	0\\
21.5749346854721	0\\
21.5742101065655	0\\
21.5734854672093	0\\
21.5727607673935	0\\
21.5720360071078	0\\
21.5713111863423	0\\
21.5705863050869	0\\
21.5698613633315	0\\
21.5691363610659	0\\
21.56841129828	0\\
21.5676861749638	0\\
21.5669609911071	0\\
21.5662357466999	0\\
21.565510441732	0\\
21.5647850761933	0\\
21.5640596500737	0\\
21.5633341633631	0\\
21.5626086160513	0\\
21.5618830081282	0\\
21.5611573395837	0\\
21.5604316104077	0\\
21.55970582059	0\\
21.5589799701206	0\\
21.5582540589891	0\\
21.5575280871856	0\\
21.5568020546999	0\\
21.5560759615218	0\\
21.5553498076411	0\\
21.5546235930478	0\\
21.5538973177317	0\\
21.5531709816826	0\\
21.5524445848904	0\\
21.5517181273449	0\\
21.5509916090359	0\\
21.5502650299533	0\\
21.5495383900868	0\\
21.5488116894265	0\\
21.548084927962	0\\
21.5473581056831	0\\
21.5466312225798	0\\
21.5459042786418	0\\
21.545177273859	0\\
21.5444502082211	0\\
21.5437230817179	0\\
21.5429958943394	0\\
21.5422686460752	0\\
21.5415413369152	0\\
21.5408139668492	0\\
21.540086535867	0\\
21.5393590439584	0\\
21.5386314911131	0\\
21.537903877321	0\\
21.5371762025719	0\\
21.5364484668555	0\\
21.5357206701616	0\\
21.53499281248	0\\
21.5342648938005	0\\
21.5335369141129	0\\
21.5328088734068	0\\
21.5320807716722	0\\
21.5313526088988	0\\
21.5306243850762	0\\
21.5298961001944	0\\
21.529167754243	0\\
21.5284393472119	0\\
21.5277108790907	0\\
21.5269823498692	0\\
21.5262537595372	0\\
21.5255251080844	0\\
21.5247963955005	0\\
21.5240676217754	0\\
21.5233387868987	0\\
21.5226098908601	0\\
21.5218809336495	0\\
21.5211519152565	0\\
21.5204228356709	0\\
21.5196936948824	0\\
21.5189644928807	0\\
21.5182352296556	0\\
21.5175059051967	0\\
21.5167765194937	0\\
21.5160470725365	0\\
21.5153175643147	0\\
21.514587994818	0\\
21.5138583640361	0\\
21.5131286719587	0\\
21.5123989185755	0\\
21.5116691038763	0\\
21.5109392278506	0\\
21.5102092904883	0\\
21.5094792917789	0\\
21.5087492317123	0\\
21.508019110278	0\\
21.5072889274657	0\\
21.5065586832652	0\\
21.505828377666	0\\
21.505098010658	0\\
21.5043675822307	0\\
21.5036370923738	0\\
21.502906541077	0\\
21.50217592833	0\\
21.5014452541224	0\\
21.5007145184438	0\\
21.499983721284	0\\
21.4992528626325	0\\
21.4985219424791	0\\
21.4977909608134	0\\
21.4970599176249	0\\
21.4963288129035	0\\
21.4955976466387	0\\
21.4948664188201	0\\
21.4941351294373	0\\
21.4934037784801	0\\
21.4926723659381	0\\
21.4919408918008	0\\
21.4912093560579	0\\
21.490477758699	0\\
21.4897460997138	0\\
21.4890143790918	0\\
21.4882825968227	0\\
21.4875507528961	0\\
21.4868188473015	0\\
21.4860868800287	0\\
21.4853548510672	0\\
21.4846227604065	0\\
21.4838906080364	0\\
21.4831583939463	0\\
21.482426118126	0\\
21.4816937805649	0\\
21.4809613812527	0\\
21.4802289201789	0\\
21.4794963973332	0\\
21.4787638127051	0\\
21.4780311662841	0\\
21.47729845806	0\\
21.4765656880222	0\\
21.4758328561603	0\\
21.4750999624638	0\\
21.4743670069225	0\\
21.4736339895257	0\\
21.472900910263	0\\
21.4721677691241	0\\
21.4714345660985	0\\
21.4707013011757	0\\
21.4699679743452	0\\
21.4692345855967	0\\
21.4685011349196	0\\
21.4677676223036	0\\
21.467034047738	0\\
21.4663004112126	0\\
21.4655667127167	0\\
21.46483295224	0\\
21.4640991297719	0\\
21.463365245302	0\\
21.4626312988199	0\\
21.4618972903149	0\\
21.4611632197767	0\\
21.4604290871948	0\\
21.4596948925586	0\\
21.4589606358577	0\\
21.4582263170816	0\\
21.4574919362198	0\\
21.4567574932617	0\\
21.456022988197	0\\
21.455288421015	0\\
21.4545537917053	0\\
21.4538191002574	0\\
21.4530843466607	0\\
21.4523495309047	0\\
21.451614652979	0\\
21.450879712873	0\\
21.4501447105761	0\\
21.4494096460779	0\\
21.4486745193678	0\\
21.4479393304352	0\\
21.4472040792698	0\\
21.4464687658608	0\\
21.4457333901978	0\\
21.4449979522703	0\\
21.4442624520676	0\\
21.4435268895793	0\\
21.4427912647948	0\\
21.4420555777035	0\\
21.4413198282949	0\\
21.4405840165584	0\\
21.4398481424835	0\\
21.4391122060596	0\\
21.4383762072761	0\\
21.4376401461225	0\\
21.4369040225882	0\\
21.4361678366626	0\\
21.4354315883351	0\\
21.4346952775952	0\\
21.4339589044323	0\\
21.4332224688358	0\\
21.432485970795	0\\
21.4317494102995	0\\
21.4310127873386	0\\
21.4302761019018	0\\
21.4295393539783	0\\
21.4288025435578	0\\
21.4280656706294	0\\
21.4273287351826	0\\
21.4265917372068	0\\
21.4258546766915	0\\
21.4251175536259	0\\
21.4243803679995	0\\
21.4236431198016	0\\
21.4229058090217	0\\
21.422168435649	0\\
21.421430999673	0\\
21.420693501083	0\\
21.4199559398685	0\\
21.4192183160187	0\\
21.418480629523	0\\
21.4177428803708	0\\
21.4170050685515	0\\
21.4162671940543	0\\
21.4155292568687	0\\
21.414791256984	0\\
21.4140531943895	0\\
21.4133150690746	0\\
21.4125768810285	0\\
21.4118386302408	0\\
21.4111003167006	0\\
21.4103619403974	0\\
21.4096235013204	0\\
21.4088849994589	0\\
21.4081464348024	0\\
21.4074078073401	0\\
21.4066691170613	0\\
21.4059303639553	0\\
21.4051915480115	0\\
21.4044526692192	0\\
21.4037137275676	0\\
21.4029747230461	0\\
21.402235655644	0\\
21.4014965253506	0\\
21.4007573321551	0\\
21.4000180760469	0\\
21.3992787570153	0\\
21.3985393750494	0\\
21.3977999301388	0\\
21.3970604222725	0\\
21.3963208514399	0\\
21.3955812176303	0\\
21.3948415208329	0\\
21.394101761037	0\\
21.3933619382319	0\\
21.3926220524069	0\\
21.3918821035511	0\\
21.3911420916539	0\\
21.3904020167045	0\\
21.3896618786922	0\\
21.3889216776063	0\\
21.3881814134359	0\\
21.3874410861703	0\\
21.3867006957988	0\\
21.3859602423106	0\\
21.3852197256949	0\\
21.3844791459411	0\\
21.3837385030382	0\\
21.3829977969756	0\\
21.3822570277424	0\\
21.381516195328	0\\
21.3807752997214	0\\
21.380034340912	0\\
21.3792933188889	0\\
21.3785522336414	0\\
21.3778110851587	0\\
21.37706987343	0\\
21.3763285984444	0\\
21.3755872601912	0\\
21.3748458586596	0\\
21.3741043938388	0\\
21.373362865718	0\\
21.3726212742863	0\\
21.371879619533	0\\
21.3711379014473	0\\
21.3703961200182	0\\
21.3696542752351	0\\
21.368912367087	0\\
21.3681703955632	0\\
21.3674283606529	0\\
21.3666862623451	0\\
21.3659441006291	0\\
21.365201875494	0\\
21.364459586929	0\\
21.3637172349232	0\\
21.3629748194659	0\\
21.3622323405461	0\\
21.361489798153	0\\
21.3607471922757	0\\
21.3600045229034	0\\
21.3592617900253	0\\
21.3585189936304	0\\
21.357776133708	0\\
21.3570332102471	0\\
21.3562902232368	0\\
21.3555471726663	0\\
21.3548040585248	0\\
21.3540608808013	0\\
21.353317639485	0\\
21.3525743345649	0\\
21.3518309660302	0\\
21.3510875338701	0\\
21.3503440380735	0\\
21.3496004786297	0\\
21.3488568555276	0\\
21.3481131687565	0\\
21.3473694183054	0\\
21.3466256041634	0\\
21.3458817263195	0\\
21.345137784763	0\\
21.3443937794828	0\\
21.3436497104681	0\\
21.3429055777079	0\\
21.3421613811913	0\\
21.3414171209074	0\\
21.3406727968452	0\\
21.3399284089939	0\\
21.3391839573424	0\\
21.3384394418799	0\\
21.3376948625954	0\\
21.336950219478	0\\
21.3362055125167	0\\
21.3354607417005	0\\
21.3347159070186	0\\
21.3339710084599	0\\
21.3332260460135	0\\
21.3324810196685	0\\
21.3317359294139	0\\
21.3309907752387	0\\
21.3302455571319	0\\
21.3295002750826	0\\
21.3287549290798	0\\
21.3280095191126	0\\
21.3272640451699	0\\
21.3265185072408	0\\
21.3257729053142	0\\
21.3250272393792	0\\
21.3242815094249	0\\
21.3235357154401	0\\
21.322789857414	0\\
21.3220439353355	0\\
21.3212979491935	0\\
21.3205518989772	0\\
21.3198057846755	0\\
21.3190596062774	0\\
21.3183133637718	0\\
21.3175670571478	0\\
21.3168206863944	0\\
21.3160742515004	0\\
21.315327752455	0\\
21.314581189247	0\\
21.3138345618654	0\\
21.3130878702993	0\\
21.3123411145374	0\\
21.311594294569	0\\
21.3108474103828	0\\
21.3101004619678	0\\
21.309353449313	0\\
21.3086063724074	0\\
21.3078592312398	0\\
21.3071120257993	0\\
21.3063647560747	0\\
21.3056174220551	0\\
21.3048700237293	0\\
21.3041225610862	0\\
21.3033750341149	0\\
21.3026274428042	0\\
21.301879787143	0\\
21.3011320671204	0\\
21.3003842827251	0\\
21.2996364339461	0\\
21.2988885207724	0\\
21.2981405431927	0\\
21.2973925011961	0\\
21.2966443947715	0\\
21.2958962239077	0\\
21.2951479885936	0\\
21.2943996888181	0\\
21.2936513245702	0\\
21.2929028958387	0\\
21.2921544026125	0\\
21.2914058448805	0\\
21.2906572226316	0\\
21.2899085358546	0\\
21.2891597845384	0\\
21.2884109686719	0\\
21.2876620882439	0\\
21.2869131432434	0\\
21.2861641336592	0\\
21.2854150594801	0\\
21.2846659206951	0\\
21.2839167172929	0\\
21.2831674492624	0\\
21.2824181165924	0\\
21.2816687192719	0\\
21.2809192572897	0\\
21.2801697306345	0\\
21.2794201392952	0\\
21.2786704832608	0\\
21.2779207625199	0\\
21.2771709770614	0\\
21.2764211268742	0\\
21.2756712119471	0\\
21.2749212322689	0\\
21.2741711878283	0\\
21.2734210786144	0\\
21.2726709046157	0\\
21.2719206658212	0\\
21.2711703622196	0\\
21.2704199937998	0\\
21.2696695605506	0\\
21.2689190624607	0\\
21.2681684995189	0\\
21.2674178717141	0\\
21.266667179035	0\\
21.2659164214704	0\\
21.2651655990091	0\\
21.2644147116399	0\\
21.2636637593515	0\\
21.2629127421328	0\\
21.2621616599724	0\\
21.2614105128592	0\\
21.2606593007819	0\\
21.2599080237293	0\\
21.2591566816902	0\\
21.2584052746533	0\\
21.2576538026073	0\\
21.256902265541	0\\
21.2561506634432	0\\
21.2553989963026	0\\
21.2546472641079	0\\
21.2538954668479	0\\
21.2531436045113	0\\
21.2523916770868	0\\
21.2516396845633	0\\
21.2508876269293	0\\
21.2501355041737	0\\
21.2493833162851	0\\
21.2486310632522	0\\
21.2478787450639	0\\
21.2471263617088	0\\
21.2463739131755	0\\
21.2456213994529	0\\
21.2448688205296	0\\
21.2441161763943	0\\
21.2433634670357	0\\
21.2426106924425	0\\
21.2418578526034	0\\
21.2411049475071	0\\
21.2403519771422	0\\
21.2395989414976	0\\
21.2388458405617	0\\
21.2380926743234	0\\
21.2373394427713	0\\
21.236586145894	0\\
21.2358327836802	0\\
21.2350793561187	0\\
21.2343258631979	0\\
21.2335723049067	0\\
21.2328186812337	0\\
21.2320649921675	0\\
21.2313112376968	0\\
21.2305574178102	0\\
21.2298035324963	0\\
21.2290495817439	0\\
21.2282955655415	0\\
21.2275414838778	0\\
21.2267873367413	0\\
21.2260331241209	0\\
21.225278846005	0\\
21.2245245023823	0\\
21.2237700932414	0\\
21.223015618571	0\\
21.2222610783596	0\\
21.2215064725958	0\\
21.2207518012684	0\\
21.2199970643658	0\\
21.2192422618767	0\\
21.2184873937896	0\\
21.2177324600932	0\\
21.2169774607761	0\\
21.2162223958269	0\\
21.2154672652341	0\\
21.2147120689864	0\\
21.2139568070722	0\\
21.2132014794803	0\\
21.2124460861991	0\\
21.2116906272173	0\\
21.2109351025234	0\\
21.210179512106	0\\
21.2094238559536	0\\
21.2086681340548	0\\
21.2079123463982	0\\
21.2071564929723	0\\
21.2064005737657	0\\
21.2056445887669	0\\
21.2048885379645	0\\
21.204132421347	0\\
21.203376238903	0\\
21.2026199906209	0\\
21.2018636764894	0\\
21.2011072964969	0\\
21.200350850632	0\\
21.1995943388832	0\\
21.1988377612391	0\\
21.1980811176881	0\\
21.1973244082188	0\\
21.1965676328197	0\\
21.1958107914792	0\\
21.195053884186	0\\
21.1942969109284	0\\
21.1935398716951	0\\
21.1927827664745	0\\
21.1920255952551	0\\
21.1912683580254	0\\
21.1905110547738	0\\
21.1897536854889	0\\
21.1889962501592	0\\
21.1882387487731	0\\
21.1874811813192	0\\
21.1867235477858	0\\
21.1859658481614	0\\
21.1852080824346	0\\
21.1844502505938	0\\
21.1836923526274	0\\
21.182934388524	0\\
21.1821763582719	0\\
21.1814182618596	0\\
21.1806600992756	0\\
21.1799018705083	0\\
21.1791435755462	0\\
21.1783852143777	0\\
21.1776267869912	0\\
21.1768682933752	0\\
21.1761097335181	0\\
21.1753511074084	0\\
21.1745924150344	0\\
21.1738336563846	0\\
21.1730748314474	0\\
21.1723159402112	0\\
21.1715569826645	0\\
21.1707979587956	0\\
21.170038868593	0\\
21.169279712045	0\\
21.1685204891401	0\\
21.1677611998667	0\\
21.1670018442131	0\\
21.1662424221677	0\\
21.165482933719	0\\
21.1647233788553	0\\
21.1639637575651	0\\
21.1632040698366	0\\
21.1624443156583	0\\
21.1616844950185	0\\
21.1609246079056	0\\
21.1601646543081	0\\
21.1594046342141	0\\
21.1586445476122	0\\
21.1578843944906	0\\
21.1571241748378	0\\
21.156363888642	0\\
21.1556035358916	0\\
21.1548431165751	0\\
21.1540826306806	0\\
21.1533220781966	0\\
21.1525614591114	0\\
21.1518007734133	0\\
21.1510400210907	0\\
21.1502792021318	0\\
21.1495183165251	0\\
21.1487573642588	0\\
21.1479963453212	0\\
21.1472352597008	0\\
21.1464741073857	0\\
21.1457128883643	0\\
21.1449516026249	0\\
21.1441902501558	0\\
21.1434288309453	0\\
21.1426673449817	0\\
21.1419057922534	0\\
21.1411441727484	0\\
21.1403824864553	0\\
21.1396207333623	0\\
21.1388589134575	0\\
21.1380970267295	0\\
21.1373350731663	0\\
21.1365730527562	0\\
21.1358109654877	0\\
21.1350488113488	0\\
21.1342865903279	0\\
21.1335243024132	0\\
21.132761947593	0\\
21.1319995258556	0\\
21.1312370371892	0\\
21.130474481582	0\\
21.1297118590223	0\\
21.1289491694983	0\\
21.1281864129984	0\\
21.1274235895106	0\\
21.1266606990232	0\\
21.1258977415246	0\\
21.1251347170028	0\\
21.1243716254461	0\\
21.1236084668428	0\\
21.122845241181	0\\
21.1220819484491	0\\
21.1213185886351	0\\
21.1205551617272	0\\
21.1197916677138	0\\
21.119028106583	0\\
21.118264478323	0\\
21.1175007829219	0\\
21.1167370203681	0\\
21.1159731906496	0\\
21.1152092937547	0\\
21.1144453296715	0\\
21.1136812983883	0\\
21.1129171998931	0\\
21.1121530341742	0\\
21.1113888012198	0\\
21.110624501018	0\\
21.1098601335569	0\\
21.1090956988248	0\\
21.1083311968098	0\\
21.1075666275	0\\
21.1068019908836	0\\
21.1060372869488	0\\
21.1052725156837	0\\
21.1045076770764	0\\
21.1037427711151	0\\
21.1029777977879	0\\
21.1022127570829	0\\
21.1014476489884	0\\
21.1006824734923	0\\
21.0999172305828	0\\
21.0991519202481	0\\
21.0983865424763	0\\
21.0976210972554	0\\
21.0968555845737	0\\
21.0960900044191	0\\
21.0953243567798	0\\
21.094558641644	0\\
21.0937928589996	0\\
21.0930270088348	0\\
21.0922610911378	0\\
21.0914951058965	0\\
21.090729053099	0\\
21.0899629327335	0\\
21.0891967447881	0\\
21.0884304892507	0\\
21.0876641661095	0\\
21.0868977753526	0\\
21.086131316968	0\\
21.0853647909437	0\\
21.0845981972679	0\\
21.0838315359286	0\\
21.0830648069139	0\\
21.0822980102117	0\\
21.0815311458102	0\\
21.0807642136974	0\\
21.0799972138613	0\\
21.07923014629	0\\
21.0784630109715	0\\
21.0776958078938	0\\
21.0769285370451	0\\
21.0761611984132	0\\
21.0753937919862	0\\
21.0746263177521	0\\
21.0738587756991	0\\
21.0730911658149	0\\
21.0723234880878	0\\
21.0715557425057	0\\
21.0707879290565	0\\
21.0700200477283	0\\
21.0692520985091	0\\
21.0684840813869	0\\
21.0677159963496	0\\
21.0669478433853	0\\
21.066179622482	0\\
21.0654113336276	0\\
21.06464297681	0\\
21.0638745520174	0\\
21.0631060592376	0\\
21.0623374984586	0\\
21.0615688696684	0\\
21.0608001728549	0\\
21.0600314080062	0\\
21.0592625751101	0\\
21.0584936741546	0\\
21.0577247051276	0\\
21.0569556680172	0\\
21.0561865628112	0\\
21.0554173894976	0\\
21.0546481480644	0\\
21.0538788384994	0\\
21.0531094607905	0\\
21.0523400149258	0\\
21.0515705008932	0\\
21.0508009186804	0\\
21.0500312682756	0\\
21.0492615496666	0\\
21.0484917628412	0\\
21.0477219077875	0\\
21.0469519844933	0\\
21.0461819929465	0\\
21.045411933135	0\\
21.0446418050467	0\\
21.0438716086695	0\\
21.0431013439913	0\\
21.042331011	0\\
21.0415606096834	0\\
21.0407901400294	0\\
21.040019602026	0\\
21.0392489956609	0\\
21.0384783209221	0\\
21.0377075777974	0\\
21.0369367662747	0\\
21.0361658863418	0\\
21.0353949379866	0\\
21.034623921197	0\\
21.0338528359607	0\\
21.0330816822657	0\\
21.0323104600998	0\\
21.0315391694508	0\\
21.0307678103065	0\\
21.0299963826549	0\\
21.0292248864837	0\\
21.0284533217807	0\\
21.0276816885338	0\\
21.0269099867308	0\\
21.0261382163595	0\\
21.0253663774077	0\\
21.0245944698633	0\\
21.023822493714	0\\
21.0230504489476	0\\
21.022278335552	0\\
21.021506153515	0\\
21.0207339028243	0\\
21.0199615834677	0\\
21.019189195433	0\\
21.0184167387081	0\\
21.0176442132806	0\\
21.0168716191384	0\\
21.0160989562692	0\\
21.0153262246608	0\\
21.0145534243011	0\\
21.0137805551776	0\\
21.0130076172783	0\\
21.0122346105908	0\\
21.011461535103	0\\
21.0106883908025	0\\
21.0099151776771	0\\
21.0091418957146	0\\
21.0083685449027	0\\
21.0075951252292	0\\
21.0068216366817	0\\
21.006048079248	0\\
21.0052744529159	0\\
21.0045007576731	0\\
21.0037269935072	0\\
21.0029531604061	0\\
21.0021792583574	0\\
21.0014052873488	0\\
21.000631247368	0\\
20.9998571384029	0\\
20.9990829604409	0\\
20.99830871347	0\\
20.9975343974776	0\\
20.9967600124517	0\\
20.9959855583797	0\\
20.9952110352495	0\\
20.9944364430487	0\\
20.993661781765	0\\
20.992887051386	0\\
20.9921122518995	0\\
20.991337383293	0\\
20.9905624455544	0\\
20.9897874386711	0\\
20.989012362631	0\\
20.9882372174216	0\\
20.9874620030306	0\\
20.9866867194457	0\\
20.9859113666544	0\\
20.9851359446445	0\\
20.9843604534036	0\\
20.9835848929193	0\\
20.9828092631793	0\\
20.9820335641711	0\\
20.9812577958825	0\\
20.980481958301	0\\
20.9797060514142	0\\
20.9789300752098	0\\
20.9781540296754	0\\
20.9773779147986	0\\
20.976601730567	0\\
20.9758254769682	0\\
20.9750491539898	0\\
20.9742727616194	0\\
20.9734962998446	0\\
20.972719768653	0\\
20.9719431680321	0\\
20.9711664979696	0\\
20.9703897584531	0\\
20.96961294947	0\\
20.9688360710081	0\\
20.9680591230548	0\\
20.9672821055977	0\\
20.9665050186244	0\\
20.9657278621225	0\\
20.9649506360795	0\\
20.9641733404829	0\\
20.9633959753204	0\\
20.9626185405793	0\\
20.9618410362474	0\\
20.9610634623122	0\\
20.9602858187611	0\\
20.9595081055817	0\\
20.9587303227616	0\\
20.9579524702882	0\\
20.9571745481491	0\\
20.9563965563319	0\\
20.9556184948239	0\\
20.9548403636128	0\\
20.9540621626861	0\\
20.9532838920312	0\\
20.9525055516357	0\\
20.951727141487	0\\
20.9509486615727	0\\
20.9501701118802	0\\
20.9493914923971	0\\
20.9486128031108	0\\
20.9478340440089	0\\
20.9470552150787	0\\
20.9462763163078	0\\
20.9454973476836	0\\
20.9447183091936	0\\
20.9439392008253	0\\
20.9431600225662	0\\
20.9423807744036	0\\
20.9416014563251	0\\
20.9408220683181	0\\
20.9400426103701	0\\
20.9392630824685	0\\
20.9384834846007	0\\
20.9377038167542	0\\
20.9369240789164	0\\
20.9361442710748	0\\
20.9353643932168	0\\
20.9345844453297	0\\
20.9338044274011	0\\
20.9330243394184	0\\
20.9322441813689	0\\
20.9314639532401	0\\
20.9306836550194	0\\
20.9299032866942	0\\
20.9291228482518	0\\
20.9283423396798	0\\
20.9275617609654	0\\
20.9267811120962	0\\
20.9260003930594	0\\
20.9252196038424	0\\
20.9244387444327	0\\
20.9236578148176	0\\
20.9228768149845	0\\
20.9220957449208	0\\
20.9213146046138	0\\
20.9205333940508	0\\
20.9197521132193	0\\
20.9189707621067	0\\
20.9181893407002	0\\
20.9174078489872	0\\
20.9166262869551	0\\
20.9158446545912	0\\
20.9150629518828	0\\
20.9142811788174	0\\
20.9134993353821	0\\
20.9127174215644	0\\
20.9119354373516	0\\
20.911153382731	0\\
20.9103712576899	0\\
20.9095890622156	0\\
20.9088067962955	0\\
20.9080244599168	0\\
20.907242053067	0\\
20.9064595757331	0\\
20.9056770279027	0\\
20.9048944095629	0\\
20.9041117207011	0\\
20.9033289613046	0\\
20.9025461313605	0\\
20.9017632308563	0\\
20.9009802597792	0\\
20.9001972181165	0\\
20.8994141058554	0\\
20.8986309229832	0\\
20.8978476694872	0\\
20.8970643453546	0\\
20.8962809505728	0\\
20.8954974851288	0\\
20.8947139490101	0\\
20.8939303422039	0\\
20.8931466646973	0\\
20.8923629164777	0\\
20.8915790975323	0\\
20.8907952078482	0\\
20.8900112474128	0\\
20.8892272162133	0\\
20.8884431142369	0\\
20.8876589414707	0\\
20.8868746979021	0\\
20.8860903835183	0\\
20.8853059983064	0\\
20.8845215422537	0\\
20.8837370153473	0\\
20.8829524175745	0\\
20.8821677489224	0\\
20.8813830093783	0\\
20.8805981989293	0\\
20.8798133175627	0\\
20.8790283652655	0\\
20.8782433420251	0\\
20.8774582478285	0\\
20.8766730826629	0\\
20.8758878465156	0\\
20.8751025393736	0\\
20.8743171612241	0\\
20.8735317120544	0\\
20.8727461918514	0\\
20.8719606006025	0\\
20.8711749382947	0\\
20.8703892049151	0\\
20.869603400451	0\\
20.8688175248895	0\\
20.8680315782176	0\\
20.8672455604226	0\\
20.8664594714915	0\\
20.8656733114115	0\\
20.8648870801697	0\\
20.8641007777532	0\\
20.8633144041491	0\\
20.8625279593445	0\\
20.8617414433266	0\\
20.8609548560824	0\\
20.860168197599	0\\
20.8593814678636	0\\
20.8585946668632	0\\
20.8578077945849	0\\
20.8570208510158	0\\
20.856233836143	0\\
20.8554467499535	0\\
20.8546595924344	0\\
20.8538723635729	0\\
20.8530850633559	0\\
20.8522976917705	0\\
20.8515102488038	0\\
20.8507227344429	0\\
20.8499351486747	0\\
20.8491474914864	0\\
20.848359762865	0\\
20.8475719627975	0\\
20.846784091271	0\\
20.8459961482725	0\\
20.845208133789	0\\
20.8444200478076	0\\
20.8436318903153	0\\
20.8428436612991	0\\
20.8420553607461	0\\
20.8412669886431	0\\
20.8404785449773	0\\
20.8396900297357	0\\
20.8389014429053	0\\
20.838112784473	0\\
20.8373240544259	0\\
20.8365352527509	0\\
20.835746379435	0\\
20.8349574344653	0\\
20.8341684178287	0\\
20.8333793295122	0\\
20.8325901695028	0\\
20.8318009377873	0\\
20.8310116343529	0\\
20.8302222591865	0\\
20.8294328122749	0\\
20.8286432936053	0\\
20.8278537031645	0\\
20.8270640409395	0\\
20.8262743069172	0\\
20.8254845010846	0\\
20.8246946234286	0\\
20.8239046739362	0\\
20.8231146525942	0\\
20.8223245593897	0\\
20.8215343943095	0\\
20.8207441573406	0\\
20.8199538484698	0\\
20.8191634676841	0\\
20.8183730149705	0\\
20.8175824903157	0\\
20.8167918937068	0\\
20.8160012251306	0\\
20.815210484574	0\\
20.8144196720239	0\\
20.8136287874672	0\\
20.8128378308907	0\\
20.8120468022815	0\\
20.8112557016262	0\\
20.8104645289119	0\\
20.8096732841254	0\\
20.8088819672535	0\\
20.8080905782832	0\\
20.8072991172012	0\\
20.8065075839945	0\\
20.8057159786498	0\\
20.8049243011541	0\\
20.8041325514942	0\\
20.8033407296569	0\\
20.8025488356291	0\\
20.8017568693976	0\\
20.8009648309492	0\\
20.8001727202708	0\\
20.7993805373492	0\\
20.7985882821712	0\\
20.7977959547236	0\\
20.7970035549933	0\\
20.796211082967	0\\
20.7954185386316	0\\
20.7946259219738	0\\
20.7938332329805	0\\
20.7930404716385	0\\
20.7922476379345	0\\
20.7914547318553	0\\
20.7906617533878	0\\
20.7898687025187	0\\
20.7890755792347	0\\
20.7882823835227	0\\
20.7874891153695	0\\
20.7866957747617	0\\
20.7859023616862	0\\
20.7851088761297	0\\
20.784315318079	0\\
20.7835216875208	0\\
20.7827279844419	0\\
20.781934208829	0\\
20.7811403606689	0\\
20.7803464399482	0\\
20.7795524466538	0\\
20.7787583807724	0\\
20.7779642422907	0\\
20.7771700311953	0\\
20.7763757474731	0\\
20.7755813911108	0\\
20.774786962095	0\\
20.7739924604125	0\\
20.7731978860499	0\\
20.7724032389941	0\\
20.7716085192316	0\\
20.7708137267491	0\\
20.7700188615335	0\\
20.7692239235713	0\\
20.7684289128492	0\\
20.7676338293539	0\\
20.7668386730721	0\\
20.7660434439905	0\\
20.7652481420957	0\\
20.7644527673744	0\\
20.7636573198132	0\\
20.7628617993988	0\\
20.762066206118	0\\
20.7612705399572	0\\
20.7604748009032	0\\
20.7596789889425	0\\
20.758883104062	0\\
20.7580871462481	0\\
20.7572911154875	0\\
20.7564950117668	0\\
20.7556988350727	0\\
20.7549025853918	0\\
20.7541062627107	0\\
20.753309867016	0\\
20.7525133982944	0\\
20.7517168565323	0\\
20.7509202417165	0\\
20.7501235538336	0\\
20.74932679287	0\\
20.7485299588125	0\\
20.7477330516475	0\\
20.7469360713618	0\\
20.7461390179418	0\\
20.7453418913742	0\\
20.7445446916455	0\\
20.7437474187423	0\\
20.7429500726511	0\\
20.7421526533585	0\\
20.7413551608512	0\\
20.7405575951155	0\\
20.7397599561381	0\\
20.7389622439056	0\\
20.7381644584044	0\\
20.7373665996211	0\\
20.7365686675423	0\\
20.7357706621544	0\\
20.734972583444	0\\
20.7341744313977	0\\
20.7333762060019	0\\
20.7325779072432	0\\
20.7317795351081	0\\
20.730981089583	0\\
20.7301825706546	0\\
20.7293839783092	0\\
20.7285853125335	0\\
20.7277865733138	0\\
20.7269877606368	0\\
20.7261888744888	0\\
20.7253899148563	0\\
20.7245908817259	0\\
20.723791775084	0\\
20.7229925949171	0\\
20.7221933412116	0\\
20.721394013954	0\\
20.7205946131309	0\\
20.7197951387285	0\\
20.7189955907335	0\\
20.7181959691322	0\\
20.7173962739111	0\\
20.7165965050566	0\\
20.7157966625551	0\\
20.7149967463932	0\\
20.7141967565572	0\\
20.7133966930336	0\\
20.7125965558088	0\\
20.7117963448691	0\\
20.7109960602011	0\\
20.7101957017912	0\\
20.7093952696256	0\\
20.708594763691	0\\
20.7077941839735	0\\
20.7069935304598	0\\
20.706192803136	0\\
20.7053920019887	0\\
20.7045911270042	0\\
20.703790178169	0\\
20.7029891554693	0\\
20.7021880588915	0\\
20.7013868884221	0\\
20.7005856440474	0\\
20.6997843257537	0\\
20.6989829335274	0\\
20.698181467355	0\\
20.6973799272226	0\\
20.6965783131167	0\\
20.6957766250237	0\\
20.6949748629298	0\\
20.6941730268214	0\\
20.6933711166848	0\\
20.6925691325064	0\\
20.6917670742725	0\\
20.6909649419694	0\\
20.6901627355834	0\\
20.6893604551009	0\\
20.6885581005081	0\\
20.6877556717913	0\\
20.686953168937	0\\
20.6861505919313	0\\
20.6853479407605	0\\
20.684545215411	0\\
20.6837424158691	0\\
20.6829395421209	0\\
20.6821365941529	0\\
20.6813335719512	0\\
20.6805304755022	0\\
20.6797273047921	0\\
20.6789240598072	0\\
20.6781207405338	0\\
20.677317346958	0\\
20.6765138790662	0\\
20.6757103368446	0\\
20.6749067202795	0\\
20.674103029357	0\\
20.6732992640635	0\\
20.6724954243852	0\\
20.6716915103083	0\\
20.670887521819	0\\
20.6700834589035	0\\
20.6692793215481	0\\
20.668475109739	0\\
20.6676708234624	0\\
20.6668664627045	0\\
20.6660620274515	0\\
20.6652575176896	0\\
20.664452933405	0\\
20.6636482745839	0\\
20.6628435412125	0\\
20.662038733277	0\\
20.6612338507635	0\\
20.6604288936582	0\\
20.6596238619474	0\\
20.6588187556171	0\\
20.6580135746535	0\\
20.6572083190429	0\\
20.6564029887713	0\\
20.6555975838249	0\\
20.6547921041899	0\\
20.6539865498523	0\\
20.6531809207985	0\\
20.6523752170144	0\\
20.6515694384862	0\\
20.6507635852001	0\\
20.6499576571422	0\\
20.6491516542986	0\\
20.6483455766554	0\\
20.6475394241987	0\\
20.6467331969147	0\\
20.6459268947895	0\\
20.6451205178091	0\\
20.6443140659596	0\\
20.6435075392273	0\\
20.642700937598	0\\
20.641894261058	0\\
20.6410875095934	0\\
20.6402806831901	0\\
20.6394737818343	0\\
20.6386668055121	0\\
20.6378597542095	0\\
20.6370526279126	0\\
20.6362454266074	0\\
20.63543815028	0\\
20.6346307989165	0\\
20.6338233725029	0\\
20.6330158710253	0\\
20.6322082944696	0\\
20.631400642822	0\\
20.6305929160685	0\\
20.629785114195	0\\
20.6289772371877	0\\
20.6281692850325	0\\
20.6273612577155	0\\
20.6265531552226	0\\
20.62574497754	0\\
20.6249367246535	0\\
20.6241283965492	0\\
20.6233199932132	0\\
20.6225115146313	0\\
20.6217029607896	0\\
20.6208943316741	0\\
20.6200856272707	0\\
20.6192768475655	0\\
20.6184679925444	0\\
20.6176590621933	0\\
20.6168500564983	0\\
20.6160409754454	0\\
20.6152318190203	0\\
20.6144225872092	0\\
20.613613279998	0\\
20.6128038973726	0\\
20.611994439319	0\\
20.6111849058231	0\\
20.6103752968708	0\\
20.6095656124482	0\\
20.608755852541	0\\
20.6079460171352	0\\
20.6071361062168	0\\
20.6063261197717	0\\
20.6055160577858	0\\
20.6047059202449	0\\
20.603895707135	0\\
20.603085418442	0\\
20.6022750541519	0\\
20.6014646142504	0\\
20.6006540987234	0\\
20.5998435075569	0\\
20.5990328407368	0\\
20.5982220982489	0\\
20.5974112800791	0\\
20.5966003862132	0\\
20.5957894166371	0\\
20.5949783713367	0\\
20.5941672502979	0\\
20.5933560535065	0\\
20.5925447809483	0\\
20.5917334326091	0\\
20.590922008475	0\\
20.5901105085315	0\\
20.5892989327647	0\\
20.5884872811603	0\\
20.5876755537042	0\\
20.5868637503821	0\\
20.58605187118	0\\
20.5852399160835	0\\
20.5844278850786	0\\
20.5836157781509	0\\
20.5828035952864	0\\
20.5819913364709	0\\
20.58117900169	0\\
20.5803665909296	0\\
20.5795541041756	0\\
20.5787415414136	0\\
20.5779289026294	0\\
20.5771161878089	0\\
20.5763033969377	0\\
20.5754905300018	0\\
20.5746775869867	0\\
20.5738645678783	0\\
20.5730514726624	0\\
20.5722383013246	0\\
20.5714250538508	0\\
20.5706117302266	0\\
20.5697983304378	0\\
20.5689848544702	0\\
20.5681713023094	0\\
20.5673576739412	0\\
20.5665439693514	0\\
20.5657301885256	0\\
20.5649163314495	0\\
20.5641023981089	0\\
20.5632883884894	0\\
20.5624743025768	0\\
20.5616601403568	0\\
20.560845901815	0\\
20.5600315869372	0\\
20.5592171957089	0\\
20.5584027281161	0\\
20.5575881841441	0\\
20.5567735637789	0\\
20.555958867006	0\\
20.5551440938111	0\\
20.5543292441798	0\\
20.5535143180979	0\\
20.5526993155509	0\\
20.5518842365245	0\\
20.5510690810044	0\\
20.5502538489762	0\\
20.5494385404256	0\\
20.5486231553381	0\\
20.5478076936994	0\\
20.5469921554951	0\\
20.5461765407109	0\\
20.5453608493324	0\\
20.5445450813451	0\\
20.5437292367347	0\\
20.5429133154868	0\\
20.542097317587	0\\
20.5412812430209	0\\
20.5404650917741	0\\
20.5396488638321	0\\
20.5388325591806	0\\
20.5380161778051	0\\
20.5371997196912	0\\
20.5363831848245	0\\
20.5355665731906	0\\
20.534749884775	0\\
20.5339331195632	0\\
20.5331162775409	0\\
20.5322993586935	0\\
20.5314823630067	0\\
20.5306652904659	0\\
20.5298481410568	0\\
20.5290309147648	0\\
20.5282136115755	0\\
20.5273962314744	0\\
20.526578774447	0\\
20.5257612404788	0\\
20.5249436295555	0\\
20.5241259416624	0\\
20.523308176785	0\\
20.522490334909	0\\
20.5216724160197	0\\
20.5208544201028	0\\
20.5200363471435	0\\
20.5192181971276	0\\
20.5183999700403	0\\
20.5175816658673	0\\
20.516763284594	0\\
20.5159448262058	0\\
20.5151262906883	0\\
20.5143076780268	0\\
20.5134889882069	0\\
20.512670221214	0\\
20.5118513770335	0\\
20.5110324556509	0\\
20.5102134570516	0\\
20.5093943812211	0\\
20.5085752281448	0\\
20.5077559978082	0\\
20.5069366901965	0\\
20.5061173052954	0\\
20.5052978430902	0\\
20.5044783035662	0\\
20.503658686709	0\\
20.5028389925039	0\\
20.5020192209363	0\\
20.5011993719916	0\\
20.5003794456552	0\\
20.4995594419125	0\\
20.4987393607489	0\\
20.4979192021497	0\\
20.4970989661004	0\\
20.4962786525862	0\\
20.4954582615926	0\\
20.494637793105	0\\
20.4938172471086	0\\
20.4929966235889	0\\
20.4921759225312	0\\
20.4913551439208	0\\
20.490534287743	0\\
20.4897133539833	0\\
20.488892342627	0\\
20.4880712536593	0\\
20.4872500870656	0\\
20.4864288428313	0\\
20.4856075209416	0\\
20.4847861213819	0\\
20.4839646441374	0\\
20.4831430891935	0\\
20.4823214565354	0\\
20.4814997461485	0\\
20.4806779580181	0\\
20.4798560921294	0\\
20.4790341484678	0\\
20.4782121270184	0\\
20.4773900277666	0\\
20.4765678506977	0\\
20.4757455957969	0\\
20.4749232630495	0\\
20.4741008524407	0\\
20.4732783639559	0\\
20.4724557975801	0\\
20.4716331532988	0\\
20.4708104310971	0\\
20.4699876309602	0\\
20.4691647528735	0\\
20.4683417968221	0\\
20.4675187627912	0\\
20.4666956507662	0\\
20.4658724607321	0\\
20.4650491926742	0\\
20.4642258465778	0\\
20.4634024224279	0\\
20.4625789202099	0\\
20.4617553399089	0\\
20.4609316815101	0\\
20.4601079449987	0\\
20.4592841303599	0\\
20.4584602375788	0\\
20.4576362666406	0\\
20.4568122175306	0\\
20.4559880902338	0\\
20.4551638847354	0\\
20.4543396010206	0\\
20.4535152390745	0\\
20.4526907988824	0\\
20.4518662804292	0\\
20.4510416837003	0\\
20.4502170086806	0\\
20.4493922553554	0\\
20.4485674237097	0\\
20.4477425137288	0\\
20.4469175253976	0\\
20.4460924587014	0\\
20.4452673136252	0\\
20.4444420901541	0\\
20.4436167882733	0\\
20.4427914079678	0\\
20.4419659492228	0\\
20.4411404120233	0\\
20.4403147963544	0\\
20.4394891022012	0\\
20.4386633295488	0\\
20.4378374783822	0\\
20.4370115486865	0\\
20.4361855404468	0\\
20.4353594536481	0\\
20.4345332882755	0\\
20.4337070443141	0\\
20.4328807217489	0\\
20.4320543205648	0\\
20.4312278407471	0\\
20.4304012822807	0\\
20.4295746451505	0\\
20.4287479293418	0\\
20.4279211348394	0\\
20.4270942616285	0\\
20.4262673096939	0\\
20.4254402790208	0\\
20.424613169594	0\\
20.4237859813988	0\\
20.4229587144199	0\\
20.4221313686424	0\\
20.4213039440514	0\\
20.4204764406317	0\\
20.4196488583684	0\\
20.4188211972464	0\\
20.4179934572507	0\\
20.4171656383663	0\\
20.4163377405781	0\\
20.4155097638712	0\\
20.4146817082303	0\\
20.4138535736406	0\\
20.4130253600868	0\\
20.4121970675541	0\\
20.4113686960273	0\\
20.4105402454913	0\\
20.409711715931	0\\
20.4088831073315	0\\
20.4080544196776	0\\
20.4072256529542	0\\
20.4063968071462	0\\
20.4055678822385	0\\
20.4047388782162	0\\
20.4039097950639	0\\
20.4030806327667	0\\
20.4022513913093	0\\
20.4014220706768	0\\
20.400592670854	0\\
20.3997631918257	0\\
20.3989336335768	0\\
20.3981039960923	0\\
20.3972742793569	0\\
20.3964444833555	0\\
20.3956146080729	0\\
20.3947846534941	0\\
20.3939546196038	0\\
20.393124506387	0\\
20.3922943138283	0\\
20.3914640419128	0\\
20.3906336906251	0\\
20.3898032599502	0\\
20.3889727498727	0\\
20.3881421603777	0\\
20.3873114914498	0\\
20.3864807430738	0\\
20.3856499152346	0\\
20.384819007917	0\\
20.3839880211058	0\\
20.3831569547856	0\\
20.3823258089415	0\\
20.381494583558	0\\
20.38066327862	0\\
20.3798318941122	0\\
20.3790004300195	0\\
20.3781688863265	0\\
20.3773372630181	0\\
20.376505560079	0\\
20.3756737774939	0\\
20.3748419152476	0\\
20.3740099733249	0\\
20.3731779517104	0\\
20.3723458503888	0\\
20.371513669345	0\\
20.3706814085637	0\\
20.3698490680295	0\\
20.3690166477271	0\\
20.3681841476414	0\\
20.3673515677569	0\\
20.3665189080584	0\\
20.3656861685306	0\\
20.3648533491582	0\\
20.3640204499258	0\\
20.3631874708182	0\\
20.36235441182	0\\
20.3615212729159	0\\
20.3606880540906	0\\
20.3598547553287	0\\
20.3590213766148	0\\
20.3581879179338	0\\
20.3573543792701	0\\
20.3565207606085	0\\
20.3556870619336	0\\
20.35485328323	0\\
20.3540194244823	0\\
20.3531854856753	0\\
20.3523514667935	0\\
20.3515173678215	0\\
20.350683188744	0\\
20.3498489295455	0\\
20.3490145902107	0\\
20.3481801707242	0\\
20.3473456710706	0\\
20.3465110912344	0\\
20.3456764312004	0\\
20.3448416909529	0\\
20.3440068704767	0\\
20.3431719697563	0\\
20.3423369887763	0\\
20.3415019275212	0\\
20.3406667859756	0\\
20.3398315641241	0\\
20.3389962619512	0\\
20.3381608794415	0\\
20.3373254165795	0\\
20.3364898733498	0\\
20.3356542497368	0\\
20.3348185457252	0\\
20.3339827612994	0\\
20.333146896444	0\\
20.3323109511435	0\\
20.3314749253825	0\\
20.3306388191453	0\\
20.3298026324166	0\\
20.3289663651808	0\\
20.3281300174224	0\\
20.3272935891259	0\\
20.3264570802758	0\\
20.3256204908566	0\\
20.3247838208528	0\\
20.3239470702488	0\\
20.3231102390291	0\\
20.3222733271782	0\\
20.3214363346805	0\\
20.3205992615205	0\\
20.3197621076826	0\\
20.3189248731513	0\\
20.3180875579111	0\\
20.3172501619463	0\\
20.3164126852414	0\\
20.3155751277808	0\\
20.314737489549	0\\
20.3138997705304	0\\
20.3130619707093	0\\
20.3122240900702	0\\
20.3113861285976	0\\
20.3105480862757	0\\
20.3097099630891	0\\
20.308871759022	0\\
20.308033474059	0\\
20.3071951081843	0\\
20.3063566613823	0\\
20.3055181336375	0\\
20.3046795249342	0\\
20.3038408352567	0\\
20.3030020645894	0\\
20.3021632129167	0\\
20.301324280223	0\\
20.3004852664925	0\\
20.2996461717096	0\\
20.2988069958587	0\\
20.2979677389241	0\\
20.2971284008901	0\\
20.2962889817411	0\\
20.2954494814613	0\\
20.2946099000351	0\\
20.2937702374468	0\\
20.2929304936806	0\\
20.292090668721	0\\
20.2912507625522	0\\
20.2904107751585	0\\
20.2895707065241	0\\
20.2887305566334	0\\
20.2878903254707	0\\
20.2870500130202	0\\
20.2862096192661	0\\
20.2853691441929	0\\
20.2845285877846	0\\
20.2836879500256	0\\
20.2828472309001	0\\
20.2820064303924	0\\
20.2811655484867	0\\
20.2803245851673	0\\
20.2794835404183	0\\
20.2786424142241	0\\
20.2778012065688	0\\
20.2769599174366	0\\
20.2761185468119	0\\
20.2752770946787	0\\
20.2744355610213	0\\
20.2735939458239	0\\
20.2727522490706	0\\
20.2719104707458	0\\
20.2710686108335	0\\
20.270226669318	0\\
20.2693846461834	0\\
20.2685425414139	0\\
20.2677003549937	0\\
20.2668580869069	0\\
20.2660157371378	0\\
20.2651733056703	0\\
20.2643307924888	0\\
20.2634881975773	0\\
20.2626455209201	0\\
20.2618027625011	0\\
20.2609599223046	0\\
20.2601170003147	0\\
20.2592739965156	0\\
20.2584309108912	0\\
20.2575877434258	0\\
20.2567444941034	0\\
20.2559011629083	0\\
20.2550577498243	0\\
20.2542142548357	0\\
20.2533706779266	0\\
20.252527019081	0\\
20.251683278283	0\\
20.2508394555167	0\\
20.2499955507662	0\\
20.2491515640155	0\\
20.2483074952487	0\\
20.2474633444498	0\\
20.2466191116029	0\\
20.2457747966921	0\\
20.2449303997014	0\\
20.2440859206149	0\\
20.2432413594165	0\\
20.2423967160903	0\\
20.2415519906203	0\\
20.2407071829906	0\\
20.2398622931851	0\\
20.239017321188	0\\
20.2381722669831	0\\
20.2373271305545	0\\
20.2364819118862	0\\
20.2356366109621	0\\
20.2347912277664	0\\
20.2339457622828	0\\
20.2331002144956	0\\
20.2322545843885	0\\
20.2314088719456	0\\
20.2305630771508	0\\
20.2297171999881	0\\
20.2288712404415	0\\
20.2280251984949	0\\
20.2271790741322	0\\
20.2263328673374	0\\
20.2254865780945	0\\
20.2246402063873	0\\
20.2237937521998	0\\
20.222947215516	0\\
20.2221005963196	0\\
20.2212538945947	0\\
20.2204071103251	0\\
20.2195602434948	0\\
20.2187132940877	0\\
20.2178662620876	0\\
20.2170191474785	0\\
20.2161719502441	0\\
20.2153246703685	0\\
20.2144773078355	0\\
20.2136298626289	0\\
20.2127823347326	0\\
20.2119347241305	0\\
20.2110870308064	0\\
20.2102392547443	0\\
20.2093913959278	0\\
20.208543454341	0\\
20.2076954299675	0\\
20.2068473227914	0\\
20.2059991327962	0\\
20.205150859966	0\\
20.2043025042845	0\\
20.2034540657356	0\\
20.202605544303	0\\
20.2017569399705	0\\
20.200908252722	0\\
20.2000594825412	0\\
20.199210629412	0\\
20.198361693318	0\\
20.1975126742432	0\\
20.1966635721713	0\\
20.195814387086	0\\
20.1949651189711	0\\
20.1941157678104	0\\
20.1932663335876	0\\
20.1924168162865	0\\
20.1915672158908	0\\
20.1907175323844	0\\
20.1898677657508	0\\
20.1890179159739	0\\
20.1881679830373	0\\
20.1873179669249	0\\
20.1864678676203	0\\
20.1856176851072	0\\
20.1847674193693	0\\
20.1839170703904	0\\
20.1830666381542	0\\
20.1822161226443	0\\
20.1813655238444	0\\
20.1805148417382	0\\
20.1796640763094	0\\
20.1788132275417	0\\
20.1779622954188	0\\
20.1771112799243	0\\
20.1762601810418	0\\
20.1754089987551	0\\
20.1745577330477	0\\
20.1737063839034	0\\
20.1728549513057	0\\
20.1720034352384	0\\
20.1711518356849	0\\
20.1703001526291	0\\
20.1694483860544	0\\
20.1685965359446	0\\
20.1677446022832	0\\
20.1668925850538	0\\
20.16604048424	0\\
20.1651882998255	0\\
20.1643360317938	0\\
20.1634836801285	0\\
20.1626312448132	0\\
20.1617787258315	0\\
20.160926123167	0\\
20.1600734368032	0\\
20.1592206667237	0\\
20.158367812912	0\\
20.1575148753517	0\\
20.1566618540264	0\\
20.1558087489196	0\\
20.1549555600148	0\\
20.1541022872956	0\\
20.1532489307455	0\\
20.1523954903481	0\\
20.1515419660868	0\\
20.1506883579452	0\\
20.1498346659067	0\\
20.148980889955	0\\
20.1481270300734	0\\
20.1472730862456	0\\
20.1464190584549	0\\
20.1455649466849	0\\
20.144710750919	0\\
20.1438564711408	0\\
20.1430021073337	0\\
20.1421476594811	0\\
20.1412931275666	0\\
20.1404385115736	0\\
20.1395838114855	0\\
20.1387290272859	0\\
20.137874158958	0\\
20.1370192064854	0\\
20.1361641698516	0\\
20.1353090490399	0\\
20.1344538440337	0\\
20.1335985548165	0\\
20.1327431813717	0\\
20.1318877236827	0\\
20.131032181733	0\\
20.1301765555058	0\\
20.1293208449846	0\\
20.1284650501529	0\\
20.1276091709939	0\\
20.126753207491	0\\
20.1258971596277	0\\
20.1250410273873	0\\
20.1241848107531	0\\
20.1233285097086	0\\
20.1224721242371	0\\
20.1216156543219	0\\
20.1207590999463	0\\
20.1199024610938	0\\
20.1190457377477	0\\
20.1181889298912	0\\
20.1173320375077	0\\
20.1164750605805	0\\
20.115617999093	0\\
20.1147608530285	0\\
20.1139036223702	0\\
20.1130463071015	0\\
20.1121889072056	0\\
20.1113314226659	0\\
20.1104738534656	0\\
20.109616199588	0\\
20.1087584610164	0\\
20.1079006377341	0\\
20.1070427297243	0\\
20.1061847369703	0\\
20.1053266594554	0\\
20.1044684971627	0\\
20.1036102500756	0\\
20.1027519181773	0\\
20.101893501451	0\\
20.1010349998799	0\\
20.1001764134473	0\\
20.0993177421365	0\\
20.0984589859305	0\\
20.0976001448127	0\\
20.0967412187662	0\\
20.0958822077742	0\\
20.09502311182	0\\
20.0941639308867	0\\
20.0933046649575	0\\
20.0924453140157	0\\
20.0915858780442	0\\
20.0907263570265	0\\
20.0898667509455	0\\
20.0890070597845	0\\
20.0881472835266	0\\
20.087287422155	0\\
20.0864274756529	0\\
20.0855674440032	0\\
20.0847073271893	0\\
20.0838471251943	0\\
20.0829868380012	0\\
20.0821264655932	0\\
20.0812660079533	0\\
20.0804054650648	0\\
20.0795448369107	0\\
20.0786841234742	0\\
20.0778233247382	0\\
20.076962440686	0\\
20.0761014713005	0\\
20.0752404165649	0\\
20.0743792764623	0\\
20.0735180509757	0\\
20.0726567400882	0\\
20.0717953437828	0\\
20.0709338620426	0\\
20.0700722948507	0\\
20.0692106421901	0\\
20.0683489040439	0\\
20.067487080395	0\\
20.0666251712265	0\\
20.0657631765215	0\\
20.0649010962629	0\\
20.0640389304338	0\\
20.0631766790172	0\\
20.0623143419961	0\\
20.0614519193535	0\\
20.0605894110724	0\\
20.0597268171357	0\\
20.0588641375265	0\\
20.0580013722277	0\\
20.0571385212224	0\\
20.0562755844935	0\\
20.0554125620239	0\\
20.0545494537966	0\\
20.0536862597946	0\\
20.0528229800008	0\\
20.0519596143982	0\\
20.0510961629696	0\\
20.0502326256982	0\\
20.0493690025666	0\\
20.048505293558	0\\
20.0476414986552	0\\
20.0467776178411	0\\
20.0459136510987	0\\
20.0450495984108	0\\
20.0441854597603	0\\
20.0433212351302	0\\
20.0424569245032	0\\
20.0415925278624	0\\
20.0407280451906	0\\
20.0398634764706	0\\
20.0389988216854	0\\
20.0381340808177	0\\
20.0372692538505	0\\
20.0364043407666	0\\
20.0355393415488	0\\
20.03467425618	0\\
20.0338090846431	0\\
20.0329438269208	0\\
20.032078482996	0\\
20.0312130528515	0\\
20.0303475364701	0\\
20.0294819338347	0\\
20.028616244928	0\\
20.0277504697328	0\\
20.026884608232	0\\
20.0260186604084	0\\
20.0251526262446	0\\
20.0242865057236	0\\
20.023420298828	0\\
20.0225540055407	0\\
20.0216876258444	0\\
20.0208211597219	0\\
20.0199546071559	0\\
20.0190879681292	0\\
20.0182212426245	0\\
20.0173544306245	0\\
20.0164875321121	0\\
20.0156205470699	0\\
20.0147534754807	0\\
20.0138863173271	0\\
20.0130190725919	0\\
20.0121517412578	0\\
20.0112843233075	0\\
20.0104168187237	0\\
20.0095492274891	0\\
20.0086815495864	0\\
20.0078137849982	0\\
20.0069459337073	0\\
20.0060779956962	0\\
20.0052099709478	0\\
20.0043418594446	0\\
20.0034736611693	0\\
20.0026053761045	0\\
20.001737004233	0\\
20.0008685455373	0\\
20	0\\
};
\addplot [color=mycolor1, forget plot]
  table[row sep=crcr]{%
20	0\\
19.9991313676039	0\\
19.9982626483314	0\\
19.9973938421654	0\\
19.9965249490882	0\\
19.9956559690827	0\\
19.9947869021314	0\\
19.9939177482168	0\\
19.9930485073216	0\\
19.9921791794283	0\\
19.9913097645196	0\\
19.9904402625781	0\\
19.9895706735862	0\\
19.9887009975266	0\\
19.9878312343818	0\\
19.9869613841344	0\\
19.986091446767	0\\
19.985221422262	0\\
19.9843513106021	0\\
19.9834811117697	0\\
19.9826108257474	0\\
19.9817404525178	0\\
19.9808699920633	0\\
19.9799994443664	0\\
19.9791288094097	0\\
19.9782580871757	0\\
19.9773872776469	0\\
19.9765163808057	0\\
19.9756453966347	0\\
19.9747743251164	0\\
19.9739031662331	0\\
19.9730319199675	0\\
19.9721605863019	0\\
19.9712891652188	0\\
19.9704176567008	0\\
19.9695460607301	0\\
19.9686743772894	0\\
19.967802606361	0\\
19.9669307479273	0\\
19.9660588019708	0\\
19.965186768474	0\\
19.9643146474192	0\\
19.9634424387888	0\\
19.9625701425653	0\\
19.9616977587311	0\\
19.9608252872686	0\\
19.9599527281601	0\\
19.9590800813881	0\\
19.9582073469349	0\\
19.9573345247829	0\\
19.9564616149145	0\\
19.9555886173121	0\\
19.954715531958	0\\
19.9538423588345	0\\
19.9529690979241	0\\
19.9520957492091	0\\
19.9512223126718	0\\
19.9503487882945	0\\
19.9494751760596	0\\
19.9486014759494	0\\
19.9477276879463	0\\
19.9468538120324	0\\
19.9459798481903	0\\
19.9451057964021	0\\
19.9442316566501	0\\
19.9433574289167	0\\
19.9424831131841	0\\
19.9416087094346	0\\
19.9407342176505	0\\
19.9398596378141	0\\
19.9389849699076	0\\
19.9381102139133	0\\
19.9372353698135	0\\
19.9363604375903	0\\
19.9354854172261	0\\
19.9346103087031	0\\
19.9337351120034	0\\
19.9328598271094	0\\
19.9319844540033	0\\
19.9311089926673	0\\
19.9302334430836	0\\
19.9293578052343	0\\
19.9284820791018	0\\
19.9276062646682	0\\
19.9267303619157	0\\
19.9258543708264	0\\
19.9249782913826	0\\
19.9241021235665	0\\
19.9232258673601	0\\
19.9223495227458	0\\
19.9214730897055	0\\
19.9205965682215	0\\
19.919719958276	0\\
19.918843259851	0\\
19.9179664729287	0\\
19.9170895974913	0\\
19.9162126335209	0\\
19.9153355809995	0\\
19.9144584399093	0\\
19.9135812102324	0\\
19.912703891951	0\\
19.9118264850471	0\\
19.9109489895027	0\\
19.9100714053001	0\\
19.9091937324213	0\\
19.9083159708483	0\\
19.9074381205632	0\\
19.9065601815482	0\\
19.9056821537852	0\\
19.9048040372563	0\\
19.9039258319436	0\\
19.9030475378291	0\\
19.9021691548948	0\\
19.9012906831228	0\\
19.9004121224952	0\\
19.8995334729938	0\\
19.8986547346008	0\\
19.8977759072982	0\\
19.896896991068	0\\
19.8960179858921	0\\
19.8951388917526	0\\
19.8942597086314	0\\
19.8933804365106	0\\
19.8925010753721	0\\
19.8916216251979	0\\
19.89074208597	0\\
19.8898624576702	0\\
19.8889827402807	0\\
19.8881029337833	0\\
19.88722303816	0\\
19.8863430533927	0\\
19.8854629794634	0\\
19.884582816354	0\\
19.8837025640463	0\\
19.8828222225224	0\\
19.8819417917642	0\\
19.8810612717535	0\\
19.8801806624723	0\\
19.8792999639024	0\\
19.8784191760257	0\\
19.8775382988242	0\\
19.8766573322797	0\\
19.8757762763741	0\\
19.8748951310892	0\\
19.874013896407	0\\
19.8731325723092	0\\
19.8722511587778	0\\
19.8713696557945	0\\
19.8704880633413	0\\
19.8696063813999	0\\
19.8687246099522	0\\
19.8678427489801	0\\
19.8669607984652	0\\
19.8660787583895	0\\
19.8651966287347	0\\
19.8643144094827	0\\
19.8634321006153	0\\
19.8625497021142	0\\
19.8616672139612	0\\
19.8607846361382	0\\
19.8599019686268	0\\
19.8590192114089	0\\
19.8581363644663	0\\
19.8572534277806	0\\
19.8563704013336	0\\
19.8554872851071	0\\
19.8546040790829	0\\
19.8537207832426	0\\
19.852837397568	0\\
19.8519539220408	0\\
19.8510703566427	0\\
19.8501867013555	0\\
19.8493029561609	0\\
19.8484191210405	0\\
19.8475351959761	0\\
19.8466511809493	0\\
19.8457670759419	0\\
19.8448828809354	0\\
19.8439985959117	0\\
19.8431142208523	0\\
19.842229755739	0\\
19.8413452005533	0\\
19.840460555277	0\\
19.8395758198917	0\\
19.838690994379	0\\
19.8378060787205	0\\
19.8369210728979	0\\
19.8360359768929	0\\
19.835150790687	0\\
19.8342655142618	0\\
19.833380147599	0\\
19.8324946906801	0\\
19.8316091434868	0\\
19.8307235060006	0\\
19.8298377782032	0\\
19.8289519600761	0\\
19.8280660516008	0\\
19.827180052759	0\\
19.8262939635322	0\\
19.825407783902	0\\
19.8245215138499	0\\
19.8236351533575	0\\
19.8227487024062	0\\
19.8218621609777	0\\
19.8209755290535	0\\
19.8200888066151	0\\
19.819201993644	0\\
19.8183150901217	0\\
19.8174280960297	0\\
19.8165410113496	0\\
19.8156538360628	0\\
19.8147665701507	0\\
19.813879213595	0\\
19.8129917663771	0\\
19.8121042284784	0\\
19.8112165998804	0\\
19.8103288805646	0\\
19.8094410705124	0\\
19.8085531697053	0\\
19.8076651781247	0\\
19.806777095752	0\\
19.8058889225688	0\\
19.8050006585563	0\\
19.8041123036961	0\\
19.8032238579696	0\\
19.8023353213581	0\\
19.8014466938431	0\\
19.8005579754059	0\\
19.799669166028	0\\
19.7987802656907	0\\
19.7978912743755	0\\
19.7970021920636	0\\
19.7961130187365	0\\
19.7952237543756	0\\
19.7943343989621	0\\
19.7934449524775	0\\
19.792555414903	0\\
19.7916657862201	0\\
19.79077606641	0\\
19.7898862554542	0\\
19.7889963533338	0\\
19.7881063600303	0\\
19.787216275525	0\\
19.786326099799	0\\
19.7854358328339	0\\
19.7845454746107	0\\
19.783655025111	0\\
19.7827644843158	0\\
19.7818738522065	0\\
19.7809831287644	0\\
19.7800923139707	0\\
19.7792014078068	0\\
19.7783104102537	0\\
19.7774193212929	0\\
19.7765281409055	0\\
19.7756368690728	0\\
19.774745505776	0\\
19.7738540509963	0\\
19.772962504715	0\\
19.7720708669133	0\\
19.7711791375723	0\\
19.7702873166733	0\\
19.7693954041975	0\\
19.7685034001261	0\\
19.7676113044402	0\\
19.766719117121	0\\
19.7658268381498	0\\
19.7649344675076	0\\
19.7640420051756	0\\
19.7631494511351	0\\
19.762256805367	0\\
19.7613640678527	0\\
19.7604712385732	0\\
19.7595783175096	0\\
19.7586853046432	0\\
19.7577921999549	0\\
19.756899003426	0\\
19.7560057150374	0\\
19.7551123347705	0\\
19.7542188626061	0\\
19.7533252985255	0\\
19.7524316425097	0\\
19.7515378945397	0\\
19.7506440545968	0\\
19.7497501226619	0\\
19.748856098716	0\\
19.7479619827403	0\\
19.7470677747159	0\\
19.7461734746236	0\\
19.7452790824447	0\\
19.7443845981601	0\\
19.7434900217508	0\\
19.7425953531979	0\\
19.7417005924824	0\\
19.7408057395853	0\\
19.7399107944876	0\\
19.7390157571703	0\\
19.7381206276144	0\\
19.7372254058009	0\\
19.7363300917107	0\\
19.7354346853249	0\\
19.7345391866244	0\\
19.7336435955901	0\\
19.7327479122031	0\\
19.7318521364442	0\\
19.7309562682945	0\\
19.7300603077348	0\\
19.7291642547461	0\\
19.7282681093094	0\\
19.7273718714055	0\\
19.7264755410153	0\\
19.7255791181198	0\\
19.7246826026999	0\\
19.7237859947365	0\\
19.7228892942104	0\\
19.7219925011026	0\\
19.7210956153939	0\\
19.7201986370652	0\\
19.7193015660974	0\\
19.7184044024714	0\\
19.7175071461679	0\\
19.7166097971678	0\\
19.7157123554521	0\\
19.7148148210015	0\\
19.7139171937968	0\\
19.713019473819	0\\
19.7121216610487	0\\
19.7112237554669	0\\
19.7103257570543	0\\
19.7094276657918	0\\
19.70852948166	0\\
19.7076312046399	0\\
19.7067328347123	0\\
19.7058343718578	0\\
19.7049358160572	0\\
19.7040371672914	0\\
19.7031384255411	0\\
19.7022395907871	0\\
19.70134066301	0\\
19.7004416421907	0\\
19.6995425283099	0\\
19.6986433213482	0\\
19.6977440212866	0\\
19.6968446281055	0\\
19.6959451417859	0\\
19.6950455623083	0\\
19.6941458896535	0\\
19.6932461238021	0\\
19.692346264735	0\\
19.6914463124326	0\\
19.6905462668759	0\\
19.6896461280453	0\\
19.6887458959215	0\\
19.6878455704853	0\\
19.6869451517173	0\\
19.6860446395981	0\\
19.6851440341083	0\\
19.6842433352287	0\\
19.6833425429398	0\\
19.6824416572222	0\\
19.6815406780566	0\\
19.6806396054236	0\\
19.6797384393038	0\\
19.6788371796778	0\\
19.6779358265261	0\\
19.6770343798294	0\\
19.6761328395683	0\\
19.6752312057233	0\\
19.674329478275	0\\
19.6734276572039	0\\
19.6725257424907	0\\
19.6716237341158	0\\
19.6707216320598	0\\
19.6698194363033	0\\
19.6689171468268	0\\
19.6680147636108	0\\
19.6671122866358	0\\
19.6662097158823	0\\
19.6653070513309	0\\
19.6644042929621	0\\
19.6635014407563	0\\
19.662598494694	0\\
19.6616954547558	0\\
19.660792320922	0\\
19.6598890931733	0\\
19.65898577149	0\\
19.6580823558526	0\\
19.6571788462415	0\\
19.6562752426372	0\\
19.6553715450202	0\\
19.6544677533709	0\\
19.6535638676697	0\\
19.652659887897	0\\
19.6517558140333	0\\
19.6508516460589	0\\
19.6499473839542	0\\
19.6490430276998	0\\
19.6481385772759	0\\
19.6472340326629	0\\
19.6463293938413	0\\
19.6454246607914	0\\
19.6445198334935	0\\
19.6436149119281	0\\
19.6427098960754	0\\
19.6418047859159	0\\
19.6408995814298	0\\
19.6399942825976	0\\
19.6390888893995	0\\
19.6381834018158	0\\
19.637277819827	0\\
19.6363721434133	0\\
19.6354663725549	0\\
19.6345605072323	0\\
19.6336545474257	0\\
19.6327484931154	0\\
19.6318423442816	0\\
19.6309361009047	0\\
19.6300297629649	0\\
19.6291233304425	0\\
19.6282168033178	0\\
19.6273101815709	0\\
19.6264034651821	0\\
19.6254966541318	0\\
19.6245897484	0\\
19.6236827479671	0\\
19.6227756528132	0\\
19.6218684629186	0\\
19.6209611782635	0\\
19.620053798828	0\\
19.6191463245924	0\\
19.6182387555369	0\\
19.6173310916416	0\\
19.6164233328867	0\\
19.6155154792524	0\\
19.6146075307189	0\\
19.6136994872663	0\\
19.6127913488747	0\\
19.6118831155244	0\\
19.6109747871954	0\\
19.6100663638679	0\\
19.609157845522	0\\
19.6082492321378	0\\
19.6073405236955	0\\
19.6064317201751	0\\
19.6055228215568	0\\
19.6046138278206	0\\
19.6037047389467	0\\
19.6027955549151	0\\
19.6018862757059	0\\
19.6009769012991	0\\
19.6000674316749	0\\
19.5991578668133	0\\
19.5982482066943	0\\
19.5973384512981	0\\
19.5964286006045	0\\
19.5955186545937	0\\
19.5946086132457	0\\
19.5936984765405	0\\
19.5927882444581	0\\
19.5918779169786	0\\
19.5909674940819	0\\
19.590056975748	0\\
19.589146361957	0\\
19.5882356526887	0\\
19.5873248479232	0\\
19.5864139476405	0\\
19.5855029518205	0\\
19.5845918604431	0\\
19.5836806734884	0\\
19.5827693909362	0\\
19.5818580127665	0\\
19.5809465389593	0\\
19.5800349694944	0\\
19.5791233043518	0\\
19.5782115435114	0\\
19.5772996869531	0\\
19.5763877346568	0\\
19.5754756866024	0\\
19.5745635427698	0\\
19.5736513031389	0\\
19.5727389676895	0\\
19.5718265364015	0\\
19.5709140092548	0\\
19.5700013862292	0\\
19.5690886673046	0\\
19.5681758524608	0\\
19.5672629416777	0\\
19.566349934935	0\\
19.5654368322127	0\\
19.5645236334905	0\\
19.5636103387483	0\\
19.5626969479658	0\\
19.5617834611229	0\\
19.5608698781993	0\\
19.5599561991748	0\\
19.5590424240292	0\\
19.5581285527424	0\\
19.5572145852939	0\\
19.5563005216637	0\\
19.5553863618315	0\\
19.554472105777	0\\
19.5535577534799	0\\
19.55264330492	0\\
19.551728760077	0\\
19.5508141189307	0\\
19.5498993814607	0\\
19.5489845476468	0\\
19.5480696174687	0\\
19.547154590906	0\\
19.5462394679385	0\\
19.5453242485459	0\\
19.5444089327077	0\\
19.5434935204037	0\\
19.5425780116136	0\\
19.5416624063171	0\\
19.5407467044936	0\\
19.539830906123	0\\
19.5389150111848	0\\
19.5379990196588	0\\
19.5370829315244	0\\
19.5361667467613	0\\
19.5352504653492	0\\
19.5343340872676	0\\
19.5334176124962	0\\
19.5325010410145	0\\
19.5315843728022	0\\
19.5306676078388	0\\
19.5297507461038	0\\
19.5288337875769	0\\
19.5279167322376	0\\
19.5269995800655	0\\
19.5260823310401	0\\
19.5251649851409	0\\
19.5242475423475	0\\
19.5233300026395	0\\
19.5224123659963	0\\
19.5214946323975	0\\
19.5205768018225	0\\
19.5196588742509	0\\
19.5187408496622	0\\
19.5178227280359	0\\
19.5169045093513	0\\
19.5159861935881	0\\
19.5150677807258	0\\
19.5141492707436	0\\
19.5132306636212	0\\
19.5123119593379	0\\
19.5113931578732	0\\
19.5104742592066	0\\
19.5095552633175	0\\
19.5086361701853	0\\
19.5077169797893	0\\
19.5067976921092	0\\
19.5058783071241	0\\
19.5049588248136	0\\
19.504039245157	0\\
19.5031195681338	0\\
19.5021997937232	0\\
19.5012799219047	0\\
19.5003599526577	0\\
19.4994398859614	0\\
19.4985197217952	0\\
19.4975994601386	0\\
19.4966791009708	0\\
19.4957586442712	0\\
19.494838090019	0\\
19.4939174381937	0\\
19.4929966887744	0\\
19.4920758417407	0\\
19.4911548970716	0\\
19.4902338547466	0\\
19.4893127147448	0\\
19.4883914770457	0\\
19.4874701416285	0\\
19.4865487084723	0\\
19.4856271775566	0\\
19.4847055488605	0\\
19.4837838223633	0\\
19.4828619980443	0\\
19.4819400758826	0\\
19.4810180558575	0\\
19.4800959379483	0\\
19.4791737221341	0\\
19.4782514083941	0\\
19.4773289967076	0\\
19.4764064870537	0\\
19.4754838794116	0\\
19.4745611737605	0\\
19.4736383700797	0\\
19.4727154683481	0\\
19.4717924685451	0\\
19.4708693706498	0\\
19.4699461746413	0\\
19.4690228804987	0\\
19.4680994882012	0\\
19.467175997728	0\\
19.466252409058	0\\
19.4653287221706	0\\
19.4644049370447	0\\
19.4634810536594	0\\
19.462557071994	0\\
19.4616329920274	0\\
19.4607088137387	0\\
19.459784537107	0\\
19.4588601621114	0\\
19.4579356887309	0\\
19.4570111169446	0\\
19.4560864467316	0\\
19.4551616780708	0\\
19.4542368109413	0\\
19.4533118453222	0\\
19.4523867811925	0\\
19.4514616185311	0\\
19.4505363573171	0\\
19.4496109975295	0\\
19.4486855391473	0\\
19.4477599821495	0\\
19.446834326515	0\\
19.4459085722228	0\\
19.4449827192519	0\\
19.4440567675813	0\\
19.4431307171899	0\\
19.4422045680566	0\\
19.4412783201604	0\\
19.4403519734803	0\\
19.4394255279951	0\\
19.4384989836838	0\\
19.4375723405253	0\\
19.4366455984985	0\\
19.4357187575823	0\\
19.4347918177556	0\\
19.4338647789972	0\\
19.4329376412861	0\\
19.4320104046012	0\\
19.4310830689212	0\\
19.4301556342251	0\\
19.4292281004917	0\\
19.4283004676999	0\\
19.4273727358285	0\\
19.4264449048563	0\\
19.4255169747621	0\\
19.4245889455248	0\\
19.4236608171232	0\\
19.4227325895361	0\\
19.4218042627423	0\\
19.4208758367206	0\\
19.4199473114497	0\\
19.4190186869085	0\\
19.4180899630757	0\\
19.41716113993	0\\
19.4162322174504	0\\
19.4153031956154	0\\
19.4143740744038	0\\
19.4134448537944	0\\
19.4125155337659	0\\
19.411586114297	0\\
19.4106565953665	0\\
19.409726976953	0\\
19.4087972590353	0\\
19.407867441592	0\\
19.4069375246019	0\\
19.4060075080436	0\\
19.4050773918957	0\\
19.4041471761371	0\\
19.4032168607463	0\\
19.4022864457019	0\\
19.4013559309827	0\\
19.4004253165673	0\\
19.3994946024343	0\\
19.3985637885623	0\\
19.39763287493	0\\
19.3967018615159	0\\
19.3957707482988	0\\
19.3948395352571	0\\
19.3939082223695	0\\
19.3929768096145	0\\
19.3920452969708	0\\
19.3911136844169	0\\
19.3901819719313	0\\
19.3892501594927	0\\
19.3883182470796	0\\
19.3873862346706	0\\
19.386454122244	0\\
19.3855219097786	0\\
19.3845895972529	0\\
19.3836571846452	0\\
19.3827246719343	0\\
19.3817920590984	0\\
19.3808593461163	0\\
19.3799265329662	0\\
19.3789936196269	0\\
19.3780606060766	0\\
19.3771274922938	0\\
19.3761942782572	0\\
19.375260963945	0\\
19.3743275493357	0\\
19.3733940344078	0\\
19.3724604191398	0\\
19.37152670351	0\\
19.3705928874968	0\\
19.3696589710787	0\\
19.3687249542341	0\\
19.3677908369414	0\\
19.3668566191789	0\\
19.3659223009251	0\\
19.3649878821583	0\\
19.364053362857	0\\
19.3631187429994	0\\
19.362184022564	0\\
19.361249201529	0\\
19.3603142798728	0\\
19.3593792575738	0\\
19.3584441346103	0\\
19.3575089109607	0\\
19.3565735866031	0\\
19.355638161516	0\\
19.3547026356776	0\\
19.3537670090663	0\\
19.3528312816602	0\\
19.3518954534378	0\\
19.3509595243772	0\\
19.3500234944568	0\\
19.3490873636548	0\\
19.3481511319495	0\\
19.347214799319	0\\
19.3462783657417	0\\
19.3453418311957	0\\
19.3444051956593	0\\
19.3434684591108	0\\
19.3425316215283	0\\
19.3415946828899	0\\
19.340657643174	0\\
19.3397205023588	0\\
19.3387832604223	0\\
19.3378459173427	0\\
19.3369084730983	0\\
19.3359709276672	0\\
19.3350332810276	0\\
19.3340955331575	0\\
19.3331576840352	0\\
19.3322197336387	0\\
19.3312816819462	0\\
19.3303435289358	0\\
19.3294052745857	0\\
19.3284669188739	0\\
19.3275284617785	0\\
19.3265899032776	0\\
19.3256512433493	0\\
19.3247124819716	0\\
19.3237736191227	0\\
19.3228346547807	0\\
19.3218955889234	0\\
19.3209564215291	0\\
19.3200171525757	0\\
19.3190777820413	0\\
19.3181383099038	0\\
19.3171987361414	0\\
19.3162590607321	0\\
19.3153192836537	0\\
19.3143794048844	0\\
19.3134394244021	0\\
19.3124993421849	0\\
19.3115591582106	0\\
19.3106188724572	0\\
19.3096784849028	0\\
19.3087379955252	0\\
19.3077974043024	0\\
19.3068567112124	0\\
19.3059159162331	0\\
19.3049750193424	0\\
19.3040340205182	0\\
19.3030929197385	0\\
19.3021517169811	0\\
19.301210412224	0\\
19.300269005445	0\\
19.299327496622	0\\
19.2983858857329	0\\
19.2974441727555	0\\
19.2965023576678	0\\
19.2955604404475	0\\
19.2946184210726	0\\
19.2936762995208	0\\
19.29273407577	0\\
19.2917917497981	0\\
19.2908493215827	0\\
19.2899067911019	0\\
19.2889641583332	0\\
19.2880214232546	0\\
19.2870785858439	0\\
19.2861356460787	0\\
19.285192603937	0\\
19.2842494593964	0\\
19.2833062124347	0\\
19.2823628630297	0\\
19.2814194111591	0\\
19.2804758568007	0\\
19.2795321999321	0\\
19.2785884405312	0\\
19.2776445785755	0\\
19.276700614043	0\\
19.2757565469111	0\\
19.2748123771577	0\\
19.2738681047604	0\\
19.2729237296969	0\\
19.2719792519448	0\\
19.2710346714819	0\\
19.2700899882858	0\\
19.2691452023342	0\\
19.2682003136046	0\\
19.2672553220747	0\\
19.2663102277222	0\\
19.2653650305247	0\\
19.2644197304598	0\\
19.263474327505	0\\
19.2625288216381	0\\
19.2615832128365	0\\
19.260637501078	0\\
19.2596916863399	0\\
19.2587457686	0\\
19.2577997478358	0\\
19.2568536240248	0\\
19.2559073971447	0\\
19.2549610671728	0\\
19.2540146340868	0\\
19.2530680978642	0\\
19.2521214584825	0\\
19.2511747159193	0\\
19.2502278701519	0\\
19.249280921158	0\\
19.248333868915	0\\
19.2473867134004	0\\
19.2464394545917	0\\
19.2454920924663	0\\
19.2445446270017	0\\
19.2435970581754	0\\
19.2426493859647	0\\
19.2417016103472	0\\
19.2407537313003	0\\
19.2398057488013	0\\
19.2388576628277	0\\
19.237909473357	0\\
19.2369611803665	0\\
19.2360127838336	0\\
19.2350642837357	0\\
19.2341156800501	0\\
19.2331669727543	0\\
19.2322181618256	0\\
19.2312692472414	0\\
19.230320228979	0\\
19.2293711070157	0\\
19.228421881329	0\\
19.227472551896	0\\
19.2265231186942	0\\
19.2255735817009	0\\
19.2246239408933	0\\
19.2236741962487	0\\
19.2227243477445	0\\
19.2217743953579	0\\
19.2208243390662	0\\
19.2198741788467	0\\
19.2189239146766	0\\
19.2179735465332	0\\
19.2170230743937	0\\
19.2160724982354	0\\
19.2151218180355	0\\
19.2141710337711	0\\
19.2132201454196	0\\
19.2122691529582	0\\
19.211318056364	0\\
19.2103668556142	0\\
19.209415550686	0\\
19.2084641415566	0\\
19.2075126282032	0\\
19.2065610106029	0\\
19.2056092887328	0\\
19.2046574625702	0\\
19.2037055320922	0\\
19.2027534972759	0\\
19.2018013580984	0\\
19.2008491145368	0\\
19.1998967665683	0\\
19.19894431417	0\\
19.1979917573189	0\\
19.1970390959921	0\\
19.1960863301668	0\\
19.1951334598199	0\\
19.1941804849286	0\\
19.1932274054699	0\\
19.1922742214209	0\\
19.1913209327586	0\\
19.19036753946	0\\
19.1894140415021	0\\
19.1884604388621	0\\
19.1875067315169	0\\
19.1865529194434	0\\
19.1855990026188	0\\
19.1846449810199	0\\
19.1836908546238	0\\
19.1827366234075	0\\
19.1817822873478	0\\
19.1808278464219	0\\
19.1798733006065	0\\
19.1789186498787	0\\
19.1779638942154	0\\
19.1770090335936	0\\
19.17605406799	0\\
19.1750989973818	0\\
19.1741438217457	0\\
19.1731885410587	0\\
19.1722331552976	0\\
19.1712776644393	0\\
19.1703220684608	0\\
19.1693663673389	0\\
19.1684105610504	0\\
19.1674546495721	0\\
19.1664986328811	0\\
19.165542510954	0\\
19.1645862837676	0\\
19.163629951299	0\\
19.1626735135247	0\\
19.1617169704217	0\\
19.1607603219667	0\\
19.1598035681366	0\\
19.1588467089081	0\\
19.157889744258	0\\
19.156932674163	0\\
19.1559754986	0\\
19.1550182175456	0\\
19.1540608309766	0\\
19.1531033388698	0\\
19.1521457412019	0\\
19.1511880379496	0\\
19.1502302290896	0\\
19.1492723145986	0\\
19.1483142944534	0\\
19.1473561686305	0\\
19.1463979371068	0\\
19.1454395998589	0\\
19.1444811568633	0\\
19.1435226080969	0\\
19.1425639535363	0\\
19.141605193158	0\\
19.1406463269388	0\\
19.1396873548552	0\\
19.1387282768839	0\\
19.1377690930015	0\\
19.1368098031847	0\\
19.1358504074099	0\\
19.1348909056538	0\\
19.133931297893	0\\
19.1329715841041	0\\
19.1320117642635	0\\
19.131051838348	0\\
19.130091806334	0\\
19.129131668198	0\\
19.1281714239167	0\\
19.1272110734664	0\\
19.1262506168239	0\\
19.1252900539655	0\\
19.1243293848678	0\\
19.1233686095073	0\\
19.1224077278604	0\\
19.1214467399037	0\\
19.1204856456135	0\\
19.1195244449665	0\\
19.118563137939	0\\
19.1176017245075	0\\
19.1166402046484	0\\
19.1156785783382	0\\
19.1147168455533	0\\
19.1137550062701	0\\
19.112793060465	0\\
19.1118310081145	0\\
19.1108688491949	0\\
19.1099065836825	0\\
19.1089442115539	0\\
19.1079817327853	0\\
19.1070191473532	0\\
19.1060564552338	0\\
19.1050936564036	0\\
19.1041307508388	0\\
19.1031677385159	0\\
19.102204619411	0\\
19.1012413935006	0\\
19.100278060761	0\\
19.0993146211684	0\\
19.0983510746992	0\\
19.0973874213296	0\\
19.0964236610359	0\\
19.0954597937943	0\\
19.0944958195813	0\\
19.0935317383729	0\\
19.0925675501454	0\\
19.0916032548751	0\\
19.0906388525383	0\\
19.089674343111	0\\
19.0887097265696	0\\
19.0877450028903	0\\
19.0867801720492	0\\
19.0858152340225	0\\
19.0848501887865	0\\
19.0838850363172	0\\
19.082919776591	0\\
19.0819544095838	0\\
19.0809889352719	0\\
19.0800233536314	0\\
19.0790576646385	0\\
19.0780918682692	0\\
19.0771259644997	0\\
19.0761599533061	0\\
19.0751938346646	0\\
19.0742276085511	0\\
19.0732612749418	0\\
19.0722948338128	0\\
19.0713282851401	0\\
19.0703616288999	0\\
19.069394865068	0\\
19.0684279936207	0\\
19.0674610145339	0\\
19.0664939277837	0\\
19.0655267333461	0\\
19.0645594311971	0\\
19.0635920213127	0\\
19.0626245036689	0\\
19.0616568782417	0\\
19.0606891450071	0\\
19.0597213039411	0\\
19.0587533550196	0\\
19.0577852982186	0\\
19.056817133514	0\\
19.0558488608818	0\\
19.054880480298	0\\
19.0539119917383	0\\
19.0529433951789	0\\
19.0519746905955	0\\
19.051005877964	0\\
19.0500369572605	0\\
19.0490679284607	0\\
19.0480987915405	0\\
19.0471295464758	0\\
19.0461601932425	0\\
19.0451907318164	0\\
19.0442211621734	0\\
19.0432514842893	0\\
19.0422816981399	0\\
19.041311803701	0\\
19.0403418009485	0\\
19.0393716898582	0\\
19.0384014704059	0\\
19.0374311425673	0\\
19.0364607063182	0\\
19.0354901616345	0\\
19.0345195084918	0\\
19.033548746866	0\\
19.0325778767328	0\\
19.0316068980678	0\\
19.030635810847	0\\
19.0296646150459	0\\
19.0286933106403	0\\
19.0277218976059	0\\
19.0267503759184	0\\
19.0257787455534	0\\
19.0248070064868	0\\
19.0238351586941	0\\
19.0228632021511	0\\
19.0218911368333	0\\
19.0209189627164	0\\
19.0199466797761	0\\
19.018974287988	0\\
19.0180017873277	0\\
19.0170291777709	0\\
19.0160564592932	0\\
19.0150836318701	0\\
19.0141106954772	0\\
19.0131376500902	0\\
19.0121644956846	0\\
19.011191232236	0\\
19.0102178597199	0\\
19.009244378112	0\\
19.0082707873876	0\\
19.0072970875225	0\\
19.006323278492	0\\
19.0053493602718	0\\
19.0043753328373	0\\
19.003401196164	0\\
19.0024269502275	0\\
19.0014525950032	0\\
19.0004781304665	0\\
18.9995035565931	0\\
18.9985288733582	0\\
18.9975540807375	0\\
18.9965791787063	0\\
18.99560416724	0\\
18.9946290463141	0\\
18.9936538159041	0\\
18.9926784759852	0\\
18.991703026533	0\\
18.9907274675229	0\\
18.9897517989301	0\\
18.9887760207301	0\\
18.9878001328983	0\\
18.9868241354099	0\\
18.9858480282405	0\\
18.9848718113653	0\\
18.9838954847596	0\\
18.9829190483989	0\\
18.9819425022583	0\\
18.9809658463133	0\\
18.9799890805391	0\\
18.979012204911	0\\
18.9780352194043	0\\
18.9770581239943	0\\
18.9760809186562	0\\
18.9751036033654	0\\
18.974126178097	0\\
18.9731486428264	0\\
18.9721709975287	0\\
18.9711932421792	0\\
18.9702153767531	0\\
18.9692374012257	0\\
18.968259315572	0\\
18.9672811197674	0\\
18.966302813787	0\\
18.9653243976059	0\\
18.9643458711994	0\\
18.9633672345426	0\\
18.9623884876107	0\\
18.9614096303788	0\\
18.9604306628221	0\\
18.9594515849156	0\\
18.9584723966345	0\\
18.957493097954	0\\
18.956513688849	0\\
18.9555341692948	0\\
18.9545545392663	0\\
18.9535747987387	0\\
18.9525949476871	0\\
18.9516149860864	0\\
18.9506349139119	0\\
18.9496547311384	0\\
18.948674437741	0\\
18.9476940336948	0\\
18.9467135189748	0\\
18.945732893556	0\\
18.9447521574133	0\\
18.9437713105219	0\\
18.9427903528566	0\\
18.9418092843924	0\\
18.9408281051044	0\\
18.9398468149674	0\\
18.9388654139565	0\\
18.9378839020465	0\\
18.9369022792125	0\\
18.9359205454292	0\\
18.9349387006717	0\\
18.9339567449148	0\\
18.9329746781335	0\\
18.9319925003027	0\\
18.9310102113971	0\\
18.9300278113918	0\\
18.9290453002615	0\\
18.9280626779811	0\\
18.9270799445255	0\\
18.9260970998695	0\\
18.9251141439879	0\\
18.9241310768557	0\\
18.9231478984474	0\\
18.9221646087381	0\\
18.9211812077025	0\\
18.9201976953153	0\\
18.9192140715514	0\\
18.9182303363855	0\\
18.9172464897923	0\\
18.9162625317468	0\\
18.9152784622235	0\\
18.9142942811972	0\\
18.9133099886427	0\\
18.9123255845346	0\\
18.9113410688478	0\\
18.9103564415568	0\\
18.9093717026364	0\\
18.9083868520613	0\\
18.9074018898061	0\\
18.9064168158455	0\\
18.9054316301542	0\\
18.9044463327068	0\\
18.903460923478	0\\
18.9024754024424	0\\
18.9014897695746	0\\
18.9005040248493	0\\
18.899518168241	0\\
18.8985321997243	0\\
18.8975461192739	0\\
18.8965599268643	0\\
18.8955736224701	0\\
18.8945872060659	0\\
18.8936006776262	0\\
18.8926140371256	0\\
18.8916272845385	0\\
18.8906404198396	0\\
18.8896534430034	0\\
18.8886663540043	0\\
18.8876791528169	0\\
18.8866918394156	0\\
18.8857044137751	0\\
18.8847168758696	0\\
18.8837292256737	0\\
18.882741463162	0\\
18.8817535883087	0\\
18.8807656010884	0\\
18.8797775014754	0\\
18.8787892894443	0\\
18.8778009649695	0\\
18.8768125280252	0\\
18.875823978586	0\\
18.8748353166263	0\\
18.8738465421203	0\\
18.8728576550426	0\\
18.8718686553674	0\\
18.8708795430691	0\\
18.8698903181221	0\\
18.8689009805006	0\\
18.8679115301791	0\\
18.8669219671318	0\\
18.8659322913331	0\\
18.8649425027572	0\\
18.8639526013785	0\\
18.8629625871713	0\\
18.8619724601097	0\\
18.8609822201681	0\\
18.8599918673208	0\\
18.859001401542	0\\
18.8580108228059	0\\
18.8570201310867	0\\
18.8560293263588	0\\
18.8550384085963	0\\
18.8540473777734	0\\
18.8530562338642	0\\
18.8520649768431	0\\
18.8510736066842	0\\
18.8500821233616	0\\
18.8490905268495	0\\
18.8480988171221	0\\
18.8471069941535	0\\
18.8461150579179	0\\
18.8451230083893	0\\
18.8441308455419	0\\
18.8431385693498	0\\
18.8421461797871	0\\
18.8411536768279	0\\
18.8401610604463	0\\
18.8391683306163	0\\
18.838175487312	0\\
18.8371825305075	0\\
18.8361894601768	0\\
18.8351962762939	0\\
18.8342029788329	0\\
18.8332095677678	0\\
18.8322160430727	0\\
18.8312224047214	0\\
18.8302286526881	0\\
18.8292347869466	0\\
18.828240807471	0\\
18.8272467142352	0\\
18.8262525072132	0\\
18.825258186379	0\\
18.8242637517064	0\\
18.8232692031695	0\\
18.822274540742	0\\
18.821279764398	0\\
18.8202848741114	0\\
18.819289869856	0\\
18.8182947516057	0\\
18.8172995193344	0\\
18.816304173016	0\\
18.8153087126244	0\\
18.8143131381333	0\\
18.8133174495166	0\\
18.8123216467482	0\\
18.8113257298018	0\\
18.8103296986513	0\\
18.8093335532706	0\\
18.8083372936333	0\\
18.8073409197133	0\\
18.8063444314844	0\\
18.8053478289202	0\\
18.8043511119947	0\\
18.8033542806815	0\\
18.8023573349544	0\\
18.8013602747871	0\\
18.8003631001533	0\\
18.7993658110267	0\\
18.7983684073811	0\\
18.7973708891902	0\\
18.7963732564276	0\\
18.795375509067	0\\
18.7943776470821	0\\
18.7933796704465	0\\
18.792381579134	0\\
18.791383373118	0\\
18.7903850523724	0\\
18.7893866168706	0\\
18.7883880665863	0\\
18.7873894014932	0\\
18.7863906215648	0\\
18.7853917267746	0\\
18.7843927170964	0\\
18.7833935925035	0\\
18.7823943529697	0\\
18.7813949984685	0\\
18.7803955289733	0\\
18.7793959444578	0\\
18.7783962448954	0\\
18.7773964302597	0\\
18.7763965005242	0\\
18.7753964556623	0\\
18.7743962956477	0\\
18.7733960204536	0\\
18.7723956300536	0\\
18.7713951244212	0\\
18.7703945035298	0\\
18.7693937673529	0\\
18.7683929158639	0\\
18.7673919490361	0\\
18.7663908668431	0\\
18.7653896692582	0\\
18.7643883562549	0\\
18.7633869278064	0\\
18.7623853838862	0\\
18.7613837244677	0\\
18.7603819495242	0\\
18.7593800590291	0\\
18.7583780529556	0\\
18.7573759312772	0\\
18.7563736939671	0\\
18.7553713409987	0\\
18.7543688723453	0\\
18.7533662879801	0\\
18.7523635878765	0\\
18.7513607720077	0\\
18.7503578403469	0\\
18.7493547928675	0\\
18.7483516295428	0\\
18.7473483503458	0\\
18.7463449552499	0\\
18.7453414442283	0\\
18.7443378172541	0\\
18.7433340743007	0\\
18.7423302153411	0\\
18.7413262403486	0\\
18.7403221492963	0\\
18.7393179421574	0\\
18.7383136189051	0\\
18.7373091795125	0\\
18.7363046239527	0\\
18.7352999521988	0\\
18.7342951642241	0\\
18.7332902600015	0\\
18.7322852395042	0\\
18.7312801027053	0\\
18.7302748495779	0\\
18.7292694800949	0\\
18.7282639942296	0\\
18.7272583919549	0\\
18.7262526732439	0\\
18.7252468380696	0\\
18.724240886405	0\\
18.7232348182232	0\\
18.7222286334971	0\\
18.7212223321998	0\\
18.7202159143042	0\\
18.7192093797833	0\\
18.7182027286101	0\\
18.7171959607576	0\\
18.7161890761987	0\\
18.7151820749062	0\\
18.7141749568533	0\\
18.7131677220127	0\\
18.7121603703574	0\\
18.7111529018603	0\\
18.7101453164943	0\\
18.7091376142322	0\\
18.708129795047	0\\
18.7071218589115	0\\
18.7061138057985	0\\
18.7051056356809	0\\
18.7040973485316	0\\
18.7030889443234	0\\
18.702080423029	0\\
18.7010717846213	0\\
18.7000630290731	0\\
18.6990541563572	0\\
18.6980451664463	0\\
18.6970360593132	0\\
18.6960268349307	0\\
18.6950174932716	0\\
18.6940080343085	0\\
18.6929984580142	0\\
18.6919887643615	0\\
18.6909789533229	0\\
18.6899690248714	0\\
18.6889589789794	0\\
18.6879488156198	0\\
18.6869385347651	0\\
18.6859281363881	0\\
18.6849176204614	0\\
18.6839069869577	0\\
18.6828962358495	0\\
18.6818853671096	0\\
18.6808743807105	0\\
18.6798632766248	0\\
18.6788520548252	0\\
18.6778407152842	0\\
18.6768292579744	0\\
18.6758176828684	0\\
18.6748059899387	0\\
18.6737941791579	0\\
18.6727822504985	0\\
18.6717702039331	0\\
18.6707580394341	0\\
18.6697457569741	0\\
18.6687333565256	0\\
18.6677208380611	0\\
18.666708201553	0\\
18.6656954469739	0\\
18.6646825742962	0\\
18.6636695834923	0\\
18.6626564745347	0\\
18.6616432473958	0\\
18.6606299020481	0\\
18.659616438464	0\\
18.6586028566158	0\\
18.657589156476	0\\
18.6565753380169	0\\
18.655561401211	0\\
18.6545473460306	0\\
18.653533172448	0\\
18.6525188804356	0\\
18.6515044699658	0\\
18.6504899410108	0\\
18.6494752935431	0\\
18.6484605275348	0\\
18.6474456429583	0\\
18.6464306397859	0\\
18.6454155179898	0\\
18.6444002775424	0\\
18.6433849184158	0\\
18.6423694405824	0\\
18.6413538440144	0\\
18.640338128684	0\\
18.6393222945635	0\\
18.6383063416249	0\\
18.6372902698406	0\\
18.6362740791828	0\\
18.6352577696236	0\\
18.6342413411351	0\\
18.6332247936896	0\\
18.6322081272592	0\\
18.6311913418161	0\\
18.6301744373324	0\\
18.6291574137801	0\\
18.6281402711315	0\\
18.6271230093586	0\\
18.6261056284335	0\\
18.6250881283283	0\\
18.624070509015	0\\
18.6230527704659	0\\
18.6220349126528	0\\
18.6210169355478	0\\
18.619998839123	0\\
18.6189806233505	0\\
18.6179622882021	0\\
18.6169438336499	0\\
18.615925259666	0\\
18.6149065662223	0\\
18.6138877532907	0\\
18.6128688208433	0\\
18.611849768852	0\\
18.6108305972887	0\\
18.6098113061254	0\\
18.608791895334	0\\
18.6077723648864	0\\
18.6067527147546	0\\
18.6057329449104	0\\
18.6047130553257	0\\
18.6036930459724	0\\
18.6026729168223	0\\
18.6016526678474	0\\
18.6006322990195	0\\
18.5996118103103	0\\
18.5985912016918	0\\
18.5975704731357	0\\
18.5965496246138	0\\
18.595528656098	0\\
18.5945075675601	0\\
18.5934863589718	0\\
18.5924650303048	0\\
18.591443581531	0\\
18.5904220126221	0\\
18.5894003235498	0\\
18.5883785142859	0\\
18.587356584802	0\\
18.5863345350699	0\\
18.5853123650613	0\\
18.5842900747479	0\\
18.5832676641013	0\\
18.5822451330932	0\\
18.5812224816953	0\\
18.5801997098792	0\\
18.5791768176166	0\\
18.5781538048791	0\\
18.5771306716382	0\\
18.5761074178657	0\\
18.5750840435331	0\\
18.5740605486119	0\\
18.5730369330739	0\\
18.5720131968905	0\\
18.5709893400333	0\\
18.5699653624738	0\\
18.5689412641836	0\\
18.5679170451343	0\\
18.5668927052972	0\\
18.565868244644	0\\
18.5648436631462	0\\
18.5638189607751	0\\
18.5627941375024	0\\
18.5617691932994	0\\
18.5607441281377	0\\
18.5597189419886	0\\
18.5586936348236	0\\
18.5576682066141	0\\
18.5566426573316	0\\
18.5556169869475	0\\
18.5545911954331	0\\
18.5535652827598	0\\
18.5525392488991	0\\
18.5515130938222	0\\
18.5504868175006	0\\
18.5494604199055	0\\
18.5484339010084	0\\
18.5474072607805	0\\
18.5463804991931	0\\
18.5453536162176	0\\
18.5443266118253	0\\
18.5432994859874	0\\
18.5422722386752	0\\
18.54124486986	0\\
18.540217379513	0\\
18.5391897676055	0\\
18.5381620341087	0\\
18.5371341789938	0\\
18.5361062022321	0\\
18.5350781037947	0\\
18.5340498836529	0\\
18.5330215417778	0\\
18.5319930781405	0\\
18.5309644927124	0\\
18.5299357854644	0\\
18.5289069563678	0\\
18.5278780053936	0\\
18.5268489325131	0\\
18.5258197376973	0\\
18.5247904209172	0\\
18.5237609821441	0\\
18.522731421349	0\\
18.5217017385029	0\\
18.5206719335769	0\\
18.5196420065421	0\\
18.5186119573694	0\\
18.51758178603	0\\
18.5165514924948	0\\
18.5155210767349	0\\
18.5144905387213	0\\
18.5134598784248	0\\
18.5124290958166	0\\
18.5113981908675	0\\
18.5103671635486	0\\
18.5093360138308	0\\
18.5083047416849	0\\
18.507273347082	0\\
18.506241829993	0\\
18.5052101903887	0\\
18.5041784282401	0\\
18.503146543518	0\\
18.5021145361933	0\\
18.5010824062369	0\\
18.5000501536196	0\\
18.4990177783122	0\\
18.4979852802857	0\\
18.4969526595107	0\\
18.4959199159582	0\\
18.494887049599	0\\
18.4938540604037	0\\
18.4928209483433	0\\
18.4917877133884	0\\
18.4907543555098	0\\
18.4897208746783	0\\
18.4886872708646	0\\
18.4876535440395	0\\
18.4866196941736	0\\
18.4855857212376	0\\
18.4845516252024	0\\
18.4835174060384	0\\
18.4824830637165	0\\
18.4814485982073	0\\
18.4804140094814	0\\
18.4793792975094	0\\
18.4783444622621	0\\
18.47730950371	0\\
18.4762744218238	0\\
18.475239216574	0\\
18.4742038879313	0\\
18.4731684358661	0\\
18.4721328603492	0\\
18.471097161351	0\\
18.4700613388421	0\\
18.469025392793	0\\
18.4679893231743	0\\
18.4669531299565	0\\
18.4659168131101	0\\
18.4648803726055	0\\
18.4638438084134	0\\
18.462807120504	0\\
18.461770308848	0\\
18.4607333734157	0\\
18.4596963141777	0\\
18.4586591311042	0\\
18.4576218241659	0\\
18.456584393333	0\\
18.455546838576	0\\
18.4545091598652	0\\
18.4534713571711	0\\
18.452433430464	0\\
18.4513953797142	0\\
18.4503572048922	0\\
18.4493189059683	0\\
18.4482804829127	0\\
18.4472419356958	0\\
18.4462032642879	0\\
18.4451644686593	0\\
18.4441255487803	0\\
18.4430865046212	0\\
18.4420473361521	0\\
18.4410080433435	0\\
18.4399686261654	0\\
18.4389290845882	0\\
18.437889418582	0\\
18.4368496281172	0\\
18.4358097131637	0\\
18.434769673692	0\\
18.433729509672	0\\
18.4326892210741	0\\
18.4316488078683	0\\
18.4306082700248	0\\
18.4295676075138	0\\
18.4285268203053	0\\
18.4274859083695	0\\
18.4264448716764	0\\
18.4254037101963	0\\
18.424362423899	0\\
18.4233210127548	0\\
18.4222794767336	0\\
18.4212378158056	0\\
18.4201960299407	0\\
18.419154119109	0\\
18.4181120832805	0\\
18.4170699224252	0\\
18.4160276365131	0\\
18.4149852255142	0\\
18.4139426893984	0\\
18.4129000281358	0\\
18.4118572416962	0\\
18.4108143300496	0\\
18.409771293166	0\\
18.4087281310152	0\\
18.4076848435672	0\\
18.4066414307918	0\\
18.405597892659	0\\
18.4045542291386	0\\
18.4035104402005	0\\
18.4024665258145	0\\
18.4014224859505	0\\
18.4003783205783	0\\
18.3993340296677	0\\
18.3982896131886	0\\
18.3972450711108	0\\
18.3962004034039	0\\
18.3951556100379	0\\
18.3941106909824	0\\
18.3930656462073	0\\
18.3920204756822	0\\
18.390975179377	0\\
18.3899297572612	0\\
18.3888842093047	0\\
18.3878385354772	0\\
18.3867927357483	0\\
18.3857468100877	0\\
18.384700758465	0\\
18.38365458085	0\\
18.3826082772123	0\\
18.3815618475215	0\\
18.3805152917472	0\\
18.3794686098591	0\\
18.3784218018267	0\\
18.3773748676197	0\\
18.3763278072076	0\\
18.37528062056	0\\
18.3742333076464	0\\
18.3731858684365	0\\
18.3721383028996	0\\
18.3710906110055	0\\
18.3700427927235	0\\
18.3689948480232	0\\
18.367946776874	0\\
18.3668985792455	0\\
18.3658502551071	0\\
18.3648018044283	0\\
18.3637532271785	0\\
18.3627045233272	0\\
18.3616556928437	0\\
18.3606067356976	0\\
18.3595576518582	0\\
18.3585084412948	0\\
18.357459103977	0\\
18.3564096398739	0\\
18.3553600489551	0\\
18.3543103311899	0\\
18.3532604865475	0\\
18.3522105149973	0\\
18.3511604165087	0\\
18.3501101910509	0\\
18.3490598385933	0\\
18.348009359105	0\\
18.3469587525554	0\\
18.3459080189138	0\\
18.3448571581493	0\\
18.3438061702313	0\\
18.3427550551289	0\\
18.3417038128113	0\\
18.3406524432479	0\\
18.3396009464076	0\\
18.3385493222599	0\\
18.3374975707737	0\\
18.3364456919182	0\\
18.3353936856627	0\\
18.3343415519763	0\\
18.3332892908279	0\\
18.3322369021869	0\\
18.3311843860223	0\\
18.3301317423031	0\\
18.3290789709985	0\\
18.3280260720775	0\\
18.3269730455092	0\\
18.3259198912626	0\\
18.3248666093068	0\\
18.3238131996108	0\\
18.3227596621435	0\\
18.3217059968741	0\\
18.3206522037714	0\\
18.3195982828045	0\\
18.3185442339423	0\\
18.3174900571538	0\\
18.316435752408	0\\
18.3153813196737	0\\
18.3143267589199	0\\
18.3132720701155	0\\
18.3122172532293	0\\
18.3111623082304	0\\
18.3101072350875	0\\
18.3090520337696	0\\
18.3079967042454	0\\
18.3069412464838	0\\
18.3058856604537	0\\
18.3048299461239	0\\
18.3037741034631	0\\
18.3027181324402	0\\
18.301662033024	0\\
18.3006058051832	0\\
18.2995494488866	0\\
18.298492964103	0\\
18.297436350801	0\\
18.2963796089495	0\\
18.2953227385171	0\\
18.2942657394725	0\\
18.2932086117844	0\\
18.2921513554216	0\\
18.2910939703527	0\\
18.2900364565463	0\\
18.2889788139711	0\\
18.2879210425957	0\\
18.2868631423888	0\\
18.285805113319	0\\
18.2847469553548	0\\
18.2836886684649	0\\
18.2826302526179	0\\
18.2815717077822	0\\
18.2805130339266	0\\
18.2794542310194	0\\
18.2783952990293	0\\
18.2773362379248	0\\
18.2762770476743	0\\
18.2752177282465	0\\
18.2741582796097	0\\
18.2730987017324	0\\
18.2720389945831	0\\
18.2709791581304	0\\
18.2699191923425	0\\
18.2688590971879	0\\
18.2677988726351	0\\
18.2667385186525	0\\
18.2656780352084	0\\
18.2646174222712	0\\
18.2635566798094	0\\
18.2624958077912	0\\
18.261434806185	0\\
18.2603736749592	0\\
18.2593124140821	0\\
18.2582510235219	0\\
18.2571895032471	0\\
18.2561278532258	0\\
18.2550660734265	0\\
18.2540041638172	0\\
18.2529421243663	0\\
18.2518799550421	0\\
18.2508176558127	0\\
18.2497552266465	0\\
18.2486926675115	0\\
18.247629978376	0\\
18.2465671592082	0\\
18.2455042099762	0\\
18.2444411306483	0\\
18.2433779211925	0\\
18.2423145815771	0\\
18.24125111177	0\\
18.2401875117396	0\\
18.2391237814537	0\\
18.2380599208807	0\\
18.2369959299884	0\\
18.235931808745	0\\
18.2348675571186	0\\
18.2338031750772	0\\
18.2327386625888	0\\
18.2316740196215	0\\
18.2306092461432	0\\
18.229544342122	0\\
18.2284793075259	0\\
18.2274141423228	0\\
18.2263488464806	0\\
18.2252834199674	0\\
18.2242178627511	0\\
18.2231521747995	0\\
18.2220863560807	0\\
18.2210204065625	0\\
18.2199543262129	0\\
18.2188881149996	0\\
18.2178217728906	0\\
18.2167552998537	0\\
18.2156886958567	0\\
18.2146219608676	0\\
18.2135550948541	0\\
18.212488097784	0\\
18.2114209696252	0\\
18.2103537103453	0\\
18.2092863199123	0\\
18.2082187982938	0\\
18.2071511454576	0\\
18.2060833613714	0\\
18.205015446003	0\\
18.20394739932	0\\
18.2028792212903	0\\
18.2018109118814	0\\
18.200742471061	0\\
18.1996738987969	0\\
18.1986051950566	0\\
18.1975363598079	0\\
18.1964673930182	0\\
18.1953982946554	0\\
18.1943290646869	0\\
18.1932597030804	0\\
18.1921902098034	0\\
18.1911205848235	0\\
18.1900508281083	0\\
18.1889809396253	0\\
18.1879109193421	0\\
18.1868407672262	0\\
18.185770483245	0\\
18.1847000673661	0\\
18.1836295195571	0\\
18.1825588397852	0\\
18.1814880280181	0\\
18.1804170842231	0\\
18.1793460083678	0\\
18.1782748004194	0\\
18.1772034603455	0\\
18.1761319881135	0\\
18.1750603836907	0\\
18.1739886470445	0\\
18.1729167781423	0\\
18.1718447769514	0\\
18.1707726434392	0\\
18.169700377573	0\\
18.1686279793201	0\\
18.1675554486479	0\\
18.1664827855236	0\\
18.1654099899145	0\\
18.1643370617879	0\\
18.163264001111	0\\
18.162190807851	0\\
18.1611174819753	0\\
18.1600440234511	0\\
18.1589704322454	0\\
18.1578967083257	0\\
18.1568228516589	0\\
18.1557488622123	0\\
18.1546747399531	0\\
18.1536004848484	0\\
18.1525260968653	0\\
18.151451575971	0\\
18.1503769221326	0\\
18.1493021353171	0\\
18.1482272154917	0\\
18.1471521626235	0\\
18.1460769766794	0\\
18.1450016576266	0\\
18.143926205432	0\\
18.1428506200628	0\\
18.1417749014859	0\\
18.1406990496683	0\\
18.1396230645771	0\\
18.1385469461791	0\\
18.1374706944414	0\\
18.1363943093309	0\\
18.1353177908146	0\\
18.1342411388593	0\\
18.133164353432	0\\
18.1320874344996	0\\
18.131010382029	0\\
18.1299331959871	0\\
18.1288558763406	0\\
18.1277784230566	0\\
18.1267008361018	0\\
18.1256231154431	0\\
18.1245452610472	0\\
18.123467272881	0\\
18.1223891509112	0\\
18.1213108951047	0\\
18.1202325054283	0\\
18.1191539818485	0\\
18.1180753243323	0\\
18.1169965328464	0\\
18.1159176073574	0\\
18.114838547832	0\\
18.113759354237	0\\
18.1126800265391	0\\
18.1116005647049	0\\
18.110520968701	0\\
18.1094412384941	0\\
18.1083613740508	0\\
18.1072813753378	0\\
18.1062012423217	0\\
18.105120974969	0\\
18.1040405732464	0\\
18.1029600371203	0\\
18.1018793665574	0\\
18.1007985615242	0\\
18.0997176219873	0\\
18.0986365479131	0\\
18.0975553392681	0\\
18.0964739960189	0\\
18.0953925181319	0\\
18.0943109055736	0\\
18.0932291583105	0\\
18.092147276309	0\\
18.0910652595355	0\\
18.0899831079564	0\\
18.0889008215382	0\\
18.0878184002472	0\\
18.0867358440499	0\\
18.0856531529126	0\\
18.0845703268016	0\\
18.0834873656833	0\\
18.082404269524	0\\
18.0813210382901	0\\
18.0802376719478	0\\
18.0791541704634	0\\
18.0780705338033	0\\
18.0769867619336	0\\
18.0759028548207	0\\
18.0748188124308	0\\
18.0737346347301	0\\
18.0726503216848	0\\
18.0715658732611	0\\
18.0704812894252	0\\
18.0693965701434	0\\
18.0683117153817	0\\
18.0672267251064	0\\
18.0661415992835	0\\
18.0650563378792	0\\
18.0639709408596	0\\
18.0628854081908	0\\
18.0617997398389	0\\
18.0607139357699	0\\
18.0596279959501	0\\
18.0585419203453	0\\
18.0574557089216	0\\
18.0563693616451	0\\
18.0552828784817	0\\
18.0541962593975	0\\
18.0531095043585	0\\
18.0520226133306	0\\
18.0509355862798	0\\
18.0498484231721	0\\
18.0487611239733	0\\
18.0476736886494	0\\
18.0465861171664	0\\
18.04549840949	0\\
18.0444105655863	0\\
18.043322585421	0\\
18.0422344689601	0\\
18.0411462161693	0\\
18.0400578270146	0\\
18.0389693014617	0\\
18.0378806394764	0\\
18.0367918410246	0\\
18.035702906072	0\\
18.0346138345844	0\\
18.0335246265275	0\\
18.0324352818671	0\\
18.031345800569	0\\
18.0302561825988	0\\
18.0291664279222	0\\
18.028076536505	0\\
18.0269865083128	0\\
18.0258963433113	0\\
18.0248060414662	0\\
18.023715602743	0\\
18.0226250271074	0\\
18.0215343145251	0\\
18.0204434649616	0\\
18.0193524783825	0\\
18.0182613547534	0\\
18.0171700940398	0\\
18.0160786962074	0\\
18.0149871612216	0\\
18.013895489048	0\\
18.012803679652	0\\
18.0117117329992	0\\
18.0106196490551	0\\
18.0095274277851	0\\
18.0084350691548	0\\
18.0073425731294	0\\
18.0062499396745	0\\
18.0051571687556	0\\
18.0040642603379	0\\
18.0029712143869	0\\
18.001878030868	0\\
18.0007847097465	0\\
17.9996912509878	0\\
17.9985976545573	0\\
17.9975039204202	0\\
17.9964100485419	0\\
17.9953160388877	0\\
17.9942218914229	0\\
17.9931276061127	0\\
17.9920331829224	0\\
17.9909386218173	0\\
17.9898439227626	0\\
17.9887490857236	0\\
17.9876541106653	0\\
17.9865589975531	0\\
17.9854637463521	0\\
17.9843683570275	0\\
17.9832728295444	0\\
17.982177163868	0\\
17.9810813599634	0\\
17.9799854177958	0\\
17.9788893373302	0\\
17.9777931185317	0\\
17.9766967613655	0\\
17.9756002657965	0\\
17.9745036317899	0\\
17.9734068593106	0\\
17.9723099483237	0\\
17.9712128987942	0\\
17.9701157106872	0\\
17.9690183839675	0\\
17.9679209186002	0\\
17.9668233145502	0\\
17.9657255717825	0\\
17.9646276902619	0\\
17.9635296699535	0\\
17.9624315108222	0\\
17.9613332128327	0\\
17.96023477595	0\\
17.959136200139	0\\
17.9580374853646	0\\
17.9569386315914	0\\
17.9558396387845	0\\
17.9547405069085	0\\
17.9536412359284	0\\
17.9525418258088	0\\
17.9514422765146	0\\
17.9503425880105	0\\
17.9492427602613	0\\
17.9481427932316	0\\
17.9470426868863	0\\
17.9459424411899	0\\
17.9448420561073	0\\
17.9437415316031	0\\
17.9426408676419	0\\
17.9415400641884	0\\
17.9404391212073	0\\
17.9393380386631	0\\
17.9382368165205	0\\
17.9371354547441	0\\
17.9360339532984	0\\
17.9349323121481	0\\
17.9338305312577	0\\
17.9327286105917	0\\
17.9316265501147	0\\
17.9305243497911	0\\
17.9294220095855	0\\
17.9283195294625	0\\
17.9272169093863	0\\
17.9261141493216	0\\
17.9250112492328	0\\
17.9239082090842	0\\
17.9228050288404	0\\
17.9217017084657	0\\
17.9205982479245	0\\
17.9194946471813	0\\
17.9183909062003	0\\
17.917287024946	0\\
17.9161830033826	0\\
17.9150788414746	0\\
17.9139745391861	0\\
17.9128700964816	0\\
17.9117655133253	0\\
17.9106607896814	0\\
17.9095559255143	0\\
17.9084509207882	0\\
17.9073457754672	0\\
17.9062404895157	0\\
17.9051350628979	0\\
17.9040294955779	0\\
17.9029237875198	0\\
17.901817938688	0\\
17.9007119490465	0\\
17.8996058185594	0\\
17.8984995471909	0\\
17.8973931349051	0\\
17.896286581666	0\\
17.8951798874378	0\\
17.8940730521846	0\\
17.8929660758703	0\\
17.8918589584591	0\\
17.8907516999149	0\\
17.8896443002018	0\\
17.8885367592838	0\\
17.8874290771248	0\\
17.8863212536888	0\\
17.8852132889398	0\\
17.8841051828417	0\\
17.8829969353584	0\\
17.881888546454	0\\
17.8807800160922	0\\
17.8796713442369	0\\
17.8785625308521	0\\
17.8774535759016	0\\
17.8763444793493	0\\
17.8752352411589	0\\
17.8741258612943	0\\
17.8730163397193	0\\
17.8719066763977	0\\
17.8707968712932	0\\
17.8696869243697	0\\
17.8685768355909	0\\
17.8674666049205	0\\
17.8663562323222	0\\
17.8652457177598	0\\
17.864135061197	0\\
17.8630242625973	0\\
17.8619133219246	0\\
17.8608022391424	0\\
17.8596910142143	0\\
17.8585796471041	0\\
17.8574681377753	0\\
17.8563564861915	0\\
17.8552446923163	0\\
17.8541327561133	0\\
17.853020677546	0\\
17.851908456578	0\\
17.8507960931727	0\\
17.8496835872938	0\\
17.8485709389046	0\\
17.8474581479687	0\\
17.8463452144496	0\\
17.8452321383107	0\\
17.8441189195155	0\\
17.8430055580273	0\\
17.8418920538096	0\\
17.8407784068258	0\\
17.8396646170393	0\\
17.8385506844135	0\\
17.8374366089117	0\\
17.8363223904972	0\\
17.8352080291334	0\\
17.8340935247837	0\\
17.8329788774112	0\\
17.8318640869793	0\\
17.8307491534513	0\\
17.8296340767904	0\\
17.8285188569599	0\\
17.827403493923	0\\
17.8262879876429	0\\
17.8251723380828	0\\
17.8240565452059	0\\
17.8229406089754	0\\
17.8218245293545	0\\
17.8207083063062	0\\
17.8195919397938	0\\
17.8184754297803	0\\
17.8173587762288	0\\
17.8162419791025	0\\
17.8151250383644	0\\
17.8140079539775	0\\
17.812890725905	0\\
17.8117733541097	0\\
17.8106558385549	0\\
17.8095381792034	0\\
17.8084203760183	0\\
17.8073024289625	0\\
17.806184337999	0\\
17.8050661030907	0\\
17.8039477242006	0\\
17.8028292012915	0\\
17.8017105343265	0\\
17.8005917232683	0\\
17.7994727680799	0\\
17.7983536687241	0\\
17.7972344251638	0\\
17.7961150373617	0\\
17.7949955052808	0\\
17.7938758288837	0\\
17.7927560081334	0\\
17.7916360429925	0\\
17.7905159334238	0\\
17.7893956793901	0\\
17.7882752808542	0\\
17.7871547377786	0\\
17.7860340501262	0\\
17.7849132178596	0\\
17.7837922409415	0\\
17.7826711193345	0\\
17.7815498530013	0\\
17.7804284419045	0\\
17.7793068860067	0\\
17.7781851852706	0\\
17.7770633396587	0\\
17.7759413491336	0\\
17.7748192136579	0\\
17.773696933194	0\\
17.7725745077045	0\\
17.771451937152	0\\
17.7703292214989	0\\
17.7692063607077	0\\
17.7680833547409	0\\
17.766960203561	0\\
17.7658369071303	0\\
17.7647134654113	0\\
17.7635898783665	0\\
17.7624661459582	0\\
17.7613422681487	0\\
17.7602182449006	0\\
17.759094076176	0\\
17.7579697619374	0\\
17.7568453021471	0\\
17.7557206967674	0\\
17.7545959457606	0\\
17.753471049089	0\\
17.7523460067147	0\\
17.7512208186002	0\\
17.7500954847076	0\\
17.7489700049991	0\\
17.7478443794369	0\\
17.7467186079833	0\\
17.7455926906005	0\\
17.7444666272505	0\\
17.7433404178955	0\\
17.7422140624977	0\\
17.7410875610191	0\\
17.739960913422	0\\
17.7388341196683	0\\
17.7377071797202	0\\
17.7365800935396	0\\
17.7354528610888	0\\
17.7343254823296	0\\
17.7331979572241	0\\
17.7320702857342	0\\
17.7309424678221	0\\
17.7298145034496	0\\
17.7286863925788	0\\
17.7275581351714	0\\
17.7264297311896	0\\
17.7253011805951	0\\
17.72417248335	0\\
17.7230436394159	0\\
17.7219146487549	0\\
17.7207855113288	0\\
17.7196562270995	0\\
17.7185267960286	0\\
17.7173972180781	0\\
17.7162674932098	0\\
17.7151376213853	0\\
17.7140076025666	0\\
17.7128774367153	0\\
17.7117471237931	0\\
17.7106166637618	0\\
17.7094860565831	0\\
17.7083553022187	0\\
17.7072244006302	0\\
17.7060933517794	0\\
17.7049621556277	0\\
17.703830812137	0\\
17.7026993212687	0\\
17.7015676829845	0\\
17.700435897246	0\\
17.6993039640147	0\\
17.6981718832521	0\\
17.69703965492	0\\
17.6959072789796	0\\
17.6947747553926	0\\
17.6936420841205	0\\
17.6925092651247	0\\
17.6913762983667	0\\
17.6902431838079	0\\
17.6891099214098	0\\
17.6879765111337	0\\
17.6868429529412	0\\
17.6857092467935	0\\
17.6845753926521	0\\
17.6834413904783	0\\
17.6823072402334	0\\
17.6811729418788	0\\
17.6800384953757	0\\
17.6789039006856	0\\
17.6777691577696	0\\
17.676634266589	0\\
17.6754992271052	0\\
17.6743640392792	0\\
17.6732287030724	0\\
17.6720932184459	0\\
17.6709575853609	0\\
17.6698218037786	0\\
17.6686858736602	0\\
17.6675497949668	0\\
17.6664135676595	0\\
17.6652771916995	0\\
17.6641406670478	0\\
17.6630039936656	0\\
17.6618671715138	0\\
17.6607302005536	0\\
17.659593080746	0\\
17.658455812052	0\\
17.6573183944326	0\\
17.6561808278488	0\\
17.6550431122616	0\\
17.6539052476319	0\\
17.6527672339208	0\\
17.651629071089	0\\
17.6504907590976	0\\
17.6493522979074	0\\
17.6482136874794	0\\
17.6470749277743	0\\
17.645936018753	0\\
17.6447969603765	0\\
17.6436577526054	0\\
17.6425183954006	0\\
17.641378888723	0\\
17.6402392325332	0\\
17.639099426792	0\\
17.6379594714602	0\\
17.6368193664985	0\\
17.6356791118677	0\\
17.6345387075284	0\\
17.6333981534412	0\\
17.632257449567	0\\
17.6311165958662	0\\
17.6299755922997	0\\
17.6288344388279	0\\
17.6276931354115	0\\
17.626551682011	0\\
17.6254100785872	0\\
17.6242683251004	0\\
17.6231264215113	0\\
17.6219843677803	0\\
17.6208421638681	0\\
17.6196998097351	0\\
17.6185573053417	0\\
17.6174146506485	0\\
17.6162718456158	0\\
17.6151288902042	0\\
17.613985784374	0\\
17.6128425280857	0\\
17.6116991212996	0\\
17.6105555639761	0\\
17.6094118560756	0\\
17.6082679975583	0\\
17.6071239883847	0\\
17.605979828515	0\\
17.6048355179096	0\\
17.6036910565287	0\\
17.6025464443325	0\\
17.6014016812813	0\\
17.6002567673354	0\\
17.599111702455	0\\
17.5979664866002	0\\
17.5968211197313	0\\
17.5956756018083	0\\
17.5945299327915	0\\
17.5933841126411	0\\
17.592238141317	0\\
17.5910920187795	0\\
17.5899457449886	0\\
17.5887993199043	0\\
17.5876527434869	0\\
17.5865060156961	0\\
17.5853591364923	0\\
17.5842121058352	0\\
17.5830649236849	0\\
17.5819175900015	0\\
17.5807701047447	0\\
17.5796224678747	0\\
17.5784746793513	0\\
17.5773267391345	0\\
17.5761786471841	0\\
17.57503040346	0\\
17.5738820079222	0\\
17.5727334605303	0\\
17.5715847612444	0\\
17.5704359100241	0\\
17.5692869068294	0\\
17.5681377516199	0\\
17.5669884443555	0\\
17.5658389849959	0\\
17.5646893735009	0\\
17.5635396098301	0\\
17.5623896939434	0\\
17.5612396258003	0\\
17.5600894053605	0\\
17.5589390325838	0\\
17.5577885074297	0\\
17.5566378298579	0\\
17.555486999828	0\\
17.5543360172995	0\\
17.5531848822322	0\\
17.5520335945854	0\\
17.5508821543189	0\\
17.549730561392	0\\
17.5485788157644	0\\
17.5474269173955	0\\
17.5462748662448	0\\
17.5451226622717	0\\
17.5439703054358	0\\
17.5428177956964	0\\
17.541665133013	0\\
17.540512317345	0\\
17.5393593486517	0\\
17.5382062268925	0\\
17.5370529520269	0\\
17.535899524014	0\\
17.5347459428133	0\\
17.533592208384	0\\
17.5324383206855	0\\
17.5312842796769	0\\
17.5301300853176	0\\
17.5289757375668	0\\
17.5278212363838	0\\
17.5266665817276	0\\
17.5255117735576	0\\
17.5243568118328	0\\
17.5232016965126	0\\
17.5220464275559	0\\
17.5208910049219	0\\
17.5197354285698	0\\
17.5185796984585	0\\
17.5174238145473	0\\
17.5162677767952	0\\
17.5151115851611	0\\
17.5139552396042	0\\
17.5127987400834	0\\
17.5116420865577	0\\
17.5104852789862	0\\
17.5093283173277	0\\
17.5081712015413	0\\
17.5070139315858	0\\
17.5058565074202	0\\
17.5046989290033	0\\
17.5035411962941	0\\
17.5023833092514	0\\
17.501225267834	0\\
17.5000670720008	0\\
17.4989087217106	0\\
17.4977502169222	0\\
17.4965915575944	0\\
17.495432743686	0\\
17.4942737751556	0\\
17.493114651962	0\\
17.491955374064	0\\
17.4907959414202	0\\
17.4896363539893	0\\
17.48847661173	0\\
17.487316714601	0\\
17.4861566625608	0\\
17.4849964555681	0\\
17.4838360935814	0\\
17.4826755765594	0\\
17.4815149044607	0\\
17.4803540772437	0\\
17.4791930948671	0\\
17.4780319572892	0\\
17.4768706644687	0\\
17.475709216364	0\\
17.4745476129336	0\\
17.4733858541359	0\\
17.4722239399293	0\\
17.4710618702724	0\\
17.4698996451234	0\\
17.4687372644407	0\\
17.4675747281828	0\\
17.466412036308	0\\
17.4652491887745	0\\
17.4640861855408	0\\
17.4629230265651	0\\
17.4617597118057	0\\
17.4605962412209	0\\
17.4594326147689	0\\
17.4582688324079	0\\
17.4571048940962	0\\
17.455940799792	0\\
17.4547765494534	0\\
17.4536121430386	0\\
17.4524475805058	0\\
17.451282861813	0\\
17.4501179869185	0\\
17.4489529557803	0\\
17.4477877683564	0\\
17.446622424605	0\\
17.4454569244841	0\\
17.4442912679517	0\\
17.4431254549659	0\\
17.4419594854845	0\\
17.4407933594657	0\\
17.4396270768674	0\\
17.4384606376475	0\\
17.4372940417639	0\\
17.4361272891746	0\\
17.4349603798374	0\\
17.4337933137103	0\\
17.4326260907511	0\\
17.4314587109176	0\\
17.4302911741676	0\\
17.4291234804591	0\\
17.4279556297497	0\\
17.4267876219972	0\\
17.4256194571595	0\\
17.4244511351941	0\\
17.423282656059	0\\
17.4221140197117	0\\
17.42094522611	0\\
17.4197762752115	0\\
17.4186071669739	0\\
17.4174379013549	0\\
17.4162684783119	0\\
17.4150988978028	0\\
17.413929159785	0\\
17.4127592642161	0\\
17.4115892110537	0\\
17.4104190002553	0\\
17.4092486317784	0\\
17.4080781055805	0\\
17.4069074216192	0\\
17.4057365798518	0\\
17.4045655802359	0\\
17.4033944227288	0\\
17.402223107288	0\\
17.4010516338709	0\\
17.3998800024348	0\\
17.3987082129372	0\\
17.3975362653353	0\\
17.3963641595866	0\\
17.3951918956482	0\\
17.3940194734776	0\\
17.3928468930319	0\\
17.3916741542685	0\\
17.3905012571445	0\\
17.3893282016173	0\\
17.388154987644	0\\
17.3869816151819	0\\
17.385808084188	0\\
17.3846343946195	0\\
17.3834605464337	0\\
17.3822865395875	0\\
17.3811123740382	0\\
17.3799380497427	0\\
17.3787635666582	0\\
17.3775889247418	0\\
17.3764141239503	0\\
17.375239164241	0\\
17.3740640455707	0\\
17.3728887678965	0\\
17.3717133311753	0\\
17.3705377353641	0\\
17.3693619804197	0\\
17.3681860662992	0\\
17.3670099929594	0\\
17.3658337603571	0\\
17.3646573684493	0\\
17.3634808171928	0\\
17.3623041065443	0\\
17.3611272364608	0\\
17.359950206899	0\\
17.3587730178157	0\\
17.3575956691676	0\\
17.3564181609115	0\\
17.355240493004	0\\
17.354062665402	0\\
17.352884678062	0\\
17.3517065309407	0\\
17.3505282239949	0\\
17.349349757181	0\\
17.3481711304558	0\\
17.3469923437759	0\\
17.3458133970977	0\\
17.3446342903779	0\\
17.343455023573	0\\
17.3422755966395	0\\
17.341096009534	0\\
17.339916262213	0\\
17.3387363546328	0\\
17.33755628675	0\\
17.336376058521	0\\
17.3351956699022	0\\
17.33401512085	0\\
17.3328344113208	0\\
17.331653541271	0\\
17.3304725106569	0\\
17.3292913194348	0\\
17.3281099675611	0\\
17.326928454992	0\\
17.3257467816839	0\\
17.3245649475929	0\\
17.3233829526754	0\\
17.3222007968875	0\\
17.3210184801855	0\\
17.3198360025255	0\\
17.3186533638637	0\\
17.3174705641563	0\\
17.3162876033594	0\\
17.315104481429	0\\
17.3139211983214	0\\
17.3127377539926	0\\
17.3115541483986	0\\
17.3103703814955	0\\
17.3091864532393	0\\
17.308002363586	0\\
17.3068181124917	0\\
17.3056336999122	0\\
17.3044491258036	0\\
17.3032643901217	0\\
17.3020794928226	0\\
17.300894433862	0\\
17.2997092131959	0\\
17.2985238307801	0\\
17.2973382865705	0\\
17.2961525805229	0\\
17.2949667125932	0\\
17.293780682737	0\\
17.2925944909102	0\\
17.2914081370686	0\\
17.2902216211678	0\\
17.2890349431637	0\\
17.2878481030118	0\\
17.2866611006679	0\\
17.2854739360876	0\\
17.2842866092266	0\\
17.2830991200405	0\\
17.2819114684849	0\\
17.2807236545155	0\\
17.2795356780877	0\\
17.2783475391572	0\\
17.2771592376795	0\\
17.27597077361	0\\
17.2747821469043	0\\
17.273593357518	0\\
17.2724044054063	0\\
17.2712152905249	0\\
17.270026012829	0\\
17.2688365722742	0\\
17.2676469688158	0\\
17.2664572024091	0\\
17.2652672730096	0\\
17.2640771805726	0\\
17.2628869250534	0\\
17.2616965064072	0\\
17.2605059245894	0\\
17.2593151795553	0\\
17.25812427126	0\\
17.2569331996588	0\\
17.2557419647069	0\\
17.2545505663595	0\\
17.2533590045718	0\\
17.2521672792988	0\\
17.2509753904959	0\\
17.2497833381179	0\\
17.2485911221202	0\\
17.2473987424576	0\\
17.2462061990854	0\\
17.2450134919585	0\\
17.2438206210319	0\\
17.2426275862607	0\\
17.2414343875999	0\\
17.2402410250043	0\\
17.239047498429	0\\
17.2378538078289	0\\
17.2366599531589	0\\
17.2354659343739	0\\
17.2342717514287	0\\
17.2330774042782	0\\
17.2318828928773	0\\
17.2306882171808	0\\
17.2294933771434	0\\
17.2282983727199	0\\
17.2271032038652	0\\
17.2259078705339	0\\
17.2247123726808	0\\
17.2235167102606	0\\
17.2223208832279	0\\
17.2211248915375	0\\
17.2199287351439	0\\
17.2187324140019	0\\
17.217535928066	0\\
17.2163392772908	0\\
17.2151424616308	0\\
17.2139454810408	0\\
17.2127483354751	0\\
17.2115510248883	0\\
17.2103535492349	0\\
17.2091559084694	0\\
17.2079581025463	0\\
17.2067601314199	0\\
17.2055619950447	0\\
17.2043636933751	0\\
17.2031652263656	0\\
17.2019665939704	0\\
17.2007677961439	0\\
17.1995688328404	0\\
17.1983697040143	0\\
17.1971704096199	0\\
17.1959709496113	0\\
17.1947713239429	0\\
17.193571532569	0\\
17.1923715754436	0\\
17.1911714525211	0\\
17.1899711637555	0\\
17.1887707091011	0\\
17.187570088512	0\\
17.1863693019423	0\\
17.1851683493461	0\\
17.1839672306775	0\\
17.1827659458906	0\\
17.1815644949394	0\\
17.1803628777779	0\\
17.1791610943601	0\\
17.1779591446401	0\\
17.1767570285717	0\\
17.175554746109	0\\
17.1743522972058	0\\
17.1731496818162	0\\
17.1719468998938	0\\
17.1707439513928	0\\
17.1695408362668	0\\
17.1683375544698	0\\
17.1671341059555	0\\
17.1659304906777	0\\
17.1647267085903	0\\
17.1635227596469	0\\
17.1623186438013	0\\
17.1611143610073	0\\
17.1599099112185	0\\
17.1587052943886	0\\
17.1575005104713	0\\
17.1562955594201	0\\
17.1550904411888	0\\
17.153885155731	0\\
17.1526797030002	0\\
17.1514740829499	0\\
17.1502682955338	0\\
17.1490623407053	0\\
17.147856218418	0\\
17.1466499286254	0\\
17.1454434712808	0\\
17.1442368463378	0\\
17.1430300537499	0\\
17.1418230934703	0\\
17.1406159654525	0\\
17.1394086696498	0\\
17.1382012060157	0\\
17.1369935745034	0\\
17.1357857750663	0\\
17.1345778076577	0\\
17.1333696722307	0\\
17.1321613687387	0\\
17.130952897135	0\\
17.1297442573726	0\\
17.1285354494049	0\\
17.127326473185	0\\
17.126117328666	0\\
17.1249080158011	0\\
17.1236985345434	0\\
17.122488884846	0\\
17.121279066662	0\\
17.1200690799444	0\\
17.1188589246463	0\\
17.1176486007207	0\\
17.1164381081206	0\\
17.115227446799	0\\
17.1140166167087	0\\
17.1128056178029	0\\
17.1115944500344	0\\
17.110383113356	0\\
17.1091716077207	0\\
17.1079599330814	0\\
17.1067480893908	0\\
17.1055360766018	0\\
17.1043238946673	0\\
17.1031115435399	0\\
17.1018990231724	0\\
17.1006863335177	0\\
17.0994734745283	0\\
17.0982604461571	0\\
17.0970472483567	0\\
17.0958338810797	0\\
17.0946203442788	0\\
17.0934066379067	0\\
17.0921927619159	0\\
17.090978716259	0\\
17.0897645008886	0\\
17.0885501157572	0\\
17.0873355608174	0\\
17.0861208360216	0\\
17.0849059413224	0\\
17.0836908766721	0\\
17.0824756420233	0\\
17.0812602373284	0\\
17.0800446625398	0\\
17.0788289176098	0\\
17.0776130024908	0\\
17.0763969171353	0\\
17.0751806614954	0\\
17.0739642355235	0\\
17.0727476391719	0\\
17.0715308723928	0\\
17.0703139351386	0\\
17.0690968273613	0\\
17.0678795490133	0\\
17.0666621000467	0\\
17.0654444804137	0\\
17.0642266900663	0\\
17.0630087289569	0\\
17.0617905970373	0\\
17.0605722942598	0\\
17.0593538205764	0\\
17.0581351759391	0\\
17.0569163602999	0\\
17.055697373611	0\\
17.0544782158241	0\\
17.0532588868914	0\\
17.0520393867647	0\\
17.050819715396	0\\
17.0495998727371	0\\
17.04837985874	0\\
17.0471596733565	0\\
17.0459393165385	0\\
17.0447187882377	0\\
17.0434980884061	0\\
17.0422772169952	0\\
17.041056173957	0\\
17.0398349592431	0\\
17.0386135728054	0\\
17.0373920145953	0\\
17.0361702845647	0\\
17.0349483826653	0\\
17.0337263088485	0\\
17.0325040630662	0\\
17.0312816452698	0\\
17.0300590554109	0\\
17.0288362934411	0\\
17.0276133593119	0\\
17.0263902529749	0\\
17.0251669743815	0\\
17.0239435234832	0\\
17.0227199002315	0\\
17.0214961045777	0\\
17.0202721364734	0\\
17.0190479958699	0\\
17.0178236827185	0\\
17.0165991969706	0\\
17.0153745385776	0\\
17.0141497074907	0\\
17.0129247036613	0\\
17.0116995270406	0\\
17.0104741775799	0\\
17.0092486552303	0\\
17.0080229599432	0\\
17.0067970916696	0\\
17.0055710503607	0\\
17.0043448359678	0\\
17.0031184484418	0\\
17.001891887734	0\\
17.0006651537954	0\\
16.999438246577	0\\
16.9982111660299	0\\
16.9969839121052	0\\
16.9957564847537	0\\
16.9945288839266	0\\
16.9933011095747	0\\
16.9920731616489	0\\
16.9908450401003	0\\
16.9896167448797	0\\
16.988388275938	0\\
16.987159633226	0\\
16.9859308166945	0\\
16.9847018262945	0\\
16.9834726619765	0\\
16.9822433236915	0\\
16.9810138113902	0\\
16.9797841250233	0\\
16.9785542645416	0\\
16.9773242298956	0\\
16.9760940210361	0\\
16.9748636379137	0\\
16.973633080479	0\\
16.9724023486827	0\\
16.9711714424753	0\\
16.9699403618073	0\\
16.9687091066294	0\\
16.967477676892	0\\
16.9662460725457	0\\
16.9650142935409	0\\
16.963782339828	0\\
16.9625502113575	0\\
16.9613179080798	0\\
16.9600854299454	0\\
16.9588527769045	0\\
16.9576199489075	0\\
16.9563869459048	0\\
16.9551537678466	0\\
16.9539204146833	0\\
16.9526868863651	0\\
16.9514531828422	0\\
16.950219304065	0\\
16.9489852499835	0\\
16.9477510205479	0\\
16.9465166157085	0\\
16.9452820354153	0\\
16.9440472796185	0\\
16.9428123482681	0\\
16.9415772413143	0\\
16.9403419587071	0\\
16.9391065003965	0\\
16.9378708663325	0\\
16.9366350564651	0\\
16.9353990707444	0\\
16.9341629091201	0\\
16.9329265715423	0\\
16.9316900579609	0\\
16.9304533683257	0\\
16.9292165025866	0\\
16.9279794606935	0\\
16.9267422425961	0\\
16.9255048482443	0\\
16.9242672775877	0\\
16.9230295305763	0\\
16.9217916071597	0\\
16.9205535072876	0\\
16.9193152309097	0\\
16.9180767779756	0\\
16.9168381484351	0\\
16.9155993422378	0\\
16.9143603593332	0\\
16.9131211996709	0\\
16.9118818632005	0\\
16.9106423498715	0\\
16.9094026596335	0\\
16.9081627924359	0\\
16.9069227482282	0\\
16.9056825269598	0\\
16.9044421285803	0\\
16.9032015530389	0\\
16.9019608002851	0\\
16.9007198702683	0\\
16.8994787629378	0\\
16.8982374782428	0\\
16.8969960161328	0\\
16.8957543765569	0\\
16.8945125594645	0\\
16.8932705648048	0\\
16.8920283925269	0\\
16.8907860425802	0\\
16.8895435149136	0\\
16.8883008094765	0\\
16.8870579262179	0\\
16.8858148650869	0\\
16.8845716260326	0\\
16.883328209004	0\\
16.8820846139503	0\\
16.8808408408203	0\\
16.8795968895632	0\\
16.8783527601278	0\\
16.8771084524632	0\\
16.8758639665183	0\\
16.8746193022418	0\\
16.8733744595829	0\\
16.8721294384902	0\\
16.8708842389127	0\\
16.8696388607992	0\\
16.8683933040984	0\\
16.8671475687591	0\\
16.8659016547301	0\\
16.8646555619602	0\\
16.863409290398	0\\
16.8621628399921	0\\
16.8609162106914	0\\
16.8596694024443	0\\
16.8584224151996	0\\
16.8571752489059	0\\
16.8559279035116	0\\
16.8546803789654	0\\
16.8534326752158	0\\
16.8521847922112	0\\
16.8509367299003	0\\
16.8496884882314	0\\
16.848440067153	0\\
16.8471914666135	0\\
16.8459426865613	0\\
16.8446937269448	0\\
16.8434445877123	0\\
16.8421952688122	0\\
16.8409457701928	0\\
16.8396960918023	0\\
16.838446233589	0\\
16.8371961955012	0\\
16.8359459774871	0\\
16.8346955794949	0\\
16.8334450014727	0\\
16.8321942433687	0\\
16.8309433051311	0\\
16.8296921867079	0\\
16.8284408880472	0\\
16.8271894090971	0\\
16.8259377498056	0\\
16.8246859101208	0\\
16.8234338899906	0\\
16.8221816893631	0\\
16.8209293081861	0\\
16.8196767464076	0\\
16.8184240039754	0\\
16.8171710808376	0\\
16.8159179769419	0\\
16.8146646922361	0\\
16.8134112266682	0\\
16.8121575801858	0\\
16.8109037527368	0\\
16.8096497442688	0\\
16.8083955547297	0\\
16.8071411840671	0\\
16.8058866322287	0\\
16.8046318991622	0\\
16.8033769848151	0\\
16.8021218891352	0\\
16.8008666120699	0\\
16.7996111535669	0\\
16.7983555135736	0\\
16.7970996920377	0\\
16.7958436889066	0\\
16.7945875041278	0\\
16.7933311376487	0\\
16.7920745894167	0\\
16.7908178593794	0\\
16.789560947484	0\\
16.7883038536779	0\\
16.7870465779084	0\\
16.7857891201229	0\\
16.7845314802687	0\\
16.783273658293	0\\
16.7820156541431	0\\
16.7807574677661	0\\
16.7794990991094	0\\
16.77824054812	0\\
16.7769818147451	0\\
16.7757228989319	0\\
16.7744638006274	0\\
16.7732045197788	0\\
16.7719450563331	0\\
16.7706854102374	0\\
16.7694255814386	0\\
16.7681655698837	0\\
16.7669053755198	0\\
16.7656449982938	0\\
16.7643844381525	0\\
16.763123695043	0\\
16.761862768912	0\\
16.7606016597064	0\\
16.7593403673731	0\\
16.7580788918589	0\\
16.7568172331105	0\\
16.7555553910747	0\\
16.7542933656982	0\\
16.7530311569279	0\\
16.7517687647102	0\\
16.750506188992	0\\
16.7492434297199	0\\
16.7479804868405	0\\
16.7467173603003	0\\
16.745454050046	0\\
16.7441905560241	0\\
16.7429268781812	0\\
16.7416630164637	0\\
16.7403989708182	0\\
16.739134741191	0\\
16.7378703275287	0\\
16.7366057297776	0\\
16.7353409478841	0\\
16.7340759817946	0\\
16.7328108314555	0\\
16.731545496813	0\\
16.7302799778134	0\\
16.7290142744031	0\\
16.7277483865282	0\\
16.726482314135	0\\
16.7252160571697	0\\
16.7239496155785	0\\
16.7226829893075	0\\
16.7214161783028	0\\
16.7201491825106	0\\
16.718882001877	0\\
16.7176146363479	0\\
16.7163470858695	0\\
16.7150793503877	0\\
16.7138114298485	0\\
16.712543324198	0\\
16.7112750333819	0\\
16.7100065573464	0\\
16.7087378960372	0\\
16.7074690494002	0\\
16.7062000173813	0\\
16.7049307999262	0\\
16.7036613969809	0\\
16.702391808491	0\\
16.7011220344023	0\\
16.6998520746606	0\\
16.6985819292115	0\\
16.6973115980008	0\\
16.696041080974	0\\
16.6947703780768	0\\
16.6934994892549	0\\
16.6922284144537	0\\
16.690957153619	0\\
16.6896857066961	0\\
16.6884140736306	0\\
16.6871422543681	0\\
16.6858702488539	0\\
16.6845980570335	0\\
16.6833256788524	0\\
16.6820531142559	0\\
16.6807803631893	0\\
16.6795074255981	0\\
16.6782343014276	0\\
16.676960990623	0\\
16.6756874931296	0\\
16.6744138088927	0\\
16.6731399378574	0\\
16.6718658799691	0\\
16.6705916351729	0\\
16.6693172034138	0\\
16.6680425846372	0\\
16.666767778788	0\\
16.6654927858113	0\\
16.6642176056522	0\\
16.6629422382557	0\\
16.6616666835669	0\\
16.6603909415306	0\\
16.659115012092	0\\
16.6578388951958	0\\
16.6565625907871	0\\
16.6552860988106	0\\
16.6540094192113	0\\
16.6527325519341	0\\
16.6514554969236	0\\
16.6501782541247	0\\
16.6489008234822	0\\
16.6476232049408	0\\
16.6463453984452	0\\
16.6450674039402	0\\
16.6437892213703	0\\
16.6425108506802	0\\
16.6412322918145	0\\
16.6399535447179	0\\
16.6386746093349	0\\
16.63739548561	0\\
16.6361161734878	0\\
16.6348366729128	0\\
16.6335569838293	0\\
16.632277106182	0\\
16.6309970399151	0\\
16.6297167849731	0\\
16.6284363413004	0\\
16.6271557088414	0\\
16.6258748875402	0\\
16.6245938773413	0\\
16.6233126781888	0\\
16.6220312900272	0\\
16.6207497128005	0\\
16.619467946453	0\\
16.6181859909289	0\\
16.6169038461722	0\\
16.6156215121272	0\\
16.614338988738	0\\
16.6130562759485	0\\
16.6117733737029	0\\
16.6104902819452	0\\
16.6092070006193	0\\
16.6079235296694	0\\
16.6066398690392	0\\
16.6053560186728	0\\
16.6040719785141	0\\
16.6027877485069	0\\
16.601503328595	0\\
16.6002187187224	0\\
16.5989339188327	0\\
16.5976489288699	0\\
16.5963637487775	0\\
16.5950783784995	0\\
16.5937928179794	0\\
16.592507067161	0\\
16.5912211259879	0\\
16.5899349944036	0\\
16.588648672352	0\\
16.5873621597764	0\\
16.5860754566205	0\\
16.5847885628278	0\\
16.5835014783418	0\\
16.5822142031059	0\\
16.5809267370637	0\\
16.5796390801585	0\\
16.5783512323337	0\\
16.5770631935328	0\\
16.5757749636991	0\\
16.5744865427758	0\\
16.5731979307064	0\\
16.571909127434	0\\
16.570620132902	0\\
16.5693309470536	0\\
16.5680415698319	0\\
16.5667520011801	0\\
16.5654622410414	0\\
16.5641722893589	0\\
16.5628821460757	0\\
16.5615918111349	0\\
16.5603012844795	0\\
16.5590105660526	0\\
16.5577196557971	0\\
16.556428553656	0\\
16.5551372595723	0\\
16.5538457734889	0\\
16.5525540953487	0\\
16.5512622250945	0\\
16.5499701626692	0\\
16.5486779080156	0\\
16.5473854610765	0\\
16.5460928217947	0\\
16.5447999901129	0\\
16.5435069659739	0\\
16.5422137493202	0\\
16.5409203400947	0\\
16.5396267382399	0\\
16.5383329436984	0\\
16.5370389564129	0\\
16.5357447763258	0\\
16.5344504033798	0\\
16.5331558375174	0\\
16.531861078681	0\\
16.530566126813	0\\
16.529270981856	0\\
16.5279756437524	0\\
16.5266801124444	0\\
16.5253843878745	0\\
16.5240884699851	0\\
16.5227923587183	0\\
16.5214960540165	0\\
16.520199555822	0\\
16.518902864077	0\\
16.5176059787236	0\\
16.5163088997041	0\\
16.5150116269606	0\\
16.5137141604352	0\\
16.51241650007	0\\
16.5111186458071	0\\
16.5098205975886	0\\
16.5085223553565	0\\
16.5072239190527	0\\
16.5059252886192	0\\
16.504626463998	0\\
16.503327445131	0\\
16.5020282319601	0\\
16.5007288244271	0\\
16.4994292224739	0\\
16.4981294260423	0\\
16.496829435074	0\\
16.4955292495109	0\\
16.4942288692947	0\\
16.492928294367	0\\
16.4916275246697	0\\
16.4903265601442	0\\
16.4890254007323	0\\
16.4877240463755	0\\
16.4864224970154	0\\
16.4851207525936	0\\
16.4838188130517	0\\
16.482516678331	0\\
16.4812143483731	0\\
16.4799118231194	0\\
16.4786091025113	0\\
16.4773061864902	0\\
16.4760030749976	0\\
16.4746997679746	0\\
16.4733962653627	0\\
16.4720925671032	0\\
16.4707886731372	0\\
16.469484583406	0\\
16.4681802978509	0\\
16.466875816413	0\\
16.4655711390334	0\\
16.4642662656534	0\\
16.4629611962139	0\\
16.4616559306561	0\\
16.460350468921	0\\
16.4590448109496	0\\
16.457738956683	0\\
16.4564329060621	0\\
16.4551266590278	0\\
16.4538202155211	0\\
16.4525135754828	0\\
16.4512067388538	0\\
16.449899705575	0\\
16.4485924755871	0\\
16.447285048831	0\\
16.4459774252473	0\\
16.4446696047769	0\\
16.4433615873604	0\\
16.4420533729384	0\\
16.4407449614517	0\\
16.4394363528409	0\\
16.4381275470465	0\\
16.4368185440091	0\\
16.4355093436693	0\\
16.4341999459675	0\\
16.4328903508443	0\\
16.4315805582402	0\\
16.4302705680955	0\\
16.4289603803506	0\\
16.427649994946	0\\
16.4263394118219	0\\
16.4250286309188	0\\
16.4237176521769	0\\
16.4224064755365	0\\
16.4210951009378	0\\
16.419783528321	0\\
16.4184717576263	0\\
16.417159788794	0\\
16.4158476217641	0\\
16.4145352564767	0\\
16.4132226928719	0\\
16.4119099308898	0\\
16.4105969704704	0\\
16.4092838115537	0\\
16.4079704540796	0\\
16.4066568979882	0\\
16.4053431432193	0\\
16.4040291897128	0\\
16.4027150374086	0\\
16.4014006862465	0\\
16.4000861361664	0\\
16.3987713871079	0\\
16.3974564390109	0\\
16.3961412918151	0\\
16.3948259454602	0\\
16.3935103998859	0\\
16.3921946550317	0\\
16.3908787108374	0\\
16.3895625672425	0\\
16.3882462241865	0\\
16.3869296816091	0\\
16.3856129394496	0\\
16.3842959976477	0\\
16.3829788561427	0\\
16.3816615148741	0\\
16.3803439737812	0\\
16.3790262328035	0\\
16.3777082918802	0\\
16.3763901509507	0\\
16.3750718099543	0\\
16.3737532688303	0\\
16.3724345275178	0\\
16.3711155859561	0\\
16.3697964440843	0\\
16.3684771018416	0\\
16.3671575591671	0\\
16.3658378159999	0\\
16.3645178722791	0\\
16.3631977279437	0\\
16.3618773829327	0\\
16.3605568371851	0\\
16.3592360906399	0\\
16.3579151432359	0\\
16.3565939949121	0\\
16.3552726456074	0\\
16.3539510952605	0\\
16.3526293438103	0\\
16.3513073911956	0\\
16.3499852373552	0\\
16.3486628822277	0\\
16.3473403257519	0\\
16.3460175678664	0\\
16.34469460851	0\\
16.3433714476211	0\\
16.3420480851385	0\\
16.3407245210006	0\\
16.339400755146	0\\
16.3380767875132	0\\
16.3367526180407	0\\
16.3354282466669	0\\
16.3341036733303	0\\
16.3327788979692	0\\
16.3314539205221	0\\
16.3301287409271	0\\
16.3288033591228	0\\
16.3274777750472	0\\
16.3261519886388	0\\
16.3248259998357	0\\
16.323499808576	0\\
16.3221734147981	0\\
16.3208468184399	0\\
16.3195200194397	0\\
16.3181930177355	0\\
16.3168658132653	0\\
16.3155384059673	0\\
16.3142107957793	0\\
16.3128829826393	0\\
16.3115549664853	0\\
16.3102267472553	0\\
16.308898324887	0\\
16.3075696993184	0\\
16.3062408704872	0\\
16.3049118383313	0\\
16.3035826027884	0\\
16.3022531637963	0\\
16.3009235212926	0\\
16.2995936752151	0\\
16.2982636255015	0\\
16.2969333720893	0\\
16.2956029149161	0\\
16.2942722539195	0\\
16.2929413890371	0\\
16.2916103202064	0\\
16.2902790473648	0\\
16.2889475704498	0\\
16.2876158893988	0\\
16.2862840041492	0\\
16.2849519146384	0\\
16.2836196208037	0\\
16.2822871225825	0\\
16.280954419912	0\\
16.2796215127294	0\\
16.2782884009721	0\\
16.2769550845771	0\\
16.2756215634816	0\\
16.2742878376229	0\\
16.2729539069379	0\\
16.2716197713638	0\\
16.2702854308377	0\\
16.2689508852964	0\\
16.2676161346771	0\\
16.2662811789167	0\\
16.2649460179521	0\\
16.2636106517202	0\\
16.262275080158	0\\
16.2609393032022	0\\
16.2596033207896	0\\
16.2582671328571	0\\
16.2569307393414	0\\
16.2555941401793	0\\
16.2542573353074	0\\
16.2529203246624	0\\
16.2515831081809	0\\
16.2502456857997	0\\
16.2489080574551	0\\
16.2475702230839	0\\
16.2462321826225	0\\
16.2448939360074	0\\
16.2435554831751	0\\
16.242216824062	0\\
16.2408779586045	0\\
16.2395388867391	0\\
16.238199608402	0\\
16.2368601235295	0\\
16.2355204320581	0\\
16.2341805339238	0\\
16.232840429063	0\\
16.2315001174119	0\\
16.2301595989065	0\\
16.2288188734832	0\\
16.2274779410779	0\\
16.2261368016267	0\\
16.2247954550658	0\\
16.2234539013311	0\\
16.2221121403586	0\\
16.2207701720843	0\\
16.2194279964441	0\\
16.218085613374	0\\
16.2167430228098	0\\
16.2154002246873	0\\
16.2140572189423	0\\
16.2127140055107	0\\
16.2113705843283	0\\
16.2100269553306	0\\
16.2086831184535	0\\
16.2073390736325	0\\
16.2059948208034	0\\
16.2046503599017	0\\
16.203305690863	0\\
16.2019608136229	0\\
16.2006157281168	0\\
16.1992704342803	0\\
16.1979249320488	0\\
16.1965792213577	0\\
16.1952333021425	0\\
16.1938871743385	0\\
16.192540837881	0\\
16.1911942927054	0\\
16.1898475387468	0\\
16.1885005759407	0\\
16.1871534042221	0\\
16.1858060235263	0\\
16.1844584337884	0\\
16.1831106349436	0\\
16.1817626269269	0\\
16.1804144096735	0\\
16.1790659831182	0\\
16.1777173471962	0\\
16.1763685018425	0\\
16.1750194469919	0\\
16.1736701825794	0\\
16.1723207085398	0\\
16.1709710248081	0\\
16.169621131319	0\\
16.1682710280074	0\\
16.1669207148079	0\\
16.1655701916554	0\\
16.1642194584844	0\\
16.1628685152298	0\\
16.1615173618261	0\\
16.160165998208	0\\
16.15881442431	0\\
16.1574626400666	0\\
16.1561106454125	0\\
16.154758440282	0\\
16.1534060246097	0\\
16.1520533983299	0\\
16.1507005613771	0\\
16.1493475136856	0\\
16.1479942551897	0\\
16.1466407858238	0\\
16.1452871055221	0\\
16.1439332142189	0\\
16.1425791118484	0\\
16.1412247983448	0\\
16.1398702736421	0\\
16.1385155376746	0\\
16.1371605903764	0\\
16.1358054316813	0\\
16.1344500615236	0\\
16.1330944798372	0\\
16.1317386865561	0\\
16.1303826816141	0\\
16.1290264649452	0\\
16.1276700364834	0\\
16.1263133961623	0\\
16.1249565439158	0\\
16.1235994796777	0\\
16.1222422033818	0\\
16.1208847149618	0\\
16.1195270143513	0\\
16.1181691014839	0\\
16.1168109762935	0\\
16.1154526387134	0\\
16.1140940886773	0\\
16.1127353261187	0\\
16.1113763509711	0\\
16.110017163168	0\\
16.1086577626428	0\\
16.1072981493289	0\\
16.1059383231597	0\\
16.1045782840685	0\\
16.1032180319887	0\\
16.1018575668535	0\\
16.1004968885961	0\\
16.0991359971498	0\\
16.0977748924478	0\\
16.0964135744232	0\\
16.0950520430092	0\\
16.0936902981388	0\\
16.0923283397451	0\\
16.0909661677611	0\\
16.0896037821199	0\\
16.0882411827543	0\\
16.0868783695973	0\\
16.0855153425819	0\\
16.0841521016408	0\\
16.082788646707	0\\
16.0814249777132	0\\
16.0800610945923	0\\
16.0786969972768	0\\
16.0773326856997	0\\
16.0759681597935	0\\
16.0746034194909	0\\
16.0732384647245	0\\
16.0718732954269	0\\
16.0705079115306	0\\
16.0691423129683	0\\
16.0677764996723	0\\
16.0664104715751	0\\
16.0650442286091	0\\
16.0636777707068	0\\
16.0623110978005	0\\
16.0609442098226	0\\
16.0595771067052	0\\
16.0582097883808	0\\
16.0568422547815	0\\
16.0554745058395	0\\
16.054106541487	0\\
16.0527383616562	0\\
16.0513699662791	0\\
16.0500013552879	0\\
16.0486325286145	0\\
16.0472634861909	0\\
16.0458942279493	0\\
16.0445247538214	0\\
16.0431550637392	0\\
16.0417851576347	0\\
16.0404150354395	0\\
16.0390446970856	0\\
16.0376741425048	0\\
16.0363033716287	0\\
16.0349323843892	0\\
16.0335611807179	0\\
16.0321897605464	0\\
16.0308181238064	0\\
16.0294462704294	0\\
16.0280742003471	0\\
16.0267019134909	0\\
16.0253294097924	0\\
16.023956689183	0\\
16.0225837515941	0\\
16.0212105969571	0\\
16.0198372252034	0\\
16.0184636362644	0\\
16.0170898300712	0\\
16.0157158065553	0\\
16.0143415656477	0\\
16.0129671072798	0\\
16.0115924313826	0\\
16.0102175378874	0\\
16.0088424267251	0\\
16.007467097827	0\\
16.006091551124	0\\
16.0047157865471	0\\
16.0033398040273	0\\
16.0019636034955	0\\
16.0005871848827	0\\
15.9992105481197	0\\
15.9978336931373	0\\
15.9964566198664	0\\
15.9950793282377	0\\
15.993701818182	0\\
15.99232408963	0\\
15.9909461425123	0\\
15.9895679767596	0\\
15.9881895923025	0\\
15.9868109890716	0\\
15.9854321669975	0\\
15.9840531260105	0\\
15.9826738660413	0\\
15.9812943870202	0\\
15.9799146888776	0\\
15.978534771544	0\\
15.9771546349497	0\\
15.975774279025	0\\
15.9743937037001	0\\
15.9730129089054	0\\
15.971631894571	0\\
15.970250660627	0\\
15.9688692070037	0\\
15.9674875336312	0\\
15.9661056404395	0\\
15.9647235273587	0\\
15.9633411943188	0\\
15.9619586412497	0\\
15.9605758680815	0\\
15.9591928747439	0\\
15.957809661167	0\\
15.9564262272805	0\\
15.9550425730142	0\\
15.953658698298	0\\
15.9522746030615	0\\
15.9508902872345	0\\
15.9495057507467	0\\
15.9481209935276	0\\
15.946736015507	0\\
15.9453508166143	0\\
15.9439653967792	0\\
15.942579755931	0\\
15.9411938939994	0\\
15.9398078109137	0\\
15.9384215066034	0\\
15.9370349809977	0\\
15.9356482340262	0\\
15.9342612656179	0\\
15.9328740757023	0\\
15.9314866642086	0\\
15.9300990310659	0\\
15.9287111762035	0\\
15.9273230995504	0\\
15.9259348010358	0\\
15.9245462805888	0\\
15.9231575381383	0\\
15.9217685736134	0\\
15.920379386943	0\\
15.9189899780561	0\\
15.9176003468815	0\\
15.9162104933482	0\\
15.9148204173849	0\\
15.9134301189204	0\\
15.9120395978836	0\\
15.9106488542031	0\\
15.9092578878076	0\\
15.9078666986258	0\\
15.9064752865863	0\\
15.9050836516177	0\\
15.9036917936485	0\\
15.9022997126073	0\\
15.9009074084225	0\\
15.8995148810226	0\\
15.8981221303361	0\\
15.8967291562912	0\\
15.8953359588164	0\\
15.89394253784	0\\
15.8925488932902	0\\
15.8911550250952	0\\
15.8897609331834	0\\
15.8883666174828	0\\
15.8869720779217	0\\
15.885577314428	0\\
15.8841823269299	0\\
15.8827871153555	0\\
15.8813916796326	0\\
15.8799960196894	0\\
15.8786001354537	0\\
15.8772040268534	0\\
15.8758076938164	0\\
15.8744111362705	0\\
15.8730143541435	0\\
15.8716173473631	0\\
15.8702201158572	0\\
15.8688226595533	0\\
15.8674249783791	0\\
15.8660270722623	0\\
15.8646289411304	0\\
15.863230584911	0\\
15.8618320035316	0\\
15.8604331969197	0\\
15.8590341650026	0\\
15.8576349077079	0\\
15.8562354249629	0\\
15.854835716695	0\\
15.8534357828313	0\\
15.8520356232993	0\\
15.8506352380262	0\\
15.849234626939	0\\
15.8478337899651	0\\
15.8464327270315	0\\
15.8450314380653	0\\
15.8436299229936	0\\
15.8422281817434	0\\
15.8408262142416	0\\
15.8394240204154	0\\
15.8380216001914	0\\
15.8366189534967	0\\
15.8352160802581	0\\
15.8338129804024	0\\
15.8324096538563	0\\
15.8310061005466	0\\
15.8296023204	0\\
15.8281983133432	0\\
15.8267940793028	0\\
15.8253896182054	0\\
15.8239849299775	0\\
15.8225800145457	0\\
15.8211748718365	0\\
15.8197695017763	0\\
15.8183639042916	0\\
15.8169580793086	0\\
15.8155520267539	0\\
15.8141457465536	0\\
15.8127392386341	0\\
15.8113325029215	0\\
15.8099255393422	0\\
15.8085183478222	0\\
15.8071109282877	0\\
15.8057032806648	0\\
15.8042954048796	0\\
15.802887300858	0\\
15.8014789685261	0\\
15.8000704078098	0\\
15.798661618635	0\\
15.7972526009276	0\\
15.7958433546135	0\\
15.7944338796184	0\\
15.7930241758682	0\\
15.7916142432885	0\\
15.7902040818051	0\\
15.7887936913436	0\\
15.7873830718296	0\\
15.7859722231888	0\\
15.7845611453467	0\\
15.7831498382288	0\\
15.7817383017606	0\\
15.7803265358675	0\\
15.778914540475	0\\
15.7775023155083	0\\
15.776089860893	0\\
15.7746771765541	0\\
15.7732642624171	0\\
15.7718511184072	0\\
15.7704377444495	0\\
15.7690241404691	0\\
15.7676103063913	0\\
15.766196242141	0\\
15.7647819476434	0\\
15.7633674228234	0\\
15.7619526676059	0\\
15.7605376819161	0\\
15.7591224656786	0\\
15.7577070188185	0\\
15.7562913412605	0\\
15.7548754329293	0\\
15.7534592937499	0\\
15.7520429236468	0\\
15.7506263225447	0\\
15.7492094903683	0\\
15.7477924270422	0\\
15.746375132491	0\\
15.7449576066391	0\\
15.7435398494111	0\\
15.7421218607314	0\\
15.7407036405245	0\\
15.7392851887147	0\\
15.7378665052263	0\\
15.7364475899837	0\\
15.7350284429112	0\\
15.733609063933	0\\
15.7321894529732	0\\
15.730769609956	0\\
15.7293495348057	0\\
15.7279292274461	0\\
15.7265086878014	0\\
15.7250879157956	0\\
15.7236669113527	0\\
15.7222456743965	0\\
15.7208242048511	0\\
15.7194025026402	0\\
15.7179805676877	0\\
15.7165583999173	0\\
15.7151359992529	0\\
15.713713365618	0\\
15.7122904989365	0\\
15.7108673991319	0\\
15.7094440661278	0\\
15.7080204998478	0\\
15.7065967002153	0\\
15.705172667154	0\\
15.7037484005872	0\\
15.7023239004384	0\\
15.7008991666309	0\\
15.699474199088	0\\
15.6980489977331	0\\
15.6966235624894	0\\
15.6951978932801	0\\
15.6937719900284	0\\
15.6923458526575	0\\
15.6909194810905	0\\
15.6894928752503	0\\
15.6880660350602	0\\
15.686638960443	0\\
15.6852116513217	0\\
15.6837841076192	0\\
15.6823563292585	0\\
15.6809283161623	0\\
15.6795000682534	0\\
15.6780715854547	0\\
15.6766428676888	0\\
15.6752139148785	0\\
15.6737847269463	0\\
15.672355303815	0\\
15.670925645407	0\\
15.6694957516449	0\\
15.6680656224513	0\\
15.6666352577485	0\\
15.665204657459	0\\
15.6637738215052	0\\
15.6623427498093	0\\
15.6609114422939	0\\
15.659479898881	0\\
15.6580481194929	0\\
15.6566161040518	0\\
15.6551838524799	0\\
15.6537513646993	0\\
15.6523186406321	0\\
15.6508856802002	0\\
15.6494524833257	0\\
15.6480190499306	0\\
15.6465853799367	0\\
15.645151473266	0\\
15.6437173298403	0\\
15.6422829495814	0\\
15.6408483324111	0\\
15.639413478251	0\\
15.637978387023	0\\
15.6365430586486	0\\
15.6351074930494	0\\
15.633671690147	0\\
15.632235649863	0\\
15.6307993721188	0\\
15.629362856836	0\\
15.6279261039358	0\\
15.6264891133398	0\\
15.6250518849691	0\\
15.6236144187452	0\\
15.6221767145893	0\\
15.6207387724226	0\\
15.6193005921663	0\\
15.6178621737416	0\\
15.6164235170695	0\\
15.614984622071	0\\
15.6135454886674	0\\
15.6121061167794	0\\
15.6106665063281	0\\
15.6092266572344	0\\
15.6077865694191	0\\
15.606346242803	0\\
15.6049056773071	0\\
15.6034648728519	0\\
15.6020238293582	0\\
15.6005825467468	0\\
15.5991410249381	0\\
15.5976992638529	0\\
15.5962572634116	0\\
15.5948150235348	0\\
15.593372544143	0\\
15.5919298251565	0\\
15.5904868664958	0\\
15.5890436680812	0\\
15.5876002298331	0\\
15.5861565516717	0\\
15.5847126335173	0\\
15.58326847529	0\\
15.58182407691	0\\
15.5803794382974	0\\
15.5789345593723	0\\
15.5774894400548	0\\
15.5760440802648	0\\
15.5745984799222	0\\
15.5731526389471	0\\
15.5717065572592	0\\
15.5702602347785	0\\
15.5688136714247	0\\
15.5673668671175	0\\
15.5659198217767	0\\
15.5644725353219	0\\
15.5630250076729	0\\
};
\addplot [color=mycolor1, forget plot]
  table[row sep=crcr]{%
15.5630250076729	0\\
15.5615772387491	0\\
15.5601292284702	0\\
15.5586809767556	0\\
15.5572324835248	0\\
15.5557837486973	0\\
15.5543347721925	0\\
15.5528855539297	0\\
15.5514360938282	0\\
15.5499863918073	0\\
15.5485364477862	0\\
15.5470862616842	0\\
15.5456358334203	0\\
15.5441851629137	0\\
15.5427342500835	0\\
15.5412830948486	0\\
15.5398316971281	0\\
15.5383800568409	0\\
15.536928173906	0\\
15.5354760482421	0\\
15.5340236797682	0\\
15.532571068403	0\\
15.5311182140652	0\\
15.5296651166736	0\\
15.5282117761468	0\\
15.5267581924035	0\\
15.5253043653622	0\\
15.5238502949414	0\\
15.5223959810598	0\\
15.5209414236356	0\\
15.5194866225874	0\\
15.5180315778335	0\\
15.5165762892922	0\\
15.515120756882	0\\
15.5136649805209	0\\
15.5122089601272	0\\
15.5107526956191	0\\
15.5092961869148	0\\
15.5078394339322	0\\
15.5063824365896	0\\
15.5049251948047	0\\
15.5034677084957	0\\
15.5020099775805	0\\
15.5005520019769	0\\
15.4990937816028	0\\
15.4976353163759	0\\
15.4961766062141	0\\
15.4947176510351	0\\
15.4932584507564	0\\
15.4917990052959	0\\
15.490339314571	0\\
15.4888793784993	0\\
15.4874191969983	0\\
15.4859587699855	0\\
15.4844980973784	0\\
15.4830371790942	0\\
15.4815760150504	0\\
15.4801146051642	0\\
15.4786529493529	0\\
15.4771910475337	0\\
15.4757288996239	0\\
15.4742665055404	0\\
15.4728038652005	0\\
15.4713409785212	0\\
15.4698778454194	0\\
15.4684144658122	0\\
15.4669508396165	0\\
15.4654869667491	0\\
15.4640228471269	0\\
15.4625584806667	0\\
15.4610938672853	0\\
15.4596290068993	0\\
15.4581638994254	0\\
15.4566985447804	0\\
15.4552329428806	0\\
15.4537670936428	0\\
15.4523009969834	0\\
15.4508346528189	0\\
15.4493680610656	0\\
15.44790122164	0\\
15.4464341344584	0\\
15.4449667994371	0\\
15.4434992164923	0\\
15.4420313855402	0\\
15.4405633064971	0\\
15.439094979279	0\\
15.437626403802	0\\
15.4361575799821	0\\
15.4346885077354	0\\
15.4332191869777	0\\
15.4317496176251	0\\
15.4302797995933	0\\
15.4288097327982	0\\
15.4273394171556	0\\
15.4258688525812	0\\
15.4243980389907	0\\
15.4229269762997	0\\
15.4214556644239	0\\
15.4199841032788	0\\
15.41851229278	0\\
15.4170402328429	0\\
15.415567923383	0\\
15.4140953643156	0\\
15.4126225555561	0\\
15.4111494970199	0\\
15.4096761886222	0\\
15.4082026302781	0\\
15.406728821903	0\\
15.4052547634119	0\\
15.4037804547199	0\\
15.402305895742	0\\
15.4008310863934	0\\
15.3993560265888	0\\
15.3978807162434	0\\
15.3964051552718	0\\
15.3949293435891	0\\
15.3934532811098	0\\
15.3919769677489	0\\
15.390500403421	0\\
15.3890235880408	0\\
15.3875465215228	0\\
15.3860692037816	0\\
15.3845916347319	0\\
15.383113814288	0\\
15.3816357423644	0\\
15.3801574188755	0\\
15.3786788437356	0\\
15.3772000168591	0\\
15.3757209381603	0\\
15.3742416075533	0\\
15.3727620249523	0\\
15.3712821902715	0\\
15.3698021034249	0\\
15.3683217643266	0\\
15.3668411728907	0\\
15.365360329031	0\\
15.3638792326614	0\\
15.3623978836959	0\\
15.3609162820483	0\\
15.3594344276324	0\\
15.3579523203618	0\\
15.3564699601503	0\\
15.3549873469116	0\\
15.3535044805592	0\\
15.3520213610067	0\\
15.3505379881676	0\\
15.3490543619555	0\\
15.3475704822836	0\\
15.3460863490655	0\\
15.3446019622144	0\\
15.3431173216436	0\\
15.3416324272664	0\\
15.3401472789961	0\\
15.3386618767457	0\\
15.3371762204284	0\\
15.3356903099572	0\\
15.3342041452452	0\\
15.3327177262054	0\\
15.3312310527506	0\\
15.3297441247939	0\\
15.328256942248	0\\
15.3267695050257	0\\
15.3252818130399	0\\
15.3237938662032	0\\
15.3223056644283	0\\
15.3208172076278	0\\
15.3193284957143	0\\
15.3178395286003	0\\
15.3163503061984	0\\
15.3148608284209	0\\
15.3133710951803	0\\
15.3118811063889	0\\
15.3103908619591	0\\
15.308900361803	0\\
15.307409605833	0\\
15.3059185939611	0\\
15.3044273260996	0\\
15.3029358021605	0\\
15.3014440220558	0\\
15.2999519856976	0\\
15.2984596929978	0\\
15.2969671438682	0\\
15.2954743382208	0\\
15.2939812759674	0\\
15.2924879570196	0\\
15.2909943812893	0\\
15.2895005486882	0\\
15.2880064591278	0\\
15.2865121125197	0\\
15.2850175087755	0\\
15.2835226478066	0\\
15.2820275295246	0\\
15.2805321538407	0\\
15.2790365206665	0\\
15.2775406299131	0\\
15.2760444814919	0\\
15.274548075314	0\\
15.2730514112906	0\\
15.2715544893329	0\\
15.2700573093519	0\\
15.2685598712587	0\\
15.2670621749643	0\\
15.2655642203796	0\\
15.2640660074154	0\\
15.2625675359827	0\\
15.2610688059923	0\\
15.2595698173549	0\\
15.2580705699811	0\\
15.2565710637818	0\\
15.2550712986675	0\\
15.2535712745487	0\\
15.2520709913361	0\\
15.25057044894	0\\
15.2490696472709	0\\
15.2475685862393	0\\
15.2460672657554	0\\
15.2445656857295	0\\
15.2430638460719	0\\
15.2415617466928	0\\
15.2400593875023	0\\
15.2385567684106	0\\
15.2370538893277	0\\
15.2355507501636	0\\
15.2340473508283	0\\
15.2325436912317	0\\
15.2310397712836	0\\
15.229535590894	0\\
15.2280311499726	0\\
15.2265264484291	0\\
15.2250214861733	0\\
15.2235162631146	0\\
15.2220107791628	0\\
15.2205050342275	0\\
15.2189990282179	0\\
15.2174927610438	0\\
15.2159862326144	0\\
15.214479442839	0\\
15.2129723916271	0\\
15.2114650788879	0\\
15.2099575045305	0\\
15.2084496684642	0\\
15.2069415705981	0\\
15.2054332108413	0\\
15.2039245891027	0\\
15.2024157052914	0\\
15.2009065593162	0\\
15.1993971510861	0\\
15.19788748051	0\\
15.1963775474965	0\\
15.1948673519545	0\\
15.1933568937926	0\\
15.1918461729195	0\\
15.1903351892438	0\\
15.188823942674	0\\
15.1873124331186	0\\
15.1858006604861	0\\
15.1842886246849	0\\
15.1827763256233	0\\
15.1812637632097	0\\
15.1797509373524	0\\
15.1782378479595	0\\
15.1767244949392	0\\
15.1752108781996	0\\
15.1736969976488	0\\
15.1721828531949	0\\
15.1706684447457	0\\
15.1691537722093	0\\
15.1676388354935	0\\
15.1661236345061	0\\
15.164608169155	0\\
15.1630924393478	0\\
15.1615764449923	0\\
15.160060185996	0\\
15.1585436622666	0\\
15.1570268737116	0\\
15.1555098202385	0\\
15.1539925017548	0\\
15.1524749181678	0\\
15.1509570693849	0\\
15.1494389553134	0\\
15.1479205758605	0\\
15.1464019309334	0\\
15.1448830204394	0\\
15.1433638442855	0\\
15.1418444023787	0\\
15.140324694626	0\\
15.1388047209345	0\\
15.137284481211	0\\
15.1357639753624	0\\
15.1342432032954	0\\
15.132722164917	0\\
15.1312008601337	0\\
15.1296792888522	0\\
15.1281574509792	0\\
15.1266353464212	0\\
15.1251129750847	0\\
15.1235903368762	0\\
15.1220674317021	0\\
15.1205442594688	0\\
15.1190208200826	0\\
15.1174971134498	0\\
15.1159731394766	0\\
15.1144488980692	0\\
15.1129243891336	0\\
15.111399612576	0\\
15.1098745683024	0\\
15.1083492562187	0\\
15.106823676231	0\\
15.1052978282449	0\\
15.1037717121665	0\\
15.1022453279014	0\\
15.1007186753554	0\\
15.0991917544342	0\\
15.0976645650434	0\\
15.0961371070885	0\\
15.0946093804751	0\\
15.0930813851086	0\\
15.0915531208946	0\\
15.0900245877383	0\\
15.0884957855452	0\\
15.0869667142204	0\\
15.0854373736692	0\\
15.0839077637968	0\\
15.0823778845083	0\\
15.0808477357087	0\\
15.0793173173032	0\\
15.0777866291967	0\\
15.076255671294	0\\
15.0747244435002	0\\
15.07319294572	0\\
15.0716611778581	0\\
15.0701291398194	0\\
15.0685968315085	0\\
15.0670642528299	0\\
15.0655314036884	0\\
15.0639982839883	0\\
15.0624648936343	0\\
15.0609312325306	0\\
15.0593973005817	0\\
15.0578630976919	0\\
15.0563286237654	0\\
15.0547938787066	0\\
15.0532588624194	0\\
15.0517235748082	0\\
15.0501880157769	0\\
15.0486521852295	0\\
15.04711608307	0\\
15.0455797092024	0\\
15.0440430635304	0\\
15.042506145958	0\\
15.0409689563888	0\\
15.0394314947265	0\\
15.0378937608749	0\\
15.0363557547376	0\\
15.034817476218	0\\
15.0332789252197	0\\
15.0317401016462	0\\
15.0302010054008	0\\
15.0286616363869	0\\
15.0271219945079	0\\
15.0255820796668	0\\
15.0240418917671	0\\
15.0225014307117	0\\
15.0209606964038	0\\
15.0194196887464	0\\
15.0178784076425	0\\
15.0163368529951	0\\
15.014795024707	0\\
15.0132529226811	0\\
15.0117105468202	0\\
15.0101678970269	0\\
15.008624973204	0\\
15.0070817752542	0\\
15.0055383030799	0\\
15.0039945565836	0\\
15.002450535668	0\\
15.0009062402354	0\\
14.9993616701881	0\\
14.9978168254284	0\\
14.9962717058588	0\\
14.9947263113812	0\\
14.993180641898	0\\
14.9916346973112	0\\
14.9900884775228	0\\
14.988541982435	0\\
14.9869952119495	0\\
14.9854481659684	0\\
14.9839008443934	0\\
14.9823532471264	0\\
14.9808053740691	0\\
14.9792572251232	0\\
14.9777088001903	0\\
14.976160099172	0\\
14.9746111219699	0\\
14.9730618684853	0\\
14.9715123386198	0\\
14.9699625322747	0\\
14.9684124493514	0\\
14.966862089751	0\\
14.9653114533748	0\\
14.963760540124	0\\
14.9622093498997	0\\
14.9606578826029	0\\
14.9591061381346	0\\
14.9575541163958	0\\
14.9560018172874	0\\
14.9544492407102	0\\
14.952896386565	0\\
14.9513432547525	0\\
14.9497898451735	0\\
14.9482361577285	0\\
14.9466821923181	0\\
14.9451279488429	0\\
14.9435734272033	0\\
14.9420186272997	0\\
14.9404635490326	0\\
14.9389081923021	0\\
14.9373525570086	0\\
14.9357966430522	0\\
14.9342404503332	0\\
14.9326839787516	0\\
14.9311272282074	0\\
14.9295701986006	0\\
14.9280128898312	0\\
14.9264553017991	0\\
14.924897434404	0\\
14.9233392875457	0\\
14.921780861124	0\\
14.9202221550385	0\\
14.9186631691889	0\\
14.9171039034746	0\\
14.9155443577952	0\\
14.9139845320501	0\\
14.9124244261388	0\\
14.9108640399605	0\\
14.9093033734145	0\\
14.9077424264002	0\\
14.9061811988166	0\\
14.9046196905628	0\\
14.903057901538	0\\
14.9014958316411	0\\
14.8999334807712	0\\
14.8983708488271	0\\
14.8968079357076	0\\
14.8952447413116	0\\
14.8936812655377	0\\
14.8921175082848	0\\
14.8905534694513	0\\
14.888989148936	0\\
14.8874245466372	0\\
14.8858596624535	0\\
14.8842944962833	0\\
14.8827290480249	0\\
14.8811633175767	0\\
14.8795973048369	0\\
14.8780310097036	0\\
14.876464432075	0\\
14.8748975718492	0\\
14.8733304289243	0\\
14.8717630031981	0\\
14.8701952945686	0\\
14.8686273029337	0\\
14.8670590281911	0\\
14.8654904702387	0\\
14.863921628974	0\\
14.8623525042948	0\\
14.8607830960987	0\\
14.8592134042831	0\\
14.8576434287455	0\\
14.8560731693833	0\\
14.854502626094	0\\
14.8529317987747	0\\
14.8513606873228	0\\
14.8497892916355	0\\
14.8482176116099	0\\
14.846645647143	0\\
14.8450733981319	0\\
14.8435008644735	0\\
14.8419280460649	0\\
14.8403549428028	0\\
14.838781554584	0\\
14.8372078813053	0\\
14.8356339228633	0\\
14.8340596791548	0\\
14.8324851500762	0\\
14.8309103355242	0\\
14.8293352353951	0\\
14.8277598495854	0\\
14.8261841779914	0\\
14.8246082205094	0\\
14.8230319770357	0\\
14.8214554474664	0\\
14.8198786316977	0\\
14.8183015296257	0\\
14.8167241411462	0\\
14.8151464661554	0\\
14.8135685045491	0\\
14.8119902562231	0\\
14.8104117210733	0\\
14.8088328989953	0\\
14.8072537898849	0\\
14.8056743936376	0\\
14.804094710149	0\\
14.8025147393146	0\\
14.8009344810299	0\\
14.7993539351902	0\\
14.7977731016909	0\\
14.7961919804272	0\\
14.7946105712944	0\\
14.7930288741875	0\\
14.7914468890018	0\\
14.7898646156323	0\\
14.7882820539739	0\\
14.7866992039216	0\\
14.7851160653702	0\\
14.7835326382146	0\\
14.7819489223495	0\\
14.7803649176696	0\\
14.7787806240696	0\\
14.777196041444	0\\
14.7756111696874	0\\
14.7740260086942	0\\
14.7724405583589	0\\
14.7708548185757	0\\
14.769268789239	0\\
14.7676824702431	0\\
14.7660958614821	0\\
14.7645089628501	0\\
14.7629217742412	0\\
14.7613342955494	0\\
14.7597465266686	0\\
14.7581584674928	0\\
14.7565701179157	0\\
14.7549814778311	0\\
14.7533925471328	0\\
14.7518033257144	0\\
14.7502138134695	0\\
14.7486240102916	0\\
14.7470339160743	0\\
14.7454435307109	0\\
14.7438528540947	0\\
14.7422618861192	0\\
14.7406706266776	0\\
14.7390790756629	0\\
14.7374872329685	0\\
14.7358950984872	0\\
14.7343026721122	0\\
14.7327099537364	0\\
14.7311169432527	0\\
14.7295236405539	0\\
14.7279300455328	0\\
14.7263361580822	0\\
14.7247419780946	0\\
14.7231475054626	0\\
14.7215527400789	0\\
14.7199576818359	0\\
14.7183623306259	0\\
14.7167666863415	0\\
14.7151707488748	0\\
14.7135745181181	0\\
14.7119779939636	0\\
14.7103811763034	0\\
14.7087840650296	0\\
14.7071866600342	0\\
14.7055889612091	0\\
14.7039909684463	0\\
14.7023926816374	0\\
14.7007941006744	0\\
14.6991952254489	0\\
14.6975960558526	0\\
14.6959965917769	0\\
14.6943968331136	0\\
14.692796779754	0\\
14.6911964315895	0\\
14.6895957885115	0\\
14.6879948504113	0\\
14.6863936171801	0\\
14.6847920887091	0\\
14.6831902648893	0\\
14.6815881456119	0\\
14.6799857307677	0\\
14.6783830202478	0\\
14.676780013943	0\\
14.6751767117441	0\\
14.6735731135418	0\\
14.6719692192268	0\\
14.6703650286898	0\\
14.6687605418212	0\\
14.6671557585117	0\\
14.6655506786516	0\\
14.6639453021314	0\\
14.6623396288413	0\\
14.6607336586716	0\\
14.6591273915125	0\\
14.6575208272541	0\\
14.6559139657866	0\\
14.6543068069999	0\\
14.6526993507839	0\\
14.6510915970286	0\\
14.6494835456239	0\\
14.6478751964594	0\\
14.6462665494249	0\\
14.64465760441	0\\
14.6430483613043	0\\
14.6414388199973	0\\
14.6398289803786	0\\
14.6382188423374	0\\
14.6366084057632	0\\
14.6349976705453	0\\
14.6333866365727	0\\
14.6317753037348	0\\
14.6301636719205	0\\
14.628551741019	0\\
14.6269395109191	0\\
14.6253269815098	0\\
14.62371415268	0\\
14.6221010243184	0\\
14.6204875963138	0\\
14.6188738685547	0\\
14.6172598409299	0\\
14.6156455133278	0\\
14.6140308856369	0\\
14.6124159577457	0\\
14.6108007295424	0\\
14.6091852009154	0\\
14.6075693717529	0\\
14.605953241943	0\\
14.6043368113739	0\\
14.6027200799336	0\\
14.60110304751	0\\
14.5994857139911	0\\
14.5978680792648	0\\
14.5962501432187	0\\
14.5946319057407	0\\
14.5930133667184	0\\
14.5913945260394	0\\
14.5897753835912	0\\
14.5881559392614	0\\
14.5865361929372	0\\
14.5849161445061	0\\
14.5832957938554	0\\
14.5816751408722	0\\
14.5800541854438	0\\
14.5784329274572	0\\
14.5768113667994	0\\
14.5751895033575	0\\
14.5735673370183	0\\
14.5719448676686	0\\
14.5703220951953	0\\
14.5686990194851	0\\
14.5670756404246	0\\
14.5654519579003	0\\
14.5638279717989	0\\
14.5622036820068	0\\
14.5605790884104	0\\
14.5589541908959	0\\
14.5573289893498	0\\
14.5557034836581	0\\
14.5540776737071	0\\
14.5524515593827	0\\
14.5508251405711	0\\
14.5491984171582	0\\
14.5475713890298	0\\
14.5459440560717	0\\
14.5443164181699	0\\
14.5426884752098	0\\
14.5410602270771	0\\
14.5394316736575	0\\
14.5378028148364	0\\
14.5361736504993	0\\
14.5345441805314	0\\
14.5329144048182	0\\
14.5312843232449	0\\
14.5296539356966	0\\
14.5280232420584	0\\
14.5263922422155	0\\
14.5247609360528	0\\
14.5231293234551	0\\
14.5214974043074	0\\
14.5198651784944	0\\
14.518232645901	0\\
14.5165998064116	0\\
14.514966659911	0\\
14.5133332062836	0\\
14.5116994454139	0\\
14.5100653771863	0\\
14.5084310014852	0\\
14.5067963181947	0\\
14.5051613271992	0\\
14.5035260283827	0\\
14.5018904216294	0\\
14.5002545068231	0\\
14.498618283848	0\\
14.4969817525877	0\\
14.4953449129262	0\\
14.4937077647472	0\\
14.4920703079343	0\\
14.4904325423713	0\\
14.4887944679415	0\\
14.4871560845285	0\\
14.4855173920158	0\\
14.4838783902866	0\\
14.4822390792242	0\\
14.480599458712	0\\
14.4789595286329	0\\
14.4773192888701	0\\
14.4756787393066	0\\
14.4740378798254	0\\
14.4723967103093	0\\
14.4707552306411	0\\
14.4691134407037	0\\
14.4674713403797	0\\
14.4658289295517	0\\
14.4641862081022	0\\
14.4625431759138	0\\
14.4608998328689	0\\
14.4592561788498	0\\
14.4576122137388	0\\
14.4559679374181	0\\
14.4543233497699	0\\
14.4526784506762	0\\
14.4510332400192	0\\
14.4493877176806	0\\
14.4477418835425	0\\
14.4460957374866	0\\
14.4444492793946	0\\
14.4428025091483	0\\
14.4411554266293	0\\
14.4395080317191	0\\
14.4378603242991	0\\
14.4362123042509	0\\
14.4345639714558	0\\
14.4329153257949	0\\
14.4312663671496	0\\
14.429617095401	0\\
14.4279675104301	0\\
14.426317612118	0\\
14.4246674003455	0\\
14.4230168749937	0\\
14.4213660359432	0\\
14.4197148830748	0\\
14.4180634162691	0\\
14.4164116354069	0\\
14.4147595403685	0\\
14.4131071310345	0\\
14.4114544072852	0\\
14.409801369001	0\\
14.4081480160622	0\\
14.4064943483488	0\\
14.4048403657411	0\\
14.4031860681191	0\\
14.4015314553628	0\\
14.3998765273521	0\\
14.3982212839668	0\\
14.3965657250867	0\\
14.3949098505915	0\\
14.3932536603609	0\\
14.3915971542745	0\\
14.3899403322116	0\\
14.3882831940519	0\\
14.3866257396745	0\\
14.3849679689589	0\\
14.3833098817843	0\\
14.3816514780297	0\\
14.3799927575744	0\\
14.3783337202972	0\\
14.3766743660772	0\\
14.3750146947933	0\\
14.3733547063242	0\\
14.3716944005487	0\\
14.3700337773455	0\\
14.3683728365931	0\\
14.3667115781701	0\\
14.365050001955	0\\
14.3633881078261	0\\
14.3617258956618	0\\
14.3600633653404	0\\
14.3584005167399	0\\
14.3567373497385	0\\
14.3550738642143	0\\
14.3534100600452	0\\
14.3517459371092	0\\
14.350081495284	0\\
14.3484167344475	0\\
14.3467516544773	0\\
14.345086255251	0\\
14.3434205366462	0\\
14.3417544985404	0\\
14.3400881408109	0\\
14.3384214633352	0\\
14.3367544659905	0\\
14.3350871486539	0\\
14.3334195112027	0\\
14.3317515535138	0\\
14.3300832754643	0\\
14.3284146769311	0\\
14.326745757791	0\\
14.3250765179207	0\\
14.3234069571971	0\\
14.3217370754966	0\\
14.320066872696	0\\
14.3183963486716	0\\
14.3167255032999	0\\
14.3150543364572	0\\
14.3133828480198	0\\
14.3117110378639	0\\
14.3100389058657	0\\
14.3083664519011	0\\
14.3066936758463	0\\
14.305020577577	0\\
14.3033471569692	0\\
14.3016734138985	0\\
14.2999993482408	0\\
14.2983249598717	0\\
14.2966502486667	0\\
14.2949752145012	0\\
14.2932998572507	0\\
14.2916241767906	0\\
14.2899481729961	0\\
14.2882718457424	0\\
14.2865951949047	0\\
14.2849182203579	0\\
14.2832409219771	0\\
14.2815632996371	0\\
14.2798853532129	0\\
14.2782070825791	0\\
14.2765284876105	0\\
14.2748495681817	0\\
14.2731703241671	0\\
14.2714907554414	0\\
14.2698108618788	0\\
14.2681306433538	0\\
14.2664500997406	0\\
14.2647692309132	0\\
14.263088036746	0\\
14.2614065171128	0\\
14.2597246718877	0\\
14.2580425009445	0\\
14.256360004157	0\\
14.254677181399	0\\
14.2529940325442	0\\
14.2513105574662	0\\
14.2496267560384	0\\
14.2479426281343	0\\
14.2462581736273	0\\
14.2445733923907	0\\
14.2428882842977	0\\
14.2412028492215	0\\
14.2395170870351	0\\
14.2378309976116	0\\
14.2361445808238	0\\
14.2344578365447	0\\
14.232770764647	0\\
14.2310833650034	0\\
14.2293956374866	0\\
14.227707581969	0\\
14.2260191983233	0\\
14.2243304864218	0\\
14.2226414461368	0\\
14.2209520773407	0\\
14.2192623799055	0\\
14.2175723537035	0\\
14.2158819986065	0\\
14.2141913144867	0\\
14.2125003012159	0\\
14.2108089586659	0\\
14.2091172867085	0\\
14.2074252852153	0\\
14.2057329540578	0\\
14.2040402931077	0\\
14.2023473022363	0\\
14.2006539813151	0\\
14.1989603302152	0\\
14.197266348808	0\\
14.1955720369645	0\\
14.1938773945558	0\\
14.192182421453	0\\
14.1904871175268	0\\
14.1887914826482	0\\
14.1870955166879	0\\
14.1853992195166	0\\
14.1837025910049	0\\
14.1820056310233	0\\
14.1803083394423	0\\
14.1786107161323	0\\
14.1769127609636	0\\
14.1752144738063	0\\
14.1735158545307	0\\
14.1718169030069	0\\
14.1701176191047	0\\
14.1684180026943	0\\
14.1667180536453	0\\
14.1650177718276	0\\
14.1633171571108	0\\
14.1616162093646	0\\
14.1599149284586	0\\
14.1582133142621	0\\
14.1565113666446	0\\
14.1548090854754	0\\
14.1531064706237	0\\
14.1514035219587	0\\
14.1497002393495	0\\
14.147996622665	0\\
14.1462926717742	0\\
14.1445883865459	0\\
14.1428837668489	0\\
14.1411788125519	0\\
14.1394735235236	0\\
14.1377678996324	0\\
14.1360619407468	0\\
14.1343556467352	0\\
14.1326490174659	0\\
14.1309420528071	0\\
14.1292347526271	0\\
14.1275271167938	0\\
14.1258191451753	0\\
14.1241108376394	0\\
14.1224021940541	0\\
14.120693214287	0\\
14.1189838982059	0\\
14.1172742456784	0\\
14.115564256572	0\\
14.1138539307541	0\\
14.1121432680921	0\\
14.1104322684533	0\\
14.108720931705	0\\
14.1070092577142	0\\
14.1052972463481	0\\
14.1035848974735	0\\
14.1018722109575	0\\
14.1001591866667	0\\
14.098445824468	0\\
14.0967321242281	0\\
14.0950180858134	0\\
14.0933037090906	0\\
14.091588993926	0\\
14.089873940186	0\\
14.0881585477368	0\\
14.0864428164447	0\\
14.0847267461758	0\\
14.083010336796	0\\
14.0812935881713	0\\
14.0795765001677	0\\
14.0778590726509	0\\
14.0761413054866	0\\
14.0744231985404	0\\
14.0727047516779	0\\
14.0709859647646	0\\
14.0692668376659	0\\
14.067547370247	0\\
14.0658275623732	0\\
14.0641074139097	0\\
14.0623869247216	0\\
14.0606660946737	0\\
14.0589449236311	0\\
14.0572234114586	0\\
14.0555015580209	0\\
14.0537793631827	0\\
14.0520568268085	0\\
14.050333948763	0\\
14.0486107289105	0\\
14.0468871671154	0\\
14.0451632632419	0\\
14.0434390171542	0\\
14.0417144287165	0\\
14.0399894977927	0\\
14.0382642242469	0\\
14.0365386079428	0\\
14.0348126487442	0\\
14.033086346515	0\\
14.0313597011185	0\\
14.0296327124185	0\\
14.0279053802784	0\\
14.0261777045615	0\\
14.0244496851311	0\\
14.0227213218505	0\\
14.0209926145828	0\\
14.019263563191	0\\
14.0175341675381	0\\
14.0158044274869	0\\
14.0140743429004	0\\
14.0123439136411	0\\
14.0106131395718	0\\
14.008882020555	0\\
14.0071505564532	0\\
14.0054187471287	0\\
14.003686592444	0\\
14.0019540922611	0\\
14.0002212464422	0\\
13.9984880548495	0\\
13.9967545173449	0\\
13.9950206337903	0\\
13.9932864040475	0\\
13.9915518279782	0\\
13.9898169054441	0\\
13.9880816363067	0\\
13.9863460204276	0\\
13.9846100576682	0\\
13.9828737478897	0\\
13.9811370909534	0\\
13.9794000867204	0\\
13.9776627350518	0\\
13.9759250358086	0\\
13.9741869888517	0\\
13.972448594042	0\\
13.97070985124	0\\
13.9689707603066	0\\
13.9672313211022	0\\
13.9654915334874	0\\
13.9637513973224	0\\
13.9620109124678	0\\
13.9602700787836	0\\
13.9585288961301	0\\
13.9567873643673	0\\
13.9550454833551	0\\
13.9533032529535	0\\
13.9515606730223	0\\
13.9498177434211	0\\
13.9480744640098	0\\
13.9463308346477	0\\
13.9445868551944	0\\
13.9428425255092	0\\
13.9410978454515	0\\
13.9393528148805	0\\
13.9376074336552	0\\
13.9358617016349	0\\
13.9341156186783	0\\
13.9323691846445	0\\
13.9306223993921	0\\
13.92887526278	0\\
13.9271277746666	0\\
13.9253799349107	0\\
13.9236317433705	0\\
13.9218831999045	0\\
13.9201343043709	0\\
13.918385056628	0\\
13.9166354565338	0\\
13.9148855039465	0\\
13.9131351987238	0\\
13.9113845407237	0\\
13.9096335298039	0\\
13.9078821658222	0\\
13.9061304486361	0\\
13.904378378103	0\\
13.9026259540805	0\\
13.9008731764259	0\\
13.8991200449964	0\\
13.8973665596491	0\\
13.8956127202412	0\\
13.8938585266297	0\\
13.8921039786714	0\\
13.8903490762231	0\\
13.8885938191417	0\\
13.8868382072836	0\\
13.8850822405056	0\\
13.883325918664	0\\
13.8815692416152	0\\
13.8798122092155	0\\
13.8780548213212	0\\
13.8762970777883	0\\
13.8745389784729	0\\
13.872780523231	0\\
13.8710217119183	0\\
13.8692625443906	0\\
13.8675030205037	0\\
13.8657431401131	0\\
13.8639829030743	0\\
13.8622223092428	0\\
13.8604613584739	0\\
13.8587000506228	0\\
13.8569383855446	0\\
13.8551763630945	0\\
13.8534139831274	0\\
13.8516512454982	0\\
13.8498881500617	0\\
13.8481246966726	0\\
13.8463608851856	0\\
13.8445967154551	0\\
13.8428321873357	0\\
13.8410673006816	0\\
13.8393020553472	0\\
13.8375364511866	0\\
13.835770488054	0\\
13.8340041658032	0\\
13.8322374842883	0\\
13.8304704433631	0\\
13.8287030428812	0\\
13.8269352826964	0\\
13.8251671626622	0\\
13.8233986826321	0\\
13.8216298424594	0\\
13.8198606419974	0\\
13.8180910810993	0\\
13.8163211596183	0\\
13.8145508774073	0\\
13.8127802343193	0\\
13.8110092302072	0\\
13.8092378649236	0\\
13.8074661383212	0\\
13.8056940502526	0\\
13.8039216005703	0\\
13.8021487891266	0\\
13.8003756157739	0\\
13.7986020803644	0\\
13.7968281827501	0\\
13.7950539227831	0\\
13.7932793003154	0\\
13.7915043151988	0\\
13.789728967285	0\\
13.7879532564256	0\\
13.7861771824724	0\\
13.7844007452767	0\\
13.78262394469	0\\
13.7808467805634	0\\
13.7790692527484	0\\
13.7772913610958	0\\
13.7755131054569	0\\
13.7737344856825	0\\
13.7719555016234	0\\
13.7701761531304	0\\
13.7683964400542	0\\
13.7666163622453	0\\
13.7648359195542	0\\
13.7630551118313	0\\
13.7612739389269	0\\
13.7594924006911	0\\
13.7577104969741	0\\
13.7559282276259	0\\
13.7541455924964	0\\
13.7523625914354	0\\
13.7505792242927	0\\
13.7487954909179	0\\
13.7470113911605	0\\
13.7452269248701	0\\
13.743442091896	0\\
13.7416568920874	0\\
13.7398713252936	0\\
13.7380853913635	0\\
13.7362990901463	0\\
13.7345124214908	0\\
13.7327253852459	0\\
13.7309379812601	0\\
13.7291502093822	0\\
13.7273620694607	0\\
13.725573561344	0\\
13.7237846848805	0\\
13.7219954399183	0\\
13.7202058263057	0\\
13.7184158438907	0\\
13.7166254925213	0\\
13.7148347720453	0\\
13.7130436823105	0\\
13.7112522231646	0\\
13.7094603944552	0\\
13.7076681960297	0\\
13.7058756277357	0\\
13.7040826894203	0\\
13.7022893809308	0\\
13.7004957021143	0\\
13.6987016528178	0\\
13.6969072328882	0\\
13.6951124421725	0\\
13.6933172805172	0\\
13.6915217477691	0\\
13.6897258437747	0\\
13.6879295683804	0\\
13.6861329214326	0\\
13.6843359027776	0\\
13.6825385122615	0\\
13.6807407497304	0\\
13.6789426150302	0\\
13.6771441080069	0\\
13.6753452285062	0\\
13.6735459763738	0\\
13.6717463514553	0\\
13.6699463535962	0\\
13.6681459826419	0\\
13.6663452384377	0\\
13.6645441208287	0\\
13.6627426296601	0\\
13.660940764777	0\\
13.6591385260242	0\\
13.6573359132465	0\\
13.6555329262887	0\\
13.6537295649954	0\\
13.6519258292111	0\\
13.6501217187802	0\\
13.6483172335472	0\\
13.6465123733561	0\\
13.6447071380513	0\\
13.6429015274766	0\\
13.6410955414761	0\\
13.6392891798937	0\\
13.6374824425729	0\\
13.6356753293576	0\\
13.6338678400913	0\\
13.6320599746174	0\\
13.6302517327792	0\\
13.6284431144202	0\\
13.6266341193833	0\\
13.6248247475117	0\\
13.6230149986484	0\\
13.6212048726362	0\\
13.6193943693179	0\\
13.6175834885362	0\\
13.6157722301336	0\\
13.6139605939527	0\\
13.6121485798358	0\\
13.6103361876251	0\\
13.6085234171629	0\\
13.6067102682913	0\\
13.6048967408522	0\\
13.6030828346875	0\\
13.601268549639	0\\
13.5994538855484	0\\
13.5976388422572	0\\
13.5958234196071	0\\
13.5940076174393	0\\
13.5921914355951	0\\
13.5903748739158	0\\
13.5885579322424	0\\
13.5867406104159	0\\
13.5849229082772	0\\
13.5831048256671	0\\
13.5812863624263	0\\
13.5794675183953	0\\
13.5776482934147	0\\
13.5758286873249	0\\
13.5740086999661	0\\
13.5721883311785	0\\
13.5703675808023	0\\
13.5685464486773	0\\
13.5667249346436	0\\
13.5649030385408	0\\
13.5630807602087	0\\
13.5612580994869	0\\
13.5594350562148	0\\
13.5576116302318	0\\
13.5557878213772	0\\
13.5539636294902	0\\
13.5521390544099	0\\
13.5503140959752	0\\
13.5484887540249	0\\
13.546663028398	0\\
13.5448369189331	0\\
13.5430104254687	0\\
13.5411835478432	0\\
13.5393562858952	0\\
13.5375286394627	0\\
13.5357006083841	0\\
13.5338721924973	0\\
13.5320433916404	0\\
13.5302142056511	0\\
13.5283846343672	0\\
13.5265546776264	0\\
13.5247243352662	0\\
13.5228936071241	0\\
13.5210624930374	0\\
13.5192309928434	0\\
13.5173991063791	0\\
13.5155668334817	0\\
13.513734173988	0\\
13.5119011277349	0\\
13.5100676945591	0\\
13.5082338742973	0\\
13.5063996667858	0\\
13.5045650718613	0\\
13.5027300893599	0\\
13.5008947191179	0\\
13.4990589609713	0\\
13.4972228147562	0\\
13.4953862803085	0\\
13.493549357464	0\\
13.4917120460583	0\\
13.489874345927	0\\
13.4880362569056	0\\
13.4861977788295	0\\
13.484358911534	0\\
13.4825196548542	0\\
13.4806800086251	0\\
13.4788399726818	0\\
13.476999546859	0\\
13.4751587309915	0\\
13.473317524914	0\\
13.471475928461	0\\
13.4696339414669	0\\
13.4677915637661	0\\
13.4659487951927	0\\
13.4641056355809	0\\
13.4622620847647	0\\
13.4604181425779	0\\
13.4585738088544	0\\
13.4567290834279	0\\
13.454883966132	0\\
13.4530384568001	0\\
13.4511925552655	0\\
13.4493462613616	0\\
13.4474995749216	0\\
13.4456524957784	0\\
13.4438050237651	0\\
13.4419571587144	0\\
13.440108900459	0\\
13.4382602488317	0\\
13.436411203665	0\\
13.4345617647912	0\\
13.4327119320426	0\\
13.4308617052515	0\\
13.4290110842499	0\\
13.4271600688698	0\\
13.4253086589432	0\\
13.4234568543017	0\\
13.421604654777	0\\
13.4197520602007	0\\
13.4178990704042	0\\
13.4160456852189	0\\
13.4141919044759	0\\
13.4123377280065	0\\
13.4104831556416	0\\
13.4086281872121	0\\
13.4067728225488	0\\
13.4049170614825	0\\
13.4030609038436	0\\
13.4012043494627	0\\
13.3993473981701	0\\
13.3974900497961	0\\
13.3956323041707	0\\
13.3937741611242	0\\
13.3919156204863	0\\
13.3900566820869	0\\
13.3881973457557	0\\
13.3863376113222	0\\
13.3844774786161	0\\
13.3826169474667	0\\
13.3807560177031	0\\
13.3788946891547	0\\
13.3770329616504	0\\
13.3751708350192	0\\
13.3733083090898	0\\
13.3714453836912	0\\
13.3695820586517	0\\
13.3677183338	0\\
13.3658542089644	0\\
13.3639896839732	0\\
13.3621247586546	0\\
13.3602594328366	0\\
13.3583937063472	0\\
13.3565275790142	0\\
13.3546610506653	0\\
13.3527941211282	0\\
13.3509267902303	0\\
13.3490590577991	0\\
13.3471909236617	0\\
13.3453223876455	0\\
13.3434534495774	0\\
13.3415841092843	0\\
13.3397143665932	0\\
13.3378442213307	0\\
13.3359736733235	0\\
13.334102722398	0\\
13.3322313683806	0\\
13.3303596110976	0\\
13.3284874503752	0\\
13.3266148860394	0\\
13.3247419179161	0\\
13.3228685458312	0\\
13.3209947696103	0\\
13.3191205890791	0\\
13.3172460040631	0\\
13.3153710143876	0\\
13.3134956198779	0\\
13.3116198203591	0\\
13.3097436156562	0\\
13.3078670055942	0\\
13.3059899899979	0\\
13.304112568692	0\\
13.302234741501	0\\
13.3003565082495	0\\
13.2984778687616	0\\
13.2965988228618	0\\
13.2947193703741	0\\
13.2928395111225	0\\
13.2909592449309	0\\
13.2890785716232	0\\
13.2871974910228	0\\
13.2853160029535	0\\
13.2834341072386	0\\
13.2815518037015	0\\
13.2796690921653	0\\
13.2777859724532	0\\
13.2759024443881	0\\
13.274018507793	0\\
13.2721341624904	0\\
13.2702494083031	0\\
13.2683642450536	0\\
13.2664786725642	0\\
13.2645926906574	0\\
13.2627062991551	0\\
13.2608194978795	0\\
13.2589322866525	0\\
13.2570446652959	0\\
13.2551566336315	0\\
13.2532681914808	0\\
13.2513793386652	0\\
13.2494900750062	0\\
13.247600400325	0\\
13.2457103144426	0\\
13.2438198171801	0\\
13.2419289083585	0\\
13.2400375877983	0\\
13.2381458553204	0\\
13.2362537107452	0\\
13.2343611538932	0\\
13.2324681845846	0\\
13.2305748026396	0\\
13.2286810078784	0\\
13.2267868001208	0\\
13.2248921791867	0\\
13.2229971448957	0\\
13.2211016970676	0\\
13.2192058355217	0\\
13.2173095600774	0\\
13.2154128705539	0\\
13.2135157667705	0\\
13.211618248546	0\\
13.2097203156994	0\\
13.2078219680493	0\\
13.2059232054146	0\\
13.2040240276136	0\\
13.2021244344649	0\\
13.2002244257866	0\\
13.198324001397	0\\
13.1964231611141	0\\
13.1945219047558	0\\
13.19262023214	0\\
13.1907181430843	0\\
13.1888156374064	0\\
13.1869127149235	0\\
13.1850093754532	0\\
13.1831056188126	0\\
13.1812014448188	0\\
13.1792968532887	0\\
13.1773918440392	0\\
13.1754864168871	0\\
13.173580571649	0\\
13.1716743081413	0\\
13.1697676261803	0\\
13.1678605255825	0\\
13.1659530061638	0\\
13.1640450677403	0\\
13.1621367101279	0\\
13.1602279331422	0\\
13.1583187365991	0\\
13.1564091203139	0\\
13.1544990841022	0\\
13.152588627779	0\\
13.1506777511597	0\\
13.1487664540593	0\\
13.1468547362925	0\\
13.1449425976743	0\\
13.1430300380193	0\\
13.1411170571421	0\\
13.139203654857	0\\
13.1372898309783	0\\
13.1353755853203	0\\
13.133460917697	0\\
13.1315458279223	0\\
13.12963031581	0\\
13.1277143811738	0\\
13.1257980238272	0\\
13.1238812435837	0\\
13.1219640402566	0\\
13.1200464136591	0\\
13.1181283636043	0\\
13.116209889905	0\\
13.1142909923742	0\\
13.1123716708244	0\\
13.1104519250684	0\\
13.1085317549184	0\\
13.1066111601868	0\\
13.1046901406859	0\\
13.1027686962276	0\\
13.100846826624	0\\
13.0989245316869	0\\
13.0970018112279	0\\
13.0950786650586	0\\
13.0931550929905	0\\
13.0912310948349	0\\
13.0893066704029	0\\
13.0873818195057	0\\
13.0854565419542	0\\
13.0835308375592	0\\
13.0816047061314	0\\
13.0796781474814	0\\
13.0777511614196	0\\
13.0758237477562	0\\
13.0738959063016	0\\
13.0719676368658	0\\
13.0700389392587	0\\
13.06810981329	0\\
13.0661802587696	0\\
13.0642502755069	0\\
13.0623198633113	0\\
13.0603890219923	0\\
13.0584577513588	0\\
13.0565260512201	0\\
13.054593921385	0\\
13.0526613616622	0\\
13.0507283718605	0\\
13.0487949517884	0\\
13.0468611012543	0\\
13.0449268200665	0\\
13.0429921080331	0\\
13.0410569649621	0\\
13.0391213906615	0\\
13.037185384939	0\\
13.0352489476022	0\\
13.0333120784587	0\\
13.0313747773158	0\\
13.0294370439808	0\\
13.0274988782609	0\\
13.0255602799629	0\\
13.0236212488938	0\\
13.0216817848602	0\\
13.0197418876689	0\\
13.0178015571262	0\\
13.0158607930386	0\\
13.0139195952122	0\\
13.0119779634531	0\\
13.0100358975673	0\\
13.0080933973606	0\\
13.0061504626387	0\\
13.0042070932072	0\\
13.0022632888714	0\\
13.0003190494368	0\\
12.9983743747084	0\\
12.9964292644913	0\\
12.9944837185904	0\\
12.9925377368106	0\\
12.9905913189564	0\\
12.9886444648323	0\\
12.9866971742428	0\\
12.9847494469921	0\\
12.9828012828844	0\\
12.9808526817235	0\\
12.9789036433135	0\\
12.9769541674579	0\\
12.9750042539604	0\\
12.9730539026245	0\\
12.9711031132534	0\\
12.9691518856505	0\\
12.9672002196186	0\\
12.9652481149609	0\\
12.96329557148	0\\
12.9613425889787	0\\
12.9593891672594	0\\
12.9574353061246	0\\
12.9554810053766	0\\
12.9535262648174	0\\
12.9515710842491	0\\
12.9496154634735	0\\
12.9476594022924	0\\
12.9457029005073	0\\
12.9437459579198	0\\
12.9417885743311	0\\
12.9398307495424	0\\
12.9378724833549	0\\
12.9359137755694	0\\
12.9339546259867	0\\
12.9319950344075	0\\
12.9300350006322	0\\
12.9280745244614	0\\
12.9261136056952	0\\
12.9241522441337	0\\
12.922190439577	0\\
12.9202281918248	0\\
12.9182655006769	0\\
12.9163023659328	0\\
12.9143387873921	0\\
12.9123747648539	0\\
12.9104102981175	0\\
12.9084453869818	0\\
12.9064800312459	0\\
12.9045142307083	0\\
12.9025479851678	0\\
12.9005812944228	0\\
12.8986141582717	0\\
12.8966465765127	0\\
12.8946785489439	0\\
12.8927100753631	0\\
12.8907411555681	0\\
12.8887717893568	0\\
12.8868019765265	0\\
12.8848317168746	0\\
12.8828610101984	0\\
12.880889856295	0\\
12.8789182549613	0\\
12.8769462059943	0\\
12.8749737091905	0\\
12.8730007643466	0\\
12.8710273712589	0\\
12.8690535297237	0\\
12.8670792395373	0\\
12.8651045004954	0\\
12.8631293123941	0\\
12.8611536750291	0\\
12.8591775881958	0\\
12.8572010516898	0\\
12.8552240653064	0\\
12.8532466288407	0\\
12.8512687420878	0\\
12.8492904048424	0\\
12.8473116168995	0\\
12.8453323780535	0\\
12.8433526880989	0\\
12.8413725468301	0\\
12.8393919540412	0\\
12.8374109095263	0\\
12.8354294130792	0\\
12.8334474644937	0\\
12.8314650635635	0\\
12.829482210082	0\\
12.8274989038425	0\\
12.8255151446383	0\\
12.8235309322623	0\\
12.8215462665075	0\\
12.8195611471666	0\\
12.8175755740324	0\\
12.8155895468971	0\\
12.8136030655533	0\\
12.8116161297931	0\\
12.8096287394084	0\\
12.8076408941914	0\\
12.8056525939336	0\\
12.8036638384268	0\\
12.8016746274624	0\\
12.7996849608318	0\\
12.7976948383261	0\\
12.7957042597364	0\\
12.7937132248536	0\\
12.7917217334685	0\\
12.7897297853717	0\\
12.7877373803537	0\\
12.7857445182047	0\\
12.7837511987151	0\\
12.7817574216747	0\\
12.7797631868736	0\\
12.7777684941015	0\\
12.775773343148	0\\
12.7737777338025	0\\
12.7717816658543	0\\
12.7697851390927	0\\
12.7677881533067	0\\
12.7657907082851	0\\
12.7637928038167	0\\
12.7617944396901	0\\
12.7597956156937	0\\
12.7577963316158	0\\
12.7557965872446	0\\
12.753796382368	0\\
12.751795716774	0\\
12.7497945902502	0\\
12.7477930025842	0\\
12.7457909535635	0\\
12.7437884429752	0\\
12.7417854706066	0\\
12.7397820362446	0\\
12.737778139676	0\\
12.7357737806875	0\\
12.7337689590657	0\\
12.7317636745968	0\\
12.7297579270673	0\\
12.7277517162631	0\\
12.7257450419703	0\\
12.7237379039745	0\\
12.7217303020615	0\\
12.7197222360167	0\\
12.7177137056255	0\\
12.715704710673	0\\
12.7136952509445	0\\
12.7116853262246	0\\
12.7096749362982	0\\
12.70766408095	0\\
12.7056527599642	0\\
12.7036409731254	0\\
12.7016287202175	0\\
12.6996160010246	0\\
12.6976028153305	0\\
12.695589162919	0\\
12.6935750435736	0\\
12.6915604570778	0\\
12.6895454032146	0\\
12.6875298817674	0\\
12.6855138925189	0\\
12.683497435252	0\\
12.6814805097494	0\\
12.6794631157935	0\\
12.6774452531667	0\\
12.6754269216511	0\\
12.6734081210289	0\\
12.6713888510818	0\\
12.6693691115917	0\\
12.6673489023401	0\\
12.6653282231085	0\\
12.6633070736781	0\\
12.66128545383	0\\
12.6592633633452	0\\
12.6572408020046	0\\
12.6552177695888	0\\
12.6531942658783	0\\
12.6511702906534	0\\
12.6491458436945	0\\
12.6471209247815	0\\
12.6450955336943	0\\
12.6430696702127	0\\
12.6410433341162	0\\
12.6390165251843	0\\
12.6369892431964	0\\
12.6349614879314	0\\
12.6329332591684	0\\
12.6309045566862	0\\
12.6288753802634	0\\
12.6268457296787	0\\
12.6248156047102	0\\
12.6227850051362	0\\
12.6207539307348	0\\
12.6187223812838	0\\
12.616690356561	0\\
12.6146578563439	0\\
12.61262488041	0\\
12.6105914285365	0\\
12.6085575005005	0\\
12.6065230960789	0\\
12.6044882150486	0\\
12.6024528571862	0\\
12.6004170222682	0\\
12.5983807100708	0\\
12.5963439203703	0\\
12.5943066529426	0\\
12.5922689075637	0\\
12.5902306840091	0\\
12.5881919820544	0\\
12.586152801475	0\\
12.5841131420461	0\\
12.5820730035427	0\\
12.5800323857398	0\\
12.5779912884121	0\\
12.5759497113342	0\\
12.5739076542805	0\\
12.5718651170252	0\\
12.5698220993425	0\\
12.5677786010062	0\\
12.5657346217903	0\\
12.5636901614683	0\\
12.5616452198136	0\\
12.5595997965996	0\\
12.5575538915994	0\\
12.5555075045861	0\\
12.5534606353323	0\\
12.5514132836109	0\\
12.5493654491942	0\\
12.5473171318547	0\\
12.5452683313644	0\\
12.5432190474955	0\\
12.5411692800198	0\\
12.5391190287089	0\\
12.5370682933345	0\\
12.5350170736679	0\\
12.5329653694802	0\\
12.5309131805426	0\\
12.5288605066259	0\\
12.5268073475008	0\\
12.524753702938	0\\
12.5226995727078	0\\
12.5206449565804	0\\
12.5185898543259	0\\
12.5165342657142	0\\
12.5144781905151	0\\
12.5124216284982	0\\
12.5103645794328	0\\
12.5083070430882	0\\
12.5062490192335	0\\
12.5041905076376	0\\
12.5021315080694	0\\
12.5000720202973	0\\
12.4980120440898	0\\
12.4959515792152	0\\
12.4938906254416	0\\
12.491829182537	0\\
12.489767250269	0\\
12.4877048284053	0\\
12.4856419167134	0\\
12.4835785149604	0\\
12.4815146229137	0\\
12.4794502403399	0\\
12.477385367006	0\\
12.4753200026786	0\\
12.4732541471241	0\\
12.4711878001087	0\\
12.4691209613987	0\\
12.4670536307598	0\\
12.464985807958	0\\
12.4629174927588	0\\
12.4608486849276	0\\
12.4587793842298	0\\
12.4567095904304	0\\
12.4546393032944	0\\
12.4525685225865	0\\
12.4504972480714	0\\
12.4484254795134	0\\
12.4463532166769	0\\
12.4442804593259	0\\
12.4422072072244	0\\
12.4401334601361	0\\
12.4380592178246	0\\
12.4359844800534	0\\
12.4339092465856	0\\
12.4318335171844	0\\
12.4297572916126	0\\
12.4276805696331	0\\
12.4256033510083	0\\
12.4235256355007	0\\
12.4214474228725	0\\
12.4193687128858	0\\
12.4172895053024	0\\
12.4152097998841	0\\
12.4131295963924	0\\
12.4110488945887	0\\
12.4089676942342	0\\
12.4068859950899	0\\
12.4048037969166	0\\
12.4027210994752	0\\
12.4006379025259	0\\
12.3985542058294	0\\
12.3964700091456	0\\
12.3943853122345	0\\
12.3923001148561	0\\
12.39021441677	0\\
12.3881282177355	0\\
12.3860415175122	0\\
12.3839543158589	0\\
12.3818666125349	0\\
12.3797784072987	0\\
12.377689699909	0\\
12.3756004901243	0\\
12.3735107777028	0\\
12.3714205624026	0\\
12.3693298439816	0\\
12.3672386221976	0\\
12.365146896808	0\\
12.3630546675704	0\\
12.3609619342419	0\\
12.3588686965795	0\\
12.3567749543401	0\\
12.3546807072804	0\\
12.3525859551568	0\\
12.3504906977258	0\\
12.3483949347435	0\\
12.3462986659659	0\\
12.3442018911487	0\\
12.3421046100476	0\\
12.340006822418	0\\
12.3379085280152	0\\
12.3358097265943	0\\
12.3337104179102	0\\
12.3316106017177	0\\
12.3295102777713	0\\
12.3274094458254	0\\
12.3253081056342	0\\
12.3232062569517	0\\
12.3211038995317	0\\
12.319001033128	0\\
12.316897657494	0\\
12.3147937723831	0\\
12.3126893775483	0\\
12.3105844727427	0\\
12.3084790577189	0\\
12.3063731322296	0\\
12.3042666960272	0\\
12.3021597488639	0\\
12.3000522904918	0\\
12.2979443206627	0\\
12.2958358391284	0\\
12.2937268456403	0\\
12.2916173399497	0\\
12.2895073218079	0\\
12.2873967909658	0\\
12.2852857471741	0\\
12.2831741901835	0\\
12.2810621197444	0\\
12.278949535607	0\\
12.2768364375214	0\\
12.2747228252374	0\\
12.2726086985048	0\\
12.270494057073	0\\
12.2683789006915	0\\
12.2662632291092	0\\
12.2641470420752	0\\
12.2620303393383	0\\
12.2599131206469	0\\
12.2577953857497	0\\
12.2556771343947	0\\
12.25355836633	0\\
12.2514390813035	0\\
12.2493192790628	0\\
12.2471989593555	0\\
12.2450781219287	0\\
12.2429567665297	0\\
12.2408348929054	0\\
12.2387125008025	0\\
12.2365895899675	0\\
12.2344661601468	0\\
12.2323422110867	0\\
12.2302177425331	0\\
12.2280927542319	0\\
12.2259672459286	0\\
12.2238412173687	0\\
12.2217146682975	0\\
12.2195875984599	0\\
12.217460007601	0\\
12.2153318954654	0\\
12.2132032617976	0\\
12.2110741063419	0\\
12.2089444288424	0\\
12.2068142290431	0\\
12.2046835066878	0\\
12.2025522615199	0\\
12.2004204932829	0\\
12.19828820172	0\\
12.196155386574	0\\
12.194022047588	0\\
12.1918881845044	0\\
12.1897537970657	0\\
12.1876188850141	0\\
12.1854834480918	0\\
12.1833474860404	0\\
12.1812109986017	0\\
12.1790739855173	0\\
12.1769364465282	0\\
12.1747983813758	0\\
12.1726597898007	0\\
12.1705206715439	0\\
12.1683810263457	0\\
12.1662408539465	0\\
12.1641001540865	0\\
12.1619589265056	0\\
12.1598171709435	0\\
12.1576748871398	0\\
12.1555320748339	0\\
12.1533887337649	0\\
12.1512448636718	0\\
12.1491004642934	0\\
12.1469555353683	0\\
12.1448100766349	0\\
12.1426640878313	0\\
12.1405175686957	0\\
12.1383705189658	0\\
12.1362229383793	0\\
12.1340748266735	0\\
12.1319261835857	0\\
12.129777008853	0\\
12.1276273022121	0\\
12.1254770633998	0\\
12.1233262921524	0\\
12.1211749882063	0\\
12.1190231512975	0\\
12.1168707811618	0\\
12.1147178775349	0\\
12.1125644401524	0\\
12.1104104687494	0\\
12.108255963061	0\\
12.1061009228222	0\\
12.1039453477676	0\\
12.1017892376316	0\\
12.0996325921486	0\\
12.0974754110527	0\\
12.0953176940777	0\\
12.0931594409574	0\\
12.0910006514252	0\\
12.0888413252145	0\\
12.0866814620582	0\\
12.0845210616894	0\\
12.0823601238407	0\\
12.0801986482446	0\\
12.0780366346334	0\\
12.0758740827393	0\\
12.073710992294	0\\
12.0715473630293	0\\
12.0693831946768	0\\
12.0672184869676	0\\
12.0650532396329	0\\
12.0628874524036	0\\
12.0607211250104	0\\
12.0585542571838	0\\
12.056386848654	0\\
12.0542188991512	0\\
12.0520504084051	0\\
12.0498813761456	0\\
12.0477118021021	0\\
12.0455416860039	0\\
12.0433710275799	0\\
12.0411998265592	0\\
12.0390280826704	0\\
12.036855795642	0\\
12.0346829652021	0\\
12.0325095910789	0\\
12.0303356730002	0\\
12.0281612106937	0\\
12.0259862038868	0\\
12.0238106523067	0\\
12.0216345556805	0\\
12.019457913735	0\\
12.0172807261968	0\\
12.0151029927924	0\\
12.0129247132479	0\\
12.0107458872894	0\\
12.0085665146426	0\\
12.0063865950332	0\\
12.0042061281866	0\\
12.0020251138278	0\\
11.999843551682	0\\
11.9976614414738	0\\
11.9954787829278	0\\
11.9932955757683	0\\
11.9911118197196	0\\
11.9889275145055	0\\
11.9867426598498	0\\
11.9845572554759	0\\
11.9823713011073	0\\
11.9801847964669	0\\
11.9779977412777	0\\
11.9758101352623	0\\
11.9736219781433	0\\
11.9714332696428	0\\
11.969244009483	0\\
11.9670541973857	0\\
11.9648638330725	0\\
11.9626729162648	0\\
11.9604814466838	0\\
11.9582894240506	0\\
11.9560968480859	0\\
11.9539037185102	0\\
11.9517100350441	0\\
11.9495157974075	0\\
11.9473210053206	0\\
11.9451256585028	0\\
11.9429297566739	0\\
11.9407332995531	0\\
11.9385362868594	0\\
11.9363387183118	0\\
11.9341405936289	0\\
11.9319419125292	0\\
11.9297426747309	0\\
11.927542879952	0\\
11.9253425279103	0\\
11.9231416183234	0\\
11.9209401509088	0\\
11.9187381253835	0\\
11.9165355414645	0\\
11.9143323988685	0\\
11.9121286973121	0\\
11.9099244365115	0\\
11.9077196161828	0\\
11.905514236042	0\\
11.9033082958046	0\\
11.9011017951861	0\\
11.8988947339017	0\\
11.8966871116664	0\\
11.8944789281949	0\\
11.892270183202	0\\
11.8900608764018	0\\
11.8878510075085	0\\
11.8856405762361	0\\
11.8834295822982	0\\
11.8812180254084	0\\
11.8790059052797	0\\
11.8767932216254	0\\
11.8745799741582	0\\
11.8723661625907	0\\
11.8701517866353	0\\
11.8679368460041	0\\
11.8657213404091	0\\
11.8635052695621	0\\
11.8612886331744	0\\
11.8590714309573	0\\
11.856853662622	0\\
11.8546353278792	0\\
11.8524164264396	0\\
11.8501969580136	0\\
11.8479769223113	0\\
11.8457563190426	0\\
11.8435351479173	0\\
11.8413134086449	0\\
11.8390911009347	0\\
11.8368682244957	0\\
11.8346447790367	0\\
11.8324207642664	0\\
11.8301961798931	0\\
11.827971025625	0\\
11.82574530117	0\\
11.8235190062358	0\\
11.82129214053	0\\
11.8190647037597	0\\
11.8168366956321	0\\
11.8146081158538	0\\
11.8123789641316	0\\
11.8101492401717	0\\
11.8079189436803	0\\
11.8056880743632	0\\
11.8034566319263	0\\
11.8012246160749	0\\
11.7989920265142	0\\
11.7967588629492	0\\
11.7945251250847	0\\
11.7922908126253	0\\
11.7900559252753	0\\
11.7878204627387	0\\
11.7855844247193	0\\
11.783347810921	0\\
11.7811106210469	0\\
11.7788728548003	0\\
11.7766345118841	0\\
11.7743955920011	0\\
11.7721560948537	0\\
11.7699160201442	0\\
11.7676753675746	0\\
11.7654341368466	0\\
11.7631923276618	0\\
11.7609499397217	0\\
11.7587069727271	0\\
11.7564634263791	0\\
11.7542193003782	0\\
11.7519745944249	0\\
11.7497293082193	0\\
11.7474834414613	0\\
11.7452369938507	0\\
11.7429899650869	0\\
11.7407423548691	0\\
11.7384941628964	0\\
11.7362453888675	0\\
11.733996032481	0\\
11.7317460934351	0\\
11.7294955714279	0\\
11.7272444661573	0\\
11.7249927773208	0\\
11.7227405046159	0\\
11.7204876477395	0\\
11.7182342063887	0\\
11.71598018026	0\\
11.7137255690499	0\\
11.7114703724546	0\\
11.70921459017	0\\
11.7069582218918	0\\
11.7047012673155	0\\
11.7024437261363	0\\
11.7001855980492	0\\
11.697926882749	0\\
11.6956675799302	0\\
11.693407689287	0\\
11.6911472105135	0\\
11.6888861433035	0\\
11.6866244873506	0\\
11.6843622423481	0\\
11.6820994079891	0\\
11.6798359839663	0\\
11.6775719699725	0\\
11.6753073657	0\\
11.6730421708409	0\\
11.670776385087	0\\
11.6685100081301	0\\
11.6662430396616	0\\
11.6639754793725	0\\
11.6617073269538	0\\
11.6594385820961	0\\
11.65716924449	0\\
11.6548993138255	0\\
11.6526287897927	0\\
11.6503576720812	0\\
11.6480859603806	0\\
11.6458136543799	0\\
11.6435407537682	0\\
11.6412672582342	0\\
11.6389931674664	0\\
11.636718481153	0\\
11.634443198982	0\\
11.6321673206411	0\\
11.629890845818	0\\
11.6276137741997	0\\
11.6253361054734	0\\
11.6230578393258	0\\
11.6207789754433	0\\
11.6184995135124	0\\
11.6162194532189	0\\
11.6139387942487	0\\
11.6116575362873	0\\
11.60937567902	0\\
11.6070932221318	0\\
11.6048101653075	0\\
11.6025265082316	0\\
11.6002422505885	0\\
11.5979573920621	0\\
11.5956719323362	0\\
11.5933858710944	0\\
11.59109920802	0\\
11.5888119427959	0\\
11.5865240751051	0\\
11.58423560463	0\\
11.5819465310529	0\\
11.5796568540558	0\\
11.5773665733206	0\\
11.5750756885287	0\\
11.5727841993614	0\\
11.5704921054999	0\\
11.5681994066247	0\\
11.5659061024165	0\\
11.5636121925556	0\\
11.5613176767218	0\\
11.5590225545951	0\\
11.5567268258549	0\\
11.5544304901804	0\\
11.5521335472507	0\\
11.5498359967445	0\\
11.5475378383403	0\\
11.5452390717163	0\\
11.5429396965505	0\\
11.5406397125206	0\\
11.5383391193041	0\\
11.5360379165783	0\\
11.5337361040199	0\\
11.5314336813058	0\\
11.5291306481124	0\\
11.5268270041159	0\\
11.5245227489921	0\\
11.5222178824168	0\\
11.5199124040654	0\\
11.5176063136129	0\\
11.5152996107344	0\\
11.5129922951044	0\\
11.5106843663973	0\\
11.5083758242872	0\\
11.506066668448	0\\
11.5037568985532	0\\
11.5014465142762	0\\
11.4991355152901	0\\
11.4968239012677	0\\
11.4945116718815	0\\
11.4921988268037	0\\
11.4898853657066	0\\
11.4875712882616	0\\
11.4852565941405	0\\
11.4829412830145	0\\
11.4806253545544	0\\
11.478308808431	0\\
11.4759916443148	0\\
11.473673861876	0\\
11.4713554607844	0\\
11.4690364407097	0\\
11.4667168013214	0\\
11.4643965422884	0\\
11.4620756632798	0\\
11.4597541639641	0\\
11.4574320440096	0\\
11.4551093030844	0\\
11.4527859408563	0\\
11.4504619569928	0\\
11.4481373511611	0\\
11.4458121230284	0\\
11.4434862722612	0\\
11.4411597985261	0\\
11.4388327014892	0\\
11.4365049808166	0\\
11.4341766361738	0\\
11.4318476672261	0\\
11.4295180736389	0\\
11.4271878550768	0\\
11.4248570112045	0\\
11.4225255416862	0\\
11.4201934461861	0\\
11.4178607243678	0\\
11.415527375895	0\\
11.4131934004307	0\\
11.4108587976379	0\\
11.4085235671795	0\\
11.4061877087176	0\\
11.4038512219145	0\\
11.4015141064321	0\\
11.3991763619319	0\\
11.3968379880752	0\\
11.3944989845232	0\\
11.3921593509365	0\\
11.3898190869757	0\\
11.3874781923009	0\\
11.3851366665722	0\\
11.3827945094492	0\\
11.3804517205913	0\\
11.3781082996576	0\\
11.3757642463069	0\\
11.3734195601979	0\\
11.3710742409889	0\\
11.3687282883377	0\\
11.3663817019022	0\\
11.3640344813399	0\\
11.3616866263079	0\\
11.3593381364631	0\\
11.3569890114621	0\\
11.3546392509614	0\\
11.3522888546169	0\\
11.3499378220845	0\\
11.3475861530196	0\\
11.3452338470775	0\\
11.3428809039131	0\\
11.3405273231812	0\\
11.3381731045361	0\\
11.3358182476318	0\\
11.3334627521223	0\\
11.3311066176611	0\\
11.3287498439014	0\\
11.3263924304962	0\\
11.3240343770983	0\\
11.3216756833599	0\\
11.3193163489333	0\\
11.3169563734704	0\\
11.3145957566225	0\\
11.3122344980412	0\\
11.3098725973772	0\\
11.3075100542815	0\\
11.3051468684043	0\\
11.3027830393958	0\\
11.3004185669059	0\\
11.2980534505841	0\\
11.2956876900797	0\\
11.2933212850418	0\\
11.290954235119	0\\
11.2885865399597	0\\
11.2862181992121	0\\
11.283849212524	0\\
11.2814795795429	0\\
11.2791092999163	0\\
11.2767383732909	0\\
11.2743667993136	0\\
11.2719945776306	0\\
11.2696217078882	0\\
11.2672481897321	0\\
11.264874022808	0\\
11.2624992067609	0\\
11.2601237412359	0\\
11.2577476258776	0\\
11.2553708603304	0\\
11.2529934442383	0\\
11.2506153772452	0\\
11.2482366589945	0\\
11.2458572891295	0\\
11.243477267293	0\\
11.2410965931276	0\\
11.2387152662756	0\\
11.2363332863791	0\\
11.2339506530799	0\\
11.2315673660192	0\\
11.2291834248383	0\\
11.226798829178	0\\
11.2244135786789	0\\
11.2220276729811	0\\
11.2196411117247	0\\
11.2172538945493	0\\
11.2148660210942	0\\
11.2124774909986	0\\
11.2100883039011	0\\
11.2076984594403	0\\
11.2053079572543	0\\
11.2029167969809	0\\
11.2005249782578	0\\
11.1981325007222	0\\
11.1957393640111	0\\
11.1933455677612	0\\
11.1909511116087	0\\
11.1885559951898	0\\
11.1861602181403	0\\
11.1837637800955	0\\
11.1813666806907	0\\
11.1789689195608	0\\
11.1765704963402	0\\
11.1741714106633	0\\
11.171771662164	0\\
11.1693712504759	0\\
11.1669701752324	0\\
11.1645684360665	0\\
11.162166032611	0\\
11.1597629644983	0\\
11.1573592313604	0\\
11.1549548328294	0\\
11.1525497685365	0\\
11.1501440381132	0\\
11.1477376411901	0\\
11.1453305773981	0\\
11.1429228463673	0\\
11.1405144477277	0\\
11.138105381109	0\\
11.1356956461405	0\\
11.1332852424514	0\\
11.1308741696703	0\\
11.1284624274257	0\\
11.1260500153457	0\\
11.1236369330582	0\\
11.1212231801907	0\\
11.1188087563702	0\\
11.1163936612238	0\\
11.113977894378	0\\
11.1115614554591	0\\
11.109144344093	0\\
11.1067265599053	0\\
11.1043081025215	0\\
11.1018889715664	0\\
11.0994691666648	0\\
11.0970486874411	0\\
11.0946275335193	0\\
11.0922057045233	0\\
11.0897832000764	0\\
11.0873600198018	0\\
11.0849361633222	0\\
11.0825116302603	0\\
11.0800864202381	0\\
11.0776605328775	0\\
11.0752339678001	0\\
11.0728067246271	0\\
11.0703788029794	0\\
11.0679502024776	0\\
11.065520922742	0\\
11.0630909633925	0\\
11.0606603240488	0\\
11.0582290043302	0\\
11.0557970038556	0\\
11.0533643222439	0\\
11.0509309591132	0\\
11.0484969140817	0\\
11.0460621867671	0\\
11.0436267767867	0\\
11.0411906837577	0\\
11.0387539072967	0\\
11.0363164470203	0\\
11.0338783025445	0\\
11.0314394734851	0\\
11.0289999594575	0\\
11.0265597600769	0\\
11.0241188749581	0\\
11.0216773037156	0\\
11.0192350459635	0\\
11.0167921013157	0\\
11.0143484693857	0\\
11.0119041497866	0\\
11.0094591421313	0\\
11.0070134460323	0\\
11.0045670611019	0\\
11.0021199869519	0\\
10.9996722231938	0\\
10.9972237694389	0\\
10.994774625298	0\\
10.9923247903817	0\\
10.9898742643003	0\\
10.9874230466635	0\\
10.9849711370811	0\\
10.9825185351622	0\\
10.9800652405158	0\\
10.9776112527503	0\\
10.9751565714741	0\\
10.972701196295	0\\
10.9702451268207	0\\
10.9677883626584	0\\
10.9653309034149	0\\
10.9628727486969	0\\
10.9604138981106	0\\
10.9579543512619	0\\
10.9554941077565	0\\
10.9530331671994	0\\
10.9505715291956	0\\
10.9481091933498	0\\
10.9456461592661	0\\
10.9431824265484	0\\
10.9407179948002	0\\
10.9382528636248	0\\
10.9357870326252	0\\
10.9333205014037	0\\
10.9308532695626	0\\
10.9283853367038	0\\
10.9259167024288	0\\
10.9234473663389	0\\
10.9209773280347	0\\
10.9185065871169	0\\
10.9160351431855	0\\
10.9135629958405	0\\
10.9110901446813	0\\
10.908616589307	0\\
10.9061423293165	0\\
10.9036673643081	0\\
10.9011916938801	0\\
10.89871531763	0\\
10.8962382351555	0\\
10.8937604460535	0\\
10.8912819499209	0\\
10.8888027463538	0\\
10.8863228349485	0\\
10.8838422153007	0\\
10.8813608870055	0\\
10.8788788496581	0\\
10.8763961028532	0\\
10.8739126461849	0\\
10.8714284792473	0\\
10.868943601634	0\\
10.8664580129382	0\\
10.8639717127529	0\\
10.8614847006706	0\\
10.8589969762836	0\\
10.8565085391836	0\\
10.8540193889622	0\\
10.8515295252106	0\\
10.8490389475195	0\\
10.8465476554795	0\\
10.8440556486806	0\\
10.8415629267125	0\\
10.8390694891647	0\\
10.8365753356262	0\\
10.8340804656858	0\\
10.8315848789316	0\\
10.8290885749518	0\\
10.8265915533339	0\\
10.8240938136652	0\\
10.8215953555326	0\\
10.8190961785227	0\\
10.8165962822216	0\\
10.8140956662152	0\\
10.8115943300891	0\\
10.8090922734282	0\\
10.8065894958175	0\\
10.8040859968412	0\\
10.8015817760835	0\\
10.7990768331279	0\\
10.796571167558	0\\
10.7940647789565	0\\
10.7915576669062	0\\
10.7890498309892	0\\
10.7865412707875	0\\
10.7840319858826	0\\
10.7815219758555	0\\
10.7790112402872	0\\
10.7764997787581	0\\
10.7739875908481	0\\
10.7714746761371	0\\
10.7689610342043	0\\
10.7664466646288	0\\
10.7639315669891	0\\
10.7614157408634	0\\
10.7588991858297	0\\
10.7563819014655	0\\
10.7538638873478	0\\
10.7513451430535	0\\
10.748825668159	0\\
10.7463054622402	0\\
10.7437845248729	0\\
10.7412628556323	0\\
10.7387404540935	0\\
10.7362173198308	0\\
10.7336934524186	0\\
10.7311688514306	0\\
10.7286435164403	0\\
10.7261174470207	0\\
10.7235906427445	0\\
10.7210631031841	0\\
10.7185348279114	0\\
10.716005816498	0\\
10.713476068515	0\\
10.7109455835334	0\\
10.7084143611235	0\\
10.7058824008554	0\\
10.7033497022989	0\\
10.7008162650232	0\\
10.6982820885973	0\\
10.6957471725898	0\\
10.6932115165689	0\\
10.6906751201023	0\\
10.6881379827575	0\\
10.6856001041016	0\\
10.6830614837013	0\\
10.6805221211227	0\\
10.6779820159319	0\\
10.6754411676944	0\\
10.6728995759753	0\\
10.6703572403393	0\\
10.667814160351	0\\
10.6652703355743	0\\
10.6627257655728	0\\
10.6601804499097	0\\
10.6576343881479	0\\
10.65508757985	0\\
10.6525400245778	0\\
10.6499917218933	0\\
10.6474426713575	0\\
10.6448928725316	0\\
10.6423423249761	0\\
10.639791028251	0\\
10.6372389819162	0\\
10.634686185531	0\\
10.6321326386544	0\\
10.6295783408451	0\\
10.6270232916612	0\\
10.6244674906605	0\\
10.6219109374006	0\\
10.6193536314383	0\\
10.6167955723304	0\\
10.6142367596331	0\\
10.6116771929024	0\\
10.6091168716935	0\\
10.6065557955617	0\\
10.6039939640616	0\\
10.6014313767476	0\\
10.5988680331734	0\\
10.5963039328926	0\\
10.5937390754583	0\\
10.5911734604233	0\\
10.5886070873397	0\\
10.5860399557596	0\\
10.5834720652345	0\\
10.5809034153154	0\\
10.5783340055531	0\\
10.5757638354979	0\\
10.5731929046998	0\\
10.5706212127082	0\\
10.5680487590724	0\\
10.5654755433409	0\\
10.5629015650621	0\\
10.560326823784	0\\
10.5577513190541	0\\
10.5551750504194	0\\
10.5525980174268	0\\
10.5500202196224	0\\
10.5474416565522	0\\
10.5448623277618	0\\
10.5422822327961	0\\
10.5397013711999	0\\
10.5371197425175	0\\
10.5345373462927	0\\
10.531954182069	0\\
10.5293702493895	0\\
10.5267855477969	0\\
10.5242000768333	0\\
10.5216138360406	0\\
10.5190268249602	0\\
10.5164390431333	0\\
10.5138504901002	0\\
10.5112611654013	0\\
10.5086710685764	0\\
10.5060801991648	0\\
10.5034885567054	0\\
10.5008961407369	0\\
10.4983029507973	0\\
10.4957089864244	0\\
10.4931142471555	0\\
10.4905187325275	0\\
10.4879224420768	0\\
10.4853253753396	0\\
10.4827275318514	0\\
10.4801289111474	0\\
10.4775295127626	0\\
10.4749293362313	0\\
10.4723283810874	0\\
10.4697266468646	0\\
10.4671241330959	0\\
10.464520839314	0\\
10.4619167650514	0\\
10.4593119098397	0\\
10.4567062732106	0\\
10.454099854695	0\\
10.4514926538235	0\\
10.4488846701264	0\\
10.4462759031333	0\\
10.4436663523737	0\\
10.4410560173764	0\\
10.43844489767	0\\
10.4358329927825	0\\
10.4332203022415	0\\
10.4306068255742	0\\
10.4279925623075	0\\
10.4253775119677	0\\
10.4227616740807	0\\
10.4201450481721	0\\
10.4175276337668	0\\
10.4149094303896	0\\
10.4122904375647	0\\
10.4096706548158	0\\
10.4070500816664	0\\
10.4044287176392	0\\
10.4018065622569	0\\
10.3991836150414	0\\
10.3965598755144	0\\
10.3939353431971	0\\
10.3913100176102	0\\
10.3886838982741	0\\
10.3860569847086	0\\
10.3834292764332	0\\
10.3808007729669	0\\
10.3781714738283	0\\
10.3755413785355	0\\
10.3729104866062	0\\
10.3702787975577	0\\
10.3676463109069	0\\
10.36501302617	0\\
10.3623789428631	0\\
10.3597440605016	0\\
10.3571083786006	0\\
10.3544718966747	0\\
10.3518346142382	0\\
10.3491965308046	0\\
10.3465576458875	0\\
10.3439179589995	0\\
10.3412774696531	0\\
10.3386361773603	0\\
10.3359940816325	0\\
10.3333511819809	0\\
10.330707477916	0\\
10.3280629689481	0\\
10.3254176545868	0\\
10.3227715343415	0\\
10.320124607721	0\\
10.3174768742336	0\\
10.3148283333873	0\\
10.3121789846896	0\\
10.3095288276475	0\\
10.3068778617676	0\\
10.304226086556	0\\
10.3015735015185	0\\
10.2989201061601	0\\
10.2962658999857	0\\
10.2936108824996	0\\
10.2909550532057	0\\
10.2882984116074	0\\
10.2856409572076	0\\
10.2829826895087	0\\
10.280323608013	0\\
10.2776637122219	0\\
10.2750030016365	0\\
10.2723414757575	0\\
10.2696791340851	0\\
10.2670159761191	0\\
10.2643520013588	0\\
10.2616872093029	0\\
10.2590215994498	0\\
10.2563551712975	0\\
10.2536879243433	0\\
10.2510198580842	0\\
10.2483509720168	0\\
10.2456812656371	0\\
10.2430107384406	0\\
10.2403393899225	0\\
10.2376672195775	0\\
10.2349942268997	0\\
10.2323204113828	0\\
10.22964577252	0\\
10.2269703098043	0\\
10.2242940227278	0\\
10.2216169107824	0\\
10.2189389734595	0\\
10.2162602102499	0\\
10.2135806206442	0\\
10.2109002041322	0\\
10.2082189602035	0\\
10.2055368883471	0\\
10.2028539880515	0\\
10.2001702588047	0\\
10.1974857000944	0\\
10.1948003114076	0\\
10.1921140922311	0\\
10.189427042051	0\\
10.1867391603529	0\\
10.1840504466221	0\\
10.1813609003432	0\\
10.1786705210007	0\\
10.1759793080781	0\\
10.1732872610589	0\\
10.1705943794257	0\\
10.1679006626611	0\\
10.1652061102467	0\\
10.162510721664	0\\
10.1598144963938	0\\
10.1571174339166	0\\
10.1544195337123	0\\
10.1517207952602	0\\
10.1490212180394	0\\
10.1463208015283	0\\
10.1436195452048	0\\
10.1409174485465	0\\
10.1382145110304	0\\
10.1355107321329	0\\
10.1328061113301	0\\
10.1301006480974	0\\
10.1273943419101	0\\
10.1246871922425	0\\
10.1219791985688	0\\
10.1192703603625	0\\
10.1165606770967	0\\
10.113850148244	0\\
10.1111387732764	0\\
10.1084265516656	0\\
10.1057134828826	0\\
10.1029995663981	0\\
10.1002848016821	0\\
10.0975691882043	0\\
10.0948527254338	0\\
10.0921354128391	0\\
10.0894172498884	0\\
10.0866982360493	0\\
10.0839783707889	0\\
10.0812576535738	0\\
10.0785360838702	0\\
10.0758136611436	0\\
10.0730903848592	0\\
10.0703662544815	0\\
10.0676412694747	0\\
10.0649154293023	0\\
10.0621887334274	0\\
10.0594611813126	0\\
10.0567327724201	0\\
10.0540035062113	0\\
10.0512733821473	0\\
10.0485423996887	0\\
10.0458105582955	0\\
10.0430778574272	0\\
10.040344296543	0\\
10.0376098751012	0\\
10.0348745925599	0\\
10.0321384483766	0\\
10.0294014420082	0\\
10.0266635729113	0\\
10.0239248405418	0\\
10.021185244355	0\\
10.018444783806	0\\
10.0157034583491	0\\
10.0129612674382	0\\
10.0102182105267	0\\
10.0074742870675	0\\
10.0047294965128	0\\
10.0019838383145	0\\
9.9992373119238	0\\
9.9964899167916	0\\
9.99374165236808	0\\
9.99099251810298	0\\
9.98824251344551	0\\
9.98549163784435	0\\
9.98273989074765	0\\
9.97998727160306	0\\
9.97723377985768	0\\
9.9744794149581	0\\
9.97172417635035	0\\
9.96896806347999	0\\
9.96621107579201	0\\
9.96345321273088	0\\
9.96069447374054	0\\
9.95793485826441	0\\
9.95517436574536	0\\
9.95241299562575	0\\
9.94965074734741	0\\
9.9468876203516	0\\
9.94412361407909	0\\
9.9413587279701	0\\
9.9385929614643	0\\
9.93582631400085	0\\
9.93305878501836	0\\
9.9302903739549	0\\
9.92752108024802	0\\
9.9247509033347	0\\
9.92197984265143	0\\
9.9192078976341	0\\
9.91643506771812	0\\
9.91366135233831	0\\
9.91088675092897	0\\
9.90811126292387	0\\
9.90533488775621	0\\
9.90255762485866	0\\
9.89977947366336	0\\
9.89700043360188	0\\
9.89422050410525	0\\
9.89143968460397	0\\
9.88865797452797	0\\
9.88587537330666	0\\
9.88309188036886	0\\
9.88030749514287	0\\
9.87752221705646	0\\
9.8747360455368	0\\
9.87194898001054	0\\
9.86916101990377	0\\
9.86637216464203	0\\
9.86358241365031	0\\
9.86079176635303	0\\
9.85800022217407	0\\
9.85520778053675	0\\
9.85241444086384	0\\
9.84962020257753	0\\
9.84682506509949	0\\
9.84402902785079	0\\
9.84123209025198	0\\
9.83843425172302	0\\
9.83563551168331	0\\
9.83283586955172	0\\
9.83003532474653	0\\
9.82723387668545	0\\
9.82443152478566	0\\
9.82162826846374	0\\
9.81882410713574	0\\
9.8160190402171	0\\
9.81321306712274	0\\
9.81040618726698	0\\
9.80759840006358	0\\
9.80478970492575	0\\
9.8019801012661	0\\
9.79916958849669	0\\
9.79635816602901	0\\
9.79354583327397	0\\
9.79073258964191	0\\
9.78791843454259	0\\
9.78510336738521	0\\
9.78228738757839	0\\
9.77947049453016	0\\
9.77665268764801	0\\
9.77383396633881	0\\
9.77101433000889	0\\
9.76819377806396	0\\
9.7653723099092	0\\
9.76254992494917	0\\
9.75972662258787	0\\
9.75690240222871	0\\
9.75407726327452	0\\
9.75125120512756	0\\
9.74842422718949	0\\
9.74559632886138	0\\
9.74276750954373	0\\
9.73993776863645	0\\
9.73710710553887	0\\
9.73427551964971	0\\
9.73144301036713	0\\
9.72860957708868	0\\
9.72577521921133	0\\
9.72293993613145	0\\
9.72010372724484	0\\
9.71726659194669	0\\
9.7144285296316	0\\
9.71158953969357	0\\
9.70874962152602	0\\
9.70590877452178	0\\
9.70306699807305	0\\
9.70022429157146	0\\
9.69738065440805	0\\
9.69453608597324	0\\
9.69169058565686	0\\
9.68884415284814	0\\
9.68599678693572	0\\
9.68314848730761	0\\
9.68029925335126	0\\
9.67744908445347	0\\
9.67459798000048	0\\
9.67174593937788	0\\
9.66889296197071	0\\
9.66603904716334	0\\
9.66318419433959	0\\
9.66032840288264	0\\
9.65747167217508	0\\
9.65461400159885	0\\
9.65175539053535	0\\
9.6488958383653	0\\
9.64603534446886	0\\
9.64317390822553	0\\
9.64031152901423	0\\
9.63744820621327	0\\
9.63458393920032	0\\
9.63171872735245	0\\
9.6288525700461	0\\
9.62598546665711	0\\
9.6231174165607	0\\
9.62024841913146	0\\
9.61737847374336	0\\
9.61450757976976	0\\
9.61163573658338	0\\
9.60876294355634	0\\
9.60588920006013	0\\
9.60301450546561	0\\
9.60013885914301	0\\
9.59726226046195	0\\
9.59438470879142	0\\
9.59150620349977	0\\
9.58862674395473	0\\
9.5857463295234	0\\
9.58286495957226	0\\
9.57998263346714	0\\
9.57709935057326	0\\
9.57421511025519	0\\
9.57132991187687	0\\
9.56844375480161	0\\
9.56555663839209	0\\
9.56266856201035	0\\
9.55977952501779	0\\
9.55688952677517	0\\
9.55399856664262	0\\
9.55110664397962	0\\
9.54821375814503	0\\
9.54531990849705	0\\
9.54242509439325	0\\
9.53952931519054	0\\
9.53663257024521	0\\
9.53373485891289	0\\
9.53083618054858	0\\
9.5279365345066	0\\
9.52503592014067	0\\
9.52213433680383	0\\
9.51923178384847	0\\
9.51632826062636	0\\
9.51342376648859	0\\
9.51051830078561	0\\
9.50761186286722	0\\
9.50470445208257	0\\
9.50179606778013	0\\
9.49888670930776	0\\
9.49597637601262	0\\
9.49306506724125	0\\
9.49015278233952	0\\
9.48723952065261	0\\
9.4843252815251	0\\
9.48141006430087	0\\
9.47849386832315	0\\
9.4755766929345	0\\
9.47265853747682	0\\
9.46973940129137	0\\
9.46681928371871	0\\
9.46389818409876	0\\
9.46097610177075	0\\
9.45805303607328	0\\
9.45512898634425	0\\
9.45220395192089	0\\
9.44927793213979	0\\
9.44635092633684	0\\
9.44342293384727	0\\
9.44049395400563	0\\
9.43756398614581	0\\
9.43463302960102	0\\
9.43170108370379	0\\
9.42876814778598	0\\
9.42583422117877	0\\
9.42289930321266	0\\
9.41996339321747	0\\
9.41702649052235	0\\
9.41408859445576	0\\
9.41114970434548	0\\
9.40820981951861	0\\
9.40526893930157	0\\
9.40232706302008	0\\
9.39938418999919	0\\
9.39644031956326	0\\
9.39349545103596	0\\
9.39054958374028	0\\
9.3876027169985	0\\
9.38465485013224	0\\
9.38170598246241	0\\
9.37875611330922	0\\
9.37580524199222	0\\
9.37285336783023	0\\
9.36990049014139	0\\
9.36694660824315	0\\
9.36399172145225	0\\
9.36103582908475	0\\
9.35807893045601	0\\
9.35512102488066	0\\
9.35216211167267	0\\
9.34920219014528	0\\
9.34624125961104	0\\
9.34327931938181	0\\
9.34031636876871	0\\
9.33735240708219	0\\
9.33438743363197	0\\
9.33142144772708	0\\
9.32845444867584	0\\
9.32548643578584	0\\
9.32251740836399	0\\
9.31954736571646	0\\
9.31657630714873	0\\
9.31360423196556	0\\
9.310631139471	0\\
9.30765702896837	0\\
9.30468189976029	0\\
9.30170575114866	0\\
9.29872858243465	0\\
9.29575039291874	0\\
9.29277118190066	0\\
9.28979094867943	0\\
9.28680969255335	0\\
9.28382741281999	0\\
9.28084410877622	0\\
9.27785977971815	0\\
9.27487442494118	0\\
9.27188804374	0\\
9.26890063540855	0\\
9.26591219924006	0\\
9.26292273452699	0\\
9.25993224056112	0\\
9.25694071663347	0\\
9.25394816203434	0\\
9.25095457605328	0\\
9.24795995797912	0\\
9.24496430709995	0\\
9.24196762270311	0\\
9.23896990407524	0\\
9.23597115050218	0\\
9.2329713612691	0\\
9.22997053566037	0\\
9.22696867295966	0\\
9.22396577244987	0\\
9.22096183341316	0\\
9.21795685513096	0\\
9.21495083688394	0\\
9.21194377795203	0\\
9.20893567761441	0\\
9.20592653514951	0\\
9.20291634983501	0\\
9.19990512094783	0\\
9.19689284776416	0\\
9.19387952955941	0\\
9.19086516560826	0\\
9.18784975518462	0\\
9.18483329756164	0\\
9.18181579201173	0\\
9.17879723780652	0\\
9.1757776342169	0\\
9.17275698051299	0\\
9.16973527596414	0\\
9.16671251983895	0\\
9.16368871140526	0\\
9.16066384993012	0\\
9.15763793467985	0\\
9.15461096491997	0\\
9.15158293991525	0\\
9.14855385892969	0\\
9.14552372122651	0\\
9.14249252606817	0\\
9.13946027271636	0\\
9.13642696043197	0\\
9.13339258847515	0\\
9.13035715610525	0\\
9.12732066258086	0\\
9.12428310715978	0\\
9.12124448909903	0\\
9.11820480765486	0\\
9.11516406208273	0\\
9.11212225163734	0\\
9.10907937557256	0\\
9.10603543314153	0\\
9.10299042359656	0\\
9.0999443461892	0\\
9.0968972001702	0\\
9.09384898478954	0\\
9.09079969929637	0\\
9.0877493429391	0\\
9.08469791496531	0\\
9.0816454146218	0\\
9.07859184115457	0\\
9.07553719380884	0\\
9.07248147182902	0\\
9.06942467445872	0\\
9.06636680094075	0\\
9.06330785051714	0\\
9.0602478224291	0\\
9.05718671591705	0\\
9.05412453022058	0\\
9.05106126457851	0\\
9.04799691822883	0\\
9.04493149040874	0\\
9.04186498035463	0\\
9.03879738730206	0\\
9.0357287104858	0\\
9.03265894913982	0\\
9.02958810249724	0\\
9.02651616979039	0\\
9.0234431502508	0\\
9.02036904310915	0\\
9.01729384759532	0\\
9.01421756293839	0\\
9.01114018836658	0\\
9.00806172310732	0\\
9.00498216638722	0\\
9.00190151743205	0\\
8.99881977546675	0\\
8.99573693971547	0\\
8.99265300940149	0\\
8.9895679837473	0\\
8.98648186197454	0\\
8.98339464330402	0\\
8.98030632695573	0\\
8.97721691214881	0\\
8.9741263981016	0\\
8.97103478403156	0\\
8.96794206915535	0\\
8.96484825268878	0\\
8.96175333384682	0\\
8.95865731184361	0\\
8.95556018589242	0\\
8.95246195520572	0\\
8.94936261899512	0\\
8.94626217647136	0\\
8.94316062684439	0\\
8.94005796932325	0\\
8.93695420311618	0\\
8.93384932743055	0\\
8.93074334147288	0\\
8.92763624444884	0\\
8.92452803556326	0\\
8.9214187140201	0\\
8.91830827902247	0\\
8.91519672977262	0\\
8.91208406547195	0\\
8.908970285321	0\\
8.90585538851944	0\\
8.90273937426608	0\\
8.89962224175889	0\\
8.89650399019495	0\\
8.89338461877049	0\\
8.89026412668086	0\\
8.88714251312055	0\\
8.88401977728319	0\\
8.88089591836153	0\\
8.87777093554744	0\\
8.87464482803194	0\\
8.87151759500515	0\\
8.86838923565635	0\\
8.8652597491739	0\\
8.86212913474532	0\\
8.85899739155723	0\\
8.85586451879538	0\\
8.85273051564463	0\\
8.84959538128897	0\\
8.84645911491149	0\\
8.8433217156944	0\\
8.84018318281904	0\\
8.83704351546583	0\\
8.83390271281434	0\\
8.83076077404322	0\\
8.82761769833023	0\\
8.82447348485225	0\\
8.82132813278526	0\\
8.81818164130435	0\\
8.81503400958371	0\\
8.81188523679662	0\\
8.80873532211547	0\\
8.80558426471177	0\\
8.80243206375608	0\\
8.7992787184181	0\\
8.7961242278666	0\\
8.79296859126947	0\\
8.78981180779367	0\\
8.78665387660525	0\\
8.78349479686937	0\\
8.78033456775026	0\\
8.77717318841124	0\\
8.77401065801473	0\\
8.77084697572221	0\\
8.76768214069428	0\\
8.76451615209059	0\\
8.76134900906988	0\\
8.75818071078997	0\\
8.75501125640776	0\\
8.75184064507923	0\\
8.74866887595942	0\\
8.74549594820247	0\\
8.74232186096157	0\\
8.73914661338899	0\\
8.73597020463607	0\\
8.73279263385321	0\\
8.7296139001899	0\\
8.72643400279467	0\\
8.72325294081512	0\\
8.72007071339793	0\\
8.71688731968883	0\\
8.7137027588326	0\\
8.7105170299731	0\\
8.70733013225323	0\\
8.70414206481495	0\\
8.70095282679929	0\\
8.69776241734632	0\\
8.69457083559515	0\\
8.69137808068398	0\\
8.68818415175	0\\
8.68498904792951	0\\
8.68179276835782	0\\
8.67859531216928	0\\
8.67539667849731	0\\
8.67219686647436	0\\
8.66899587523192	0\\
8.66579370390052	0\\
8.66259035160971	0\\
8.65938581748811	0\\
8.65618010066336	0\\
8.65297320026214	0\\
8.64976511541013	0\\
8.64655584523208	0\\
8.64334538885177	0\\
8.64013374539197	0\\
8.63692091397451	0\\
8.63370689372024	0\\
8.63049168374902	0\\
8.62727528317975	0\\
8.62405769113033	0\\
8.62083890671771	0\\
8.61761892905782	0\\
8.61439775726564	0\\
8.61117539045515	0\\
8.60795182773933	0\\
8.60472706823021	0\\
8.60150111103879	0\\
8.59827395527509	0\\
8.59504560004816	0\\
8.59181604446603	0\\
8.58858528763575	0\\
8.58535332866337	0\\
8.58212016665393	0\\
8.57888580071149	0\\
8.57565022993909	0\\
8.57241345343878	0\\
8.5691754703116	0\\
8.56593627965759	0\\
8.56269588057578	0\\
8.55945427216418	0\\
8.5562114535198	0\\
8.55296742373865	0\\
8.54972218191571	0\\
8.54647572714495	0\\
8.54322805851931	0\\
8.53997917513075	0\\
8.53672907607016	0\\
8.53347776042746	0\\
8.5302252272915	0\\
8.52697147575015	0\\
8.52371650489023	0\\
8.52046031379752	0\\
8.51720290155681	0\\
8.51394426725182	0\\
8.51068440996527	0\\
8.50742332877883	0\\
8.50416102277313	0\\
8.50089749102778	0\\
8.49763273262134	0\\
8.49436674663134	0\\
8.49109953213426	0\\
8.48783108820555	0\\
8.4845614139196	0\\
8.48129050834976	0\\
8.47801837056833	0\\
8.47474499964658	0\\
8.47147039465471	0\\
8.46819455466187	0\\
8.46491747873616	0\\
8.46163916594462	0\\
8.45835961535325	0\\
8.45507882602696	0\\
8.45179679702964	0\\
8.44851352742409	0\\
8.44522901627205	0\\
8.44194326263421	0\\
8.43865626557017	0\\
8.43536802413848	0\\
8.43207853739662	0\\
8.42878780440099	0\\
8.42549582420693	0\\
8.42220259586869	0\\
8.41890811843944	0\\
8.41561239097131	0\\
8.4123154125153	0\\
8.40901718212136	0\\
8.40571769883836	0\\
8.40241696171406	0\\
8.39911496979516	0\\
8.39581172212725	0\\
8.39250721775486	0\\
8.38920145572141	0\\
8.38589443506921	0\\
8.38258615483951	0\\
8.37927661407245	0\\
8.37596581180707	0\\
8.37265374708131	0\\
8.36934041893201	0\\
8.36602582639491	0\\
8.36270996850464	0\\
8.35939284429474	0\\
8.35607445279762	0\\
8.35275479304459	0\\
8.34943386406586	0\\
8.34611166489051	0\\
8.34278819454651	0\\
8.33946345206073	0\\
8.33613743645889	0\\
8.33281014676562	0\\
8.32948158200442	0\\
8.32615174119765	0\\
8.32282062336658	0\\
8.31948822753132	0\\
8.31615455271087	0\\
8.31281959792309	0\\
8.30948336218471	0\\
8.30614584451135	0\\
8.30280704391745	0\\
8.29946695941636	0\\
8.29612559002025	0\\
8.29278293474018	0\\
8.28943899258605	0\\
8.28609376256663	0\\
8.28274724368953	0\\
8.27939943496123	0\\
8.27605033538703	0\\
8.27269994397112	0\\
8.2693482597165	0\\
8.26599528162504	0\\
8.26264100869744	0\\
8.25928543993326	0\\
8.25592857433087	0\\
8.2525704108875	0\\
8.24921094859923	0\\
8.24585018646093	0\\
8.24248812346634	0\\
8.23912475860803	0\\
8.23576009087738	0\\
8.2323941192646	0\\
8.22902684275875	0\\
8.22565826034769	0\\
8.2222883710181	0\\
8.21891717375549	0\\
8.2155446675442	0\\
8.21217085136736	0\\
8.20879572420693	0\\
8.20541928504369	0\\
8.20204153285721	0\\
8.19866246662589	0\\
8.19528208532692	0\\
8.19190038793631	0\\
8.18851737342887	0\\
8.1851330407782	0\\
8.1817473889567	0\\
8.1783604169356	0\\
8.17497212368488	0\\
8.17158250817335	0\\
8.16819156936859	0\\
8.16479930623699	0\\
8.16140571774371	0\\
8.1580108028527	0\\
8.15461456052671	0\\
8.15121698972725	0\\
8.14781808941463	0\\
8.14441785854793	0\\
8.14101629608501	0\\
8.13761340098249	0\\
8.1342091721958	0\\
8.13080360867911	0\\
8.12739670938535	0\\
8.12398847326626	0\\
8.1205788992723	0\\
8.11716798635273	0\\
8.11375573345555	0\\
8.11034213952752	0\\
8.10692720351418	0\\
8.10351092435979	0\\
8.10009330100739	0\\
8.09667433239876	0\\
8.09325401747444	0\\
8.08983235517372	0\\
8.08640934443461	0\\
8.0829849841939	0\\
8.0795592733871	0\\
8.07613221094845	0\\
8.07270379581096	0\\
8.06927402690635	0\\
8.06584290316509	0\\
8.06241042351636	0\\
8.05897658688809	0\\
8.05554139220695	0\\
8.05210483839829	0\\
8.04866692438624	0\\
8.0452276490936	0\\
8.04178701144193	0\\
8.03834501035149	0\\
8.03490164474126	0\\
8.03145691352892	0\\
8.02801081563088	0\\
8.02456334996226	0\\
8.02111451543688	0\\
8.01766431096725	0\\
8.01421273546463	0\\
8.01075978783892	0\\
8.00730546699878	0\\
8.00384977185152	0\\
8.00039270130317	0\\
7.99693425425845	0\\
7.99347442962076	0\\
7.99001322629221	0\\
7.98655064317358	0\\
7.98308667916433	0\\
7.97962133316263	0\\
7.97615460406529	0\\
7.97268649076784	0\\
7.96921699216447	0\\
7.96574610714802	0\\
7.96227383461005	0\\
7.95880017344075	0\\
7.955325122529	0\\
7.95184868076233	0\\
7.94837084702696	0\\
7.94489162020773	0\\
7.94141099918817	0\\
7.93792898285048	0\\
7.93444557007547	0\\
7.93096075974264	0\\
7.92747455073013	0\\
7.92398694191473	0\\
7.92049793217186	0\\
7.91700752037562	0\\
7.91351570539872	0\\
7.91002248611252	0\\
7.90652786138702	0\\
7.90303183009085	0\\
7.89953439109129	0\\
7.89603554325422	0\\
7.89253528544418	0\\
7.88903361652433	0\\
7.88553053535643	0\\
7.88202604080089	0\\
7.87852013171674	0\\
7.87501280696161	0\\
7.87150406539175	0\\
7.86799390586204	0\\
7.86448232722595	0\\
7.86096932833556	0\\
7.85745490804159	0\\
7.85393906519332	0\\
7.85042179863864	0\\
7.84690310722408	0\\
7.84338298979472	0\\
7.83986144519426	0\\
7.83633847226498	0\\
7.83281406984776	0\\
7.82928823678207	0\\
7.82576097190595	0\\
7.82223227405605	0\\
7.81870214206758	0\\
7.81517057477434	0\\
7.81163757100871	0\\
7.80810312960162	0\\
7.8045672493826	0\\
7.80102992917975	0\\
7.79749116781971	0\\
7.79395096412772	0\\
7.79040931692755	0\\
7.78686622504156	0\\
7.78332168729065	0\\
7.77977570249428	0\\
7.77622826947047	0\\
7.77267938703578	0\\
7.76912905400533	0\\
7.76557726919278	0\\
7.76202403141033	0\\
7.75846933946873	0\\
7.75491319217727	0\\
7.75135558834377	0\\
7.74779652677459	0\\
7.74423600627461	0\\
7.74067402564726	0\\
7.73711058369449	0\\
7.73354567921675	0\\
7.72997931101306	0\\
7.72641147788092	0\\
7.72284217861637	0\\
7.71927141201395	0\\
7.71569917686671	0\\
7.71212547196624	0\\
7.70855029610261	0\\
7.7049736480644	0\\
7.7013955266387	0\\
7.69781593061108	0\\
7.69423485876565	0\\
7.69065230988497	0\\
7.68706828275012	0\\
7.68348277614067	0\\
7.67989578883466	0\\
7.67630731960863	0\\
7.67271736723759	0\\
7.66912593049506	0\\
7.66553300815301	0\\
7.66193859898188	0\\
7.65834270175062	0\\
7.65474531522661	0\\
7.65114643817572	0\\
7.64754606936227	0\\
7.64394420754907	0\\
7.64034085149737	0\\
7.63673599996687	0\\
7.63312965171574	0\\
7.6295218055006	0\\
7.62591246007652	0\\
7.62230161419701	0\\
7.61868926661404	0\\
7.615075416078	0\\
7.61146006133774	0\\
7.60784320114055	0\\
7.60422483423212	0\\
7.60060495935661	0\\
7.5969835752566	0\\
7.59336068067308	0\\
7.58973627434548	0\\
7.58611035501164	0\\
7.58248292140784	0\\
7.57885397226875	0\\
7.57522350632746	0\\
7.5715915223155	0\\
7.56795801896275	0\\
7.56432299499756	0\\
7.56068644914663	0\\
7.55704838013509	0\\
7.55340878668646	0\\
7.54976766752266	0\\
7.54612502136398	0\\
7.54248084692912	0\\
7.53883514293517	0\\
7.5351879080976	0\\
7.53153914113024	0\\
7.52788884074532	0\\
7.52423700565346	0\\
7.52058363456361	0\\
7.51692872618312	0\\
7.51327227921771	0\\
7.50961429237145	0\\
7.50595476434678	0\\
7.5022936938445	0\\
7.49863107956376	0\\
7.49496692020208	0\\
7.4913012144553	0\\
7.48763396101764	0\\
7.48396515858165	0\\
7.48029480583823	0\\
7.47662290147661	0\\
7.47294944418435	0\\
7.46927443264738	0\\
7.46559786554992	0\\
7.46191974157454	0\\
7.45824005940213	0\\
7.45455881771191	0\\
7.45087601518141	0\\
7.44719165048648	0\\
7.44350572230128	0\\
7.4398182292983	0\\
7.43612917014832	0\\
7.43243854352043	0\\
7.42874634808202	0\\
7.42505258249879	0\\
7.42135724543473	0\\
7.41766033555212	0\\
7.41396185151154	0\\
7.41026179197185	0\\
7.40656015559021	0\\
7.40285694102204	0\\
7.39915214692106	0\\
7.39544577193926	0\\
7.39173781472689	0\\
7.38802827393249	0\\
7.38431714820286	0\\
7.38060443618306	0\\
7.37689013651643	0\\
7.37317424784454	0\\
7.36945676880724	0\\
7.36573769804262	0\\
7.36201703418703	0\\
7.35829477587505	0\\
7.35457092173953	0\\
7.35084547041153	0\\
7.34711842052038	0\\
7.34338977069361	0\\
7.33965951955702	0\\
7.3359276657346	0\\
7.33219420784859	0\\
7.32845914451945	0\\
7.32472247436586	0\\
7.32098419600471	0\\
7.3172443080511	0\\
7.31350280911836	0\\
7.309759697818	0\\
7.30601497275975	0\\
7.30226863255154	0\\
7.29852067579951	0\\
7.29477110110797	0\\
7.29101990707944	0\\
7.28726709231462	0\\
7.28351265541239	0\\
7.27975659496983	0\\
7.27599890958219	0\\
7.27223959784289	0\\
7.26847865834353	0\\
7.26471608967388	0\\
7.26095189042187	0\\
7.2571860591736	0\\
7.25341859451334	0\\
7.24964949502349	0\\
7.24587875928462	0\\
7.24210638587546	0\\
7.23833237337287	0\\
7.23455672035186	0\\
7.23077942538558	0\\
7.22700048704533	0\\
7.22321990390052	0\\
7.21943767451872	0\\
7.2156537974656	0\\
7.21186827130498	0\\
7.20808109459878	0\\
7.20429226590705	0\\
7.20050178378795	0\\
7.19670964679776	0\\
7.19291585349086	0\\
7.18912040241973	0\\
7.18532329213497	0\\
7.18152452118526	0\\
7.17772408811738	0\\
7.17392199147621	0\\
7.17011822980471	0\\
7.16631280164392	0\\
7.16250570553297	0\\
7.15869694000908	0\\
7.15488650360751	0\\
7.15107439486163	0\\
7.14726061230286	0\\
7.14344515446067	0\\
7.13962801986262	0\\
7.13580920703432	0\\
7.13198871449942	0\\
7.12816654077963	0\\
7.1243426843947	0\\
7.12051714386245	0\\
7.11668991769872	0\\
7.11286100441738	0\\
7.10903040253035	0\\
7.10519811054757	0\\
7.10136412697701	0\\
7.09752845032468	0\\
7.09369107909457	0\\
7.08985201178873	0\\
7.08601124690719	0\\
7.08216878294802	0\\
7.07832461840726	0\\
7.07447875177898	0\\
7.07063118155524	0\\
7.06678190622609	0\\
7.06293092427959	0\\
7.05907823420175	0\\
7.05522383447662	0\\
7.05136772358617	0\\
7.0475099000104	0\\
7.04365036222725	0\\
7.03978910871264	0\\
7.03592613794047	0\\
7.03206144838258	0\\
7.02819503850878	0\\
7.02432690678684	0\\
7.02045705168248	0\\
7.01658547165935	0\\
7.01271216517909	0\\
7.00883713070122	0\\
7.00496036668326	0\\
7.0010818715806	0\\
6.99720164384662	0\\
6.99331968193259	0\\
6.98943598428771	0\\
6.98555054935911	0\\
6.9816633755918	0\\
6.97777446142876	0\\
6.97388380531082	0\\
6.96999140567675	0\\
6.96609726096321	0\\
6.96220136960475	0\\
6.95830373003383	0\\
6.95440434068076	0\\
6.95050319997379	0\\
6.94660030633901	0\\
6.9426956582004	0\\
6.93878925397981	0\\
6.93488109209698	0\\
6.93097117096948	0\\
6.92705948901277	0\\
6.92314604464017	0\\
6.91923083626282	0\\
6.91531386228976	0\\
6.91139512112784	0\\
6.90747461118177	0\\
6.90355233085408	0\\
6.89962827854516	0\\
6.89570245265321	0\\
6.89177485157428	0\\
6.88784547370221	0\\
6.8839143174287	0\\
6.87998138114323	0\\
6.8760466632331	0\\
6.87211016208344	0\\
6.86817187607715	0\\
6.86423180359495	0\\
6.86028994301535	0\\
6.85634629271466	0\\
6.85240085106696	0\\
6.84845361644413	0\\
6.84450458721581	0\\
6.84055376174943	0\\
6.83660113841021	0\\
6.83264671556109	0\\
6.8286904915628	0\\
6.82473246477385	0\\
6.82077263355046	0\\
6.81681099624663	0\\
6.81284755121411	0\\
6.80888229680237	0\\
6.80491523135863	0\\
6.80094635322786	0\\
6.79697566075274	0\\
6.79300315227368	0\\
6.78902882612881	0\\
6.78505268065399	0\\
6.78107471418278	0\\
6.77709492504646	0\\
6.77311331157401	0\\
6.7691298720921	0\\
6.76514460492511	0\\
6.76115750839512	0\\
6.75716858082189	0\\
6.75317782052285	0\\
6.74918522581312	0\\
6.74519079500551	0\\
6.74119452641049	0\\
6.73719641833619	0\\
6.7331964690884	0\\
6.72919467697059	0\\
6.72519104028387	0\\
6.72118555732699	0\\
6.71717822639636	0\\
6.71316904578603	0\\
6.70915801378769	0\\
6.70514512869064	0\\
6.70113038878183	0\\
6.69711379234583	0\\
6.69309533766483	0\\
6.68907502301862	0\\
6.68505284668462	0\\
6.68102880693784	0\\
6.6770029020509	0\\
6.67297513029402	0\\
6.66894548993501	0\\
6.66491397923926	0\\
6.66088059646974	0\\
6.65684533988703	0\\
6.65280820774925	0\\
6.64876919831211	0\\
6.64472830982887	0\\
6.64068554055036	0\\
6.63664088872498	0\\
6.63259435259865	0\\
6.62854593041486	0\\
6.62449562041465	0\\
6.62044342083657	0\\
6.61638932991674	0\\
6.61233334588877	0\\
6.60827546698382	0\\
6.60421569143056	0\\
6.60015401745518	0\\
6.59609044328139	0\\
6.59202496713038	0\\
6.58795758722085	0\\
6.58388830176902	0\\
6.57981710898857	0\\
6.57574400709069	0\\
6.57166899428404	0\\
6.56759206877475	0\\
6.56351322876645	0\\
6.55943247246021	0\\
6.55534979805458	0\\
6.55126520374556	0\\
6.54717868772661	0\\
6.54309024818863	0\\
6.53899988331998	0\\
6.53490759130644	0\\
6.53081337033124	0\\
6.52671721857503	0\\
6.52261913421589	0\\
6.51851911542932	0\\
6.51441716038824	0\\
6.51031326726296	0\\
6.50620743422122	0\\
6.50209965942815	0\\
6.49798994104627	0\\
6.49387827723549	0\\
6.48976466615313	0\\
6.48564910595385	0\\
6.48153159478973	0\\
6.47741213081018	0\\
6.473290712162	0\\
6.46916733698935	0\\
6.46504200343374	0\\
6.46091470963403	0\\
6.45678545372642	0\\
6.45265423384447	0\\
6.44852104811905	0\\
6.44438589467839	0\\
6.44024877164801	0\\
6.43610967715078	0\\
6.43196860930688	0\\
6.42782556623378	0\\
6.42368054604628	0\\
6.41953354685647	0\\
6.41538456677373	0\\
6.41123360390473	0\\
6.40708065635344	0\\
6.40292572222108	0\\
6.39876879960617	0\\
6.39460988660449	0\\
6.39044898130908	0\\
6.38628608181024	0\\
6.38212118619553	0\\
6.37795429254974	0\\
6.37378539895492	0\\
6.36961450349035	0\\
6.36544160423254	0\\
6.36126669925523	0\\
6.35708978662938	0\\
6.35291086442317	0\\
6.34872993070198	0\\
6.3445469835284	0\\
6.34036202096223	0\\
6.33617504106044	0\\
6.33198604187722	0\\
6.32779502146391	0\\
6.32360197786905	0\\
6.31940690913836	0\\
6.31520981331469	0\\
6.3110106884381	0\\
6.30680953254577	0\\
6.30260634367204	0\\
6.2984011198484	0\\
6.29419385910348	0\\
6.28998455946303	0\\
6.28577321894996	0\\
6.28155983558426	0\\
6.27734440738307	0\\
6.27312693236063	0\\
6.26890740852828	0\\
6.26468583389448	0\\
6.26046220646476	0\\
6.25623652424176	0\\
6.25200878522519	0\\
6.24777898741184	0\\
6.24354712879557	0\\
6.23931320736733	0\\
6.23507722111509	0\\
6.2308391680239	0\\
6.22659904607586	0\\
6.22235685325011	0\\
6.21811258752283	0\\
6.21386624686721	0\\
6.2096178292535	0\\
6.20536733264895	0\\
6.20111475501783	0\\
6.19686009432141	0\\
6.19260334851798	0\\
6.1883445155628	0\\
6.18408359340815	0\\
6.17982058000328	0\\
6.17555547329442	0\\
6.17128827122478	0\\
6.16701897173452	0\\
6.16274757276077	0\\
6.15847407223763	0\\
6.15419846809614	0\\
6.14992075826426	0\\
6.14564094066692	0\\
6.14135901322597	0\\
6.13707497386017	0\\
6.13278882048523	0\\
6.12850055101375	0\\
6.12421016335523	0\\
6.11991765541609	0\\
6.11562302509965	0\\
6.11132627030608	0\\
6.10702738893248	0\\
6.10272637887279	0\\
6.09842323801784	0\\
6.09411796425531	0\\
6.08981055546975	0\\
6.08550100954257	0\\
6.08118932435199	0\\
6.07687549777309	0\\
6.0725595276778	0\\
6.06824141193484	0\\
6.06392114840978	0\\
6.05959873496498	0\\
6.05527416945963	0\\
6.05094744974971	0\\
6.04661857368799	0\\
6.04228753912402	0\\
6.03795434390416	0\\
6.03361898587153	0\\
6.029281462866	0\\
6.02494177272423	0\\
6.02059991327962	0\\
};
\addplot [color=mycolor1, forget plot]
  table[row sep=crcr]{%
6.02059991327962	0\\
6.01625588236234	0\\
6.01190967779927	0\\
6.00756129741405	0\\
6.00321073902705	0\\
5.99885800045534	0\\
5.99450307951274	0\\
5.99014597400975	0\\
5.9857866817536	0\\
5.98142520054819	0\\
5.97706152819413	0\\
5.97269566248871	0\\
5.96832760122589	0\\
5.9639573421963	0\\
5.95958488318725	0\\
5.95521022198268	0\\
5.9508333563632	0\\
5.94645428410605	0\\
5.94207300298513	0\\
5.93768951077094	0\\
5.93330380523062	0\\
5.92891588412793	0\\
5.92452574522321	0\\
5.92013338627345	0\\
5.91573880503219	0\\
5.91134199924958	0\\
5.90694296667236	0\\
5.90254170504382	0\\
5.89813821210385	0\\
5.89373248558886	0\\
5.88932452323186	0\\
5.88491432276236	0\\
5.88050188190645	0\\
5.87608719838673	0\\
5.87167026992234	0\\
5.86725109422891	0\\
5.86282966901862	0\\
5.85840599200013	0\\
5.8539800608786	0\\
5.84955187335568	0\\
5.84512142712952	0\\
5.84068871989473	0\\
5.83625374934238	0\\
5.83181651316002	0\\
5.82737700903165	0\\
5.82293523463771	0\\
5.81849118765509	0\\
5.81404486575709	0\\
5.80959626661346	0\\
5.80514538789036	0\\
5.80069222725036	0\\
5.79623678235243	0\\
5.79177905085193	0\\
5.78731903040063	0\\
5.78285671864666	0\\
5.77839211323453	0\\
5.77392521180512	0\\
5.76945601199565	0\\
5.76498451143972	0\\
5.76051070776726	0\\
5.75603459860452	0\\
5.75155618157411	0\\
5.74707545429493	0\\
5.74259241438222	0\\
5.7381070594475	0\\
5.7336193870986	0\\
5.72912939493966	0\\
5.72463708057106	0\\
5.72014244158949	0\\
5.71564547558789	0\\
5.71114618015548	0\\
5.70664455287769	0\\
5.70214059133624	0\\
5.69763429310906	0\\
5.69312565577031	0\\
5.68861467689039	0\\
5.68410135403588	0\\
5.6795856847696	0\\
5.67506766665053	0\\
5.67054729723387	0\\
5.66602457407099	0\\
5.66149949470943	0\\
5.6569720566929	0\\
5.65244225756125	0\\
5.64791009485051	0\\
5.64337556609283	0\\
5.6388386688165	0\\
5.63429940054592	0\\
5.62975775880162	0\\
5.62521374110026	0\\
5.62066734495455	0\\
5.61611856787334	0\\
5.61156740736152	0\\
5.60701386092011	0\\
5.60245792604615	0\\
5.59789960023276	0\\
5.59333888096911	0\\
5.58877576574041	0\\
5.5842102520279	0\\
5.57964233730886	0\\
5.57507201905658	0\\
5.57049929474035	0\\
5.56592416182548	0\\
5.56134661777325	0\\
5.55676666004095	0\\
5.55218428608182	0\\
5.54759949334509	0\\
5.54301227927594	0\\
5.53842264131548	0\\
5.53383057690079	0\\
5.52923608346488	0\\
5.52463915843667	0\\
5.520039799241	0\\
5.51543800329863	0\\
5.51083376802619	0\\
5.50622709083623	0\\
5.50161796913717	0\\
5.4970064003333	0\\
5.49239238182476	0\\
5.48777591100758	0\\
5.4831569852736	0\\
5.47853560201051	0\\
5.47391175860184	0\\
5.46928545242693	0\\
5.46465668086091	0\\
5.46002544127475	0\\
5.45539173103519	0\\
5.45075554750475	0\\
5.44611688804173	0\\
5.4414757500002	0\\
5.43683213072998	0\\
5.43218602757664	0\\
5.42753743788149	0\\
5.42288635898157	0\\
5.41823278820962	0\\
5.41357672289413	0\\
5.40891816035925	0\\
5.40425709792485	0\\
5.39959353290648	0\\
5.39492746261534	0\\
5.39025888435833	0\\
5.38558779543797	0\\
5.38091419315246	0\\
5.37623807479561	0\\
5.37155943765686	0\\
5.36687827902129	0\\
5.36219459616957	0\\
5.35750838637795	0\\
5.35281964691831	0\\
5.34812837505808	0\\
5.34343456806028	0\\
5.33873822318346	0\\
5.33403933768176	0\\
5.32933790880483	0\\
5.32463393379787	0\\
5.31992740990158	0\\
5.31521833435221	0\\
5.31050670438148	0\\
5.3057925172166	0\\
5.30107577008029	0\\
5.29635646019073	0\\
5.29163458476155	0\\
5.28691014100185	0\\
5.28218312611617	0\\
5.27745353730447	0\\
5.27272137176216	0\\
5.26798662668005	0\\
5.26324929924433	0\\
5.25850938663663	0\\
5.25376688603393	0\\
5.24902179460859	0\\
5.24427410952834	0\\
5.23952382795625	0\\
5.23477094705076	0\\
5.2300154639656	0\\
5.22525737584987	0\\
5.22049667984795	0\\
5.21573337309953	0\\
5.21096745273959	0\\
5.2061989158984	0\\
5.20142775970149	0\\
5.19665398126967	0\\
5.19187757771897	0\\
5.18709854616069	0\\
5.18231688370132	0\\
5.17753258744263	0\\
5.17274565448153	0\\
5.16795608191017	0\\
5.16316386681589	0\\
5.15836900628117	0\\
5.15357149738369	0\\
5.14877133719628	0\\
5.14396852278689	0\\
5.13916305121864	0\\
5.13435491954974	0\\
5.12954412483353	0\\
5.12473066411846	0\\
5.11991453444804	0\\
5.11509573286088	0\\
5.11027425639067	0\\
5.10545010206612	0\\
5.10062326691103	0\\
5.0957937479442	0\\
5.09096154217948	0\\
5.08612664662571	0\\
5.08128905828676	0\\
5.07644877416147	0\\
5.07160579124366	0\\
5.06676010652213	0\\
5.06191171698063	0\\
5.05706061959786	0\\
5.05220681134746	0\\
5.04735028919798	0\\
5.04249105011288	0\\
5.03762909105055	0\\
5.03276440896424	0\\
5.02789700080209	0\\
5.02302686350709	0\\
5.01815399401712	0\\
5.01327838926487	0\\
5.00840004617788	0\\
5.0035189616785	0\\
4.9986351326839	0\\
4.99374855610603	0\\
4.98885922885164	0\\
4.98396714782226	0\\
4.97907230991415	0\\
4.97417471201836	0\\
4.96927435102064	0\\
4.96437122380149	0\\
4.95946532723613	0\\
4.95455665819446	0\\
4.94964521354109	0\\
4.94473099013528	0\\
4.939813984831	0\\
4.93489419447683	0\\
4.92997161591602	0\\
4.92504624598644	0\\
4.92011808152058	0\\
4.91518711934554	0\\
4.910253356283	0\\
4.90531678914922	0\\
4.90037741475506	0\\
4.8954352299059	0\\
4.89049023140168	0\\
4.88554241603686	0\\
4.88059178060043	0\\
4.8756383218759	0\\
4.87068203664124	0\\
4.86572292166892	0\\
4.86076097372589	0\\
4.85579618957353	0\\
4.85082856596769	0\\
4.84585809965862	0\\
4.84088478739102	0\\
4.83590862590397	0\\
4.83092961193097	0\\
4.82594774219986	0\\
4.82096301343289	0\\
4.81597542234662	0\\
4.810984965652	0\\
4.80599164005425	0\\
4.80099544225295	0\\
4.79599636894197	0\\
4.79099441680946	0\\
4.78598958253785	0\\
4.78098186280383	0\\
4.77597125427834	0\\
4.77095775362656	0\\
4.76594135750788	0\\
4.76092206257591	0\\
4.75589986547845	0\\
4.75087476285749	0\\
4.74584675134917	0\\
4.74081582758382	0\\
4.73578198818586	0\\
4.73074522977388	0\\
4.72570554896057	0\\
4.72066294235272	0\\
4.71561740655121	0\\
4.71056893815098	0\\
4.70551753374105	0\\
4.70046318990447	0\\
4.69540590321833	0\\
4.69034567025373	0\\
4.68528248757579	0\\
4.68021635174359	0\\
4.67514725931021	0\\
4.67007520682269	0\\
4.66500019082201	0\\
4.65992220784308	0\\
4.65484125441474	0\\
4.64975732705973	0\\
4.64467042229467	0\\
4.63958053663008	0\\
4.63448766657033	0\\
4.62939180861363	0\\
4.62429295925202	0\\
4.61919111497138	0\\
4.61408627225138	0\\
4.60897842756548	0\\
4.60386757738091	0\\
4.59875371815868	0\\
4.59363684635352	0\\
4.5885169584139	0\\
4.58339405078202	0\\
4.57826811989376	0\\
4.57313916217871	0\\
4.56800717406009	0\\
4.56287215195483	0\\
4.55773409227347	0\\
4.55259299142017	0\\
4.54744884579273	0\\
4.54230165178251	0\\
4.53715140577447	0\\
4.53199810414715	0\\
4.52684174327262	0\\
4.52168231951648	0\\
4.51651982923787	0\\
4.51135426878942	0\\
4.50618563451726	0\\
4.50101392276098	0\\
4.49583912985363	0\\
4.49066125212172	0\\
4.48548028588515	0\\
4.48029622745728	0\\
4.47510907314482	0\\
4.46991881924789	0\\
4.46472546205995	0\\
4.45952899786783	0\\
4.45432942295166	0\\
4.44912673358493	0\\
4.4439209260344	0\\
4.43871199656011	0\\
4.43349994141537	0\\
4.42828475684677	0\\
4.4230664390941	0\\
4.41784498439039	0\\
4.41262038896184	0\\
4.40739264902789	0\\
4.4021617608011	0\\
4.39692772048722	0\\
4.39169052428509	0\\
4.38645016838673	0\\
4.38120664897723	0\\
4.37595996223475	0\\
4.37071010433056	0\\
4.36545707142895	0\\
4.36020085968727	0\\
4.35494146525587	0\\
4.34967888427812	0\\
4.34441311289038	0\\
4.33914414722194	0\\
4.33387198339509	0\\
4.32859661752502	0\\
4.32331804571986	0\\
4.31803626408063	0\\
4.31275126870124	0\\
4.30746305566844	0\\
4.30217162106186	0\\
4.29687696095396	0\\
4.29157907140998	0\\
4.28627794848799	0\\
4.28097358823883	0\\
4.27566598670608	0\\
4.2703551399261	0\\
4.26504104392793	0\\
4.25972369473336	0\\
4.25440308835685	0\\
4.24907922080552	0\\
4.24375208807916	0\\
4.23842168617019	0\\
4.23308801106365	0\\
4.22775105873718	0\\
4.22241082516099	0\\
4.21706730629786	0\\
4.21172049810313	0\\
4.20637039652464	0\\
4.20101699750274	0\\
4.1956602969703	0\\
4.19030029085262	0\\
4.18493697506748	0\\
4.17957034552507	0\\
4.17420039812802	0\\
4.16882712877135	0\\
4.16345053334244	0\\
4.15807060772103	0\\
4.15268734777923	0\\
4.14730074938144	0\\
4.14191080838436	0\\
4.13651752063699	0\\
4.13112088198059	0\\
4.12572088824865	0\\
4.12031753526689	0\\
4.11491081885324	0\\
4.10950073481782	0\\
4.10408727896289	0\\
4.0986704470829	0\\
4.09325023496438	0\\
4.08782663838599	0\\
4.0823996531185	0\\
4.07696927492469	0\\
4.07153549955945	0\\
4.06609832276966	0\\
4.06065774029421	0\\
4.055213747864	0\\
4.04976634120187	0\\
4.04431551602263	0\\
4.038861268033	0\\
4.03340359293163	0\\
4.02794248640903	0\\
4.02247794414759	0\\
4.01700996182155	0\\
4.01153853509696	0\\
4.0060636596317	0\\
4.00058533107541	0\\
3.99510354506949	0\\
3.98961829724712	0\\
3.98412958323315	0\\
3.97863739864418	0\\
3.97314173908845	0\\
3.96764260016589	0\\
3.96213997746803	0\\
3.95663386657806	0\\
3.95112426307073	0\\
3.94561116251239	0\\
3.94009456046091	0\\
3.93457445246574	0\\
3.92905083406778	0\\
3.92352370079947	0\\
3.91799304818467	0\\
3.91245887173873	0\\
3.90692116696839	0\\
3.9013799293718	0\\
3.89583515443849	0\\
3.89028683764935	0\\
3.88473497447658	0\\
3.87917956038374	0\\
3.87362059082563	0\\
3.86805806124835	0\\
3.86249196708923	0\\
3.85692230377683	0\\
3.85134906673091	0\\
3.84577225136241	0\\
3.8401918530734	0\\
3.83460786725713	0\\
3.82902028929791	0\\
3.82342911457117	0\\
3.81783433844339	0\\
3.8122359562721	0\\
3.80663396340583	0\\
3.80102835518412	0\\
3.79541912693748	0\\
3.78980627398735	0\\
3.78418979164612	0\\
3.77856967521707	0\\
3.77294591999435	0\\
3.76731852126297	0\\
3.76168747429876	0\\
3.75605277436839	0\\
3.75041441672926	0\\
3.74477239662957	0\\
3.73912670930825	0\\
3.7334773499949	0\\
3.72782431390986	0\\
3.72216759626411	0\\
3.71650719225924	0\\
3.7108430970875	0\\
3.7051753059317	0\\
3.69950381396522	0\\
3.69382861635198	0\\
3.6881497082464	0\\
3.68246708479342	0\\
3.67678074112842	0\\
3.67109067237723	0\\
3.66539687365609	0\\
3.65969934007163	0\\
3.65399806672085	0\\
3.64829304869108	0\\
3.64258428105997	0\\
3.63687175889545	0\\
3.63115547725573	0\\
3.62543543118923	0\\
3.61971161573461	0\\
3.61398402592069	0\\
3.60825265676648	0\\
3.60251750328108	0\\
3.59677856046373	0\\
3.59103582330375	0\\
3.58528928678051	0\\
3.57953894586339	0\\
3.5737847955118	0\\
3.5680268306751	0\\
3.56226504629264	0\\
3.55649943729364	0\\
3.55072999859724	0\\
3.54495672511247	0\\
3.53917961173816	0\\
3.53339865336299	0\\
3.52761384486541	0\\
3.52182518111362	0\\
3.51603265696559	0\\
3.51023626726895	0\\
3.50443600686105	0\\
3.49863187056885	0\\
3.49282385320897	0\\
3.4870119495876	0\\
3.48119615450051	0\\
3.475376462733	0\\
3.46955286905989	0\\
3.46372536824548	0\\
3.45789395504352	0\\
3.4520586241972	0\\
3.44621937043908	0\\
3.44037618849113	0\\
3.43452907306462	0\\
3.42867801886017	0\\
3.42282302056764	0\\
3.41696407286619	0\\
3.41110117042417	0\\
3.40523430789915	0\\
3.39936347993785	0\\
3.39348868117614	0\\
3.38760990623899	0\\
3.38172714974045	0\\
3.37584040628364	0\\
3.36994967046065	0\\
3.36405493685262	0\\
3.3581562000296	0\\
3.3522534545506	0\\
3.34634669496352	0\\
3.34043591580513	0\\
3.33452111160103	0\\
3.32860227686565	0\\
3.32267940610218	0\\
3.31675249380256	0\\
3.31082153444746	0\\
3.30488652250622	0\\
3.29894745243683	0\\
3.29300431868593	0\\
3.28705711568874	0\\
3.28110583786903	0\\
3.27515047963912	0\\
3.2691910353998	0\\
3.26322749954037	0\\
3.25725986643852	0\\
3.25128813046038	0\\
3.24531228596043	0\\
3.2393323272815	0\\
3.23334824875472	0\\
3.2273600446995	0\\
3.22136770942349	0\\
3.21537123722256	0\\
3.20937062238075	0\\
3.20336585917024	0\\
3.19735694185133	0\\
3.19134386467241	0\\
3.18532662186988	0\\
3.17930520766821	0\\
3.17327961627978	0\\
3.16724984190499	0\\
3.1612158787321	0\\
3.15517772093728	0\\
3.14913536268451	0\\
3.14308879812563	0\\
3.13703802140022	0\\
3.13098302663563	0\\
3.12492380794689	0\\
3.11886035943673	0\\
3.11279267519553	0\\
3.10672074930124	0\\
3.10064457581941	0\\
3.09456414880311	0\\
3.08847946229294	0\\
3.08239051031694	0\\
3.07629728689058	0\\
3.07019978601675	0\\
3.06409800168569	0\\
3.05799192787495	0\\
3.0518815585494	0\\
3.04576688766113	0\\
3.03964790914948	0\\
3.03352461694096	0\\
3.02739700494921	0\\
3.021265067075	0\\
3.01512879720618	0\\
3.00898818921761	0\\
3.00284323697117	0\\
2.9966939343157	0\\
2.99054027508696	0\\
2.9843822531076	0\\
2.97821986218713	0\\
2.97205309612187	0\\
2.96588194869491	0\\
2.9597064136761	0\\
2.95352648482198	0\\
2.94734215587573	0\\
2.9411534205672	0\\
2.9349602726128	0\\
2.92876270571549	0\\
2.92256071356476	0\\
2.91635428983655	0\\
2.91014342819325	0\\
2.90392812228364	0\\
2.89770836574284	0\\
2.89148415219233	0\\
2.88525547523981	0\\
2.87902232847927	0\\
2.87278470549087	0\\
2.86654259984093	0\\
2.8602960050819	0\\
2.85404491475231	0\\
2.84778932237672	0\\
2.8415292214657	0\\
2.83526460551576	0\\
2.82899546800935	0\\
2.82272180241478	0\\
2.81644360218621	0\\
2.81016086076359	0\\
2.80387357157263	0\\
2.79758172802473	0\\
2.79128532351699	0\\
2.78498435143214	0\\
2.77867880513848	0\\
2.77236867798985	0\\
2.76605396332563	0\\
2.75973465447063	0\\
2.7534107447351	0\\
2.74708222741466	0\\
2.74074909579025	0\\
2.73441134312814	0\\
2.7280689626798	0\\
2.72172194768195	0\\
2.71537029135645	0\\
2.70901398691028	0\\
2.7026530275355	0\\
2.6962874064092	0\\
2.68991711669347	0\\
2.68354215153533	0\\
2.67716250406669	0\\
2.67077816740435	0\\
2.66438913464989	0\\
2.65799539888966	0\\
2.65159695319474	0\\
2.64519379062089	0\\
2.63878590420849	0\\
2.63237328698251	0\\
2.62595593195246	0\\
2.61953383211234	0\\
2.61310698044061	0\\
2.60667536990012	0\\
2.60023899343809	0\\
2.59379784398602	0\\
2.58735191445971	0\\
2.58090119775916	0\\
2.57444568676854	0\\
2.56798537435613	0\\
2.56152025337431	0\\
2.55505031665947	0\\
2.54857555703198	0\\
2.54209596729615	0\\
2.53561154024018	0\\
2.52912226863609	0\\
2.52262814523969	0\\
2.51612916279054	0\\
2.50962531401188	0\\
2.5031165916106	0\\
2.49660298827719	0\\
2.49008449668564	0\\
2.4835611094935	0\\
2.47703281934172	0\\
2.47049961885464	0\\
2.46396150063997	0\\
2.45741845728871	0\\
2.45087048137509	0\\
2.44431756545653	0\\
2.43775970207362	0\\
2.43119688375002	0\\
2.42462910299243	0\\
2.41805635229055	0\\
2.411478624117	0\\
2.40489591092731	0\\
2.39830820515982	0\\
2.39171549923568	0\\
2.38511778555873	0\\
2.37851505651553	0\\
2.37190730447524	0\\
2.36529452178959	0\\
2.35867670079283	0\\
2.35205383380169	0\\
2.34542591311529	0\\
2.33879293101512	0\\
2.33215487976497	0\\
2.32551175161089	0\\
2.3188635387811	0\\
2.312210233486	0\\
2.30555182791803	0\\
2.29888831425169	0\\
2.29221968464346	0\\
2.28554593123173	0\\
2.27886704613674	0\\
2.27218302146056	0\\
2.26549384928701	0\\
2.2587995216816	0\\
2.25210003069149	0\\
2.24539536834541	0\\
2.23868552665363	0\\
2.23197049760788	0\\
2.22525027318131	0\\
2.21852484532841	0\\
2.21179420598498	0\\
2.20505834706806	0\\
2.19831726047587	0\\
2.19157093808773	0\\
2.18481937176407	0\\
2.17806255334627	0\\
2.17130047465669	0\\
2.16453312749857	0\\
2.15776050365597	0\\
2.15098259489373	0\\
2.14419939295737	0\\
2.13741088957308	0\\
2.13061707644763	0\\
2.12381794526831	0\\
2.11701348770287	0\\
2.11020369539948	0\\
2.10338855998663	0\\
2.09656807307311	0\\
2.0897422262479	0\\
2.08291101108016	0\\
2.07607441911914	0\\
2.0692324418941	0\\
2.06238507091428	0\\
2.05553229766883	0\\
2.04867411362673	0\\
2.04181051023673	0\\
2.03494147892733	0\\
2.02806701110661	0\\
2.02118709816231	0\\
2.01430173146163	0\\
2.00741090235126	0\\
2.00051460215725	0\\
1.993612822185	0\\
1.98670555371916	0\\
1.97979278802355	0\\
1.97287451634114	0\\
1.96595072989395	0\\
1.959021419883	0\\
1.95208657748822	0\\
1.9451461938684	0\\
1.93820026016113	0\\
1.93124876748271	0\\
1.92429170692811	0\\
1.91732906957085	0\\
1.91036084646302	0\\
1.9033870286351	0\\
1.896407607096	0\\
1.88942257283289	0\\
1.88243191681123	0\\
1.8754356299746	0\\
1.8684337032447	0\\
1.86142612752127	0\\
1.85441289368198	0\\
1.84739399258241	0\\
1.84036941505594	0\\
1.83333915191369	0\\
1.82630319394446	0\\
1.81926153191463	0\\
1.81221415656813	0\\
1.80516105862633	0\\
1.79810222878796	0\\
1.79103765772908	0\\
1.78396733610298	0\\
1.77689125454008	0\\
1.76980940364792	0\\
1.76272177401103	0\\
1.75562835619085	0\\
1.74852914072571	0\\
1.74142411813071	0\\
1.73431327889765	0\\
1.72719661349497	0\\
1.72007411236764	0\\
1.71294576593713	0\\
1.7058115646013	0\\
1.69867149873432	0\\
1.69152555868662	0\\
1.68437373478478	0\\
1.67721601733146	0\\
1.67005239660535	0\\
1.66288286286104	0\\
1.655707406329	0\\
1.64852601721544	0\\
1.64133868570226	0\\
1.63414540194698	0\\
1.62694615608265	0\\
1.61974093821774	0\\
1.61252973843612	0\\
1.60531254679689	0\\
1.59808935333441	0\\
1.59086014805812	0\\
1.5836249209525	0\\
1.57638366197697	0\\
1.56913636106585	0\\
1.56188300812821	0\\
1.55462359304784	0\\
1.54735810568313	0\\
1.54008653586701	0\\
1.53280887340684	0\\
1.52552510808435	0\\
1.51823522965555	0\\
1.51093922785061	0\\
1.50363709237383	0\\
1.4963288129035	0\\
1.48901437909182	0\\
1.48169378056488	0\\
1.47436700692245	0\\
1.46703404773802	0\\
1.45969489255861	0\\
1.45234953090473	0\\
1.4449979522703	0\\
1.43764014612251	0\\
1.43027610190178	0\\
1.42290580902166	0\\
1.41552925686869	0\\
1.40814643480239	0\\
1.4007573321551	0\\
1.39336193823191	0\\
1.38596024231059	0\\
1.37855223364144	0\\
1.37113790144726	0\\
1.36371723492323	0\\
1.3562902232368	0\\
1.34885685552761	0\\
1.34141712090741	0\\
1.33397100845991	0\\
1.32651850724076	0\\
1.31905960627739	0\\
1.31159429456897	0\\
1.30412256108624	0\\
1.29664439477148	0\\
1.28915978453837	0\\
1.28166871927192	0\\
1.27417118782835	0\\
1.26666717903499	0\\
1.25915668169021	0\\
1.25163968456326	0\\
1.24411617639425	0\\
1.23658614589398	0\\
1.22904958174387	0\\
1.22150647259584	0\\
1.21395680707223	0\\
1.2064005737657	0\\
1.1988377612391	0\\
1.19126835802536	0\\
1.18369235262742	0\\
1.17610973351814	0\\
1.16852048914011	0\\
1.16092460790564	0\\
1.15332207819658	0\\
1.14571288836429	0\\
1.13809702672945	0\\
1.13047448158201	0\\
1.12284524118105	0\\
1.1152092937547	0\\
1.1075666275	0\\
1.09991723058283	0\\
1.09226109113775	0\\
1.08459819726795	0\\
1.07692853704505	0\\
1.06925209850911	0\\
1.06156886966839	0\\
1.05387883849936	0\\
1.04618199294647	0\\
1.03847832092213	0\\
1.03076781030655	0\\
1.02305044894763	0\\
1.01532622466085	0\\
1.00759512522916	0\\
0.999857138402851	0\\
0.992112251899463	0\\
0.984360453403633	0\\
0.976601730567003	0\\
0.968836071008089	0\\
0.961063462312181	0\\
0.9532838920312	0\\
0.945497347683591	0\\
0.937703816754201	0\\
0.929903286694166	0\\
0.922095744920774	0\\
0.914281178817354	0\\
0.906459575733148	0\\
0.898630922983201	0\\
0.89079520784822	0\\
0.882952417574458	0\\
0.875102539373588	0\\
0.86724556042259	0\\
0.859381467863603	0\\
0.851510248803814	0\\
0.843631890315324	0\\
0.835746379435035	0\\
0.827853703164501	0\\
0.819953848469813	0\\
0.812046802281461	0\\
0.804132551494222	0\\
0.796211082967008	0\\
0.788282383522744	0\\
0.780346439948238	0\\
0.772403238994056	0\\
0.764452767374369	0\\
0.756495011766839	0\\
0.748529958812472	0\\
0.740557595115499	0\\
0.732577907243223	0\\
0.724590881725892	0\\
0.716596505056562	0\\
0.708594763690966	0\\
0.700585644047363	0\\
0.692569132506409	0\\
0.684545215411011	0\\
0.676513879066207	0\\
0.668475109738995	0\\
0.660428893658215	0\\
0.652375217014397	0\\
0.644314065959631	0\\
0.636245426607408	0\\
0.628169285032484	0\\
0.620085627270735	0\\
0.611994439319021	0\\
0.603895707135025	0\\
0.595789416637114	0\\
0.587675553704192	0\\
0.57955410417556	0\\
0.571425053850753	0\\
0.563288388489399	0\\
0.555144093811068	0\\
0.54699215549513	0\\
0.538832559180588	0\\
0.530665290465937	0\\
0.522490334909004	0\\
0.514307678026813	0\\
0.506117305295405	0\\
0.497919202149701	0\\
0.489713353983338	0\\
0.481499746148525	0\\
0.473278363955869	0\\
0.465049192674231	0\\
0.456812217530555	0\\
0.44856742370973	0\\
0.440314796354406	0\\
0.432054320564846	0\\
0.42378598139876	0\\
0.415509763871157	0\\
0.407225652954157	0\\
0.398933633576848	0\\
0.390633690625111	0\\
0.382325808941456	0\\
0.374009973324867	0\\
0.365686168530619	0\\
0.357354379270116	0\\
0.349014590210723	0\\
0.340666785975607	0\\
0.33231095114355	0\\
0.323947070248783	0\\
0.315575127780819	0\\
0.307195108184285	0\\
0.298806995858732	0\\
0.290410775158476	0\\
0.282006430392411	0\\
0.273593945823851	0\\
0.265173305670332	0\\
0.256744494103446	0\\
0.248307495248658	0\\
0.239862293185139	0\\
0.231408871945565	0\\
0.222947215515951	0\\
0.214477307835461	0\\
0.205999132796239	0\\
0.197512674243204	0\\
0.189017915973881	0\\
0.180514841738204	0\\
0.172003435238351	0\\
0.163483680128529	0\\
0.154955560014801	0\\
0.146419058454891	0\\
0.137874158958009	0\\
0.129320844984636	0\\
0.120759099946345	0\\
0.112188907205608	0\\
0.103610250075606	0\\
0.0950231118200223	0\\
0.0864274756528535	0\\
0.0778233247382095	0\\
0.0692106421901298	0\\
0.0605894110723611	0\\
0.0519596143981738	0\\
0.0433212351301526	0\\
0.0346742561800106	0\\
0.0260186604083634	0\\
0.0173544306245402	0\\
0.00868154958637186	0\\
0	0\\
-0.00869023548035287	0\\
-0.0173891742525778	0\\
-0.0260968337668847	0\\
-0.0348132315260254	0\\
-0.04353838508549	0\\
-0.0522723120537338	0\\
-0.0610150300923756	0\\
-0.0697665569164269	0\\
-0.0785269102944925	0\\
-0.0872961080490018	0\\
-0.096074168056411	0\\
-0.104861108247438	0\\
-0.113656946607264	0\\
-0.122461701175776	0\\
-0.131275390047764	0\\
-0.14009803137317	0\\
-0.148929643357287	0\\
-0.157770244261007	0\\
-0.166619852401029	0\\
-0.175478486150103	0\\
-0.184346163937243	0\\
-0.193222904247971	0\\
-0.202108725624538	0\\
-0.211003646666164	0\\
-0.219907686029263	0\\
-0.22882086242769	0\\
-0.237743194632961	0\\
-0.246674701474509	0\\
-0.255615401839902	0\\
-0.264565314675103	0\\
-0.273524458984693	0\\
-0.282492853832127	0\\
-0.291470518339966	0\\
-0.300457471690133	0\\
-0.309453733124148	0\\
-0.318459321943385	0\\
-0.327474257509309	0\\
-0.336498559243741	0\\
-0.345532246629092	0\\
-0.354575339208632	0\\
-0.363627856586727	0\\
-0.372689818429112	0\\
-0.381761244463128	0\\
-0.390842154477999	0\\
-0.399932568325073	0\\
-0.409032505918098	0\\
-0.418141987233471	0\\
-0.427261032310513	0\\
-0.436389661251721	0\\
-0.445527894223045	0\\
-0.454675751454146	0\\
-0.463833253238675	0\\
-0.473000419934531	0\\
-0.482177271964145	0\\
-0.49136382981474	0\\
-0.500560114038621	0\\
-0.509766145253432	0\\
-0.518981944142453	0\\
-0.528207531454861	0\\
-0.537442928006027	0\\
-0.546688154677781	0\\
-0.555943232418711	0\\
-0.565208182244434	0\\
-0.574483025237896	0\\
-0.583767782549644	0\\
-0.593062475398133	0\\
-0.602367125070001	0\\
-0.611681752920373	0\\
-0.621006380373147	0\\
-0.630341028921298	0\\
-0.639685720127164	0\\
-0.649040475622759	0\\
-0.658405317110058	0\\
-0.667780266361314	0\\
-0.677165345219347	0\\
-0.686560575597867	0\\
-0.695965979481758	0\\
-0.705381578927414	0\\
-0.714807396063021	0\\
-0.724243453088895	0\\
-0.733689772277774	0\\
-0.74314637597515	0\\
-0.752613286599578	0\\
-0.762090526642991	0\\
-0.771578118671034	0\\
-0.78107608532337	0\\
-0.790584449314021	0\\
-0.800103233431675	0\\
-0.809632460540035	0\\
-0.819172153578127	0\\
-0.828722335560651	0\\
-0.838283029578296	0\\
-0.847854258798095	0\\
-0.857436046463737	0\\
-0.867028415895933	0\\
-0.876631390492733	0\\
-0.886244993729884	0\\
-0.895869249161164	0\\
-0.90550418041874	0\\
-0.915149811213501	0\\
-0.924806165335424	0\\
-0.934473266653911	0\\
-0.944151139118158	0\\
-0.953839806757495	0\\
-0.96353929368176	0\\
-0.973249624081646	0\\
-0.982970822229071	0\\
-0.992702912477538	0\\
-1.0024459192625	0\\
-1.01219986710174	0\\
-1.02196478059573	0\\
-1.03174068442798	0\\
-1.04152760336547	0\\
-1.05132556225898	0\\
-1.06113458604349	0\\
-1.07095469973854	0\\
-1.08078592844863	0\\
-1.0906282973636	0\\
-1.10048183175904	0\\
-1.11034655699663	0\\
-1.12022249852456	0\\
-1.13010968187795	0\\
-1.14000813267919	0\\
-1.14991787663838	0\\
-1.15983893955373	0\\
-1.16977134731194	0\\
-1.17971512588861	0\\
-1.18967030134865	0\\
-1.19963689984674	0\\
-1.20961494762763	0\\
-1.21960447102667	0\\
-1.22960549647016	0\\
-1.23961805047579	0\\
-1.24964215965307	0\\
-1.25967785070372	0\\
-1.26972515042213	0\\
-1.27978408569581	0\\
-1.28985468350574	0\\
-1.2999369709269	0\\
-1.31003097512864	0\\
-1.32013672337515	0\\
-1.33025424302589	0\\
-1.34038356153604	0\\
-1.35052470645693	0\\
-1.36067770543655	0\\
-1.3708425862199	0\\
-1.38101937664954	0\\
-1.391208104666	0\\
-1.40140879830824	0\\
-1.41162148571414	0\\
-1.42184619512095	0\\
-1.43208295486572	0\\
-1.44233179338586	0\\
-1.45259273921953	0\\
-1.46286582100615	0\\
-1.4731510674869	0\\
-1.48344850750515	0\\
-1.49375817000701	0\\
-1.50408008404176	0\\
-1.51441427876237	0\\
-1.52476078342599	0\\
-1.53511962739447	0\\
-1.5454908401348	0\\
-1.55587445121967	0\\
-1.56627049032796	0\\
-1.57667898724522	0\\
-1.58709997186425	0\\
-1.59753347418552	0\\
-1.60797952431778	0\\
-1.61843815247852	0\\
-1.62890938899453	0\\
-1.6393932643024	0\\
-1.64988980894907	0\\
-1.66039905359235	0\\
-1.6709210290015	0\\
-1.68145576605768	0\\
-1.6920032957546	0\\
-1.70256364919899	0\\
-1.71313685761119	0\\
-1.72372295232567	0\\
-1.73432196479163	0\\
-1.74493392657354	0\\
-1.75555886935169	0\\
-1.76619682492278	0\\
-1.77684782520047	0\\
-1.78751190221597	0\\
-1.79818908811864	0\\
-1.80887941517649	0\\
-1.81958291577688	0\\
-1.830299622427	0\\
-1.84102956775455	0\\
-1.85177278450828	0\\
-1.86252930555859	0\\
-1.87329916389819	0\\
-1.88408239264263	0\\
-1.89487902503097	0\\
-1.90568909442638	0\\
-1.91651263431673	0\\
-1.92734967831525	0\\
-1.93820026016113	0\\
-1.94906441372017	0\\
-1.95994217298541	0\\
-1.97083357207775	0\\
-1.98173864524662	0\\
-1.99265742687059	0\\
-2.00358995145807	0\\
-2.01453625364792	0\\
-2.02549636821013	0\\
-2.03647033004647	0\\
-2.04745817419117	0\\
-2.05845993581159	0\\
-2.06947565020889	0\\
-2.08050535281871	0\\
-2.09154907921184	0\\
-2.10260686509495	0\\
-2.11367874631123	0\\
-2.12476475884113	0\\
-2.13586493880304	0\\
-2.14697932245399	0\\
-2.15810794619039	0\\
-2.16925084654871	0\\
-2.18040806020622	0\\
-2.19157962398171	0\\
-2.20276557483623	0\\
-2.21396594987379	0\\
-2.22518078634215	0\\
-2.2364101216335	0\\
-2.24765399328528	0\\
-2.25891243898086	0\\
-2.27018549655036	0\\
-2.28147320397138	0\\
-2.29277559936976	0\\
-2.30409272102038	0\\
-2.31542460734792	0\\
-2.32677129692765	0\\
-2.3381328284862	0\\
-2.34950924090239	0\\
-2.36090057320799	0\\
-2.37230686458854	0\\
-2.38372815438417	0\\
-2.39516448209039	0\\
-2.40661588735893	0\\
-2.41808240999854	0\\
-2.42956408997587	0\\
-2.44106096741623	0\\
-2.45257308260452	0\\
-2.46410047598599	0\\
-2.47564318816716	0\\
-2.48720125991663	0\\
-2.498774732166	0\\
-2.51036364601067	0\\
-2.52196804271077	0\\
-2.53358796369202	0\\
-2.54522345054662	0\\
-2.55687454503414	0\\
-2.56854128908243	0\\
-2.58022372478849	0\\
-2.59192189441946	0\\
-2.60363584041344	0\\
-2.61536560538048	0\\
-2.62711123210348	0\\
-2.63887276353917	0\\
-2.65065024281897	0\\
-2.66244371325002	0\\
-2.6742532183161	0\\
-2.68607880167859	0\\
-2.69792050717744	0\\
-2.70977837883216	0\\
-2.72165246084279	0\\
-2.73354279759088	0\\
-2.74544943364051	0\\
-2.75737241373926	0\\
-2.76931178281924	0\\
-2.78126758599813	0\\
-2.79323986858012	0\\
-2.80522867605706	0\\
-2.81723405410938	0\\
-2.82925604860722	0\\
-2.84129470561142	0\\
-2.85335007137463	0\\
-2.86542219234235	0\\
-2.87751111515399	0\\
-2.889616886644	0\\
-2.90173955384289	0\\
-2.91387916397839	0\\
-2.92603576447651	0\\
-2.93820940296269	0\\
-2.95040012726287	0\\
-2.96260798540467	0\\
-2.97483302561849	0\\
-2.98707529633867	0\\
-2.99933484620462	0\\
-3.01161172406201	0\\
-3.02390597896393	0\\
-3.03621766017203	0\\
-3.04854681715776	0\\
-3.06089349960352	0\\
-3.0732577574039	0\\
-3.08563964066683	0\\
-3.09803919971486	0\\
-3.11045648508637	0\\
-3.12289154753678	0\\
-3.13534443803981	0\\
-3.14781520778876	0\\
-3.16030390819772	0\\
-3.1728105909029	0\\
-3.18533530776386	0\\
-3.19787811086484	0\\
-3.21043905251603	0\\
-3.22301818525489	0\\
-3.23561556184748	0\\
-3.24823123528977	0\\
-3.26086525880899	0\\
-3.27351768586497	0\\
-3.28618857015149	0\\
-3.29887796559768	0\\
-3.31158592636935	0\\
-3.32431250687042	0\\
-3.3370577617443	0\\
-3.34982174587527	0\\
-3.36260451438997	0\\
-3.37540612265873	0\\
-3.38822662629711	0\\
-3.40106608116728	0\\
-3.4139245433795	0\\
-3.4268020692936	0\\
-3.43969871552046	0\\
-3.4526145389235	0\\
-3.46554959662016	0\\
-3.47850394598347	0\\
-3.49147764464354	0\\
-3.50447075048909	0\\
-3.51748332166902	0\\
-3.53051541659398	-1.15811939055388e-05\\
-3.54356709393791	-7.23827152451176e-05\\
-3.55663841263965	-0.000185300955495804\\
-3.56972943190454	-0.000350337947264509\\
-3.582840211206	-0.000567496661445672\\
-3.5959708102872	-0.000836781007399546\\
-3.60912128916262	-0.00115819583318005\\
-3.6222917081198	-0.00153174692583443\\
-3.63548212772089	-0.00195744101174773\\
-3.64869260880438	-0.0024352857570418\\
-3.66192321248679	-0.00296528976804235\\
-3.67517400016434	-0.00354746259178908\\
-3.68844503351465	-0.00418181471661977\\
-3.70173637449852	-0.00486835757278982\\
-3.71504808536159	-0.00560710353316019\\
-3.72838022863616	-0.00639806591394796\\
-3.74173286714289	-0.0072412589755232\\
-3.75510606399261	-0.00813669792326486\\
-3.76849988258813	-0.00908439890847959\\
-3.78191438662599	-0.0100843790293701\\
-3.79534964009832	-0.0111366563320688\\
-3.80880570729464	-0.0122412498117208\\
-3.82228265280376	-0.0133981794136321\\
-3.83578054151556	-0.0146074660344718\\
-3.84929943862293	-0.0158691315235299\\
-3.86283940962365	-0.0171831986840413\\
-3.87640052032225	-0.0185496912745537\\
-3.889982836832	-0.0199686340103775\\
-3.90358642557675	-0.021440052565063\\
-3.91721135329299	-0.0229639735719676\\
-3.93085768703172	-0.0245404246258592\\
-3.94452549416049	-0.0261694342845914\\
-3.95821484236534	-0.027851032070834\\
-3.9719257996529	-0.0295852484738618\\
-3.9856584343523	-0.0313721149514037\\
-3.99941281511731	-0.0332116639315603\\
-4.01318901092836	-0.0351039288147634\\
-4.02698709109462	-0.0370489439758198\\
-4.04080712525608	-0.039046744765993\\
-4.05464918338567	-0.0410973675151629\\
-4.0685133357914	-0.0432008495340396\\
-4.0823996531185	-0.0453572291164417\\
-4.09630820635152	-0.0475665455416286\\
-4.11023906681661	-0.0498288390767117\\
-4.12419230618362	-0.0521441509791062\\
-4.1381679964684	-0.0545125234990719\\
-4.15216621003492	-0.0569339998822907\\
-4.16618701959764	-0.0594086243725251\\
-4.18023049822368	-0.0619364422143368\\
-4.19429671933517	-0.0645174996558674\\
-4.20838575671149	-0.0671518439516861\\
-4.22249768449166	-0.0698395233656985\\
-4.23663257717665	-0.0725805871741272\\
-4.2507905096317	-0.0753750856685514\\
-4.26497155708877	-0.078223070159019\\
-4.27917579514892	-0.0811245929772203\\
-4.29340329978466	-0.0840797074797294\\
-4.30765414734249	-0.0870884680513211\\
-4.3219284145453	-0.0901509301083387\\
-4.33622617849485	-0.0932671501021509\\
-4.35054751667427	-0.0964371855226607\\
-4.36489250695062	-0.0996610949018904\\
-4.37926122757736	-0.102938937817639\\
-4.39365375719697	-0.106270774897205\\
-4.40807017484351	-0.109656667821179\\
-4.42251055994521	-0.113096679327319\\
-4.43697499232713	-0.116590873214477\\
-4.45146355221377	-0.120139314346617\\
-4.46597632023178	-0.1237420686569\\
-4.48051337741262	-0.127399203151831\\
-4.49507480519527	-0.131110785915505\\
-4.50966068542901	-0.134876886113895\\
-4.52427110037613	-0.138697573999246\\
-4.53890613271475	-0.142572920914527\\
-4.5535658655416	-0.146502999297957\\
-4.56825038237489	-0.150487882687624\\
-4.58295976715711	-0.154527645726165\\
-4.59769410425797	-0.158622364165535\\
-4.61245347847723	-0.162772114871844\\
-4.62723797504771	-0.166976975830293\\
-4.64204767963819	-0.171237026150161\\
-4.65688267835639	-0.175552346069899\\
-4.67174305775201	-0.179923016962299\\
-4.68662890481972	-0.184349121339734\\
-4.70154030700223	-0.188830742859491\\
-4.71647735219339	-0.19336796632919\\
-4.73144012874125	-0.197960877712275\\
-4.74642872545128	-0.202609564133612\\
-4.76144323158942	-0.207314113885141\\
-4.77648373688537	-0.212074616431654\\
-4.79155033153576	-0.216891162416622\\
-4.80664310620739	-0.22176384366814\\
-4.82176215204053	-0.226692753204942\\
-4.8369075606522	-0.231677985242528\\
-4.85207942413951	-0.236719635199344\\
-4.86727783508304	-0.241817799703103\\
-4.88250288655017	-0.246972576597158\\
-4.89775467209858	-0.252184064946992\\
-4.91303328577962	-0.257452365046787\\
-4.92833882214187	-0.262777578426113\\
-4.94367137623457	-0.268159807856674\\
-4.95903104361123	-0.273599157359197\\
-4.97441792033315	-0.279095732210384\\
-4.98983210297308	-0.284649638949984\\
-5.00527368861878	-0.290260985387951\\
-5.02074277487677	-0.295929880611726\\
-5.03623945987599	-0.301656434993584\\
-5.05176384227154	-0.307440760198134\\
-5.06731602124842	-0.313282969189878\\
-5.08289609652542	-0.319183176240905\\
-5.09850416835885	-0.32514149693868\\
-5.11414033754648	-0.331158048193946\\
-5.1298047054314	-0.337232948248733\\
-5.14549737390603	-0.343366316684482\\
-5.16121844541602	-0.349558274430267\\
-5.1769680229643	-0.355808943771157\\
-5.19274621011512	-0.362118448356661\\
-5.20855311099816	-0.368486913209311\\
-5.22438883031262	-0.374914464733357\\
-5.24025347333139	-0.381401230723575\\
-5.25614714590525	-0.387947340374194\\
-5.27206995446715	-0.394552924287962\\
-5.2880220060364	-0.4012181144853\\
-5.30400340822306	-0.407943044413621\\
-5.32001426923226	-0.414727848956742\\
-5.33605469786861	-0.421572664444436\\
-5.35212480354063	-0.428477628662116\\
-5.36822469626523	-0.435442880860638\\
-5.38435448667222	-0.442468561766245\\
-5.40051428600889	-0.44955481359064\\
-5.4167042061446	-0.456701780041191\\
-5.43292435957543	-0.463909606331278\\
-5.44917485942887	-0.471178439190777\\
-5.46545581946855	-0.47850842687667\\
-5.48176735409904	-0.485899719183825\\
-5.49810957837062	-0.493352467455884\\
-5.51448260798422	-0.500866824596326\\
-5.53088655929628	-0.508442945079656\\
-5.54732154932375	-0.516080984962757\\
-5.56378769574907	-0.523781101896372\\
-5.58028511692522	-0.531543455136773\\
-5.59681393188086	-0.539368205557533\\
-5.61337426032547	-0.547255515661517\\
-5.62996622265451	-0.555205549592965\\
-5.64658993995476	-0.563218473149789\\
-5.66324553400951	-0.571294453796002\\
-5.67993312730402	-0.579433660674319\\
-5.69665284303084	-0.587636264618922\\
-5.71340480509534	-0.595902438168403\\
-5.73018913812115	-0.604232355578862\\
-5.74700596745577	-0.612626192837197\\
-5.76385541917618	-0.621084127674541\\
-5.78073762009448	-0.629606339579912\\
-5.79765269776367	-0.638193009814013\\
-5.81460078048338	-0.646844321423227\\
-5.83158199730574	-0.655560459253796\\
-5.84859647804127	-0.664341609966194\\
-5.86564435326482	-0.67318796204966\\
-5.88272575432161	-0.682099705836971\\
-5.89984081333328	-0.691077033519363\\
-5.91698966320402	-0.700120139161685\\
-5.93417243762677	-0.709229218717721\\
-5.95138927108949	-0.71840447004575\\
-5.96864029888145	-0.727646092924276\\
-5.98592565709961	-0.736954289068002\\
-6.00324548265508	-0.746329262143978\\
-6.02059991327962	-0.755771217787996\\
-6.0379890875322	-0.76528036362118\\
-6.05541314480565	-0.774856909266815\\
-6.07287222533336	-0.784501066367376\\
-6.09036647019605	-0.79421304860181\\
-6.10789602132862	-0.80399307170302\\
-6.12546102152706	-0.813841353475616\\
-6.1430616144554	-0.823758113813863\\
-6.16069794465279	-0.833743574719901\\
-6.17837015754063	-0.84379796032219\\
-6.19607839942973	-0.853921496894215\\
-6.21382281752759	-0.864114412873422\\
-6.23160355994579	-0.874376938880434\\
-6.24942077570731	-0.884709307738493\\
-6.26727461475413	-0.895111754493201\\
-6.28516522795473	-0.905584516432485\\
-6.30309276711175	-0.916127833106869\\
-6.32105738496976	-0.926741946349988\\
-6.33905923522301	-0.937427100299405\\
-6.35709847252336	-0.948183541417684\\
-6.37517525248826	-0.959011518513778\\
-6.39328973170873	-0.969911282764669\\
-6.41144206775762	-0.980883087737345\\
-6.42963241919772	-0.991927189411032\\
-6.44786094559014	-1.00304384619977\\
-6.46612780750267	-1.01423331897526\\
-6.4844331665183	-1.02549587109006\\
-6.50277718524377	-1.03683176840104\\
-6.52116002731824	-1.04824127929325\\
-6.53958185742208	-1.05972467470401\\
-6.55804284128565	-1.07128222814741\\
-6.57654314569833	-1.08291421573913\\
-6.59508293851752	-1.09462091622155\\
-6.61366238867776	-1.1064026109893\\
-6.6322816662	-1.11825958411507\\
-6.65094094220092	-1.13019212237584\\
-6.66964038890238	-1.14220051527946\\
-6.68838017964094	-1.15428505509157\\
-6.70716048887749	-1.16644603686294\\
-6.72598149220704	-1.17868375845715\\
-6.74484336636852	-1.19099852057872\\
-6.76374628925477	-1.20339062680152\\
-6.78269043992262	-1.21586038359774\\
-6.80167599860299	-1.22840810036709\\
-6.8207031467113	-1.24103408946655\\
-6.83977206685775	-1.2537386662405\\
-6.85888294285792	-1.26652214905121\\
-6.87803595974336	-1.27938485930989\\
-6.89723130377236	-1.29232712150805\\
-6.91646916244079	-1.3053492632494\\
-6.93574972449313	-1.31845161528217\\
-6.95507317993354	-1.33163451153185\\
-6.97443972003712	-1.34489828913454\\
-6.99384953736127	-1.35824328847056\\
-7.01330282575716	-1.37166985319877\\
-7.03279978038137	-1.3851783302912\\
-7.05234059770761	-1.39876907006834\\
-7.07192547553861	-1.4124424262348\\
-7.09155461301816	-1.42619875591556\\
-7.11122821064323	-1.44003841969278\\
-7.13094647027625	-1.45396178164305\\
-7.15070959515757	-1.4679692093753\\
-7.17051778991801	-1.48206107406914\\
-7.19037126059156	-1.49623775051388\\
-7.21027021462828	-1.51049961714806\\
-7.23021486090725	-1.52484705609956\\
-7.25020540974979	-1.53928045322636\\
-7.27024207293269	-1.55380019815781\\
-7.29032506370176	-1.56840668433665\\
-7.31045459678537	-1.58310030906148\\
-7.33063088840827	-1.59788147353003\\
-7.35085415630551	-1.61275058288299\\
-7.37112461973656	-1.62770804624853\\
-7.39144249949952	-1.64275427678743\\
-7.41180801794562	-1.65788969173902\\
-7.43222139899377	-1.67311471246765\\
-7.45268286814535	-1.68842976451002\\
-7.47319265249915	-1.70383527762311\\
-7.49375098076652	-1.71933168583293\\
-7.51435808328663	-1.73491942748394\\
-7.53501419204199	-1.75059894528932\\
-7.55571954067409	-1.76637068638188\\
-7.5764743644993	-1.78223510236592\\
-7.59727890052485	-1.79819264936973\\
-7.61813338746515	-1.81424378809904\\
-7.63903806575815	-1.83038898389119\\
-7.65999317758202	-1.84662870677024\\
-7.68099896687198	-1.86296343150285\\
-7.70205567933731	-1.87939363765512\\
-7.72316356247862	-1.89591980965029\\
-7.74432286560529	-1.91254243682729\\
-7.76553383985316	-1.92926201350033\\
-7.7867967382024	-1.94607903901936\\
-7.8081118154956	-1.96299401783149\\
-7.82947932845611	-1.98000745954343\\
-7.85089953570663	-1.99711987898489\\
-7.8723726977879	-2.014331796273\\
-7.89389907717781	-2.03164373687782\\
-7.9154789383106	-2.04905623168881\\
-7.93711254759635	-2.0665698170825\\
-7.95880017344075	-2.08418503499111\\
-7.98054208626503	-2.10190243297244\\
-8.00233855852624	-2.11972256428075\\
-8.0241898647377	-2.13764598793896\\
-8.04609628148975	-2.15567326881189\\
-8.0680580874708	-2.1738049776808\\
-8.09007556348852	-2.19204169131914\\
-8.11214899249147	-2.21038399256952\\
-8.13427865959085	-2.22883247042201\\
-8.15646485208266	-2.24738772009372\\
-8.17870785947001	-2.26605034310969\\
-8.20100797348585	-2.28482094738524\\
-8.22336548811585	-2.30370014730952\\
-8.24578069962177	-2.32268856383069\\
-8.2682539065649	-2.34178682454234\\
-8.29078540982999	-2.36099556377157\\
-8.31337551264938	-2.38031542266837\\
-8.33602452062754	-2.39974704929676\\
-8.35873274176582	-2.41929109872724\\
-8.38150048648761	-2.43894823313108\\
-8.40432806766379	-2.45871912187603\\
-8.42721580063855	-2.47860444162386\\
-8.45016400325549	-2.49860487642944\\
-8.47317299588414	-2.51872111784169\\
-8.49624310144678	-2.53895386500622\\
-8.51937464544562	-2.55930382476976\\
-8.54256795599039	-2.57977171178648\\
-8.56582336382625	-2.60035824862619\\
-8.58914120236205	-2.62106416588439\\
-8.61252180769908	-2.64189020229438\\
-8.6359655186601	-2.6628371048413\\
-8.65947267681879	-2.68390562887821\\
-8.68304362652965	-2.70509653824435\\
-8.70667871495821	-2.72641060538542\\
-8.73037829211178	-2.7478486114761\\
-8.75414271087051	-2.76941134654483\\
-8.77797232701888	-2.79109960960074\\
-8.80186749927775	-2.8129142087631\\
-8.82582858933668	-2.83485596139293\\
-8.84985596188684	-2.85692569422724\\
-8.87394998465425	-2.87912424351559\\
-8.89811102843362	-2.9014524551593\\
-8.92233946712251	-2.92391118485318\\
-8.94663567775614	-2.94650129822999\\
-8.97100004054249	-2.96922367100751\\
-8.99543293889812	-2.9920791891385\\
-9.01993475948424	-3.01506874896338\\
-9.04450589224355	-3.0381932573659\\
-9.06914673043738	-3.06145363193172\\
-9.09385767068352	-3.08485080111008\\
-9.11863911299448	-3.10838570437849\\
-9.1434914608164	-3.13205929241068\\
-9.16841512106838	-3.15587252724775\\
-9.19341050418253	-3.17982638247266\\
-9.21847802414447	-3.2039218433881\\
-9.24361809853452	-3.22815990719791\\
-9.2688311485694	-3.25254158319192\\
-9.29411759914459	-3.27706789293464\\
-9.3194778788773	-3.30173987045747\\
-9.34491242015005	-3.3265585624549\\
-9.37042165915489	-3.35152502848453\\
-9.39600603593836	-3.37664034117112\\
-9.4216659944469	-3.40190558641476\\
-9.44740198257323	-3.42732186360324\\
-9.47321445220312	-3.45289028582864\\
-9.4991038592631	-3.47861198010854\\
-9.52507066376871	-3.50448808761145\\
-9.5511153298736	-3.53051541659398\\
-9.57723832591927	-3.55663841263965\\
-9.60344012448563	-3.582840211206\\
-9.62972120244225	-3.60912128916262\\
-9.65608204100052	-3.63548212772089\\
-9.68252312576642	-3.66192321248679\\
-9.70904494679428	-3.68844503351466\\
-9.73564799864122	-3.71504808536159\\
-9.76233278042251	-3.74173286714289\\
-9.78909979586776	-3.76849988258813\\
-9.81594955337794	-3.79534964009832\\
-9.84288256608338	-3.82228265280376\\
-9.86989935190256	-3.84929943862294\\
-9.89700043360188	-3.87640052032225\\
-9.92418633885638	-3.90358642557675\\
-9.95145760031134	-3.93085768703172\\
-9.97881475564497	-3.95821484236535\\
-10.0062583476319	-3.9856584343523\\
-10.033788924208	-4.01318901092837\\
-10.0614070385357	-4.04080712525608\\
-10.089113249071	-4.06851333579141\\
-10.1169081196311	-4.09630820635152\\
-10.1447922194633	-4.12419230618363\\
-10.1727661233145	-4.15216621003492\\
-10.2008304115033	-4.18023049822368\\
-10.2289856699911	-4.20838575671149\\
-10.2572324904563	-4.23663257717665\\
-10.2855714703684	-4.26497155708877\\
-10.3140032130643	-4.29340329978466\\
-10.3425283278249	-4.3219284145453\\
-10.3711474299539	-4.35054751667428\\
-10.399861140857	-4.37926122757736\\
-10.4286700881231	-4.40807017484351\\
-10.4575749056067	-4.43697499232713\\
-10.4865762335114	-4.46597632023179\\
-10.5156747184749	-4.49507480519527\\
-10.5448710136558	-4.52427110037613\\
-10.5741657788212	-4.5535658655416\\
-10.6035596804367	-4.58295976715712\\
-10.6330533917569	-4.61245347847723\\
-10.6626475929178	-4.64204767963819\\
-10.6923429710316	-4.67174305775201\\
-10.7221402202819	-4.70154030700223\\
-10.7520400420209	-4.73144012874125\\
-10.782043144869	-4.76144323158942\\
-10.8121502448154	-4.79155033153576\\
-10.8423620653202	-4.82176215204053\\
-10.8726793374191	-4.85207942413951\\
-10.9031027998298	-4.88250288655017\\
-10.9336331990592	-4.91303328577962\\
-10.9642712895142	-4.94367137623457\\
-10.9950178336128	-4.97441792033315\\
-11.0258736018984	-5.00527368861878\\
-11.0568393731556	-5.03623945987599\\
-11.087915934528	-5.06731602124842\\
-11.1191040816385	-5.09850416835885\\
-11.150404618711	-5.1298047054314\\
-11.1818183586956	-5.16121844541602\\
-11.2133461233947	-5.19274621011512\\
-11.2449887435922	-5.22438883031262\\
-11.2767470591849	-5.25614714590525\\
-11.308621919316	-5.2880220060364\\
-11.3406141825119	-5.32001426923226\\
-11.3727247168203	-5.35212480354063\\
-11.4049543999518	-5.38435448667221\\
-11.4373041194242	-5.4167042061446\\
-11.4697747727085	-5.44917485942887\\
-11.5023672673787	-5.48176735409904\\
-11.5350825212638	-5.51448260798422\\
-11.5679214626034	-5.54732154932375\\
-11.6008850302048	-5.58028511692521\\
-11.6339741736051	-5.61337426032547\\
-11.6671898532344	-5.64658993995475\\
-11.7005330405836	-5.67993312730402\\
-11.734004718375	-5.71340480509534\\
-11.7676058807354	-5.74700596745577\\
-11.8013375333741	-5.78073762009448\\
-11.835200693763	-5.81460078048338\\
-11.8691963913209	-5.84859647804127\\
-11.9033256676012	-5.88272575432161\\
-11.9375895764836	-5.91698966320401\\
-11.9719891843691	-5.95138927108949\\
-12.0065255703792	-5.98592565709961\\
-12.0411998265592	-6.02059991327962\\
-12.0760130580853	-6.05541314480565\\
-12.1109663834757	-6.09036647019605\\
-12.1460609348067	-6.12546102152706\\
-12.1812978579324	-6.16069794465279\\
-12.2166783127093	-6.19607839942972\\
-12.2522034732254	-6.23160355994579\\
-12.2878745280338	-6.26727461475413\\
-12.3236926803914	-6.30309276711175\\
-12.3596591485026	-6.339059235223\\
-12.3957751657679	-6.37517525248826\\
-12.4320419810372	-6.41144206775762\\
-12.4684608588698	-6.44786094559014\\
-12.5050330797979	-6.4844331665183\\
-12.5417599405979	-6.52116002731824\\
-12.5786427545653	-6.55804284128565\\
-12.6156828517971	-6.59508293851752\\
-12.6528815794796	-6.63228166619999\\
-12.690240302182	-6.66964038890238\\
-12.7277604021571	-6.70716048887749\\
-12.7654432796481	-6.74484336636852\\
-12.8032903532022	-6.78269043992261\\
-12.8413030599909	-6.8207031467113\\
-12.8794828561375	-6.85888294285792\\
-12.917831217052	-6.89723130377236\\
-12.9563496377727	-6.93574972449312\\
-12.9950396333167	-6.97443972003712\\
-13.0339027390368	-7.01330282575716\\
-13.0729405109872	-7.05234059770761\\
-13.1121545262978	-7.09155461301816\\
-13.1515463835559	-7.13094647027625\\
-13.1911177031976	-7.17051778991801\\
-13.2308701279079	-7.21027021462828\\
-13.2708053230294	-7.25020540974978\\
-13.3109249769814	-7.29032506370176\\
-13.3512308016879	-7.33063088840827\\
-13.3917245330162	-7.37112461973656\\
-13.4324079312252	-7.41180801794562\\
-13.473282781425	-7.45268286814535\\
-13.5143508940461	-7.49375098076652\\
-13.5556141053216	-7.53501419204199\\
-13.5970742777789	-7.57647436449929\\
-13.6387333007448	-7.61813338746515\\
-13.6805930908616	-7.65999317758202\\
-13.7226555926169	-7.70205567933731\\
-13.7649227788849	-7.74432286560529\\
-13.807396651482	-7.7867967382024\\
-13.8500792417357	-7.82947932845611\\
-13.8929726110675	-7.8723726977879\\
-13.9360788515902	-7.9154789383106\\
-13.9794000867204	-7.95880017344075\\
-14.0229384718059	-8.00233855852624\\
-14.0666961947694	-8.04609628148976\\
-14.1106754767681	-8.09007556348852\\
-14.1548785728705	-8.13427865959086\\
-14.1993077727496	-8.17870785947001\\
-14.2439654013955	-8.22336548811586\\
-14.2888538198445	-8.2682539065649\\
-14.333975425929	-8.31337551264939\\
-14.3793326550454	-8.35873274176582\\
-14.4249279809434	-8.4043280676638\\
-14.4707639165351	-8.45016400325549\\
-14.5168430147264	-8.49624310144678\\
-14.56316786927	-8.54256795599039\\
-14.6097411156417	-8.58914120236205\\
-14.6565654319397	-8.6359655186601\\
-14.7036435398093	-8.68304362652965\\
-14.7509782053914	-8.73037829211178\\
-14.7985722402985	-8.77797232701888\\
-14.8464285026163	-8.82582858933668\\
-14.8945498979339	-8.87394998465426\\
-14.9429393804021	-8.92233946712251\\
-14.9915999538221	-8.9710000405425\\
-15.0405346727639	-9.01993475948424\\
-15.089746643717	-9.06914673043738\\
-15.1392390262741	-9.11863911299448\\
-15.189015034348	-9.16841512106839\\
-15.2390779374241	-9.21847802414447\\
-15.289431061849	-9.2688311485694\\
-15.3400777921569	-9.3194778788773\\
-15.3910215724345	-9.3704216591549\\
-15.4422659077265	-9.4216659944469\\
-15.4938143654827	-9.47321445220312\\
-15.5456705770483	-9.52507066376871\\
-15.5978382391989	-9.57723832591928\\
-15.6503211157219	-9.62972120244225\\
-15.703123039046	-9.68252312576642\\
-15.7562479119208	-9.73564799864122\\
-15.8096997091474	-9.78909979586776\\
-15.863482479363	-9.84288256608338\\
-15.9176003468815	-9.89700043360189\\
-15.972057513591	-9.95145760031134\\
-16.0268582609116	-10.0062583476319\\
-16.0820069518153	-10.0614070385357\\
-16.1375080329108	-10.1169081196311\\
-16.1933660365942	-10.1727661233145\\
-16.2495855832707	-10.2289856699911\\
-16.306171383648	-10.2855714703684\\
-16.3631282411045	-10.3425283278249\\
-16.4204610541366	-10.399861140857\\
-16.4781748188864	-10.4575749056067\\
-16.5362746317545	-10.5156747184749\\
-16.5947656921008	-10.5741657788212\\
-16.6536533050365	-10.6330533917569\\
-16.7129428843112	-10.6923429710316\\
-16.7726399553005	-10.7520400420209\\
-16.832750158095	-10.8121502448154\\
-16.8932792506988	-10.8726793374191\\
-16.9542331123389	-10.9336331990592\\
-17.0156177468924	-10.9950178336128\\
-17.0774392864352	-11.0568393731556\\
-17.1397039949181	-11.1191040816385\\
-17.2024182719753	-11.1818183586956\\
-17.2655886568719	-11.2449887435922\\
-17.3292218325956	-11.308621919316\\
-17.3933246300999	-11.3727247168203\\
-17.4579040327038	-11.4373041194242\\
-17.5229671806583	-11.5023672673787\\
-17.588521375883	-11.5679214626034\\
-17.6545740868847	-11.6339741736051\\
-17.7211329538633	-11.7005330405836\\
-17.788205794015	-11.7676058807354\\
-17.8558006070426	-11.835200693763\\
-17.9239255808809	-11.9033256676012\\
-17.9925890976487	-11.9719891843691\\
-18.0617997398389	-12.0411998265592\\
-18.1315662967553	-12.1109663834757\\
-18.201897771212	-12.1812978579324\\
-18.272803386505	-12.2522034732254\\
-18.344292593671	-12.3236926803914\\
-18.4163750790475	-12.3957751657679\\
-18.4890607721494	-12.4684608588698\\
-18.5623598538775	-12.5417599405979\\
-18.6362827650768	-12.6156828517971\\
-18.7108402154616	-12.690240302182\\
-18.7860431929278	-12.7654432796481\\
-18.8619029732705	-12.8413030599909\\
-18.9384311303316	-12.917831217052\\
-19.0156395465964	-12.9950396333167\\
-19.0935404242669	-13.0729405109872\\
-19.1721462968355	-13.1515463835559\\
-19.2514700411875	-13.2308701279079\\
-19.331524890261	-13.3109249769814\\
-19.4123244462958	-13.3917245330162\\
-19.4938826947046	-13.473282781425\\
-19.5762140186012	-13.5556141053216\\
-19.6593332140244	-13.6387333007448\\
-19.7432555058966	-13.7226555926169\\
-19.8279965647616	-13.807396651482\\
-19.9135725243471	-13.8929726110675\\
-20	-13.9794000867204\\
-20.087296108049	-14.0666961947694\\
-20.1754784861501	-14.1548785728705\\
-20.2645653146751	-14.2439654013955\\
-20.3545753392086	-14.333975425929\\
-20.445527894223	-14.4249279809434\\
-20.537442928006	-14.5168430147264\\
-20.6303410289213	-14.6097411156417\\
-20.7242434530889	-14.7036435398093\\
-20.8191721535781	-14.7985722402985\\
-20.9151498112135	-14.8945498979339\\
-21.0121998671017	-14.9915999538221\\
-21.1103465569966	-15.089746643717\\
-21.2096149476276	-15.189015034348\\
-21.3100309751286	-15.289431061849\\
-21.4116214857141	-15.3910215724345\\
-21.5144142787624	-15.4938143654827\\
-21.6184381524785	-15.5978382391989\\
-21.7237229523257	-15.703123039046\\
-21.830299622427	-15.8096997091474\\
-21.9382002601611	-15.9176003468815\\
-22.0474581741912	-16.0268582609116\\
-22.1581079461904	-16.1375080329108\\
-22.2701854965504	-16.2495855832707\\
-22.3837281543842	-16.3631282411045\\
-22.498774732166	-16.4781748188864\\
-22.6153656053805	-16.5947656921008\\
-22.7335427975909	-16.7129428843113\\
-22.8533500713746	-16.832750158095\\
-22.9748330256185	-16.9542331123389\\
-23.0980391997149	-17.0774392864352\\
-23.2230181852549	-17.2024182719753\\
-23.3498217458753	-17.3292218325956\\
-23.4785039459835	-17.4579040327039\\
-23.6091212891626	-17.588521375883\\
-23.7417328671429	-17.7211329538633\\
-23.8764005203222	-17.8558006070426\\
-24.0131890109284	-17.9925890976488\\
-24.1521662100349	-18.1315662967553\\
-24.2934032997847	-18.272803386505\\
-24.4369749923271	-18.4163750790475\\
-24.5829597671571	-18.5623598538775\\
-24.7314401287412	-18.7108402154616\\
-24.8825028865502	-18.8619029732706\\
-25.036239459876	-19.0156395465964\\
-25.1927462101151	-19.1721462968355\\
-25.3521248035406	-19.331524890261\\
-25.5144826079842	-19.4938826947046\\
-25.679933127304	-19.6593332140244\\
-25.8485964780413	-19.8279965647617\\
-26.0205999132796	-20\\
-26.1960783994297	-20.1754784861501\\
-26.3751752524882	-20.3545753392086\\
-26.5580428412857	-20.537442928006\\
-26.7448433663685	-20.7242434530889\\
-26.9357497244931	-20.9151498112135\\
-27.1309464702762	-21.1103465569966\\
-27.3306308884083	-21.3100309751287\\
-27.535014192042	-21.5144142787624\\
-27.7443228656053	-21.7237229523257\\
-27.9588001734407	-21.9382002601611\\
-28.17870785947	-22.1581079461904\\
-28.4043280676638	-22.3837281543842\\
-28.6359655186601	-22.6153656053805\\
-28.8739499846542	-22.8533500713746\\
-29.1186391129945	-23.0980391997149\\
-29.3704216591549	-23.3498217458753\\
-29.6297212024423	-23.6091212891627\\
-29.8970004336019	-23.8764005203222\\
-30.1727661233146	-24.1521662100349\\
-30.4575749056067	-24.4369749923271\\
-30.7520400420209	-24.7314401287413\\
-31.0568393731556	-25.036239459876\\
-31.3727247168203	-25.3521248035407\\
-31.7005330405836	-25.679933127304\\
-32.0411998265593	-26.0205999132797\\
-32.3957751657679	-26.3751752524882\\
-32.7654432796481	-26.7448433663685\\
-33.1515463835559	-27.1309464702762\\
-33.5556141053216	-27.5350141920419\\
-33.9794000867204	-27.9588001734407\\
-34.4249279809434	-28.4043280676637\\
-34.8945498979339	-28.8739499846542\\
-35.3910215724345	-29.3704216591548\\
-35.9176003468815	-29.8970004336019\\
-36.4781748188863	-30.4575749056067\\
-37.0774392864352	-31.0568393731556\\
-37.7211329538632	-31.7005330405836\\
-38.4163750790475	-32.3957751657679\\
-39.1721462968354	-33.1515463835558\\
-40	-33.9794000867204\\
-40.9151498112134	-34.8945498979338\\
-41.9382002601611	-35.9176003468815\\
-43.0980391997147	-37.0774392864351\\
-44.4369749923271	-38.4163750790475\\
-46.0205999132794	-39.9999999999998\\
-47.9588001734407	-41.9382002601611\\
-50.4575749056064	-44.4369749923268\\
-53.9794000867204	-47.9588001734407\\
-59.999999999999	-53.9794000867194\\
-inf	-inf\\
-60	-53.9794000867204\\
-53.9794000867204	-47.9588001734407\\
-50.4575749056067	-44.4369749923271\\
-47.9588001734407	-41.9382002601611\\
-46.0205999132796	-40\\
-44.4369749923271	-38.4163750790475\\
-43.0980391997149	-37.0774392864352\\
-41.9382002601611	-35.9176003468815\\
-40.9151498112135	-34.8945498979339\\
-40	-33.9794000867204\\
-39.1721462968355	-33.1515463835559\\
-38.4163750790475	-32.3957751657679\\
-37.7211329538633	-31.7005330405836\\
-37.0774392864352	-31.0568393731556\\
-36.4781748188864	-30.4575749056067\\
-35.9176003468815	-29.8970004336019\\
-35.3910215724345	-29.3704216591549\\
-34.8945498979339	-28.8739499846542\\
-34.4249279809434	-28.4043280676638\\
-33.9794000867204	-27.9588001734407\\
-33.5556141053216	-27.535014192042\\
-33.1515463835559	-27.1309464702762\\
-32.7654432796481	-26.7448433663685\\
-32.3957751657679	-26.3751752524882\\
-32.0411998265592	-26.0205999132796\\
-31.7005330405836	-25.679933127304\\
-31.3727247168202	-25.3521248035406\\
-31.0568393731556	-25.036239459876\\
-30.7520400420209	-24.7314401287412\\
-30.4575749056067	-24.4369749923271\\
-30.1727661233145	-24.1521662100349\\
-29.8970004336019	-23.8764005203222\\
-29.6297212024422	-23.6091212891626\\
-29.3704216591549	-23.3498217458753\\
-29.1186391129945	-23.0980391997149\\
-28.8739499846542	-22.8533500713746\\
-28.6359655186601	-22.6153656053805\\
-28.4043280676638	-22.3837281543842\\
-28.17870785947	-22.1581079461904\\
-27.9588001734407	-21.9382002601611\\
-27.7443228656053	-21.7237229523257\\
-27.535014192042	-21.5144142787624\\
-27.3306308884083	-21.3100309751287\\
-27.1309464702763	-21.1103465569966\\
-26.9357497244931	-20.9151498112135\\
-26.7448433663685	-20.7242434530889\\
-26.5580428412857	-20.537442928006\\
-26.3751752524883	-20.3545753392086\\
-26.1960783994297	-20.1754784861501\\
-26.0205999132796	-20\\
-25.8485964780413	-19.8279965647617\\
-25.679933127304	-19.6593332140244\\
-25.5144826079842	-19.4938826947046\\
-25.3521248035406	-19.331524890261\\
-25.1927462101151	-19.1721462968355\\
-25.036239459876	-19.0156395465964\\
-24.8825028865502	-18.8619029732706\\
-24.7314401287413	-18.7108402154616\\
-24.5829597671571	-18.5623598538775\\
-24.4369749923271	-18.4163750790475\\
-24.2934032997847	-18.272803386505\\
-24.1521662100349	-18.1315662967553\\
-24.0131890109284	-17.9925890976488\\
-23.8764005203223	-17.8558006070426\\
-23.7417328671429	-17.7211329538633\\
-23.6091212891626	-17.588521375883\\
-23.4785039459835	-17.4579040327039\\
-23.3498217458753	-17.3292218325957\\
-23.2230181852549	-17.2024182719753\\
-23.0980391997149	-17.0774392864352\\
-22.9748330256185	-16.9542331123389\\
-22.8533500713746	-16.832750158095\\
-22.7335427975909	-16.7129428843113\\
-22.6153656053805	-16.5947656921009\\
-22.498774732166	-16.4781748188864\\
-22.3837281543842	-16.3631282411046\\
-22.2701854965504	-16.2495855832707\\
-22.1581079461904	-16.1375080329108\\
-22.0474581741912	-16.0268582609116\\
-21.9382002601611	-15.9176003468815\\
-21.830299622427	-15.8096997091474\\
-21.7237229523257	-15.703123039046\\
-21.6184381524785	-15.5978382391989\\
-21.5144142787624	-15.4938143654827\\
-21.4116214857141	-15.3910215724345\\
-21.3100309751287	-15.289431061849\\
-21.2096149476276	-15.189015034348\\
-21.1103465569966	-15.089746643717\\
-21.0121998671017	-14.9915999538221\\
-20.9151498112135	-14.8945498979339\\
-20.8191721535781	-14.7985722402985\\
-20.7242434530889	-14.7036435398093\\
-20.6303410289213	-14.6097411156417\\
-20.537442928006	-14.5168430147264\\
-20.445527894223	-14.4249279809434\\
-20.3545753392086	-14.333975425929\\
-20.2645653146751	-14.2439654013955\\
-20.1754784861501	-14.1548785728705\\
-20.087296108049	-14.0666961947694\\
-20	-13.9794000867204\\
-19.9135725243471	-13.8929726110675\\
-19.8279965647616	-13.807396651482\\
-19.7432555058966	-13.7226555926169\\
-19.6593332140244	-13.6387333007448\\
-19.5762140186012	-13.5556141053216\\
-19.4938826947046	-13.473282781425\\
-19.4123244462958	-13.3917245330162\\
-19.331524890261	-13.3109249769814\\
-19.2514700411875	-13.2308701279079\\
-19.1721462968355	-13.1515463835559\\
-19.0935404242669	-13.0729405109872\\
-19.0156395465964	-12.9950396333167\\
-18.9384311303316	-12.917831217052\\
-18.8619029732705	-12.8413030599909\\
-18.7860431929278	-12.7654432796481\\
-18.7108402154616	-12.690240302182\\
-18.6362827650768	-12.6156828517971\\
-18.5623598538775	-12.5417599405979\\
-18.4890607721494	-12.4684608588698\\
-18.4163750790475	-12.3957751657679\\
-18.344292593671	-12.3236926803914\\
-18.272803386505	-12.2522034732254\\
-18.201897771212	-12.1812978579324\\
-18.1315662967553	-12.1109663834757\\
-18.0617997398389	-12.0411998265592\\
-17.9925890976487	-11.9719891843691\\
-17.9239255808809	-11.9033256676012\\
-17.8558006070426	-11.835200693763\\
-17.788205794015	-11.7676058807354\\
-17.7211329538633	-11.7005330405836\\
-17.6545740868847	-11.6339741736051\\
-17.588521375883	-11.5679214626034\\
-17.5229671806583	-11.5023672673787\\
-17.4579040327038	-11.4373041194242\\
-17.3933246300999	-11.3727247168203\\
-17.3292218325956	-11.308621919316\\
-17.2655886568719	-11.2449887435922\\
-17.2024182719753	-11.1818183586956\\
-17.1397039949181	-11.1191040816385\\
-17.0774392864352	-11.0568393731556\\
-17.0156177468924	-10.9950178336128\\
-16.9542331123389	-10.9336331990592\\
-16.8932792506988	-10.8726793374191\\
-16.832750158095	-10.8121502448154\\
-16.7726399553005	-10.7520400420209\\
-16.7129428843113	-10.6923429710316\\
-16.6536533050365	-10.6330533917569\\
-16.5947656921009	-10.5741657788212\\
-16.5362746317545	-10.5156747184749\\
-16.4781748188864	-10.4575749056067\\
-16.4204610541366	-10.399861140857\\
-16.3631282411045	-10.3425283278249\\
-16.306171383648	-10.2855714703684\\
-16.2495855832707	-10.2289856699911\\
-16.1933660365942	-10.1727661233145\\
-16.1375080329108	-10.1169081196311\\
-16.0820069518153	-10.0614070385357\\
-16.0268582609115	-10.0062583476319\\
-15.972057513591	-9.95145760031134\\
-15.9176003468815	-9.89700043360188\\
-15.863482479363	-9.84288256608338\\
-15.8096997091474	-9.78909979586776\\
-15.7562479119208	-9.73564799864122\\
-15.703123039046	-9.68252312576642\\
-15.6503211157219	-9.62972120244225\\
-15.5978382391989	-9.57723832591927\\
-15.5456705770483	-9.52507066376871\\
-15.4938143654827	-9.47321445220312\\
-15.4422659077265	-9.42166599444691\\
-15.3910215724345	-9.3704216591549\\
-15.3400777921569	-9.3194778788773\\
-15.289431061849	-9.2688311485694\\
-15.2390779374241	-9.21847802414447\\
-15.189015034348	-9.16841512106839\\
-15.1392390262741	-9.11863911299449\\
-15.089746643717	-9.06914673043738\\
-15.0405346727639	-9.01993475948425\\
-14.9915999538221	-8.9710000405425\\
-14.9429393804021	-8.92233946712252\\
-14.8945498979339	-8.87394998465426\\
-14.8464285026163	-8.82582858933669\\
-14.7985722402985	-8.77797232701888\\
-14.7509782053914	-8.73037829211179\\
-14.7036435398093	-8.68304362652965\\
-14.6565654319397	-8.6359655186601\\
-14.6097411156417	-8.58914120236205\\
-14.56316786927	-8.5425679559904\\
-14.5168430147264	-8.49624310144678\\
-14.4707639165351	-8.4501640032555\\
-14.4249279809434	-8.4043280676638\\
-14.3793326550455	-8.35873274176583\\
-14.333975425929	-8.31337551264939\\
-14.2888538198445	-8.2682539065649\\
-14.2439654013955	-8.22336548811586\\
-14.1993077727496	-8.17870785947002\\
-14.1548785728705	-8.13427865959086\\
-14.1106754767681	-8.09007556348852\\
-14.0666961947694	-8.04609628148976\\
-14.0229384718059	-8.00233855852625\\
-13.9794000867204	-7.95880017344075\\
-13.9360788515902	-7.9154789383106\\
-13.8929726110675	-7.8723726977879\\
-13.8500792417357	-7.82947932845612\\
-13.807396651482	-7.7867967382024\\
-13.7649227788849	-7.74432286560529\\
-13.7226555926169	-7.70205567933731\\
-13.6805930908616	-7.65999317758202\\
-13.6387333007448	-7.61813338746515\\
-13.5970742777789	-7.5764743644993\\
-13.5556141053216	-7.53501419204199\\
-13.5143508940461	-7.49375098076652\\
-13.473282781425	-7.45268286814535\\
-13.4324079312252	-7.41180801794562\\
-13.3917245330162	-7.37112461973656\\
-13.3512308016879	-7.33063088840827\\
-13.3109249769814	-7.29032506370176\\
-13.2708053230294	-7.25020540974979\\
-13.2308701279079	-7.21027021462828\\
-13.1911177031976	-7.17051778991801\\
-13.1515463835559	-7.13094647027625\\
-13.1121545262978	-7.09155461301816\\
-13.0729405109872	-7.05234059770761\\
-13.0339027390368	-7.01330282575716\\
-12.9950396333167	-6.97443972003712\\
-12.9563496377727	-6.93574972449313\\
-12.917831217052	-6.89723130377236\\
-12.8794828561375	-6.85888294285792\\
-12.8413030599909	-6.8207031467113\\
-12.8032903532022	-6.78269043992262\\
-12.7654432796481	-6.74484336636852\\
-12.7277604021571	-6.70716048887749\\
-12.690240302182	-6.66964038890238\\
-12.6528815794796	-6.6322816662\\
-12.6156828517971	-6.59508293851752\\
-12.5786427545653	-6.55804284128565\\
-12.5417599405979	-6.52116002731824\\
-12.5050330797979	-6.4844331665183\\
-12.4684608588698	-6.44786094559014\\
-12.4320419810372	-6.41144206775762\\
-12.3957751657679	-6.37517525248826\\
-12.3596591485026	-6.33905923522301\\
-12.3236926803914	-6.30309276711175\\
-12.2878745280338	-6.26727461475413\\
-12.2522034732254	-6.23160355994579\\
-12.2166783127093	-6.19607839942973\\
-12.1812978579324	-6.16069794465279\\
-12.1460609348067	-6.12546102152706\\
-12.1109663834757	-6.09036647019605\\
-12.0760130580853	-6.05541314480565\\
-12.0411998265592	-6.02059991327962\\
-12.0065255703792	-5.98592565709961\\
-11.9719891843691	-5.95138927108949\\
-11.9375895764836	-5.91698966320402\\
-11.9033256676012	-5.88272575432161\\
-11.8691963913209	-5.84859647804127\\
-11.835200693763	-5.81460078048338\\
-11.8013375333741	-5.78073762009448\\
-11.7676058807354	-5.74700596745577\\
-11.734004718375	-5.71340480509534\\
-11.7005330405836	-5.67993312730402\\
-11.6671898532344	-5.64658993995476\\
-11.6339741736051	-5.61337426032547\\
-11.6008850302048	-5.58028511692522\\
-11.5679214626034	-5.54732154932375\\
-11.5350825212638	-5.51448260798422\\
-11.5023672673787	-5.48176735409904\\
-11.4697747727085	-5.44917485942887\\
-11.4373041194242	-5.4167042061446\\
-11.4049543999518	-5.38435448667222\\
-11.3727247168203	-5.35212480354063\\
-11.3406141825119	-5.32001426923226\\
-11.308621919316	-5.2880220060364\\
-11.2767470591849	-5.25614714590525\\
-11.2449887435922	-5.22438883031262\\
-11.2133461233947	-5.19274621011512\\
-11.1818183586956	-5.16121844541602\\
-11.150404618711	-5.1298047054314\\
-11.1191040816385	-5.09850416835885\\
-11.087915934528	-5.06731602124842\\
-11.0568393731556	-5.03623945987599\\
-11.0258736018984	-5.00527368861878\\
-10.9950178336128	-4.97441792033315\\
-10.9642712895142	-4.94367137623457\\
-10.9336331990592	-4.91303328577962\\
-10.9031027998298	-4.88250288655017\\
-10.8726793374191	-4.85207942413951\\
-10.8423620653202	-4.82176215204053\\
-10.8121502448154	-4.79155033153576\\
-10.782043144869	-4.76144323158942\\
-10.7520400420209	-4.73144012874125\\
-10.7221402202819	-4.70154030700223\\
-10.6923429710316	-4.67174305775201\\
-10.6626475929178	-4.64204767963819\\
-10.6330533917569	-4.61245347847723\\
-10.6035596804367	-4.58295976715711\\
-10.5741657788212	-4.5535658655416\\
-10.5448710136558	-4.52427110037613\\
-10.5156747184749	-4.49507480519527\\
-10.4865762335114	-4.46597632023179\\
-10.4575749056068	-4.43697499232713\\
-10.4286700881231	-4.40807017484351\\
-10.399861140857	-4.37926122757737\\
-10.3711474299539	-4.35054751667428\\
-10.3425283278249	-4.3219284145453\\
-10.3140032130643	-4.29340329978466\\
-10.2855714703684	-4.26497155708878\\
-10.2572324904563	-4.23663257717665\\
-10.2289856699911	-4.20838575671149\\
-10.2008304115033	-4.18023049822368\\
-10.1727661233145	-4.15216621003492\\
-10.1447922194633	-4.12419230618363\\
-10.1169081196311	-4.09630820635152\\
-10.089113249071	-4.06851333579141\\
-10.0614070385357	-4.04080712525608\\
-10.033788924208	-4.01318901092837\\
-10.0062583476319	-3.9856584343523\\
-9.97881475564497	-3.95821484236535\\
-9.95145760031135	-3.93085768703172\\
-9.92418633885638	-3.90358642557675\\
-9.89700043360188	-3.87640052032226\\
-9.86989935190256	-3.84929943862294\\
-9.84288256608338	-3.82228265280376\\
-9.81594955337794	-3.79534964009832\\
-9.78909979586776	-3.76849988258813\\
-9.76233278042251	-3.74173286714289\\
-9.73564799864122	-3.7150480853616\\
-9.70904494679428	-3.68844503351466\\
-9.68252312576642	-3.6619232124868\\
-9.65608204100052	-3.63548212772089\\
-9.62972120244225	-3.60912128916263\\
-9.60344012448563	-3.582840211206\\
-9.57723832591928	-3.55663841263965\\
-9.5511153298736	-3.53051541659398\\
-9.52507066376871	-3.50448808761146\\
-9.4991038592631	-3.47861198010854\\
-9.47321445220312	-3.45289028582865\\
-9.44740198257323	-3.42732186360324\\
-9.42166599444691	-3.40190558641477\\
-9.39600603593836	-3.37664034117112\\
-9.3704216591549	-3.35152502848453\\
-9.34491242015005	-3.3265585624549\\
-9.3194778788773	-3.30173987045747\\
-9.29411759914459	-3.27706789293464\\
-9.2688311485694	-3.25254158319192\\
-9.24361809853452	-3.22815990719791\\
-9.21847802414447	-3.2039218433881\\
-9.19341050418253	-3.17982638247266\\
-9.16841512106838	-3.15587252724775\\
-9.1434914608164	-3.13205929241068\\
-9.11863911299449	-3.10838570437849\\
-9.09385767068352	-3.08485080111008\\
-9.06914673043738	-3.06145363193173\\
-9.04450589224355	-3.0381932573659\\
-9.01993475948424	-3.01506874896338\\
-8.99543293889812	-2.9920791891385\\
-8.9710000405425	-2.96922367100751\\
-8.94663567775614	-2.94650129822999\\
-8.92233946712251	-2.92391118485318\\
-8.89811102843362	-2.9014524551593\\
-8.87394998465425	-2.87912424351559\\
-8.84985596188684	-2.85692569422724\\
-8.82582858933669	-2.83485596139294\\
-8.80186749927775	-2.8129142087631\\
-8.77797232701888	-2.79109960960074\\
-8.75414271087051	-2.76941134654483\\
-8.73037829211179	-2.74784861147611\\
-8.70667871495821	-2.72641060538542\\
-8.68304362652965	-2.70509653824435\\
-8.65947267681879	-2.68390562887821\\
-8.6359655186601	-2.6628371048413\\
-8.61252180769908	-2.64189020229438\\
-8.58914120236205	-2.62106416588439\\
-8.56582336382625	-2.60035824862619\\
-8.5425679559904	-2.57977171178648\\
-8.51937464544562	-2.55930382476976\\
-8.49624310144678	-2.53895386500622\\
-8.47317299588414	-2.51872111784169\\
-8.45016400325549	-2.49860487642944\\
-8.42721580063855	-2.47860444162386\\
-8.4043280676638	-2.45871912187603\\
-8.38150048648761	-2.43894823313108\\
-8.35873274176583	-2.41929109872724\\
-8.33602452062754	-2.39974704929676\\
-8.31337551264938	-2.38031542266837\\
-8.29078540982999	-2.36099556377157\\
-8.2682539065649	-2.34178682454234\\
-8.24578069962177	-2.32268856383069\\
-8.22336548811585	-2.30370014730952\\
-8.20100797348585	-2.28482094738524\\
-8.17870785947002	-2.2660503431097\\
-8.15646485208266	-2.24738772009372\\
-8.13427865959085	-2.22883247042201\\
-8.11214899249147	-2.21038399256952\\
-8.09007556348852	-2.19204169131914\\
-8.0680580874708	-2.1738049776808\\
-8.04609628148975	-2.15567326881189\\
-8.0241898647377	-2.13764598793896\\
-8.00233855852624	-2.11972256428075\\
-7.98054208626503	-2.10190243297244\\
-7.95880017344075	-2.08418503499112\\
-7.93711254759635	-2.0665698170825\\
-7.9154789383106	-2.04905623168881\\
-7.89389907717781	-2.03164373687782\\
-7.8723726977879	-2.014331796273\\
-7.85089953570663	-1.99711987898489\\
-7.82947932845612	-1.98000745954343\\
-7.8081118154956	-1.96299401783149\\
-7.7867967382024	-1.94607903901936\\
-7.76553383985316	-1.92926201350033\\
-7.74432286560529	-1.91254243682729\\
-7.72316356247862	-1.89591980965029\\
-7.70205567933731	-1.87939363765512\\
-7.68099896687198	-1.86296343150285\\
-7.65999317758202	-1.84662870677024\\
-7.63903806575815	-1.83038898389119\\
-7.61813338746514	-1.81424378809904\\
-7.59727890052485	-1.79819264936973\\
-7.57647436449929	-1.78223510236592\\
-7.55571954067409	-1.76637068638188\\
-7.53501419204199	-1.75059894528931\\
-7.51435808328663	-1.73491942748394\\
-7.49375098076652	-1.71933168583293\\
-7.47319265249915	-1.70383527762311\\
-7.45268286814535	-1.68842976451002\\
-7.43222139899377	-1.67311471246765\\
-7.41180801794562	-1.65788969173902\\
-7.39144249949952	-1.64275427678743\\
-7.37112461973656	-1.62770804624853\\
-7.35085415630552	-1.61275058288299\\
-7.33063088840827	-1.59788147353003\\
-7.31045459678537	-1.58310030906148\\
-7.29032506370176	-1.56840668433665\\
-7.27024207293269	-1.55380019815782\\
-7.25020540974979	-1.53928045322636\\
-7.23021486090725	-1.52484705609956\\
-7.21027021462828	-1.51049961714806\\
-7.19037126059157	-1.49623775051388\\
-7.17051778991801	-1.48206107406914\\
-7.15070959515757	-1.4679692093753\\
-7.13094647027625	-1.45396178164305\\
-7.11122821064323	-1.44003841969278\\
-7.09155461301816	-1.42619875591556\\
-7.07192547553861	-1.4124424262348\\
-7.05234059770761	-1.39876907006834\\
-7.03279978038137	-1.38517833029121\\
-7.01330282575716	-1.37166985319877\\
-6.99384953736127	-1.35824328847056\\
-6.97443972003712	-1.34489828913454\\
-6.95507317993354	-1.33163451153186\\
-6.93574972449313	-1.31845161528217\\
-6.91646916244079	-1.3053492632494\\
-6.89723130377236	-1.29232712150805\\
-6.87803595974336	-1.27938485930989\\
-6.85888294285792	-1.26652214905121\\
-6.83977206685775	-1.2537386662405\\
-6.8207031467113	-1.24103408946655\\
-6.801675998603	-1.22840810036709\\
-6.78269043992262	-1.21586038359774\\
-6.76374628925478	-1.20339062680153\\
-6.74484336636852	-1.19099852057872\\
-6.72598149220704	-1.17868375845715\\
-6.70716048887749	-1.16644603686294\\
-6.68838017964094	-1.15428505509157\\
-6.66964038890238	-1.14220051527946\\
-6.65094094220092	-1.13019212237585\\
-6.6322816662	-1.11825958411507\\
-6.61366238867776	-1.1064026109893\\
-6.59508293851752	-1.09462091622155\\
-6.57654314569833	-1.08291421573913\\
-6.55804284128565	-1.07128222814741\\
-6.53958185742208	-1.05972467470401\\
-6.52116002731824	-1.04824127929325\\
-6.50277718524377	-1.03683176840104\\
-6.4844331665183	-1.02549587109006\\
-6.46612780750267	-1.01423331897526\\
-6.44786094559014	-1.00304384619977\\
-6.42963241919772	-0.991927189411033\\
-6.41144206775762	-0.980883087737345\\
-6.39328973170874	-0.969911282764671\\
-6.37517525248826	-0.959011518513778\\
-6.35709847252337	-0.948183541417684\\
-6.33905923522301	-0.937427100299405\\
-6.32105738496976	-0.926741946349989\\
-6.30309276711175	-0.916127833106869\\
-6.28516522795473	-0.905584516432486\\
-6.26727461475413	-0.895111754493201\\
-6.24942077570731	-0.884709307738494\\
-6.23160355994579	-0.874376938880434\\
-6.2138228175276	-0.864114412873422\\
-6.19607839942973	-0.853921496894215\\
-6.17837015754063	-0.84379796032219\\
-6.16069794465279	-0.833743574719901\\
-6.1430616144554	-0.823758113813863\\
-6.12546102152706	-0.813841353475616\\
-6.10789602132863	-0.803993071703021\\
-6.09036647019605	-0.79421304860181\\
-6.07287222533336	-0.784501066367377\\
-6.05541314480565	-0.774856909266815\\
-6.0379890875322	-0.765280363621181\\
-6.02059991327962	-0.755771217787996\\
-6.00324548265509	-0.746329262143978\\
-5.98592565709961	-0.736954289068002\\
-5.96864029888145	-0.727646092924279\\
-5.95138927108949	-0.71840447004575\\
-5.93417243762677	-0.709229218717721\\
-5.91698966320402	-0.700120139161685\\
-5.89984081333328	-0.691077033519365\\
-5.88272575432161	-0.682099705836971\\
-5.86564435326483	-0.673187962049661\\
-5.84859647804127	-0.664341609966194\\
-5.83158199730575	-0.655560459253798\\
-5.81460078048338	-0.646844321423227\\
-5.79765269776367	-0.638193009814013\\
-5.78073762009448	-0.629606339579912\\
-5.76385541917618	-0.621084127674543\\
-5.74700596745577	-0.612626192837197\\
-5.73018913812115	-0.604232355578862\\
-5.71340480509534	-0.595902438168403\\
-5.69665284303084	-0.587636264618922\\
-5.67993312730402	-0.579433660674319\\
-5.66324553400951	-0.571294453796004\\
-5.64658993995476	-0.563218473149789\\
-5.62996622265452	-0.555205549592967\\
-5.61337426032547	-0.547255515661517\\
-5.59681393188086	-0.539368205557534\\
-5.58028511692522	-0.531543455136773\\
-5.56378769574907	-0.523781101896374\\
-5.54732154932375	-0.516080984962757\\
-5.53088655929629	-0.508442945079659\\
-5.51448260798422	-0.500866824596326\\
-5.49810957837062	-0.493352467455886\\
-5.48176735409904	-0.485899719183825\\
-5.46545581946856	-0.478508426876671\\
-5.44917485942887	-0.471178439190777\\
-5.43292435957543	-0.463909606331279\\
-5.4167042061446	-0.456701780041191\\
-5.40051428600889	-0.44955481359064\\
-5.38435448667222	-0.442468561766245\\
-5.36822469626523	-0.435442880860638\\
-5.35212480354063	-0.428477628662116\\
-5.33605469786861	-0.421572664444436\\
-5.32001426923226	-0.414727848956742\\
-5.30400340822306	-0.407943044413621\\
-5.2880220060364	-0.4012181144853\\
-5.27206995446715	-0.394552924287962\\
-5.25614714590525	-0.387947340374194\\
-5.24025347333139	-0.381401230723575\\
-5.22438883031262	-0.374914464733357\\
-5.20855311099816	-0.368486913209311\\
-5.19274621011512	-0.362118448356661\\
-5.1769680229643	-0.355808943771157\\
-5.16121844541602	-0.349558274430267\\
-5.14549737390603	-0.343366316684482\\
-5.1298047054314	-0.337232948248733\\
-5.11414033754648	-0.331158048193946\\
-5.09850416835885	-0.32514149693868\\
-5.08289609652542	-0.319183176240905\\
-5.06731602124842	-0.313282969189878\\
-5.05176384227154	-0.307440760198134\\
-5.03623945987599	-0.301656434993584\\
-5.02074277487677	-0.295929880611726\\
-5.00527368861878	-0.290260985387951\\
-4.98983210297308	-0.284649638949984\\
-4.97441792033315	-0.279095732210384\\
-4.95903104361123	-0.273599157359197\\
-4.94367137623457	-0.268159807856674\\
-4.92833882214187	-0.262777578426113\\
-4.91303328577962	-0.257452365046787\\
-4.89775467209858	-0.252184064946992\\
-4.88250288655017	-0.246972576597158\\
-4.86727783508304	-0.241817799703103\\
-4.85207942413951	-0.236719635199344\\
-4.8369075606522	-0.231677985242528\\
-4.82176215204053	-0.226692753204942\\
-4.80664310620739	-0.22176384366814\\
-4.79155033153576	-0.216891162416622\\
-4.77648373688537	-0.212074616431654\\
-4.76144323158942	-0.207314113885141\\
-4.74642872545128	-0.202609564133612\\
-4.73144012874125	-0.197960877712275\\
-4.71647735219339	-0.19336796632919\\
-4.70154030700223	-0.188830742859491\\
-4.68662890481972	-0.184349121339734\\
-4.67174305775201	-0.1799230169623\\
-4.65688267835639	-0.175552346069899\\
-4.64204767963819	-0.171237026150161\\
-4.62723797504771	-0.166976975830293\\
-4.61245347847723	-0.162772114871846\\
-4.59769410425797	-0.158622364165535\\
-4.58295976715712	-0.154527645726166\\
-4.56825038237489	-0.150487882687624\\
-4.55356586554161	-0.146502999297957\\
-4.53890613271475	-0.142572920914527\\
-4.52427110037613	-0.138697573999247\\
-4.50966068542901	-0.134876886113895\\
-4.49507480519527	-0.131110785915505\\
-4.48051337741262	-0.127399203151831\\
-4.46597632023178	-0.1237420686569\\
-4.45146355221377	-0.120139314346617\\
-4.43697499232713	-0.116590873214477\\
-4.42251055994521	-0.113096679327319\\
-4.40807017484351	-0.109656667821179\\
-4.39365375719697	-0.106270774897205\\
-4.37926122757736	-0.10293893781764\\
-4.36489250695062	-0.0996610949018904\\
-4.35054751667428	-0.0964371855226607\\
-4.33622617849485	-0.0932671501021509\\
-4.3219284145453	-0.0901509301083387\\
-4.30765414734249	-0.0870884680513211\\
-4.29340329978466	-0.0840797074797304\\
-4.27917579514892	-0.0811245929772203\\
-4.26497155708878	-0.07822307015902\\
-4.2507905096317	-0.0753750856685514\\
-4.23663257717665	-0.0725805871741272\\
-4.22249768449166	-0.0698395233656985\\
-4.20838575671149	-0.0671518439516861\\
-4.19429671933517	-0.0645174996558674\\
-4.18023049822368	-0.0619364422143368\\
-4.16618701959764	-0.0594086243725251\\
-4.15216621003492	-0.0569339998822907\\
-4.1381679964684	-0.0545125234990719\\
-4.12419230618363	-0.0521441509791062\\
-4.11023906681661	-0.0498288390767117\\
-4.09630820635152	-0.0475665455416286\\
-4.0823996531185	-0.0453572291164417\\
-4.06851333579141	-0.0432008495340396\\
-4.05464918338567	-0.0410973675151629\\
-4.04080712525608	-0.039046744765993\\
-4.02698709109462	-0.0370489439758198\\
-4.01318901092837	-0.0351039288147634\\
-3.99941281511731	-0.0332116639315603\\
-3.9856584343523	-0.0313721149514047\\
-3.9719257996529	-0.0295852484738618\\
-3.95821484236535	-0.027851032070834\\
-3.94452549416049	-0.0261694342845914\\
-3.93085768703172	-0.0245404246258592\\
-3.91721135329299	-0.0229639735719676\\
-3.90358642557675	-0.021440052565063\\
-3.889982836832	-0.0199686340103775\\
-3.87640052032226	-0.0185496912745556\\
-3.86283940962365	-0.0171831986840413\\
-3.84929943862293	-0.0158691315235299\\
-3.83578054151556	-0.0146074660344718\\
-3.82228265280376	-0.0133981794136331\\
-3.80880570729464	-0.0122412498117208\\
-3.79534964009832	-0.0111366563320688\\
-3.78191438662599	-0.0100843790293701\\
-3.76849988258813	-0.00908439890847959\\
-3.75510606399261	-0.00813669792326486\\
-3.74173286714289	-0.0072412589755232\\
-3.72838022863616	-0.00639806591394796\\
-3.7150480853616	-0.00560710353316019\\
-3.70173637449852	-0.00486835757278982\\
-3.68844503351466	-0.0041818147166217\\
-3.67517400016434	-0.00354746259178908\\
-3.6619232124868	-0.00296528976804235\\
-3.64869260880438	-0.0024352857570418\\
-3.63548212772089	-0.00195744101174773\\
-3.6222917081198	-0.00153174692583443\\
-3.60912128916263	-0.00115819583318005\\
-3.5959708102872	-0.000836781007399546\\
-3.582840211206	-0.000567496661445672\\
-3.56972943190454	-0.000350337947264509\\
-3.55663841263965	-0.000185300955495804\\
-3.54356709393791	-7.23827152451176e-05\\
-3.53051541659398	-1.15811939055388e-05\\
-3.51748332166902	0\\
-3.50447075048909	0\\
-3.49147764464354	0\\
-3.47850394598347	0\\
-3.46554959662016	0\\
-3.4526145389235	0\\
-3.43969871552046	0\\
-3.4268020692936	0\\
-3.4139245433795	0\\
-3.40106608116728	0\\
-3.38822662629711	0\\
-3.37540612265873	0\\
-3.36260451438997	0\\
-3.34982174587527	0\\
-3.3370577617443	0\\
-3.32431250687042	0\\
-3.31158592636935	0\\
-3.29887796559768	0\\
-3.28618857015149	0\\
-3.27351768586497	0\\
-3.26086525880899	0\\
-3.24823123528977	0\\
-3.23561556184748	0\\
-3.22301818525489	0\\
-3.21043905251603	0\\
-3.19787811086484	0\\
-3.18533530776386	0\\
-3.1728105909029	0\\
-3.16030390819772	0\\
-3.14781520778876	0\\
-3.13534443803981	0\\
-3.12289154753678	0\\
-3.11045648508637	0\\
-3.09803919971486	0\\
-3.08563964066683	0\\
-3.0732577574039	0\\
-3.06089349960352	0\\
-3.04854681715776	0\\
-3.03621766017203	0\\
-3.02390597896393	0\\
-3.01161172406201	0\\
-2.99933484620462	0\\
-2.98707529633867	0\\
-2.97483302561849	0\\
-2.96260798540467	0\\
-2.95040012726287	0\\
-2.93820940296269	0\\
-2.92603576447651	0\\
-2.91387916397839	0\\
-2.90173955384289	0\\
-2.889616886644	0\\
-2.87751111515399	0\\
-2.86542219234235	0\\
-2.85335007137463	0\\
-2.84129470561142	0\\
-2.82925604860722	0\\
-2.81723405410938	0\\
-2.80522867605706	0\\
-2.79323986858013	0\\
-2.78126758599813	0\\
-2.76931178281924	0\\
-2.75737241373926	0\\
-2.74544943364051	0\\
-2.73354279759088	0\\
-2.72165246084279	0\\
-2.70977837883216	0\\
-2.69792050717744	0\\
-2.68607880167859	0\\
-2.6742532183161	0\\
-2.66244371325002	0\\
-2.65065024281897	0\\
-2.63887276353917	0\\
-2.62711123210349	0\\
-2.61536560538048	0\\
-2.60363584041344	0\\
-2.59192189441946	0\\
-2.58022372478849	0\\
-2.56854128908243	0\\
-2.55687454503414	0\\
-2.54522345054662	0\\
-2.53358796369202	0\\
-2.52196804271077	0\\
-2.51036364601067	0\\
-2.498774732166	0\\
-2.48720125991663	0\\
-2.47564318816716	0\\
-2.46410047598599	0\\
-2.45257308260452	0\\
-2.44106096741623	0\\
-2.42956408997587	0\\
-2.41808240999854	0\\
-2.40661588735893	0\\
-2.39516448209039	0\\
-2.38372815438417	0\\
-2.37230686458854	0\\
-2.36090057320799	0\\
-2.34950924090239	0\\
-2.3381328284862	0\\
-2.32677129692765	0\\
-2.31542460734792	0\\
-2.30409272102038	0\\
-2.29277559936976	0\\
-2.28147320397138	0\\
-2.27018549655036	0\\
-2.25891243898086	0\\
-2.24765399328528	0\\
-2.2364101216335	0\\
-2.22518078634215	0\\
-2.21396594987379	0\\
-2.20276557483623	0\\
-2.19157962398171	0\\
-2.18040806020622	0\\
-2.16925084654871	0\\
-2.15810794619039	0\\
-2.14697932245399	0\\
-2.13586493880304	0\\
-2.12476475884113	0\\
-2.11367874631123	0\\
-2.10260686509495	0\\
-2.09154907921184	0\\
-2.08050535281871	0\\
-2.06947565020889	0\\
-2.05845993581159	0\\
-2.04745817419117	0\\
-2.03647033004647	0\\
-2.02549636821013	0\\
-2.01453625364792	0\\
-2.00358995145807	0\\
-1.99265742687059	0\\
-1.98173864524662	0\\
-1.97083357207775	0\\
-1.95994217298541	0\\
-1.94906441372017	0\\
-1.93820026016113	0\\
-1.92734967831525	0\\
-1.91651263431673	0\\
-1.90568909442638	0\\
-1.89487902503097	0\\
-1.88408239264263	0\\
-1.87329916389819	0\\
-1.86252930555859	0\\
-1.85177278450828	0\\
-1.84102956775455	0\\
-1.830299622427	0\\
-1.81958291577688	0\\
-1.80887941517649	0\\
-1.79818908811864	0\\
-1.78751190221597	0\\
-1.77684782520047	0\\
-1.76619682492278	0\\
-1.75555886935169	0\\
-1.74493392657354	0\\
-1.73432196479163	0\\
-1.72372295232567	0\\
-1.71313685761119	0\\
-1.70256364919899	0\\
-1.6920032957546	0\\
-1.68145576605768	0\\
-1.6709210290015	0\\
-1.66039905359236	0\\
-1.64988980894907	0\\
-1.6393932643024	0\\
-1.62890938899453	0\\
-1.61843815247852	0\\
-1.60797952431778	0\\
-1.59753347418552	0\\
-1.58709997186425	0\\
-1.57667898724523	0\\
-1.56627049032796	0\\
-1.55587445121967	0\\
-1.5454908401348	0\\
-1.53511962739447	0\\
-1.52476078342599	0\\
-1.51441427876237	0\\
-1.50408008404176	0\\
-1.49375817000701	0\\
-1.48344850750515	0\\
-1.4731510674869	0\\
-1.46286582100615	0\\
-1.45259273921953	0\\
-1.44233179338586	0\\
-1.43208295486572	0\\
-1.42184619512095	0\\
-1.41162148571415	0\\
-1.40140879830824	0\\
-1.391208104666	0\\
-1.38101937664954	0\\
-1.3708425862199	0\\
-1.36067770543655	0\\
-1.35052470645694	0\\
-1.34038356153604	0\\
-1.33025424302589	0\\
-1.32013672337515	0\\
-1.31003097512865	0\\
-1.2999369709269	0\\
-1.28985468350574	0\\
-1.27978408569581	0\\
-1.26972515042213	0\\
-1.25967785070372	0\\
-1.24964215965307	0\\
-1.23961805047579	0\\
-1.22960549647016	0\\
-1.21960447102667	0\\
-1.20961494762763	0\\
-1.19963689984674	0\\
-1.18967030134866	0\\
-1.17971512588861	0\\
-1.16977134731194	0\\
-1.15983893955373	0\\
-1.14991787663839	0\\
-1.14000813267919	0\\
-1.13010968187795	0\\
-1.12022249852456	0\\
-1.11034655699663	0\\
-1.10048183175904	0\\
-1.09062829736361	0\\
-1.08078592844863	0\\
-1.07095469973854	0\\
-1.06113458604349	0\\
-1.05132556225898	0\\
-1.04152760336547	0\\
-1.03174068442798	0\\
-1.02196478059573	0\\
-1.01219986710174	0\\
-1.0024459192625	0\\
-0.992702912477539	0\\
-0.982970822229071	0\\
-0.973249624081646	0\\
-0.96353929368176	0\\
-0.953839806757496	0\\
-0.944151139118158	0\\
-0.934473266653912	0\\
-0.924806165335424	0\\
-0.915149811213502	0\\
-0.90550418041874	0\\
-0.895869249161165	0\\
-0.886244993729884	0\\
-0.876631390492734	0\\
-0.867028415895933	0\\
-0.857436046463738	0\\
-0.847854258798095	0\\
-0.838283029578297	0\\
-0.828722335560651	0\\
-0.819172153578128	0\\
-0.809632460540035	0\\
-0.800103233431676	0\\
-0.790584449314021	0\\
-0.781076085323371	0\\
-0.771578118671034	0\\
-0.762090526642992	0\\
-0.752613286599578	0\\
-0.743146375975151	0\\
-0.733689772277774	0\\
-0.724243453088894	0\\
-0.714807396063021	0\\
-0.705381578927413	0\\
-0.695965979481758	0\\
-0.686560575597866	0\\
-0.677165345219347	0\\
-0.667780266361313	0\\
-0.658405317110058	0\\
-0.649040475622759	0\\
-0.639685720127164	0\\
-0.630341028921298	0\\
-0.621006380373147	0\\
-0.611681752920373	0\\
-0.602367125070001	0\\
-0.593062475398133	0\\
-0.583767782549644	0\\
-0.574483025237896	0\\
-0.565208182244434	0\\
-0.555943232418711	0\\
-0.546688154677781	0\\
-0.537442928006027	0\\
-0.528207531454862	0\\
-0.518981944142453	0\\
-0.509766145253433	0\\
-0.500560114038621	0\\
-0.491363829814742	0\\
-0.482177271964145	0\\
-0.473000419934532	0\\
-0.463833253238675	0\\
-0.454675751454147	0\\
-0.445527894223045	0\\
-0.436389661251722	0\\
-0.427261032310513	0\\
-0.418141987233472	0\\
-0.409032505918098	0\\
-0.399932568325074	0\\
-0.390842154477999	0\\
-0.381761244463129	0\\
-0.372689818429112	0\\
-0.363627856586728	0\\
-0.354575339208632	0\\
-0.345532246629093	0\\
-0.336498559243741	0\\
-0.32747425750931	0\\
-0.318459321943385	0\\
-0.309453733124149	0\\
-0.300457471690133	0\\
-0.291470518339967	0\\
-0.282492853832127	0\\
-0.273524458984694	0\\
-0.264565314675103	0\\
-0.255615401839903	0\\
-0.246674701474509	0\\
-0.237743194632962	0\\
-0.22882086242769	0\\
-0.219907686029264	0\\
-0.211003646666164	0\\
-0.202108725624539	0\\
-0.193222904247971	0\\
-0.184346163937244	0\\
-0.175478486150103	0\\
-0.16661985240103	0\\
-0.157770244261007	0\\
-0.148929643357288	0\\
-0.14009803137317	0\\
-0.131275390047765	0\\
-0.122461701175776	0\\
-0.113656946607265	0\\
-0.104861108247438	0\\
-0.096074168056412	0\\
-0.0872961080490018	0\\
-0.0785269102944935	0\\
-0.0697665569164269	0\\
-0.0610150300923765	0\\
-0.0522723120537338	0\\
-0.0435383850854909	0\\
-0.0348132315260254	0\\
-0.0260968337668856	0\\
-0.0173891742525778	0\\
-0.00869023548035383	0\\
0	0\\
};
\end{axis}
\end{tikzpicture}%
Up to the threshold of 1/3 the input is multiplied by two and the characteristic curve is in its linear region. Between input values of 1/3 up to 2/3, the characteristic curve produces a soft compression described by the middle term of above equation. Above input values of 2/3 the output value is set to one. The corresponding {\bfseries Matlab} code for overdrive with symmetrical soft clipping is shown next.
\begin{lstlisting}
function y=symclip(audio)
% "Overdrive" simulation with symmetrical clipping

N = length(audio);
th = 1 / 3; % threshold for symmetrical soft clipping 
for n = 1: 1: N,
	if abs(audio(n)) < th
		y(n) = 2 * audio(n);
	end
	if abs(audio(n))>=th
		if audio(n) > 0
			y(n) = (3 - (2 - audio(n) * 3) .^ 2) / 3; 
		end
		if audio(n) < 0
			y(n) = -(3 - (2 - abs(audio(n)) * 3) .^	2) / 3; 
		end
	end 
	if abs(audio(n)) > 2 * th
		if audio(n) > 0
			y(n) = 1;
		end
		if audio(n) < 0
			y(n) = -1;
		end
	end
end
\end{lstlisting}
{\bfseries Destortion.} A nonlinearity suitable for the simulation of distortion is given by
\[
f(x)=\operatorname{sgn}(x)\left(1-\mathrm{e}^{-|x|}\right).
\]
The static characteristic curve is illustrated in following figure.

% This file was created by matlab2tikz.
%
%The latest updates can be retrieved from
%  http://www.mathworks.com/matlabcentral/fileexchange/22022-matlab2tikz-matlab2tikz
%where you can also make suggestions and rate matlab2tikz.
%
\definecolor{mycolor1}{rgb}{0.00000,0.44700,0.74100}%
%
\begin{tikzpicture}

\begin{axis}[%
width=1.703in,
height=1.573in,
at={(0.711in,0.424in)},
scale only axis,
xmin=-1,
xmax=1,
xlabel style={font=\color{white!15!black}},
xlabel={x},
ymin=-1.1,
ymax=1.1,
ylabel style={font=\color{white!15!black}},
ylabel={y},
%axis background/.style={fill=white},
title style={font=\bfseries},
title={Static characteristic: y=f(x)},
xmajorgrids,
ymajorgrids
]
\addplot [color=mycolor1, forget plot]
  table[row sep=crcr]{%
-1	-0.997769123041\\
-0.999	-0.997655697543029\\
-0.998	-0.997542191249927\\
-0.997	-0.997428603675465\\
-0.996	-0.997314934330491\\
-0.995	-0.997201182722906\\
-0.994	-0.997087348357651\\
-0.993	-0.996973430736687\\
-0.992	-0.996859429358978\\
-0.991	-0.996745343720473\\
-0.99	-0.996631173314088\\
-0.989	-0.996516917629687\\
-0.988	-0.996402576154062\\
-0.987	-0.996288148370921\\
-0.986	-0.99617363376086\\
-0.985	-0.996059031801352\\
-0.984	-0.995944341966726\\
-0.983	-0.995829563728145\\
-0.982	-0.995714696553592\\
-0.981	-0.995599739907846\\
-0.98	-0.995484693252467\\
-0.979	-0.995369556045773\\
-0.978	-0.995254327742824\\
-0.977	-0.995139007795398\\
-0.976	-0.995023595651977\\
-0.975	-0.994908090757721\\
-0.974	-0.994792492554452\\
-0.973	-0.994676800480633\\
-0.972	-0.994561013971349\\
-0.971	-0.994445132458282\\
-0.97	-0.994329155369697\\
-0.969	-0.994213082130417\\
-0.968	-0.994096912161804\\
-0.967	-0.993980644881736\\
-0.966	-0.993864279704591\\
-0.965	-0.993747816041219\\
-0.964	-0.993631253298928\\
-0.963	-0.993514590881456\\
-0.962	-0.993397828188955\\
-0.961	-0.993280964617967\\
-0.96	-0.9931639995614\\
-0.959	-0.993046932408511\\
-0.958	-0.992929762544881\\
-0.957	-0.992812489352392\\
-0.956	-0.992695112209208\\
-0.955	-0.99257763048975\\
-0.954	-0.992460043564674\\
-0.953	-0.992342350800849\\
-0.952	-0.992224551561334\\
-0.951	-0.992106645205353\\
-0.95	-0.991988631088276\\
-0.949	-0.991870508561593\\
-0.948	-0.991752276972891\\
-0.947	-0.991633935665831\\
-0.946	-0.991515483980124\\
-0.945	-0.991396921251507\\
-0.944	-0.99127824681172\\
-0.943	-0.991159459988481\\
-0.942	-0.991040560105463\\
-0.941	-0.990921546482267\\
-0.94	-0.990802418434401\\
-0.939	-0.990683175273253\\
-0.938	-0.990563816306068\\
-0.937	-0.99044434083592\\
-0.936	-0.990324748161691\\
-0.935	-0.990205037578042\\
-0.934	-0.99008520837539\\
-0.933	-0.989965259839881\\
-0.932	-0.989845191253367\\
-0.931	-0.989725001893375\\
-0.93	-0.989604691033087\\
-0.929	-0.989484257941309\\
-0.928	-0.989363701882449\\
-0.927	-0.989243022116485\\
-0.926	-0.989122217898944\\
-0.925	-0.989001288480872\\
-0.924	-0.988880233108807\\
-0.923	-0.988759051024753\\
-0.922	-0.988637741466155\\
-0.921	-0.988516303665864\\
-0.92	-0.988394736852119\\
-0.919	-0.98827304024851\\
-0.918	-0.988151213073959\\
-0.917	-0.988029254542685\\
-0.916	-0.987907163864177\\
-0.915	-0.98778494024317\\
-0.914	-0.98766258287961\\
-0.913	-0.987540090968631\\
-0.912	-0.98741746370052\\
-0.911	-0.987294700260694\\
-0.91	-0.987171799829666\\
-0.909	-0.987048761583019\\
-0.908	-0.986925584691373\\
-0.907	-0.986802268320357\\
-0.906	-0.98667881163058\\
-0.905	-0.986555213777598\\
-0.904	-0.986431473911886\\
-0.903	-0.986307591178806\\
-0.902	-0.986183564718578\\
-0.901	-0.986059393666245\\
-0.9	-0.985935077151649\\
-0.899	-0.98581061429939\\
-0.898	-0.985686004228805\\
-0.897	-0.985561246053928\\
-0.896	-0.985436338883461\\
-0.895	-0.985311281820745\\
-0.894	-0.985186073963722\\
-0.893	-0.985060714404907\\
-0.892	-0.984935202231352\\
-0.891	-0.984809536524617\\
-0.89	-0.984683716360733\\
-0.889	-0.984557740810173\\
-0.888	-0.984431608937812\\
-0.887	-0.984305319802902\\
-0.886	-0.984178872459029\\
-0.885	-0.984052265954088\\
-0.884	-0.983925499330241\\
-0.883	-0.983798571623886\\
-0.882	-0.983671481865624\\
-0.881	-0.98354422908022\\
-0.88	-0.983416812286571\\
-0.879	-0.983289230497669\\
-0.878	-0.983161482720567\\
-0.877	-0.983033567956342\\
-0.876	-0.982905485200059\\
-0.875	-0.982777233440737\\
-0.874	-0.982648811661308\\
-0.873	-0.982520218838586\\
-0.872	-0.982391453943226\\
-0.871	-0.982262515939689\\
-0.87	-0.982133403786203\\
-0.869	-0.982004116434728\\
-0.868	-0.981874652830916\\
-0.867	-0.981745011914073\\
-0.866	-0.981615192617125\\
-0.865	-0.981485193866572\\
-0.864	-0.981355014582457\\
-0.863	-0.981224653678322\\
-0.862	-0.981094110061171\\
-0.861	-0.980963382631431\\
-0.86	-0.980832470282911\\
-0.859	-0.980701371902763\\
-0.858	-0.980570086371442\\
-0.857	-0.980438612562666\\
-0.856	-0.980306949343374\\
-0.855	-0.980175095573686\\
-0.854	-0.980043050106864\\
-0.853	-0.979910811789268\\
-0.852	-0.979778379460313\\
-0.851	-0.979645751952434\\
-0.85	-0.979512928091036\\
-0.849	-0.979379906694457\\
-0.848	-0.979246686573922\\
-0.847	-0.979113266533504\\
-0.846	-0.978979645370079\\
-0.845	-0.978845821873279\\
-0.844	-0.978711794825456\\
-0.843	-0.978577563001633\\
-0.842	-0.97844312516946\\
-0.841	-0.978308480089171\\
-0.84	-0.97817362651354\\
-0.839	-0.978038563187834\\
-0.838	-0.977903288849769\\
-0.837	-0.977767802229467\\
-0.836	-0.977632102049404\\
-0.835	-0.97749618702437\\
-0.834	-0.977360055861421\\
-0.833	-0.977223707259831\\
-0.832	-0.977087139911045\\
-0.831	-0.976950352498636\\
-0.83	-0.976813343698253\\
-0.829	-0.976676112177575\\
-0.828	-0.976538656596263\\
-0.827	-0.976400975605913\\
-0.826	-0.976263067850004\\
-0.825	-0.976124931963853\\
-0.824	-0.975986566574565\\
-0.823	-0.97584797030098\\
-0.822	-0.975709141753629\\
-0.821	-0.975570079534681\\
-0.82	-0.97543078223789\\
-0.819	-0.975291248448551\\
-0.818	-0.975151476743442\\
-0.817	-0.975011465690778\\
-0.816	-0.974871213850156\\
-0.815	-0.974730719772506\\
-0.814	-0.974589982000037\\
-0.813	-0.974448999066184\\
-0.812	-0.974307769495558\\
-0.811	-0.974166291803889\\
-0.81	-0.974024564497977\\
-0.809	-0.973882586075633\\
-0.808	-0.97374035502563\\
-0.807	-0.973597869827645\\
-0.806	-0.973455128952207\\
-0.805	-0.97331213086064\\
-0.804	-0.973168874005008\\
-0.803	-0.973025356828058\\
-0.802	-0.972881577763169\\
-0.801	-0.972737535234289\\
-0.8	-0.972593227655882\\
-0.799	-0.972448653432871\\
-0.798	-0.972303810960578\\
-0.797	-0.972158698624671\\
-0.796	-0.9720133148011\\
-0.795	-0.971867657856042\\
-0.794	-0.971721726145843\\
-0.793	-0.971575518016955\\
-0.792	-0.971429031805883\\
-0.791	-0.971282265839115\\
-0.79	-0.971135218433074\\
-0.789	-0.970987887894047\\
-0.788	-0.970840272518129\\
-0.787	-0.970692370591163\\
-0.786	-0.970544180388673\\
-0.785	-0.970395700175807\\
-0.784	-0.970246928207272\\
-0.783	-0.970097862727272\\
-0.782	-0.969948501969444\\
-0.781	-0.969798844156797\\
-0.78	-0.969648887501642\\
-0.779	-0.969498630205535\\
-0.778	-0.969348070459208\\
-0.777	-0.969197206442506\\
-0.776	-0.969046036324316\\
-0.775	-0.968894558262511\\
-0.774	-0.968742770403874\\
-0.773	-0.968590670884036\\
-0.772	-0.968438257827409\\
-0.771	-0.968285529347118\\
-0.77	-0.968132483544931\\
-0.769	-0.967979118511195\\
-0.768	-0.967825432324761\\
-0.767	-0.967671423052922\\
-0.766	-0.967517088751338\\
-0.765	-0.967362427463967\\
-0.764	-0.967207437222999\\
-0.763	-0.967052116048777\\
-0.762	-0.966896461949735\\
-0.761	-0.966740472922318\\
-0.76	-0.966584146950916\\
-0.759	-0.966427482007787\\
-0.758	-0.966270476052987\\
-0.757	-0.966113127034296\\
-0.756	-0.965955432887142\\
-0.755	-0.965797391534531\\
-0.754	-0.965639000886968\\
-0.753	-0.965480258842382\\
-0.752	-0.965321163286053\\
-0.751	-0.965161712090537\\
-0.75	-0.965001903115582\\
-0.749	-0.96484173420806\\
-0.748	-0.964681203201882\\
-0.747	-0.964520307917927\\
-0.746	-0.964359046163958\\
-0.745	-0.964197415734545\\
-0.744	-0.964035414410985\\
-0.743	-0.963873039961225\\
-0.742	-0.963710290139778\\
-0.741	-0.963547162687642\\
-0.74	-0.963383655332224\\
-0.739	-0.963219765787251\\
-0.738	-0.963055491752692\\
-0.737	-0.962890830914677\\
-0.736	-0.962725780945407\\
-0.735	-0.962560339503077\\
-0.734	-0.962394504231787\\
-0.733	-0.962228272761461\\
-0.732	-0.962061642707759\\
-0.731	-0.961894611671991\\
-0.73	-0.961727177241033\\
-0.729	-0.961559336987238\\
-0.728	-0.961391088468351\\
-0.727	-0.961222429227416\\
-0.726	-0.961053356792693\\
-0.725	-0.960883868677568\\
-0.724	-0.960713962380461\\
-0.723	-0.960543635384737\\
-0.722	-0.960372885158618\\
-0.721	-0.960201709155087\\
-0.72	-0.960030104811801\\
-0.719	-0.959858069550995\\
-0.718	-0.959685600779392\\
-0.717	-0.959512695888109\\
-0.716	-0.959339352252562\\
-0.715	-0.959165567232372\\
-0.714	-0.95899133817127\\
-0.713	-0.958816662397002\\
-0.712	-0.958641537221232\\
-0.711	-0.958465959939445\\
-0.71	-0.958289927830852\\
-0.709	-0.958113438158289\\
-0.708	-0.957936488168118\\
-0.707	-0.957759075090132\\
-0.706	-0.957581196137451\\
-0.705	-0.957402848506425\\
-0.704	-0.957224029376532\\
-0.703	-0.957044735910272\\
-0.702	-0.956864965253074\\
-0.701	-0.956684714533185\\
-0.7	-0.95650398086157\\
-0.699	-0.956322761331809\\
-0.698	-0.956141053019989\\
-0.697	-0.955958852984603\\
-0.696	-0.95577615826644\\
-0.695	-0.955592965888481\\
-0.694	-0.955409272855793\\
-0.693	-0.955225076155417\\
-0.692	-0.955040372756262\\
-0.691	-0.954855159608997\\
-0.69	-0.954669433645939\\
-0.689	-0.954483191780946\\
-0.688	-0.954296430909299\\
-0.687	-0.954109147907598\\
-0.686	-0.953921339633647\\
-0.685	-0.953733002926337\\
-0.684	-0.953544134605537\\
-0.683	-0.953354731471979\\
-0.682	-0.95316479030714\\
-0.681	-0.952974307873127\\
-0.68	-0.952783280912565\\
-0.679	-0.952591706148473\\
-0.678	-0.952399580284148\\
-0.677	-0.95220690000305\\
-0.676	-0.952013661968679\\
-0.675	-0.951819862824456\\
-0.674	-0.951625499193601\\
-0.673	-0.951430567679013\\
-0.672	-0.951235064863148\\
-0.671	-0.951038987307893\\
-0.67	-0.950842331554447\\
-0.669	-0.950645094123192\\
-0.668	-0.95044727151357\\
-0.667	-0.950248860203957\\
-0.666	-0.950049856651534\\
-0.665	-0.949850257292163\\
-0.664	-0.949650058540256\\
-0.663	-0.949449256788648\\
-0.662	-0.949247848408464\\
-0.661	-0.949045829748992\\
-0.66	-0.948843197137549\\
-0.659	-0.948639946879351\\
-0.658	-0.948436075257376\\
-0.657	-0.948231578532234\\
-0.656	-0.948026452942034\\
-0.655	-0.947820694702241\\
-0.654	-0.947614300005548\\
-0.653	-0.947407265021735\\
-0.652	-0.94719958589753\\
-0.651	-0.946991258756473\\
-0.65	-0.946782279698776\\
-0.649	-0.94657264480118\\
-0.648	-0.946362350116818\\
-0.647	-0.946151391675069\\
-0.646	-0.945939765481417\\
-0.645	-0.945727467517307\\
-0.644	-0.94551449374\\
-0.643	-0.945300840082428\\
-0.642	-0.945086502453047\\
-0.641	-0.94487147673569\\
-0.64	-0.944655758789419\\
-0.639	-0.944439344448375\\
-0.638	-0.94422222952163\\
-0.637	-0.944004409793033\\
-0.636	-0.943785881021061\\
-0.635	-0.943566638938666\\
-0.634	-0.943346679253121\\
-0.633	-0.943125997645862\\
-0.632	-0.94290458977234\\
-0.631	-0.942682451261858\\
-0.63	-0.942459577717416\\
-0.629	-0.942235964715554\\
-0.628	-0.94201160780619\\
-0.627	-0.941786502512461\\
-0.626	-0.941560644330565\\
-0.625	-0.941334028729592\\
-0.624	-0.941106651151367\\
-0.623	-0.940878507010284\\
-0.622	-0.94064959169314\\
-0.621	-0.94041990055897\\
-0.62	-0.940189428938878\\
-0.619	-0.939958172135872\\
-0.618	-0.939726125424694\\
-0.617	-0.939493284051647\\
-0.616	-0.939259643234427\\
-0.615	-0.939025198161952\\
-0.614	-0.938789943994182\\
-0.613	-0.938553875861955\\
-0.612	-0.938316988866802\\
-0.611	-0.938079278080777\\
-0.61	-0.937840738546276\\
-0.609	-0.937601365275862\\
-0.608	-0.937361153252082\\
-0.607	-0.937120097427287\\
-0.606	-0.936878192723453\\
-0.605	-0.936635434031995\\
-0.604	-0.936391816213585\\
-0.603	-0.936147334097966\\
-0.602	-0.935901982483766\\
-0.601	-0.935655756138311\\
-0.6	-0.935408649797437\\
-0.599	-0.935160658165299\\
-0.598	-0.934911775914185\\
-0.597	-0.934661997684314\\
-0.596	-0.934411318083657\\
-0.595	-0.93415973168773\\
-0.594	-0.933907233039407\\
-0.593	-0.93365381664872\\
-0.592	-0.933399476992663\\
-0.591	-0.933144208514991\\
-0.59	-0.932888005626023\\
-0.589	-0.932630862702437\\
-0.588	-0.932372774087071\\
-0.587	-0.932113734088718\\
-0.586	-0.931853736981921\\
-0.585	-0.931592777006766\\
-0.584	-0.931330848368678\\
-0.583	-0.931067945238208\\
-0.582	-0.930804061750825\\
-0.581	-0.930539192006707\\
-0.58	-0.930273330070524\\
-0.579	-0.930006469971229\\
-0.578	-0.929738605701841\\
-0.577	-0.929469731219227\\
-0.576	-0.929199840443889\\
-0.575	-0.928928927259739\\
-0.574	-0.928656985513885\\
-0.573	-0.928384009016405\\
-0.572	-0.928109991540127\\
-0.571	-0.927834926820402\\
-0.57	-0.927558808554882\\
-0.569	-0.927281630403291\\
-0.568	-0.927003385987195\\
-0.567	-0.926724068889777\\
-0.566	-0.926443672655602\\
-0.565	-0.926162190790385\\
-0.564	-0.92587961676076\\
-0.563	-0.925595943994042\\
-0.562	-0.925311165877993\\
-0.561	-0.925025275760578\\
-0.56	-0.924738266949735\\
-0.559	-0.924450132713126\\
-0.558	-0.924160866277898\\
-0.557	-0.923870460830439\\
-0.556	-0.923578909516133\\
-0.555	-0.92328620543911\\
-0.554	-0.922992341662005\\
-0.553	-0.922697311205699\\
-0.552	-0.922401107049075\\
-0.551	-0.922103722128763\\
-0.55	-0.921805149338884\\
-0.549	-0.921505381530796\\
-0.548	-0.921204411512837\\
-0.547	-0.920902232050063\\
-0.546	-0.920598835863994\\
-0.545	-0.920294215632343\\
-0.544	-0.919988363988761\\
-0.543	-0.919681273522566\\
-0.542	-0.919372936778479\\
-0.541	-0.919063346256355\\
-0.54	-0.918752494410912\\
-0.539	-0.918440373651461\\
-0.538	-0.918126976341631\\
-0.537	-0.917812294799098\\
-0.536	-0.917496321295301\\
-0.535	-0.917179048055172\\
-0.534	-0.91686046725685\\
-0.533	-0.916540571031403\\
-0.532	-0.916219351462543\\
-0.531	-0.915896800586342\\
-0.53	-0.915572910390944\\
-0.529	-0.915247672816278\\
-0.528	-0.914921079753767\\
-0.527	-0.914593123046036\\
-0.526	-0.914263794486619\\
-0.525	-0.913933085819664\\
-0.524	-0.913600988739633\\
-0.523	-0.913267494891006\\
-0.522	-0.912932595867979\\
-0.521	-0.912596283214163\\
-0.52	-0.912258548422276\\
-0.519	-0.911919382933841\\
-0.518	-0.911578778138873\\
-0.517	-0.911236725375574\\
-0.516	-0.910893215930019\\
-0.515	-0.910548241035841\\
-0.514	-0.910201791873917\\
-0.513	-0.909853859572051\\
-0.512	-0.909504435204654\\
-0.511	-0.909153509792421\\
-0.51	-0.90880107430201\\
-0.509	-0.908447119645717\\
-0.508	-0.908091636681146\\
-0.507	-0.907734616210883\\
-0.506	-0.907376048982164\\
-0.505	-0.907015925686539\\
-0.504	-0.906654236959543\\
-0.503	-0.906290973380353\\
-0.502	-0.905926125471452\\
-0.501	-0.905559683698286\\
-0.5	-0.905191638468922\\
-0.499	-0.904821980133705\\
-0.498	-0.904450698984904\\
-0.497	-0.904077785256368\\
-0.496	-0.903703229123175\\
-0.495	-0.903327020701273\\
-0.494	-0.90294915004713\\
-0.493	-0.902569607157371\\
-0.492	-0.902188381968424\\
-0.491	-0.90180546435615\\
-0.49	-0.901420844135485\\
-0.489	-0.90103451106007\\
-0.488	-0.900646454821884\\
-0.487	-0.900256665050871\\
-0.486	-0.899865131314568\\
-0.485	-0.899471843117729\\
-0.484	-0.899076789901947\\
-0.483	-0.898679961045275\\
-0.482	-0.898281345861841\\
-0.481	-0.897880933601467\\
-0.48	-0.89747871344928\\
-0.479	-0.89707467452532\\
-0.478	-0.896668805884153\\
-0.477	-0.896261096514477\\
-0.476	-0.895851535338719\\
-0.475	-0.895440111212645\\
-0.474	-0.895026812924952\\
-0.473	-0.894611629196869\\
-0.472	-0.894194548681745\\
-0.471	-0.89377555996465\\
-0.47	-0.893354651561955\\
-0.469	-0.892931811920922\\
-0.468	-0.892507029419289\\
-0.467	-0.892080292364852\\
-0.466	-0.891651588995041\\
-0.465	-0.8912209074765\\
-0.464	-0.890788235904657\\
-0.463	-0.890353562303302\\
-0.462	-0.889916874624146\\
-0.461	-0.889478160746399\\
-0.46	-0.889037408476323\\
-0.459	-0.888594605546801\\
-0.458	-0.888149739616889\\
-0.457	-0.887702798271377\\
-0.456	-0.88725376902034\\
-0.455	-0.886802639298685\\
-0.454	-0.886349396465706\\
-0.453	-0.885894027804623\\
-0.452	-0.885436520522124\\
-0.451	-0.88497686174791\\
-0.45	-0.884515038534225\\
-0.449	-0.884051037855395\\
-0.448	-0.883584846607357\\
-0.447	-0.883116451607185\\
-0.446	-0.88264583959262\\
-0.445	-0.882172997221589\\
-0.444	-0.881697911071726\\
-0.443	-0.881220567639891\\
-0.442	-0.880740953341678\\
-0.441	-0.880259054510931\\
-0.44	-0.879774857399253\\
-0.439	-0.879288348175504\\
-0.438	-0.878799512925311\\
-0.437	-0.878308337650564\\
-0.436	-0.87781480826891\\
-0.435	-0.877318910613249\\
-0.434	-0.876820630431222\\
-0.433	-0.8763199533847\\
-0.432	-0.875816865049266\\
-0.431	-0.875311350913697\\
-0.43	-0.874803396379439\\
-0.429	-0.874292986760086\\
-0.428	-0.873780107280846\\
-0.427	-0.873264743078015\\
-0.426	-0.872746879198435\\
-0.425	-0.872226500598962\\
-0.424	-0.871703592145921\\
-0.423	-0.871178138614563\\
-0.422	-0.870650124688513\\
-0.421	-0.870119534959225\\
-0.42	-0.869586353925421\\
-0.419	-0.869050565992538\\
-0.418	-0.868512155472163\\
-0.417	-0.867971106581469\\
-0.416	-0.86742740344265\\
-0.415	-0.866881030082345\\
-0.414	-0.866331970431063\\
-0.413	-0.865780208322609\\
-0.412	-0.8652257274935\\
-0.411	-0.864668511582374\\
-0.41	-0.864108544129411\\
-0.409	-0.863545808575732\\
-0.408	-0.862980288262808\\
-0.407	-0.862411966431856\\
-0.406	-0.861840826223242\\
-0.405	-0.861266850675865\\
-0.404	-0.860690022726555\\
-0.403	-0.860110325209455\\
-0.402	-0.859527740855401\\
-0.401	-0.858942252291306\\
-0.4	-0.858353842039529\\
-0.399	-0.857762492517247\\
-0.398	-0.857168186035825\\
-0.397	-0.856570904800175\\
-0.396	-0.855970630908121\\
-0.395	-0.855367346349747\\
-0.394	-0.854761033006756\\
-0.393	-0.854151672651812\\
-0.392	-0.853539246947887\\
-0.391	-0.852923737447601\\
-0.39	-0.852305125592556\\
-0.389	-0.85168339271267\\
-0.388	-0.851058520025501\\
-0.387	-0.850430488635578\\
-0.386	-0.849799279533712\\
-0.385	-0.849164873596319\\
-0.384	-0.848527251584727\\
-0.383	-0.847886394144486\\
-0.382	-0.84724228180467\\
-0.381	-0.846594894977175\\
-0.38	-0.845944213956016\\
-0.379	-0.845290218916618\\
-0.378	-0.844632889915099\\
-0.377	-0.843972206887555\\
-0.376	-0.843308149649337\\
-0.375	-0.842640697894322\\
-0.374	-0.841969831194187\\
-0.373	-0.841295528997668\\
-0.372	-0.840617770629824\\
-0.371	-0.839936535291293\\
-0.37	-0.839251802057538\\
-0.369	-0.8385635498781\\
-0.368	-0.837871757575838\\
-0.367	-0.837176403846164\\
-0.366	-0.83647746725628\\
-0.365	-0.835774926244404\\
-0.364	-0.835068759118994\\
-0.363	-0.834358944057969\\
-0.362	-0.833645459107921\\
-0.361	-0.832928282183324\\
-0.36	-0.832207391065744\\
-0.359	-0.831482763403032\\
-0.358	-0.830754376708525\\
-0.357	-0.830022208360236\\
-0.356	-0.829286235600034\\
-0.355	-0.828546435532831\\
-0.354	-0.827802785125757\\
-0.353	-0.827055261207327\\
-0.352	-0.82630384046661\\
-0.351	-0.825548499452389\\
-0.35	-0.824789214572316\\
-0.349	-0.824025962092066\\
-0.348	-0.823258718134476\\
-0.347	-0.822487458678693\\
-0.346	-0.821712159559303\\
-0.345	-0.820932796465465\\
-0.344	-0.820149344940035\\
-0.343	-0.819361780378684\\
-0.342	-0.818570078029012\\
-0.341	-0.817774212989662\\
-0.34	-0.816974160209417\\
-0.339	-0.8161698944863\\
-0.338	-0.815361390466669\\
-0.337	-0.814548622644304\\
-0.336	-0.813731565359487\\
-0.335	-0.812910192798075\\
-0.334	-0.812084478990582\\
-0.333	-0.811254397811229\\
-0.332	-0.810419922977017\\
-0.331	-0.809581028046772\\
-0.33	-0.808737686420196\\
-0.329	-0.807889871336912\\
-0.328	-0.807037555875494\\
-0.327	-0.806180712952505\\
-0.326	-0.805319315321519\\
-0.325	-0.804453335572138\\
-0.324	-0.803582746129008\\
-0.323	-0.802707519250827\\
-0.322	-0.801827627029343\\
-0.321	-0.800943041388351\\
-0.32	-0.800053734082685\\
-0.319	-0.799159676697195\\
-0.318	-0.798260840645731\\
-0.317	-0.797357197170107\\
-0.316	-0.796448717339073\\
-0.315	-0.795535372047266\\
-0.314	-0.794617132014169\\
-0.313	-0.793693967783052\\
-0.312	-0.792765849719914\\
-0.311	-0.791832748012414\\
-0.31	-0.790894632668802\\
-0.309	-0.789951473516835\\
-0.308	-0.789003240202692\\
-0.307	-0.788049902189882\\
-0.306	-0.787091428758145\\
-0.305	-0.786127789002344\\
-0.304	-0.785158951831355\\
-0.303	-0.784184885966946\\
-0.302	-0.783205559942652\\
-0.301	-0.78222094210264\\
-0.3	-0.781231000600572\\
-0.299	-0.780235703398459\\
-0.298	-0.779235018265505\\
-0.297	-0.778228912776947\\
-0.296	-0.77721735431289\\
-0.295	-0.776200310057132\\
-0.294	-0.775177746995978\\
-0.293	-0.774149631917061\\
-0.292	-0.773115931408137\\
-0.291	-0.772076611855886\\
-0.29	-0.771031639444703\\
-0.289	-0.76998098015548\\
-0.288	-0.768924599764379\\
-0.287	-0.767862463841603\\
-0.286	-0.766794537750154\\
-0.285	-0.76572078664459\\
-0.284	-0.764641175469765\\
-0.283	-0.763555668959571\\
-0.282	-0.762464231635667\\
-0.281	-0.761366827806202\\
-0.28	-0.760263421564531\\
-0.279	-0.759153976787921\\
-0.278	-0.758038457136251\\
-0.277	-0.756916826050703\\
-0.276	-0.755789046752448\\
-0.275	-0.754655082241321\\
-0.274	-0.753514895294488\\
-0.273	-0.752368448465105\\
-0.272	-0.751215704080975\\
-0.271	-0.750056624243185\\
-0.27	-0.748891170824747\\
-0.269	-0.747719305469223\\
-0.268	-0.746540989589344\\
-0.267	-0.745356184365623\\
-0.266	-0.744164850744954\\
-0.265	-0.742966949439208\\
-0.264	-0.741762440923821\\
-0.263	-0.740551285436366\\
-0.262	-0.739333442975125\\
-0.261	-0.738108873297651\\
-0.26	-0.736877535919312\\
-0.259	-0.735639390111841\\
-0.258	-0.734394394901865\\
-0.257	-0.733142509069435\\
-0.256	-0.731883691146534\\
-0.255	-0.730617899415594\\
-0.254	-0.729345091907985\\
-0.253	-0.728065226402511\\
-0.252	-0.726778260423886\\
-0.251	-0.725484151241206\\
-0.25	-0.724182855866413\\
-0.249	-0.722874331052743\\
-0.248	-0.721558533293172\\
-0.247	-0.720235418818849\\
-0.246	-0.718904943597522\\
-0.245	-0.717567063331949\\
-0.244	-0.716221733458307\\
-0.243	-0.714868909144586\\
-0.242	-0.713508545288976\\
-0.241	-0.712140596518242\\
-0.24	-0.71076501718609\\
-0.239	-0.709381761371528\\
-0.238	-0.707990782877208\\
-0.237	-0.706592035227763\\
-0.236	-0.705185471668139\\
-0.235	-0.703771045161907\\
-0.234	-0.702348708389569\\
-0.233	-0.70091841374686\\
-0.232	-0.699480113343028\\
-0.231	-0.698033758999115\\
-0.23	-0.696579302246219\\
-0.229	-0.69511669432375\\
-0.228	-0.693645886177677\\
-0.227	-0.692166828458756\\
-0.226	-0.690679471520763\\
-0.225	-0.689183765418698\\
-0.224	-0.687679659906989\\
-0.223	-0.686167104437686\\
-0.222	-0.684646048158642\\
-0.221	-0.683116439911675\\
-0.22	-0.681578228230735\\
-0.219	-0.680031361340045\\
-0.218	-0.678475787152242\\
-0.217	-0.676911453266497\\
-0.216	-0.675338306966633\\
-0.215	-0.673756295219223\\
-0.214	-0.672165364671685\\
-0.213	-0.670565461650357\\
-0.212	-0.66895653215857\\
-0.211	-0.667338521874698\\
-0.21	-0.665711376150207\\
-0.209	-0.664075040007686\\
-0.208	-0.662429458138869\\
-0.207	-0.66077457490264\\
-0.206	-0.659110334323036\\
-0.205	-0.657436680087226\\
-0.204	-0.655753555543489\\
-0.203	-0.65406090369917\\
-0.202	-0.65235866721863\\
-0.201	-0.650646788421183\\
-0.2	-0.648925209279018\\
-0.199	-0.647193871415111\\
-0.198	-0.645452716101123\\
-0.197	-0.643701684255285\\
-0.196	-0.641940716440272\\
-0.195	-0.640169752861063\\
-0.194	-0.63838873336279\\
-0.193	-0.636597597428569\\
-0.192	-0.634796284177324\\
-0.191	-0.632984732361593\\
-0.19	-0.631162880365327\\
-0.189	-0.629330666201667\\
-0.188	-0.627488027510717\\
-0.187	-0.625634901557296\\
-0.186	-0.623771225228679\\
-0.185	-0.62189693503233\\
-0.184	-0.620011967093608\\
-0.183	-0.618116257153476\\
-0.182	-0.616209740566181\\
-0.181	-0.614292352296932\\
-0.18	-0.612364026919555\\
-0.179	-0.610424698614137\\
-0.178	-0.608474301164663\\
-0.177	-0.606512767956623\\
-0.176	-0.604540031974621\\
-0.175	-0.60255602579996\\
-0.174	-0.600560681608214\\
-0.173	-0.598553931166787\\
-0.172	-0.596535705832457\\
-0.171	-0.594505936548905\\
-0.17	-0.592464553844229\\
-0.169	-0.590411487828442\\
-0.168	-0.588346668190955\\
-0.167	-0.586270024198051\\
-0.166	-0.584181484690331\\
-0.165	-0.582080978080159\\
-0.164	-0.57996843234908\\
-0.163	-0.577843775045231\\
-0.162	-0.575706933280731\\
-0.161	-0.573557833729055\\
-0.16	-0.571396402622399\\
-0.159	-0.56922256574902\\
-0.158	-0.567036248450567\\
-0.157	-0.564837375619392\\
-0.156	-0.562625871695846\\
-0.155	-0.560401660665559\\
-0.154	-0.558164666056706\\
-0.153	-0.555914810937249\\
-0.152	-0.553652017912173\\
-0.151	-0.551376209120694\\
-0.15	-0.549087306233461\\
-0.149	-0.546785230449734\\
-0.148	-0.544469902494547\\
-0.147	-0.542141242615853\\
-0.146	-0.539799170581658\\
-0.145	-0.537443605677125\\
-0.144	-0.535074466701674\\
-0.143	-0.532691671966057\\
-0.142	-0.530295139289418\\
-0.141	-0.527884785996331\\
-0.14	-0.525460528913828\\
-0.139	-0.523022284368404\\
-0.138	-0.520569968183003\\
-0.137	-0.518103495673987\\
-0.136	-0.515622781648091\\
-0.135	-0.513127740399353\\
-0.134	-0.510618285706029\\
-0.133	-0.50809433082749\\
-0.132	-0.505555788501098\\
-0.131	-0.503002570939067\\
-0.13	-0.500434589825299\\
-0.129	-0.497851756312207\\
-0.128	-0.495253981017518\\
-0.127	-0.49264117402105\\
-0.126	-0.490013244861481\\
-0.125	-0.487370102533087\\
-0.124	-0.48471165548247\\
-0.123	-0.48203781160526\\
-0.122	-0.4793484782428\\
-0.121	-0.476643562178808\\
-0.12	-0.473922969636025\\
-0.119	-0.471186606272837\\
-0.118	-0.468434377179878\\
-0.117	-0.465666186876614\\
-0.116	-0.462881939307905\\
-0.115	-0.46008153784055\\
-0.114	-0.457264885259804\\
-0.113	-0.454431883765881\\
-0.112	-0.451582434970431\\
-0.111	-0.448716439893001\\
-0.11	-0.445833798957471\\
-0.109	-0.442934411988464\\
-0.108	-0.440018178207749\\
-0.107	-0.437084996230605\\
-0.106	-0.434134764062175\\
-0.105	-0.431167379093792\\
-0.104	-0.42818273809929\\
-0.103	-0.425180737231279\\
-0.102	-0.422161272017415\\
-0.101	-0.419124237356636\\
-0.1	-0.416069527515376\\
-0.099	-0.412997036123762\\
-0.098	-0.409906656171783\\
-0.097	-0.406798280005438\\
-0.096	-0.40367179932286\\
-0.095	-0.400527105170417\\
-0.094	-0.39736408793879\\
-0.093	-0.39418263735903\\
-0.092	-0.390982642498582\\
-0.091	-0.387763991757296\\
-0.09	-0.384526572863409\\
-0.089	-0.381270272869502\\
-0.088	-0.377994978148434\\
-0.087	-0.374700574389253\\
-0.086	-0.371386946593078\\
-0.085	-0.368053979068961\\
-0.084	-0.364701555429722\\
-0.083	-0.361329558587759\\
-0.082	-0.357937870750832\\
-0.081	-0.354526373417824\\
-0.08	-0.351094947374473\\
-0.079	-0.347643472689082\\
-0.078	-0.344171828708199\\
-0.077	-0.340679894052279\\
-0.076	-0.337167546611306\\
-0.075	-0.333634663540404\\
-0.074	-0.330081121255412\\
-0.073	-0.326506795428435\\
-0.072	-0.322911560983367\\
-0.071	-0.319295292091392\\
-0.07	-0.315657862166449\\
-0.069	-0.311999143860679\\
-0.0679999999999999	-0.308319009059838\\
-0.0669999999999999	-0.304617328878688\\
-0.0659999999999999	-0.300893973656352\\
-0.0649999999999999	-0.297148812951652\\
-0.0639999999999999	-0.293381715538407\\
-0.0629999999999999	-0.289592549400716\\
-0.0619999999999999	-0.2857811817282\\
-0.0609999999999999	-0.28194747891122\\
-0.0599999999999999	-0.278091306536072\\
-0.0589999999999999	-0.274212529380145\\
-0.0579999999999999	-0.270311011407054\\
-0.0569999999999999	-0.26638661576174\\
-0.0559999999999999	-0.262439204765548\\
-0.0549999999999999	-0.258468639911266\\
-0.0539999999999999	-0.254474781858142\\
-0.0529999999999999	-0.250457490426864\\
-0.0519999999999999	-0.246416624594518\\
-0.0509999999999999	-0.24235204248951\\
-0.0499999999999999	-0.238263601386454\\
-0.0489999999999999	-0.23415115770104\\
-0.0479999999999999	-0.230014566984863\\
-0.0469999999999999	-0.22585368392022\\
-0.0459999999999999	-0.221668362314882\\
-0.0449999999999999	-0.217458455096832\\
-0.0439999999999999	-0.213223814308964\\
-0.0429999999999999	-0.208964291103764\\
-0.0419999999999999	-0.204679735737948\\
-0.0409999999999999	-0.20036999756707\\
-0.04	-0.196034925040102\\
-0.039	-0.191674365693975\\
-0.038	-0.187288166148092\\
-0.037	-0.182876172098807\\
-0.036	-0.178438228313868\\
-0.035	-0.173974178626832\\
-0.034	-0.169483865931437\\
-0.033	-0.164967132175952\\
-0.032	-0.160423818357486\\
-0.031	-0.155853764516261\\
-0.03	-0.151256809729855\\
-0.029	-0.146632792107412\\
-0.028	-0.141981548783807\\
-0.027	-0.13730291591379\\
-0.026	-0.132596728666083\\
-0.025	-0.127862821217448\\
-0.024	-0.123101026746716\\
-0.023	-0.118311177428782\\
-0.022	-0.113493104428564\\
-0.021	-0.108646637894926\\
-0.02	-0.103771606954558\\
-0.019	-0.0988678397058326\\
-0.018	-0.0939351632126087\\
-0.017	-0.0889734034980109\\
-0.016	-0.0839823855381645\\
-0.015	-0.0789619332558947\\
-0.014	-0.0739118695143879\\
-0.013	-0.0688320161108146\\
-0.012	-0.0637221937699149\\
-0.011	-0.058582222137543\\
-0.01	-0.0534119197741763\\
-0.00900000000000001	-0.0482111041483816\\
-0.00800000000000001	-0.0429795916302457\\
-0.00700000000000001	-0.0377171974847639\\
-0.00600000000000001	-0.0324237358651893\\
-0.005	-0.0270990198063427\\
-0.004	-0.0217428612178816\\
-0.003	-0.0163550708775294\\
-0.002	-0.0109354584242625\\
-0.001	-0.00548383235145825\\
0	0\\
0.001	0.00548383235145825\\
0.002	0.0109354584242625\\
0.003	0.0163550708775294\\
0.004	0.0217428612178816\\
0.005	0.0270990198063427\\
0.00600000000000001	0.0324237358651893\\
0.00700000000000001	0.0377171974847639\\
0.00800000000000001	0.0429795916302457\\
0.00900000000000001	0.0482111041483816\\
0.01	0.0534119197741763\\
0.011	0.058582222137543\\
0.012	0.0637221937699149\\
0.013	0.0688320161108146\\
0.014	0.0739118695143879\\
0.015	0.0789619332558947\\
0.016	0.0839823855381645\\
0.017	0.0889734034980109\\
0.018	0.0939351632126087\\
0.019	0.0988678397058326\\
0.02	0.103771606954558\\
0.021	0.108646637894926\\
0.022	0.113493104428564\\
0.023	0.118311177428782\\
0.024	0.123101026746716\\
0.025	0.127862821217448\\
0.026	0.132596728666083\\
0.027	0.13730291591379\\
0.028	0.141981548783807\\
0.029	0.146632792107412\\
0.03	0.151256809729855\\
0.031	0.155853764516261\\
0.032	0.160423818357486\\
0.033	0.164967132175952\\
0.034	0.169483865931437\\
0.035	0.173974178626832\\
0.036	0.178438228313868\\
0.037	0.182876172098807\\
0.038	0.187288166148092\\
0.039	0.191674365693975\\
0.04	0.196034925040102\\
0.0409999999999999	0.20036999756707\\
0.0419999999999999	0.204679735737948\\
0.0429999999999999	0.208964291103764\\
0.0439999999999999	0.213223814308964\\
0.0449999999999999	0.217458455096832\\
0.0459999999999999	0.221668362314882\\
0.0469999999999999	0.22585368392022\\
0.0479999999999999	0.230014566984863\\
0.0489999999999999	0.23415115770104\\
0.0499999999999999	0.238263601386454\\
0.0509999999999999	0.24235204248951\\
0.0519999999999999	0.246416624594518\\
0.0529999999999999	0.250457490426864\\
0.0539999999999999	0.254474781858142\\
0.0549999999999999	0.258468639911266\\
0.0559999999999999	0.262439204765548\\
0.0569999999999999	0.26638661576174\\
0.0579999999999999	0.270311011407054\\
0.0589999999999999	0.274212529380145\\
0.0599999999999999	0.278091306536072\\
0.0609999999999999	0.28194747891122\\
0.0619999999999999	0.2857811817282\\
0.0629999999999999	0.289592549400716\\
0.0639999999999999	0.293381715538407\\
0.0649999999999999	0.297148812951652\\
0.0659999999999999	0.300893973656352\\
0.0669999999999999	0.304617328878688\\
0.0679999999999999	0.308319009059838\\
0.069	0.311999143860679\\
0.07	0.315657862166449\\
0.071	0.319295292091392\\
0.072	0.322911560983367\\
0.073	0.326506795428435\\
0.074	0.330081121255412\\
0.075	0.333634663540404\\
0.076	0.337167546611306\\
0.077	0.340679894052279\\
0.078	0.344171828708199\\
0.079	0.347643472689082\\
0.08	0.351094947374473\\
0.081	0.354526373417824\\
0.082	0.357937870750832\\
0.083	0.361329558587759\\
0.084	0.364701555429722\\
0.085	0.368053979068961\\
0.086	0.371386946593078\\
0.087	0.374700574389253\\
0.088	0.377994978148434\\
0.089	0.381270272869502\\
0.09	0.384526572863409\\
0.091	0.387763991757296\\
0.092	0.390982642498582\\
0.093	0.39418263735903\\
0.094	0.39736408793879\\
0.095	0.400527105170417\\
0.096	0.40367179932286\\
0.097	0.406798280005438\\
0.098	0.409906656171783\\
0.099	0.412997036123762\\
0.1	0.416069527515376\\
0.101	0.419124237356636\\
0.102	0.422161272017415\\
0.103	0.425180737231279\\
0.104	0.42818273809929\\
0.105	0.431167379093792\\
0.106	0.434134764062175\\
0.107	0.437084996230605\\
0.108	0.440018178207749\\
0.109	0.442934411988464\\
0.11	0.445833798957471\\
0.111	0.448716439893001\\
0.112	0.451582434970431\\
0.113	0.454431883765881\\
0.114	0.457264885259804\\
0.115	0.46008153784055\\
0.116	0.462881939307905\\
0.117	0.465666186876614\\
0.118	0.468434377179878\\
0.119	0.471186606272837\\
0.12	0.473922969636025\\
0.121	0.476643562178808\\
0.122	0.4793484782428\\
0.123	0.48203781160526\\
0.124	0.48471165548247\\
0.125	0.487370102533087\\
0.126	0.490013244861481\\
0.127	0.49264117402105\\
0.128	0.495253981017518\\
0.129	0.497851756312207\\
0.13	0.500434589825299\\
0.131	0.503002570939067\\
0.132	0.505555788501098\\
0.133	0.50809433082749\\
0.134	0.510618285706029\\
0.135	0.513127740399353\\
0.136	0.515622781648091\\
0.137	0.518103495673987\\
0.138	0.520569968183003\\
0.139	0.523022284368404\\
0.14	0.525460528913828\\
0.141	0.527884785996331\\
0.142	0.530295139289418\\
0.143	0.532691671966057\\
0.144	0.535074466701674\\
0.145	0.537443605677125\\
0.146	0.539799170581658\\
0.147	0.542141242615853\\
0.148	0.544469902494547\\
0.149	0.546785230449734\\
0.15	0.549087306233461\\
0.151	0.551376209120694\\
0.152	0.553652017912173\\
0.153	0.555914810937249\\
0.154	0.558164666056706\\
0.155	0.560401660665559\\
0.156	0.562625871695846\\
0.157	0.564837375619392\\
0.158	0.567036248450567\\
0.159	0.56922256574902\\
0.16	0.571396402622399\\
0.161	0.573557833729055\\
0.162	0.575706933280731\\
0.163	0.577843775045231\\
0.164	0.57996843234908\\
0.165	0.582080978080159\\
0.166	0.584181484690331\\
0.167	0.586270024198051\\
0.168	0.588346668190955\\
0.169	0.590411487828442\\
0.17	0.592464553844229\\
0.171	0.594505936548905\\
0.172	0.596535705832457\\
0.173	0.598553931166787\\
0.174	0.600560681608214\\
0.175	0.60255602579996\\
0.176	0.604540031974621\\
0.177	0.606512767956623\\
0.178	0.608474301164663\\
0.179	0.610424698614137\\
0.18	0.612364026919555\\
0.181	0.614292352296932\\
0.182	0.616209740566181\\
0.183	0.618116257153476\\
0.184	0.620011967093608\\
0.185	0.62189693503233\\
0.186	0.623771225228679\\
0.187	0.625634901557296\\
0.188	0.627488027510717\\
0.189	0.629330666201667\\
0.19	0.631162880365327\\
0.191	0.632984732361593\\
0.192	0.634796284177324\\
0.193	0.636597597428569\\
0.194	0.63838873336279\\
0.195	0.640169752861063\\
0.196	0.641940716440272\\
0.197	0.643701684255285\\
0.198	0.645452716101123\\
0.199	0.647193871415111\\
0.2	0.648925209279018\\
0.201	0.650646788421183\\
0.202	0.65235866721863\\
0.203	0.65406090369917\\
0.204	0.655753555543489\\
0.205	0.657436680087226\\
0.206	0.659110334323036\\
0.207	0.66077457490264\\
0.208	0.662429458138869\\
0.209	0.664075040007686\\
0.21	0.665711376150207\\
0.211	0.667338521874698\\
0.212	0.66895653215857\\
0.213	0.670565461650357\\
0.214	0.672165364671685\\
0.215	0.673756295219223\\
0.216	0.675338306966633\\
0.217	0.676911453266497\\
0.218	0.678475787152242\\
0.219	0.680031361340045\\
0.22	0.681578228230735\\
0.221	0.683116439911675\\
0.222	0.684646048158642\\
0.223	0.686167104437686\\
0.224	0.687679659906989\\
0.225	0.689183765418698\\
0.226	0.690679471520763\\
0.227	0.692166828458756\\
0.228	0.693645886177677\\
0.229	0.69511669432375\\
0.23	0.696579302246219\\
0.231	0.698033758999115\\
0.232	0.699480113343028\\
0.233	0.70091841374686\\
0.234	0.702348708389569\\
0.235	0.703771045161907\\
0.236	0.705185471668139\\
0.237	0.706592035227763\\
0.238	0.707990782877208\\
0.239	0.709381761371528\\
0.24	0.71076501718609\\
0.241	0.712140596518242\\
0.242	0.713508545288976\\
0.243	0.714868909144586\\
0.244	0.716221733458307\\
0.245	0.717567063331949\\
0.246	0.718904943597522\\
0.247	0.720235418818849\\
0.248	0.721558533293172\\
0.249	0.722874331052743\\
0.25	0.724182855866413\\
0.251	0.725484151241206\\
0.252	0.726778260423886\\
0.253	0.728065226402511\\
0.254	0.729345091907985\\
0.255	0.730617899415594\\
0.256	0.731883691146534\\
0.257	0.733142509069435\\
0.258	0.734394394901865\\
0.259	0.735639390111841\\
0.26	0.736877535919312\\
0.261	0.738108873297651\\
0.262	0.739333442975125\\
0.263	0.740551285436366\\
0.264	0.741762440923821\\
0.265	0.742966949439208\\
0.266	0.744164850744954\\
0.267	0.745356184365623\\
0.268	0.746540989589344\\
0.269	0.747719305469223\\
0.27	0.748891170824747\\
0.271	0.750056624243185\\
0.272	0.751215704080975\\
0.273	0.752368448465105\\
0.274	0.753514895294488\\
0.275	0.754655082241321\\
0.276	0.755789046752448\\
0.277	0.756916826050703\\
0.278	0.758038457136251\\
0.279	0.759153976787921\\
0.28	0.760263421564531\\
0.281	0.761366827806202\\
0.282	0.762464231635667\\
0.283	0.763555668959571\\
0.284	0.764641175469765\\
0.285	0.76572078664459\\
0.286	0.766794537750154\\
0.287	0.767862463841603\\
0.288	0.768924599764379\\
0.289	0.76998098015548\\
0.29	0.771031639444703\\
0.291	0.772076611855886\\
0.292	0.773115931408137\\
0.293	0.774149631917061\\
0.294	0.775177746995978\\
0.295	0.776200310057132\\
0.296	0.77721735431289\\
0.297	0.778228912776947\\
0.298	0.779235018265505\\
0.299	0.780235703398459\\
0.3	0.781231000600572\\
0.301	0.78222094210264\\
0.302	0.783205559942652\\
0.303	0.784184885966946\\
0.304	0.785158951831355\\
0.305	0.786127789002344\\
0.306	0.787091428758145\\
0.307	0.788049902189882\\
0.308	0.789003240202692\\
0.309	0.789951473516835\\
0.31	0.790894632668802\\
0.311	0.791832748012414\\
0.312	0.792765849719914\\
0.313	0.793693967783052\\
0.314	0.794617132014169\\
0.315	0.795535372047266\\
0.316	0.796448717339073\\
0.317	0.797357197170107\\
0.318	0.798260840645731\\
0.319	0.799159676697195\\
0.32	0.800053734082685\\
0.321	0.800943041388351\\
0.322	0.801827627029343\\
0.323	0.802707519250827\\
0.324	0.803582746129008\\
0.325	0.804453335572138\\
0.326	0.805319315321519\\
0.327	0.806180712952505\\
0.328	0.807037555875494\\
0.329	0.807889871336912\\
0.33	0.808737686420196\\
0.331	0.809581028046772\\
0.332	0.810419922977017\\
0.333	0.811254397811229\\
0.334	0.812084478990582\\
0.335	0.812910192798075\\
0.336	0.813731565359487\\
0.337	0.814548622644304\\
0.338	0.815361390466669\\
0.339	0.8161698944863\\
0.34	0.816974160209417\\
0.341	0.817774212989662\\
0.342	0.818570078029012\\
0.343	0.819361780378684\\
0.344	0.820149344940035\\
0.345	0.820932796465465\\
0.346	0.821712159559303\\
0.347	0.822487458678693\\
0.348	0.823258718134476\\
0.349	0.824025962092066\\
0.35	0.824789214572316\\
0.351	0.825548499452389\\
0.352	0.82630384046661\\
0.353	0.827055261207327\\
0.354	0.827802785125757\\
0.355	0.828546435532831\\
0.356	0.829286235600034\\
0.357	0.830022208360236\\
0.358	0.830754376708525\\
0.359	0.831482763403032\\
0.36	0.832207391065744\\
0.361	0.832928282183324\\
0.362	0.833645459107921\\
0.363	0.834358944057969\\
0.364	0.835068759118994\\
0.365	0.835774926244404\\
0.366	0.83647746725628\\
0.367	0.837176403846164\\
0.368	0.837871757575838\\
0.369	0.8385635498781\\
0.37	0.839251802057538\\
0.371	0.839936535291293\\
0.372	0.840617770629824\\
0.373	0.841295528997668\\
0.374	0.841969831194187\\
0.375	0.842640697894322\\
0.376	0.843308149649337\\
0.377	0.843972206887555\\
0.378	0.844632889915099\\
0.379	0.845290218916618\\
0.38	0.845944213956016\\
0.381	0.846594894977175\\
0.382	0.84724228180467\\
0.383	0.847886394144486\\
0.384	0.848527251584727\\
0.385	0.849164873596319\\
0.386	0.849799279533712\\
0.387	0.850430488635578\\
0.388	0.851058520025501\\
0.389	0.85168339271267\\
0.39	0.852305125592556\\
0.391	0.852923737447601\\
0.392	0.853539246947887\\
0.393	0.854151672651812\\
0.394	0.854761033006756\\
0.395	0.855367346349747\\
0.396	0.855970630908121\\
0.397	0.856570904800175\\
0.398	0.857168186035825\\
0.399	0.857762492517247\\
0.4	0.858353842039529\\
0.401	0.858942252291306\\
0.402	0.859527740855401\\
0.403	0.860110325209455\\
0.404	0.860690022726555\\
0.405	0.861266850675865\\
0.406	0.861840826223242\\
0.407	0.862411966431856\\
0.408	0.862980288262808\\
0.409	0.863545808575732\\
0.41	0.864108544129411\\
0.411	0.864668511582374\\
0.412	0.8652257274935\\
0.413	0.865780208322609\\
0.414	0.866331970431063\\
0.415	0.866881030082345\\
0.416	0.86742740344265\\
0.417	0.867971106581469\\
0.418	0.868512155472163\\
0.419	0.869050565992538\\
0.42	0.869586353925421\\
0.421	0.870119534959225\\
0.422	0.870650124688513\\
0.423	0.871178138614563\\
0.424	0.871703592145921\\
0.425	0.872226500598962\\
0.426	0.872746879198435\\
0.427	0.873264743078015\\
0.428	0.873780107280846\\
0.429	0.874292986760086\\
0.43	0.874803396379439\\
0.431	0.875311350913697\\
0.432	0.875816865049266\\
0.433	0.8763199533847\\
0.434	0.876820630431222\\
0.435	0.877318910613249\\
0.436	0.87781480826891\\
0.437	0.878308337650564\\
0.438	0.878799512925311\\
0.439	0.879288348175504\\
0.44	0.879774857399253\\
0.441	0.880259054510931\\
0.442	0.880740953341678\\
0.443	0.881220567639891\\
0.444	0.881697911071726\\
0.445	0.882172997221589\\
0.446	0.88264583959262\\
0.447	0.883116451607185\\
0.448	0.883584846607357\\
0.449	0.884051037855395\\
0.45	0.884515038534225\\
0.451	0.88497686174791\\
0.452	0.885436520522124\\
0.453	0.885894027804623\\
0.454	0.886349396465706\\
0.455	0.886802639298685\\
0.456	0.88725376902034\\
0.457	0.887702798271377\\
0.458	0.888149739616889\\
0.459	0.888594605546801\\
0.46	0.889037408476323\\
0.461	0.889478160746399\\
0.462	0.889916874624146\\
0.463	0.890353562303302\\
0.464	0.890788235904657\\
0.465	0.8912209074765\\
0.466	0.891651588995041\\
0.467	0.892080292364852\\
0.468	0.892507029419289\\
0.469	0.892931811920922\\
0.47	0.893354651561955\\
0.471	0.89377555996465\\
0.472	0.894194548681745\\
0.473	0.894611629196869\\
0.474	0.895026812924952\\
0.475	0.895440111212645\\
0.476	0.895851535338719\\
0.477	0.896261096514477\\
0.478	0.896668805884153\\
0.479	0.89707467452532\\
0.48	0.89747871344928\\
0.481	0.897880933601467\\
0.482	0.898281345861841\\
0.483	0.898679961045275\\
0.484	0.899076789901947\\
0.485	0.899471843117729\\
0.486	0.899865131314568\\
0.487	0.900256665050871\\
0.488	0.900646454821884\\
0.489	0.90103451106007\\
0.49	0.901420844135485\\
0.491	0.90180546435615\\
0.492	0.902188381968424\\
0.493	0.902569607157371\\
0.494	0.90294915004713\\
0.495	0.903327020701273\\
0.496	0.903703229123175\\
0.497	0.904077785256368\\
0.498	0.904450698984904\\
0.499	0.904821980133705\\
0.5	0.905191638468922\\
0.501	0.905559683698286\\
0.502	0.905926125471452\\
0.503	0.906290973380353\\
0.504	0.906654236959543\\
0.505	0.907015925686539\\
0.506	0.907376048982164\\
0.507	0.907734616210883\\
0.508	0.908091636681146\\
0.509	0.908447119645717\\
0.51	0.90880107430201\\
0.511	0.909153509792421\\
0.512	0.909504435204654\\
0.513	0.909853859572051\\
0.514	0.910201791873917\\
0.515	0.910548241035841\\
0.516	0.910893215930019\\
0.517	0.911236725375574\\
0.518	0.911578778138873\\
0.519	0.911919382933841\\
0.52	0.912258548422276\\
0.521	0.912596283214163\\
0.522	0.912932595867979\\
0.523	0.913267494891006\\
0.524	0.913600988739633\\
0.525	0.913933085819664\\
0.526	0.914263794486619\\
0.527	0.914593123046036\\
0.528	0.914921079753767\\
0.529	0.915247672816278\\
0.53	0.915572910390944\\
0.531	0.915896800586342\\
0.532	0.916219351462543\\
0.533	0.916540571031403\\
0.534	0.91686046725685\\
0.535	0.917179048055172\\
0.536	0.917496321295301\\
0.537	0.917812294799098\\
0.538	0.918126976341631\\
0.539	0.918440373651461\\
0.54	0.918752494410912\\
0.541	0.919063346256355\\
0.542	0.919372936778479\\
0.543	0.919681273522566\\
0.544	0.919988363988761\\
0.545	0.920294215632343\\
0.546	0.920598835863994\\
0.547	0.920902232050063\\
0.548	0.921204411512837\\
0.549	0.921505381530796\\
0.55	0.921805149338884\\
0.551	0.922103722128763\\
0.552	0.922401107049075\\
0.553	0.922697311205699\\
0.554	0.922992341662005\\
0.555	0.92328620543911\\
0.556	0.923578909516133\\
0.557	0.923870460830439\\
0.558	0.924160866277898\\
0.559	0.924450132713126\\
0.56	0.924738266949735\\
0.561	0.925025275760578\\
0.562	0.925311165877993\\
0.563	0.925595943994042\\
0.564	0.92587961676076\\
0.565	0.926162190790385\\
0.566	0.926443672655602\\
0.567	0.926724068889777\\
0.568	0.927003385987195\\
0.569	0.927281630403291\\
0.57	0.927558808554882\\
0.571	0.927834926820402\\
0.572	0.928109991540127\\
0.573	0.928384009016405\\
0.574	0.928656985513885\\
0.575	0.928928927259739\\
0.576	0.929199840443889\\
0.577	0.929469731219227\\
0.578	0.929738605701841\\
0.579	0.930006469971229\\
0.58	0.930273330070524\\
0.581	0.930539192006707\\
0.582	0.930804061750825\\
0.583	0.931067945238208\\
0.584	0.931330848368678\\
0.585	0.931592777006766\\
0.586	0.931853736981921\\
0.587	0.932113734088718\\
0.588	0.932372774087071\\
0.589	0.932630862702437\\
0.59	0.932888005626023\\
0.591	0.933144208514991\\
0.592	0.933399476992663\\
0.593	0.93365381664872\\
0.594	0.933907233039407\\
0.595	0.93415973168773\\
0.596	0.934411318083657\\
0.597	0.934661997684314\\
0.598	0.934911775914185\\
0.599	0.935160658165299\\
0.6	0.935408649797437\\
0.601	0.935655756138311\\
0.602	0.935901982483766\\
0.603	0.936147334097966\\
0.604	0.936391816213585\\
0.605	0.936635434031995\\
0.606	0.936878192723453\\
0.607	0.937120097427287\\
0.608	0.937361153252082\\
0.609	0.937601365275862\\
0.61	0.937840738546276\\
0.611	0.938079278080777\\
0.612	0.938316988866802\\
0.613	0.938553875861955\\
0.614	0.938789943994182\\
0.615	0.939025198161952\\
0.616	0.939259643234427\\
0.617	0.939493284051647\\
0.618	0.939726125424694\\
0.619	0.939958172135872\\
0.62	0.940189428938878\\
0.621	0.94041990055897\\
0.622	0.94064959169314\\
0.623	0.940878507010284\\
0.624	0.941106651151367\\
0.625	0.941334028729592\\
0.626	0.941560644330565\\
0.627	0.941786502512461\\
0.628	0.94201160780619\\
0.629	0.942235964715554\\
0.63	0.942459577717416\\
0.631	0.942682451261858\\
0.632	0.94290458977234\\
0.633	0.943125997645862\\
0.634	0.943346679253121\\
0.635	0.943566638938666\\
0.636	0.943785881021061\\
0.637	0.944004409793033\\
0.638	0.94422222952163\\
0.639	0.944439344448375\\
0.64	0.944655758789419\\
0.641	0.94487147673569\\
0.642	0.945086502453047\\
0.643	0.945300840082428\\
0.644	0.94551449374\\
0.645	0.945727467517307\\
0.646	0.945939765481417\\
0.647	0.946151391675069\\
0.648	0.946362350116818\\
0.649	0.94657264480118\\
0.65	0.946782279698776\\
0.651	0.946991258756473\\
0.652	0.94719958589753\\
0.653	0.947407265021735\\
0.654	0.947614300005548\\
0.655	0.947820694702241\\
0.656	0.948026452942034\\
0.657	0.948231578532234\\
0.658	0.948436075257376\\
0.659	0.948639946879351\\
0.66	0.948843197137549\\
0.661	0.949045829748992\\
0.662	0.949247848408464\\
0.663	0.949449256788648\\
0.664	0.949650058540256\\
0.665	0.949850257292163\\
0.666	0.950049856651534\\
0.667	0.950248860203957\\
0.668	0.95044727151357\\
0.669	0.950645094123192\\
0.67	0.950842331554447\\
0.671	0.951038987307893\\
0.672	0.951235064863148\\
0.673	0.951430567679013\\
0.674	0.951625499193601\\
0.675	0.951819862824456\\
0.676	0.952013661968679\\
0.677	0.95220690000305\\
0.678	0.952399580284148\\
0.679	0.952591706148473\\
0.68	0.952783280912565\\
0.681	0.952974307873127\\
0.682	0.95316479030714\\
0.683	0.953354731471979\\
0.684	0.953544134605537\\
0.685	0.953733002926337\\
0.686	0.953921339633647\\
0.687	0.954109147907598\\
0.688	0.954296430909299\\
0.689	0.954483191780946\\
0.69	0.954669433645939\\
0.691	0.954855159608997\\
0.692	0.955040372756262\\
0.693	0.955225076155417\\
0.694	0.955409272855793\\
0.695	0.955592965888481\\
0.696	0.95577615826644\\
0.697	0.955958852984603\\
0.698	0.956141053019989\\
0.699	0.956322761331809\\
0.7	0.95650398086157\\
0.701	0.956684714533185\\
0.702	0.956864965253074\\
0.703	0.957044735910272\\
0.704	0.957224029376532\\
0.705	0.957402848506425\\
0.706	0.957581196137451\\
0.707	0.957759075090132\\
0.708	0.957936488168118\\
0.709	0.958113438158289\\
0.71	0.958289927830852\\
0.711	0.958465959939445\\
0.712	0.958641537221232\\
0.713	0.958816662397002\\
0.714	0.95899133817127\\
0.715	0.959165567232372\\
0.716	0.959339352252562\\
0.717	0.959512695888109\\
0.718	0.959685600779392\\
0.719	0.959858069550995\\
0.72	0.960030104811801\\
0.721	0.960201709155087\\
0.722	0.960372885158618\\
0.723	0.960543635384737\\
0.724	0.960713962380461\\
0.725	0.960883868677568\\
0.726	0.961053356792693\\
0.727	0.961222429227416\\
0.728	0.961391088468351\\
0.729	0.961559336987238\\
0.73	0.961727177241033\\
0.731	0.961894611671991\\
0.732	0.962061642707759\\
0.733	0.962228272761461\\
0.734	0.962394504231787\\
0.735	0.962560339503077\\
0.736	0.962725780945407\\
0.737	0.962890830914677\\
0.738	0.963055491752692\\
0.739	0.963219765787251\\
0.74	0.963383655332224\\
0.741	0.963547162687642\\
0.742	0.963710290139778\\
0.743	0.963873039961225\\
0.744	0.964035414410985\\
0.745	0.964197415734545\\
0.746	0.964359046163958\\
0.747	0.964520307917927\\
0.748	0.964681203201882\\
0.749	0.96484173420806\\
0.75	0.965001903115582\\
0.751	0.965161712090537\\
0.752	0.965321163286053\\
0.753	0.965480258842382\\
0.754	0.965639000886968\\
0.755	0.965797391534531\\
0.756	0.965955432887142\\
0.757	0.966113127034296\\
0.758	0.966270476052987\\
0.759	0.966427482007787\\
0.76	0.966584146950916\\
0.761	0.966740472922318\\
0.762	0.966896461949735\\
0.763	0.967052116048777\\
0.764	0.967207437222999\\
0.765	0.967362427463967\\
0.766	0.967517088751338\\
0.767	0.967671423052922\\
0.768	0.967825432324761\\
0.769	0.967979118511195\\
0.77	0.968132483544931\\
0.771	0.968285529347118\\
0.772	0.968438257827409\\
0.773	0.968590670884036\\
0.774	0.968742770403874\\
0.775	0.968894558262511\\
0.776	0.969046036324316\\
0.777	0.969197206442506\\
0.778	0.969348070459208\\
0.779	0.969498630205535\\
0.78	0.969648887501642\\
0.781	0.969798844156797\\
0.782	0.969948501969444\\
0.783	0.970097862727272\\
0.784	0.970246928207272\\
0.785	0.970395700175807\\
0.786	0.970544180388673\\
0.787	0.970692370591163\\
0.788	0.970840272518129\\
0.789	0.970987887894047\\
0.79	0.971135218433074\\
0.791	0.971282265839115\\
0.792	0.971429031805883\\
0.793	0.971575518016955\\
0.794	0.971721726145843\\
0.795	0.971867657856042\\
0.796	0.9720133148011\\
0.797	0.972158698624671\\
0.798	0.972303810960578\\
0.799	0.972448653432871\\
0.8	0.972593227655882\\
0.801	0.972737535234289\\
0.802	0.972881577763169\\
0.803	0.973025356828058\\
0.804	0.973168874005008\\
0.805	0.97331213086064\\
0.806	0.973455128952207\\
0.807	0.973597869827645\\
0.808	0.97374035502563\\
0.809	0.973882586075633\\
0.81	0.974024564497977\\
0.811	0.974166291803889\\
0.812	0.974307769495558\\
0.813	0.974448999066184\\
0.814	0.974589982000037\\
0.815	0.974730719772506\\
0.816	0.974871213850156\\
0.817	0.975011465690778\\
0.818	0.975151476743442\\
0.819	0.975291248448551\\
0.82	0.97543078223789\\
0.821	0.975570079534681\\
0.822	0.975709141753629\\
0.823	0.97584797030098\\
0.824	0.975986566574565\\
0.825	0.976124931963853\\
0.826	0.976263067850004\\
0.827	0.976400975605913\\
0.828	0.976538656596263\\
0.829	0.976676112177575\\
0.83	0.976813343698253\\
0.831	0.976950352498636\\
0.832	0.977087139911045\\
0.833	0.977223707259831\\
0.834	0.977360055861421\\
0.835	0.97749618702437\\
0.836	0.977632102049404\\
0.837	0.977767802229467\\
0.838	0.977903288849769\\
0.839	0.978038563187834\\
0.84	0.97817362651354\\
0.841	0.978308480089171\\
0.842	0.97844312516946\\
0.843	0.978577563001633\\
0.844	0.978711794825456\\
0.845	0.978845821873279\\
0.846	0.978979645370079\\
0.847	0.979113266533504\\
0.848	0.979246686573922\\
0.849	0.979379906694457\\
0.85	0.979512928091036\\
0.851	0.979645751952434\\
0.852	0.979778379460313\\
0.853	0.979910811789268\\
0.854	0.980043050106864\\
0.855	0.980175095573686\\
0.856	0.980306949343374\\
0.857	0.980438612562666\\
0.858	0.980570086371442\\
0.859	0.980701371902763\\
0.86	0.980832470282911\\
0.861	0.980963382631431\\
0.862	0.981094110061171\\
0.863	0.981224653678322\\
0.864	0.981355014582457\\
0.865	0.981485193866572\\
0.866	0.981615192617125\\
0.867	0.981745011914073\\
0.868	0.981874652830916\\
0.869	0.982004116434728\\
0.87	0.982133403786203\\
0.871	0.982262515939689\\
0.872	0.982391453943226\\
0.873	0.982520218838586\\
0.874	0.982648811661308\\
0.875	0.982777233440737\\
0.876	0.982905485200059\\
0.877	0.983033567956342\\
0.878	0.983161482720567\\
0.879	0.983289230497669\\
0.88	0.983416812286571\\
0.881	0.98354422908022\\
0.882	0.983671481865624\\
0.883	0.983798571623886\\
0.884	0.983925499330241\\
0.885	0.984052265954088\\
0.886	0.984178872459029\\
0.887	0.984305319802902\\
0.888	0.984431608937812\\
0.889	0.984557740810173\\
0.89	0.984683716360733\\
0.891	0.984809536524617\\
0.892	0.984935202231352\\
0.893	0.985060714404907\\
0.894	0.985186073963722\\
0.895	0.985311281820745\\
0.896	0.985436338883461\\
0.897	0.985561246053928\\
0.898	0.985686004228805\\
0.899	0.98581061429939\\
0.9	0.985935077151649\\
0.901	0.986059393666245\\
0.902	0.986183564718578\\
0.903	0.986307591178806\\
0.904	0.986431473911886\\
0.905	0.986555213777598\\
0.906	0.98667881163058\\
0.907	0.986802268320357\\
0.908	0.986925584691373\\
0.909	0.987048761583019\\
0.91	0.987171799829666\\
0.911	0.987294700260694\\
0.912	0.98741746370052\\
0.913	0.987540090968631\\
0.914	0.98766258287961\\
0.915	0.98778494024317\\
0.916	0.987907163864177\\
0.917	0.988029254542685\\
0.918	0.988151213073959\\
0.919	0.98827304024851\\
0.92	0.988394736852119\\
0.921	0.988516303665864\\
0.922	0.988637741466155\\
0.923	0.988759051024753\\
0.924	0.988880233108807\\
0.925	0.989001288480872\\
0.926	0.989122217898944\\
0.927	0.989243022116485\\
0.928	0.989363701882449\\
0.929	0.989484257941309\\
0.93	0.989604691033087\\
0.931	0.989725001893375\\
0.932	0.989845191253367\\
0.933	0.989965259839881\\
0.934	0.99008520837539\\
0.935	0.990205037578042\\
0.936	0.990324748161691\\
0.937	0.99044434083592\\
0.938	0.990563816306068\\
0.939	0.990683175273253\\
0.94	0.990802418434401\\
0.941	0.990921546482267\\
0.942	0.991040560105463\\
0.943	0.991159459988481\\
0.944	0.99127824681172\\
0.945	0.991396921251507\\
0.946	0.991515483980124\\
0.947	0.991633935665831\\
0.948	0.991752276972891\\
0.949	0.991870508561593\\
0.95	0.991988631088276\\
0.951	0.992106645205353\\
0.952	0.992224551561334\\
0.953	0.992342350800849\\
0.954	0.992460043564674\\
0.955	0.99257763048975\\
0.956	0.992695112209208\\
0.957	0.992812489352392\\
0.958	0.992929762544881\\
0.959	0.993046932408511\\
0.96	0.9931639995614\\
0.961	0.993280964617967\\
0.962	0.993397828188955\\
0.963	0.993514590881456\\
0.964	0.993631253298928\\
0.965	0.993747816041219\\
0.966	0.993864279704591\\
0.967	0.993980644881736\\
0.968	0.994096912161804\\
0.969	0.994213082130417\\
0.97	0.994329155369697\\
0.971	0.994445132458282\\
0.972	0.994561013971349\\
0.973	0.994676800480633\\
0.974	0.994792492554452\\
0.975	0.994908090757721\\
0.976	0.995023595651977\\
0.977	0.995139007795398\\
0.978	0.995254327742824\\
0.979	0.995369556045773\\
0.98	0.995484693252467\\
0.981	0.995599739907846\\
0.982	0.995714696553592\\
0.983	0.995829563728145\\
0.984	0.995944341966726\\
0.985	0.996059031801352\\
0.986	0.99617363376086\\
0.987	0.996288148370921\\
0.988	0.996402576154062\\
0.989	0.996516917629687\\
0.99	0.996631173314088\\
0.991	0.996745343720473\\
0.992	0.996859429358978\\
0.993	0.996973430736687\\
0.994	0.997087348357651\\
0.995	0.997201182722906\\
0.996	0.997314934330491\\
0.997	0.997428603675465\\
0.998	0.997542191249927\\
0.999	0.997655697543029\\
1	0.997769123041\\
};
\end{axis}

\begin{axis}[%
width=1.703in,
height=1.573in,
at={(2.998in,0.424in)},
scale only axis,
unbounded coords=jump,
xmin=-30,
xmax=0,
xlabel style={font=\color{white!15!black}},
xlabel={X in dB},
ymin=-30,
ymax=1,
ylabel style={font=\color{white!15!black}},
ylabel={Y in dB},
%axis background/.style={fill=white},
title style={font=\bfseries},
title={Log. output over input level},
xmajorgrids,
ymajorgrids
]
\addplot [color=mycolor1, forget plot]
  table[row sep=crcr]{%
29.5424250943932	11.82129214053\\
29.542135559913	11.8210694225557\\
29.5418460157812	11.8208466988705\\
29.5415564619971	11.820623969474\\
29.5412668985601	11.820401234366\\
29.5409773254695	11.8201784935462\\
29.5406877427247	11.8199557470143\\
29.540398150325	11.81973299477\\
29.5401085482699	11.819510236813\\
29.5398189365586	11.819287473143\\
29.5395293151905	11.8190647037597\\
29.5392396841651	11.8188419286629\\
29.5389500434815	11.8186191478522\\
29.5386603931392	11.8183963613273\\
29.5383707331376	11.8181735690879\\
29.538081063476	11.8179507711339\\
29.5377913841537	11.8177279674648\\
29.5375016951701	11.8175051580803\\
29.5372119965246	11.8172823429802\\
29.5369222882165	11.8170595221643\\
29.5366325702452	11.8168366956321\\
29.53634284261	11.8166138633833\\
29.5360531053103	11.8163910254178\\
29.5357633583455	11.8161681817352\\
29.5354736017148	11.8159453323353\\
29.5351838354177	11.8157224772176\\
29.5348940594535	11.8154996163819\\
29.5346042738216	11.815276749828\\
29.5343144785213	11.8150538775556\\
29.5340246735519	11.8148309995643\\
29.5337348589129	11.8146081158538\\
29.5334450346036	11.8143852264239\\
29.5331552006232	11.8141623312743\\
29.5328653569713	11.8139394304046\\
29.5325755036471	11.8137165238146\\
29.5322856406501	11.813493611504\\
29.5319957679795	11.8132706934725\\
29.5317058856346	11.8130477697198\\
29.531415993615	11.8128248402456\\
29.5311260919198	11.8126019050496\\
29.5308361805486	11.8123789641316\\
29.5305462595005	11.8121560174911\\
29.5302563287751	11.811933065128\\
29.5299663883715	11.8117101070419\\
29.5296764382893	11.8114871432326\\
29.5293864785277	11.8112641736997\\
29.5290965090861	11.8110411984429\\
29.5288065299638	11.8108182174621\\
29.5285165411602	11.8105952307568\\
29.5282265426747	11.8103722383267\\
29.5279365345066	11.8101492401717\\
29.5276465166553	11.8099262362913\\
29.52735648912	11.8097032266853\\
29.5270664519003	11.8094802113534\\
29.5267764049953	11.8092571902953\\
29.5264863484046	11.8090341635108\\
29.5261962821273	11.8088111309994\\
29.525906206163	11.808588092761\\
29.5256161205108	11.8083650487952\\
29.5253260251703	11.8081419991017\\
29.5250359201407	11.8079189436803\\
29.5247458054214	11.8076958825306\\
29.5244556810117	11.8074728156523\\
29.524165546911	11.8072497430452\\
29.5238754031187	11.807026664709\\
29.5235852496341	11.8068035806433\\
29.5232950864565	11.806580490848\\
29.5230049135853	11.8063573953225\\
29.5227147310199	11.8061342940668\\
29.5224245387597	11.8059111870805\\
29.5221343368038	11.8056880743632\\
29.5218441251518	11.8054649559148\\
29.521553903803	11.8052418317349\\
29.5212636727566	11.8050187018231\\
29.5209734320122	11.8047955661793\\
29.520683181569	11.8045724248032\\
29.5203929214263	11.8043492776943\\
29.5201026515836	11.8041261248525\\
29.5198123720401	11.8039029662774\\
29.5195220827953	11.8036798019688\\
29.5192317838485	11.8034566319263\\
29.518941475199	11.8032334561497\\
29.5186511568462	11.8030102746386\\
29.5183608287894	11.8027870873928\\
29.518070491028	11.802563894412\\
29.5177801435613	11.8023406956958\\
29.5174897863887	11.802117491244\\
29.5171994195096	11.8018942810563\\
29.5169090429233	11.8016710651324\\
29.5166186566291	11.801447843472\\
29.5163282606264	11.8012246160748\\
29.5160378549145	11.8010013829406\\
29.5157474394928	11.8007781440689\\
29.5154570143607	11.8005548994595\\
29.5151665795175	11.8003316491122\\
29.5148761349625	11.8001083930266\\
29.5145856806951	11.7998851312024\\
29.5142952167147	11.7996618636394\\
29.5140047430206	11.7994385903372\\
29.5137142596121	11.7992153112955\\
29.5134237664886	11.7989920265142\\
29.5131332636495	11.7987687359927\\
29.512842751094	11.7985454397309\\
29.5125522288216	11.7983221377285\\
29.5122616968316	11.7980988299852\\
29.5119711551234	11.7978755165007\\
29.5116806036962	11.7976521972746\\
29.5113900425495	11.7974288723067\\
29.5110994716826	11.7972055415967\\
29.5108088910949	11.7969822051443\\
29.5105183007856	11.7967588629492\\
29.5102277007542	11.7965355150111\\
29.509937091	11.7963121613297\\
29.5096464715223	11.7960888019047\\
29.5093558423205	11.7958654367359\\
29.509065203394	11.7956420658229\\
29.5087745547421	11.7954186891654\\
29.5084838963641	11.7951953067631\\
29.5081932282594	11.7949719186158\\
29.5079025504273	11.7947485247231\\
29.5076118628672	11.7945251250847\\
29.5073211655785	11.7943017197004\\
29.5070304585604	11.7940783085699\\
29.5067397418124	11.7938548916928\\
29.5064490153338	11.7936314690689\\
29.5061582791239	11.7934080406979\\
29.5058675331821	11.7931846065794\\
29.5055767775077	11.7929611667132\\
29.5052860121001	11.792737721099\\
29.5049952369586	11.7925142697365\\
29.5047044520826	11.7922908126253\\
29.5044136574714	11.7920673497653\\
29.5041228531244	11.791843881156\\
29.5038320390409	11.7916204067973\\
29.5035412152202	11.7913969266887\\
29.5032503816618	11.7911734408301\\
29.502959538365	11.7909499492211\\
29.502668685329	11.7907264518614\\
29.5023778225533	11.7905029487507\\
29.5020869500373	11.7902794398888\\
29.5017960677801	11.7900559252753\\
29.5015051757813	11.7898324049099\\
29.5012142740401	11.7896088787923\\
29.500923362556	11.7893853469223\\
29.5006324413281	11.7891618092996\\
29.500341510356	11.7889382659238\\
29.5000505696389	11.7887147167946\\
29.4997596191761	11.7884911619118\\
29.4994686589672	11.7882676012751\\
29.4991776890112	11.7880440348842\\
29.4988867093078	11.7878204627387\\
29.498595719856	11.7875968848383\\
29.4983047206555	11.7873733011829\\
29.4980137117053	11.787149711772\\
29.497722693005	11.7869261166054\\
29.4974316645539	11.7867025156829\\
29.4971406263512	11.786478909004\\
29.4968495783964	11.7862552965685\\
29.4965585206888	11.786031678376\\
29.4962674532278	11.7858080544264\\
29.4959763760126	11.7855844247193\\
29.4956852890427	11.7853607892545\\
29.4953941923174	11.7851371480315\\
29.495103085836	11.7849135010501\\
29.4948119695978	11.7846898483101\\
29.4945208436023	11.7844661898111\\
29.4942297078488	11.7842425255528\\
29.4939385623366	11.7840188555349\\
29.493647407065	11.7837951797572\\
29.4933562420335	11.7835714982193\\
29.4930650672413	11.783347810921\\
29.4927738826878	11.7831241178618\\
29.4924826883723	11.7829004190417\\
29.4921914842942	11.7826767144602\\
29.4919002704529	11.782453004117\\
29.4916090468476	11.7822292880119\\
29.4913178134777	11.7820055661446\\
29.4910265703427	11.7817818385147\\
29.4907353174417	11.781558105122\\
29.4904440547742	11.7813343659661\\
29.4901527823395	11.7811106210469\\
29.489861500137	11.7808868703639\\
29.4895702081659	11.7806631139169\\
29.4892789064257	11.7804393517055\\
29.4889875949157	11.7802155837296\\
29.4886962736352	11.7799918099887\\
29.4884049425835	11.7797680304826\\
29.4881136017601	11.7795442452111\\
29.4878222511643	11.7793204541737\\
29.4875308907953	11.7790966573702\\
29.4872395206526	11.7788728548003\\
29.4869481407355	11.7786490464637\\
29.4866567510433	11.7784252323601\\
29.4863653515755	11.7782014124893\\
29.4860739423312	11.7779775868508\\
29.4857825233099	11.7777537554445\\
29.485491094511	11.77752991827\\
29.4851996559336	11.777306075327\\
29.4849082075773	11.7770822266152\\
29.4846167494414	11.7768583721344\\
29.4843252815251	11.7766345118841\\
29.4840338038279	11.7764106458643\\
29.483742316349	11.7761867740744\\
29.4834508190879	11.7759628965143\\
29.4831593120438	11.7757390131836\\
29.4828677952161	11.7755151240821\\
29.4825762686042	11.7752912292094\\
29.4822847322074	11.7750673285652\\
29.481993186025	11.7748434221494\\
29.4817016300564	11.7746195099614\\
29.4814100643009	11.7743955920011\\
29.4811184887578	11.7741716682682\\
29.4808269034266	11.7739477387623\\
29.4805353083066	11.7737238034832\\
29.480243703397	11.7734998624306\\
29.4799520886973	11.7732759156041\\
29.4796604642067	11.7730519630035\\
29.4793688299247	11.7728280046284\\
29.4790771858505	11.7726040404786\\
29.4787855319836	11.7723800705539\\
29.4784938683231	11.7721560948537\\
29.4782021948686	11.771932113378\\
29.4779105116193	11.7717081261263\\
29.4776188185746	11.7714841330984\\
29.4773271157338	11.7712601342941\\
29.4770354030963	11.7710361297129\\
29.4767436806614	11.7708121193545\\
29.4764519484284	11.7705881032188\\
29.4761602063967	11.7703640813054\\
29.4758684545656	11.7701400536139\\
29.4755766929345	11.7699160201442\\
29.4752849215027	11.7696919808959\\
29.4749931402696	11.7694679358686\\
29.4747013492344	11.7692438850622\\
29.4744095483966	11.7690198284762\\
29.4741177377554	11.7687957661105\\
29.4738259173103	11.7685716979647\\
29.4735340870606	11.7683476240385\\
29.4732422470055	11.7681235443316\\
29.4729503971445	11.7678994588437\\
29.4726585374768	11.7676753675746\\
29.4723666680019	11.7674512705238\\
29.472074788719	11.7672271676912\\
29.4717828996276	11.7670030590764\\
29.4714910007269	11.7667789446791\\
29.4711990920163	11.766554824499\\
29.4709071734951	11.7663306985359\\
29.4706152451626	11.7661065667894\\
29.4703233070183	11.7658824292592\\
29.4700313590615	11.765658285945\\
29.4697394012914	11.7654341368466\\
29.4694474337074	11.7652099819636\\
29.4691554563089	11.7649858212957\\
29.4688634690952	11.7647616548427\\
29.4685714720657	11.7645374826042\\
29.4682794652197	11.76431330458\\
29.4679874485565	11.7640891207697\\
29.4676954220754	11.763864931173\\
29.4674033857759	11.7636407357897\\
29.4671113396572	11.7634165346194\\
29.4668192837187	11.7631923276618\\
29.4665272179597	11.7629681149167\\
29.4662351423796	11.7627438963838\\
29.4659430569777	11.7625196720627\\
29.4656509617534	11.7622954419531\\
29.4653588567059	11.7620712060549\\
29.4650667418347	11.7618469643675\\
29.464774617139	11.7616227168908\\
29.4644824826182	11.7613984636245\\
29.4641903382717	11.7611742045682\\
29.4638981840988	11.7609499397217\\
29.4636060200987	11.7607256690846\\
29.463313846271	11.7605013926567\\
29.4630216626148	11.7602771104376\\
29.4627294691295	11.7600528224271\\
29.4624372658146	11.7598285286249\\
29.4621450526692	11.7596042290306\\
29.4618528296928	11.759379923644\\
29.4615605968847	11.7591556124647\\
29.4612683542443	11.7589312954925\\
29.4609761017708	11.7587069727271\\
29.4606838394635	11.7584826441682\\
29.460391567322	11.7582583098154\\
29.4600992853454	11.7580339696684\\
29.4598069935332	11.7578096237271\\
29.4595146918846	11.757585271991\\
29.459222380399	11.7573609144599\\
29.4589300590757	11.7571365511334\\
29.4586377279141	11.7569121820113\\
29.4583453869135	11.7566878070933\\
29.4580530360733	11.7564634263791\\
29.4577606753927	11.7562390398683\\
29.4574683048712	11.7560146475607\\
29.457175924508	11.755790249456\\
29.4568835343026	11.7555658455539\\
29.4565911342541	11.755341435854\\
29.456298724362	11.7551170203561\\
29.4560063046257	11.7548925990599\\
29.4557138750444	11.754668171965\\
29.4554214356175	11.7544437390712\\
29.4551289863442	11.7542193003782\\
29.4548365272241	11.7539948558857\\
29.4545440582563	11.7537704055933\\
29.4542515794403	11.7535459495009\\
29.4539590907754	11.753321487608\\
29.4536665922608	11.7530970199144\\
29.453374083896	11.7528725464197\\
29.4530815656803	11.7526480671237\\
29.4527890376129	11.7524235820261\\
29.4524964996934	11.7521990911266\\
29.4522039519209	11.7519745944249\\
29.4519113942948	11.7517500919206\\
29.4516188268145	11.7515255836136\\
29.4513262494793	11.7513010695034\\
29.4510336622886	11.7510765495897\\
29.4507410652415	11.7508520238724\\
29.4504484583377	11.750627492351\\
29.4501558415762	11.7504029550253\\
29.4498632149565	11.750178411895\\
29.4495705784779	11.7499538629597\\
29.4492779321398	11.7497293082193\\
29.4489852759414	11.7495047476733\\
29.4486926098822	11.7492801813215\\
29.4483999339614	11.7490556091635\\
29.4481072481784	11.7488310311992\\
29.4478145525326	11.7486064474281\\
29.4475218470232	11.74838185785\\
29.4472291316496	11.7481572624646\\
29.4469364064111	11.7479326612715\\
29.4466436713071	11.7477080542705\\
29.4463509263368	11.7474834414613\\
29.4460581714998	11.7472588228436\\
29.4457654067951	11.747034198417\\
29.4454726322223	11.7468095681813\\
29.4451798477807	11.7465849321362\\
29.4448870534695	11.7463602902814\\
29.4445942492881	11.7461356426165\\
29.4443014352358	11.7459109891413\\
29.444008611312	11.7456863298554\\
29.4437157775161	11.7454616647587\\
29.4434229338473	11.7452369938507\\
29.4431300803049	11.7450123171312\\
29.4428372168884	11.7447876345998\\
29.442544343597	11.7445629462563\\
29.4422514604301	11.7443382521004\\
29.4419585673871	11.7441135521317\\
29.4416656644671	11.74388884635\\
29.4413727516697	11.743664134755\\
29.4410798289941	11.7434394173463\\
29.4407868964396	11.7432146941237\\
29.4404939540056	11.7429899650869\\
29.4402010016915	11.7427652302355\\
29.4399080394964	11.7425404895693\\
29.4396150674199	11.7423157430879\\
29.4393220854612	11.7420909907911\\
29.4390290936196	11.7418662326785\\
29.4387360918946	11.74164146875\\
29.4384430802853	11.741416699005\\
29.4381500587912	11.7411919234434\\
29.4378570274116	11.7409671420649\\
29.4375639861458	11.7407423548691\\
29.4372709349932	11.7405175618558\\
29.436977873953	11.7402927630246\\
29.4366848030247	11.7400679583753\\
29.4363917222075	11.7398431479075\\
29.4360986315007	11.7396183316209\\
29.4358055309038	11.7393935095154\\
29.4355124204161	11.7391686815904\\
29.4352193000368	11.7389438478458\\
29.4349261697653	11.7387190082812\\
29.434633029601	11.7384941628964\\
29.4343398795432	11.738269311691\\
29.4340467195911	11.7380444546648\\
29.4337535497442	11.7378195918173\\
29.4334603700018	11.7375947231485\\
29.4331671803632	11.7373698486578\\
29.4328739808277	11.7371449683451\\
29.4325807713947	11.73692008221\\
29.4322875520635	11.7366951902523\\
29.4319943228334	11.7364702924715\\
29.4317010837038	11.7362453888675\\
29.431407834674	11.7360204794399\\
29.4311145757433	11.7357955641885\\
29.430821306911	11.7355706431128\\
29.4305280281766	11.7353457162127\\
29.4302347395392	11.7351207834878\\
29.4299414409984	11.7348958449378\\
29.4296481325533	11.7346709005624\\
29.4293548142033	11.7344459503614\\
29.4290614859477	11.7342209943343\\
29.428768147786	11.733996032481\\
29.4284747997173	11.733771064801\\
29.4281814417411	11.7335460912942\\
29.4278880738567	11.7333211119602\\
29.4275946960633	11.7330961267987\\
29.4273013083604	11.7328711358093\\
29.4270079107472	11.7326461389919\\
29.4267145032231	11.7324211363461\\
29.4264210857875	11.7321961278715\\
29.4261276584396	11.731971113568\\
29.4258342211788	11.7317460934351\\
29.4255407740044	11.7315210674726\\
29.4252473169157	11.7312960356802\\
29.4249538499121	11.7310709980576\\
29.4246603729929	11.7308459546045\\
29.4243668861574	11.7306209053206\\
29.4240733894051	11.7303958502055\\
29.423779882735	11.730170789259\\
29.4234863661468	11.7299457224808\\
29.4231928396395	11.7297206498705\\
29.4228993032127	11.7294955714279\\
29.4226057568655	11.7292704871527\\
29.4223122005974	11.7290453970446\\
29.4220186344077	11.7288203011032\\
29.4217250582956	11.7285951993282\\
29.4214314722606	11.7283700917194\\
29.421137876302	11.7281449782765\\
29.420844270419	11.7279198589991\\
29.420550654611	11.727694733887\\
29.4202570288774	11.7274696029398\\
29.4199633932175	11.7272444661573\\
29.4196697476305	11.7270193235391\\
29.4193760921159	11.726794175085\\
29.419082426673	11.7265690207945\\
29.418788751301	11.7263438606676\\
29.4184950659994	11.7261186947037\\
29.4182013707674	11.7258935229027\\
29.4179076656043	11.7256683452642\\
29.4176139505096	11.7254431617879\\
29.4173202254825	11.7252179724736\\
29.4170264905224	11.7249927773208\\
29.4167327456285	11.7247675763294\\
29.4164389908003	11.724542369499\\
29.416145226037	11.7243171568293\\
29.4158514513379	11.72409193832\\
29.4155576667025	11.7238667139708\\
29.41526387213	11.7236414837814\\
29.4149700676197	11.7234162477516\\
29.4146762531711	11.7231910058809\\
29.4143824287833	11.7229657581691\\
29.4140885944558	11.7227405046159\\
29.4137947501878	11.7225152452209\\
29.4135008959787	11.722289979984\\
29.4132070318279	11.7220647089047\\
29.4129131577346	11.7218394319829\\
29.4126192736981	11.721614149218\\
29.4123253797179	11.72138886061\\
29.4120314757932	11.7211635661584\\
29.4117375619233	11.720938265863\\
29.4114436381076	11.7207129597235\\
29.4111497043455	11.7204876477395\\
29.4108557606362	11.7202623299108\\
29.410561806979	11.720037006237\\
29.4102678433733	11.7198116767179\\
29.4099738698184	11.7195863413531\\
29.4096798863137	11.7193610001424\\
29.4093858928584	11.7191356530854\\
29.409091889452	11.7189103001818\\
29.4087978760936	11.7186849414314\\
29.4085038527827	11.7184595768338\\
29.4082098195186	11.7182342063887\\
29.4079157763006	11.7180088300958\\
29.407621723128	11.7177834479549\\
29.4073276600001	11.7175580599656\\
29.4070335869164	11.7173326661276\\
29.406739503876	11.7171072664405\\
29.4064454108783	11.7168818609042\\
29.4061513079228	11.7166564495183\\
29.4058571950086	11.7164310322825\\
29.405563072135	11.7162056091965\\
29.4052689393016	11.71598018026\\
29.4049747965074	11.7157547454727\\
29.404680643752	11.7155293048342\\
29.4043864810346	11.7153038583444\\
29.4040923083545	11.7150784060028\\
29.403798125711	11.7148529478091\\
29.4035039331036	11.7146274837632\\
29.4032097305314	11.7144020138646\\
29.4029155179939	11.714176538113\\
29.4026212954904	11.7139510565083\\
29.4023270630201	11.7137255690499\\
29.4020328205824	11.7135000757378\\
29.4017385681767	11.7132745765714\\
29.4014443058022	11.7130490715506\\
29.4011500334583	11.7128235606751\\
29.4008557511443	11.7125980439444\\
29.4005614588595	11.7123725213584\\
29.4002671566033	11.7121469929168\\
29.3999728443749	11.7119214586191\\
29.3996785221738	11.7116959184652\\
29.3993841899992	11.7114703724546\\
29.3990898478504	11.7112448205872\\
29.3987954957268	11.7110192628626\\
29.3985011336278	11.7107936992805\\
29.3982067615525	11.7105681298405\\
29.3979123795004	11.7103425545425\\
29.3976179874707	11.710116973386\\
29.3973235854628	11.7098913863709\\
29.3970291734761	11.7096657934967\\
29.3967347515098	11.7094401947632\\
29.3964403195633	11.70921459017\\
29.3961458776358	11.7089889797169\\
29.3958514257268	11.7087633634036\\
29.3955569638355	11.7085377412297\\
29.3952624919612	11.708312113195\\
29.3949680101033	11.7080864792991\\
29.3946735182612	11.7078608395418\\
29.394379016434	11.7076351939227\\
29.3940845046212	11.7074095424416\\
29.3937899828221	11.707183885098\\
29.393495451036	11.7069582218918\\
29.3932009092621	11.7067325528226\\
29.3929063575	11.7065068778902\\
29.3926117957488	11.7062811970941\\
29.3923172240078	11.7060555104341\\
29.3920226422765	11.70582981791\\
29.3917280505541	11.7056041195213\\
29.39143344884	11.7053784152678\\
29.3911388371334	11.7051527051492\\
29.3908442154337	11.7049269891652\\
29.3905495837403	11.7047012673155\\
29.3902549420524	11.7044755395997\\
29.3899602903693	11.7042498060177\\
29.3896656286904	11.7040240665689\\
29.3893709570151	11.7037983212532\\
29.3890762753425	11.7035725700703\\
29.3887815836721	11.7033468130199\\
29.3884868820032	11.7031210501015\\
29.3881921703351	11.702895281315\\
29.3878974486671	11.70266950666\\
29.3876027169985	11.7024437261363\\
29.3873079753287	11.7022179397435\\
29.3870132236569	11.7019921474813\\
29.3867184619826	11.7017663493494\\
29.3864236903049	11.7015405453475\\
29.3861289086234	11.7013147354753\\
29.3858341169371	11.7010889197325\\
29.3855393152456	11.7008630981188\\
29.385244503548	11.7006372706339\\
29.3849496818438	11.7004114372775\\
29.3846548501322	11.7001855980492\\
29.3843600084126	11.6999597529488\\
29.3840651566843	11.699733901976\\
29.3837702949466	11.6995080451304\\
29.3834754231988	11.6992821824118\\
29.3831805414402	11.6990563138198\\
29.3828856496703	11.6988304393542\\
29.3825907478882	11.6986045590146\\
29.3822958360933	11.6983786728008\\
29.3820009142849	11.6981527807123\\
29.3817059824624	11.697926882749\\
29.3814110406251	11.6977009789105\\
29.3811160887722	11.6974750691965\\
29.3808211269032	11.6972491536067\\
29.3805261550172	11.6970232321408\\
29.3802311731137	11.6967973047985\\
29.379936181192	11.6965713715795\\
29.3796411792514	11.6963454324834\\
29.3793461672912	11.6961194875101\\
29.3790511453107	11.6958935366591\\
29.3787561133092	11.6956675799302\\
29.3784610712861	11.695441617323\\
29.3781660192407	11.6952156488373\\
29.3778709571723	11.6949896744727\\
29.3775758850802	11.6947636942289\\
29.3772808029638	11.6945377081057\\
29.3769857108223	11.6943117161027\\
29.376690608655	11.6940857182197\\
29.3763954964614	11.6938597144562\\
29.3761003742407	11.6936337048121\\
29.3758052419922	11.693407689287\\
29.3755100997153	11.6931816678805\\
29.3752149474092	11.6929556405925\\
29.3749197850734	11.6927296074225\\
29.374624612707	11.6925035683704\\
29.3743294303094	11.6922775234357\\
29.37403423788	11.6920514726181\\
29.3737390354181	11.6918254159174\\
29.3734438229229	11.6915993533333\\
29.3731486003939	11.6913732848654\\
29.3728533678302	11.6911472105135\\
29.3725581252313	11.6909211302772\\
29.3722628725964	11.6906950441562\\
29.3719676099249	11.6904689521503\\
29.3716723372161	11.690242854259\\
29.3713770544693	11.6900167504822\\
29.3710817616838	11.6897906408195\\
29.370786458859	11.6895645252706\\
29.3704911459941	11.6893384038351\\
29.3701958230884	11.6891122765129\\
29.3699004901414	11.6888861433035\\
29.3696051471523	11.6886600042067\\
29.3693097941203	11.6884338592222\\
29.369014431045	11.6882077083496\\
29.3687190579255	11.6879815515887\\
29.3684236747611	11.6877553889391\\
29.3681282815513	11.6875292204006\\
29.3678328782952	11.6873030459727\\
29.3675374649923	11.6870768656553\\
29.3672420416419	11.6868506794481\\
29.3669466082431	11.6866244873506\\
29.3666511647955	11.6863982893627\\
29.3663557112983	11.6861720854839\\
29.3660602477507	11.685945875714\\
29.3657647741522	11.6857196600527\\
29.365469290502	11.6854934384997\\
29.3651737967995	11.6852672110547\\
29.3648782930439	11.6850409777173\\
29.3645827792347	11.6848147384873\\
29.364287255371	11.6845884933643\\
29.3639917214523	11.6843622423481\\
29.3636961774777	11.6841359854383\\
29.3634006234468	11.6839097226346\\
29.3631050593587	11.6836834539368\\
29.3628094852128	11.6834571793445\\
29.3625139010084	11.6832308988573\\
29.3622183067447	11.6830046124751\\
29.3619227024213	11.6827783201975\\
29.3616270880372	11.6825520220241\\
29.361331463592	11.6823257179548\\
29.3610358290848	11.6820994079891\\
29.3607401845149	11.6818730921267\\
29.3604445298818	11.6816467703674\\
29.3601488651847	11.6814204427109\\
29.359853190423	11.6811941091568\\
29.3595575055959	11.6809677697049\\
29.3592618107027	11.6807414243547\\
29.3589661057428	11.6805150731061\\
29.3586703907155	11.6802887159587\\
29.3583746656202	11.6800623529122\\
29.358078930456	11.6798359839663\\
29.3577831852224	11.6796096091207\\
29.3574874299186	11.6793832283751\\
29.357191664544	11.6791568417291\\
29.3568958890979	11.6789304491826\\
29.3566001035795	11.678704050735\\
29.3563043079883	11.6784776463863\\
29.3560085023235	11.6782512361359\\
29.3557126865844	11.6780248199837\\
29.3554168607703	11.6777983979294\\
29.3551210248807	11.6775719699725\\
29.3548251789147	11.6773455361129\\
29.3545293228716	11.6771190963502\\
29.3542334567509	11.676892650684\\
29.3539375805518	11.6766661991142\\
29.3536416942736	11.6764397416403\\
29.3533457979157	11.6762132782621\\
29.3530498914773	11.6759868089793\\
29.3527539749578	11.6757603337916\\
29.3524580483565	11.6755338526986\\
29.3521621116727	11.6753073657\\
29.3518661649056	11.6750808727956\\
29.3515702080547	11.674854373985\\
29.3512742411193	11.6746278692679\\
29.3509782640985	11.674401358644\\
29.3506822769918	11.674174842113\\
29.3503862797985	11.6739483196746\\
29.3500902725179	11.6737217913285\\
29.3497942551493	11.6734952570744\\
29.349498227692	11.673268716912\\
29.3492021901453	11.6730421708409\\
29.3489061425085	11.6728156188608\\
29.348610084781	11.6725890609715\\
29.348314016962	11.6723624971727\\
29.3480179390509	11.672135927464\\
29.347721851047	11.6719093518451\\
29.3474257529496	11.6716827703157\\
29.347129644758	11.6714561828755\\
29.3468335264715	11.6712295895242\\
29.3465373980894	11.6710029902615\\
29.346241259611	11.670776385087\\
29.3459451110358	11.6705497740006\\
29.3456489523628	11.6703231570018\\
29.3453527835916	11.6700965340903\\
29.3450566047213	11.6698699052659\\
29.3447604157513	11.6696432705283\\
29.344464216681	11.669416629877\\
29.3441680075095	11.6691899833119\\
29.3438717882363	11.6689633308326\\
29.3435755588606	11.6687366724388\\
29.3432793193818	11.6685100081301\\
29.3429830697991	11.6682833379064\\
29.342686810112	11.6680566617672\\
29.3423905403195	11.6678299797123\\
29.3420942604212	11.6676032917414\\
29.3417979704163	11.6673765978541\\
29.3415016703041	11.6671498980501\\
29.341205360084	11.6669231923292\\
29.3409090397551	11.666696480691\\
29.3406127093169	11.6664697631352\\
29.3403163687687	11.6662430396616\\
29.3400200181097	11.6660163102697\\
29.3397236573394	11.6657895749593\\
29.3394272864569	11.66556283373\\
29.3391309054616	11.6653360865817\\
29.3388345143528	11.6651093335139\\
29.3385381131298	11.6648825745263\\
29.3382417017919	11.6646558096187\\
29.3379452803385	11.6644290387907\\
29.3376488487688	11.6642022620421\\
29.3373524070822	11.6639754793725\\
29.3370559552779	11.6637486907815\\
29.3367594933553	11.663521896269\\
29.3364630213137	11.6632950958345\\
29.3361665391524	11.6630682894778\\
29.3358700468707	11.6628414771986\\
29.3355735444679	11.6626146589966\\
29.3352770319433	11.6623878348714\\
29.3349805092963	11.6621610048227\\
29.334683976526	11.6619341688503\\
29.334387433632	11.6617073269537\\
29.3340908806134	11.6614804791328\\
29.3337943174695	11.6612536253872\\
29.3334977441997	11.6610267657166\\
29.3332011608033	11.6607999001207\\
29.3329045672796	11.6605730285991\\
29.3326079636279	11.6603461511516\\
29.3323113498474	11.6601192677779\\
29.3320147259376	11.6598923784776\\
29.3317180918977	11.6596654832505\\
29.3314214477271	11.6594385820961\\
29.331124793425	11.6592116750143\\
29.3308281289907	11.6589847620047\\
29.3305314544236	11.658757843067\\
29.3302347697229	11.6585309182008\\
29.329938074888	11.658303987406\\
29.3296413699182	11.6580770506821\\
29.3293446548128	11.6578501080288\\
29.329047929571	11.657623159446\\
29.3287511941923	11.6573962049331\\
29.3284544486758	11.65716924449\\
29.328157693021	11.6569422781163\\
29.3278609272271	11.6567153058117\\
29.3275641512934	11.6564883275759\\
29.3272673652193	11.6562613434086\\
29.326970569004	11.6560343533094\\
29.3266737626468	11.6558073572782\\
29.3263769461471	11.6555803553144\\
29.3260801195042	11.655353347418\\
29.3257832827173	11.6551263335884\\
29.3254864357858	11.6548993138255\\
29.325189578709	11.654672288129\\
29.3248927114862	11.6544452564984\\
29.3245958341167	11.6542182189335\\
29.3242989465998	11.653991175434\\
29.3240020489347	11.6537641259996\\
29.3237051411209	11.6535370706299\\
29.3234082231577	11.6533100093247\\
29.3231112950442	11.6530829420836\\
29.3228143567799	11.6528558689064\\
29.322517408364	11.6526287897927\\
29.3222204497958	11.6524017047422\\
29.3219234810747	11.6521746137547\\
29.3216265022	11.6519475168297\\
29.3213295131709	11.651720413967\\
29.3210325139868	11.6514933051662\\
29.320735504647	11.6512661904271\\
29.3204384851507	11.6510390697494\\
29.3201414554973	11.6508119431327\\
29.3198444156862	11.6505848105768\\
29.3195473657165	11.6503576720812\\
29.3192503055876	11.6501305276458\\
29.3189532352988	11.6499033772702\\
29.3186561548494	11.649676220954\\
29.3183590642387	11.6494490586971\\
29.3180619634661	11.649221890499\\
29.3177648525307	11.6489947163594\\
29.317467731432	11.6487675362781\\
29.3171706001693	11.6485403502547\\
29.3168734587417	11.648313158289\\
29.3165763071487	11.6480859603806\\
29.3162791453896	11.6478587565291\\
29.3159819734636	11.6476315467344\\
29.3156847913701	11.647404330996\\
29.3153875991083	11.6471771093137\\
29.3150903966776	11.6469498816872\\
29.3147931840772	11.6467226481161\\
29.3144959613065	11.6464954086001\\
29.3141987283648	11.6462681631389\\
29.3139014852514	11.6460409117323\\
29.3136042319656	11.6458136543799\\
29.3133069685066	11.6455863910813\\
29.3130096948738	11.6453591218364\\
29.3127124110666	11.6451318466447\\
29.3124151170841	11.6449045655059\\
29.3121178129257	11.6446772784199\\
29.3118204985908	11.6444499853861\\
29.3115231740785	11.6442226864044\\
29.3112258393882	11.6439953814743\\
29.3109284945193	11.6437680705957\\
29.310631139471	11.6435407537682\\
29.3103337742426	11.6433134309914\\
29.3100363988334	11.6430861022651\\
29.3097390132428	11.642858767589\\
29.3094416174699	11.6426314269627\\
29.3091442115143	11.6424040803859\\
29.308846795375	11.6421767278584\\
29.3085493690515	11.6419493693797\\
29.308251932543	11.6417220049497\\
29.3079544858489	11.641494634568\\
29.3076570289684	11.6412672582342\\
29.3073595619008	11.6410398759481\\
29.3070620846455	11.6408124877093\\
29.3067645972018	11.6405850935176\\
29.3064670995689	11.6403576933726\\
29.3061695917461	11.640130287274\\
29.3058720737328	11.6399028752215\\
29.3055745455283	11.6396754572148\\
29.3052770071318	11.6394480332536\\
29.3049794585427	11.6392206033375\\
29.3046818997603	11.6389931674664\\
29.3043843307838	11.6387657256397\\
29.3040867516126	11.6385382778573\\
29.303789162246	11.6383108241188\\
29.3034915626832	11.6380833644239\\
29.3031939529236	11.6378558987723\\
29.3028963329664	11.6376284271638\\
29.3025987028111	11.6374009495978\\
29.3023010624568	11.6371734660743\\
29.3020034119029	11.6369459765927\\
29.3017057511487	11.636718481153\\
29.3014080801934	11.6364909797546\\
29.3011103990364	11.6362634723974\\
29.300812707677	11.6360359590809\\
29.3005150061145	11.635808439805\\
29.3002172943482	11.6355809145692\\
29.2999195723773	11.6353533833733\\
29.2996218402012	11.6351258462169\\
29.2993240978192	11.6348983030998\\
29.2990263452306	11.6346707540216\\
29.2987285824347	11.634443198982\\
29.2984308094307	11.6342156379807\\
29.298133026218	11.6339880710174\\
29.2978352327959	11.6337604980918\\
29.2975374291637	11.6335329192035\\
29.2972396153207	11.6333053343523\\
29.2969417912661	11.6330777435379\\
29.2966439569994	11.6328501467598\\
29.2963461125197	11.6326225440179\\
29.2960482578264	11.6323949353118\\
29.2957503929187	11.6321673206411\\
29.2954525177961	11.6319397000057\\
29.2951546324577	11.6317120734051\\
29.2948567369029	11.6314844408391\\
29.294558831131	11.6312568023073\\
29.2942609151413	11.6310291578094\\
29.293962988933	11.6308015073451\\
29.2936650525055	11.6305738509142\\
29.2933671058581	11.6303461885162\\
29.29306914899	11.6301185201509\\
29.2927711819007	11.629890845818\\
29.2924732045892	11.6296631655171\\
29.2921752170551	11.629435479248\\
29.2918772192975	11.6292077870102\\
29.2915792113158	11.6289800888036\\
29.2912811931093	11.6287523846278\\
29.2909831646772	11.6285246744825\\
29.2906851260189	11.6282969583673\\
29.2903870771336	11.628069236282\\
29.2900890180207	11.6278415082263\\
29.2897909486794	11.6276137741997\\
29.2894928691091	11.6273860342021\\
29.2891947793091	11.6271582882331\\
29.2888966792786	11.6269305362924\\
29.2885985690169	11.6267027783797\\
29.2883004485234	11.6264750144947\\
29.2880023177973	11.626247244637\\
29.287704176838	11.6260194688063\\
29.2874060256447	11.6257916870024\\
29.2871078642167	11.6255638992248\\
29.2868096925533	11.6253361054734\\
29.2865115106539	11.6251083057478\\
29.2862133185177	11.6248805000476\\
29.2859151161441	11.6246526883726\\
29.2856169035322	11.6244248707225\\
29.2853186806815	11.6241970470968\\
29.2850204475912	11.6239692174954\\
29.2847222042606	11.623741381918\\
29.284423950689	11.623513540364\\
29.2841256868757	11.6232856928334\\
29.28382741282	11.6230578393258\\
29.2835291285212	11.6228299798408\\
29.2832308339786	11.6226021143781\\
29.2829325291915	11.6223742429375\\
29.2826342141591	11.6221463655185\\
29.2823358888809	11.621918482121\\
29.282037553356	11.6216905927446\\
29.2817392075838	11.6214626973889\\
29.2814408515636	11.6212347960537\\
29.2811424852946	11.6210068887386\\
29.2808441087762	11.6207789754433\\
29.2805457220077	11.6205510561676\\
29.2802473249883	11.6203231309111\\
29.2799489177173	11.6200951996734\\
29.2796505001941	11.6198672624544\\
29.279352072418	11.6196393192535\\
29.2790536343882	11.6194113700707\\
29.278755186104	11.6191834149054\\
29.2784567275647	11.6189554537575\\
29.2781582587697	11.6187274866266\\
29.2778597797181	11.6184995135124\\
29.2775612904094	11.6182715344145\\
29.2772627908428	11.6180435493327\\
29.2769642810176	11.6178155582667\\
29.2766657609331	11.6175875612161\\
29.2763672305886	11.6173595581806\\
29.2760686899834	11.6171315491599\\
29.2757701391167	11.6169035341537\\
29.2754715779879	11.6166755131617\\
29.2751730065963	11.6164474861835\\
29.2748744249412	11.6162194532189\\
29.2745758330218	11.6159914142675\\
29.2742772308375	11.6157633693291\\
29.2739786183875	11.6155353184032\\
29.2736799956711	11.6153072614897\\
29.2733813626877	11.6150791985881\\
29.2730827194366	11.6148511296981\\
29.2727840659169	11.6146230548196\\
29.2724854021281	11.614394973952\\
29.2721867280693	11.6141668870952\\
29.27188804374	11.6139387942487\\
29.2715893491394	11.6137106954124\\
29.2712906442667	11.6134825905858\\
29.2709919291214	11.6132544797687\\
29.2706932037026	11.6130263629607\\
29.2703944680097	11.6127982401616\\
29.270095722042	11.612570111371\\
29.2697969657987	11.6123419765886\\
29.2694981992792	11.612113835814\\
29.2691994224827	11.6118856890471\\
29.2689006354086	11.6116575362873\\
29.268601838056	11.6114293775346\\
29.2683030304245	11.6112012127884\\
29.2680042125131	11.6109730420486\\
29.2677053843212	11.6107448653148\\
29.2674065458481	11.6105166825866\\
29.2671076970932	11.6102884938638\\
29.2668088380556	11.6100602991461\\
29.2665099687347	11.6098320984331\\
29.2662110891297	11.6096038917245\\
29.2659121992401	11.60937567902\\
29.2656132990649	11.6091474603194\\
29.2653143886037	11.6089192356221\\
29.2650154678555	11.6086910049281\\
29.2647165368198	11.6084627682369\\
29.2644175954958	11.6082345255482\\
29.2641186438829	11.6080062768618\\
29.2638196819802	11.6077780221772\\
29.2635207097872	11.6075497614943\\
29.263221727303	11.6073214948126\\
29.262922734527	11.6070932221318\\
29.2626237314585	11.6068649434517\\
29.2623247180967	11.6066366587719\\
29.262025694441	11.6064083680922\\
29.2617266604907	11.6061800714121\\
29.261427616245	11.6059517687314\\
29.2611285617032	11.6057234600498\\
29.2608294968646	11.6054951453669\\
29.2605304217285	11.6052668246824\\
29.2602313362943	11.6050384979961\\
29.2599322405611	11.6048101653075\\
29.2596331345283	11.6045818266165\\
29.2593340181952	11.6043534819226\\
29.2590348915611	11.6041251312255\\
29.2587357546252	11.603896774525\\
29.2584366073869	11.6036684118208\\
29.2581374498453	11.6034400431124\\
29.257838282	11.6032116683996\\
29.25753910385	11.6029832876821\\
29.2572399153947	11.6027549009596\\
29.2569407166335	11.6025265082316\\
29.2566415075655	11.6022981094981\\
29.2563422881901	11.6020697047585\\
29.2560430585065	11.6018412940126\\
29.2557438185142	11.6016128772601\\
29.2554445682122	11.6013844545007\\
29.2551453076	11.601156025734\\
29.2548460366768	11.6009275909598\\
29.254546755442	11.6006991501777\\
29.2542474638947	11.6004707033873\\
29.2539481620343	11.6002422505885\\
29.2536488498602	11.6000137917808\\
29.2533495273714	11.599785326964\\
29.2530501945675	11.5995568561377\\
29.2527508514476	11.5993283793016\\
29.252451498011	11.5990998964554\\
29.252152134257	11.5988714075989\\
29.251852760185	11.5986429127315\\
29.2515533757942	11.5984144118532\\
29.2512539810838	11.5981859049635\\
29.2509545760533	11.5979573920621\\
29.2506551607018	11.5977288731487\\
29.2503557350287	11.597500348223\\
29.2500562990332	11.5972718172846\\
29.2497568527146	11.5970432803334\\
29.2494573960723	11.5968147373688\\
29.2491579291055	11.5965861883907\\
29.2488584518135	11.5963576333987\\
29.2485589641956	11.5961290723925\\
29.248259466251	11.5959005053718\\
29.2479599579791	11.5956719323362\\
29.2476604393792	11.5954433532855\\
29.2473609104505	11.5952147682192\\
29.2470613711923	11.5949861771372\\
29.2467618216039	11.5947575800391\\
29.2464622616847	11.5945289769246\\
29.2461626914338	11.5943003677933\\
29.2458631108506	11.594071752645\\
29.2455635199344	11.5938431314792\\
29.2452639186844	11.5936145042958\\
29.2449643070999	11.5933858710944\\
29.2446646851803	11.5931572318747\\
29.2443650529248	11.5929285866363\\
29.2440654103327	11.5926999353789\\
29.2437657574033	11.5924712781023\\
29.2434660941358	11.5922426148061\\
29.2431664205296	11.5920139454899\\
29.242866736584	11.5917852701536\\
29.2425670422981	11.5915565887967\\
29.2422673376714	11.5913279014189\\
29.2419676227031	11.59109920802\\
29.2416678973925	11.5908705085995\\
29.2413681617389	11.5906418031573\\
29.2410684157415	11.5904130916929\\
29.2407686593997	11.5901843742061\\
29.2404688927127	11.5899556506965\\
29.2401691156798	11.5897269211638\\
29.2398693283003	11.5894981856077\\
29.2395695305736	11.5892694440279\\
29.2392697224988	11.5890406964241\\
29.2389699040752	11.5888119427959\\
29.2386700753023	11.5885831831431\\
29.2383702361791	11.5883544174653\\
29.2380703867051	11.5881256457621\\
29.2377705268795	11.5878968680334\\
29.2374706567015	11.5876680842787\\
29.2371707761706	11.5874392944978\\
29.2368708852859	11.5872104986903\\
29.2365709840467	11.5869816968558\\
29.2362710724524	11.5867528889942\\
29.2359711505022	11.5865240751051\\
29.2356712181954	11.5862952551881\\
29.2353712755313	11.586066429243\\
29.2350713225091	11.5858375972694\\
29.2347713591282	11.585608759267\\
29.2344713853879	11.5853799152354\\
29.2341714012873	11.5851510651745\\
29.2338714068259	11.5849222090838\\
29.2335714020029	11.584693346963\\
29.2332713868175	11.5844644788118\\
29.2329713612691	11.58423560463\\
29.2326713253569	11.5840067244171\\
29.2323712790803	11.5837778381729\\
29.2320712224385	11.583548945897\\
29.2317711554308	11.5833200475892\\
29.2314710780564	11.583091143249\\
29.2311709903148	11.5828622328763\\
29.230870892205	11.5826333164706\\
29.2305707837265	11.5824043940317\\
29.2302706648786	11.5821754655592\\
29.2299705356604	11.5819465310529\\
29.2296703960713	11.5817175905123\\
29.2293702461105	11.5814886439372\\
29.2290700857774	11.5812596913273\\
29.2287699150713	11.5810307326823\\
29.2284697339913	11.5808017680017\\
29.2281695425369	11.5805727972854\\
29.2278693407072	11.580343820533\\
29.2275691285016	11.5801148377441\\
29.2272689059193	11.5798858489185\\
29.2269686729597	11.5796568540558\\
29.2266684296219	11.5794278531558\\
29.2263681759054	11.579198846218\\
29.2260679118093	11.5789698332422\\
29.225767637333	11.5787408142281\\
29.2254673524757	11.5785117891754\\
29.2251670572368	11.5782827580836\\
29.2248667516155	11.5780537209526\\
29.224566435611	11.577824677782\\
29.2242661092227	11.5775956285714\\
29.2239657724499	11.5773665733206\\
29.2236654252918	11.5771375120292\\
29.2233650677477	11.5769084446969\\
29.2230646998169	11.5766793713235\\
29.2227643214986	11.5764502919085\\
29.2224639327923	11.5762212064516\\
29.222163533697	11.5759921149526\\
29.2218631242122	11.5757630174112\\
29.2215627043371	11.5755339138269\\
29.221262274071	11.5753048041995\\
29.2209618334132	11.5750756885287\\
29.2206613823628	11.5748465668141\\
29.2203609209194	11.5746174390555\\
29.220060449082	11.5743883052524\\
29.21975996685	11.5741591654047\\
29.2194594742227	11.5739300195119\\
29.2191589711993	11.5737008675738\\
29.2188584577792	11.57347170959\\
29.2185579339616	11.5732425455602\\
29.2182573997457	11.5730133754841\\
29.217956855131	11.5727841993614\\
29.2176563001166	11.5725550171918\\
29.2173557347018	11.5723258289749\\
29.2170551588859	11.5720966347104\\
29.2167545726682	11.571867434398\\
29.216453976048	11.5716382280374\\
29.2161533690246	11.5714090156282\\
29.2158527515972	11.5711797971702\\
29.2155521237651	11.570950572663\\
29.2152514855276	11.5707213421064\\
29.2149508368839	11.5704921054999\\
29.2146501778335	11.5702628628432\\
29.2143495083754	11.5700336141361\\
29.2140488285091	11.5698043593783\\
29.2137481382337	11.5695750985693\\
29.2134474375486	11.5693458317089\\
29.2131467264531	11.5691165587968\\
29.2128460049464	11.5688872798327\\
29.2125452730278	11.5686579948162\\
29.2122445306966	11.5684287037469\\
29.211943777952	11.5681994066247\\
29.2116430147934	11.5679701034492\\
29.21134224122	11.5677407942199\\
29.2110414572312	11.5675114789368\\
29.2107406628261	11.5672821575993\\
29.210439858004	11.5670528302072\\
29.2101390427643	11.5668234967602\\
29.2098382171062	11.566594157258\\
29.209537381029	11.5663648117001\\
29.209236534532	11.5661354600864\\
29.2089356776144	11.5659061024165\\
29.2086348102756	11.5656767386901\\
29.2083339325147	11.5654473689068\\
29.2080330443312	11.5652179930663\\
29.2077321457242	11.5649886111684\\
29.207431236693	11.5647592232126\\
29.207130317237	11.5645298291988\\
29.2068293873553	11.5643004291264\\
29.2065284470473	11.5640710229953\\
29.2062274963123	11.5638416108051\\
29.2059265351495	11.5636121925556\\
29.2056255635582	11.5633827682462\\
29.2053245815377	11.5631533378769\\
29.2050235890872	11.5629239014471\\
29.2047225862061	11.5626944589567\\
29.2044215728936	11.5624650104053\\
29.204120549149	11.5622355557925\\
29.2038195149716	11.5620060951181\\
29.2035184703606	11.5617766283818\\
29.2032174153153	11.5615471555831\\
29.202916349835	11.5613176767218\\
29.202615273919	11.5610881917976\\
29.2023141875665	11.5608587008102\\
29.2020130907769	11.5606292037592\\
29.2017119835494	11.5603997006442\\
29.2014108658832	11.5601701914651\\
29.2011097377777	11.5599406762215\\
29.2008085992322	11.5597111549129\\
29.2005074502458	11.5594816275393\\
29.2002062908179	11.5592520941001\\
29.1999051209478	11.5590225545951\\
29.1996039406348	11.558793009024\\
29.199302749878	11.5585634573864\\
29.1990015486768	11.558333899682\\
29.1987003370305	11.5581043359106\\
29.1983991149383	11.5578747660717\\
29.1980978823996	11.5576451901651\\
29.1977966394135	11.5574156081904\\
29.1974953859794	11.5571860201474\\
29.1971941220965	11.5569564260356\\
29.1968928477642	11.5567268258549\\
29.1965915629816	11.5564972196048\\
29.1962902677481	11.556267607285\\
29.1959889620629	11.5560379888952\\
29.1956876459254	11.5558083644352\\
29.1953863193347	11.5555787339045\\
29.1950849822902	11.5553490973028\\
29.1947836347912	11.5551194546299\\
29.1944822768369	11.5548898058855\\
29.1941809084265	11.5546601510691\\
29.1938795295594	11.5544304901804\\
29.1935781402349	11.5542008232192\\
29.1932767404521	11.5539711501852\\
29.1929753302105	11.5537414710779\\
29.1926739095092	11.5535117858971\\
29.1923724783475	11.5532820946425\\
29.1920710367247	11.5530523973138\\
29.1917695846402	11.5528226939105\\
29.191468122093	11.5525929844325\\
29.1911666490827	11.5523632688793\\
29.1908651656083	11.5521335472507\\
29.1905636716691	11.5519038195463\\
29.1902621672646	11.5516740857659\\
29.1899606523938	11.551444345909\\
29.1896591270562	11.5512145999754\\
29.1893575912509	11.5509848479648\\
29.1890560449773	11.5507550898767\\
29.1887544882346	11.550525325711\\
29.188452921022	11.5502955554673\\
29.188151343339	11.5500657791452\\
29.1878497551846	11.5498359967445\\
29.1875481565583	11.5496062082648\\
29.1872465474592	11.5493764137058\\
29.1869449278867	11.5491466130672\\
29.1866432978401	11.5489168063486\\
29.1863416573185	11.5486869935498\\
29.1860400063213	11.5484571746703\\
29.1857383448477	11.54822734971\\
29.1854366728971	11.5479975186684\\
29.1851349904686	11.5477676815453\\
29.1848332975616	11.5475378383403\\
29.1845315941754	11.5473079890531\\
29.1842298803091	11.5470781336834\\
29.1839281559621	11.5468482722308\\
29.1836264211337	11.5466184046951\\
29.1833246758231	11.5463885310759\\
29.1830229200295	11.5461586513729\\
29.1827211537524	11.5459287655857\\
29.1824193769908	11.5456988737141\\
29.1821175897442	11.5454689757578\\
29.1818157920117	11.5452390717163\\
29.1815139837927	11.5450091615894\\
29.1812121650864	11.5447792453768\\
29.1809103358921	11.5445493230781\\
29.1806084962091	11.544319394693\\
29.1803066460366	11.5440894602213\\
29.1800047853739	11.5438595196625\\
29.1797029142203	11.5436295730163\\
29.179401032575	11.5433996202825\\
29.1790991404373	11.5431696614606\\
29.1787972378065	11.5429396965505\\
29.1784953246819	11.5427097255517\\
29.1781934010627	11.542479748464\\
29.1778914669482	11.5422497652869\\
29.1775895223377	11.5420197760203\\
29.1772875672304	11.5417897806637\\
29.1769856016256	11.5415597792169\\
29.1766836255226	11.5413297716795\\
29.1763816389206	11.5410997580512\\
29.176079641819	11.5408697383317\\
29.1757776342169	11.5406397125206\\
29.1754756161137	11.5404096806177\\
29.1751735875086	11.5401796426226\\
29.174871548401	11.539949598535\\
29.1745694987899	11.5397195483545\\
29.1742674386749	11.5394894920809\\
29.173965368055	11.5392594297138\\
29.1736632869296	11.5390293612529\\
29.173361195298	11.5387992866979\\
29.1730590931593	11.5385692060484\\
29.172756980513	11.5383391193041\\
29.1724548573582	11.5381090264648\\
29.1721527236942	11.53787892753\\
29.1718505795203	11.5376488224996\\
29.1715484248358	11.537418711373\\
29.1712462596399	11.5371885941501\\
29.1709440839319	11.5369584708304\\
29.170641897711	11.5367283414137\\
29.1703397009766	11.5364982058997\\
29.1700374937279	11.536268064288\\
29.1697352759641	11.5360379165783\\
29.1694330476846	11.5358077627702\\
29.1691308088886	11.5355776028635\\
29.1688285595754	11.5353474368579\\
29.1685262997442	11.5351172647529\\
29.1682240293943	11.5348870865483\\
29.167921748525	11.5346569022438\\
29.1676194571356	11.534426711839\\
29.1673171552252	11.5341965153337\\
29.1670148427933	11.5339663127274\\
29.166712519839	11.5337361040199\\
29.1664101863616	11.5335058892108\\
29.1661078423603	11.5332756682999\\
29.1658054878346	11.5330454412868\\
29.1655031227835	11.5328152081711\\
29.1652007472065	11.5325849689525\\
29.1648983611026	11.5323547236308\\
29.1645959644713	11.5321244722056\\
29.1642935573118	11.5318942146766\\
29.1639911396234	11.5316639510435\\
29.1636887114053	11.5314336813058\\
29.1633862726567	11.5312034054634\\
29.163083823377	11.5309731235158\\
29.1627813635654	11.5307428354628\\
29.1624788932212	11.5305125413041\\
29.1621764123437	11.5302822410393\\
29.1618739209321	11.530051934668\\
29.1615714189857	11.52982162219\\
29.1612689065037	11.529591303605\\
29.1609663834854	11.5293609789125\\
29.1606638499301	11.5291306481124\\
29.1603613058371	11.5289003112042\\
29.1600587512056	11.5286699681877\\
29.1597561860348	11.5284396190625\\
29.1594536103241	11.5282092638283\\
29.1591510240728	11.5279789024848\\
29.15884842728	11.5277485350316\\
29.158545819945	11.5275181614684\\
29.1582432020672	11.527287781795\\
29.1579405736457	11.5270573960109\\
29.1576379346798	11.5268270041159\\
29.1573352851689	11.5265966061096\\
29.1570326251122	11.5263662019917\\
29.1567299545088	11.5261357917618\\
29.1564272733582	11.5259053754198\\
29.1561245816595	11.5256749529651\\
29.1558218794121	11.5254445243976\\
29.1555191666152	11.5252140897169\\
29.155216443268	11.5249836489226\\
29.1549137093698	11.5247532020144\\
29.15461096492	11.5245227489921\\
29.1543082099177	11.5242922898552\\
29.1540054443622	11.5240618246035\\
29.1537026682527	11.5238313532367\\
29.1533998815887	11.5236008757543\\
29.1530970843692	11.5233703921562\\
29.1527942765936	11.5231399024419\\
29.1524914582611	11.5229094066111\\
29.1521886293711	11.5226789046636\\
29.1518857899227	11.5224483965989\\
29.1515829399153	11.5222178824168\\
29.151280079348	11.5219873621169\\
29.1509772082202	11.521756835699\\
29.1506743265311	11.5215263031626\\
29.15037143428	11.5212957645075\\
29.1500685314661	11.5210652197334\\
29.1497656180888	11.5208346688398\\
29.1494626941472	11.5206041118266\\
29.1491597596406	11.5203735486933\\
29.1488568145684	11.5201429794397\\
29.1485538589297	11.5199124040654\\
29.1482508927238	11.51968182257\\
29.1479479159501	11.5194512349533\\
29.1476449286076	11.519220641215\\
29.1473419306958	11.5189900413547\\
29.1470389222139	11.5187594353721\\
29.1467359031611	11.5185288232668\\
29.1464328735367	11.5182982050386\\
29.1461298333399	11.5180675806871\\
29.1458267825701	11.517836950212\\
29.1455237212265	11.5176063136129\\
29.1452206493083	11.5173756708896\\
29.1449175668149	11.5171450220417\\
29.1446144737454	11.5169143670689\\
29.1443113700991	11.5166837059709\\
29.1440082558754	11.5164530387472\\
29.1437051310734	11.5162223653977\\
29.1434019956924	11.515991685922\\
29.1430988497317	11.5157610003198\\
29.1427956931905	11.5155303085907\\
29.1424925260682	11.5152996107344\\
29.1421893483639	11.5150689067506\\
29.1418861600769	11.5148381966389\\
29.1415829612066	11.5146074803991\\
29.141279751752	11.5143767580308\\
29.1409765317126	11.5141460295337\\
29.1406733010876	11.5139152949074\\
29.1403700598762	11.5136845541517\\
29.1400668080777	11.5134538072661\\
29.1397635456913	11.5132230542505\\
29.1394602727164	11.5129922951044\\
29.1391569891521	11.5127615298275\\
29.1388536949978	11.5125307584196\\
29.1385503902526	11.5122999808802\\
29.1382470749159	11.512069197209\\
29.137943748987	11.5118384074058\\
29.137640412465	11.5116076114702\\
29.1373370653492	11.5113768094019\\
29.137033707639	11.5111460012005\\
29.1367303393335	11.5109151868658\\
29.136426960432	11.5106843663973\\
29.1361235709338	11.5104535397948\\
29.1358201708381	11.5102227070579\\
29.1355167601442	11.5099918681864\\
29.1352133388514	11.5097610231799\\
29.1349099069589	11.509530172038\\
29.134606464466	11.5092993147605\\
29.1343030113719	11.5090684513469\\
29.1339995476758	11.5088375817971\\
29.1336960733772	11.5086067061106\\
29.1333925884752	11.5083758242872\\
29.133089092969	11.5081449363265\\
29.1327855868579	11.5079140422281\\
29.1324820701413	11.5076831419918\\
29.1321785428183	11.5074522356173\\
29.1318750048882	11.5072213231041\\
29.1315714563502	11.506990404452\\
29.1312678972037	11.5067594796607\\
29.1309643274479	11.5065285487298\\
29.130660747082	11.506297611659\\
29.1303571561053	11.506066668448\\
29.130053554517	11.5058357190964\\
29.1297499423165	11.5056047636039\\
29.1294463195029	11.5053738019702\\
29.1291426860756	11.505142834195\\
29.1288390420337	11.504911860278\\
29.1285353873766	11.5046808802187\\
29.1282317221035	11.5044498940169\\
29.1279280462137	11.5042189016723\\
29.1276243597064	11.5039879031845\\
29.1273206625809	11.5037568985532\\
29.1270169548364	11.5035258877781\\
29.1267132364721	11.5032948708589\\
29.1264095074875	11.5030638477951\\
29.1261057678816	11.5028328185866\\
29.1258020176538	11.5026017832329\\
29.1254982568033	11.5023707417338\\
29.1251944853294	11.5021396940889\\
29.1248907032314	11.5019086402979\\
29.1245869105084	11.5016775803605\\
29.1242831071598	11.5014465142762\\
29.1239792931847	11.5012154420449\\
29.1236754685826	11.5009843636662\\
29.1233716333525	11.5007532791398\\
29.1230677874939	11.5005221884652\\
29.1227639310058	11.5002910916423\\
29.1224600638877	11.5000599886706\\
29.1221561861387	11.4998288795499\\
29.121852297758	11.4995977642798\\
29.1215483987451	11.49936664286\\
29.121244489099	11.4991355152901\\
29.1209405688191	11.4989043815699\\
29.1206366379047	11.498673241699\\
29.1203326963549	11.4984420956771\\
29.120028744169	11.4982109435038\\
29.1197247813464	11.4979797851789\\
29.1194208078862	11.497748620702\\
29.1191168237877	11.4975174500727\\
29.1188128290501	11.4972862732908\\
29.1185088236727	11.4970550903559\\
29.1182048076549	11.4968239012677\\
29.1179007809957	11.4965927060258\\
29.1175967436945	11.49636150463\\
29.1172926957506	11.4961302970799\\
29.1169886371631	11.4958990833752\\
29.1166845679314	11.4956678635155\\
29.1163804880546	11.4954366375006\\
29.1160763975321	11.49520540533\\
29.1157722963632	11.4949741670035\\
29.1154681845469	11.4947429225208\\
29.1151640620827	11.4945116718815\\
29.1148599289698	11.4942804150852\\
29.1145557852074	11.4940491521317\\
29.1142516307947	11.4938178830207\\
29.1139474657311	11.4935866077517\\
29.1136432900158	11.4933553263245\\
29.113339103648	11.4931240387388\\
29.113034906627	11.4928927449942\\
29.1127306989521	11.4926614450903\\
29.1124264806224	11.492430139027\\
29.1121222516373	11.4921988268037\\
29.111818011996	11.4919675084203\\
29.1115137616978	11.4917361838764\\
29.1112095007419	11.4915048531716\\
29.1109052291276	11.4912735163056\\
29.110600946854	11.4910421732781\\
29.1102966539206	11.4908108240888\\
29.1099923503265	11.4905794687373\\
29.1096880360709	11.4903481072233\\
29.1093837111532	11.4901167395465\\
29.1090793755726	11.4898853657065\\
29.1087750293283	11.4896539857031\\
29.1084706724196	11.4894225995359\\
29.1081663048457	11.4891912072045\\
29.107861926606	11.4889598087087\\
29.1075575376995	11.4887284040481\\
29.1072531381257	11.4884969932224\\
29.1069487278838	11.4882655762312\\
29.1066443069729	11.4880341530742\\
29.1063398753924	11.4878027237511\\
29.1060354331415	11.4875712882616\\
29.1057309802195	11.4873398466054\\
29.1054265166256	11.487108398782\\
29.105122042359	11.4868769447913\\
29.1048175574191	11.4866454846328\\
29.104513061805	11.4864140183062\\
29.1042085555161	11.4861825458112\\
29.1039040385515	11.4859510671474\\
29.1035995109106	11.4857195823146\\
29.1032949725925	11.4854880913124\\
29.1029904235966	11.4852565941405\\
29.102685863922	11.4850250907986\\
29.102381293568	11.4847935812862\\
29.102076712534	11.4845620656032\\
29.101772120819	11.4843305437491\\
29.1014675184225	11.4840990157237\\
29.1011629053436	11.4838674815265\\
29.1008582815815	11.4836359411574\\
29.1005536471356	11.4834043946159\\
29.1002490020051	11.4831728419017\\
29.0999443461892	11.4829412830145\\
29.0996396796872	11.4827097179539\\
29.0993350024983	11.4824781467197\\
29.0990303146218	11.4822465693115\\
29.098725616057	11.4820149857289\\
29.098420906803	11.4817833959717\\
29.0981161868592	11.4815518000396\\
29.0978114562247	11.4813201979321\\
29.0975067148989	11.4810885896489\\
29.097201962881	11.4808569751898\\
29.0968972001702	11.4806253545544\\
29.0965924267658	11.4803937277423\\
29.096287642667	11.4801620947533\\
29.0959828478731	11.479930455587\\
29.0956780423834	11.4796988102431\\
29.095373226197	11.4794671587212\\
29.0950683993132	11.479235501021\\
29.0947635617314	11.4790038371423\\
29.0944587134506	11.4787721670846\\
29.0941538544703	11.4785404908476\\
29.0938489847895	11.478308808431\\
29.0935441044077	11.4780771198345\\
29.0932392133239	11.4778454250577\\
29.0929343115376	11.4776137241004\\
29.0926293990478	11.4773820169621\\
29.092324475854	11.4771503036426\\
29.0920195419552	11.4769185841415\\
29.0917145973508	11.4766868584585\\
29.0914096420401	11.4764551265933\\
29.0911046760222	11.4762233885455\\
29.0907996992964	11.4759916443148\\
29.0904947118619	11.4757598939009\\
29.0901897137181	11.4755281373034\\
29.0898847048641	11.475296374522\\
29.0895796852993	11.4750646055565\\
29.0892746550228	11.4748328304063\\
29.0889696140338	11.4746010490713\\
29.0886645623318	11.4743692615511\\
29.0883594999158	11.4741374678454\\
29.0880544267852	11.4739056679538\\
29.0877493429391	11.473673861876\\
29.0874442483769	11.4734420496116\\
29.0871391430978	11.4732102311604\\
29.086834027101	11.472978406522\\
29.0865289003857	11.4727465756961\\
29.0862237629513	11.4725147386824\\
29.085918614797	11.4722828954805\\
29.085613455922	11.47205104609\\
29.0853082863255	11.4718191905108\\
29.0850031060069	11.4715873287423\\
29.0846979149653	11.4713554607844\\
29.0843927132	11.4711235866366\\
29.0840875007103	11.4708917062987\\
29.0837822774954	11.4706598197702\\
29.0834770435545	11.470427927051\\
29.0831717988869	11.4701960281406\\
29.0828665434918	11.4699641230387\\
29.0825612773685	11.469732211745\\
29.0822560005163	11.4695002942591\\
29.0819507129343	11.4692683705808\\
29.0816454146218	11.4690364407097\\
29.0813401055781	11.4688045046454\\
29.0810347858024	11.4685725623877\\
29.0807294552939	11.4683406139362\\
29.080424114052	11.4681086592905\\
29.0801187620758	11.4678766984504\\
29.0798133993646	11.4676447314155\\
29.0795080259176	11.4674127581855\\
29.0792026417341	11.46718077876\\
29.0788972468134	11.4669487931387\\
29.0785918411546	11.4667168013214\\
29.078286424757	11.4664848033075\\
29.0779809976199	11.4662527990969\\
29.0776755597425	11.4660207886892\\
29.077370111124	11.4657887720841\\
29.0770646517637	11.4655567492812\\
29.0767591816609	11.4653247202801\\
29.0764537008147	11.4650926850807\\
29.0761482092245	11.4648606436825\\
29.0758427068895	11.4646285960852\\
29.0755371938088	11.4643965422884\\
29.0752316699819	11.4641644822919\\
29.0749261354078	11.4639324160954\\
29.0746205900859	11.4637003436984\\
29.0743150340153	11.4634682651006\\
29.0740094671954	11.4632361803018\\
29.0737038896254	11.4630040893016\\
29.0733983013045	11.4627719920996\\
29.073092702232	11.4625398886955\\
29.0727870924071	11.462307779089\\
29.072481471829	11.4620756632798\\
29.072175840497	11.4618435412675\\
29.0718701984104	11.4616114130519\\
29.0715645455684	11.4613792786324\\
29.0712588819701	11.461147138009\\
29.070953207615	11.4609149911811\\
29.0706475225021	11.4606828381485\\
29.0703418266308	11.4604506789108\\
29.0700361200004	11.4602185134677\\
29.0697304026099	11.4599863418189\\
29.0694246744587	11.4597541639641\\
29.0691189355461	11.4595219799029\\
29.0688131858712	11.4592897896349\\
29.0685074254333	11.4590575931599\\
29.0682016542317	11.4588253904775\\
29.0678958722655	11.4585931815874\\
29.0675900795341	11.4583609664893\\
29.0672842760367	11.4581287451828\\
29.0669784617725	11.4578965176675\\
29.0666726367408	11.4576642839433\\
29.0663668009408	11.4574320440096\\
29.0660609543717	11.4571997978662\\
29.0657550970328	11.4569675455128\\
29.0654492289233	11.4567352869491\\
29.0651433500425	11.4565030221746\\
29.0648374603896	11.4562707511891\\
29.0645315599638	11.4560384739922\\
29.0642256487645	11.4558061905836\\
29.0639197267907	11.455573900963\\
29.0636137940419	11.4553416051301\\
29.0633078505171	11.4551093030844\\
29.0630018962158	11.4548769948257\\
29.062695931137	11.4546446803536\\
29.06238995528	11.4544123596679\\
29.0620839686442	11.4541800327681\\
29.0617779712286	11.453947699654\\
29.0614719630327	11.4537153603251\\
29.0611659440555	11.4534830147813\\
29.0608599142963	11.4532506630221\\
29.0605538737545	11.4530183050472\\
29.0602478224291	11.4527859408563\\
29.0599417603195	11.452553570449\\
29.0596356874249	11.452321193825\\
29.0593296037446	11.4520888109841\\
29.0590235092777	11.4518564219257\\
29.0587174040236	11.4516240266497\\
29.0584112879814	11.4513916251557\\
29.0581051611504	11.4511592174433\\
29.0577990235299	11.4509268035123\\
29.057492875119	11.4506943833622\\
29.057186715917	11.4504619569928\\
29.0568805459232	11.4502295244036\\
29.0565743651369	11.4499970855945\\
29.0562681735571	11.4497646405651\\
29.0559619711832	11.4495321893149\\
29.0556557580144	11.4492997318438\\
29.05534953405	11.4490672681512\\
29.0550432992892	11.4488347982371\\
29.0547370537312	11.4486023221009\\
29.0544307973752	11.4483698397423\\
29.0541245302206	11.4481373511611\\
29.0538182522665	11.4479048563569\\
29.0535119635122	11.4476723553293\\
29.0532056639569	11.4474398480781\\
29.0528993535999	11.4472073346028\\
29.0525930324403	11.4469748149032\\
29.0522867004776	11.4467422889789\\
29.0519803577107	11.4465097568296\\
29.0516740041391	11.446277218455\\
29.051367639762	11.4460446738547\\
29.0510612645785	11.4458121230284\\
29.050754878588	11.4455795659757\\
29.0504484817896	11.4453470026963\\
29.0501420741826	11.44511443319\\
29.0498356557663	11.4448818574563\\
29.0495292265398	11.4446492754949\\
29.0492227865024	11.4444166873055\\
29.0489163356534	11.4441840928878\\
29.048609873992	11.4439514922414\\
29.0483034015174	11.443718885366\\
29.0479969182288	11.4434862722612\\
29.0476904241256	11.4432536529267\\
29.0473839192069	11.4430210273623\\
29.0470774034719	11.4427883955675\\
29.04677087692	11.442555757542\\
29.0464643395503	11.4423231132855\\
29.0461577913621	11.4420904627976\\
29.0458512323546	11.4418578060781\\
29.045544662527	11.4416251431266\\
29.0452380818787	11.4413924739427\\
29.0449314904087	11.4411597985261\\
29.0446248881165	11.4409271168765\\
29.0443182750011	11.4406944289935\\
29.0440116510618	11.4404617348769\\
29.043705016298	11.4402290345263\\
29.0433983707087	11.4399963279413\\
29.0430917142933	11.4397636151216\\
29.0427850470509	11.4395308960669\\
29.0424783689809	11.4392981707768\\
29.0421716800823	11.439065439251\\
29.0418649803546	11.4388327014892\\
29.0415582697969	11.4385999574911\\
29.0412515484085	11.4383672072563\\
29.0409448161885	11.4381344507844\\
29.0406380731363	11.4379016880752\\
29.040331319251	11.4376689191283\\
29.0400245545319	11.4374361439433\\
29.0397177789782	11.43720336252\\
29.0394109925892	11.436970574858\\
29.0391041953641	11.436737780957\\
29.0387973873021	11.4365049808166\\
29.0384905684024	11.4362721744365\\
29.0381837386644	11.4360393618163\\
29.0378768980872	11.4358065429558\\
29.0375700466701	11.4355737178546\\
29.0372631844123	11.4353408865123\\
29.036956311313	11.4351080489287\\
29.0366494273714	11.4348752051034\\
29.0363425325869	11.434642355036\\
29.0360356269586	11.4344094987262\\
29.0357287104858	11.4341766361738\\
29.0354217831677	11.4339437673782\\
29.0351148450035	11.4337108923393\\
29.0348078959925	11.4334780110567\\
29.0345009361339	11.43324512353\\
29.0341939654269	11.433012229759\\
29.0338869838708	11.4327793297432\\
29.0335799914648	11.4325464234823\\
29.0332729882082	11.4323135109761\\
29.0329659741001	11.4320805922242\\
29.0326589491398	11.4318476672261\\
29.0323519133266	11.4316147359817\\
29.0320448666596	11.4313817984906\\
29.0317378091381	11.4311488547524\\
29.0314307407613	11.4309159047668\\
29.0311236615286	11.4306829485334\\
29.030816571439	11.430449986052\\
29.0305094704918	11.4302170173222\\
29.0302023586863	11.4299840423436\\
29.0298952360217	11.4297510611159\\
29.0295881024972	11.4295180736389\\
29.0292809581121	11.4292850799121\\
29.0289738028656	11.4290520799352\\
29.0286666367569	11.4288190737079\\
29.0283594597852	11.4285860612298\\
29.0280522719499	11.4283530425006\\
29.02774507325	11.4281200175201\\
29.0274378636849	11.4278869862877\\
29.0271306432538	11.4276539488033\\
29.0268234119559	11.4274209050664\\
29.0265161697904	11.4271878550768\\
29.0262089167566	11.4269547988341\\
29.0259016528537	11.4267217363379\\
29.0255943780809	11.4264886675879\\
29.0252870924375	11.4262555925839\\
29.0249797959228	11.4260225113254\\
29.0246724885358	11.4257894238121\\
29.0243651702759	11.4255563300437\\
29.0240578411423	11.4253232300199\\
29.0237505011341	11.4250901237402\\
29.0234431502508	11.4248570112045\\
29.0231357884914	11.4246238924123\\
29.0228284158553	11.4243907673632\\
29.0225210323415	11.4241576360571\\
29.0222136379495	11.4239244984935\\
29.0219062326783	11.4236913546721\\
29.0215988165273	11.4234582045925\\
29.0212913894957	11.4232250482545\\
29.0209839515826	11.4229918856577\\
29.0206765027873	11.4227587168017\\
29.0203690431091	11.4225255416862\\
29.0200615725472	11.4222923603109\\
29.0197540911008	11.4220591726755\\
29.0194465987691	11.4218259787796\\
29.0191390955514	11.4215927786228\\
29.0188315814469	11.4213595722049\\
29.0185240564548	11.4211263595255\\
29.0182165205744	11.4208931405842\\
29.0179089738049	11.4206599153808\\
29.0176014161454	11.4204266839149\\
29.0172938475953	11.4201934461861\\
29.0169862681538	11.4199602021941\\
29.0166786778201	11.4197269519387\\
29.0163710765934	11.4194936954193\\
29.0160634644729	11.4192604326358\\
29.015755841458	11.4190271635878\\
29.0154482075477	11.4187938882749\\
29.0151405627414	11.4185606066967\\
29.0148329070383	11.4183273188531\\
29.0145252404375	11.4180940247436\\
29.0142175629384	11.4178607243678\\
29.0139098745401	11.4176274177256\\
29.0136021752419	11.4173941048164\\
29.013294465043	11.41716078564\\
29.0129867439427	11.4169274601961\\
29.0126790119401	11.4166941284843\\
29.0123712690345	11.4164607905042\\
29.0120635152251	11.4162274462556\\
29.0117557505112	11.4159940957381\\
29.0114479748919	11.4157607389513\\
29.0111401883666	11.415527375895\\
29.0108323909343	11.4152940065687\\
29.0105245825945	11.4150606309722\\
29.0102167633462	11.4148272491051\\
29.0099089331887	11.4145938609671\\
29.0096010921212	11.4143604665578\\
29.009293240143	11.4141270658769\\
29.0089853772533	11.4138936589241\\
29.0086775034513	11.413660245699\\
29.0083696187362	11.4134268262013\\
29.0080617231073	11.4131934004307\\
29.0077538165638	11.4129599683868\\
29.0074458991049	11.4127265300692\\
29.0071379707299	11.4124930854777\\
29.0068300314379	11.4122596346119\\
29.0065220812282	11.4120261774714\\
29.0062141201001	11.411792714056\\
29.0059061480527	11.4115592443653\\
29.0055981650853	11.411325768399\\
29.005290171197	11.4110922861566\\
29.0049821663872	11.4108587976379\\
29.0046741506551	11.4106253028426\\
29.0043661239998	11.4103918017703\\
29.0040580864206	11.4101582944207\\
29.0037500379167	11.4099247807933\\
29.0034419784874	11.409691260888\\
29.0031339081319	11.4094577347043\\
29.0028258268494	11.409224202242\\
29.0025177346391	11.4089906635006\\
29.0022096315002	11.4087571184799\\
29.001901517432	11.4085235671795\\
29.0015933924338	11.408290009599\\
29.0012852565046	11.4080564457382\\
29.0009771096438	11.4078228755966\\
29.0006689518506	11.407589299174\\
29.0003607831241	11.4073557164701\\
29.0000526034637	11.4071221274844\\
28.9997444128686	11.4068885322166\\
28.9994362113379	11.4066549306664\\
28.9991279988708	11.4064213228336\\
28.9988197754668	11.4061877087176\\
28.9985115411248	11.4059540883182\\
28.9982032958442	11.4057204616351\\
28.9978950396242	11.4054868286679\\
28.997586772464	11.4052531894162\\
28.9972784943629	11.4050195438798\\
28.99697020532	11.4047858920583\\
28.9966619053346	11.4045522339514\\
28.9963535944059	11.4043185695587\\
28.9960452725331	11.4040848988798\\
28.9957369397155	11.4038512219145\\
28.9954285959522	11.4036175386624\\
28.9951202412426	11.4033838491232\\
28.9948118755857	11.4031501532965\\
28.9945034989809	11.402916451182\\
28.9941951114274	11.4026827427794\\
28.9938867129244	11.4024490280882\\
28.993578303471	11.4022153071083\\
28.9932698830666	11.4019815798392\\
28.9929614517104	11.4017478462805\\
28.9926530094015	11.4015141064321\\
28.9923445561392	11.4012803602934\\
28.9920360919228	11.4010466078643\\
28.9917276167514	11.4008128491443\\
28.9914191306242	11.4005790841331\\
28.9911106335406	11.4003453128304\\
28.9908021254996	11.4001115352358\\
28.9904936065006	11.399877751349\\
28.9901850765427	11.3996439611696\\
28.9898765356252	11.3994101646974\\
28.9895679837473	11.3991763619319\\
28.9892594209082	11.3989425528728\\
28.9889508471072	11.3987087375199\\
28.9886422623434	11.3984749158727\\
28.988333666616	11.398241087931\\
28.9880250599244	11.3980072536943\\
28.9877164422677	11.3977734131624\\
28.9874078136452	11.3975395663348\\
28.987099174056	11.3973057132113\\
28.9867905234994	11.3970718537916\\
28.9864818619745	11.3968379880752\\
28.9861731894807	11.3966041160619\\
28.9858645060172	11.3963702377513\\
28.9855558115831	11.3961363531431\\
28.9852471061777	11.3959024622368\\
28.9849383898002	11.3956685650323\\
28.9846296624499	11.3954346615292\\
28.9843209241258	11.395200751727\\
28.9840121748274	11.3949668356255\\
28.9837034145537	11.3947329132243\\
28.983394643304	11.3944989845232\\
28.9830858610776	11.3942650495217\\
28.9827770678736	11.3940311082195\\
28.9824682636912	11.3937971606162\\
28.9821594485297	11.3935632067117\\
28.9818506223884	11.3933292465054\\
28.9815417852663	11.393095279997\\
28.9812329371628	11.3928613071863\\
28.9809240780771	11.3926273280729\\
28.9806152080083	11.3923933426564\\
28.9803063269557	11.3921593509365\\
28.9799974349186	11.3919253529128\\
28.979688531896	11.3916913485851\\
28.9793796178874	11.391457337953\\
28.9790706928918	11.3912233210161\\
28.9787617569084	11.3909892977741\\
28.9784528099366	11.3907552682266\\
28.9781438519755	11.3905212323734\\
28.9778348830244	11.3902871902141\\
28.9775259030824	11.3900531417483\\
28.9772169121488	11.3898190869757\\
28.9769079102228	11.3895850258959\\
28.9765988973036	11.3893509585087\\
28.9762898733905	11.3891168848137\\
28.9759808384826	11.3888828048105\\
28.9756717925791	11.3886487184988\\
28.9753627356794	11.3884146258782\\
28.9750536677825	11.3881805269485\\
28.9747445888878	11.3879464217093\\
28.9744354989944	11.3877123101602\\
28.9741263981016	11.3874781923009\\
28.9738172862085	11.3872440681311\\
28.9735081633145	11.3870099376504\\
28.9731990294186	11.3867758008584\\
28.9728898845202	11.386541657755\\
28.9725807286184	11.3863075083396\\
28.9722715617124	11.3860733526119\\
28.9719623838016	11.3858391905717\\
28.971653194885	11.3856050222186\\
28.9713439949619	11.3853708475522\\
28.9710347840316	11.3851366665722\\
28.9707255620932	11.3849024792783\\
28.9704163291459	11.38466828567\\
28.970107085189	11.3844340857472\\
28.9697978302217	11.3841998795094\\
28.9694885642432	11.3839656669563\\
28.9691792872527	11.3837314480875\\
28.9688699992495	11.3834972229028\\
28.9685607002327	11.3832629914017\\
28.9682513902016	11.3830287535839\\
28.9679420691554	11.3827945094492\\
28.9676327370932	11.3825602589971\\
28.9673233940144	11.3823260022273\\
28.9670140399181	11.3820917391395\\
28.9667046748035	11.3818574697333\\
28.9663952986699	11.3816231940084\\
28.9660859115165	11.3813889119644\\
28.9657765133424	11.3811546236011\\
28.965467104147	11.380920328918\\
28.9651576839294	11.3806860279148\\
28.9648482526888	11.3804517205913\\
28.9645388104244	11.3802174069469\\
28.9642293571356	11.3799830869815\\
28.9639198928214	11.3797487606946\\
28.963610417481	11.379514428086\\
28.9633009311138	11.3792800891552\\
28.9629914337189	11.379045743902\\
28.9626819252956	11.3788113923259\\
28.962372405843	11.3785770344267\\
28.9620628753603	11.3783426702041\\
28.9617533338468	11.3781082996576\\
28.9614437813017	11.3778739227869\\
28.9611342177242	11.3776395395917\\
28.9608246431136	11.3774051500717\\
28.9605150574689	11.3771707542265\\
28.9602054607895	11.3769363520557\\
28.9598958530746	11.3767019435591\\
28.9595862343233	11.3764675287362\\
28.9592766045349	11.3762331075868\\
28.9589669637086	11.3759986801105\\
28.9586573118436	11.3757642463069\\
28.9583476489391	11.3755298061758\\
28.9580379749944	11.3752953597167\\
28.9577282900086	11.3750609069294\\
28.957418593981	11.3748264478135\\
28.9571088869107	11.3745919823686\\
28.956799168797	11.3743575105944\\
28.9564894396392	11.3741230324906\\
28.9561796994363	11.3738885480568\\
28.9558699481876	11.3736540572927\\
28.9555601858924	11.3734195601979\\
28.9552504125499	11.3731850567722\\
28.9549406281591	11.3729505470151\\
28.9546308327195	11.3727160309263\\
28.9543210262301	11.3724815085055\\
28.9540112086903	11.3722469797523\\
28.9537013800991	11.3720124446664\\
28.9533915404559	11.3717779032474\\
28.9530816897597	11.3715433554951\\
28.95277182801	11.371308801409\\
28.9524619552057	11.3710742409889\\
28.9521520713463	11.3708396742343\\
28.9518421764308	11.3706051011449\\
28.9515322704584	11.3703705217205\\
28.9512223534285	11.3701359359606\\
28.9509124253402	11.3699013438649\\
28.9506024861927	11.3696667454331\\
28.9502925359853	11.3694321406648\\
28.949982574717	11.3691975295597\\
28.9496726023872	11.3689629121175\\
28.9493626189951	11.3687282883377\\
28.9490526245399	11.3684936582201\\
28.9487426190207	11.3682590217643\\
28.9484326024368	11.36802437897\\
28.9481225747874	11.3677897298368\\
28.9478125360717	11.3675550743644\\
28.947502486289	11.3673204125525\\
28.9471924254384	11.3670857444006\\
28.9468823535191	11.3668510699085\\
28.9465722705303	11.3666163890758\\
28.9462621764714	11.3663817019022\\
28.9459520713414	11.3661470083874\\
28.9456419551395	11.3659123085309\\
28.9453318278651	11.3656776023324\\
28.9450216895172	11.3654428897917\\
28.9447115400952	11.3652081709083\\
28.9444013795982	11.364973445682\\
28.9440912080254	11.3647387141123\\
28.943781025376	11.364503976199\\
28.9434708316493	11.3642692319416\\
28.9431606268444	11.3640344813399\\
28.9428504109606	11.3637997243935\\
28.942540183997	11.363564961102\\
28.9422299459529	11.3633301914652\\
28.9419196968275	11.3630954154827\\
28.9416094366201	11.362860633154\\
28.9412991653297	11.362625844479\\
28.9409888829556	11.3623910494572\\
28.940678589497	11.3621562480883\\
28.9403682849532	11.361921440372\\
28.9400579693232	11.3616866263079\\
28.9397476426065	11.3614518058956\\
28.9394373048021	11.3612169791349\\
28.9391269559092	11.3609821460254\\
28.9388165959271	11.3607473065667\\
28.938506224855	11.3605124607585\\
28.938195842692	11.3602776086005\\
28.9378854494375	11.3600427500923\\
28.9375750450905	11.3598078852335\\
28.9372646296503	11.3595730140239\\
28.9369542031162	11.3593381364631\\
28.9366437654872	11.3591032525507\\
28.9363333167627	11.3588683622864\\
28.9360228569418	11.3586334656698\\
28.9357123860238	11.3583985627007\\
28.9354019040078	11.3581636533786\\
28.935091410893	11.3579287377032\\
28.9347809066787	11.3576938156743\\
28.934470391364	11.3574588872913\\
28.9341598649482	11.357223952554\\
28.9338493274305	11.3569890114621\\
28.9335387788101	11.3567540640152\\
28.9332282190862	11.356519110213\\
28.9329176482579	11.356284150055\\
28.9326070663246	11.3560491835411\\
28.9322964732853	11.3558142106707\\
28.9319858691394	11.3555792314437\\
28.931675253886	11.3553442458595\\
28.9313646275243	11.355109253918\\
28.9310539900535	11.3548742556187\\
28.9307433414729	11.3546392509614\\
28.9304326817816	11.3544042399456\\
28.9301220109788	11.354169222571\\
28.9298113290638	11.3539341988373\\
28.9295006360357	11.3536991687441\\
28.9291899318938	11.3534641322911\\
28.9288792166373	11.353229089478\\
28.9285684902653	11.3529940403043\\
28.9282577527771	11.3527589847698\\
28.9279470041719	11.3525239228741\\
28.9276362444488	11.3522888546169\\
28.9273254736072	11.3520537799978\\
28.9270146916461	11.3518186990165\\
28.9267038985648	11.3515836116726\\
28.9263930943626	11.3513485179658\\
28.9260822790385	11.3511134178957\\
28.9257714525918	11.350878311462\\
28.9254606150217	11.3506431986644\\
28.9251497663275	11.3504080795025\\
28.9248389065083	11.350172953976\\
28.9245280355633	11.3499378220845\\
28.9242171534917	11.3497026838276\\
28.9239062602927	11.3494675392051\\
28.9235953559656	11.3492323882166\\
28.9232844405095	11.3489972308617\\
28.9229735139237	11.3487620671401\\
28.9226625762073	11.3485268970515\\
28.9223516273595	11.3482917205955\\
28.9220406673796	11.3480565377717\\
28.9217296962667	11.3478213485798\\
28.9214187140201	11.3475861530196\\
28.9211077206389	11.3473509510905\\
28.9207967161224	11.3471157427924\\
28.9204857004698	11.3468805281247\\
28.9201746736802	11.3466453070873\\
28.9198636357529	11.3464100796797\\
28.9195525866871	11.3461748459017\\
28.9192415264819	11.3459396057527\\
28.9189304551366	11.3457043592326\\
28.9186193726504	11.345469106341\\
28.9183082790225	11.3452338470775\\
28.917997174252	11.3449985814418\\
28.9176860583382	11.3447633094334\\
28.9173749312803	11.3445280310522\\
28.9170637930775	11.3442927462978\\
28.916752643729	11.3440574551697\\
28.9164414832339	11.3438221576676\\
28.9161303115916	11.3435868537913\\
28.9158191288011	11.3433515435404\\
28.9155079348617	11.3431162269144\\
28.9151967297726	11.3428809039131\\
28.914885513533	11.3426455745362\\
28.9145742861421	11.3424102387832\\
28.9142630475991	11.3421748966539\\
28.9139517979032	11.3419395481478\\
28.9136405370536	11.3417041932647\\
28.9133292650495	11.3414688320042\\
28.91301798189	11.3412334643659\\
28.9127066875745	11.3409980903496\\
28.9123953821021	11.3407627099548\\
28.912084065472	11.3405273231812\\
28.9117727376833	11.3402919300285\\
28.9114613987354	11.3400565304963\\
28.9111500486273	11.3398211245843\\
28.9108386873584	11.3395857122921\\
28.9105273149278	11.3393502936194\\
28.9102159313346	11.3391148685658\\
28.9099045365782	11.3388794371311\\
28.9095931306576	11.3386439993148\\
28.9092817135722	11.3384085551165\\
28.908970285321	11.3381731045361\\
28.9086588459033	11.337937647573\\
28.9083473953184	11.337702184227\\
28.9080359335653	11.3374667144977\\
28.9077244606433	11.3372312383848\\
28.9074129765516	11.336995755888\\
28.9071014812894	11.3367602670068\\
28.9067899748559	11.3365247717409\\
28.9064784572502	11.33628927009\\
28.9061669284717	11.3360537620538\\
28.9058553885194	11.3358182476318\\
28.9055438373927	11.3355827268238\\
28.9052322750906	11.3353471996294\\
28.9049207016123	11.3351116660483\\
28.9046091169572	11.3348761260801\\
28.9042975211243	11.3346405797244\\
28.903985914113	11.3344050269809\\
28.9036742959222	11.3341694678493\\
28.9033626665514	11.3339339023293\\
28.9030510259996	11.3336983304204\\
28.9027393742661	11.3334627521223\\
28.90242771135	11.3332271674348\\
28.9021160372506	11.3329915763573\\
28.9018043519671	11.3327559788897\\
28.9014926554986	11.3325203750315\\
28.9011809478444	11.3322847647823\\
28.9008692290036	11.332049148142\\
28.9005574989755	11.33181352511\\
28.9002457577592	11.3315778956861\\
28.8999340053539	11.3313422598699\\
28.8996222417589	11.3311066176611\\
28.8993104669733	11.3308709690593\\
28.8989986809964	11.3306353140641\\
28.8986868838272	11.3303996526753\\
28.8983750754651	11.3301639848925\\
28.8980632559092	11.3299283107152\\
28.8977514251588	11.3296926301433\\
28.8974395832129	11.3294569431763\\
28.8971277300709	11.3292212498139\\
28.8968158657318	11.3289855500557\\
28.896503990195	11.3287498439014\\
28.8961921034595	11.3285141313507\\
28.8958802055246	11.3282784124031\\
28.8955682963896	11.3280426870584\\
28.8952563760535	11.3278069553162\\
28.8949444445156	11.3275712171762\\
28.894632501775	11.327335472638\\
28.894320547831	11.3270997217012\\
28.8940085826828	11.3268639643656\\
28.8936966063296	11.3266282006307\\
28.8933846187705	11.3263924304962\\
28.8930726200047	11.3261566539618\\
28.8927606100315	11.3259208710272\\
28.8924485888501	11.3256850816919\\
28.8921365564595	11.3254492859556\\
28.8918245128591	11.3252134838181\\
28.8915124580481	11.3249776752788\\
28.8912003920255	11.3247418603376\\
28.8908883147907	11.324506038994\\
28.8905762263427	11.3242702112476\\
28.8902641266809	11.3240343770983\\
28.8899520158043	11.3237985365455\\
28.8896398937122	11.3235626895889\\
28.8893277604039	11.3233268362283\\
28.8890156158784	11.3230909764632\\
28.8887034601349	11.3228551102934\\
28.8883912931727	11.3226192377184\\
28.888079114991	11.3223833587379\\
28.887766925589	11.3221474733515\\
28.8874547249657	11.321911581559\\
28.8871425131206	11.3216756833599\\
28.8868302900526	11.321439778754\\
28.8865180557611	11.3212038677408\\
28.8862058102451	11.32096795032\\
28.885893553504	11.3207320264914\\
28.8855812855369	11.3204960962544\\
28.885269006343	11.3202601596088\\
28.8849567159215	11.3200242165543\\
28.8846444142716	11.3197882670904\\
28.8843321013924	11.3195523112169\\
28.8840197772832	11.3193163489333\\
28.8837074419431	11.3190803802394\\
28.8833950953715	11.3188444051348\\
28.8830827375673	11.3186084236191\\
28.8827703685299	11.318372435692\\
28.8824579882584	11.3181364413532\\
28.882145596752	11.3179004406022\\
28.8818331940099	11.3176644334388\\
28.8815207800314	11.3174284198626\\
28.8812083548155	11.3171923998732\\
28.8808959183615	11.3169563734704\\
28.8805834706686	11.3167203406536\\
28.880271011736	11.3164843014227\\
28.8799585415628	11.3162482557773\\
28.8796460601483	11.3160122037169\\
28.8793335674916	11.3157761452413\\
28.8790210635919	11.3155400803502\\
28.8787085484485	11.315304009043\\
28.8783960220605	11.3150679313196\\
28.8780834844271	11.3148318471796\\
28.8777709355474	11.3145957566225\\
28.8774583754208	11.3143596596482\\
28.8771458040464	11.3141235562561\\
28.8768332214233	11.313887446446\\
28.8765206275507	11.3136513302176\\
28.876208022428	11.3134152075704\\
28.8758954060541	11.3131790785041\\
28.8755827784284	11.3129429430184\\
28.87527013955	11.3127068011129\\
28.8749574894181	11.3124706527873\\
28.8746448280319	11.3122344980412\\
28.8743321553906	11.3119983368743\\
28.8740194714934	11.3117621692862\\
28.8737067763395	11.3115259952766\\
28.873394069928	11.3112898148451\\
28.8730813522581	11.3110536279914\\
28.8727686233291	11.3108174347151\\
28.8724558831401	11.3105812350159\\
28.8721431316903	11.3103450288934\\
28.8718303689789	11.3101088163473\\
28.8715175950052	11.3098725973772\\
28.8712048097681	11.3096363719829\\
28.8708920132671	11.3094001401638\\
28.8705792055012	11.3091639019198\\
28.8702663864697	11.3089276572503\\
28.8699535561717	11.3086914061552\\
28.8696407146064	11.308455148634\\
28.8693278617731	11.3082188846863\\
28.8690149976708	11.3079826143119\\
28.8687021222988	11.3077463375104\\
28.8683892356563	11.3075100542815\\
28.8680763377425	11.3072737646247\\
28.8677634285565	11.3070374685397\\
28.8674505080976	11.3068011660263\\
28.8671375763649	11.3065648570839\\
28.8668246333576	11.3063285417124\\
28.8665116790749	11.3060922199113\\
28.866198713516	11.3058558916803\\
28.8658857366801	11.305619557019\\
28.8655727485663	11.3053832159271\\
28.8652597491739	11.3051468684043\\
28.864946738502	11.3049105144501\\
28.8646337165499	11.3046741540643\\
28.8643206833167	11.3044377872464\\
28.8640076388015	11.3042014139962\\
28.8636945830037	11.3039650343133\\
28.8633815159223	11.3037286481973\\
28.8630684375566	11.3034922556479\\
28.8627553479058	11.3032558566648\\
28.8624422469689	11.3030194512475\\
28.8621291347453	11.3027830393958\\
28.8618160112341	11.3025466211092\\
28.8615028764345	11.3023101963875\\
28.8611897303457	11.3020737652303\\
28.8608765729668	11.3018373276372\\
28.8605634042971	11.3016008836079\\
28.8602502243357	11.301364433142\\
28.8599370330818	11.3011279762392\\
28.8596238305347	11.3008915128992\\
28.8593106166934	11.3006550431215\\
28.8589973915572	11.3004185669059\\
28.8586841551253	11.3001820842519\\
28.8583709073968	11.2999455951593\\
28.858057648371	11.2997090996276\\
28.857744378047	11.2994725976566\\
28.857431096424	11.2992360892459\\
28.8571178035011	11.2989995743951\\
28.8568044992777	11.2987630531038\\
28.8564911837528	11.2985265253718\\
28.8561778569256	11.2982899911987\\
28.8558645187954	11.2980534505841\\
28.8555511693613	11.2978169035277\\
28.8552378086224	11.2975803500291\\
28.8549244365781	11.2973437900879\\
28.8546110532274	11.2971072237039\\
28.8542976585696	11.2968706508767\\
28.8539842526038	11.2966340716058\\
28.8536708353292	11.2963974858911\\
28.8533574067451	11.2961608937321\\
28.8530439668505	11.2959242951284\\
28.8527305156446	11.2956876900797\\
28.8524170531268	11.2954510785857\\
28.852103579296	11.2952144606461\\
28.8517900941516	11.2949778362604\\
28.8514765976927	11.2947412054283\\
28.8511630899185	11.2945045681494\\
28.8508495708281	11.2942679244235\\
28.8505360404208	11.2940312742501\\
28.8502224986957	11.2937946176289\\
28.849908945652	11.2935579545596\\
28.849595381289	11.2933212850418\\
28.8492818056057	11.2930846090751\\
28.8489682186014	11.2928479266593\\
28.8486546202752	11.2926112377938\\
28.8483410106264	11.2923745424785\\
28.848027389654	11.2921378407129\\
28.8477137573574	11.2919011324968\\
28.8474001137357	11.2916644178296\\
28.847086458788	11.2914276967112\\
28.8467727925135	11.2911909691411\\
28.8464591149115	11.290954235119\\
28.8461454259811	11.2907174946445\\
28.8458317257214	11.2904807477173\\
28.8455180141318	11.290243994337\\
28.8452042912112	11.2900072345033\\
28.844890556959	11.2897704682159\\
28.8445768113744	11.2895336954743\\
28.8442630544564	11.2892969162782\\
28.8439492862043	11.2890601306273\\
28.8436355066172	11.2888233385213\\
28.8433217156944	11.2885865399597\\
28.843007913435	11.2883497349422\\
28.8426940998382	11.2881129234685\\
28.8423802749032	11.2878761055382\\
28.8420664386291	11.2876392811509\\
28.8417525910152	11.2874024503064\\
28.8414387320606	11.2871656130042\\
28.8411248617645	11.2869287692441\\
28.8408109801261	11.2866919190255\\
28.8404970871445	11.2864550623483\\
28.840183182819	11.2862181992121\\
28.8398692671488	11.2859813296164\\
28.8395553401329	11.2857444535609\\
28.8392414017706	11.2855075710454\\
28.838927452061	11.2852706820694\\
28.8386134910034	11.2850337866326\\
28.8382995185969	11.2847968847346\\
28.8379855348408	11.2845599763751\\
28.8376715397341	11.2843230615537\\
28.837357533276	11.2840861402701\\
28.8370435154658	11.283849212524\\
28.8367294863027	11.2836122783149\\
28.8364154457857	11.2833753376425\\
28.8361013939141	11.2831383905064\\
28.835787330687	11.2829014369064\\
28.8354732561037	11.2826644768421\\
28.8351591701633	11.2824275103131\\
28.834845072865	11.282190537319\\
28.8345309642079	11.2819535578595\\
28.8342168441913	11.2817165719343\\
28.8339027128143	11.2814795795429\\
28.8335885700762	11.2812425806851\\
28.833274415976	11.2810055753605\\
28.832960250513	11.2807685635688\\
28.8326460736863	11.2805315453095\\
28.8323318854951	11.2802945205824\\
28.8320176859386	11.280057489387\\
28.831703475016	11.2798204517231\\
28.8313892527265	11.2795834075902\\
28.8310750190691	11.279346356988\\
28.8307607740432	11.2791092999163\\
28.8304465176479	11.2788722363745\\
28.8301322498823	11.2786351663624\\
28.8298179707457	11.2783980898796\\
28.8295036802371	11.2781610069257\\
28.8291893783559	11.2779239175005\\
28.8288750651011	11.2776868216035\\
28.828560740472	11.2774497192344\\
28.8282464044677	11.2772126103928\\
28.8279320570874	11.2769754950784\\
28.8276176983302	11.2767383732909\\
28.8273033281954	11.2765012450298\\
28.8269889466822	11.2762641102949\\
28.8266745537896	11.2760269690857\\
28.826360149517	11.275789821402\\
28.8260457338633	11.2755526672433\\
28.825731306828	11.2753155066094\\
28.82541686841	11.2750783394997\\
28.8251024186086	11.2748411659141\\
28.8247879574229	11.2746039858522\\
28.8244734848523	11.2743667993136\\
28.8241590008957	11.2741296062978\\
28.8238445055524	11.2738924068047\\
28.8235299988215	11.2736552008339\\
28.8232154807023	11.2734179883849\\
28.8229009511939	11.2731807694574\\
28.8225864102955	11.2729435440511\\
28.8222718580063	11.2727063121656\\
28.8219572943254	11.2724690738006\\
28.821642719252	11.2722318289557\\
28.8213281327853	11.2719945776306\\
28.8210135349244	11.2717573198249\\
28.8206989256686	11.2715200555382\\
28.820384305017	11.2712827847703\\
28.8200696729688	11.2710455075206\\
28.8197550295231	11.270808223789\\
28.8194403746792	11.270570933575\\
28.8191257084362	11.2703336368782\\
28.8188110307933	11.2700963336984\\
28.8184963417496	11.2698590240352\\
28.8181816413044	11.2696217078882\\
28.8178669294567	11.2693843852571\\
28.8175522062058	11.2691470561414\\
28.8172374715509	11.268909720541\\
28.8169227254911	11.2686723784553\\
28.8166079680256	11.268435029884\\
28.8162931991536	11.2681976748269\\
28.8159784188742	11.2679603132835\\
28.8156636271867	11.2677229452535\\
28.8153488240901	11.2674855707365\\
28.8150340095837	11.2672481897321\\
28.8147191836666	11.2670108022401\\
28.8144043463381	11.2667734082601\\
28.8140894975972	11.2665360077917\\
28.8137746374432	11.2662986008345\\
28.8134597658752	11.2660611873882\\
28.8131448828924	11.2658237674525\\
28.8128299884939	11.265586341027\\
28.8125150826791	11.2653489081113\\
28.8122001654469	11.2651114687051\\
28.8118852367966	11.264874022808\\
28.8115702967274	11.2646365704196\\
28.8112553452384	11.2643991115397\\
28.8109403823288	11.2641616461679\\
28.8106254079978	11.2639241743038\\
28.8103104222446	11.263686695947\\
28.8099954250682	11.2634492110972\\
28.809680416468	11.263211719754\\
28.809365396443	11.2629742219172\\
28.8090503649924	11.2627367175862\\
28.8087353221155	11.2624992067609\\
28.8084202678113	11.2622616894408\\
28.8081052020791	11.2620241656255\\
28.807790124918	11.2617866353147\\
28.8074750363271	11.2615490985081\\
28.8071599363058	11.2613115552053\\
28.806844824853	11.261074005406\\
28.8065297019681	11.2608364491097\\
28.8062145676501	11.2605988863162\\
28.8058994218983	11.260361317025\\
28.8055842647118	11.2601237412359\\
28.8052690960897	11.2598861589484\\
28.8049539160313	11.2596485701623\\
28.8046387245358	11.2594109748771\\
28.8043235216022	11.2591733730925\\
28.8040083072298	11.2589357648081\\
28.8036930814177	11.2586981500236\\
28.8033778441651	11.2584605287387\\
28.8030625954712	11.258222900953\\
28.8027473353351	11.257985266666\\
28.8024320637561	11.2577476258776\\
28.8021167807332	11.2575099785872\\
28.8018014862656	11.2572723247946\\
28.8014861803526	11.2570346644994\\
28.8011708629933	11.2567969977013\\
28.8008555341868	11.2565593243998\\
28.8005401939323	11.2563216445947\\
28.8002248422291	11.2560839582855\\
28.7999094790762	11.255846265472\\
28.7995941044728	11.2556085661537\\
28.7992787184181	11.2553708603304\\
28.7989633209113	11.2551331480016\\
28.7986479119515	11.254895429167\\
28.7983324915379	11.2546577038262\\
28.7980170596697	11.2544199719789\\
28.797701616346	11.2541822336247\\
28.7973861615661	11.2539444887633\\
28.797070695329	11.2537067373943\\
28.7967552176339	11.2534689795174\\
28.7964397284801	11.2532312151322\\
28.7961242278666	11.2529934442383\\
28.7958087157927	11.2527556668354\\
28.7954931922575	11.2525178829232\\
28.7951776572601	11.2522800925012\\
28.7948621107999	11.2520422955692\\
28.7945465528758	11.2518044921267\\
28.7942309834871	11.2515666821734\\
28.793915402633	11.251328865709\\
28.7935998103125	11.251091042733\\
28.793284206525	11.2508532132452\\
28.7929685912695	11.2506153772452\\
28.7926529645452	11.2503775347326\\
28.7923373263513	11.2501396857071\\
28.7920216766869	11.2499018301683\\
28.7917060155513	11.2496639681158\\
28.7913903429435	11.2494260995494\\
28.7910746588628	11.2491882244686\\
28.7907589633083	11.248950342873\\
28.7904432562791	11.2487124547624\\
28.7901275377746	11.2484745601364\\
28.7898118077937	11.2482366589945\\
28.7894960663357	11.2479987513366\\
28.7891803133997	11.2477608371621\\
28.7888645489849	11.2475229164708\\
28.7885487730905	11.2472849892622\\
28.7882329857157	11.2470470555361\\
28.7879171868595	11.2468091152921\\
28.7876013765213	11.2465711685298\\
28.7872855547	11.2463332152488\\
28.7869697213949	11.2460952554488\\
28.7866538766053	11.2458572891295\\
28.7863380203301	11.2456193162904\\
28.7860221525686	11.2453813369313\\
28.78570627332	11.2451433510518\\
28.7853903825834	11.2449053586515\\
28.785074480358	11.24466735973\\
28.7847585666429	11.244429354287\\
28.7844426414373	11.2441913423222\\
28.7841267047404	11.2439533238351\\
28.7838107565514	11.2437152988255\\
28.7834947968694	11.243477267293\\
28.7831788256935	11.2432392292371\\
28.782862843023	11.2430011846576\\
28.7825468488569	11.2427631335541\\
28.7822308431945	11.2425250759263\\
28.781914826035	11.2422870117737\\
28.7815987973774	11.2420489410961\\
28.781282757221	11.241810863893\\
28.7809667055649	11.2415727801641\\
28.7806506424083	11.2413346899091\\
28.7803345677503	11.2410965931276\\
28.7800184815901	11.2408584898192\\
28.7797023839268	11.2406203799836\\
28.7793862747597	11.2403822636204\\
28.7790701540879	11.2401441407292\\
28.7787540219106	11.2399060113098\\
28.7784378782268	11.2396678753617\\
28.7781217230359	11.2394297328846\\
28.7778055563368	11.2391915838781\\
28.7774893781289	11.2389534283419\\
28.7771731884112	11.2387152662756\\
28.776856987183	11.2384770976789\\
28.7765407744434	11.2382389225514\\
28.7762245501915	11.2380007408927\\
28.7759083144265	11.2377625527024\\
28.7755920671476	11.2375243579803\\
28.7752758083539	11.237286156726\\
28.7749595380446	11.2370479489391\\
28.7746432562189	11.2368097346192\\
28.7743269628759	11.236571513766\\
28.7740106580147	11.2363332863791\\
28.7736943416346	11.2360950524583\\
28.7733780137347	11.235856812003\\
28.7730616743142	11.235618565013\\
28.7727453233722	11.2353803114878\\
28.7724289609079	11.2351420514273\\
28.7721125869204	11.2349037848309\\
28.771796201409	11.2346655116983\\
28.7714798043727	11.2344272320292\\
28.7711633958107	11.2341889458231\\
28.7708469757222	11.2339506530799\\
28.7705305441064	11.233712353799\\
28.7702141009623	11.2334740479801\\
28.7698976462893	11.2332357356229\\
28.7695811800863	11.2329974167271\\
28.7692647023527	11.2327590912921\\
28.7689482130875	11.2325207593178\\
28.7686317122899	11.2322824208037\\
28.768315199959	11.2320440757495\\
28.7679986760941	11.2318057241548\\
28.7676821406943	11.2315673660192\\
28.7673655937587	11.2313290013425\\
28.7670490352865	11.2310906301241\\
28.7667324652769	11.2308522523639\\
28.766415883729	11.2306138680614\\
28.766099290642	11.2303754772162\\
28.765782686015	11.2301370798281\\
28.7654660698472	11.2298986758966\\
28.7651494421378	11.2296602654214\\
28.7648328028858	11.2294218484021\\
28.7645161520906	11.2291834248383\\
28.7641994897512	11.2289449947298\\
28.7638828158667	11.2287065580761\\
28.7635661304365	11.2284681148769\\
28.7632494334595	11.2282296651319\\
28.762932724935	11.2279912088406\\
28.7626160048621	11.2277527460027\\
28.76229927324	11.2275142766178\\
28.7619825300678	11.2272758006857\\
28.7616657753447	11.2270373182059\\
28.7613490090699	11.226798829178\\
28.7610322312425	11.2265603336018\\
28.7607154418616	11.2263218314768\\
28.7603986409265	11.2260833228027\\
28.7600818284362	11.2258448075792\\
28.75976500439	11.2256062858058\\
28.759448168787	11.2253677574822\\
28.7591313216263	11.2251292226081\\
28.7588144629071	11.2248906811831\\
28.7584975926286	11.2246521332068\\
28.75818071079	11.2244135786789\\
28.7578638173903	11.224175017599\\
28.7575469124287	11.2239364499667\\
28.7572299959044	11.2236978757818\\
28.7569130678166	11.2234592950438\\
28.7565961281644	11.2232207077523\\
28.7562791769469	11.2229821139071\\
28.7559622141634	11.2227435135077\\
28.7556452398129	11.2225049065538\\
28.7553282538946	11.2222662930451\\
28.7550112564078	11.2220276729811\\
28.7546942473514	11.2217890463616\\
28.7543772267248	11.2215504131861\\
28.754060194527	11.2213117734542\\
28.7537431507572	11.2210731271658\\
28.7534260954146	11.2208344743203\\
28.7531090284983	11.2205958149174\\
28.7527919500074	11.2203571489567\\
28.7524748599412	11.220118476438\\
28.7521577582987	11.2198797973608\\
28.7518406450792	11.2196411117247\\
28.7515235202818	11.2194024195295\\
28.7512063839056	11.2191637207747\\
28.7508892359498	11.21892501546\\
28.7505720764136	11.218686303585\\
28.7502549052961	11.2184475851494\\
28.7499377225964	11.2182088601528\\
28.7496205283138	11.2179701285949\\
28.7493033224473	11.2177313904752\\
28.7489861049961	11.2174926457935\\
28.7486688759594	11.2172538945493\\
28.7483516353364	11.2170151367423\\
28.7480343831261	11.2167763723722\\
28.7477171193277	11.2165376014386\\
28.7473998439405	11.216298823941\\
28.7470825569635	11.2160600398793\\
28.7467652583959	11.2158212492529\\
28.7464479482368	11.2155824520616\\
28.7461306264855	11.2153436483049\\
28.745813293141	11.2151048379826\\
28.7454959482025	11.2148660210942\\
28.7451785916691	11.2146271976395\\
28.7448612235401	11.2143883676179\\
28.7445438438146	11.2141495310292\\
28.7442264524916	11.2139106878731\\
28.7439090495705	11.2136718381491\\
28.7435916350502	11.2134329818568\\
28.7432742089301	11.213194118996\\
28.7429567712091	11.2129552495663\\
28.7426393218866	11.2127163735673\\
28.7423218609616	11.2124774909986\\
28.7420043884332	11.2122386018599\\
28.7416869043007	11.2119997061508\\
28.7413694085632	11.211760803871\\
28.7410519012198	11.2115218950201\\
28.7407343822698	11.2112829795977\\
28.7404168517121	11.2110440576035\\
28.7400993095461	11.2108051290371\\
28.7397817557708	11.2105661938981\\
28.7394641903854	11.2103272521863\\
28.739146613389	11.2100883039011\\
28.7388290247808	11.2098493490424\\
28.73851142456	11.2096103876096\\
28.7381938127257	11.2093714196025\\
28.737876189277	11.2091324450207\\
28.7375585542132	11.2088934638637\\
28.7372409075332	11.2086544761314\\
28.7369232492364	11.2084154818232\\
28.7366055793219	11.2081764809389\\
28.7362878977887	11.2079374734781\\
28.7359702046361	11.2076984594403\\
28.7356524998632	11.2074594388253\\
28.7353347834691	11.2072204116327\\
28.735017055453	11.2069813778621\\
28.7346993158141	11.2067423375132\\
28.7343815645515	11.2065032905856\\
28.7340638016644	11.2062642370789\\
28.7337460271518	11.2060251769928\\
28.733428241013	11.2057861103269\\
28.7331104432471	11.2055470370808\\
28.7327926338532	11.2053079572543\\
28.7324748128305	11.2050688708469\\
28.7321569801782	11.2048297778582\\
28.7318391358954	11.2045906782879\\
28.7315212799812	11.2043515721357\\
28.7312034124348	11.2041124594012\\
28.7308855332554	11.203873340084\\
28.730567642442	11.2036342141837\\
28.7302497399939	11.2033950817\\
28.7299318259101	11.2031559426325\\
28.7296139001899	11.2029167969809\\
28.7292959628324	11.2026776447449\\
28.7289780138367	11.2024384859239\\
28.7286600532019	11.2021993205177\\
28.7283420809274	11.201960148526\\
28.728024097012	11.2017209699483\\
28.7277061014551	11.2014817847843\\
28.7273880942558	11.2012425930336\\
28.7270700754132	11.2010033946959\\
28.7267520449264	11.2007641897707\\
28.7264340027947	11.2005249782578\\
28.7261159490171	11.2002857601568\\
28.7257978835928	11.2000465354673\\
28.725479806521	11.199807304189\\
28.7251617178007	11.1995680663214\\
28.7248436174312	11.1993288218643\\
28.7245255054116	11.1990895708172\\
28.7242073817411	11.1988503131798\\
28.7238892464187	11.1986110489518\\
28.7235710994437	11.1983717781327\\
28.7232529408151	11.1981325007222\\
28.7229347705322	11.19789321672\\
28.722616588594	11.1976539261257\\
28.7222983949998	11.1974146289389\\
28.7219801897486	11.1971753251592\\
28.7216619728396	11.1969360147863\\
28.721343744272	11.1966966978199\\
28.7210255040448	11.1964573742595\\
28.7207072521574	11.1962180441048\\
28.7203889886087	11.1959787073555\\
28.7200707133979	11.1957393640111\\
28.7197524265243	11.1955000140714\\
28.7194341279868	11.1952606575359\\
28.7191158177847	11.1950212944043\\
28.7187974959172	11.1947819246762\\
28.7184791623833	11.1945425483512\\
28.7181608171822	11.1943031654291\\
28.7178424603131	11.1940637759094\\
28.717524091775	11.1938243797917\\
28.7172057115672	11.1935849770758\\
28.7168873196888	11.1933455677612\\
28.7165689161389	11.1931061518475\\
28.7162505009167	11.1928667293345\\
28.7159320740213	11.1926273002217\\
28.7156136354519	11.1923878645088\\
28.7152951852076	11.1921484221955\\
28.7149767232875	11.1919089732813\\
28.7146582496908	11.1916695177658\\
28.7143397644167	11.1914300556488\\
28.7140212674642	11.1911905869299\\
28.7137027588326	11.1909511116087\\
28.7133842385209	11.1907116296848\\
28.7130657065284	11.1904721411579\\
28.7127471628541	11.1902326460276\\
28.7124286074972	11.1899931442935\\
28.7121100404568	11.1897536359553\\
28.7117914617321	11.1895141210126\\
28.7114728713223	11.1892745994651\\
28.7111542692264	11.1890350713124\\
28.7108356554436	11.1887955365541\\
28.7105170299731	11.1885559951898\\
28.710198392814	11.1883164472192\\
28.7098797439654	11.188076892642\\
28.7095610834265	11.1878373314577\\
28.7092424111964	11.1875977636661\\
28.7089237272743	11.1873581892666\\
28.7086050316593	11.1871186082591\\
28.7082863243506	11.186879020643\\
28.7079676053472	11.1866394264181\\
28.7076488746484	11.186399825584\\
28.7073301322532	11.1861602181402\\
28.7070113781609	11.1859206040866\\
28.7066926123705	11.1856809834226\\
28.7063738348812	11.1854413561479\\
28.7060550456922	11.1852017222622\\
28.7057362448026	11.1849620817651\\
28.7054174322114	11.1847224346562\\
28.705098607918	11.1844827809351\\
28.7047797719213	11.1842431206016\\
28.7044609242206	11.1840034536552\\
28.704142064815	11.1837637800955\\
28.7038231937035	11.1835240999223\\
28.7035043108855	11.183284413135\\
28.70318541636	11.1830447197335\\
28.7028665101261	11.1828050197172\\
28.702547592183	11.1825653130859\\
28.7022286625299	11.1823255998392\\
28.7019097211658	11.1820858799767\\
28.7015907680899	11.181846153498\\
28.7012718033013	11.1816064204028\\
28.7009528267993	11.1813666806907\\
28.7006338385829	11.1811269343614\\
28.7003148386512	11.1808871814144\\
28.6999958270035	11.1806474218495\\
28.6996768036388	11.1804076556662\\
28.6993577685563	11.1801678828642\\
28.6990387217551	11.1799281034432\\
28.6987196632343	11.1796883174027\\
28.6984005929932	11.1794485247423\\
28.6980815110308	11.1792087254618\\
28.6977624173463	11.1789689195608\\
28.6974433119388	11.1787291070388\\
28.6971241948075	11.1784892878956\\
28.6968050659515	11.1782494621308\\
28.6964859253699	11.1780096297439\\
28.6961667730619	11.1777697907347\\
28.6958476090265	11.1775299451027\\
28.6955284332631	11.1772900928477\\
28.6952092457706	11.1770502339691\\
28.6948900465482	11.1768103684668\\
28.6945708355952	11.1765704963402\\
28.6942516129105	11.1763306175891\\
28.6939323784933	11.1760907322131\\
28.6936131323428	11.1758508402117\\
28.6932938744582	11.1756109415847\\
28.6929746048385	11.1753710363317\\
28.6926553234828	11.1751311244523\\
28.6923360303905	11.1748912059461\\
28.6920167255604	11.1746512808128\\
28.6916974089919	11.174411349052\\
28.691378080684	11.1741714106633\\
28.6910587406358	11.1739314656464\\
28.6907393888466	11.173691514001\\
28.6904200253155	11.1734515557265\\
28.6901006500415	11.1732115908228\\
28.6897812630238	11.1729716192893\\
28.6894618642616	11.1727316411258\\
28.689142453754	11.1724916563319\\
28.6888230315001	11.1722516649072\\
28.688503597499	11.1720116668514\\
28.68818415175	11.171771662164\\
28.6878646942521	11.1715316508447\\
28.6875452250045	11.1712916328932\\
28.6872257440063	11.1710516083091\\
28.6869062512566	11.1708115770919\\
28.6865867467546	11.1705715392414\\
28.6862672304994	11.1703314947572\\
28.6859477024902	11.170091443639\\
28.685628162726	11.1698513858862\\
28.6853086112061	11.1696113214986\\
28.6849890479295	11.1693712504759\\
28.6846694728954	11.1691311728176\\
28.684349886103	11.1688910885234\\
28.6840302875512	11.1686509975929\\
28.6837106772394	11.1684109000257\\
28.6833910551666	11.1681707958215\\
28.683071421332	11.1679306849799\\
28.6827517757347	11.1676905675006\\
28.6824321183738	11.1674504433831\\
28.6821124492485	11.1672103126272\\
28.6817927683578	11.1669701752324\\
28.681473075701	11.1667300311984\\
28.6811533712772	11.1664898805248\\
28.6808336550854	11.1662497232112\\
28.6805139271249	11.1660095592573\\
28.6801941873948	11.1657693886627\\
28.6798744358942	11.165529211427\\
28.6795546726222	11.16528902755\\
28.6792348975779	11.1650488370311\\
28.6789151107606	11.1648086398701\\
28.6785953121693	11.1645684360665\\
28.6782755018031	11.1643282256201\\
28.6779556796613	11.1640880085303\\
28.6776358457429	11.163847784797\\
28.6773160000471	11.1636075544196\\
28.6769961425729	11.1633673173979\\
28.6766762733196	11.1631270737315\\
28.6763563922863	11.1628868234199\\
28.676036499472	11.1626465664629\\
28.675716594876	11.1624063028601\\
28.6753966784973	11.162166032611\\
28.6750767503351	11.1619257557154\\
28.6747568103886	11.1616854721728\\
28.6744368586568	11.1614451819829\\
28.6741168951389	11.1612048851454\\
28.673796919834	11.1609645816598\\
28.6734769327412	11.1607242715258\\
28.6731569338598	11.160483954743\\
28.6728369231887	11.160243631311\\
28.6725169007272	11.1600033012296\\
28.6721968664744	11.1597629644983\\
28.6718768204293	11.1595226211167\\
28.6715567625912	11.1592822710845\\
28.6712366929592	11.1590419144013\\
28.6709166115324	11.1588015510668\\
28.6705965183099	11.1585611810805\\
28.6702764132909	11.1583208044421\\
28.6699562964744	11.1580804211513\\
28.6696361678597	11.1578400312077\\
28.6693160274458	11.1575996346108\\
28.6689958752319	11.1573592313604\\
28.6686757112171	11.1571188214561\\
28.6683555354006	11.1568784048975\\
28.6680353477814	11.1566379816842\\
28.6677151483588	11.1563975518158\\
28.6673949371317	11.1561571152921\\
28.6670747140994	11.1559166721126\\
28.6667544792611	11.1556762222769\\
28.6664342326157	11.1554357657847\\
28.6661139741625	11.1551953026357\\
28.6657937039005	11.1549548328294\\
28.665473421829	11.1547143563654\\
28.665153127947	11.1544738732435\\
28.6648328222536	11.1542333834633\\
28.6645125047481	11.1539928870243\\
28.6641921754295	11.1537523839262\\
28.6638718342969	11.1535118741687\\
28.6635514813495	11.1532713577513\\
28.6632311165864	11.1530308346737\\
28.6629107400068	11.1527903049356\\
28.6625903516097	11.1525497685365\\
28.6622699513943	11.1523092254762\\
28.6619495393598	11.1520686757541\\
28.6616291155051	11.15182811937\\
28.6613086798296	11.1515875563235\\
28.6609882323323	11.1513469866142\\
28.6606677730123	11.1511064102418\\
28.6603473018687	11.1508658272058\\
28.6600268189008	11.150625237506\\
28.6597063241075	11.1503846411419\\
28.6593858174881	11.1501440381132\\
28.6590652990417	11.1499034284194\\
28.6587447687673	11.1496628120603\\
28.6584242266642	11.1494221890355\\
28.6581036727315	11.1491815593446\\
28.6577831069682	11.1489409229872\\
28.6574625293735	11.1487002799629\\
28.6571419399466	11.1484596302715\\
28.6568213386865	11.1482189739124\\
28.6565007255924	11.1479783108855\\
28.6561801006634	11.1477376411901\\
28.6558594638986	11.1474969648261\\
28.6555388152972	11.1472562817931\\
28.6552181548583	11.1470155920906\\
28.654897482581	11.1467748957183\\
28.6545767984644	11.1465341926758\\
28.6542561025077	11.1462934829628\\
28.65393539471	11.1460527665789\\
28.6536146750705	11.1458120435238\\
28.6532939435881	11.1455713137969\\
28.6529732002621	11.1453305773981\\
28.6526524450917	11.1450898343268\\
28.6523316780758	11.1448490845829\\
28.6520108992137	11.1446083281657\\
28.6516901085044	11.1443675650751\\
28.6513693059472	11.1441267953106\\
28.651048491541	11.1438860188719\\
28.6507276652851	11.1436452357586\\
28.6504068271786	11.1434044459703\\
28.6500859772205	11.1431636495066\\
28.6497651154101	11.1429228463673\\
28.6494442417464	11.1426820365518\\
28.6491233562286	11.1424412200599\\
28.6488024588558	11.1422003968911\\
28.6484815496271	11.1419595670452\\
28.6481606285416	11.1417187305217\\
28.6478396955985	11.1414778873202\\
28.6475187507968	11.1412370374405\\
28.6471977941358	11.1409961808821\\
28.6468768256145	11.1407553176446\\
28.6465558452321	11.1405144477277\\
28.6462348529876	11.140273571131\\
28.6459138488803	11.1400326878541\\
28.6455928329091	11.1397917978967\\
28.6452718050733	11.1395509012584\\
28.644950765372	11.1393099979389\\
28.6446297138043	11.1390690879376\\
28.6443086503693	11.1388281712544\\
28.6439875750661	11.1385872478888\\
28.6436664878939	11.1383463178404\\
28.6433453888518	11.138105381109\\
28.6430242779388	11.137864437694\\
28.6427031551543	11.1376234875951\\
28.6423820204971	11.137382530812\\
28.6420608739665	11.1371415673443\\
28.6417397155617	11.1369005971916\\
28.6414185452816	11.1366596203536\\
28.6410973631255	11.1364186368298\\
28.6407761690924	11.13617764662\\
28.6404549631815	11.1359366497237\\
28.640133745392	11.1356956461405\\
28.6398125157228	11.1354546358701\\
28.6394912741732	11.1352136189122\\
28.6391700207423	11.1349725952663\\
28.6388487554291	11.1347315649321\\
28.6385274782329	11.1344905279092\\
28.6382061891527	11.1342494841973\\
28.6378848881876	11.1340084337959\\
28.6375635753368	11.1337673767047\\
28.6372422505994	11.1335263129233\\
28.6369209139745	11.1332852424514\\
28.6365995654613	11.1330441652885\\
28.6362782050588	11.1328030814344\\
28.6359568327661	11.1325619908886\\
28.6356354485825	11.1323208936508\\
28.635314052507	11.1320797897205\\
28.6349926445387	11.1318386790975\\
28.6346712246767	11.1315975617814\\
28.6343497929203	11.1313564377717\\
28.6340283492684	11.1311153070681\\
28.6337068937202	11.1308741696703\\
28.6333854262749	11.1306330255778\\
28.6330639469315	11.1303918747903\\
28.6327424556892	11.1301507173075\\
28.6324209525471	11.1299095531289\\
28.6320994375043	11.1296683822542\\
28.6317779105599	11.129427204683\\
28.6314563717131	11.1291860204149\\
28.6311348209629	11.1289448294495\\
28.6308132583085	11.1287036317866\\
28.630491683749	11.1284624274257\\
28.6301700972836	11.1282212163664\\
28.6298484989112	11.1279799986084\\
28.6295268886311	11.1277387741514\\
28.6292052664425	11.1274975429948\\
28.6288836323442	11.1272563051384\\
28.6285619863357	11.1270150605818\\
28.6282403284158	11.1267738093246\\
28.6279186585838	11.1265325513664\\
28.6275969768387	11.1262912867069\\
28.6272752831797	11.1260500153457\\
28.626953577606	11.1258087372825\\
28.6266318601165	11.1255674525168\\
28.6263101307105	11.1253261610482\\
28.625988389387	11.1250848628765\\
28.6256666361452	11.1248435580012\\
28.6253448709842	11.124602246422\\
28.6250230939031	11.1243609281384\\
28.624701304901	11.1241196031502\\
28.624379503977	11.1238782714569\\
28.6240576911303	11.1236369330582\\
28.62373586636	11.1233955879537\\
28.6234140296651	11.123154236143\\
28.6230921810449	11.1229128776258\\
28.6227703204984	11.1226715124017\\
28.6224484480247	11.1224301404702\\
28.6221265636229	11.1221887618311\\
28.6218046672922	11.121947376484\\
28.6214827590317	11.1217059844285\\
28.6211608388405	11.1214645856641\\
28.6208389067177	11.1212231801906\\
28.6205169626624	11.1209817680076\\
28.6201950066738	11.1207403491147\\
28.619873038751	11.1204989235115\\
28.619551058893	11.1202574911977\\
28.619229067099	11.1200160521729\\
28.6189070633681	11.1197746064366\\
28.6185850476994	11.1195331539886\\
28.6182630200921	11.1192916948284\\
28.6179409805452	11.1190502289558\\
28.6176189290578	11.1188087563702\\
28.6172968656292	11.1185672770714\\
28.6169747902583	11.118325791059\\
28.6166527029444	11.1180842983325\\
28.6163306036864	11.1178427988917\\
28.6160084924836	11.1176012927361\\
28.6156863693351	11.1173597798654\\
28.6153642342399	11.1171182602792\\
28.6150420871972	11.1168767339771\\
28.6147199282061	11.1166352009588\\
28.6143977572656	11.1163936612238\\
28.6140755743751	11.1161521147719\\
28.6137533795334	11.1159105616026\\
28.6134311727398	11.1156690017155\\
28.6131089539933	11.1154274351104\\
28.6127867232931	11.1151858617867\\
28.6124644806383	11.1149442817442\\
28.612142226028	11.1147026949824\\
28.6118199594613	11.1144611015011\\
28.6114976809373	11.1142195012997\\
28.6111753904552	11.113977894378\\
28.610853088014	11.1137362807356\\
28.6105307736128	11.1134946603721\\
28.6102084472508	11.113253033287\\
28.6098861089272	11.1130113994802\\
28.6095637586409	11.1127697589511\\
28.6092413963911	11.1125281116994\\
28.608919022177	11.1122864577247\\
28.6085966359975	11.1120447970267\\
28.608274237852	11.1118031296049\\
28.6079518277393	11.1115614554591\\
28.6076294056588	11.1113197745888\\
28.6073069716094	11.1110780869936\\
28.6069845255903	11.1108363926732\\
28.6066620676006	11.1105946916273\\
28.6063395976395	11.1103529838553\\
28.6060171157059	11.110111269357\\
28.6056946217991	11.109869548132\\
28.6053721159182	11.10962782018\\
28.6050495980622	11.1093860855004\\
28.6047270682302	11.109144344093\\
28.6044045264214	11.1089025959574\\
28.604081972635	11.1086608410932\\
28.6037594068699	11.1084190795\\
28.6034368291253	11.1081773111775\\
28.6031142394004	11.1079355361253\\
28.6027916376942	11.107693754343\\
28.6024690240058	11.1074519658302\\
28.6021463983344	11.1072101705866\\
28.601823760679	11.1069683686117\\
28.6015011110388	11.1067265599053\\
28.6011784494129	11.106484744467\\
28.6008557758003	11.1062429222962\\
28.6005330902003	11.1060010933928\\
28.6002103926118	11.1057592577563\\
28.5998876830341	11.1055174153864\\
28.5995649614662	11.1052755662825\\
28.5992422279073	11.1050337104445\\
28.5989194823564	11.1047918478719\\
28.5985967248126	11.1045499785644\\
28.5982739552751	11.1043081025215\\
28.597951173743	11.1040662197428\\
28.5976283802153	11.1038243302281\\
28.5973055746912	11.103582433977\\
28.5969827571698	11.103340530989\\
28.5966599276503	11.1030986212638\\
28.5963370861316	11.102856704801\\
28.5960142326129	11.1026147816002\\
28.5956913670934	11.1023728516611\\
28.5953684895721	11.1021309149833\\
28.5950456000482	11.1018889715664\\
28.5947226985207	11.10164702141\\
28.5943997849887	11.1014050645138\\
28.5940768594514	11.1011631008774\\
28.5937539219079	11.1009211305004\\
28.5934309723573	11.1006791533824\\
28.5931080107986	11.100437169523\\
28.5927850372311	11.100195178922\\
28.5924620516537	11.0999531815789\\
28.5921390540657	11.0997111774932\\
28.591816044466	11.0994691666648\\
28.5914930228539	11.0992271490931\\
28.5911699892284	11.0989851247778\\
28.5908469435887	11.0987430937186\\
28.5905238859337	11.098501055915\\
28.5902008162628	11.0982590113667\\
28.5898777345748	11.0980169600733\\
28.5895546408691	11.0977749020344\\
28.5892315351446	11.0975328372496\\
28.5889084174004	11.0972907657187\\
28.5885852876358	11.0970486874411\\
28.5882621458497	11.0968066024165\\
28.5879389920413	11.0965645106446\\
28.5876158262097	11.096322412125\\
28.5872926483539	11.0960803068572\\
28.5869694584732	11.095838194841\\
28.5866462565666	11.0955960760759\\
28.5863230426332	11.0953539505615\\
28.5859998166722	11.0951118182976\\
28.5856765786825	11.0948696792836\\
28.5853533286634	11.0946275335193\\
28.5850300666139	11.0943853810043\\
28.5847067925331	11.0941432217381\\
28.5843835064202	11.0939010557204\\
28.5840602082742	11.0936588829509\\
28.5837368980943	11.0934167034291\\
28.5834135758795	11.0931745171547\\
28.5830902416289	11.0929323241273\\
28.5827668953418	11.0926901243465\\
28.5824435370171	11.092447917812\\
28.5821201666539	11.0922057045233\\
28.5817967842515	11.0919634844801\\
28.5814733898088	11.091721257682\\
28.581149983325	11.0914790241287\\
28.5808265647991	11.0912367838197\\
28.5805031342304	11.0909945367547\\
28.5801796916178	11.0907522829334\\
28.5798562369605	11.0905100223552\\
28.5795327702577	11.0902677550199\\
28.5792092915083	11.0900254809271\\
28.5788858007115	11.0897832000764\\
28.5785622978664	11.0895409124674\\
28.5782387829721	11.0892986180997\\
28.5779152560278	11.0890563169731\\
28.5775917170324	11.088814009087\\
28.5772681659851	11.0885716944411\\
28.5769446028851	11.0883293730351\\
28.5766210277314	11.0880870448685\\
28.5762974405231	11.087844709941\\
28.5759738412593	11.0876023682522\\
28.5756502299391	11.0873600198017\\
28.5753266065616	11.0871176645892\\
28.575002971126	11.0868753026143\\
28.5746793236313	11.0866329338766\\
28.5743556640766	11.0863905583757\\
28.5740319924611	11.0861481761112\\
28.5737083087838	11.0859057870828\\
28.5733846130438	11.08566339129\\
28.5730609052402	11.0854209887326\\
28.5727371853722	11.0851785794101\\
28.5724134534388	11.0849361633222\\
28.5720897094391	11.0846937404685\\
28.5717659533723	11.0844513108485\\
28.5714421852374	11.084208874462\\
28.5711184050335	11.0839664313085\\
28.5707946127598	11.0837239813876\\
28.5704708084153	11.0834815246991\\
28.5701469919991	11.0832390612425\\
28.5698231635104	11.0829965910174\\
28.5694993229482	11.0827541140234\\
28.5691754703116	11.0825116302603\\
28.5688516055998	11.0822691397275\\
28.5685277288118	11.0820266424247\\
28.5682038399467	11.0817841383516\\
28.5678799390036	11.0815416275078\\
28.5675560259817	11.0812991098928\\
28.5672321008801	11.0810565855064\\
28.5669081636977	11.080814054348\\
28.5665842144338	11.0805715164175\\
28.5662602530874	11.0803289717143\\
28.5659362796576	11.0800864202381\\
28.5656122941435	11.0798438619885\\
28.5652882965443	11.0796012969651\\
28.564964286859	11.0793587251676\\
28.5646402650867	11.0791161465956\\
28.5643162312265	11.0788735612487\\
28.5639921852775	11.0786309691265\\
28.5636681272389	11.0783883702287\\
28.5633440571096	11.0781457645548\\
28.5630199748889	11.0779031521046\\
28.5626958805758	11.0776605328775\\
28.5623717741693	11.0774179068733\\
28.5620476556687	11.0771752740915\\
28.561723525073	11.0769326345318\\
28.5613993823813	11.0766899881937\\
28.5610752275926	11.0764473350771\\
28.5607510607062	11.0762046751813\\
28.560426881721	11.0759620085061\\
28.5601026906362	11.0757193350511\\
28.5597784874509	11.0754766548159\\
28.5594542721642	11.0752339678001\\
28.5591300447751	11.0749912740033\\
28.5588058052828	11.0747485734253\\
28.5584815536864	11.0745058660655\\
28.558157289985	11.0742631519236\\
28.5578330141776	11.0740204309992\\
28.5575087262633	11.073777703292\\
28.5571844262413	11.0735349688016\\
28.5568601141107	11.0732922275275\\
28.5565357898705	11.0730494794695\\
28.5562114535198	11.0728067246271\\
28.5558871050578	11.0725639629999\\
28.5555627444835	11.0723211945877\\
28.555238371796	11.0720784193899\\
28.5549139869944	11.0718356374062\\
28.5545895900779	11.0715928486363\\
28.5542651810454	11.0713500530797\\
28.5539407598962	11.0711072507361\\
28.5536163266292	11.0708644416051\\
28.5532918812437	11.0706216256863\\
28.5529674237387	11.0703788029794\\
28.5526429541132	11.0701359734839\\
28.5523184723664	11.0698931371995\\
28.5519939784974	11.0696502941258\\
28.5516694725052	11.0694074442624\\
28.551344954389	11.069164587609\\
28.5510204241479	11.0689217241651\\
28.5506958817809	11.0686788539304\\
28.5503713272871	11.0684359769045\\
28.5500467606657	11.068193093087\\
28.5497221819157	11.0679502024776\\
28.5493975910362	11.0677073050758\\
28.5490729880264	11.0674644008813\\
28.5487483728853	11.0672214898938\\
28.5484237456119	11.0669785721127\\
28.5480991062055	11.0667356475378\\
28.547774454665	11.0664927161686\\
28.5474497909896	11.0662497780048\\
28.5471251151784	11.0660068330461\\
28.5468004272305	11.0657638812919\\
28.5464757271449	11.065520922742\\
28.5461510149208	11.0652779573959\\
28.5458262905573	11.0650349852534\\
28.5455015540533	11.0647920063139\\
28.5451768054081	11.0645490205772\\
28.5448520446207	11.0643060280427\\
28.5445272716902	11.0640630287103\\
28.5442024866158	11.0638200225794\\
28.5438776893964	11.0635770096498\\
28.5435528800312	11.0633339899209\\
28.5432280585193	11.0630909633925\\
28.5429032248598	11.0628479300642\\
28.5425783790517	11.0626048899355\\
28.5422535210942	11.0623618430061\\
28.5419286509863	11.0621187892756\\
28.5416037687272	11.0618757287437\\
28.5412788743159	11.0616326614099\\
28.5409539677515	11.0613895872739\\
28.5406290490331	11.0611465063353\\
28.5403041181598	11.0609034185938\\
28.5399791751307	11.0606603240488\\
28.5396542199449	11.0604172227001\\
28.5393292526015	11.0601741145473\\
28.5390042730995	11.0599309995899\\
28.5386792814381	11.0596878778277\\
28.5383542776163	11.0594447492602\\
28.5380292616333	11.059201613887\\
28.537704233488	11.0589584717078\\
28.5373791931797	11.0587153227222\\
28.5370541407074	11.0584721669298\\
28.5367290760702	11.0582290043302\\
28.5364039992671	11.057985834923\\
28.5360789102973	11.0577426587079\\
28.5357538091599	11.0574994756845\\
28.5354286958539	11.0572562858524\\
28.5351035703785	11.0570130892112\\
28.5347784327327	11.0567698857606\\
28.5344532829156	11.0565266755001\\
28.5341281209263	11.0562834584293\\
28.5338029467639	11.056040234548\\
28.5334777604275	11.0557970038556\\
28.5331525619161	11.0555537663519\\
28.5328273512289	11.0553105220365\\
28.5325021283649	11.0550672709089\\
28.5321768933233	11.0548240129688\\
28.5318516461031	11.0545807482158\\
28.5315263867034	11.0543374766495\\
28.5312011151233	11.0540941982695\\
28.5308758313619	11.0538509130755\\
28.5305505354183	11.0536076210671\\
28.5302252272915	11.0533643222439\\
28.5298999069807	11.0531210166054\\
28.5295745744849	11.0528777041515\\
28.5292492298032	11.0526343848815\\
28.5289238729347	11.0523910587952\\
28.5285985038786	11.0521477258923\\
28.5282731226338	11.0519043861722\\
28.5279477291994	11.0516610396346\\
28.5276223235747	11.0514176862792\\
28.5272969057585	11.0511743261055\\
28.5269714757502	11.0509309591132\\
28.5266460335486	11.0506875853019\\
28.5263205791529	11.0504442046712\\
28.5259951125622	11.0502008172207\\
28.5256696337756	11.0499574229501\\
28.5253441427921	11.0497140218589\\
28.5250186396109	11.0494706139469\\
28.524693124231	11.0492271992135\\
28.5243675966516	11.0489837776584\\
28.5240420568716	11.0487403492813\\
28.5237165048902	11.0484969140817\\
28.5233909407065	11.0482534720593\\
28.5230653643196	11.0480100232137\\
28.5227397757285	11.0477665675444\\
28.5224141749323	11.0475231050513\\
28.5220885619302	11.0472796357337\\
28.5217629367211	11.0470361595914\\
28.5214372993042	11.046792676624\\
28.5211116496786	11.046549186831\\
28.5207859878433	11.0463056902122\\
28.5204603137975	11.0460621867671\\
28.5201346275402	11.0458186764953\\
28.5198089290705	11.0455751593965\\
28.5194832183875	11.0453316354703\\
28.5191574954903	11.0450881047163\\
28.5188317603779	11.044844567134\\
28.5185060130495	11.0446010227232\\
28.5181802535041	11.0443574714835\\
28.5178544817408	11.0441139134144\\
28.5175286977586	11.0438703485156\\
28.5172029015568	11.0436267767867\\
28.5168770931343	11.0433831982273\\
28.5165512724903	11.0431396128371\\
28.5162254396238	11.0428960206155\\
28.5158995945339	11.0426524215624\\
28.5155737372197	11.0424088156772\\
28.5152478676802	11.0421652029597\\
28.5149219859146	11.0419215834093\\
28.514596091922	11.0416779570258\\
28.5142701857013	11.0414343238087\\
28.5139442672518	11.0411906837577\\
28.5136183365725	11.0409470368724\\
28.5132923936624	11.0407033831523\\
28.5129664385207	11.0404597225972\\
28.5126404711464	11.0402160552066\\
28.5123144915386	11.0399723809802\\
28.5119884996964	11.0397286999175\\
28.5116624956189	11.0394850120182\\
28.5113364793052	11.0392413172819\\
28.5110104507543	11.0389976157082\\
28.5106844099653	11.0387539072967\\
28.5103583569373	11.0385101920471\\
28.5100322916694	11.038266469959\\
28.5097062141606	11.0380227410319\\
28.5093801244101	11.0377790052655\\
28.509054022417	11.0375352626594\\
28.5087279081802	11.0372915132133\\
28.5084017816989	11.0370477569267\\
28.5080756429722	11.0368039937993\\
28.5077494919992	11.0365602238306\\
28.5074233287788	11.0363164470203\\
28.5070971533103	11.036072663368\\
28.5067709655927	11.0358288728734\\
28.506444765625	11.035585075536\\
28.5061185534064	11.0353412713554\\
28.5057923289359	11.0350974603313\\
28.5054660922126	11.0348536424633\\
28.5051398432356	11.034609817751\\
28.504813582004	11.034365986194\\
28.5044873085168	11.034122147792\\
28.5041610227731	11.0338783025445\\
28.5038347247721	11.0336344504512\\
28.5035084145127	11.0333905915116\\
28.5031820919941	11.0331467257254\\
28.5028557572153	11.0329028530923\\
28.5025294101755	11.0326589736118\\
28.5022030508736	11.0324150872835\\
28.5018766793089	11.0321711941071\\
28.5015502954802	11.0319272940821\\
28.5012238993868	11.0316833872082\\
28.5008974910278	11.0314394734851\\
28.5005710704021	11.0311955529122\\
28.5002446375089	11.0309516254893\\
28.4999181923472	11.030707691216\\
28.4995917349162	11.0304637500918\\
28.4992652652148	11.0302198021164\\
28.4989387832423	11.0299758472893\\
28.4986122889976	11.0297318856103\\
28.4982857824798	11.0294879170789\\
28.497959263688	11.0292439416948\\
28.4976327326213	11.0289999594575\\
28.4973061892788	11.0287559703667\\
28.4969796336596	11.0285119744219\\
28.4966530657626	11.0282679716229\\
28.4963264855871	11.0280239619692\\
28.495999893132	11.0277799454604\\
28.4956732883964	11.0275359220962\\
28.4953466713795	11.0272918918761\\
28.4950200420803	11.0270478547998\\
28.4946934004979	11.0268038108668\\
28.4943667466313	11.0265597600769\\
28.4940400804797	11.0263157024296\\
28.4937134020421	11.0260716379245\\
28.4933867113175	11.0258275665613\\
28.4930600083051	11.0255834883396\\
28.4927332930039	11.0253394032589\\
28.4924065654131	11.0250953113189\\
28.4920798255316	11.0248512125192\\
28.4917530733586	11.0246071068594\\
28.4914263088931	11.0243629943392\\
28.4910995321343	11.0241188749581\\
28.4907727430811	11.0238747487158\\
28.4904459417327	11.0236306156119\\
28.4901191280881	11.0233864756459\\
28.4897923021464	11.0231423288176\\
28.4894654639067	11.0228981751264\\
28.489138613368	11.0226540145721\\
28.4888117505295	11.0224098471543\\
28.4884848753902	11.0221656728725\\
28.4881579879492	11.0219214917264\\
28.4878310882056	11.0216773037156\\
28.4875041761583	11.0214331088397\\
28.4871772518066	11.0211889070983\\
28.4868503151494	11.020944698491\\
28.4865233661859	11.0207004830175\\
28.4861964049151	11.0204562606774\\
28.4858694313361	11.0202120314702\\
28.485542445448	11.0199677953956\\
28.4852154472498	11.0197235524532\\
28.4848884367407	11.0194793026427\\
28.4845614139196	11.0192350459635\\
28.4842343787857	11.0189907824154\\
28.483907331338	11.018746511998\\
28.4835802715756	11.0185022347108\\
28.4832531994977	11.0182579505535\\
28.4829261151031	11.0180136595257\\
28.4825990183911	11.017769361627\\
28.4822719093607	11.0175250568571\\
28.4819447880109	11.0172807452155\\
28.481617654341	11.0170364267018\\
28.4812905083498	11.0167921013157\\
28.4809633500365	11.0165477690568\\
28.4806361794001	11.0163034299246\\
28.4803089964398	11.0160590839189\\
28.4799818011546	11.0158147310392\\
28.4796545935435	11.0155703712851\\
28.4793273736057	11.0153260046563\\
28.4790001413402	11.0150816311523\\
28.4786728967461	11.0148372507728\\
28.4783456398225	11.0145928635174\\
28.4780183705683	11.0143484693857\\
28.4776910889828	11.0141040683772\\
28.4773637950649	11.0138596604917\\
28.4770364888138	11.0136152457288\\
28.4767091702285	11.013370824088\\
28.4763818393081	11.0131263955689\\
28.4760544960516	11.0128819601712\\
28.4757271404581	11.0126375178946\\
28.4753997725268	11.0123930687385\\
28.4750723922566	11.0121486127026\\
28.4747449996466	11.0119041497866\\
28.4744175946959	11.0116596799899\\
28.4740901774036	11.0114152033124\\
28.4737627477687	11.0111707197535\\
28.4734353057904	11.0109262293129\\
28.4731078514676	11.0106817319901\\
28.4727803847994	11.0104372277849\\
28.472452905785	11.0101927166968\\
28.4721254144234	11.0099481987254\\
28.4717979107136	11.0097036738704\\
28.4714703946547	11.0094591421313\\
28.4711428662458	11.0092146035077\\
28.470815325486	11.0089700579994\\
28.4704877723743	11.0087255056058\\
28.4701602069099	11.0084809463266\\
28.4698326290916	11.0082363801615\\
28.4695050389187	11.0079918071099\\
28.4691774363903	11.0077472271716\\
28.4688498215052	11.0075026403462\\
28.4685221942627	11.0072580466332\\
28.4681945546619	11.0070134460323\\
28.4678669027017	11.0067688385431\\
28.4675392383812	11.0065242241652\\
28.4672115616995	11.0062796028981\\
28.4668838726557	11.0060349747416\\
28.4665561712489	11.0057903396953\\
28.4662284574781	11.0055456977587\\
28.4659007313423	11.0053010489314\\
28.4655729928407	11.0050563932131\\
28.4652452419723	11.0048117306034\\
28.4649174787362	11.0045670611019\\
28.4645897031314	11.0043223847082\\
28.464261915157	11.0040777014219\\
28.4639341148121	11.0038330112426\\
28.4636063020957	11.00358831417\\
28.463278477007	11.0033436102036\\
28.4629506395449	11.0030988993431\\
28.4626227897085	11.002854181588\\
28.462294927497	11.0026094569381\\
28.4619670529094	11.0023647253928\\
28.4616391659446	11.0021199869519\\
28.4613112666019	11.0018752416148\\
28.4609833548802	11.0016304893813\\
28.4606554307787	11.001385730251\\
28.4603274942964	11.0011409642234\\
28.4599995454323	11.0008961912981\\
28.4596715841856	11.0006514114748\\
28.4593436105553	11.0004066247532\\
28.4590156245404	11.0001618311327\\
28.45868762614	10.999917030613\\
28.4583596153532	10.9996722231938\\
28.4580315921791	10.9994274088746\\
28.4577035566167	10.999182587655\\
28.4573755086651	10.9989377595347\\
28.4570474483233	10.9986929245132\\
28.4567193755905	10.9984480825902\\
28.4563912904656	10.9982032337653\\
28.4560631929477	10.9979583780381\\
28.4557350830359	10.9977135154083\\
28.4554069607293	10.9974686458753\\
28.455078826027	10.9972237694389\\
28.4547506789279	10.9969788860986\\
28.4544225194312	10.996733995854\\
28.4540943475358	10.9964890987048\\
28.453766163241	10.9962441946506\\
28.4534379665457	10.995999283691\\
28.4531097574491	10.9957543658256\\
28.45278153595	10.9955094410539\\
28.4524533020478	10.9952645093757\\
28.4521250557413	10.9950195707905\\
28.4517967970296	10.994774625298\\
28.4514685259119	10.9945296728977\\
28.4511402423872	10.9942847135892\\
28.4508119464545	10.9940397473722\\
28.4504836381129	10.9937947742463\\
28.4501553173615	10.9935497942111\\
28.4498269841993	10.9933048072662\\
28.4494986386254	10.9930598134112\\
28.4491702806389	10.9928148126457\\
28.4488419102387	10.9925698049693\\
28.4485135274241	10.9923247903817\\
28.448185132194	10.9920797688824\\
28.4478567245474	10.9918347404711\\
28.4475283044836	10.9915897051474\\
28.4471998720014	10.9913446629108\\
28.4468714271001	10.991099613761\\
28.4465429697785	10.9908545576976\\
28.4462145000359	10.9906094947202\\
28.4458860178712	10.9903644248285\\
28.4455575232836	10.990119348022\\
28.4452290162721	10.9898742643003\\
28.4449004968356	10.989629173663\\
28.4445719649734	10.9893840761098\\
28.4442434206845	10.9891389716403\\
28.4439148639678	10.9888938602541\\
28.4435862948226	10.9886487419507\\
28.4432577132478	10.9884036167298\\
28.4429291192425	10.9881584845911\\
28.4426005128058	10.987913345534\\
28.4422718939366	10.9876681995583\\
28.4419432626342	10.9874230466635\\
28.4416146188975	10.9871778868493\\
28.4412859627256	10.9869327201152\\
28.4409572941176	10.9866875464609\\
28.4406286130724	10.9864423658859\\
28.4402999195893	10.98619717839\\
28.4399712136672	10.9859519839726\\
28.4396424953051	10.9857067826334\\
28.4393137645022	10.985461574372\\
28.4389850212576	10.9852163591881\\
28.4386562655702	10.9849711370811\\
28.4383274974391	10.9847259080508\\
28.4379987168634	10.9844806720968\\
28.4376699238421	10.9842354292186\\
28.4373411183744	10.9839901794158\\
28.4370123004592	10.9837449226882\\
28.4366834700956	10.9834996590352\\
28.4363546272827	10.9832543884565\\
28.4360257720195	10.9830091109517\\
28.435696904305	10.9827638265204\\
28.4353680241385	10.9825185351622\\
28.4350391315188	10.9822732368768\\
28.4347102264451	10.9820279316637\\
28.4343813089164	10.9817826195225\\
28.4340523789318	10.9815373004529\\
28.4337234364903	10.9812919744544\\
28.433394481591	10.9810466415267\\
28.4330655142329	10.9808013016694\\
28.4327365344151	10.980555954882\\
28.4324075421367	10.9803106011643\\
28.4320785373966	10.9800652405158\\
28.4317495201941	10.979819872936\\
28.431420490528	10.9795744984247\\
28.4310914483975	10.9793291169814\\
28.4307623938017	10.9790837286058\\
28.4304333267395	10.9788383332974\\
28.4301042472101	10.9785929310558\\
28.4297751552125	10.9783475218807\\
28.4294460507457	10.9781021057717\\
28.4291169338089	10.9778566827284\\
28.428787804401	10.9776112527503\\
28.4284586625211	10.9773658158371\\
28.4281295081683	10.9771203719885\\
28.4278003413416	10.9768749212039\\
28.4274711620401	10.9766294634831\\
28.4271419702629	10.9763839988255\\
28.4268127660089	10.9761385272309\\
28.4264835492772	10.9758930486989\\
28.426154320067	10.975647563229\\
28.4258250783772	10.9754020708209\\
28.4254958242069	10.9751565714741\\
28.4251665575552	10.9749110651883\\
28.4248372784211	10.9746655519631\\
28.4245079868037	10.974420031798\\
28.4241786827019	10.9741745046928\\
28.423849366115	10.9739289706469\\
28.4235200370419	10.9736834296601\\
28.4231906954816	10.9734378817319\\
28.4228613414333	10.9731923268619\\
28.422531974896	10.9729467650497\\
28.4222025958687	10.972701196295\\
28.4218732043505	10.9724556205974\\
28.4215438003404	10.9722100379563\\
28.4212143838376	10.9719644483716\\
28.420884954841	10.9717188518427\\
28.4205555133497	10.9714732483693\\
28.4202260593627	10.971227637951\\
28.4198965928792	10.9709820205874\\
28.4195671138981	10.970736396278\\
28.4192376224185	10.9704907650226\\
28.4189081184394	10.9702451268207\\
28.41857860196	10.9699994816719\\
28.4182490729793	10.9697538295759\\
28.4179195314962	10.9695081705321\\
28.41758997751	10.9692625045403\\
28.4172604110195	10.9690168316001\\
28.4169308320239	10.968771151711\\
28.4166012405223	10.9685254648727\\
28.4162716365136	10.9682797710847\\
28.4159420199969	10.9680340703468\\
28.4156123909713	10.9677883626584\\
28.4152827494358	10.9675426480192\\
28.4149530953895	10.9672969264287\\
28.4146234288314	10.9670511978867\\
28.4142937497606	10.9668054623927\\
28.4139640581761	10.9665597199464\\
28.413634354077	10.9663139705472\\
28.4133046374623	10.9660682141949\\
28.4129749083311	10.965822450889\\
28.4126451666824	10.9655766806291\\
28.4123154125153	10.9653309034149\\
28.4119856458288	10.965085119246\\
28.411655866622	10.9648393281219\\
28.4113260748939	10.9645935300422\\
28.4109962706435	10.9643477250067\\
28.41066645387	10.9641019130148\\
28.4103366245723	10.9638560940662\\
28.4100067827496	10.9636102681606\\
28.4096769284008	10.9633644352974\\
28.4093470615251	10.9631185954763\\
28.4090171821214	10.9628727486969\\
28.4086872901888	10.9626268949589\\
28.4083573857263	10.9623810342617\\
28.4080274687331	10.9621351666051\\
28.4076975392082	10.9618892919886\\
28.4073675971505	10.9616434104119\\
28.4070376425592	10.9613975218745\\
28.4067076754332	10.9611516263761\\
28.4063776957718	10.9609057239162\\
28.4060477035738	10.9606598144945\\
28.4057176988384	10.9604138981106\\
28.4053876815645	10.9601679747641\\
28.4050576517513	10.9599220444545\\
28.4047276093978	10.9596761071815\\
28.404397554503	10.9594301629447\\
28.4040674870659	10.9591842117437\\
28.4037374070857	10.9589382535781\\
28.4034073145614	10.9586922884476\\
28.403077209492	10.9584463163516\\
28.4027470918765	10.9582003372899\\
28.4024169617141	10.9579543512619\\
28.4020868190037	10.9577083582675\\
28.4017566637444	10.957462358306\\
28.4014264959352	10.9572163513772\\
28.4010963155753	10.9569703374807\\
28.4007661226636	10.956724316616\\
28.4004359171991	10.9564782887827\\
28.400105699181	10.9562322539805\\
28.3997754686083	10.955986212209\\
28.39944522548	10.9557401634678\\
28.3991149697952	10.9554941077564\\
28.3987847015528	10.9552480450746\\
28.3984544207521	10.9550019754218\\
28.3981241273919	10.9547558987977\\
28.3977938214714	10.9545098152019\\
28.3974635029895	10.954263724634\\
28.3971331719454	10.9540176270937\\
28.3968028283381	10.9537715225804\\
28.3964724721666	10.9535254110939\\
28.39614210343	10.9532792926337\\
28.3958117221273	10.9530331671994\\
28.3954813282575	10.9527870347906\\
28.3951509218197	10.952540895407\\
28.394820502813	10.9522947490481\\
28.3944900712364	10.9520485957136\\
28.3941596270889	10.9518024354031\\
28.3938291703696	10.951556268116\\
28.3934987010774	10.9513100938522\\
28.3931682192116	10.9510639126111\\
28.3928377247711	10.9508177243924\\
28.3925072177549	10.9505715291956\\
28.3921766981621	10.9503253270205\\
28.3918461659917	10.9500791178665\\
28.3915156212428	10.9498329017333\\
28.3911850639144	10.9495866786205\\
28.3908544940056	10.9493404485277\\
28.3905239115154	10.9490942114545\\
28.3901933164428	10.9488479674005\\
28.3898627087869	10.9486017163653\\
28.3895320885468	10.9483554583485\\
28.3892014557214	10.9481091933498\\
28.3888708103098	10.9478629213687\\
28.3885401523111	10.9476166424047\\
28.3882094817243	10.9473703564577\\
28.3878787985484	10.947124063527\\
28.3875481027825	10.9468777636124\\
28.3872173944257	10.9466314567134\\
28.3868866734769	10.9463851428297\\
28.3865559399352	10.9461388219608\\
28.3862251937996	10.9458924941064\\
28.3858944350692	10.9456461592661\\
28.3855636637431	10.9453998174394\\
28.3852328798202	10.9451534686259\\
28.3849020832996	10.9449071128253\\
28.3845712741804	10.9446607500372\\
28.3842404524615	10.9444143802612\\
28.3839096181421	10.9441680034969\\
28.3835787712212	10.9439216197438\\
28.3832479116977	10.9436752290016\\
28.3829170395708	10.9434288312699\\
28.3825861548395	10.9431824265483\\
28.3822552575028	10.9429360148364\\
28.3819243475598	10.9426895961338\\
28.3815934250095	10.9424431704401\\
28.381262489851	10.942196737755\\
28.3809315420832	10.9419502980779\\
28.3806005817052	10.9417038514085\\
28.3802696087161	10.9414573977464\\
28.379938623115	10.9412109370913\\
28.3796076249007	10.9409644694427\\
28.3792766140725	10.9407179948002\\
28.3789455906292	10.9404715131634\\
28.37861455457	10.940225024532\\
28.3782835058939	10.9399785289055\\
28.3779524446	10.9397320262836\\
28.3776213706872	10.9394855166657\\
28.3772902841546	10.9392390000517\\
28.3769591850012	10.9389924764409\\
28.3766280732262	10.9387459458332\\
28.3762969488284	10.9384994082279\\
28.3759658118071	10.9382528636248\\
28.3756346621611	10.9380063120235\\
28.3753034998896	10.9377597534236\\
28.3749723249915	10.9375131878246\\
28.3746411374659	10.9372666152261\\
28.3743099373119	10.9370200356279\\
28.3739787245285	10.9367734490294\\
28.3736474991147	10.9365268554303\\
28.3733162610695	10.9362802548301\\
28.372985010392	10.9360336472285\\
28.3726537470813	10.9357870326252\\
28.3723224711363	10.9355404110196\\
28.3719911825562	10.9352937824113\\
28.3716598813398	10.9350471468001\\
28.3713285674864	10.9348005041855\\
28.3709972409948	10.934553854567\\
28.3706659018642	10.9343071979443\\
28.3703345500936	10.9340605343171\\
28.370003185682	10.9338138636848\\
28.3696718086285	10.9335671860471\\
28.369340418932	10.9333205014037\\
28.3690090165917	10.933073809754\\
28.3686776016065	10.9328271110978\\
28.3683461739755	10.9325804054345\\
28.3680147336977	10.9323336927639\\
28.3676832807722	10.9320869730855\\
28.367351815198	10.9318402463989\\
28.3670203369742	10.9315935127037\\
28.3666888460997	10.9313467719995\\
28.3663573425736	10.931100024286\\
28.3660258263949	10.9308532695626\\
28.3656942975627	10.9306065078291\\
28.365362756076	10.930359739085\\
28.3650312019339	10.93011296333\\
28.3646996351353	10.9298661805635\\
28.3643680556793	10.9296193907853\\
28.364036463565	10.929372593995\\
28.3637048587913	10.929125790192\\
28.3633732413573	10.9288789793761\\
28.3630416112621	10.9286321615469\\
28.3627099685046	10.9283853367038\\
28.362378313084	10.9281385048466\\
28.3620466449992	10.9278916659749\\
28.3617149642492	10.9276448200882\\
28.3613832708332	10.9273979671861\\
28.3610515647501	10.9271511072683\\
28.360719845999	10.9269042403343\\
28.3603881145788	10.9266573663838\\
28.3600563704887	10.9264104854163\\
28.3597246137277	10.9261635974314\\
28.3593928442947	10.9259167024288\\
28.3590610621889	10.9256698004081\\
28.3587292674093	10.9254228913688\\
28.3583974599548	10.9251759753106\\
28.3580656398246	10.924929052233\\
28.3577338070176	10.9246821221357\\
28.3574019615329	10.9244351850182\\
28.3570701033695	10.9241882408802\\
28.3567382325265	10.9239412897212\\
28.3564063490029	10.9236943315409\\
28.3560744527976	10.9234473663389\\
28.3557425439098	10.9232003941147\\
28.3554106223385	10.9229534148679\\
28.3550786880827	10.9227064285983\\
28.3547467411414	10.9224594353052\\
28.3544147815137	10.9222124349885\\
28.3540828091986	10.9219654276476\\
28.3537508241951	10.9217184132822\\
28.3534188265022	10.9214713918918\\
28.353086816119	10.9212243634761\\
28.3527547930446	10.9209773280347\\
28.3524227572779	10.9207302855671\\
28.352090708818	10.920483236073\\
28.3517586476638	10.920236179552\\
28.3514265738145	10.9199891160036\\
28.3510944872691	10.9197420454275\\
28.3507623880266	10.9194949678232\\
28.3504302760859	10.9192478831904\\
28.3500981514463	10.9190007915287\\
28.3497660141066	10.9187536928376\\
28.3494338640659	10.9185065871169\\
28.3491017013232	10.9182594743659\\
28.3487695258776	10.9180123545845\\
28.3484373377281	10.9177652277721\\
28.3481051368737	10.9175180939283\\
28.3477729233135	10.9172709530529\\
28.3474406970464	10.9170238051453\\
28.3471084580716	10.9167766502051\\
28.346776206388	10.9165294882321\\
28.3464439419946	10.9162823192257\\
28.3461116648905	10.9160351431855\\
28.3457793750747	10.9157879601112\\
28.3454470725463	10.9155407700024\\
28.3451147573043	10.9152935728587\\
28.3447824293476	10.9150463686796\\
28.3444500886754	10.9147991574647\\
28.3441177352866	10.9145519392138\\
28.3437853691803	10.9143047139263\\
28.3434529903555	10.9140574816018\\
28.3431205988112	10.91381024224\\
28.3427881945465	10.9135629958405\\
28.3424557775604	10.9133157424028\\
28.3421233478519	10.9130684819266\\
28.34179090542	10.9128212144115\\
28.3414584502637	10.912573939857\\
28.3411259823822	10.9123266582627\\
28.3407935017744	10.9120793696283\\
28.3404610084393	10.9118320739534\\
28.340128502376	10.9115847712375\\
28.3397959835834	10.9113374614803\\
28.3394634520607	10.9110901446813\\
28.3391309078069	10.9108428208402\\
28.3387983508208	10.9105954899565\\
28.3384657811017	10.9103481520298\\
28.3381331986485	10.9101008070599\\
28.3378006034603	10.9098534550461\\
28.3374679955359	10.9096060959882\\
28.3371353748746	10.9093587298858\\
28.3368027414753	10.9091113567383\\
28.3364700953371	10.9088639765456\\
28.3361374364589	10.908616589307\\
28.3358047648398	10.9083691950223\\
28.3354720804788	10.9081217936911\\
28.3351393833749	10.9078743853128\\
28.3348066735272	10.9076269698872\\
28.3344739509347	10.9073795474139\\
28.3341412155964	10.9071321178923\\
28.3338084675113	10.9068846813222\\
28.3334757066784	10.9066372377032\\
28.3331429330969	10.9063897870347\\
28.3328101467656	10.9061423293165\\
28.3324773476837	10.9058948645481\\
28.3321445358501	10.9056473927291\\
28.3318117112639	10.9053999138591\\
28.331478873924	10.9051524279378\\
28.3311460238296	10.9049049349646\\
28.3308131609796	10.9046574349393\\
28.3304802853731	10.9044099278614\\
28.330147397009	10.9041624137305\\
28.3298144958864	10.9039148925462\\
28.3294815820044	10.9036673643081\\
28.3291486553619	10.9034198290158\\
28.328815715958	10.903172286669\\
28.3284827637917	10.9029247372671\\
28.328149798862	10.9026771808098\\
28.3278168211679	10.9024296172967\\
28.3274838307084	10.9021820467274\\
28.3271508274827	10.9019344691015\\
28.3268178114896	10.9016868844185\\
28.3264847827283	10.9014392926782\\
28.3261517411977	10.9011916938801\\
28.3258186868968	10.9009440880237\\
28.3254856198247	10.9006964751087\\
28.3251525399804	10.9004488551347\\
28.324819447363	10.9002012281012\\
28.3244863419714	10.899953594008\\
28.3241532238046	10.8997059528544\\
28.3238200928617	10.8994583046403\\
28.3234869491418	10.8992106493651\\
28.3231537926437	10.8989629870285\\
28.3228206233666	10.89871531763\\
28.3224874413094	10.8984676411693\\
28.3221542464712	10.898219957646\\
28.3218210388511	10.8979722670596\\
28.3214878184479	10.8977245694097\\
28.3211545852608	10.897476864696\\
28.3208213392887	10.8972291529181\\
28.3204880805307	10.8969814340754\\
28.3201548089858	10.8967337081677\\
28.319821524653	10.8964859751946\\
28.3194882275313	10.8962382351555\\
28.3191549176198	10.8959904880502\\
28.3188215949175	10.8957427338782\\
28.3184882594233	10.8954949726392\\
28.3181549111363	10.8952472043326\\
28.3178215500556	10.8949994289582\\
28.3174881761801	10.8947516465154\\
28.3171547895089	10.894503857004\\
28.3168213900409	10.8942560604235\\
28.3164879777752	10.8940082567735\\
28.3161545527109	10.8937604460535\\
28.3158211148468	10.8935126282633\\
28.3154876641821	10.8932648034024\\
28.3151542007158	10.8930169714703\\
28.3148207244469	10.8927691324668\\
28.3144872353743	10.8925212863913\\
28.3141537334972	10.8922734332434\\
28.3138202188145	10.8920255730229\\
28.3134866913252	10.8917777057292\\
28.3131531510284	10.891529831362\\
28.3128195979231	10.8912819499209\\
28.3124860320083	10.8910340614054\\
28.3121524532829	10.8907861658151\\
28.3118188617462	10.8905382631497\\
28.3114852573969	10.8902903534088\\
28.3111516402342	10.8900424365919\\
28.3108180102571	10.8897945126986\\
28.3104843674646	10.8895465817286\\
28.3101507118557	10.8892986436814\\
28.3098170434294	10.8890506985566\\
28.3094833621847	10.8888027463538\\
28.3091496681207	10.8885547870727\\
28.3088159612364	10.8883068207128\\
28.3084822415307	10.8880588472737\\
28.3081485090027	10.887810866755\\
28.3078147636515	10.8875628791562\\
28.3074810054759	10.8873148844771\\
28.3071472344752	10.8870668827172\\
28.3068134506481	10.8868188738761\\
28.3064796539938	10.8865708579533\\
28.3061458445113	10.8863228349485\\
28.3058120221997	10.8860748048613\\
28.3054781870578	10.8858267676913\\
28.3051443390847	10.8855787234381\\
28.3048104782795	10.8853306721011\\
28.3044766046411	10.8850826136802\\
28.3041427181686	10.8848345481748\\
28.303808818861	10.8845864755846\\
28.3034749067172	10.8843383959091\\
28.3031409817364	10.8840903091479\\
28.3028070439175	10.8838422153006\\
28.3024730932595	10.8835941143669\\
28.3021391297614	10.8833460063464\\
28.3018051534223	10.8830978912385\\
28.3014711642412	10.8828497690429\\
28.301137162217	10.8826016397593\\
28.3008031473489	10.8823535033871\\
28.3004691196357	10.8821053599261\\
28.3001350790766	10.8818572093758\\
28.2998010256704	10.8816090517357\\
28.2994669594164	10.8813608870055\\
};
\addplot [color=mycolor1, forget plot]
  table[row sep=crcr]{%
28.2994669594164	10.8813608870055\\
28.2991328803133	10.8811127151848\\
28.2987987883604	10.8808645362732\\
28.2984646835565	10.8806163502702\\
28.2981305659007	10.8803681571755\\
28.297796435392	10.8801199569887\\
28.2974622920294	10.8798717497093\\
28.2971281358119	10.8796235353369\\
28.2967939667385	10.8793753138712\\
28.2964597848083	10.8791270853118\\
28.2961255900203	10.8788788496581\\
28.2957913823734	10.8786306069099\\
28.2954571618666	10.8783823570667\\
28.2951229284991	10.8781341001281\\
28.2947886822697	10.8778858360938\\
28.2944544231776	10.8776375649632\\
28.2941201512216	10.877389286736\\
28.2937858664009	10.8771410014118\\
28.2934515687144	10.8768927089902\\
28.2931172581612	10.8766444094708\\
28.2927829347402	10.8763961028532\\
28.2924485984504	10.8761477891369\\
28.292114249291	10.8758994683216\\
28.2917798872608	10.8756511404068\\
28.2914455123589	10.8754028053922\\
28.2911111245843	10.8751544632773\\
28.290776723936	10.8749061140618\\
28.290442310413	10.8746577577452\\
28.2901078840144	10.8744093943271\\
28.289773444739	10.8741610238071\\
28.2894389925861	10.8739126461849\\
28.2891045275544	10.87366426146\\
28.2887700496431	10.8734158696319\\
28.2884355588512	10.8731674707004\\
28.2881010551777	10.8729190646649\\
28.2877665386215	10.8726706515252\\
28.2874320091817	10.8724222312807\\
28.2870974668573	10.872173803931\\
28.2867629116474	10.8719253694759\\
28.2864283435508	10.8716769279148\\
28.2860937625666	10.8714284792473\\
28.2857591686939	10.8711800234731\\
28.2854245619316	10.8709315605917\\
28.2850899422787	10.8706830906028\\
28.2847553097343	10.8704346135058\\
28.2844206642974	10.8701861293005\\
28.2840860059669	10.8699376379864\\
28.2837513347418	10.8696891395631\\
28.2834166506212	10.8694406340302\\
28.2830819536041	10.8691921213873\\
28.2827472436895	10.868943601634\\
28.2824125208764	10.8686950747698\\
28.2820777851638	10.8684465407945\\
28.2817430365507	10.8681979997074\\
28.281408275036	10.8679494515084\\
28.2810735006189	10.8677008961969\\
28.2807387132983	10.8674523337725\\
28.2804039130733	10.8672037642348\\
28.2800690999427	10.8669551875835\\
28.2797342739057	10.8667066038181\\
28.2793994349612	10.8664580129382\\
28.2790645831083	10.8662094149435\\
28.2787297183459	10.8659608098334\\
28.2783948406731	10.8657121976076\\
28.2780599500888	10.8654635782657\\
28.2777250465921	10.8652149518072\\
28.2773901301819	10.8649663182319\\
28.2770552008573	10.8647176775392\\
28.2767202586173	10.8644690297287\\
28.2763853034609	10.8642203748001\\
28.276050335387	10.8639717127529\\
28.2757153543948	10.8637230435868\\
28.2753803604831	10.8634743673013\\
28.275045353651	10.863225683896\\
28.2747103338975	10.8629769933705\\
28.2743753012216	10.8627282957244\\
28.2740402556223	10.8624795909573\\
28.2737051970986	10.8622308790688\\
28.2733701256495	10.8619821600584\\
28.273035041274	10.8617334339259\\
28.2726999439711	10.8614847006706\\
28.2723648337399	10.8612359602924\\
28.2720297105792	10.8609872127907\\
28.2716945744882	10.8607384581651\\
28.2713594254658	10.8604896964152\\
28.271024263511	10.8602409275407\\
28.2706890886228	10.8599921515411\\
28.2703539008003	10.859743368416\\
28.2700187000424	10.859494578165\\
28.2696834863481	10.8592457807876\\
28.2693482597165	10.8589969762836\\
28.2690130201465	10.8587481646524\\
28.2686777676371	10.8584993458937\\
28.2683425021874	10.858250520007\\
28.2680072237963	10.858001686992\\
28.2676719324628	10.8577528468482\\
28.267336628186	10.8575039995753\\
28.2670013109648	10.8572551451727\\
28.2666659807982	10.8570062836402\\
28.2663306376853	10.8567574149773\\
28.265995281625	10.8565085391836\\
28.2656599126164	10.8562596562587\\
28.2653245306584	10.8560107662021\\
28.26498913575	10.8557618690135\\
28.2646537278903	10.8555129646925\\
28.2643183070783	10.8552640532386\\
28.2639828733128	10.8550151346515\\
28.263647426593	10.8547662089307\\
28.2633119669179	10.8545172760758\\
28.2629764942863	10.8542683360865\\
28.2626410086974	10.8540193889622\\
28.2623055101502	10.8537704347027\\
28.2619699986436	10.8535214733074\\
28.2616344741766	10.853272504776\\
28.2612989367482	10.8530235291081\\
28.2609633863575	10.8527745463032\\
28.2606278230034	10.852525556361\\
28.2602922466849	10.8522765592811\\
28.2599566574011	10.852027555063\\
28.2596210551509	10.8517785437063\\
28.2592854399333	10.8515295252106\\
28.2589498117473	10.8512804995755\\
28.2586141705919	10.8510314668006\\
28.2582785164662	10.8507824268855\\
28.257942849369	10.8505333798298\\
28.2576071692995	10.8502843256331\\
28.2572714762566	10.8500352642949\\
28.2569357702392	10.8497861958149\\
28.2566000512465	10.8495371201925\\
28.2562643192774	10.8492880374276\\
28.2559285743309	10.8490389475195\\
28.2555928164059	10.848789850468\\
28.2552570455016	10.8485407462726\\
28.2549212616168	10.8482916349328\\
28.2545854647506	10.8480425164483\\
28.254249654902	10.8477933908187\\
28.25391383207	10.8475442580436\\
28.2535779962535	10.8472951181225\\
28.2532421474516	10.8470459710551\\
28.2529062856633	10.8467968168409\\
28.2525704108875	10.8465476554795\\
28.2522345231233	10.8462984869705\\
28.2518986223696	10.8460493113135\\
28.2515627086254	10.8458001285081\\
28.2512267818898	10.8455509385539\\
28.2508908421618	10.8453017414505\\
28.2505548894402	10.8450525371974\\
28.2502189237242	10.8448033257943\\
28.2498829450127	10.8445541072407\\
28.2495469533047	10.8443048815363\\
28.2492109485992	10.8440556486806\\
28.2488749308952	10.8438064086731\\
28.2485389001918	10.8435571615136\\
28.2482028564878	10.8433079072016\\
28.2478667997822	10.8430586457366\\
28.2475307300742	10.8428093771183\\
28.2471946473626	10.8425601013463\\
28.2468585516465	10.8423108184201\\
28.2465224429249	10.8420615283394\\
28.2461863211967	10.8418122311036\\
28.2458501864609	10.8415629267125\\
28.2455140387166	10.8413136151656\\
28.2451778779627	10.8410642964625\\
28.2448417041983	10.8408149706027\\
28.2445055174222	10.840565637586\\
28.2441693176336	10.8403162974118\\
28.2438331048314	10.8400669500797\\
28.2434968790146	10.8398175955894\\
28.2431606401821	10.8395682339404\\
28.242824388333	10.8393188651323\\
28.2424881234663	10.8390694891647\\
28.242151845581	10.8388201060372\\
28.241815554676	10.8385707157494\\
28.2414792507504	10.8383213183009\\
28.2411429338031	10.8380719136912\\
28.2408066038332	10.83782250192\\
28.2404702608395	10.8375730829868\\
28.2401339048212	10.8373236568913\\
28.2397975357772	10.8370742236329\\
28.2394611537065	10.8368247832114\\
28.239124758608	10.8365753356262\\
28.2387883504809	10.8363258808771\\
28.238451929324	10.8360764189635\\
28.2381154951364	10.835826949885\\
28.237779047917	10.8355774736413\\
28.2374425876648	10.835327990232\\
28.2371061143789	10.8350784996565\\
28.2367696280582	10.8348290019146\\
28.2364331287017	10.8345794970058\\
28.2360966163085	10.8343299849296\\
28.2357600908774	10.8340804656858\\
28.2354235524075	10.8338309392738\\
28.2350870008977	10.8335814056932\\
28.2347504363471	10.8333318649437\\
28.2344138587547	10.8330823170248\\
28.2340772681194	10.8328327619361\\
28.2337406644402	10.8325831996773\\
28.2334040477162	10.8323336302478\\
28.2330674179462	10.8320840536473\\
28.2327307751294	10.8318344698754\\
28.2323941192646	10.8315848789316\\
28.2320574503509	10.8313352808156\\
28.2317207683873	10.8310856755269\\
28.2313840733726	10.8308360630651\\
28.2310473653061	10.8305864434298\\
28.2307106441865	10.8303368166207\\
28.230373910013	10.8300871826372\\
28.2300371627845	10.829837541479\\
28.2297004024999	10.8295878931456\\
28.2293636291584	10.8293382376367\\
28.2290268427587	10.8290885749518\\
28.2286900433001	10.8288389050905\\
28.2283532307814	10.8285892280524\\
28.2280164052016	10.8283395438372\\
28.2276795665597	10.8280898524443\\
28.2273427148547	10.8278401538734\\
28.2270058500856	10.827590448124\\
28.2266689722513	10.8273407351958\\
28.2263320813509	10.8270910150883\\
28.2259951773834	10.8268412878011\\
28.2256582603477	10.8265915533339\\
28.2253213302428	10.8263418116861\\
28.2249843870677	10.8260920628574\\
28.2246474308214	10.8258423068474\\
28.2243104615029	10.8255925436556\\
28.2239734791111	10.8253427732817\\
28.2236364836451	10.8250929957252\\
28.2232994751038	10.8248432109857\\
28.2229624534862	10.8245934190628\\
28.2226254187913	10.8243436199561\\
28.2222883710181	10.8240938136652\\
28.2219513101656	10.8238440001896\\
28.2216142362327	10.823594179529\\
28.2212771492185	10.8233443516829\\
28.2209400491219	10.8230945166509\\
28.2206029359419	10.8228446744327\\
28.2202658096775	10.8225948250277\\
28.2199286703276	10.8223449684357\\
28.2195915178914	10.822095104656\\
28.2192543523677	10.8218452336885\\
28.2189171737555	10.8215953555326\\
28.2185799820538	10.8213454701879\\
28.2182427772617	10.821095577654\\
28.217905559378	10.8208456779305\\
28.2175683284017	10.8205957710171\\
28.217231084332	10.8203458569131\\
28.2168938271676	10.8200959356184\\
28.2165565569077	10.8198460071324\\
28.2162192735511	10.8195960714547\\
28.215881977097	10.8193461285849\\
28.2155446675442	10.8190961785227\\
28.2152073448917	10.8188462212675\\
28.2148700091386	10.818596256819\\
28.2145326602838	10.8183462851767\\
28.2141952983263	10.8180963063404\\
28.2138579232651	10.8178463203094\\
28.2135205350991	10.8175963270835\\
28.2131831338273	10.8173463266621\\
28.2128457194488	10.817096319045\\
28.2125082919625	10.8168463042316\\
28.2121708513674	10.8165962822216\\
28.2118333976624	10.8163462530145\\
28.2114959308466	10.81609621661\\
28.2111584509189	10.8158461730076\\
28.2108209578783	10.8155961222069\\
28.2104834517238	10.8153460642075\\
28.2101459324543	10.8150959990089\\
28.209808400069	10.8148459266109\\
28.2094708545666	10.8145958470128\\
28.2091332959463	10.8143457602144\\
28.2087957242069	10.8140956662152\\
28.2084581393476	10.8138455650149\\
28.2081205413671	10.8135954566129\\
28.2077829302647	10.8133453410088\\
28.2074453060391	10.8130952182024\\
28.2071076686894	10.8128450881931\\
28.2067700182146	10.8125949509805\\
28.2064323546136	10.8123448065642\\
28.2060946778855	10.8120946549438\\
28.2057569880292	10.8118444961189\\
28.2054192850437	10.8115943300891\\
28.2050815689279	10.8113441568539\\
28.2047438396809	10.811093976413\\
28.2044060973016	10.8108437887659\\
28.2040683417891	10.8105935939122\\
28.2037305731422	10.8103433918514\\
28.2033927913599	10.8100931825833\\
28.2030549964414	10.8098429661073\\
28.2027171883854	10.8095927424231\\
28.202379367191	10.8093425115302\\
28.2020415328572	10.8090922734282\\
28.201703685383	10.8088420281168\\
28.2013658247672	10.8085917755954\\
28.201027951009	10.8083415158637\\
28.2006900641073	10.8080912489212\\
28.2003521640611	10.8078409747676\\
28.2000142508692	10.8075906934024\\
28.1996763245308	10.8073404048252\\
28.1993383850448	10.8070901090356\\
28.1990004324102	10.8068398060331\\
28.1986624666259	10.8065894958175\\
28.1983244876909	10.8063391783882\\
28.1979864956043	10.8060888537448\\
28.1976484903649	10.8058385218869\\
28.1973104719717	10.8055881828141\\
28.1969724404238	10.805337836526\\
28.1966343957201	10.8050874830222\\
28.1962963378596	10.8048371223022\\
28.1959582668413	10.8045867543657\\
28.195620182664	10.8043363792121\\
28.1952820853269	10.8040859968412\\
28.1949439748289	10.8038356072525\\
28.1946058511689	10.8035852104455\\
28.194267714346	10.8033348064199\\
28.1939295643591	10.8030843951752\\
28.1935914012071	10.802833976711\\
28.1932532248891	10.802583551027\\
28.1929150354041	10.8023331181226\\
28.192576832751	10.8020826779975\\
28.1922386169287	10.8018322306513\\
28.1919003879363	10.8015817760835\\
28.1915621457728	10.8013313142937\\
28.191223890437	10.8010808452815\\
28.190885621928	10.8008303690465\\
28.1905473402448	10.8005798855883\\
28.1902090453863	10.8003293949064\\
28.1898707373515	10.8000788970005\\
28.1895324161394	10.7998283918701\\
28.189194081749	10.7995778795149\\
28.1888557341791	10.7993273599343\\
28.1885173734289	10.7990768331279\\
28.1881789994972	10.7988262990955\\
28.1878406123831	10.7985757578364\\
28.1875022120854	10.7983252093504\\
28.1871637986033	10.798074653637\\
28.1868253719356	10.7978240906958\\
28.1864869320814	10.7975735205264\\
28.1861484790396	10.7973229431283\\
28.1858100128091	10.7970723585011\\
28.185471533389	10.7968217666445\\
28.1851330407782	10.796571167558\\
28.1847945349757	10.7963205612411\\
28.1844560159805	10.7960699476935\\
28.1841174837915	10.7958193269148\\
28.1837789384077	10.7955686989045\\
28.1834403798281	10.7953180636623\\
28.1831018080516	10.7950674211876\\
28.1827632230773	10.7948167714801\\
28.182424624904	10.7945661145394\\
28.1820860135308	10.794315450365\\
28.1817473889567	10.7940647789565\\
28.1814087511806	10.7938141003136\\
28.1810701002014	10.7935634144357\\
28.1807314360182	10.7933127213225\\
28.1803927586299	10.7930620209736\\
28.1800540680354	10.7928113133886\\
28.1797153642339	10.7925605985669\\
28.1793766472241	10.7923098765083\\
28.1790379170052	10.7920591472122\\
28.178699173576	10.7918084106783\\
28.1783604169356	10.7915576669062\\
28.1780216470829	10.7913069158954\\
28.1776828640168	10.7910561576455\\
28.1773440677364	10.7908053921561\\
28.1770052582406	10.7905546194268\\
28.1766664355284	10.7903038394571\\
28.1763275995987	10.7900530522467\\
28.1759887504505	10.7898022577952\\
28.1756498880829	10.789551456102\\
28.1753110124947	10.7893006471668\\
28.1749721236849	10.7890498309892\\
28.1746332216525	10.7887990075688\\
28.1742943063965	10.788548176905\\
28.1739553779158	10.7882973389976\\
28.1736164362094	10.7880464938461\\
28.1732774812762	10.7877956414501\\
28.1729385131153	10.7875447818091\\
28.1725995317256	10.7872939149228\\
28.1722605371061	10.7870430407907\\
28.1719215292557	10.7867921594124\\
28.1715825081734	10.7865412707875\\
28.1712434738581	10.7862903749156\\
28.1709044263089	10.7860394717962\\
28.1705653655247	10.7857885614289\\
28.1702262915045	10.7855376438134\\
28.1698872042472	10.7852867189491\\
28.1695481037518	10.7850357868357\\
28.1692089900173	10.7847848474728\\
28.1688698630426	10.7845339008599\\
28.1685307228267	10.7842829469966\\
28.1681915693686	10.7840319858826\\
28.1678524026672	10.7837810175173\\
28.1675132227215	10.7835300419003\\
28.1671740295305	10.7832790590313\\
28.1668348230931	10.7830280689099\\
28.1664956034083	10.7827770715355\\
28.1661563704751	10.7825260669078\\
28.1658171242923	10.7822750550264\\
28.1654778648591	10.7820240358908\\
28.1651385921743	10.7817730095007\\
28.164799306237	10.7815219758555\\
28.164460007046	10.781270934955\\
28.1641206946004	10.7810198867986\\
28.1637813688991	10.780768831386\\
28.1634420299411	10.7805177687167\\
28.1631026777253	10.7802666987903\\
28.1627633122507	10.7800156216063\\
28.1624239335163	10.7797645371645\\
28.162084541521	10.7795134454643\\
28.1617451362638	10.7792623465054\\
28.1614057177437	10.7790112402872\\
28.1610662859596	10.7787601268095\\
28.1607268409105	10.7785090060717\\
28.1603873825953	10.7782578780735\\
28.1600479110131	10.7780067428144\\
28.1597084261627	10.777755600294\\
28.1593689280432	10.7775044505119\\
28.1590294166535	10.7772532934677\\
28.1586898919925	10.7770021291609\\
28.1583503540593	10.7767509575912\\
28.1580108028527	10.7764997787581\\
28.1576712383718	10.7762485926611\\
28.1573316606155	10.7759973993\\
28.1569920695828	10.7757461986742\\
28.1566524652726	10.7754949907833\\
28.156312847684	10.775243775627\\
28.1559732168157	10.7749925532047\\
28.1556335726669	10.7747413235161\\
28.1552939152365	10.7744900865608\\
28.1549542445235	10.7742388423382\\
28.1546145605267	10.7739875908481\\
28.1542748632452	10.77373633209\\
28.1539351526779	10.7734850660635\\
28.1535954288239	10.7732337927681\\
28.1532556916819	10.7729825122034\\
28.1529159412511	10.772731224369\\
28.1525761775304	10.7724799292646\\
28.1522364005186	10.7722286268896\\
28.1518966102149	10.7719773172436\\
28.1515568066181	10.7717260003263\\
28.1512169897272	10.7714746761371\\
28.1508771595412	10.7712233446758\\
28.1505373160591	10.7709720059418\\
28.1501974592797	10.7707206599347\\
28.149857589202	10.7704693066542\\
28.1495177058251	10.7702179460998\\
28.1491778091479	10.769966578271\\
28.1488378991692	10.7697152031675\\
28.1484979758882	10.7694638207888\\
28.1481580393036	10.7692124311346\\
28.1478180894146	10.7689610342043\\
28.1474781262201	10.7687096299977\\
28.1471381497189	10.7684582185141\\
28.1467981599102	10.7682067997533\\
28.1464581567928	10.7679553737148\\
28.1461181403656	10.7677039403982\\
28.1457781106278	10.7674524998031\\
28.1454380675781	10.767201051929\\
28.1450980112156	10.7669495967756\\
28.1447579415392	10.7666981343423\\
28.1444178585479	10.7664466646288\\
28.1440777622407	10.7661951876347\\
28.1437376526164	10.7659437033595\\
28.1433975296741	10.7656922118028\\
28.1430573934127	10.7654407129642\\
28.1427172438312	10.7651892068433\\
28.1423770809285	10.7649376934397\\
28.1420369047035	10.7646861727528\\
28.1416967151554	10.7644346447824\\
28.1413565122829	10.764183109528\\
28.141016296085	10.7639315669891\\
28.1406760665608	10.7636800171653\\
28.1403358237091	10.7634284600563\\
28.1399955675289	10.7631768956616\\
28.1396552980192	10.7629253239808\\
28.139315015179	10.7626737450134\\
28.1389747190071	10.762422158759\\
28.1386344095025	10.7621705652173\\
28.1382940866643	10.7619189643877\\
28.1379537504913	10.7616673562699\\
28.1376134009825	10.7614157408634\\
28.1372730381369	10.7611641181679\\
28.1369326619533	10.7609124881829\\
28.1365922724309	10.7606608509079\\
28.1362518695685	10.7604092063426\\
28.135911453365	10.7601575544865\\
28.1355710238195	10.7599058953392\\
28.1352305809309	10.7596542289003\\
28.1348901246981	10.7594025551694\\
28.1345496551201	10.759150874146\\
28.1342091721958	10.7588991858297\\
28.1338686759242	10.7586474902202\\
28.1335281663043	10.7583957873169\\
28.1331876433351	10.7581440771194\\
28.1328471070153	10.7578923596274\\
28.1325065573441	10.7576406348403\\
28.1321659943204	10.7573889027579\\
28.131825417943	10.7571371633796\\
28.1314848282111	10.7568854167051\\
28.1311442251235	10.7566336627338\\
28.1308036086791	10.7563819014655\\
28.130462978877	10.7561301328996\\
28.130122335716	10.7558783570358\\
28.1297816791952	10.7556265738735\\
28.1294410093135	10.7553747834125\\
28.1291003260698	10.7551229856523\\
28.1287596294632	10.7548711805924\\
28.1284189194924	10.7546193682324\\
28.1280781961566	10.7543675485719\\
28.1277374594546	10.7541157216105\\
28.1273967093854	10.7538638873478\\
28.1270559459479	10.7536120457833\\
28.1267151691411	10.7533601969166\\
28.126374378964	10.7531083407473\\
28.1260335754155	10.752856477275\\
28.1256927584945	10.7526046064992\\
28.1253519282001	10.7523527284195\\
28.1250110845311	10.7521008430355\\
28.1246702274865	10.7518489503468\\
28.1243293570652	10.751597050353\\
28.1239884732663	10.7513451430535\\
28.1236475760886	10.751093228448\\
28.1233066655311	10.7508413065362\\
28.1229657415927	10.7505893773174\\
28.1226248042725	10.7503374407914\\
28.1222838535693	10.7500854969577\\
28.1219428894821	10.7498335458159\\
28.1216019120098	10.7495815873656\\
28.1212609211515	10.7493296216062\\
28.120919916906	10.7490776485375\\
28.1205788992723	10.748825668159\\
28.1202378682493	10.7485736804702\\
28.1198968238361	10.7483216854707\\
28.1195557660315	10.7480696831602\\
28.1192146948344	10.7478176735381\\
28.1188736102439	10.7475656566041\\
28.1185325122589	10.7473136323577\\
28.1181914008784	10.7470616007986\\
28.1178502761012	10.7468095619262\\
28.1175091379263	10.7465575157402\\
28.1171679863527	10.7463054622402\\
28.1168268213794	10.7460534014256\\
28.1164856430052	10.7458013332962\\
28.1161444512291	10.7455492578514\\
28.1158032460501	10.7452971750909\\
28.1154620274671	10.7450450850141\\
28.1151207954791	10.7447929876208\\
28.114779550085	10.7445408829104\\
28.1144382912837	10.7442887708826\\
28.1140970190743	10.7440366515369\\
28.1137557334555	10.7437845248729\\
28.1134144344265	10.7435323908901\\
28.1130731219861	10.7432802495882\\
28.1127317961333	10.7430281009668\\
28.1123904568671	10.7427759450253\\
28.1120491041863	10.7425237817633\\
28.1117077380899	10.7422716111805\\
28.1113663585769	10.7420194332765\\
28.1110249656462	10.7417672480507\\
28.1106835592968	10.7415150555028\\
28.1103421395275	10.7412628556323\\
28.1100007063374	10.7410106484389\\
28.1096592597254	10.740758433922\\
28.1093177996905	10.7405062120814\\
28.1089763262315	10.7402539829164\\
28.1086348393475	10.7400017464268\\
28.1082933390373	10.739749502612\\
28.1079518252999	10.7394972514718\\
28.1076102981343	10.7392449930055\\
28.1072687575394	10.7389927272129\\
28.1069272035142	10.7387404540935\\
28.1065856360575	10.7384881736468\\
28.1062440551684	10.7382358858725\\
28.1059024608457	10.73798359077\\
28.1055608530885	10.7377312883391\\
28.1052192318956	10.7374789785792\\
28.1048775972661	10.73722666149\\
28.1045359491988	10.7369743370709\\
28.1041942876927	10.7367220053217\\
28.1038526127467	10.7364696662418\\
28.1035109243598	10.7362173198308\\
28.1031692225309	10.7359649660884\\
28.102827507259	10.735712605014\\
28.102485778543	10.7354602366073\\
28.1021440363818	10.7352078608677\\
28.1018022807745	10.734955477795\\
28.1014605117198	10.7347030873887\\
28.1011187292168	10.7344506896483\\
28.1007769332645	10.7341982845735\\
28.1004351238617	10.7339458721637\\
28.1000933010074	10.7336934524186\\
28.0997514647005	10.7334410253377\\
28.09940961494	10.7331885909207\\
28.0990677517249	10.7329361491671\\
28.098725875054	10.7326837000764\\
28.0983839849263	10.7324312436482\\
28.0980420813407	10.7321787798822\\
28.0977001642963	10.7319263087778\\
28.0973582337918	10.7316738303347\\
28.0970162898263	10.7314213445525\\
28.0966743323988	10.7311688514306\\
28.096332361508	10.7309163509687\\
28.0959903771531	10.7306638431664\\
28.0956483793329	10.7304113280232\\
28.0953063680463	10.7301588055386\\
28.0949643432923	10.7299062757124\\
28.0946223050699	10.729653738544\\
28.094280253378	10.729401194033\\
28.0939381882155	10.729148642179\\
28.0935961095813	10.7288960829816\\
28.0932540174744	10.7286435164403\\
28.0929119118938	10.7283909425547\\
28.0925697928384	10.7281383613244\\
28.092227660307	10.7278857727489\\
28.0918855142988	10.7276331768279\\
28.0915433548125	10.7273805735608\\
28.0912011818471	10.7271279629474\\
28.0908589954016	10.726875344987\\
28.0905167954749	10.7266227196794\\
28.090174582066	10.7263700870241\\
28.0898323551737	10.7261174470207\\
28.0894901147971	10.7258647996687\\
28.089147860935	10.7256121449676\\
28.0888055935864	10.7253594829172\\
28.0884633127502	10.7251068135169\\
28.0881210184254	10.7248541367664\\
28.0877787106109	10.7246014526651\\
28.0874363893056	10.7243487612127\\
28.0870940545085	10.7240960624088\\
28.0867517062185	10.7238433562529\\
28.0864093444346	10.7235906427445\\
28.0860669691557	10.7233379218833\\
28.0857245803806	10.7230851936689\\
28.0853821781085	10.7228324581007\\
28.0850397623381	10.7225797151784\\
28.0846973330685	10.7223269649016\\
28.0843548902985	10.7220742072698\\
28.0840124340271	10.7218214422825\\
28.0836699642533	10.7215686699395\\
28.0833274809759	10.7213158902401\\
28.0829849841939	10.7210631031841\\
28.0826424739063	10.7208103087709\\
28.0822999501119	10.7205575070002\\
28.0819574128098	10.7203046978716\\
28.0816148619988	10.7200518813845\\
28.0812722976778	10.7197990575385\\
28.0809297198459	10.7195462263334\\
28.080587128502	10.7192933877685\\
28.0802445236449	10.7190405418435\\
28.0799019052736	10.7187876885579\\
28.0795592733871	10.7185348279114\\
28.0792166279843	10.7182819599034\\
28.078873969064	10.7180290845337\\
28.0785312966254	10.7177762018016\\
28.0781886106672	10.7175233117069\\
28.0778459111884	10.7172704142491\\
28.077503198188	10.7170175094276\\
28.0771604716649	10.7167645972423\\
28.0768177316179	10.7165116776925\\
28.0764749780461	10.7162587507778\\
28.0761322109485	10.716005816498\\
28.0757894303238	10.7157528748524\\
28.075446636171	10.7154999258407\\
28.0751038284892	10.7152469694624\\
28.0747610072771	10.7149940057171\\
28.0744181725338	10.7147410346045\\
28.0740753242582	10.714488056124\\
28.0737324624491	10.7142350702752\\
28.0733895871056	10.7139820770577\\
28.0730466982266	10.7137290764712\\
28.072703795811	10.713476068515\\
28.0723608798577	10.7132230531889\\
28.0720179503656	10.7129700304923\\
28.0716750073338	10.7127170004249\\
28.071332050761	10.7124639629862\\
28.0709890806464	10.7122109181759\\
28.0706460969887	10.7119578659933\\
28.0703030997869	10.7117048064383\\
28.06996008904	10.7114517395102\\
28.0696170647468	10.7111986652087\\
28.0692740269063	10.7109455835334\\
28.0689309755175	10.7106924944837\\
28.0685879105793	10.7104393980594\\
28.0682448320905	10.7101862942599\\
28.0679017400502	10.7099331830849\\
28.0675586344572	10.7096800645338\\
28.0672155153105	10.7094269386063\\
28.066872382609	10.709173805302\\
28.0665292363516	10.7089206646203\\
28.0661860765374	10.708667516561\\
28.0658429031651	10.7084143611235\\
28.0654997162337	10.7081611983074\\
28.0651565157422	10.7079080281123\\
28.0648133016895	10.7076548505377\\
28.0644700740745	10.7074016655833\\
28.0641268328961	10.7071484732486\\
28.0637835781533	10.7068952735331\\
28.063440309845	10.7066420664365\\
28.0630970279701	10.7063888519583\\
28.0627537325276	10.7061356300981\\
28.0624104235164	10.7058824008554\\
28.0620671009353	10.7056291642298\\
28.0617237647834	10.705375920221\\
28.0613804150595	10.7051226688283\\
28.0610370517627	10.7048694100515\\
28.0606936748917	10.7046161438901\\
28.0603502844456	10.7043628703437\\
28.0600068804233	10.7041095894117\\
28.0596634628236	10.7038563010939\\
28.0593200316456	10.7036030053898\\
28.0589765868881	10.7033497022989\\
28.0586331285501	10.7030963918208\\
28.0582896566305	10.7028430739551\\
28.0579461711282	10.7025897487013\\
28.0576026720421	10.702336416059\\
28.0572591593713	10.7020830760279\\
28.0569156331145	10.7018297286074\\
28.0565720932708	10.7015763737971\\
28.056228539839	10.7013230115966\\
28.0558849728181	10.7010696420054\\
28.0555413922069	10.7008162650232\\
28.0551977980046	10.7005628806495\\
28.0548541902098	10.7003094888838\\
28.0545105688217	10.7000560897258\\
28.054166933839	10.699802683175\\
28.0538232852608	10.699549269231\\
28.0534796230859	10.6992958478933\\
28.0531359473133	10.6990424191615\\
28.0527922579419	10.6987889830352\\
28.0524485549705	10.698535539514\\
28.0521048383983	10.6982820885973\\
28.051761108224	10.6980286302849\\
28.0514173644466	10.6977751645762\\
28.051073607065	10.6975216914708\\
28.0507298360782	10.6972682109683\\
28.050386051485	10.6970147230683\\
28.0500422532844	10.6967612277703\\
28.0496984414753	10.6965077250738\\
28.0493546160566	10.6962542149786\\
28.0490107770273	10.6960006974841\\
28.0486669243862	10.6957471725898\\
28.0483230581324	10.6954936402955\\
28.0479791782647	10.6952401006005\\
28.047635284782	10.6949865535046\\
28.0472913776832	10.6947329990073\\
28.0469474569674	10.694479437108\\
28.0466035226333	10.6942258678065\\
28.04625957468	10.6939722911023\\
28.0459156131063	10.6937187069949\\
28.0455716379112	10.6934651154839\\
28.0452276490936	10.6932115165689\\
28.0448836466524	10.6929579102494\\
28.0445396305865	10.6927042965251\\
28.0441956008949	10.6924506753954\\
28.0438515575765	10.6921970468599\\
28.0435075006301	10.6919434109183\\
28.0431634300548	10.6916897675701\\
28.0428193458494	10.6914361168148\\
28.0424752480128	10.691182458652\\
28.042131136544	10.6909287930813\\
28.0417870114419	10.6906751201023\\
28.0414428727055	10.6904214397145\\
28.0410987203335	10.6901677519175\\
28.040754554325	10.6899140567108\\
28.0404103746789	10.689660354094\\
28.0400661813941	10.6894066440668\\
28.0397219744695	10.6891529266285\\
28.039377753904	10.6888992017789\\
28.0390335196966	10.6886454695175\\
28.0386892718461	10.6883917298439\\
28.0383450103515	10.6881379827575\\
28.0380007352117	10.6878842282581\\
28.0376564464256	10.6876304663451\\
28.0373121439922	10.6873766970181\\
28.0369678279104	10.6871229202768\\
28.036623498179	10.6868691361205\\
28.036279154797	10.6866153445491\\
28.0359347977633	10.6863615455618\\
28.0355904270768	10.6861077391585\\
28.0352460427365	10.6858539253386\\
28.0349016447413	10.6856001041016\\
28.03455723309	10.6853462754472\\
28.0342128077816	10.685092439375\\
28.0338683688151	10.6848385958844\\
28.0335239161892	10.6845847449751\\
28.033179449903	10.6843308866466\\
28.0328349699554	10.6840770208985\\
28.0324904763453	10.6838231477303\\
28.0321459690715	10.6835692671417\\
28.0318014481331	10.6833153791321\\
28.0314569135289	10.6830614837012\\
28.0311123652579	10.6828075808486\\
28.0307678033189	10.6825536705737\\
28.0304232277108	10.6822997528761\\
28.0300786384327	10.6820458277555\\
28.0297340354834	10.6817918952114\\
28.0293894188618	10.6815379552433\\
28.0290447885668	10.6812840078508\\
28.0287001445974	10.6810300530335\\
28.0283554869525	10.6807760907909\\
28.0280108156309	10.6805221211227\\
28.0276661306316	10.6802681440283\\
28.0273214319535	10.6800141595074\\
28.0269767195956	10.6797601675595\\
28.0266319935567	10.6795061681842\\
28.0262872538357	10.679252161381\\
28.0259425004317	10.6789981471495\\
28.0255977333434	10.6787441254893\\
28.0252529525698	10.6784900963999\\
28.0249081581097	10.6782360598809\\
28.0245633499623	10.6779820159319\\
28.0242185281262	10.6777279645524\\
28.0238736926005	10.677473905742\\
28.0235288433841	10.6772198395003\\
28.0231839804758	10.6769657658268\\
28.0228391038746	10.6767116847211\\
28.0224942135794	10.6764575961828\\
28.0221493095891	10.6762035002114\\
28.0218043919027	10.6759493968065\\
28.021459460519	10.6756952859676\\
28.0211145154369	10.6754411676944\\
28.0207695566554	10.6751870419863\\
28.0204245841733	10.674932908843\\
28.0200795979896	10.674678768264\\
28.0197345981033	10.6744246202489\\
28.0193895845131	10.6741704647972\\
28.0190445572181	10.6739163019085\\
28.018699516217	10.6736621315824\\
28.018354461509	10.6734079538185\\
28.0180093930927	10.6731537686162\\
28.0176643109673	10.6728995759753\\
28.0173192151315	10.6726453758951\\
28.0169741055842	10.6723911683754\\
28.0166289823245	10.6721369534156\\
28.0162838453512	10.6718827310153\\
28.0159386946632	10.6716285011741\\
28.0155935302594	10.6713742638916\\
28.0152483521387	10.6711200191673\\
28.0149031603001	10.6708657670008\\
28.0145579547424	10.6706115073916\\
28.0142127354646	10.6703572403393\\
28.0138675024656	10.6701029658436\\
28.0135222557442	10.6698486839038\\
28.0131769952995	10.6695943945197\\
28.0128317211302	10.6693400976907\\
28.0124864332353	10.6690857934165\\
28.0121411316138	10.6688314816965\\
28.0117958162645	10.6685771625305\\
28.0114504871863	10.6683228359178\\
28.0111051443781	10.6680685018581\\
28.0107597878389	10.667814160351\\
28.0104144175676	10.667559811396\\
28.010069033563	10.6673054549927\\
28.0097236358241	10.6670510911406\\
28.0093782243498	10.6667967198394\\
28.0090327991389	10.6665423410885\\
28.0086873601905	10.6662879548875\\
28.0083419075033	10.6660335612361\\
28.0079964410764	10.6657791601337\\
28.0076509609086	10.6655247515799\\
28.0073054669988	10.6652703355743\\
28.0069599593459	10.6650159121165\\
28.0066144379489	10.6647614812059\\
28.0062689028066	10.6645070428423\\
28.005923353918	10.6642525970251\\
28.0055777912819	10.6639981437539\\
28.0052322148973	10.6637436830283\\
28.004886624763	10.6634892148478\\
28.004541020878	10.663234739212\\
28.0041954032412	10.6629802561205\\
28.0038497718515	10.6627257655728\\
28.0035041267078	10.6624712675684\\
28.003158467809	10.6622167621071\\
28.0028127951539	10.6619622491882\\
28.0024671087416	10.6617077288114\\
28.0021214085709	10.6614532009762\\
28.0017756946407	10.6611986656823\\
28.0014299669499	10.6609441229291\\
28.0010842254975	10.6606895727162\\
28.0007384702823	10.6604350150432\\
28.0003927013032	10.6601804499097\\
28.0000469185591	10.6599258773152\\
27.999701122049	10.6596712972592\\
27.9993553117717	10.6594167097414\\
27.9990094877262	10.6591621147613\\
27.9986636499114	10.6589075123185\\
27.9983177983261	10.6586529024125\\
27.9979719329692	10.6583982850429\\
27.9976260538397	10.6581436602092\\
27.9972801609365	10.657889027911\\
27.9969342542584	10.6576343881479\\
27.9965883338045	10.6573797409195\\
27.9962423995735	10.6571250862253\\
27.9958964515643	10.6568704240648\\
27.995550489776	10.6566157544376\\
27.9952045142073	10.6563610773433\\
27.9948585248572	10.6561063927815\\
27.9945125217246	10.6558517007517\\
27.9941665048084	10.6555970012535\\
27.9938204741075	10.6553422942864\\
27.9934744296208	10.65508757985\\
27.9931283713471	10.6548328579438\\
27.9927822992855	10.6545781285675\\
27.9924362134347	10.6543233917205\\
27.9920901137938	10.6540686474025\\
27.9917440003615	10.653813895613\\
27.9913978731369	10.6535591363516\\
27.9910517321187	10.6533043696178\\
27.9907055773059	10.6530495954112\\
27.9903594086975	10.6527948137314\\
27.9900132262922	10.6525400245778\\
27.989667030089	10.6522852279502\\
27.9893208200869	10.6520304238479\\
27.9889745962846	10.6517756122707\\
27.9886283586812	10.651520793218\\
27.9882821072754	10.6512659666895\\
27.9879358420662	10.6510111326846\\
27.9875895630526	10.650756291203\\
27.9872432702333	10.6505014422442\\
27.9868969636074	10.6502465858078\\
27.9865506431736	10.6499917218932\\
27.9862043089309	10.6497368505002\\
27.9858579608782	10.6494819716282\\
27.9855115990144	10.6492270852769\\
27.9851652233384	10.6489721914457\\
27.9848188338491	10.6487172901342\\
27.9844724305454	10.6484623813421\\
27.9841260134262	10.6482074650687\\
27.9837795824903	10.6479525413138\\
27.9834331377367	10.6476976100769\\
27.9830866791643	10.6474426713575\\
27.982740206772	10.6471877251553\\
27.9823937205587	10.6469327714696\\
27.9820472205232	10.6466778103002\\
27.9817007066645	10.6464228416466\\
27.9813541789815	10.6461678655083\\
27.981007637473	10.6459128818849\\
27.980661082138	10.645657890776\\
27.9803145129753	10.6454028921811\\
27.9799679299839	10.6451478860998\\
27.9796213331626	10.6448928725316\\
27.9792747225104	10.6446378514762\\
27.9789280980261	10.644382822933\\
27.9785814597087	10.6441277869016\\
27.978234807557	10.6438727433816\\
27.9778881415699	10.6436176923725\\
27.9775414617464	10.643362633874\\
27.9771947680852	10.6431075678855\\
27.9768480605854	10.6428524944067\\
27.9765013392458	10.642597413437\\
27.9761546040653	10.6423423249761\\
27.9758078550428	10.6420872290235\\
27.9754610921772	10.6418321255787\\
27.9751143154674	10.6415770146414\\
27.9747675249122	10.641321896211\\
27.9744207205107	10.6410667702872\\
27.9740739022616	10.6408116368695\\
27.9737270701639	10.6405564959575\\
27.9733802242164	10.6403013475507\\
27.9730333644181	10.6400461916487\\
27.9726864907678	10.639791028251\\
27.9723396032645	10.6395358573572\\
27.971992701907	10.6392806789669\\
27.9716457866943	10.6390254930796\\
27.9712988576252	10.6387702996949\\
27.9709519146986	10.6385150988123\\
27.9706049579134	10.6382598904315\\
27.9702579872685	10.6380046745519\\
27.9699110027627	10.6377494511731\\
27.9695640043951	10.6374942202947\\
27.9692169921645	10.6372389819162\\
27.9688699660697	10.6369837360372\\
27.9685229261097	10.6367284826573\\
27.9681758722833	10.636473221776\\
27.9678288045895	10.6362179533928\\
27.9674817230271	10.6359626775074\\
27.9671346275951	10.6357073941193\\
27.9667875182923	10.6354521032281\\
27.9664403951175	10.6351968048332\\
27.9660932580698	10.6349414989344\\
27.965746107148	10.634686185531\\
27.965398942351	10.6344308646227\\
27.9650517636776	10.6341755362091\\
27.9647045711269	10.6339202002897\\
27.9643573646975	10.6336648568641\\
27.9640101443886	10.6334095059317\\
27.9636629101988	10.6331541474923\\
27.9633156621272	10.6328987815453\\
27.9629684001726	10.6326434080902\\
27.9626211243339	10.6323880271268\\
27.9622738346101	10.6321326386544\\
27.9619265309999	10.6318772426728\\
27.9615792135022	10.6316218391813\\
27.9612318821161	10.6313664281797\\
27.9608845368403	10.6311110096674\\
27.9605371776737	10.630855583644\\
27.9601898046153	10.630600150109\\
27.9598424176639	10.6303447090621\\
27.9594950168184	10.6300892605028\\
27.9591476020777	10.6298338044306\\
27.9588001734408	10.6295783408451\\
27.9584527309063	10.6293228697459\\
27.9581052744734	10.6290673911324\\
27.9577578041408	10.6288119050044\\
27.9574103199074	10.6285564113612\\
27.9570628217722	10.6283009102026\\
27.956715309734	10.628045401528\\
27.9563677837917	10.627789885337\\
27.9560202439441	10.6275343616292\\
27.9556726901903	10.627278830404\\
27.955325122529	10.6270232916612\\
27.9549775409592	10.6267677454002\\
27.9546299454797	10.6265121916206\\
27.9542823360894	10.6262566303219\\
27.9539347127873	10.6260010615037\\
27.9535870755721	10.6257454851656\\
27.9532394244428	10.6254899013071\\
27.9528917593983	10.6252343099278\\
27.9525440804375	10.6249787110272\\
27.9521963875592	10.624723104605\\
27.9518486807623	10.6244674906605\\
27.9515009600458	10.6242118691935\\
27.9511532254084	10.6239562402035\\
27.9508054768491	10.6237006036899\\
27.9504577143668	10.6234449596525\\
27.9501099379604	10.6231893080906\\
27.9497621476287	10.622933649004\\
27.9494143433706	10.6226779823921\\
27.949066525185	10.6224223082546\\
27.9487186930709	10.6221666265908\\
27.948370847027	10.6219109374006\\
27.9480229870522	10.6216552406832\\
27.9476751131455	10.6213995364385\\
27.9473272253058	10.6211438246658\\
27.9469793235318	10.6208881053647\\
27.9466314078226	10.6206323785349\\
27.9462834781769	10.6203766441758\\
27.9459355345937	10.6201209022871\\
27.9455875770719	10.6198651528682\\
27.9452396056102	10.6196093959188\\
27.9448916202077	10.6193536314383\\
27.9445436208632	10.6190978594264\\
27.9441956075756	10.6188420798826\\
27.9438475803437	10.6185862928064\\
27.9434995391665	10.6183304981975\\
27.9431514840428	10.6180746960553\\
27.9428034149715	10.6178188863794\\
27.9424553319516	10.6175630691695\\
27.9421072349817	10.6173072444249\\
27.941759124061	10.6170514121454\\
27.9414109991882	10.6167955723304\\
27.9410628603622	10.6165397249795\\
27.9407147075819	10.6162838700923\\
27.9403665408462	10.6160280076683\\
27.9400183601539	10.6157721377071\\
27.939670165504	10.6155162602082\\
27.9393219568953	10.6152603751712\\
27.9389737343268	10.6150044825956\\
27.9386254977972	10.6147485824811\\
27.9382772473054	10.614492674827\\
27.9379289828505	10.6142367596331\\
27.9375807044311	10.6139808368989\\
27.9372324120463	10.6137249066239\\
27.9368841056949	10.6134689688076\\
27.9365357853757	10.6132130234497\\
27.9361874510876	10.6129570705497\\
27.9358391028296	10.6127011101071\\
27.9354907406005	10.6124451421215\\
27.9351423643992	10.6121891665925\\
27.9347939742246	10.6119331835196\\
27.9344455700755	10.6116771929024\\
27.9340971519508	10.6114211947403\\
27.9337487198494	10.6111651890331\\
27.9334002737702	10.6109091757801\\
27.9330518137121	10.6106531549811\\
27.9327033396739	10.6103971266355\\
27.9323548516545	10.6101410907429\\
27.9320063496528	10.6098850473028\\
27.9316578336677	10.6096289963149\\
27.931309303698	10.6093729377786\\
27.9309607597426	10.6091168716935\\
27.9306122018005	10.6088607980592\\
27.9302636298704	10.6086047168752\\
27.9299150439513	10.6083486281412\\
27.9295664440421	10.6080925318565\\
27.9292178301415	10.6078364280208\\
27.9288692022486	10.6075803166337\\
27.9285205603621	10.6073241976947\\
27.9281719044809	10.6070680712033\\
27.927823234604	10.6068119371591\\
27.9274745507301	10.6065557955617\\
27.9271258528583	10.6062996464106\\
27.9267771409872	10.6060434897054\\
27.926428415116	10.6057873254456\\
27.9260796752433	10.6055311536308\\
27.925730921368	10.6052749742606\\
27.9253821534892	10.6050187873344\\
27.9250333716055	10.6047625928518\\
27.924684575716	10.6045063908125\\
27.9243357658194	10.6042501812159\\
27.9239869419147	10.6039939640616\\
27.9236381040007	10.6037377393492\\
27.9232892520764	10.6034815070782\\
27.9229403861405	10.6032252672482\\
27.9225915061919	10.6029690198587\\
27.9222426122296	10.6027127649092\\
27.9218937042523	10.6024565023994\\
27.9215447822591	10.6022002323288\\
27.9211958462487	10.6019439546969\\
27.92084689622	10.6016876695033\\
27.9204979321719	10.6014313767476\\
27.9201489541032	10.6011750764292\\
27.9197999620129	10.6009187685478\\
27.9194509558999	10.6006624531029\\
27.9191019357629	10.6004061300941\\
27.9187529016008	10.6001497995209\\
27.9184038534126	10.5998934613828\\
27.9180547911971	10.5996371156795\\
27.9177057149532	10.5993807624105\\
27.9173566246798	10.5991244015752\\
27.9170075203756	10.5988680331734\\
27.9166584020397	10.5986116572045\\
27.9163092696708	10.5983552736681\\
27.9159601232678	10.5980988825637\\
27.9156109628297	10.5978424838909\\
27.9152617883553	10.5975860776492\\
27.9149125998434	10.5973296638383\\
27.9145633972929	10.5970732424576\\
27.9142141807027	10.5968168135067\\
27.9138649500717	10.5965603769852\\
27.9135157053987	10.5963039328926\\
27.9131664466827	10.5960474812285\\
27.9128171739224	10.5957910219923\\
27.9124678871167	10.5955345551838\\
27.9121185862646	10.5952780808024\\
27.9117692713649	10.5950215988476\\
27.9114199424164	10.5947651093191\\
27.9110705994181	10.5945086122163\\
27.9107212423687	10.5942521075389\\
27.9103718712673	10.5939955952864\\
27.9100224861125	10.5937390754583\\
27.9096730869034	10.5934825480542\\
27.9093236736387	10.5932260130737\\
27.9089742463174	10.5929694705163\\
27.9086248049384	10.5927129203815\\
27.9082753495004	10.5924563626689\\
27.9079258800023	10.592199797378\\
27.9075763964431	10.5919432245085\\
27.9072268988215	10.5916866440598\\
27.9068773871366	10.5914300560316\\
27.906527861387	10.5911734604233\\
27.9061783215718	10.5909168572345\\
27.9058287676897	10.5906602464648\\
27.9054791997396	10.5904036281137\\
27.9051296177204	10.5901470021807\\
27.904780021631	10.5898903686655\\
27.9044304114703	10.5896337275676\\
27.904080787237	10.5893770788865\\
27.9037311489301	10.5891204226218\\
27.9033814965484	10.588863758773\\
27.9030318300908	10.5886070873397\\
27.9026821495562	10.5883504083214\\
27.9023324549435	10.5880937217178\\
27.9019827462514	10.5878370275282\\
27.9016330234789	10.5875803257524\\
27.9012832866248	10.5873236163898\\
27.9009335356881	10.5870668994399\\
27.9005837706675	10.5868101749025\\
27.9002339915619	10.5865534427769\\
27.8998841983702	10.5862967030627\\
27.8995343910913	10.5860399557596\\
27.899184569724	10.585783200867\\
27.8988347342672	10.5855264383845\\
27.8984848847197	10.5852696683117\\
27.8981350210805	10.5850128906481\\
27.8977851433484	10.5847561053932\\
27.8974352515222	10.5844993125466\\
27.8970853456008	10.5842425121079\\
27.8967354255831	10.5839857040766\\
27.896385491468	10.5837288884523\\
27.8960355432542	10.5834720652345\\
27.8956855809407	10.5832152344227\\
27.8953356045264	10.5829583960165\\
27.8949856140101	10.5827015500155\\
27.8946356093906	10.5824446964192\\
27.8942855906668	10.5821878352272\\
27.8939355578377	10.581930966439\\
27.893585510902	10.5816740900542\\
27.8932354498586	10.5814172060723\\
27.8928853747064	10.5811603144928\\
27.8925352854442	10.5809034153154\\
27.8921851820709	10.5806465085395\\
27.8918350645854	10.5803895941648\\
27.8914849329865	10.5801326721907\\
27.8911347872731	10.5798757426168\\
27.8907846274441	10.5796188054427\\
27.8904344534983	10.579361860668\\
27.8900842654345	10.5791049082921\\
27.8897340632517	10.5788479483146\\
27.8893838469487	10.5785909807351\\
27.8890336165243	10.5783340055531\\
27.8886833719775	10.5780770227682\\
27.888333113307	10.5778200323799\\
27.8879828405118	10.5775630343878\\
27.8876325535907	10.5773060287914\\
27.8872822525426	10.5770490155902\\
27.8869319373663	10.5767919947839\\
27.8865816080606	10.576534966372\\
27.8862312646245	10.576277930354\\
27.8858809070568	10.5760208867294\\
27.8855305353564	10.5757638354979\\
27.8851801495221	10.575506776659\\
27.8848297495528	10.5752497102121\\
27.8844793354474	10.574992636157\\
27.8841289072046	10.574735554493\\
27.8837784648234	10.5744784652199\\
27.8834280083026	10.574221368337\\
27.8830775376411	10.573964263844\\
27.8827270528378	10.5737071517405\\
27.8823765538914	10.5734500320259\\
27.8820260408009	10.5731929046998\\
27.8816755135651	10.5729357697618\\
27.8813249721829	10.5726786272114\\
27.8809744166531	10.5724214770481\\
27.8806238469746	10.5721643192716\\
27.8802732631463	10.5719071538814\\
27.8799226651669	10.5716499808769\\
27.8795720530354	10.5713928002578\\
27.8792214267507	10.5711356120237\\
27.8788707863115	10.5708784161739\\
27.8785201317167	10.5706212127082\\
27.8781694629653	10.5703640016261\\
27.877818780056	10.570106782927\\
27.8774680829877	10.5698495566106\\
27.8771173717592	10.5695923226765\\
27.8767666463695	10.569335081124\\
27.8764159068173	10.5690778319529\\
27.8760651531016	10.5688205751626\\
27.8757143852212	10.5685633107527\\
27.8753636031749	10.5683060387228\\
27.8750128069616	10.5680487590723\\
27.8746619965802	10.5677914718009\\
27.8743111720294	10.5675341769081\\
27.8739603333083	10.5672768743935\\
27.8736094804155	10.5670195642565\\
27.8732586133501	10.5667622464968\\
27.8729077321107	10.5665049211138\\
27.8725568366964	10.5662475881072\\
27.8722059271058	10.5659902474765\\
27.871855003338	10.5657328992212\\
27.8715040653918	10.5654755433409\\
27.8711531132659	10.5652181798351\\
27.8708021469593	10.5649608087034\\
27.8704511664708	10.5647034299453\\
27.8701001717992	10.5644460435604\\
27.8697491629435	10.5641886495482\\
27.8693981399025	10.5639312479082\\
27.8690471026749	10.5636738386401\\
27.8686960512598	10.5634164217434\\
27.8683449856559	10.5631589972175\\
27.867993905862	10.5629015650621\\
27.8676428118771	10.5626441252767\\
27.8672917037	10.5623866778609\\
27.8669405813296	10.5621292228142\\
27.8665894447646	10.5618717601361\\
27.866238294004	10.5616142898263\\
27.8658871290466	10.5613568118841\\
27.8655359498912	10.5610993263093\\
27.8651847565368	10.5608418331013\\
27.864833548982	10.5605843322597\\
27.8644823272259	10.560326823784\\
27.8641310912673	10.5600693076738\\
27.863779841105	10.5598117839287\\
27.8634285767378	10.5595542525481\\
27.8630772981646	10.5592967135316\\
27.8627260053843	10.5590391668788\\
27.8623746983958	10.5587816125893\\
27.8620233771977	10.5585240506625\\
27.8616720417891	10.558266481098\\
27.8613206921688	10.5580089038953\\
27.8609693283356	10.5577513190541\\
27.8606179502883	10.5574937265738\\
27.8602665580259	10.557236126454\\
27.8599151515471	10.5569785186943\\
27.8595637308508	10.5567209032942\\
27.8592122959359	10.5564632802532\\
27.8588608468012	10.5562056495709\\
27.8585093834456	10.5559480112468\\
27.8581579058679	10.5556903652805\\
27.8578064140669	10.5554327116715\\
27.8574549080416	10.5551750504194\\
27.8571033877907	10.5549173815238\\
27.8567518533131	10.5546597049841\\
27.8564003046077	10.5544020207999\\
27.8560487416733	10.5541443289707\\
27.8556971645087	10.5538866294962\\
27.8553455731128	10.5536289223758\\
27.8549939674845	10.5533712076091\\
27.8546423476226	10.5531134851957\\
27.8542907135259	10.5528557551351\\
27.8539390651933	10.5525980174268\\
27.8535874026237	10.5523402720703\\
27.8532357258158	10.5520825190654\\
27.8528840347685	10.5518247584113\\
27.8525323294808	10.5515669901078\\
27.8521806099513	10.5513092141544\\
27.8518288761791	10.5510514305506\\
27.8514771281628	10.550793639296\\
27.8511253659014	10.55053584039\\
27.8507735893938	10.5502780338323\\
27.8504217986386	10.5500202196224\\
27.8500699936349	10.5497623977598\\
27.8497181743815	10.5495045682442\\
27.8493663408771	10.5492467310749\\
27.8490144931206	10.5489888862517\\
27.848662631111	10.5487310337739\\
27.848310754847	10.5484731736413\\
27.8479588643274	10.5482153058532\\
27.8476069595512	10.5479574304093\\
27.8472550405171	10.5476995473092\\
27.8469031072241	10.5474416565522\\
27.8465511596709	10.5471837581381\\
27.8461991978564	10.5469258520663\\
27.8458472217794	10.5466679383364\\
27.8454952314388	10.546410016948\\
27.8451432268335	10.5461520879005\\
27.8447912079622	10.5458941511935\\
27.8444391748239	10.5456362068266\\
27.8440871274173	10.5453782547993\\
27.8437350657413	10.5451202951112\\
27.8433829897947	10.5448623277618\\
27.8430308995765	10.5446043527506\\
27.8426787950853	10.5443463700772\\
27.8423266763202	10.5440883797412\\
27.8419745432799	10.543830381742\\
27.8416223959632	10.5435723760793\\
27.841270234369	10.5433143627525\\
27.8409180584962	10.5430563417613\\
27.8405658683436	10.5427983131051\\
27.84021366391	10.5425402767835\\
27.8398614451943	10.5422822327961\\
27.8395092121952	10.5420241811424\\
27.8391569649118	10.5417661218219\\
27.8388047033427	10.5415080548342\\
27.8384524274869	10.5412499801788\\
27.8381001373431	10.5409918978553\\
27.8377478329103	10.5407338078632\\
27.8373955141872	10.5404757102021\\
27.8370431811728	10.5402176048715\\
27.8366908338657	10.5399594918709\\
27.836338472265	10.5397013711999\\
27.8359860963694	10.5394432428581\\
27.8356337061777	10.5391851068449\\
27.8352813016888	10.53892696316\\
27.8349288829016	10.5386688118028\\
27.8345764498149	10.538410652773\\
27.8342240024274	10.53815248607\\
27.8338715407382	10.5378943116934\\
27.8335190647459	10.5376361296428\\
27.8331665744495	10.5373779399176\\
27.8328140698478	10.5371197425175\\
27.8324615509395	10.5368615374419\\
27.8321090177237	10.5366033246905\\
27.831756470199	10.5363451042628\\
27.8314039083643	10.5360868761582\\
27.8310513322186	10.5358286403764\\
27.8306987417605	10.5355703969169\\
27.830346136989	10.5353121457792\\
27.8299935179028	10.535053886963\\
27.8296408845009	10.5347956204676\\
27.8292882367821	10.5345373462927\\
27.8289355747451	10.5342790644378\\
27.8285828983889	10.5340207749025\\
27.8282302077122	10.5337624776863\\
27.827877502714	10.5335041727887\\
27.827524783393	10.5332458602093\\
27.8271720497481	10.5329875399476\\
27.8268193017781	10.5327292120032\\
27.8264665394818	10.5324708763756\\
27.8261137628582	10.5322125330644\\
27.825760971906	10.531954182069\\
27.825408166624	10.5316958233891\\
27.8250553470111	10.5314374570242\\
27.8247025130662	10.5311790829739\\
27.824349664788	10.5309207012375\\
27.8239968021755	10.5306623118148\\
27.8236439252274	10.5304039147053\\
27.8232910339425	10.5301455099085\\
27.8229381283198	10.5298870974239\\
27.822585208358	10.529628677251\\
27.8222322740561	10.5293702493895\\
27.8218793254127	10.5291118138389\\
27.8215263624267	10.5288533705987\\
27.8211733850971	10.5285949196684\\
27.8208203934226	10.5283364610477\\
27.820467387402	10.528077994736\\
27.8201143670342	10.5278195207328\\
27.819761332318	10.5275610390378\\
27.8194082832523	10.5273025496504\\
27.8190552198359	10.5270440525703\\
27.8187021420676	10.5267855477969\\
27.8183490499462	10.5265270353298\\
27.8179959434707	10.5262685151685\\
27.8176428226397	10.5260099873126\\
27.8172896874522	10.5257514517616\\
27.816936537907	10.5254929085151\\
27.8165833740029	10.5252343575725\\
27.8162301957388	10.5249757989336\\
27.8158770031134	10.5247172325977\\
27.8155237961256	10.5244586585644\\
27.8151705747743	10.5242000768333\\
27.8148173390583	10.5239414874039\\
27.8144640889764	10.5236828902757\\
27.8141108245274	10.5234242854484\\
27.8137575457102	10.5231656729214\\
27.8134042525236	10.5229070526942\\
27.8130509449664	10.5226484247665\\
27.8126976230375	10.5223897891377\\
27.8123442867357	10.5221311458075\\
27.8119909360598	10.5218724947752\\
27.8116375710087	10.5216138360406\\
27.8112841915812	10.5213551696031\\
27.810930797776	10.5210964954622\\
27.8105773895922	10.5208378136176\\
27.8102239670284	10.5205791240687\\
27.8098705300835	10.5203204268151\\
27.8095170787563	10.5200617218563\\
27.8091636130457	10.5198030091919\\
27.8088101329505	10.5195442888214\\
27.8084566384695	10.5192855607443\\
27.8081031296016	10.5190268249602\\
27.8077496063456	10.5187680814687\\
27.8073960687002	10.5185093302693\\
27.8070425166645	10.5182505713614\\
27.806688950237	10.5179918047448\\
27.8063353694168	10.5177330304188\\
27.8059817742026	10.517474248383\\
27.8056281645933	10.5172154586371\\
27.8052745405876	10.5169566611804\\
27.8049209021844	10.5166978560126\\
27.8045672493826	10.5164390431333\\
27.8042135821809	10.5161802225418\\
27.8038599005783	10.5159213942378\\
27.8035062045734	10.5156625582209\\
27.8031524941652	10.5154037144905\\
27.8027987693525	10.5151448630462\\
27.802445030134	10.5148860038876\\
27.8020912765087	10.5146271370141\\
27.8017375084753	10.5143682624254\\
27.8013837260327	10.5141093801209\\
27.8010299291797	10.5138504901002\\
27.8006761179152	10.5135915923629\\
27.8003222922379	10.5133326869084\\
27.7999684521467	10.5130737737364\\
27.7996145976403	10.5128148528463\\
27.7992607287178	10.5125559242378\\
27.7989068453777	10.5122969879102\\
27.7985529476191	10.5120380438633\\
27.7981990354407	10.5117790920965\\
27.7978451088413	10.5115201326093\\
27.7974911678197	10.5112611654013\\
27.7971372123748	10.5110021904721\\
27.7967832425055	10.5107432078212\\
27.7964292582105	10.510484217448\\
27.7960752594886	10.5102252193523\\
27.7957212463387	10.5099662135334\\
27.7953672187596	10.509707199991\\
27.7950131767501	10.5094481787245\\
27.7946591203091	10.5091891497336\\
27.7943050494354	10.5089301130177\\
27.7939509641277	10.5086710685764\\
27.793596864385	10.5084120164092\\
27.793242750206	10.5081529565158\\
27.7928886215896	10.5078938888955\\
27.7925344785345	10.507634813548\\
27.7921803210397	10.5073757304727\\
27.7918261491039	10.5071166396693\\
27.791471962726	10.5068575411373\\
27.7911177619047	10.5065984348762\\
27.790763546639	10.5063393208855\\
27.7904093169275	10.5060801991648\\
27.7900550727693	10.5058210697136\\
27.7897008141629	10.5055619325315\\
27.7893465411074	10.505302787618\\
27.7889922536014	10.5050436349726\\
27.7886379516439	10.5047844745949\\
27.7882836352337	10.5045253064844\\
27.7879293043695	10.5042661306407\\
27.7875749590502	10.5040069470633\\
27.7872205992746	10.5037477557517\\
27.7868662250416	10.5034885567054\\
27.7865118363499	10.5032293499241\\
27.7861574331983	10.5029701354072\\
27.7858030155858	10.5027109131543\\
27.7854485835111	10.502451683165\\
27.785094136973	10.5021924454387\\
27.7847396759703	10.501933199975\\
27.784385200502	10.5016739467734\\
27.7840307105667	10.5014146858335\\
27.7836762061633	10.5011554171548\\
27.7833216872906	10.5008961407369\\
27.7829671539476	10.5006368565793\\
27.7826126061328	10.5003775646815\\
27.7822580438453	10.500118265043\\
27.7819034670837	10.4998589576635\\
27.781548875847	10.4995996425424\\
27.7811942701339	10.4993403196793\\
27.7808396499433	10.4990809890737\\
27.7804850152739	10.4988216507251\\
27.7801303661246	10.4985623046332\\
27.7797757024943	10.4983029507973\\
27.7794210243816	10.4980435892172\\
27.7790663317856	10.4977842198922\\
27.7787116247048	10.497524842822\\
27.7783569031383	10.4972654580061\\
27.7780021670847	10.497006065444\\
27.777647416543	10.4967466651352\\
27.7772926515119	10.4964872570793\\
27.7769378719902	10.4962278412759\\
27.7765830779768	10.4959684177244\\
27.7762282694705	10.4957089864244\\
27.77587344647	10.4954495473755\\
27.7755186089743	10.4951901005772\\
27.7751637569821	10.4949306460289\\
27.7748088904922	10.4946711837303\\
27.7744540095035	10.4944117136809\\
27.7740991140148	10.4941522358803\\
27.7737442040248	10.4938927503278\\
27.7733892795324	10.4936332570232\\
27.7730343405365	10.4933737559659\\
27.7726793870358	10.4931142471555\\
27.7723244190291	10.4928547305915\\
27.7719694365153	10.4925952062735\\
27.7716144394932	10.4923356742009\\
27.7712594279615	10.4920761343734\\
27.7709044019191	10.4918165867904\\
27.7705493613649	10.4915570314515\\
27.7701943062976	10.4912974683562\\
27.769839236716	10.4910378975041\\
27.769484152619	10.4907783188947\\
27.7691290540053	10.4905187325275\\
27.7687739408739	10.4902591384021\\
27.7684188132234	10.489999536518\\
27.7680636710527	10.4897399268748\\
27.7677085143606	10.4894803094719\\
27.767353343146	10.489220684309\\
27.7669981574076	10.4889610513855\\
27.7666429571443	10.488701410701\\
27.7662877423549	10.488441762255\\
27.7659325130381	10.4881821060471\\
27.7655772691928	10.4879224420768\\
27.7652220108178	10.4876627703437\\
27.7648667379119	10.4874030908472\\
27.764511450474	10.4871434035869\\
27.7641561485028	10.4868837085623\\
27.7638008319971	10.486624005773\\
27.7634455009558	10.4863642952186\\
27.7630901553776	10.4861045768985\\
27.7627347952615	10.4858448508122\\
27.7623794206061	10.4855851169594\\
27.7620240314103	10.4853253753396\\
27.761668627673	10.4850656259522\\
27.7613132093928	10.4848058687969\\
27.7609577765687	10.4845461038731\\
27.7606023291994	10.4842863311804\\
27.7602468672838	10.4840265507184\\
27.7598913908206	10.4837667624865\\
27.7595358998087	10.4835069664843\\
27.7591803942469	10.4832471627114\\
27.758824874134	10.4829873511672\\
27.7584693394687	10.4827275318514\\
27.75811379025	10.4824677047634\\
27.7577582264766	10.4822078699027\\
27.7574026481472	10.481948027269\\
27.7570470552609	10.4816881768617\\
27.7566914478162	10.4814283186804\\
27.7563358258121	10.4811684527247\\
27.7559801892474	10.4809085789939\\
27.7556245381208	10.4806486974878\\
27.7552688724311	10.4803888082058\\
27.7549131921773	10.4801289111474\\
27.754557497358	10.4798690063123\\
27.7542017879721	10.4796090936998\\
27.7538460640184	10.4793491733097\\
27.7534903254957	10.4790892451413\\
27.7531345724028	10.4788293091942\\
27.7527788047386	10.4785693654681\\
27.7524230225017	10.4783094139623\\
27.7520672256911	10.4780494546764\\
27.7517114143055	10.47778948761\\
27.7513555883438	10.4775295127626\\
27.7509997478047	10.4772695301338\\
27.750643892687	10.477009539723\\
27.7502880229896	10.4767495415298\\
27.7499321387113	10.4764895355538\\
27.7495762398508	10.4762295217944\\
27.749220326407	10.4759695002512\\
27.7488643983787	10.4757094709238\\
27.7485084557646	10.4754494338116\\
27.7481524985637	10.4751893889143\\
27.7477965267746	10.4749293362313\\
27.7474405403962	10.4746692757622\\
27.7470845394273	10.4744092075064\\
27.7467285238667	10.4741491314637\\
27.7463724937133	10.4738890476334\\
27.7460164489657	10.4736289560151\\
27.7456603896228	10.4733688566083\\
27.7453043156835	10.4731087494126\\
27.7449482271465	10.4728486344276\\
27.7445921240106	10.4725885116527\\
27.7442360062746	10.4723283810874\\
27.7438798739374	10.4720682427314\\
27.7435237269976	10.4718080965841\\
27.7431675654543	10.4715479426451\\
27.742811389306	10.4712877809139\\
27.7424551985517	10.47102761139\\
27.7420989931901	10.470767434073\\
27.7417427732201	10.4705072489625\\
27.7413865386404	10.4702470560578\\
27.7410302894499	10.4699868553587\\
27.7406740256473	10.4697266468646\\
27.7403177472314	10.469466430575\\
27.7399614542011	10.4692062064894\\
27.7396051465551	10.4689459746075\\
27.7392488242923	10.4686857349287\\
27.7388924874115	10.4684254874526\\
27.7385361359114	10.4681652321787\\
27.7381797697908	10.4679049691066\\
27.7378233890486	10.4676446982357\\
27.7374669936836	10.4673844195656\\
27.7371105836945	10.4671241330959\\
27.7367541590801	10.466863838826\\
27.7363977198393	10.4666035367555\\
27.7360412659709	10.466343226884\\
27.7356847974736	10.4660829092109\\
27.7353283143463	10.4658225837358\\
27.7349718165877	10.4655622504583\\
27.7346153041966	10.4653019093778\\
27.7342587771719	10.4650415604939\\
27.7339022355124	10.4647812038061\\
27.7335456792168	10.464520839314\\
27.7331891082839	10.4642604670171\\
27.7328325227126	10.4640000869149\\
27.7324759225016	10.463739699007\\
27.7321193076497	10.4634793032929\\
27.7317626781558	10.463218899772\\
27.7314060340186	10.4629584884441\\
27.7310493752369	10.4626980693085\\
27.7306927018096	10.4624376423648\\
27.7303360137354	10.4621772076126\\
27.7299793110131	10.4619167650514\\
27.7296225936415	10.4616563146806\\
27.7292658616194	10.4613958564999\\
27.7289091149457	10.4611353905088\\
27.728552353619	10.4608749167068\\
27.7281955776383	10.4606144350934\\
27.7278387870022	10.4603539456682\\
27.7274819817097	10.4600934484307\\
27.7271251617594	10.4598329433805\\
27.7267683271502	10.4595724305169\\
27.7264114778809	10.4593119098397\\
27.7260546139503	10.4590513813483\\
27.7256977353572	10.4587908450423\\
27.7253408421003	10.4585303009212\\
27.7249839341785	10.4582697489844\\
27.7246270115905	10.4580091892317\\
27.7242700743352	10.4577486216624\\
27.7239131224113	10.4574880462761\\
27.7235561558177	10.4572274630723\\
27.7231991745531	10.4569668720507\\
27.7228421786164	10.4567062732106\\
27.7224851680062	10.4564456665517\\
27.7221281427215	10.4561850520734\\
27.721771102761	10.4559244297753\\
27.7214140481235	10.455663799657\\
27.7210569788078	10.4554031617179\\
27.7206998948127	10.4551425159576\\
27.7203427961369	10.4548818623757\\
27.7199856827794	10.4546212009716\\
27.7196285547388	10.4543605317448\\
27.7192714120139	10.454099854695\\
27.7189142546037	10.4538391698216\\
27.7185570825067	10.4535784771242\\
27.7181998957219	10.4533177766023\\
27.7178426942481	10.4530570682554\\
27.7174854780839	10.4527963520831\\
27.7171282472283	10.4525356280849\\
27.71677100168	10.4522748962603\\
27.7164137414377	10.4520141566088\\
27.7160564665004	10.4517534091301\\
27.7156991768667	10.4514926538235\\
27.7153418725355	10.4512318906887\\
27.7149845535056	10.4509711197252\\
27.7146272197757	10.4507103409325\\
27.7142698713446	10.4504495543101\\
27.7139125082112	10.4501887598576\\
27.7135551303742	10.4499279575744\\
27.7131977378324	10.4496671474602\\
27.7128403305847	10.4494063295144\\
27.7124829086297	10.4491455037367\\
27.7121254719662	10.4488846701264\\
27.7117680205932	10.4486238286832\\
27.7114105545093	10.4483629794065\\
27.7110530737134	10.448102122296\\
27.7106955782042	10.447841257351\\
27.7103380679805	10.4475803845713\\
27.7099805430411	10.4473195039562\\
27.7096230033849	10.4470586155053\\
27.7092654490105	10.4467977192182\\
27.7089078799168	10.4465368150944\\
27.7085502961026	10.4462759031333\\
27.7081926975666	10.4460149833346\\
27.7078350843077	10.4457540556978\\
27.7074774563246	10.4454931202224\\
27.7071198136162	10.4452321769079\\
27.7067621561811	10.4449712257538\\
27.7064044840182	10.4447102667597\\
27.7060467971263	10.4444492999251\\
27.7056890955042	10.4441883252496\\
27.7053313791506	10.4439273427326\\
27.7049736480644	10.4436663523737\\
27.7046159022443	10.4434053541725\\
27.7042581416891	10.4431443481284\\
27.7039003663976	10.4428833342409\\
27.7035425763686	10.4426223125097\\
27.7031847716008	10.4423612829343\\
27.7028269520932	10.442100245514\\
27.7024691178443	10.4418392002486\\
27.7021112688531	10.4415781471375\\
27.7017534051183	10.4413170861803\\
27.7013955266387	10.4410560173764\\
27.7010376334131	10.4407949407255\\
27.7006797254402	10.440533856227\\
27.7003218027189	10.4402727638804\\
27.6999638652479	10.4400116636854\\
27.6996059130261	10.4397505556414\\
27.6992479460521	10.4394894397479\\
27.6988899643249	10.4392283160045\\
27.6985319678431	10.4389671844107\\
27.6981739566056	10.438706044966\\
27.6978159306111	10.43844489767\\
27.6974578898584	10.4381837425222\\
27.6970998343464	10.4379225795221\\
27.6967417640737	10.4376614086692\\
27.6963836790392	10.4374002299632\\
27.6960255792416	10.4371390434034\\
27.6956674646798	10.4368778489894\\
27.6953093353526	10.4366166467208\\
27.6949511912586	10.4363554365971\\
27.6945930323967	10.4360942186178\\
27.6942348587657	10.4358329927825\\
27.6938766703643	10.4355717590906\\
27.6935184671913	10.4353105175417\\
27.6931602492456	10.4350492681353\\
27.6928020165258	10.4347880108709\\
27.6924437690309	10.4345267457482\\
27.6920855067594	10.4342654727665\\
27.6917272297104	10.4340041919255\\
27.6913689378824	10.4337429032246\\
27.6910106312744	10.4334816066634\\
27.690652309885	10.4332203022415\\
27.6902939737131	10.4329589899582\\
27.6899356227574	10.4326976698133\\
27.6895772570167	10.4324363418061\\
27.6892188764898	10.4321750059362\\
27.6888604811756	10.4319136622032\\
27.6885020710726	10.4316523106065\\
27.6881436461799	10.4313909511458\\
27.687785206496	10.4311295838205\\
27.6874267520198	10.4308682086301\\
27.6870682827501	10.4306068255742\\
27.6867097986857	10.4303454346523\\
27.6863512998253	10.430084035864\\
27.6859927861677	10.4298226292087\\
27.6856342577117	10.429561214686\\
27.6852757144561	10.4292997922954\\
27.6849171563996	10.4290383620364\\
27.684558583541	10.4287769239086\\
27.6841999958792	10.4285154779115\\
27.6838413934128	10.4282540240447\\
27.6834827761407	10.4279925623075\\
27.6831241440616	10.4277310926996\\
27.6827654971743	10.4274696152206\\
27.6824068354776	10.4272081298698\\
27.6820481589703	10.4269466366469\\
27.6816894676512	10.4266851355514\\
27.6813307615189	10.4264236265827\\
27.6809720405724	10.4261621097405\\
27.6806133048103	10.4259005850243\\
27.6802545542315	10.4256390524335\\
27.6798957888347	10.4253775119677\\
27.6795370086187	10.4251159636265\\
27.6791782135822	10.4248544074093\\
27.6788194037242	10.4245928433156\\
27.6784605790432	10.4243312713451\\
27.6781017395382	10.4240696914973\\
27.6777428852078	10.4238081037715\\
27.6773840160509	10.4235465081675\\
27.6770251320662	10.4232849046847\\
27.6766662332525	10.4230232933226\\
27.6763073196086	10.4227616740807\\
27.6759483911333	10.4225000469587\\
27.6755894478252	10.4222384119559\\
27.6752304896833	10.421976769072\\
27.6748715167063	10.4217151183065\\
27.6745125288929	10.4214534596588\\
27.6741535262419	10.4211917931285\\
27.6737945087521	10.4209301187152\\
27.6734354764223	10.4206684364183\\
27.6730764292512	10.4204067462375\\
27.6727173672376	10.4201450481721\\
27.6723582903803	10.4198833422217\\
27.6719991986781	10.419621628386\\
27.6716400921297	10.4193599066643\\
27.6712809707339	10.4190981770562\\
27.6709218344895	10.4188364395612\\
27.6705626833952	10.4185746941789\\
27.6702035174499	10.4183129409088\\
27.6698443366522	10.4180511797504\\
27.669485141001	10.4177894107033\\
27.6691259304951	10.4175276337668\\
27.6687667051331	10.4172658489407\\
27.6684074649139	10.4170040562244\\
27.6680482098363	10.4167422556174\\
27.667688939899	10.4164804471192\\
27.6673296551008	10.4162186307295\\
27.6669703554404	10.4159568064476\\
27.6666110409167	10.4156949742732\\
27.6662517115284	10.4154331342057\\
27.6658923672742	10.4151712862447\\
27.665533008153	10.4149094303896\\
27.6651736341635	10.4146475666402\\
27.6648142453045	10.4143856949957\\
27.6644548415747	10.4141238154558\\
27.664095422973	10.4138619280201\\
27.663735989498	10.4136000326879\\
27.6633765411486	10.4133381294589\\
27.6630170779235	10.4130762183325\\
27.6626575998215	10.4128142993084\\
27.6622981068414	10.4125523723859\\
27.6619385989819	10.4122904375647\\
27.6615790762418	10.4120284948443\\
27.6612195386199	10.4117665442241\\
27.6608599861149	10.4115045857037\\
27.6605004187256	10.4112426192827\\
27.6601408364508	10.4109806449606\\
27.6597812392892	10.4107186627368\\
27.6594216272396	10.4104566726109\\
27.6590620003008	10.4101946745824\\
27.6587023584716	10.4099326686509\\
27.6583427017506	10.4096706548158\\
27.6579830301367	10.4094086330768\\
27.6576233436287	10.4091466034332\\
27.6572636422253	10.4088845658848\\
27.6569039259252	10.4086225204308\\
27.6565441947273	10.408360467071\\
27.6561844486303	10.4080984058047\\
27.655824687633	10.4078363366316\\
27.655464911734	10.4075742595512\\
27.6551051209323	10.4073121745629\\
27.6547453152266	10.4070500816664\\
27.6543854946156	10.406787980861\\
27.6540256590981	10.4065258721464\\
27.6536658086729	10.4062637555221\\
27.6533059433386	10.4060016309875\\
27.6529460630942	10.4057394985423\\
27.6525861679383	10.4054773581859\\
27.6522262578698	10.4052152099179\\
27.6518663328873	10.4049530537378\\
27.6515063929897	10.404690889645\\
27.6511464381757	10.4044287176392\\
27.6507864684441	10.4041665377198\\
27.6504264837936	10.4039043498864\\
27.650066484223	10.4036421541385\\
27.6497064697311	10.4033799504756\\
27.6493464403166	10.4031177388973\\
27.6489863959783	10.402855519403\\
27.648626336715	10.4025932919923\\
27.6482662625253	10.4023310566647\\
27.6479061734082	10.4020688134197\\
27.6475460693623	10.4018065622568\\
27.6471859503864	10.4015443031757\\
27.6468258164792	10.4012820361757\\
27.6464656676396	10.4010197612564\\
27.6461055038663	10.4007574784174\\
27.645745325158	10.4004951876581\\
27.6453851315136	10.4002328889781\\
27.6450249229317	10.3999705823769\\
27.6446646994111	10.3997082678541\\
27.6443044609507	10.399445945409\\
27.6439442075491	10.3991836150414\\
27.6435839392051	10.3989212767506\\
27.6432236559175	10.3986589305362\\
27.642863357685	10.3983965763978\\
27.6425030445064	10.3981342143348\\
27.6421427163805	10.3978718443467\\
27.641782373306	10.3976094664332\\
27.6414220152817	10.3973470805937\\
27.6410616423063	10.3970846868277\\
27.6407012543786	10.3968222851347\\
27.6403408514974	10.3965598755144\\
27.6399804336613	10.3962974579661\\
27.6396200008693	10.3960350324895\\
27.63925955312	10.395772599084\\
27.6388990904121	10.3955101577491\\
27.6385386127445	10.3952477084844\\
27.638178120116	10.3949852512895\\
27.6378176125252	10.3947227861637\\
27.6374570899709	10.3944603131067\\
27.6370965524518	10.394197832118\\
27.6367359999669	10.3939353431971\\
27.6363754325147	10.3936728463434\\
27.6360148500941	10.3934103415566\\
27.6356542527037	10.3931478288361\\
27.6352936403425	10.3928853081815\\
27.6349330130091	10.3926227795923\\
27.6345723707022	10.392360243068\\
27.6342117134207	10.3920976986081\\
27.6338510411633	10.3918351462121\\
27.6334903539287	10.3915725858797\\
27.6331296517157	10.3913100176102\\
27.6327689345231	10.3910474414032\\
27.6324082023496	10.3907848572583\\
27.632047455194	10.3905222651748\\
27.631686693055	10.3902596651525\\
27.6313259159314	10.3899970571908\\
27.630965123822	10.3897344412891\\
27.6306043167254	10.3894718174471\\
27.6302434946405	10.3892091856643\\
27.629882657566	10.3889465459401\\
27.6295218055006	10.3886838982741\\
27.6291609384431	10.3884212426658\\
27.6288000563924	10.3881585791147\\
27.628439159347	10.3878959076204\\
27.6280782473058	10.3876332281823\\
27.6277173202675	10.3873705408\\
27.627356378231	10.387107845473\\
27.6269954211948	10.3868451422009\\
27.6266344491579	10.3865824309831\\
27.6262734621188	10.3863197118191\\
27.6259124600765	10.3860569847086\\
27.6255514430296	10.3857942496509\\
27.625190410977	10.3855315066457\\
27.6248293639172	10.3852687556924\\
27.6244683018492	10.3850059967905\\
27.6241072247717	10.3847432299397\\
27.6237461326833	10.3844804551393\\
27.6233850255829	10.3842176723889\\
27.6230239034692	10.3839548816881\\
27.622662766341	10.3836920830364\\
27.622301614197	10.3834292764332\\
27.621940447036	10.3831664618781\\
27.6215792648567	10.3829036393706\\
27.6212180676578	10.3826408089102\\
27.6208568554382	10.3823779704965\\
27.6204956281965	10.382115124129\\
27.6201343859316	10.3818522698071\\
27.6197731286421	10.3815894075305\\
27.6194118563269	10.3813265372985\\
27.6190505689846	10.3810636591108\\
27.618689266614	10.3808007729669\\
27.618327949214	10.3805378788662\\
27.6179666167831	10.3802749768084\\
27.6176052693202	10.3800120667928\\
27.617243906824	10.3797491488191\\
27.6168825292933	10.3794862228867\\
27.6165211367268	10.3792232889952\\
27.6161597291233	10.3789603471441\\
27.6157983064815	10.3786973973329\\
27.6154368688001	10.3784344395611\\
27.615075416078	10.3781714738283\\
27.6147139483138	10.3779085001339\\
27.6143524655064	10.3776455184775\\
27.6139909676543	10.3773825288586\\
27.6136294547565	10.3771195312767\\
27.6132679268117	10.3768565257313\\
27.6129063838185	10.376593512222\\
27.6125448257758	10.3763304907482\\
27.6121832526822	10.3760674613095\\
27.6118216645366	10.3758044239055\\
27.6114600613377	10.3755413785355\\
27.6110984430843	10.3752783251992\\
27.610736809775	10.375015263896\\
27.6103751614086	10.3747521946255\\
27.6100134979839	10.3744891173872\\
27.6096518194996	10.3742260321806\\
27.6092901259545	10.3739629390052\\
27.6089284173473	10.3736998378606\\
27.6085666936767	10.3734367287462\\
27.6082049549416	10.3731736116615\\
27.6078432011405	10.3729104866062\\
27.6074814322724	10.3726473535797\\
27.6071196483359	10.3723842125815\\
27.6067578493298	10.3721210636112\\
27.6063960352528	10.3718579066682\\
27.6060342061037	10.3715947417521\\
27.6056723618812	10.3713315688624\\
27.605310502584	10.3710683879985\\
27.604948628211	10.3708051991602\\
27.6045867387607	10.3705420023467\\
27.6042248342321	10.3702787975577\\
27.6038629146238	10.3700155847927\\
27.6035009799346	10.3697523640512\\
27.6031390301632	10.3694891353327\\
27.6027770653083	10.3692258986368\\
27.6024150853688	10.3689626539628\\
27.6020530903433	10.3686994013105\\
27.6016910802306	10.3684361406792\\
27.6013290550294	10.3681728720684\\
27.6009670147385	10.3679095954779\\
27.6006049593566	10.3676463109069\\
27.6002428888825	10.367383018355\\
27.5998808033148	10.3671197178219\\
27.5995187026524	10.3668564093069\\
27.599156586894	10.3665930928095\\
27.5987944560383	10.3663297683294\\
27.5984323100841	10.366066435866\\
27.5980701490301	10.3658030954189\\
27.597707972875	10.3655397469875\\
27.5973457816176	10.3652763905713\\
27.5969835752566	10.36501302617\\
27.5966213537908	10.3647496537829\\
27.5962591172189	10.3644862734097\\
27.5958968655396	10.3642228850498\\
27.5955345987518	10.3639594887028\\
27.595172316854	10.3636960843681\\
27.5948100198451	10.3634326720452\\
27.5944477077239	10.3631692517338\\
27.5940853804889	10.3629058234333\\
27.5937230381391	10.3626423871432\\
27.5933606806731	10.3623789428631\\
27.5929983080896	10.3621154905924\\
27.5926359203875	10.3618520303306\\
27.5922735175653	10.3615885620774\\
27.591911099622	10.3613250858322\\
27.5915486665561	10.3610616015944\\
27.5911862183666	10.3607981093637\\
27.590823755052	10.3605346091396\\
27.5904612766111	10.3602711009215\\
27.5900987830427	10.360007584709\\
27.5897362743455	10.3597440605016\\
27.5893737505182	10.3594805282987\\
27.5890112115596	10.3592169881\\
27.5886486574684	10.358953439905\\
27.5882860882434	10.3586898837131\\
27.5879235038833	10.3584263195238\\
27.5875609043868	10.3581627473368\\
27.5871982897526	10.3578991671514\\
27.5868356599795	10.3576355789672\\
27.5864730150663	10.3573719827838\\
27.5861103550116	10.3571083786006\\
27.5857476798143	10.3568447664171\\
27.5853849894729	10.3565811462329\\
27.5850222839864	10.3563175180475\\
27.5846595633533	10.3560538818604\\
27.5842968275725	10.3557902376711\\
27.5839340766427	10.3555265854791\\
27.5835713105626	10.355262925284\\
27.5832085293309	10.3549992570852\\
27.5828457329464	10.3547355808823\\
27.5824829214078	10.3544718966747\\
27.5821200947139	10.3542082044621\\
27.5817572528634	10.3539445042438\\
27.5813943958549	10.3536807960195\\
27.5810315236873	10.3534170797886\\
27.5806686363593	10.3531533555506\\
27.5803057338697	10.3528896233052\\
27.579942816217	10.3526258830517\\
27.5795798834002	10.3523621347897\\
27.5792169354178	10.3520983785186\\
27.5788539722687	10.3518346142382\\
27.5784909939516	10.3515708419477\\
27.5781280004652	10.3513070616468\\
27.5777649918083	10.3510432733349\\
27.5774019679795	10.3507794770116\\
27.5770389289776	10.3505156726764\\
27.5766758748014	10.3502518603288\\
27.5763128054495	10.3499880399683\\
27.5759497209208	10.3497242115945\\
27.5755866212138	10.3494603752067\\
27.5752235063275	10.3491965308046\\
27.5748603762604	10.3489326783877\\
27.5744972310113	10.3486688179555\\
27.5741340705791	10.3484049495074\\
27.5737708949623	10.3481410730431\\
27.5734077041597	10.347877188562\\
27.57304449817	10.3476132960635\\
27.572681276992	10.3473493955474\\
27.5723180406245	10.347085487013\\
27.5719547890661	10.3468215704598\\
27.5715915223155	10.3465576458875\\
27.5712282403715	10.3462937132954\\
27.5708649432329	10.3460297726831\\
27.5705016308983	10.3457658240502\\
27.5701383033665	10.3455018673961\\
27.5697749606362	10.3452379027203\\
27.5694116027061	10.3449739300224\\
27.569048229575	10.3447099493018\\
27.5686848412416	10.3444459605581\\
27.5683214377046	10.3441819637909\\
27.5679580189628	10.3439179589995\\
27.5675945850148	10.3436539461835\\
27.5672311358594	10.3433899253425\\
27.5668676714954	10.3431258964759\\
27.5665041919214	10.3428618595833\\
27.5661406971363	10.3425978146641\\
27.5657771871386	10.3423337617179\\
27.5654136619272	10.3420697007442\\
27.5650501215007	10.3418056317425\\
27.5646865658579	10.3415415547123\\
27.5643229949976	10.3412774696531\\
27.5639594089183	10.3410133765645\\
27.563595807619	10.3407492754459\\
27.5632321910983	10.3404851662968\\
27.5628685593548	10.3402210491169\\
27.5625049123875	10.3399569239055\\
27.5621412501948	10.3396927906622\\
27.5617775727757	10.3394286493865\\
27.5614138801288	10.339164500078\\
27.5610501722529	10.338900342736\\
27.5606864491466	10.3386361773603\\
27.5603227108088	10.3383720039501\\
27.559958957238	10.3381078225051\\
27.5595951884331	10.3378436330248\\
27.5592314043927	10.3375794355087\\
27.5588676051157	10.3373152299563\\
27.5585037906006	10.3370510163671\\
27.5581399608463	10.3367867947406\\
27.5577761158515	10.3365225650763\\
27.5574122556148	10.3362583273738\\
27.5570483801351	10.3359940816325\\
27.556684489411	10.335729827852\\
27.5563205834412	10.3354655660317\\
27.5559566622245	10.3352012961713\\
27.5555927257596	10.3349370182701\\
27.5552287740453	10.3346727323277\\
27.5548648070801	10.3344084383437\\
27.554500824863	10.3341441363175\\
27.5541368273925	10.3338798262486\\
27.5537728146674	10.3336155081365\\
27.5534087866865	10.3333511819809\\
27.5530447434484	10.333086847781\\
27.5526806849518	10.3328225055366\\
27.5523166111956	10.3325581552471\\
27.5519525221784	10.3322937969119\\
27.5515884178989	10.3320294305307\\
27.5512242983559	10.3317650561029\\
27.550860163548	10.331500673628\\
27.550496013474	10.3312362831055\\
27.5501318481327	10.330971884535\\
27.5497676675227	10.330707477916\\
27.5494034716427	10.3304430632479\\
27.5490392604915	10.3301786405303\\
27.5486750340678	10.3299142097627\\
27.5483107923703	10.3296497709446\\
27.5479465353977	10.3293853240755\\
27.5475822631488	10.329120869155\\
27.5472179756222	10.3288564061824\\
27.5468536728168	10.3285919351574\\
27.5464893547311	10.3283274560795\\
27.546125021364	10.3280629689481\\
27.5457606727141	10.3277984737627\\
27.5453963087801	10.327533970523\\
27.5450319295609	10.3272694592283\\
27.544667535055	10.3270049398782\\
27.5443031252612	10.3267404124723\\
27.5439387001783	10.3264758770099\\
27.5435742598049	10.3262113334907\\
27.5432098041397	10.3259467819141\\
27.5428453331816	10.3256822222796\\
27.5424808469291	10.3254176545868\\
27.5421163453811	10.3251530788352\\
27.5417518285361	10.3248884950242\\
27.5413872963931	10.3246239031534\\
27.5410227489505	10.3243593032223\\
27.5406581862073	10.3240946952303\\
27.540293608162	10.3238300791771\\
27.5399290148135	10.3235654550621\\
27.5395644061603	10.3233008228848\\
27.5391997822013	10.3230361826448\\
27.5388351429352	10.3227715343415\\
27.5384704883606	10.3225068779744\\
27.5381058184763	10.3222422135431\\
27.537741133281	10.3219775410471\\
27.5373764327734	10.3217128604858\\
27.5370117169522	10.3214481718588\\
27.5366469858161	10.3211834751656\\
27.5362822393639	10.3209187704058\\
27.5359174775943	10.3206540575787\\
27.535552700506	10.3203893366839\\
27.5351879080976	10.320124607721\\
27.5348231003679	10.3198598706893\\
27.5344582773157	10.3195951255886\\
27.5340934389396	10.3193303724181\\
27.5337285852383	10.3190656111776\\
27.5333637162106	10.3188008418664\\
27.5329988318551	10.318536064484\\
27.5326339321707	10.3182712790301\\
27.5322690171559	10.318006485504\\
27.5319040868095	10.3177416839053\\
27.5315391411302	10.3174768742336\\
27.5311741801168	10.3172120564882\\
27.5308092037679	10.3169472306688\\
27.5304442120822	10.3166823967748\\
27.5300792050585	10.3164175548057\\
27.5297141826955	10.3161527047611\\
27.5293491449919	10.3158878466404\\
27.5289840919463	10.3156229804432\\
27.5286190235576	10.3153581061689\\
27.5282539398243	10.3150932238171\\
27.5278888407453	10.3148283333873\\
27.5275237263192	10.314563434879\\
27.5271585965448	10.3142985282916\\
27.5267934514207	10.3140336136248\\
27.5264282909457	10.313768690878\\
27.5260631151184	10.3135037600507\\
27.5256979239376	10.3132388211424\\
27.525332717402	10.3129738741526\\
27.5249674955103	10.3127089190809\\
27.5246022582612	10.3124439559267\\
27.5242370056535	10.3121789846896\\
27.5238717376857	10.311914005369\\
27.5235064543567	10.3116490179645\\
27.5231411556651	10.3113840224756\\
27.5227758416096	10.3111190189017\\
27.522410512189	10.3108540072424\\
27.522045167402	10.3105889874972\\
27.5216798072472	10.3103239596657\\
27.5213144317234	10.3100589237472\\
27.5209490408293	10.3097938797413\\
27.5205836345636	10.3095288276475\\
27.520218212925	10.3092637674654\\
27.5198527759122	10.3089986991944\\
27.5194873235238	10.308733622834\\
27.5191218557587	10.3084685383837\\
27.5187563726155	10.3082034458431\\
27.518390874093	10.3079383452117\\
27.5180253601898	10.3076732364889\\
27.5176598309046	10.3074081196743\\
27.5172942862361	10.3071429947674\\
27.5169287261831	10.3068778617676\\
27.5165631507443	10.3066127206745\\
27.5161975599183	10.3063475714876\\
27.5158319537038	10.3060824142065\\
27.5154663320997	10.3058172488305\\
27.5151006951045	10.3055520753592\\
27.514735042717	10.3052868937921\\
27.5143693749358	10.3050217041287\\
27.5140036917598	10.3047565063686\\
27.5136379931875	10.3044913005112\\
27.5132722792177	10.304226086556\\
27.5129065498491	10.3039608645026\\
27.5125408050804	10.3036956343504\\
27.5121750449103	10.303430396099\\
27.5118092693375	10.3031651497478\\
27.5114434783607	10.3028998952964\\
27.5110776719787	10.3026346327443\\
27.51071185019	10.3023693620909\\
27.5103460129934	10.3021040833358\\
27.5099801603877	10.3018387964785\\
27.5096142923714	10.3015735015185\\
27.5092484089434	10.3013081984552\\
27.5088825101023	10.3010428872883\\
27.5085165958469	10.3007775680171\\
27.5081506661757	10.3005122406413\\
27.5077847210876	10.3002469051603\\
27.5074187605812	10.2999815615736\\
27.5070527846552	10.2997162098807\\
27.5066867933083	10.2994508500812\\
27.5063207865393	10.2991854821745\\
27.5059547643468	10.2989201061601\\
27.5055887267295	10.2986547220375\\
27.5052226736861	10.2983893298063\\
27.5048566052154	10.298123929466\\
27.504490521316	10.297858521016\\
27.5041244219866	10.2975931044559\\
27.5037583072259	10.2973276797851\\
27.5033921770327	10.2970622470032\\
27.5030260314056	10.2967968061096\\
27.5026598703433	10.296531357104\\
27.5022936938445	10.2962658999857\\
27.5019275019079	10.2960004347543\\
27.5015612945323	10.2957349614093\\
27.5011950717162	10.2954694799502\\
27.5008288334585	10.2952039903765\\
27.5004625797578	10.2949384926876\\
27.5000963106128	10.2946729868832\\
27.4997300260222	10.2944074729627\\
27.4993637259847	10.2941419509256\\
27.498997410499	10.2938764207714\\
27.4986310795638	10.2936108824996\\
27.4982647331777	10.2933453361098\\
27.4978983713396	10.2930797816014\\
27.4975319940481	10.2928142189739\\
27.4971656013018	10.2925486482268\\
27.4967991930995	10.2922830693597\\
27.4964327694399	10.292017482372\\
27.4960663303216	10.2917518872633\\
27.4956998757435	10.291486284033\\
27.495333405704	10.2912206726806\\
27.4949669202021	10.2909550532057\\
27.4946004192363	10.2906894256078\\
27.4942339028053	10.2904237898863\\
27.4938673709079	10.2901581460408\\
27.4935008235427	10.2898924940707\\
27.4931342607084	10.2896268339757\\
27.4927676824038	10.289361165755\\
27.4924010886274	10.2890954894084\\
27.4920344793781	10.2888298049353\\
27.4916678546545	10.2885641123351\\
27.4913012144553	10.2882984116074\\
27.4909345587792	10.2880327027517\\
27.4905678876249	10.2877669857674\\
27.490201200991	10.2875012606542\\
27.4898344988763	10.2872355274115\\
27.4894677812795	10.2869697860387\\
27.4891010481993	10.2867040365355\\
27.4887342996342	10.2864382789013\\
27.4883675355832	10.2861725131355\\
27.4880007560447	10.2859067392378\\
27.4876339610176	10.2856409572076\\
27.4872671505006	10.2853751670443\\
27.4869003244922	10.2851093687476\\
27.4865334829912	10.284843562317\\
27.4861666259963	10.2845777477518\\
27.4857997535062	10.2843119250516\\
27.4854328655196	10.284046094216\\
27.4850659620351	10.2837802552444\\
27.4846990430515	10.2835144081363\\
27.4843321085675	10.2832485528913\\
27.4839651585817	10.2829826895087\\
27.4835981930928	10.2827168179883\\
27.4832312120995	10.2824509383293\\
27.4828642156005	10.2821850505314\\
27.4824972035945	10.281919154594\\
27.4821301760803	10.2816532505166\\
27.4817631330563	10.2813873382988\\
27.4813960745215	10.28112141794\\
27.4810290004744	10.2808554894398\\
27.4806619109138	10.2805895527976\\
27.4802948058382	10.280323608013\\
27.4799276852465	10.2800576550854\\
27.4795605491373	10.2797916940143\\
27.4791933975093	10.2795257247993\\
27.4788262303612	10.2792597474399\\
27.4784590476917	10.2789937619355\\
27.4780918494994	10.2787277682857\\
27.4777246357831	10.2784617664899\\
27.4773574065414	10.2781957565476\\
27.476990161773	10.2779297384585\\
27.4766229014766	10.2776637122219\\
27.4762556256509	10.2773976778373\\
27.4758883342946	10.2771316353043\\
27.4755210274064	10.2768655846223\\
27.4751537049849	10.276599525791\\
27.4747863670289	10.2763334588096\\
27.474419013537	10.2760673836779\\
27.4740516445079	10.2758013003952\\
27.4736842599403	10.275535208961\\
27.4733168598329	10.275269109375\\
27.4729494441844	10.2750030016365\\
27.4725820129934	10.274736885745\\
27.4722145662586	10.2744707617001\\
27.4718471039788	10.2742046295013\\
27.4714796261526	10.2739384891481\\
27.4711121327787	10.2736723406399\\
27.4707446238557	10.2734061839763\\
27.4703770993824	10.2731400191567\\
27.4700095593575	10.2728738461808\\
27.4696420037796	10.2726076650478\\
27.4692744326474	10.2723414757575\\
27.4689068459596	10.2720752783092\\
27.4685392437149	10.2718090727025\\
27.4681716259119	10.2715428589368\\
27.4678039925494	10.2712766370118\\
27.467436343626	10.2710104069267\\
27.4670686791404	10.2707441686813\\
27.4667009990914	10.2704779222749\\
27.4663333034775	10.2702116677071\\
27.4659655922974	10.2699454049773\\
27.4655978655499	10.2696791340851\\
27.4652301232336	10.26941285503\\
27.4648623653473	10.2691465678114\\
27.4644945918895	10.2688802724289\\
27.4641268028589	10.268613968882\\
27.4637589982544	10.2683476571701\\
27.4633911780744	10.2680813372928\\
27.4630233423178	10.2678150092496\\
27.4626554909831	10.2675486730399\\
27.4622876240692	10.2672823286632\\
27.4619197415745	10.2670159761191\\
27.4615518434979	10.2667496154071\\
27.4611839298381	10.2664832465266\\
27.4608160005936	10.2662168694772\\
27.4604480557631	10.2659504842584\\
27.4600800953455	10.2656840908696\\
27.4597121193392	10.2654176893103\\
27.4593441277431	10.2651512795801\\
27.4589761205558	10.2648848616784\\
27.4586080977759	10.2646184356048\\
27.4582400594021	10.2643520013588\\
27.4578720054332	10.2640855589398\\
27.4575039358678	10.2638191083473\\
27.4571358507046	10.2635526495809\\
27.4567677499422	10.26328618264\\
27.4563996335794	10.2630197075241\\
27.4560315016147	10.2627532242328\\
27.455663354047	10.2624867327655\\
27.4552951908748	10.2622202331218\\
27.4549270120969	10.2619537253011\\
27.4545588177119	10.2616872093029\\
27.4541906077185	10.2614206851267\\
27.4538223821154	10.2611541527721\\
27.4534541409012	10.2608876122385\\
27.4530858840747	10.2606210635254\\
27.4527176116344	10.2603545066324\\
27.4523493235791	10.2600879415588\\
27.4519810199075	10.2598213683043\\
27.4516127006183	10.2595547868683\\
27.45124436571	10.2592881972503\\
27.4508760151814	10.2590215994498\\
27.4505076490312	10.2587549934663\\
27.450139267258	10.2584883792994\\
27.4497708698605	10.2582217569484\\
27.4494024568374	10.2579551264129\\
27.4490340281873	10.2576884876924\\
27.448665583909	10.2574218407865\\
27.4482971240011	10.2571551856945\\
27.4479286484622	10.256888522416\\
27.4475601572911	10.2566218509505\\
27.4471916504865	10.2563551712975\\
27.4468231280469	10.2560884834564\\
27.4464545899711	10.2558217874269\\
27.4460860362578	10.2555550832083\\
27.4457174669055	10.2552883708002\\
27.4453488819131	10.255021650202\\
27.4449802812791	10.2547549214134\\
27.4446116650023	10.2544881844336\\
27.4442430330812	10.2542214392624\\
27.4438743855147	10.2539546858991\\
27.4435057223013	10.2536879243433\\
27.4431370434397	10.2534211545944\\
27.4427683489286	10.253154376652\\
27.4423996387667	10.2528875905155\\
27.4420309129526	10.2526207961844\\
27.441662171485	10.2523539936583\\
27.4412934143627	10.2520871829366\\
27.4409246415841	10.2518203640189\\
27.4405558531481	10.2515535369046\\
27.4401870490533	10.2512867015932\\
27.4398182292983	10.2510198580842\\
27.4394493938819	10.2507530063772\\
27.4390805428026	10.2504861464715\\
27.4387116760593	10.2502192783668\\
27.4383427936504	10.2499524020625\\
27.4379738955748	10.249685517558\\
27.4376049818311	10.249418624853\\
27.4372360524179	10.2491517239469\\
27.4368671073339	10.2488848148391\\
27.4364981465778	10.2486178975293\\
27.4361291701483	10.2483509720168\\
27.435760178044	10.2480840383012\\
27.4353911702636	10.247817096382\\
27.4350221468057	10.2475501462586\\
27.4346531076691	10.2472831879306\\
27.4342840528524	10.2470162213975\\
27.4339149823542	10.2467492466587\\
27.4335458961732	10.2464822637138\\
27.4331767943082	10.2462152725622\\
27.4328076767577	10.2459482732035\\
27.4324385435204	10.2456812656371\\
27.4320693945951	10.2454142498625\\
27.4317002299803	10.2451472258793\\
27.4313310496747	10.2448801936869\\
27.430961853677	10.2446131532848\\
27.4305926419859	10.2443461046725\\
27.4302234146	10.2440790478495\\
27.429854171518	10.2438119828153\\
27.4294849127386	10.2435449095695\\
27.4291156382603	10.2432778281114\\
27.428746348082	10.2430107384406\\
27.4283770422022	10.2427436405566\\
27.4280077206197	10.2424765344589\\
27.427638383333	10.2422094201469\\
27.4272690303409	10.2419422976203\\
27.426899661642	10.2416751668783\\
27.4265302772349	10.2414080279207\\
27.4261608771184	10.2411408807468\\
27.4257914612911	10.2408737253561\\
27.4254220297517	10.2406065617482\\
27.4250525824988	10.2403393899225\\
27.4246831195311	10.2400722098786\\
27.4243136408472	10.2398050216159\\
27.4239441464459	10.2395378251339\\
27.4235746363257	10.2392706204321\\
27.4232051104854	10.23900340751\\
27.4228355689236	10.2387361863671\\
27.422466011639	10.2384689570029\\
27.4220964386302	10.2382017194169\\
27.4217268498959	10.2379344736086\\
27.4213572454347	10.2376672195775\\
27.4209876252454	10.237399957323\\
27.4206179893265	10.2371326868447\\
27.4202483376768	10.2368654081421\\
27.4198786702949	10.2365981212146\\
27.4195089871794	10.2363308260618\\
27.4191392883291	10.2360635226831\\
27.4187695737425	10.2357962110781\\
27.4183998434184	10.2355288912461\\
27.4180300973554	10.2352615631868\\
27.4176603355521	10.2349942268997\\
27.4172905580073	10.2347268823841\\
27.4169207647195	10.2344595296396\\
27.4165509556875	10.2341921686657\\
27.4161811309099	10.2339247994619\\
27.4158112903854	10.2336574220277\\
27.4154414341125	10.2333900363626\\
27.4150715620901	10.2331226424661\\
27.4147016743166	10.2328552403376\\
27.4143317707909	10.2325878299766\\
27.4139618515115	10.2323204113828\\
27.4135919164772	10.2320529845554\\
27.4132219656865	10.2317855494941\\
27.4128519991381	10.2315181061983\\
27.4124820168307	10.2312506546676\\
27.412112018763	10.2309831949013\\
27.4117420049335	10.2307157268991\\
27.411371975341	10.2304482506603\\
27.4110019299841	10.2301807661846\\
27.4106318688615	10.2299132734713\\
27.4102617919719	10.22964577252\\
27.4098916993137	10.2293782633302\\
27.4095215908859	10.2291107459014\\
27.4091514666869	10.228843220233\\
27.4087813267155	10.2285756863245\\
27.4084111709703	10.2283081441755\\
27.4080409994499	10.2280405937854\\
27.4076708121531	10.2277730351538\\
27.4073006090784	10.22750546828\\
27.4069303902246	10.2272378931637\\
27.4065601555902	10.2269703098043\\
27.406189905174	10.2267027182012\\
27.4058196389746	10.2264351183541\\
27.4054493569906	10.2261675102623\\
27.4050790592207	10.2258998939254\\
27.4047087456635	10.2256322693428\\
27.4043384163178	10.2253646365142\\
27.4039680711821	10.2250969954388\\
27.4035977102552	10.2248293461163\\
27.4032273335356	10.2245616885461\\
27.402856941022	10.2242940227278\\
27.4024865327131	10.2240263486607\\
27.4021161086076	10.2237586663445\\
27.401745668704	10.2234909757785\\
27.401375213001	10.2232232769623\\
27.4010047414974	10.2229555698955\\
27.4006342541916	10.2226878545773\\
27.4002637510825	10.2224201310075\\
27.3998932321686	10.2221523991854\\
27.3995226974485	10.2218846591105\\
27.3991521469211	10.2216169107824\\
27.3987815805848	10.2213491542005\\
27.3984109984383	10.2210813893643\\
27.3980404004804	10.2208136162733\\
27.3976697867096	10.220545834927\\
27.3972991571246	10.2202780453249\\
27.396928511724	10.2200102474665\\
27.3965578505066	10.2197424413512\\
27.3961871734708	10.2194746269787\\
27.3958164806155	10.2192068043482\\
27.3954457719393	10.2189389734595\\
27.3950750474407	10.2186711343118\\
27.3947043071184	10.2184032869048\\
27.3943335509712	10.2181354312379\\
27.3939627789976	10.2178675673106\\
27.3935919911963	10.2175996951224\\
27.393221187566	10.2173318146727\\
27.3928503681052	10.2170639259612\\
27.3924795328127	10.2167960289872\\
27.392108681687	10.2165281237503\\
27.3917378147269	10.2162602102499\\
27.391366931931	10.2159922884856\\
27.3909960332978	10.2157243584568\\
27.3906251188262	10.215456420163\\
27.3902541885147	10.2151884736037\\
27.389883242362	10.2149205187784\\
27.3895122803667	10.2146525556867\\
27.3891413025274	10.2143845843279\\
27.3887703088429	10.2141166047015\\
27.3883992993117	10.2138486168071\\
27.3880282739325	10.2135806206442\\
27.387657232704	10.2133126162122\\
27.3872861756247	10.2130446035106\\
27.3869151026934	10.212776582539\\
27.3865440139087	10.2125085532967\\
27.3861729092692	10.2122405157834\\
27.3858017887736	10.2119724699984\\
27.3854306524205	10.2117044159413\\
27.3850595002086	10.2114363536116\\
27.3846883321365	10.2111682830088\\
27.3843171482029	10.2109002041322\\
27.3839459484063	10.2106321169816\\
27.3835747327455	10.2103640215562\\
27.3832035012191	10.2100959178556\\
27.3828322538257	10.2098278058794\\
27.382460990564	10.2095596856269\\
27.3820897114326	10.2092915570977\\
27.3817184164302	10.2090234202913\\
27.3813471055553	10.2087552752071\\
27.3809757788067	10.2084871218447\\
27.3806044361831	10.2082189602035\\
27.3802330776829	10.2079507902831\\
27.3798617033049	10.2076826120828\\
27.3794903130478	10.2074144256023\\
27.3791189069101	10.2071462308409\\
27.3787474848905	10.2068780277982\\
27.3783760469876	10.2066098164737\\
27.3780045932001	10.2063415968669\\
27.3776331235267	10.2060733689771\\
27.3772616379659	10.2058051328041\\
27.3768901365164	10.2055368883471\\
27.3765186191769	10.2052686356057\\
27.376147085946	10.2050003745795\\
27.3757755368223	10.2047321052678\\
27.3754039718044	10.2044638276702\\
27.3750323908911	10.2041955417862\\
27.3746607940809	10.2039272476152\\
27.3742891813725	10.2036589451567\\
27.3739175527646	10.2033906344103\\
27.3735459082557	10.2031223153754\\
27.3731742478445	10.2028539880515\\
27.3728025715297	10.202585652438\\
27.3724308793099	10.2023173085346\\
27.3720591711836	10.2020489563406\\
27.3716874471497	10.2017805958556\\
27.3713157072066	10.201512227079\\
27.3709439513531	10.2012438500103\\
27.3705721795878	10.2009754646491\\
27.3702003919093	10.2007070709947\\
27.3698285883162	10.2004386690468\\
27.3694567688072	10.2001702588047\\
27.369084933381	10.199901840268\\
27.3687130820361	10.1996334134361\\
27.3683412147712	10.1993649783086\\
27.367969331585	10.1990965348849\\
27.367597432476	10.1988280831645\\
27.367225517443	10.1985596231469\\
27.3668535864845	10.1982911548317\\
27.3664816395992	10.1980226782181\\
27.3661096767857	10.1977541933059\\
27.3657376980426	10.1974857000944\\
27.3653657033687	10.1972171985831\\
27.3649936927624	10.1969486887716\\
27.3646216662226	10.1966801706592\\
27.3642496237477	10.1964116442456\\
27.3638775653365	10.1961431095301\\
27.3635054909875	10.1958745665124\\
27.3631334006995	10.1956060151917\\
27.3627612944709	10.1953374555677\\
27.3623891723006	10.1950688876399\\
27.362017034187	10.1948003114076\\
27.3616448801289	10.1945317268705\\
27.3612727101249	10.194263134028\\
27.3609005241735	10.1939945328795\\
27.3605283222735	10.1937259234246\\
27.3601561044235	10.1934573056627\\
27.3597838706221	10.1931886795934\\
27.3594116208679	10.1929200452161\\
27.3590393551596	10.1926514025303\\
27.3586670734957	10.1923827515355\\
27.3582947758751	10.1921140922311\\
27.3579224622961	10.1918454246167\\
27.3575501327576	10.1915767486918\\
27.3571777872581	10.1913080644558\\
27.3568054257963	10.1910393719083\\
27.3564330483708	10.1907706710486\\
27.3560606549801	10.1905019618764\\
27.3556882456231	10.190233244391\\
27.3553158202982	10.189964518592\\
27.3549433790041	10.1896957844788\\
27.3545709217395	10.189427042051\\
27.354198448503	10.189158291308\\
27.3538259592932	10.1888895322493\\
27.3534534541087	10.1886207648743\\
27.3530809329481	10.1883519891827\\
27.3527083958102	10.1880832051738\\
27.3523358426935	10.1878144128472\\
27.3519632735967	10.1875456122023\\
27.3515906885183	10.1872768032386\\
27.351218087457	10.1870079859557\\
27.3508454704115	10.1867391603529\\
27.3504728373804	10.1864703264298\\
27.3501001883622	10.1862014841858\\
27.3497275233557	10.1859326336205\\
27.3493548423595	10.1856637747333\\
27.3489821453721	10.1853949075238\\
27.3486094323922	10.1851260319913\\
27.3482367034185	10.1848571481354\\
27.3478639584495	10.1845882559556\\
27.3474911974839	10.1843193554513\\
27.3471184205204	10.1840504466221\\
27.3467456275575	10.1837815294673\\
27.3463728185938	10.1835126039866\\
27.3459999936281	10.1832436701794\\
27.3456271526589	10.1829747280452\\
27.3452542956848	10.1827057775834\\
27.3448814227045	10.1824368187935\\
27.3445085337166	10.1821678516751\\
27.3441356287198	10.1818988762276\\
27.3437627077125	10.1816298924505\\
27.3433897706936	10.1813609003432\\
27.3430168176616	10.1810918999054\\
27.3426438486151	10.1808228911363\\
27.3422708635527	10.1805538740356\\
27.3418978624732	10.1802848486027\\
27.341524845375	10.1800158148372\\
27.3411518122569	10.1797467727384\\
27.3407787631174	10.1794777223058\\
27.3404056979551	10.179208663539\\
27.3400326167688	10.1789395964375\\
27.339659519557	10.1786705210007\\
27.3392864063184	10.178401437228\\
27.3389132770515	10.1781323451191\\
27.338540131755	10.1778632446733\\
27.3381669704275	10.1775941358902\\
27.3377937930676	10.1773250187692\\
27.337420599674	10.1770558933098\\
27.3370473902453	10.1767867595115\\
27.3366741647801	10.1765176173738\\
27.336300923277	10.1762484668962\\
27.3359276657346	10.1759793080781\\
27.3355543921516	10.1757101409191\\
27.3351811025266	10.1754409654185\\
27.3348077968582	10.175171781576\\
27.334434475145	10.174902589391\\
27.3340611373857	10.1746333888629\\
27.3336877835789	10.1743641799912\\
27.3333144137231	10.1740949627755\\
27.3329410278171	10.1738257372153\\
27.3325676258594	10.1735565033099\\
27.3321942078486	10.1732872610589\\
27.3318207737834	10.1730180104617\\
27.3314473236624	10.1727487515179\\
27.3310738574842	10.1724794842269\\
27.3307003752474	10.1722102085882\\
27.3303268769507	10.1719409246014\\
27.3299533625926	10.1716716322658\\
27.3295798321719	10.1714023315809\\
27.329206285687	10.1711330225463\\
27.3288327231367	10.1708637051614\\
27.3284591445195	10.1705943794257\\
27.328085549834	10.1703250453387\\
27.3277119390789	10.1700557028999\\
27.3273383122528	10.1697863521087\\
27.3269646693544	10.1695169929647\\
27.3265910103821	10.1692476254672\\
27.3262173353347	10.1689782496159\\
27.3258436442108	10.1687088654101\\
27.3254699370089	10.1684394728494\\
27.3250962137277	10.1681700719332\\
27.3247224743659	10.1679006626611\\
27.3243487189219	10.1676312450324\\
27.3239749473945	10.1673618190468\\
27.3236011597823	10.1670923847036\\
27.3232273560838	10.1668229420023\\
27.3228535362977	10.1665534909425\\
27.3224797004227	10.1662840315236\\
27.3221058484572	10.1660145637451\\
27.3217319804	10.1657450876064\\
27.3213580962496	10.1654756031071\\
27.3209841960047	10.1652061102467\\
27.3206102796639	10.1649366090246\\
27.3202363472257	10.1646670994402\\
27.3198623986889	10.1643975814932\\
27.319488434052	10.1641280551829\\
27.3191144533136	10.1638585205088\\
27.3187404564723	10.1635889774705\\
27.3183664435269	10.1633194260673\\
27.3179924144758	10.1630498662989\\
27.3176183693176	10.1627802981646\\
27.3172443080511	10.162510721664\\
27.3168702306748	10.1622411367965\\
27.3164961371873	10.1619715435616\\
27.3161220275872	10.1617019419588\\
27.3157479018732	10.1614323319875\\
27.3153737600439	10.1611627136473\\
27.3149996020978	10.1608930869376\\
27.3146254280336	10.160623451858\\
27.3142512378499	10.1603538084078\\
27.3138770315452	10.1600841565866\\
27.3135028091184	10.1598144963938\\
27.3131285705678	10.159544827829\\
27.3127543158921	10.1592751508916\\
27.3123800450901	10.1590054655811\\
27.3120057581601	10.1587357718969\\
27.311631455101	10.1584660698386\\
27.3112571359112	10.1581963594056\\
27.3108828005894	10.1579266405975\\
27.3105084491342	10.1576569134136\\
27.3101340815442	10.1573871778535\\
27.309759697818	10.1571174339166\\
27.3093852979542	10.1568476816025\\
27.3090108819515	10.1565779209106\\
27.3086364498084	10.1563081518403\\
27.3082620015236	10.1560383743913\\
27.3078875370956	10.1557685885629\\
27.307513056523	10.1554987943546\\
27.3071385598046	10.1552289917659\\
27.3067640469388	10.1549591807963\\
27.3063895179243	10.1546893614453\\
27.3060149727597	10.1544195337123\\
27.3056404114437	10.1541496975968\\
27.3052658339747	10.1538798530983\\
27.3048912403514	10.1536100002164\\
27.3045166305725	10.1533401389503\\
27.3041420046365	10.1530702692998\\
27.303767362542	10.1528003912641\\
27.3033927042876	10.1525305048428\\
27.3030180298721	10.1522606100354\\
27.3026433392938	10.1519907068414\\
27.3022686325515	10.1517207952602\\
27.3018939096438	10.1514508752913\\
27.3015191705693	10.1511809469342\\
27.3011444153265	10.1509110101884\\
27.3007696439141	10.1506410650533\\
27.3003948563307	10.1503711115285\\
27.3000200525749	10.1501011496133\\
27.2996452326452	10.1498311793074\\
27.2992703965404	10.1495612006101\\
27.298895544259	10.1492912135209\\
27.2985206757995	10.1490212180394\\
27.2981457911607	10.148751214165\\
27.2977708903411	10.1484812018971\\
27.2973959733392	10.1482111812353\\
27.2970210401539	10.147941152179\\
27.2966460907835	10.1476711147277\\
27.2962711252267	10.147401068881\\
27.2958961434822	10.1471310146381\\
27.2955211455485	10.1468609519988\\
27.2951461314242	10.1465908809623\\
27.294771101108	10.1463208015283\\
27.2943960545984	10.1460507136961\\
27.294020991894	10.1457806174653\\
27.2936459129934	10.1455105128353\\
27.2932708178953	10.1452403998056\\
27.2928957065982	10.1449702783757\\
27.2925205791008	10.1447001485451\\
27.2921454354016	10.1444300103132\\
27.2917702754992	10.1441598636796\\
27.2913950993923	10.1438897086436\\
27.2910199070794	10.1436195452048\\
27.2906446985592	10.1433493733627\\
27.2902694738302	10.1430791931167\\
27.289894232891	10.1428090044663\\
27.2895189757403	10.142538807411\\
27.2891437023766	10.1422686019502\\
27.2887684127985	10.1419983880835\\
27.2883931070047	10.1417281658103\\
27.2880177849937	10.1414579351301\\
27.2876424467641	10.1411876960423\\
27.2872670923146	10.1409174485465\\
27.2868917216437	10.1406471926421\\
27.28651633475	10.1403769283287\\
27.2861409316322	10.1401066556055\\
27.2857655122887	10.1398363744723\\
27.2853900767183	10.1395660849284\\
27.2850146249195	10.1392957869733\\
27.2846391568909	10.1390254806064\\
27.2842636726311	10.1387551658273\\
27.2838881721388	10.1384848426355\\
27.2835126554124	10.1382145110304\\
27.2831371224506	10.1379441710114\\
27.282761573252	10.1376738225781\\
27.2823860078152	10.13740346573\\
27.2820104261388	10.1371331004665\\
27.2816348282214	10.136862726787\\
27.2812592140616	10.1365923446911\\
27.2808835836579	10.1363219541783\\
27.280507937009	10.136051555248\\
27.2801322741134	10.1357811478997\\
27.2797565949698	10.1355107321329\\
27.2793808995768	10.135240307947\\
27.2790051879329	10.1349698753415\\
27.2786294600367	10.1346994343159\\
27.2782537158869	10.1344289848697\\
27.277877955482	10.1341585270024\\
27.2775021788206	10.1338880607134\\
27.2771263859014	10.1336175860021\\
27.2767505767228	10.1333471028682\\
27.2763747512836	10.133076611311\\
27.2759989095822	10.13280611133\\
27.2756230516173	10.1325356029248\\
27.2752471773875	10.1322650860947\\
27.2748712868914	10.1319945608393\\
27.2744953801275	10.131724027158\\
27.2741194570945	10.1314534850503\\
27.2737435177909	10.1311829345156\\
27.2733675622154	10.1309123755536\\
27.2729915903665	10.1306418081635\\
27.2726156022428	10.130371232345\\
27.2722395978429	10.1301006480974\\
27.2718635771654	10.1298300554203\\
27.2714875402089	10.1295594543132\\
27.271111486972	10.1292888447754\\
27.2707354174533	10.1290182268065\\
27.2703593316513	10.1287476004059\\
27.2699832295647	10.1284769655732\\
27.269607111192	10.1282063223078\\
27.2692309765319	10.1279356706091\\
27.2688548255828	10.1276650104768\\
27.2684786583435	10.1273943419101\\
27.2681024748125	10.1271236649086\\
27.2677262749884	10.1268529794718\\
27.2673500588697	10.1265822855992\\
27.2669738264551	10.1263115832901\\
27.2665975777432	10.1260408725442\\
27.2662213127325	10.1257701533608\\
27.2658450314216	10.1254994257395\\
27.2654687338092	10.1252286896797\\
27.2650924198937	10.1249579451809\\
27.2647160896739	10.1246871922425\\
27.2643397431482	10.1244164308641\\
27.2639633803153	10.1241456610451\\
27.2635870011737	10.123874882785\\
27.2632106057221	10.1236040960832\\
27.262834193959	10.1233333009393\\
27.262457765883	10.1230624973527\\
27.2620813214927	10.1227916853229\\
27.2617048607867	10.1225208648493\\
27.2613283837636	10.1222500359315\\
27.2609518904219	10.1219791985688\\
27.2605753807602	10.1217083527608\\
27.2601988547772	10.121437498507\\
27.2598223124714	10.1211666358068\\
27.2594457538414	10.1208957646597\\
27.2590691788857	10.1206248850652\\
27.258692587603	10.1203539970227\\
27.2583159799919	10.1200831005317\\
27.2579393560509	10.1198121955917\\
27.2575627157786	10.1195412822021\\
27.2571860591736	10.1192703603625\\
27.2568093862345	10.1189994300723\\
27.2564326969598	10.118728491331\\
27.2560559913482	10.118457544138\\
27.2556792693983	10.1181865884929\\
27.2553025311085	10.117915624395\\
27.2549257764776	10.1176446518439\\
27.254549005504	10.1173736708391\\
27.2541722181864	10.11710268138\\
27.2537954145233	10.116831683466\\
27.2534185945133	10.1165606770967\\
27.2530417581551	10.1162896622716\\
27.2526649054471	10.11601863899\\
27.252288036388	10.1157476072515\\
27.2519111509764	10.1154765670556\\
27.2515342492108	10.1152055184017\\
27.2511573310898	10.1149344612893\\
27.250780396612	10.1146633957178\\
27.2504034457759	10.1143923216868\\
27.2500264785802	10.1141212391957\\
27.2496494950235	10.113850148244\\
27.2492724951042	10.1135790488311\\
27.2488954788211	10.1133079409566\\
27.2485184461726	10.1130368246198\\
27.2481413971574	10.1127656998203\\
27.247764331774	10.1124945665576\\
27.247387250021	10.112223424831\\
27.247010151897	10.1119522746402\\
27.2466330374005	10.1116811159845\\
27.2462559065302	10.1114099488634\\
27.2458787592846	10.1111387732764\\
27.2455015956623	10.110867589223\\
27.2451244156619	10.1105963967026\\
27.244747219282	10.1103251957148\\
27.244370006521	10.1100539862589\\
27.2439927773777	10.1097827683345\\
27.2436155318506	10.109511541941\\
27.2432382699382	10.1092403070779\\
27.2428609916392	10.1089690637446\\
27.2424836969521	10.1086978119407\\
27.2421063858755	10.1084265516656\\
27.2417290584079	10.1081552829188\\
27.241351714548	10.1078840056997\\
27.2409743542943	10.1076127200079\\
27.2405969776454	10.1073414258427\\
27.2402195845999	10.1070701232037\\
27.2398421751563	10.1067988120904\\
27.2394647493132	10.1065274925021\\
27.2390873070692	10.1062561644384\\
27.2387098484229	10.1059848278988\\
27.2383323733729	10.1057134828826\\
27.2379548819176	10.1054421293895\\
27.2375773740558	10.1051707674188\\
27.2371998497859	10.10489939697\\
27.2368223091065	10.1046280180427\\
27.2364447520163	10.1043566306362\\
27.2360671785137	10.10408523475\\
27.2356895885975	10.1038138303837\\
27.235311982266	10.1035424175366\\
27.2349343595179	10.1032709962082\\
27.2345567203519	10.1029995663981\\
27.2341790647663	10.1027281281057\\
27.2338013927599	10.1024566813304\\
27.2334237043312	10.1021852260718\\
27.2330459994787	10.1019137623292\\
27.2326682782011	10.1016422901022\\
27.2322905404969	10.1013708093902\\
27.2319127863647	10.1010993201928\\
27.231535015803	10.1008278225093\\
27.2311572288104	10.1005563163393\\
27.2307794253856	10.1002848016821\\
27.230401605527	10.1000132785374\\
27.2300237692332	10.0997417469045\\
27.2296459165028	10.0994702067829\\
27.2292680473344	10.0991986581722\\
27.2288901617266	10.0989271010716\\
27.2285122596778	10.0986555354808\\
27.2281343411868	10.0983839613992\\
27.2277564062519	10.0981123788263\\
27.2273784548719	10.0978407877615\\
27.2270004870453	10.0975691882043\\
27.2266225027707	10.0972975801542\\
27.2262445020465	10.0970259636106\\
27.2258664848715	10.0967543385731\\
27.2254884512441	10.096482705041\\
27.225110401163	10.0962110630138\\
27.2247323346266	10.0959394124911\\
27.2243542516336	10.0956677534723\\
27.2239761521826	10.0953960859568\\
27.223598036272	10.0951244099441\\
27.2232199039005	10.0948527254338\\
27.2228417550667	10.0945810324251\\
27.222463589769	10.0943093309178\\
27.2220854080061	10.0940376209111\\
27.2217072097765	10.0937659024046\\
27.2213289950788	10.0934941753977\\
27.2209507639116	10.0932224398899\\
27.2205725162734	10.0929506958806\\
27.2201942521629	10.0926789433695\\
27.2198159715784	10.0924071823558\\
27.2194376745187	10.0921354128391\\
27.2190593609823	10.0918636348188\\
27.2186810309677	10.0915918482944\\
27.2183026844736	10.0913200532655\\
27.2179243214984	10.0910482497313\\
27.2175459420408	10.0907764376915\\
27.2171675460993	10.0905046171454\\
27.2167891336724	10.0902327880926\\
27.2164107047588	10.0899609505325\\
27.216032259357	10.0896891044646\\
27.2156537974656	10.0894172498884\\
27.2152753190831	10.0891453868032\\
27.2148968242081	10.0888735152087\\
27.2145183128391	10.0886016351042\\
27.2141397849747	10.0883297464892\\
27.2137612406136	10.0880578493632\\
27.2133826797541	10.0877859437256\\
27.213004102395	10.087514029576\\
27.2126255085347	10.0872421069137\\
27.2122468981718	10.0869701757384\\
27.211868271305	10.0866982360493\\
27.2114896279327	10.086426287846\\
27.2111109680534	10.086154331128\\
27.2107322916659	10.0858823658947\\
27.2103535987686	10.0856103921455\\
27.20997488936	10.0853384098801\\
27.2095961634388	10.0850664190977\\
27.2092174210036	10.084794419798\\
27.2088386620528	10.0845224119803\\
27.208459886585	10.0842503956441\\
27.2080810945988	10.0839783707889\\
27.2077022860927	10.0837063374142\\
27.2073234610654	10.0834342955193\\
27.2069446195153	10.0831622451039\\
27.206565761441	10.0828901861673\\
27.2061868868412	10.0826181187091\\
27.2058079957142	10.0823460427286\\
27.2054290880588	10.0820739582254\\
27.2050501638734	10.0818018651988\\
27.2046712231566	10.0815297636485\\
27.204292265907	10.0812576535738\\
27.2039132921232	10.0809855349743\\
27.2035343018036	10.0807134078493\\
27.2031552949468	10.0804412721983\\
27.2027762715515	10.0801691280209\\
27.2023972316161	10.0798969753165\\
27.2020181751392	10.0796248140845\\
27.2016391021194	10.0793526443244\\
27.2012600125552	10.0790804660357\\
27.2008809064452	10.0788082792178\\
27.2005017837879	10.0785360838702\\
27.200122644582	10.0782638799924\\
27.1997434888258	10.0779916675838\\
27.1993643165181	10.077719446644\\
27.1989851276573	10.0774472171723\\
27.198605922242	10.0771749791682\\
27.1982267002708	10.0769027326312\\
27.1978474617423	10.0766304775608\\
27.1974682066548	10.0763582139564\\
27.1970889350072	10.0760859418176\\
27.1967096467978	10.0758136611436\\
27.1963303420252	10.0755413719341\\
27.195951020688	10.0752690741885\\
27.1955716827848	10.0749967679062\\
27.195192328314	10.0747244530868\\
27.1948129572743	10.0744521297296\\
27.1944335696642	10.0741797978342\\
27.1940541654823	10.0739074573999\\
27.193674744727	10.0736351084264\\
27.193295307397	10.073362750913\\
27.1929158534909	10.0730903848592\\
27.192536383007	10.0728180102644\\
27.1921568959441	10.0725456271282\\
27.1917773923007	10.07227323545\\
27.1913978720752	10.0720008352293\\
27.1910183352664	10.0717284264655\\
27.1906387818726	10.071456009158\\
27.1902592118925	10.0711835833065\\
27.1898796253246	10.0709111489102\\
27.1895000221675	10.0706387059687\\
27.1891204024197	10.0703662544815\\
27.1887407660798	10.070093794448\\
27.1883611131463	10.0698213258676\\
27.1879814436177	10.0695488487399\\
27.1876017574927	10.0692763630643\\
27.1872220547697	10.0690038688402\\
27.1868423354474	10.0687313660672\\
27.1864625995241	10.0684588547447\\
27.1860828469986	10.0681863348721\\
27.1857030778694	10.0679138064489\\
27.185323292135	10.0676412694747\\
27.1849434897939	10.0673687239487\\
27.1845636708447	10.0670961698706\\
27.184183835286	10.0668236072397\\
27.1838039831162	10.0665510360556\\
27.183424114334	10.0662784563177\\
27.1830442289379	10.0660058680254\\
27.1826643269264	10.0657332711783\\
27.1822844082981	10.0654606657757\\
27.1819044730516	10.0651880518172\\
27.1815245211853	10.0649154293023\\
27.1811445526978	10.0646427982302\\
27.1807645675877	10.0643701586007\\
27.1803845658535	10.064097510413\\
27.1800045474937	10.0638248536667\\
27.179624512507	10.0635521883612\\
27.1792444608918	10.063279514496\\
27.1788643926467	10.0630068320706\\
27.1784843077702	10.0627341410843\\
27.1781042062609	10.0624614415368\\
27.1777240881174	10.0621887334274\\
27.1773439533381	10.0619160167556\\
27.1769638019216	10.0616432915208\\
27.1765836338665	10.0613705577226\\
27.1762034491713	10.0610978153604\\
27.1758232478345	10.0608250644336\\
27.1754430298547	10.0605523049417\\
27.1750627952304	10.0602795368842\\
27.1746825439602	10.0600067602606\\
27.1743022760426	10.0597339750702\\
27.1739219914762	10.0594611813126\\
27.1735416902595	10.0591883789873\\
27.173161372391	10.0589155680936\\
27.1727810378693	10.058642748631\\
27.1724006866929	10.0583699205991\\
27.1720203188604	10.0580970839972\\
27.1716399343703	10.0578242388249\\
27.1712595332211	10.0575513850815\\
27.1708791154114	10.0572785227666\\
27.1704986809398	10.0570056518797\\
27.1701182298047	10.0567327724201\\
27.1697377620047	10.0564598843873\\
27.1693572775384	10.0561869877808\\
27.1689767764043	10.0559140826001\\
27.1685962586009	10.0556411688447\\
27.1682157241267	10.0553682465139\\
27.1678351729804	10.0550953156072\\
27.1674546051604	10.0548223761242\\
27.1670740206653	10.0545494280642\\
27.1666934194936	10.0542764714267\\
27.1663128016439	10.0540035062113\\
27.1659321671147	10.0537305324172\\
27.1655515159045	10.0534575500441\\
27.1651708480119	10.0531845590914\\
27.1647901634354	10.0529115595585\\
27.1644094621736	10.0526385514449\\
27.1640287442249	10.05236553475\\
27.163648009588	10.0520925094733\\
27.1632672582614	10.0518194756144\\
27.1628864902435	10.0515464331725\\
27.162505705533	10.0512733821473\\
27.1621249041283	10.0510003225381\\
27.1617440860281	10.0507272543444\\
27.1613632512308	10.0504541775657\\
27.1609823997349	10.0501810922014\\
27.1606015315391	10.049907998251\\
27.1602206466419	10.049634895714\\
27.1598397450417	10.0493617845898\\
27.1594588267372	10.0490886648779\\
27.1590778917268	10.0488155365777\\
27.1586969400091	10.0485423996887\\
27.1583159715826	10.0482692542103\\
27.1579349864459	10.0479961001421\\
27.1575539845975	10.0477229374834\\
27.1571729660359	10.0474497662338\\
27.1567919307596	10.0471765863926\\
27.1564108787673	10.0469033979595\\
27.1560298100574	10.0466302009337\\
27.1556487246284	10.0463569953148\\
27.1552676224789	10.0460837811022\\
27.1548865036075	10.0458105582955\\
27.1545053680126	10.0455373268939\\
27.1541242156928	10.0452640868971\\
27.1537430466467	10.0449908383045\\
27.1533618608727	10.0447175811154\\
27.1529806583694	10.0444443153295\\
27.1525994391352	10.0441710409461\\
27.1522182031689	10.0438977579647\\
27.1518369504688	10.0436244663848\\
27.1514556810336	10.0433511662058\\
27.1510743948616	10.0430778574272\\
27.1506930919516	10.0428045400484\\
27.1503117723019	10.042531214069\\
27.1499304359112	10.0422578794882\\
27.149549082778	10.0419845363057\\
27.1491677129007	10.0417111845209\\
27.148786326278	10.0414378241332\\
27.1484049229083	10.0411644551421\\
27.1480235027902	10.0408910775471\\
27.1476420659222	10.0406176913475\\
27.1472606123029	10.0403442965429\\
27.1468791419307	10.0400708931328\\
27.1464976548042	10.0397974811165\\
27.1461161509219	10.0395240604936\\
27.1457346302824	10.0392506312635\\
27.1453530928842	10.0389771934256\\
27.1449715387257	10.0387037469795\\
27.1445899678057	10.0384302919245\\
27.1442083801224	10.0381568282601\\
27.1438267756746	10.0378833559859\\
27.1434451544607	10.0376098751012\\
27.1430635164792	10.0373363856055\\
27.1426818617287	10.0370628874982\\
27.1423001902077	10.0367893807789\\
27.1419185019147	10.036515865447\\
27.1415367968482	10.0362423415019\\
27.1411550750069	10.0359688089431\\
27.1407733363891	10.03569526777\\
27.1403915809934	10.0354217179821\\
27.1400098088184	10.035148159579\\
27.1396280198626	10.0348745925599\\
27.1392462141245	10.0346010169244\\
27.1388643916026	10.034327432672\\
27.1384825522954	10.034053839802\\
27.1381006962016	10.033780238314\\
27.1377188233195	10.0335066282074\\
27.1373369336477	10.0332330094817\\
27.1369550271848	10.0329593821363\\
27.1365731039292	10.0326857461707\\
27.1361911638796	10.0324121015843\\
27.1358092070343	10.0321384483766\\
27.135427233392	10.031864786547\\
27.1350452429511	10.0315911160951\\
27.1346632357103	10.0313174370202\\
27.1342812116679	10.0310437493219\\
27.1338991708225	10.0307700529995\\
27.1335171131727	10.0304963480526\\
27.133135038717	10.0302226344805\\
27.1327529474539	10.0299489122828\\
27.1323708393818	10.0296751814589\\
27.1319887144994	10.0294014420082\\
27.1316065728052	10.0291276939303\\
27.1312244142976	10.0288539372245\\
27.1308422389752	10.0285801718904\\
27.1304600468365	10.0283063979274\\
27.1300778378801	10.0280326153348\\
27.1296956121044	10.0277588241123\\
27.129313369508	10.0274850242593\\
27.1289311100894	10.0272112157751\\
27.1285488338471	10.0269373986593\\
27.1281665407796	10.0266635729113\\
27.1277842308855	10.0263897385306\\
27.1274019041633	10.0261158955166\\
27.1270195606114	10.0258420438688\\
27.1266372002284	10.0255681835867\\
27.1262548230129	10.0252943146696\\
27.1258724289633	10.0250204371171\\
27.1254900180782	10.0247465509287\\
27.125107590356	10.0244726561036\\
27.1247251457954	10.0241987526415\\
27.1243426843947	10.0239248405418\\
27.1239602061526	10.0236509198039\\
27.1235777110675	10.0233769904272\\
27.123195199138	10.0231030524113\\
27.1228126703626	10.0228291057557\\
27.1224301247397	10.0225551504596\\
27.122047562268	10.0222811865227\\
27.1216649829459	10.0220072139443\\
27.1212823867719	10.0217332327239\\
27.1208997737446	10.021459242861\\
27.1205171438625	10.021185244355\\
27.120134497124	10.0209112372054\\
27.1197518335278	10.0206372214117\\
27.1193691530722	10.0203631969732\\
27.1189864557559	10.0200891638894\\
27.1186037415774	10.0198151221599\\
27.1182210105351	10.019541071784\\
27.1178382626275	10.0192670127612\\
27.1174554978533	10.018992945091\\
27.1170727162108	10.0187188687728\\
27.1166899176987	10.018444783806\\
27.1163071023154	10.0181706901902\\
27.1159242700594	10.0178965879247\\
27.1155414209293	10.0176224770091\\
27.1151585549236	10.0173483574428\\
27.1147756720407	10.0170742292252\\
27.1143927722792	10.0168000923558\\
27.1140098556376	10.016525946834\\
27.1136269221145	10.0162517926593\\
27.1132439717082	10.0159776298312\\
27.1128610044174	10.0157034583491\\
27.1124780202405	10.0154292782125\\
27.1120950191761	10.0151550894207\\
27.1117120012226	10.0148808919734\\
27.1113289663787	10.0146066858698\\
27.1109459146427	10.0143324711096\\
27.1105628460132	10.014058247692\\
27.1101797604887	10.0137840156167\\
27.1097966580677	10.013509774883\\
27.1094135387488	10.0132355254903\\
27.1090304025303	10.0129612674382\\
27.108647249411	10.0126870007261\\
27.1082640793892	10.0124127253535\\
27.1078808924634	10.0121384413197\\
27.1074976886322	10.0118641486243\\
27.1071144678941	10.0115898472668\\
27.1067312302476	10.0113155372464\\
27.1063479756912	10.0110412185628\\
27.1059647042234	10.0107668912153\\
27.1055814158427	10.0104925552035\\
27.1051981105476	10.0102182105267\\
27.1048147883366	10.0099438571845\\
27.1044314492083	10.0096694951762\\
27.1040480931611	10.0093951245014\\
27.1036647201935	10.0091207451594\\
27.1032813303041	10.0088463571498\\
27.1028979234913	10.008571960472\\
27.1025144997538	10.0082975551253\\
27.1021310590898	10.0080231411094\\
27.1017476014981	10.0077487184237\\
27.101364126977	10.0074742870675\\
27.1009806355251	10.0071998470403\\
27.1005971271409	10.0069253983417\\
27.1002136018229	10.006650940971\\
27.0998300595696	10.0063764749277\\
27.0994465003795	10.0061020002113\\
27.0990629242511	10.0058275168212\\
27.0986793311829	10.0055530247568\\
27.0982957211735	10.0052785240177\\
27.0979120942212	10.0050040146032\\
27.0975284503247	10.0047294965128\\
27.0971447894824	10.004454969746\\
27.0967611116928	10.0041804343022\\
27.0963774169544	10.0039058901808\\
27.0959937052658	10.0036313373814\\
27.0956099766254	10.0033567759034\\
27.0952262310318	10.0030822057461\\
27.0948424684834	10.0028076269092\\
27.0944586889787	10.0025330393919\\
27.0940748925163	10.0022584431939\\
27.0936910790946	10.0019838383145\\
27.0933072487121	10.0017092247531\\
27.0929234013674	10.0014346025093\\
27.092539537059	10.0011599715824\\
27.0921556557853	10.000885331972\\
27.0917717575448	10.0006106836775\\
27.0913878423361	10.0003360266983\\
27.0910039101576	10.0000613610339\\
27.0906199610079	9.99978668668368\\
27.0902359948855	9.99951200364717\\
27.0898520117887	9.9992373119238\\
27.0894680117162	9.99896261151302\\
27.0890839946665	9.99868790241427\\
27.088699960638	9.99841318462701\\
27.0883159096293	9.99813845815068\\
27.0879318416387	9.99786372298474\\
27.087547756665	9.99758897912864\\
27.0871636547064	9.99731422658182\\
27.0867795357616	9.99703946534374\\
27.086395399829	9.99676469541385\\
27.0860112469072	9.99648991679159\\
27.0856270769946	9.99621512947643\\
27.0852428900897	9.99594033346779\\
27.084858686191	9.99566552876515\\
27.084474465297	9.99539071536795\\
27.0840902274063	9.99511589327563\\
27.0837059725172	9.99484106248764\\
27.0833217006284	9.99456622300345\\
27.0829374117382	9.99429137482249\\
27.0825531058453	9.99401651794421\\
27.082168782948	9.99374165236808\\
27.0817844430449	9.99346677809352\\
27.0814000861346	9.99319189512\\
27.0810157122153	9.99291700344696\\
27.0806313212858	9.99264210307386\\
27.0802469133444	9.99236719400013\\
27.0798624883897	9.99209227622524\\
27.0794780464201	9.99181734974863\\
27.0790935874342	9.99154241456975\\
27.0787091114304	9.99126747068805\\
27.0783246184073	9.99099251810298\\
27.0779401083632	9.99071755681398\\
27.0775555812968	9.99044258682051\\
27.0771710372065	9.99016760812201\\
27.0767864760909	9.98989262071794\\
27.0764018979483	9.98961762460773\\
27.0760173027773	9.98934261979085\\
27.0756326905763	9.98906760626674\\
27.075248061344	9.98879258403485\\
27.0748634150787	9.98851755309462\\
27.074478751779	9.98824251344551\\
27.0740940714433	9.98796746508696\\
27.0737093740702	9.98769240801842\\
27.073324659658	9.98741734223934\\
27.0729399282055	9.98714226774918\\
27.0725551797109	9.98686718454737\\
27.0721704141728	9.98659209263336\\
27.0717856315897	9.98631699200661\\
27.0714008319601	9.98604188266655\\
27.0710160152824	9.98576676461265\\
27.0706311815552	9.98549163784435\\
27.070246330777	9.98521650236109\\
27.0698614629462	9.98494135816232\\
27.0694765780613	9.98466620524749\\
27.0690916761209	9.98439104361605\\
27.0687067571233	9.98411587326745\\
27.0683218210671	9.98384069420113\\
27.0679368679508	9.98356550641655\\
27.0675518977729	9.98329030991314\\
27.0671669105318	9.98301510469036\\
27.0667819062261	9.98273989074765\\
27.0663968848542	9.98246466808447\\
27.0660118464146	9.98218943670025\\
27.0656267909058	9.98191419659445\\
27.0652417183263	9.98163894776651\\
27.0648566286746	9.98136369021589\\
27.0644715219491	9.98108842394202\\
27.0640863981484	9.98081314894435\\
27.0637012572709	9.98053786522234\\
27.0633160993152	9.98026257277543\\
27.0629309242796	9.97998727160306\\
27.0625457321627	9.97971196170468\\
27.062160522963	9.97943664307975\\
27.061775296679	9.9791613157277\\
27.0613900533091	9.97888597964798\\
27.0610047928518	9.97861063484005\\
27.0606195153056	9.97833528130333\\
27.0602342206691	9.97805991903729\\
27.0598489089406	9.97778454804138\\
27.0594635801186	9.97750916831502\\
27.0590782342018	9.97723377985768\\
27.0586928711884	9.9769583826688\\
27.0583074910771	9.97668297674782\\
27.0579220938662	9.97640756209419\\
27.0575366795543	9.97613213870736\\
27.0571512481399	9.97585670658677\\
27.0567657996215	9.97558126573187\\
27.0563803339974	9.97530581614211\\
27.0559948512663	9.97503035781693\\
27.0556093514265	9.97475489075577\\
27.0552238344766	9.97447941495809\\
27.0548383004151	9.97420393042333\\
27.0544527492404	9.97392843715093\\
27.054067180951	9.97365293514035\\
27.0536815955454	9.97337742439101\\
27.053295993022	9.97310190490238\\
27.0529103733794	9.9728263766739\\
27.0525247366161	9.97255083970501\\
27.0521390827304	9.97227529399516\\
27.051753411721	9.97199973954379\\
27.0513677235862	9.97172417635035\\
27.0509820183245	9.97144860441429\\
27.0505962959345	9.97117302373504\\
27.0502105564146	9.97089743431206\\
27.0498247997633	9.97062183614479\\
27.049439025979	9.97034622923267\\
27.0490532350603	9.97007061357516\\
27.0486674270056	9.96979498917169\\
27.0482816018134	9.96951935602171\\
27.0478957594822	9.96924371412466\\
27.0475099000104	9.96896806347999\\
27.0471240233965	9.96869240408715\\
27.0467381296391	9.96841673594557\\
27.0463522187365	9.96814105905471\\
27.0459662906873	9.96786537341401\\
27.04558034549	9.96758967902291\\
27.0451943831429	9.96731397588086\\
27.0448084036446	9.9670382639873\\
27.0444224069936	9.96676254334167\\
27.0440363931883	9.96648681394343\\
27.0436503622273	9.96621107579201\\
27.0432643141089	9.96593532888686\\
27.0428782488317	9.96565957322742\\
27.0424921663941	9.96538380881314\\
27.0421060667946	9.96510803564346\\
27.0417199500317	9.96483225371783\\
27.0413338161039	9.96455646303568\\
27.0409476650096	9.96428066359647\\
27.0405614967473	9.96400485539964\\
27.0401753113155	9.96372903844462\\
27.0397891087126	9.96345321273088\\
27.0394028889372	9.96317737825784\\
27.0390166519877	9.96290153502495\\
27.0386303978625	9.96262568303166\\
27.0382441265602	9.9623498222774\\
27.0378578380792	9.96207395276163\\
27.037471532418	9.96179807448379\\
27.0370852095751	9.96152218744331\\
27.0366988695488	9.96124629163965\\
27.0363125123378	9.96097038707224\\
27.0359261379405	9.96069447374054\\
27.0355397463553	9.96041855164397\\
27.0351533375807	9.960142620782\\
27.0347669116152	9.95986668115405\\
27.0343804684572	9.95959073275957\\
27.0339940081053	9.959314775598\\
27.0336075305579	9.9590388096688\\
27.0332210358135	9.95876283497139\\
27.0328345238704	9.95848685150523\\
27.0324479947273	9.95821085926975\\
27.0320614483826	9.9579348582644\\
27.0316748848347	9.95765884848863\\
27.0312883040821	9.95738282994187\\
27.0309017061233	9.95710680262356\\
27.0305150909567	9.95683076653315\\
27.0301284585809	9.95655472167009\\
27.0297418089942	9.9562786680338\\
27.0293551421952	9.95600260562375\\
27.0289684581823	9.95572653443936\\
27.028581756954	9.95545045448008\\
27.0281950385088	9.95517436574536\\
27.0278083028451	9.95489826823463\\
27.0274215499613	9.95462216194733\\
27.027034779856	9.95434604688292\\
27.0266479925277	9.95406992304083\\
27.0262611879747	9.9537937904205\\
27.0258743661955	9.95351764902138\\
27.0254875271887	9.9532414988429\\
27.0251006709527	9.9529653398845\\
27.0247137974859	9.95268917214564\\
27.0243269067868	9.95241299562575\\
27.0239399988539	9.95213681032428\\
27.0235530736857	9.95186061624065\\
27.0231661312805	9.95158441337433\\
27.022779171637	9.95130820172474\\
27.0223921947534	9.95103198129133\\
27.0220052006284	9.95075575207354\\
27.0216181892603	9.95047951407081\\
27.0212311606477	9.95020326728258\\
27.0208441147889	9.9499270117083\\
27.0204570516825	9.9496507473474\\
27.0200699713269	9.94937447419933\\
27.0196828737205	9.94909819226353\\
27.019295758862	9.94882190153943\\
27.0189086267496	9.94854560202648\\
27.0185214773819	9.94826929372412\\
27.0181343107573	9.9479929766318\\
27.0177471268743	9.94771665074894\\
27.0173599257313	9.94744031607499\\
27.0169727073268	9.9471639726094\\
27.0165854716594	9.9468876203516\\
27.0161982187273	9.94661125930103\\
27.0158109485291	9.94633488945714\\
27.0154236610633	9.94605851081936\\
27.0150363563283	9.94578212338713\\
27.0146490343226	9.94550572715991\\
27.0142616950446	9.94522932213711\\
27.0138743384928	9.94495290831819\\
27.0134869646656	9.94467648570259\\
27.0130995735616	9.94440005428974\\
27.0127121651791	9.94412361407909\\
27.0123247395166	9.94384716507007\\
27.0119372965727	9.94357070726213\\
27.0115498363456	9.94329424065471\\
27.011162358834	9.94301776524724\\
27.0107748640362	9.94274128103916\\
27.0103873519508	9.94246478802992\\
27.0099998225761	9.94218828621895\\
27.0096122759106	9.9419117756057\\
27.0092247119528	9.9416352561896\\
27.0088371307012	9.94135872797009\\
27.0084495321542	9.94108219094662\\
27.0080619163102	9.94080564511862\\
27.0076742831678	9.94052909048553\\
27.0072866327253	9.94025252704679\\
27.0068989649812	9.93997595480184\\
27.0065112799341	9.93969937375011\\
27.0061235775822	9.93942278389106\\
27.0057358579242	9.93914618522411\\
27.0053481209584	9.93886957774871\\
27.0049603666833	9.9385929614643\\
27.0045725950973	9.93831633637031\\
27.004184806199	9.93803970246617\\
27.0037969999867	9.93776305975135\\
27.0034091764589	9.93748640822526\\
27.0030213356141	9.93720974788735\\
27.0026334774508	9.93693307873706\\
27.0022456019673	9.93665640077383\\
27.0018577091621	9.93637971399709\\
27.0014697990337	9.93610301840628\\
27.0010818715806	9.93582631400085\\
27.0006939268012	9.93554960078022\\
27.0003059646938	9.93527287874385\\
26.9999179852571	9.93499614789116\\
26.9995299884895	9.93471940822159\\
26.9991419743893	9.93444265973459\\
26.998753942955	9.93416590242959\\
26.9983658941852	9.93388913630603\\
26.9979778280782	9.93361236136334\\
26.9975897446326	9.93333557760097\\
26.9972016438466	9.93305878501836\\
26.9968135257189	9.93278198361493\\
26.9964253902478	9.93250517339013\\
26.9960372374318	9.9322283543434\\
26.9956490672694	9.93195152647417\\
26.995260879759	9.93167468978189\\
26.994872674899	9.93139784426598\\
26.9944844526879	9.93112098992589\\
26.9940962131242	9.93084412676106\\
26.9937079562063	9.93056725477091\\
26.9933196819326	9.9302903739549\\
26.9929313903016	9.93001348431245\\
26.9925430813117	9.929736585843\\
26.9921547549615	9.929459678546\\
26.9917664112493	9.92918276242087\\
26.9913780501735	9.92890583746706\\
26.9909896717327	9.928628903684\\
26.9906012759253	9.92835196107113\\
26.9902128627497	9.92807500962788\\
26.9898244322043	9.9277980493537\\
26.9894359842877	9.92752108024801\\
26.9890475189983	9.92724410231027\\
26.9886590363344	9.92696711553989\\
26.9882705362946	9.92669011993633\\
26.9878820188774	9.92641311549901\\
26.987493484081	9.92613610222737\\
26.9871049319041	9.92585908012085\\
26.986716362345	9.92558204917889\\
26.9863277754021	9.92530500940092\\
26.985939171074	9.92502796078638\\
26.9855505493591	9.9247509033347\\
26.9851619102558	9.92447383704533\\
26.9847732537625	9.92419676191769\\
26.9843845798778	9.92391967795122\\
26.9839958886	9.92364258514536\\
26.9836071799276	9.92336548349955\\
26.983218453859	9.92308837301322\\
26.9828297103927	9.92281125368581\\
26.9824409495271	9.92253412551675\\
26.9820521712606	9.92225698850547\\
26.9816633755918	9.92197984265142\\
26.981274562519	9.92170268795404\\
26.9808857320407	9.92142552441274\\
26.9804968841553	9.92114835202698\\
26.9801080188613	9.92087117079619\\
26.9797191361572	9.92059398071979\\
26.9793302360412	9.92031678179724\\
26.978941318512	9.92003957402796\\
26.9785523835678	9.91976235741138\\
26.9781634312073	9.91948513194695\\
26.9777744614288	9.9192078976341\\
26.9773854742307	9.91893065447226\\
26.9769964696115	9.91865340246087\\
26.9766074475696	9.91837614159937\\
26.9762184081036	9.91809887188719\\
26.9758293512117	9.91782159332375\\
26.9754402768925	9.91754430590851\\
26.9750511851443	9.91726700964089\\
26.9746620759657	9.91698970452033\\
26.9742729493551	9.91671239054625\\
26.9738838053108	9.91643506771812\\
26.9734946438314	9.91615773603534\\
26.9731054649153	9.91588039549735\\
26.9727162685609	9.9156030461036\\
26.9723270547666	9.91532568785352\\
26.971937823531	9.91504832074653\\
26.9715485748523	9.91477094478208\\
26.9711593087291	9.9144935599596\\
26.9707700251599	9.91421616627852\\
26.9703807241429	9.91393876373827\\
26.9699914056768	9.9136613523383\\
26.9696020697598	9.91338393207804\\
26.9692127163905	9.91310650295691\\
26.9688233455672	9.91282906497436\\
26.9684339572885	9.91255161812981\\
26.9680445515527	9.9122741624227\\
26.9676551283583	9.91199669785247\\
26.9672656877037	9.91171922441855\\
26.9668762295874	9.91144174212037\\
26.9664867540077	9.91116425095736\\
26.9660972609632	9.91088675092897\\
26.9657077504522	9.91060924203461\\
26.9653182224732	9.91033172427374\\
26.9649286770247	9.91005419764577\\
26.964539114105	9.90977666215015\\
26.9641495337125	9.9094991177863\\
26.9637599358458	9.90922156455366\\
26.9633703205032	9.90894400245166\\
26.9629806876833	9.90866643147974\\
26.9625910373843	9.90838885163733\\
26.9622013696048	9.90811126292386\\
26.9618116843431	9.90783366533877\\
26.9614219815978	9.90755605888148\\
26.9610322613672	9.90727844355144\\
26.9606425236497	9.90700081934807\\
26.9602527684439	9.9067231862708\\
26.9598629957481	9.90644554431907\\
26.9594732055608	9.90616789349232\\
26.9590833978803	9.90589023378997\\
26.9586935727052	9.90561256521145\\
26.9583037300338	9.90533488775621\\
26.9579138698646	9.90505720142366\\
26.9575239921961	9.90477950621325\\
26.9571340970265	9.90450180212441\\
26.9567441843544	9.90422408915656\\
26.9563542541782	9.90394636730915\\
26.9559643064964	9.9036686365816\\
26.9555743413073	9.90339089697334\\
26.9551843586094	9.90311314848381\\
26.954794358401	9.90283539111244\\
26.9544043406808	9.90255762485866\\
26.9540143054469	9.90227984972191\\
26.953624252698	9.90200206570161\\
26.9532341824324	9.90172427279719\\
26.9528440946485	9.90144647100809\\
26.9524539893448	9.90116866033375\\
26.9520638665197	9.90089084077358\\
26.9516737261716	9.90061301232703\\
26.951283568299	9.90033517499352\\
26.9508933929003	9.90005732877249\\
26.9505031999738	9.89977947366336\\
26.9501129895181	9.89950160966557\\
26.9497227615315	9.89922373677856\\
26.9493325160124	9.89894585500174\\
26.9489422529594	9.89866796433456\\
26.9485519723708	9.89839006477644\\
26.9481616742451	9.89811215632681\\
26.9477713585806	9.89783423898512\\
26.9473810253758	9.89755631275077\\
26.9469906746291	9.89727837762322\\
26.946600306339	9.89700043360188\\
26.9462099205038	9.89672248068619\\
26.9458195171221	9.89644451887558\\
26.9454290961921	9.89616654816948\\
26.9450386577124	9.89588856856733\\
26.9446482016813	9.89561058006854\\
26.9442577280973	9.89533258267256\\
26.9438672369587	9.89505457637881\\
26.9434767282641	9.89477656118672\\
26.9430862020119	9.89449853709572\\
26.9426956582004	9.89422050410525\\
26.9423050968281	9.89394246221473\\
26.9419145178934	9.8936644114236\\
26.9415239213947	9.89338635173128\\
26.9411333073304	9.89310828313721\\
26.9407426756991	9.89283020564081\\
26.940352026499	9.89255211924151\\
26.9399613597286	9.89227402393875\\
26.9395706753863	9.89199591973195\\
26.9391799734706	9.89171780662055\\
26.9387892539798	9.89143968460397\\
26.9383985169124	9.89116155368164\\
26.9380077622668	9.890883413853\\
26.9376169900415	9.89060526511747\\
26.9372262002347	9.89032710747448\\
26.936835392845	9.89004894092346\\
26.9364445678708	9.88977076546384\\
26.9360537253105	9.88949258109506\\
26.9356628651624	9.88921438781653\\
26.9352719874251	9.88893618562769\\
26.934881092097	9.88865797452797\\
26.9344901791764	9.8883797545168\\
26.9340992486617	9.8881015255936\\
26.9337083005515	9.88782328775781\\
26.933317334844	9.88754504100886\\
26.9329263515378	9.88726678534616\\
26.9325353506312	9.88698852076916\\
26.9321443321227	9.88671024727728\\
26.9317532960106	9.88643196486995\\
26.9313622422934	9.8861536735466\\
26.9309711709695	9.88587537330665\\
26.9305800820373	9.88559706414954\\
26.9301889754952	9.8853187460747\\
26.9297978513417	9.88504041908154\\
26.9294067095751	9.88476208316951\\
26.9290155501939	9.88448373833803\\
26.9286243731965	9.88420538458653\\
26.9282331785812	9.88392702191443\\
26.9278419663466	9.88364865032116\\
26.927450736491	9.88337026980616\\
26.9270594890128	9.88309188036885\\
26.9266682239104	9.88281348200866\\
26.9262769411823	9.88253507472502\\
26.9258856408269	9.88225665851735\\
26.9254943228426	9.88197823338508\\
26.9251029872277	9.88169979932764\\
26.9247116339808	9.88142135634446\\
26.9243202631001	9.88114290443496\\
26.9239288745842	9.88086444359858\\
26.9235374684314	9.88058597383474\\
26.9231460446402	9.88030749514287\\
26.9227546032089	9.8800290075224\\
26.922363144136	9.87975051097275\\
26.9219716674198	9.87947200549335\\
26.9215801730588	9.87919349108362\\
26.9211886610515	9.87891496774301\\
26.9207971313961	9.87863643547092\\
26.9204055840911	9.8783578942668\\
26.920014019135	9.87807934413006\\
26.9196224365261	9.87780078506014\\
26.9192308362628	9.87752221705646\\
26.9188392183436	9.87724364011844\\
26.9184475827668	9.87696505424552\\
26.9180559295309	9.87668645943712\\
26.9176642586343	9.87640785569268\\
26.9172725700753	9.8761292430116\\
26.9168808638524	9.87585062139333\\
26.916489139964	9.87557199083729\\
26.9160973984085	9.8752933513429\\
26.9157056391843	9.87501470290959\\
26.9153138622898	9.8747360455368\\
26.9149220677234	9.87445737922393\\
26.9145302554835	9.87417870397043\\
26.9141384255686	9.87390001977571\\
26.9137465779769	9.87362132663921\\
26.9133547127071	9.87334262456035\\
26.9129628297573	9.87306391353855\\
26.9125709291261	9.87278519357324\\
26.9121790108119	9.87250646466385\\
26.911787074813	9.87222772680981\\
26.9113951211278	9.87194898001053\\
26.9110031497549	9.87167022426546\\
26.9106111606924	9.871391459574\\
26.910219153939	9.87111268593559\\
26.9098271294929	9.87083390334965\\
26.9094350873526	9.87055511181561\\
26.9090430275165	9.87027631133289\\
26.9086509499829	9.86999750190092\\
26.9082588547504	9.86971868351913\\
26.9078667418172	9.86943985618693\\
26.9074746111818	9.86916101990377\\
26.9070824628426	9.86888217466905\\
26.9066902967979	9.86860332048221\\
26.9062981130463	9.86832445734267\\
26.905905911586	9.86804558524986\\
26.9055136924155	9.86776670420319\\
26.9051214555333	9.86748781420211\\
26.9047292009376	9.86720891524602\\
26.9043369286269	9.86693000733437\\
26.9039446385996	9.86665109046656\\
26.9035523308541	9.86637216464203\\
26.9031600053888	9.8660932298602\\
26.902767662202	9.86581428612049\\
26.9023753012923	9.86553533342234\\
26.9019829226579	9.86525637176516\\
26.9015905262973	9.86497740114837\\
26.9011981122089	9.86469842157142\\
26.9008056803911	9.8644194330337\\
26.9004132308423	9.86414043553467\\
26.9000207635608	9.86386142907372\\
26.8996282785452	9.8635824136503\\
26.8992357757936	9.86330338926383\\
26.8988432553047	9.86302435591372\\
26.8984507170767	9.86274531359941\\
26.8980581611081	9.86246626232031\\
26.8976655873973	9.86218720207585\\
26.8972729959426	9.86190813286546\\
26.8968803867424	9.86162905468856\\
26.8964877597952	9.86134996754457\\
26.8960951150993	9.86107087143292\\
26.8957024526532	9.86079176635302\\
26.8953097724552	9.86051265230432\\
26.8949170745038	9.86023352928621\\
26.8945243587972	9.85995439729814\\
26.894131625334	9.85967525633952\\
26.8937388741125	9.85939610640979\\
26.893346105131	9.85911694750835\\
26.8929533183881	9.85883777963463\\
26.8925605138821	9.85855860278807\\
26.8921676916113	9.85827941696807\\
26.8917748515743	9.85800022217407\\
26.8913819937693	9.85772101840548\\
26.8909891181947	9.85744180566174\\
26.8905962248491	9.85716258394225\\
26.8902033137306	9.85688335324646\\
26.8898103848379	9.85660411357377\\
26.8894174381691	9.85632486492361\\
26.8890244737228	9.85604560729541\\
26.8886314914972	9.85576634068858\\
26.8882384914909	9.85548706510255\\
26.8878454737022	9.85520778053675\\
26.8874524381295	9.85492848699059\\
26.8870593847711	9.8546491844635\\
26.8866663136255	9.85436987295489\\
26.8862732246911	9.8540905524642\\
26.8858801179662	9.85381122299084\\
26.8854869934493	9.85353188453424\\
26.8850938511387	9.85325253709381\\
26.8847006910328	9.85297318066899\\
26.88430751313	9.85269381525919\\
26.8839143174287	9.85241444086384\\
26.8835211039273	9.85213505748234\\
26.8831278726241	9.85185566511414\\
26.8827346235176	9.85157626375865\\
26.8823413566062	9.8512968534153\\
26.8819480718882	9.85101743408349\\
26.881554769362	9.85073800576266\\
26.881161449026	9.85045856845223\\
26.8807681108787	9.85017912215162\\
26.8803747549183	9.84989966686024\\
26.8799813811432	9.84962020257753\\
26.879587989552	9.8493407293029\\
26.8791945801428	9.84906124703578\\
26.8788011529142	9.84878175577558\\
26.8784077078645	9.84850225552173\\
26.8780142449921	9.84822274627365\\
26.8776207642954	9.84794322803075\\
26.8772272657728	9.84766370079247\\
26.8768337494226	9.84738416455821\\
26.8764402152432	9.84710461932741\\
26.8760466632331	9.84682506509949\\
26.8756530933906	9.84654550187385\\
26.875259505714	9.84626592964993\\
26.8748659002019	9.84598634842715\\
26.8744722768525	9.84570675820492\\
26.8740786356642	9.84542715898267\\
26.8736849766355	9.84514755075982\\
26.8732912997647	9.84486793353579\\
26.8728976050501	9.84458830730999\\
26.8725038924903	9.84430867208185\\
26.8721101620834	9.84402902785079\\
26.8717164138281	9.84374937461623\\
26.8713226477225	9.84346971237759\\
26.8709288637651	9.84319004113429\\
26.8705350619543	9.84291036088575\\
26.8701412422885	9.84263067163139\\
26.869747404766	9.84235097337063\\
26.8693535493853	9.84207126610288\\
26.8689596761446	9.84179154982758\\
26.8685657850425	9.84151182454414\\
26.8681718760771	9.84123209025198\\
26.8677779492471	9.84095234695051\\
26.8673840045506	9.84067259463917\\
26.8669900419862	9.84039283331736\\
26.8665960615521	9.84011306298451\\
26.8662020632468	9.83983328364004\\
26.8658080470687	9.83955349528336\\
26.865414013016	9.8392736979139\\
26.8650199610873	9.83899389153108\\
26.8646258912808	9.83871407613431\\
26.8642318035949	9.83843425172301\\
26.8638376980281	9.83815441829661\\
26.8634435745787	9.83787457585452\\
26.8630494332451	9.83759472439616\\
26.8626552740257	9.83731486392096\\
26.8622610969188	9.83703499442831\\
26.8618669019227	9.83675511591766\\
26.861472689036	9.83647522838842\\
26.861078458257	9.83619533184\\
26.860684209584	9.83591542627182\\
26.8602899430154	9.83563551168331\\
26.8598956585495	9.83535558807388\\
26.8595013561849	9.83507565544295\\
26.8591070359198	9.83479571378994\\
26.8587126977526	9.83451576311427\\
26.8583183416817	9.83423580341535\\
26.8579239677055	9.8339558346926\\
26.8575295758223	9.83367585694545\\
26.8571351660305	9.83339587017331\\
26.8567407383285	9.83311587437559\\
26.8563462927147	9.83283586955172\\
26.8559518291874	9.83255585570112\\
26.8555573477449	9.83227583282319\\
26.8551628483858	9.83199580091737\\
26.8547683311083	9.83171575998307\\
26.8543737959109	9.8314357100197\\
26.8539792427918	9.83115565102669\\
26.8535846717495	9.83087558300345\\
26.8531900827824	9.8305955059494\\
26.8527954758887	9.83031541986395\\
26.852400851067	9.83003532474653\\
26.8520062083154	9.82975522059655\\
26.8516115476325	9.82947510741342\\
26.8512168690166	9.82919498519658\\
26.8508221724661	9.82891485394543\\
26.8504274579793	9.82863471365939\\
26.8500327255546	9.82835456433788\\
26.8496379751903	9.82807440598031\\
26.8492432068849	9.8277942385861\\
26.8488484206367	9.82751406215468\\
26.8484536164441	9.82723387668545\\
};
\addplot [color=mycolor1, forget plot]
  table[row sep=crcr]{%
26.8484536164441	9.82723387668545\\
26.8480587943055	9.82695368217784\\
26.8476639542191	9.82667347863125\\
26.8472690961834	9.82639326604511\\
26.8468742201968	9.82611304441884\\
26.8464793262576	9.82583281375184\\
26.8460844143642	9.82555257404355\\
26.845689484515	9.82527232529336\\
26.8452945367082	9.82499206750071\\
26.8448995709424	9.824711800665\\
26.8445045872158	9.82443152478566\\
26.8441095855268	9.82415123986209\\
26.8437145658739	9.82387094589372\\
26.8433195282552	9.82359064287997\\
26.8429244726693	9.82331033082024\\
26.8425293991145	9.82303000971395\\
26.8421343075891	9.82274967956053\\
26.8417391980916	9.82246934035938\\
26.8413440706202	9.82218899210992\\
26.8409489251734	9.82190863481157\\
26.8405537617494	9.82162826846374\\
26.8401585803468	9.82134789306586\\
26.8397633809638	9.82106750861733\\
26.8393681635988	9.82078711511757\\
26.8389729282501	9.82050671256599\\
26.8385776749162	9.82022630096202\\
26.8381824035954	9.81994588030506\\
26.837787114286	9.81966545059455\\
26.8373918069864	9.81938501182987\\
26.836996481695	9.81910456401046\\
26.8366011384102	9.81882410713573\\
26.8362057771303	9.8185436412051\\
26.8358103978536	9.81826316621797\\
26.8354150005785	9.81798268217377\\
26.8350195853035	9.81770218907191\\
26.8346241520268	9.8174216869118\\
26.8342287007468	9.81714117569286\\
26.8338332314619	9.81686065541451\\
26.8334377441704	9.81658012607615\\
26.8330422388706	9.81629958767721\\
26.8326467155611	9.8160190402171\\
26.83225117424	9.81573848369523\\
26.8318556149059	9.81545791811102\\
26.8314600375569	9.81517734346388\\
26.8310644421915	9.81489675975322\\
26.8306688288081	9.81461616697847\\
26.830273197405	9.81433556513904\\
26.8298775479806	9.81405495423433\\
26.8294818805332	9.81377433426377\\
26.8290861950611	9.81349370522677\\
26.8286904915628	9.81321306712273\\
26.8282947700366	9.81293241995109\\
26.8278990304808	9.81265176371124\\
26.8275032728939	9.81237109840262\\
26.8271074972741	9.81209042402461\\
26.8267117036198	9.81180974057665\\
26.8263158919295	9.81152904805815\\
26.8259200622013	9.81124834646852\\
26.8255242144338	9.81096763580717\\
26.8251283486251	9.81068691607351\\
26.8247324647738	9.81040618726697\\
26.8243365628782	9.81012544938696\\
26.8239406429366	9.80984470243287\\
26.8235447049473	9.80956394640414\\
26.8231487489087	9.80928318130018\\
26.8227527748193	9.80900240712039\\
26.8223567826772	9.80872162386419\\
26.821960772481	9.808440831531\\
26.8215647442289	9.80816003012023\\
26.8211686979192	9.80787921963128\\
26.8207726335505	9.80759840006358\\
26.8203765511209	9.80731757141653\\
26.8199804506288	9.80703673368955\\
26.8195843320727	9.80675588688206\\
26.8191881954508	9.80647503099346\\
26.8187920407616	9.80619416602316\\
26.8183958680033	9.80591329197059\\
26.8179996771742	9.80563240883515\\
26.8176034682729	9.80535151661625\\
26.8172072412976	9.80507061531331\\
26.8168109962466	9.80478970492575\\
26.8164147331184	9.80450878545296\\
26.8160184519112	9.80422785689436\\
26.8156221526234	9.80394691924938\\
26.8152258352534	9.80366597251742\\
26.8148294997996	9.80338501669788\\
26.8144331462601	9.80310405179019\\
26.8140367746335	9.80282307779375\\
26.8136403849181	9.80254209470798\\
26.8132439771122	9.8022611025323\\
26.8128475512141	9.8019801012661\\
26.8124511072222	9.8016990909088\\
26.8120546451349	9.80141807145982\\
26.8116581649506	9.80113704291857\\
26.8112616666674	9.80085600528445\\
26.8108651502839	9.80057495855688\\
26.8104686157983	9.80029390273528\\
26.810072063209	9.80001283781904\\
26.8096754925144	9.79973176380759\\
26.8092789037127	9.79945068070034\\
26.8088822968024	9.79916958849669\\
26.8084856717817	9.79888848719606\\
26.8080890286491	9.79860737679786\\
26.8076923674029	9.7983262573015\\
26.8072956880413	9.79804512870639\\
26.8068989905629	9.79776399101194\\
26.8065022749658	9.79748284421756\\
26.8061055412485	9.79720168832267\\
26.8057087894094	9.79692052332668\\
26.8053120194466	9.79663934922899\\
26.8049152313586	9.79635816602901\\
26.8045184251438	9.79607697372616\\
26.8041216008004	9.79579577231985\\
26.8037247583269	9.79551456180949\\
26.8033278977215	9.79523334219449\\
26.8029310189827	9.79495211347425\\
26.8025341221086	9.79467087564819\\
26.8021372070978	9.79438962871573\\
26.8017402739485	9.79410837267626\\
26.8013433226591	9.79382710752921\\
26.8009463532279	9.79354583327397\\
26.8005493656532	9.79326454990996\\
26.8001523599335	9.7929832574366\\
26.799755336067	9.79270195585328\\
26.799358294052	9.79242064515942\\
26.798961233887	9.79213932535444\\
26.7985641555703	9.79185799643774\\
26.7981670591001	9.79157665840872\\
26.7977699444749	9.7912953112668\\
26.797372811693	9.7910139550114\\
26.7969756607527	9.79073258964191\\
26.7965784916524	9.79045121515774\\
26.7961813043904	9.79016983155832\\
26.795784098965	9.78988843884304\\
26.7953868753746	9.78960703701132\\
26.7949896336175	9.78932562606257\\
26.7945923736921	9.78904420599618\\
26.7941950955966	9.78876277681159\\
26.7937977993295	9.78848133850818\\
26.7934004848891	9.78819989108538\\
26.7930031522737	9.78791843454259\\
26.7926058014816	9.78763696887922\\
26.7922084325112	9.78735549409467\\
26.7918110453608	9.78707401018837\\
26.7914136400288	9.78679251715971\\
26.7910162165134	9.7865110150081\\
26.7906187748131	9.78622950373296\\
26.7902213149262	9.7859479833337\\
26.7898238368509	9.78566645380971\\
26.7894263405857	9.78538491516041\\
26.7890288261288	9.78510336738521\\
26.7886312934787	9.78482181048351\\
26.7882337426335	9.78454024445473\\
26.7878361735918	9.78425866929828\\
26.7874385863517	9.78397708501355\\
26.7870409809117	9.78369549159996\\
26.7866433572701	9.78341388905692\\
26.7862457154252	9.78313227738384\\
26.7858480553753	9.78285065658011\\
26.7854503771188	9.78256902664516\\
26.785052680654	9.78228738757839\\
26.7846549659792	9.7820057393792\\
26.7842572330928	9.78172408204701\\
26.7838594819931	9.78144241558122\\
26.7834617126785	9.78116073998123\\
26.7830639251472	9.78087905524647\\
26.7826661193976	9.78059736137633\\
26.782268295428	9.78031565837022\\
26.7818704532368	9.78003394622755\\
26.7814725928223	9.77975222494773\\
26.7810747141828	9.77947049453016\\
26.7806768173166	9.77918875497426\\
26.7802789022222	9.77890700627942\\
26.7798809688977	9.77862524844505\\
26.7794830173416	9.77834348147057\\
26.7790850475522	9.77806170535538\\
26.7786870595277	9.77777992009889\\
26.7782890532666	9.7774981257005\\
26.7778910287671	9.77721632215961\\
26.7774929860276	9.77693450947565\\
26.7770949250465	9.77665268764801\\
26.7766968458219	9.7763708566761\\
26.7762987483524	9.77608901655932\\
26.7759006326361	9.77580716729709\\
26.7755024986715	9.77552530888881\\
26.7751043464568	9.77524344133388\\
26.7747061759904	9.77496156463172\\
26.7743079872706	9.77467967878172\\
26.7739097802957	9.7743977837833\\
26.773511555064	9.77411587963586\\
26.773113311574	9.77383396633881\\
26.7727150498239	9.77355204389155\\
26.772316769812	9.77327011229349\\
26.7719184715366	9.77298817154404\\
26.7715201549962	9.77270622164259\\
26.7711218201889	9.77242426258857\\
26.7707234671132	9.77214229438136\\
26.7703250957673	9.77186031702039\\
26.7699267061496	9.77157833050504\\
26.7695282982585	9.77129633483474\\
26.7691298720921	9.77101433000889\\
26.7687314276489	9.77073231602688\\
26.7683329649272	9.77045029288813\\
26.7679344839252	9.77016826059204\\
26.7675359846414	9.76988621913801\\
26.7671374670741	9.76960416852546\\
26.7667389312215	9.76932210875379\\
26.7663403770819	9.76904003982239\\
26.7659418046538	9.76875796173069\\
26.7655432139354	9.76847587447807\\
26.7651446049251	9.76819377806396\\
26.7647459776212	9.76791167248775\\
26.7643473320219	9.76762955774884\\
26.7639486681256	9.76734743384664\\
26.7635499859307	9.76706530078057\\
26.7631512854355	9.76678315855001\\
26.7627525666382	9.76650100715438\\
26.7623538295372	9.76621884659308\\
26.7619550741308	9.76593667686551\\
26.7615563004173	9.76565449797108\\
26.7611575083951	9.7653723099092\\
26.7607586980625	9.76509011267926\\
26.7603598694177	9.76480790628067\\
26.7599610224592	9.76452569071285\\
26.7595621571852	9.76424346597518\\
26.759163273594	9.76396123206708\\
26.758764371684	9.76367898898794\\
26.7583654514535	9.76339673673718\\
26.7579665129007	9.76311447531419\\
26.7575675560241	9.76283220471839\\
26.7571685808219	9.76254992494917\\
26.7567695872924	9.76226763600593\\
26.756370575434	9.76198533788809\\
26.755971545245	9.76170303059504\\
26.7555724967237	9.76142071412619\\
26.7551734298683	9.76113838848095\\
26.7547743446773	9.76085605365871\\
26.754375241149	9.76057370965887\\
26.7539761192816	9.76029135648086\\
26.7535769790734	9.76000899412405\\
26.7531778205228	9.75972662258787\\
26.7527786436282	9.7594442418717\\
26.7523794483877	9.75916185197496\\
26.7519802347998	9.75887945289706\\
26.7515810028627	9.75859704463738\\
26.7511817525747	9.75831462719533\\
26.7507824839342	9.75803220057032\\
26.7503831969395	9.75774976476175\\
26.7499838915889	9.75746731976903\\
26.7495845678806	9.75718486559154\\
26.7491852258131	9.75690240222871\\
26.7487858653846	9.75661992967993\\
26.7483864865935	9.75633744794459\\
26.7479870894379	9.75605495702212\\
26.7475876739164	9.7557724569119\\
26.7471882400271	9.75548994761334\\
26.7467887877684	9.75520742912584\\
26.7463893171386	9.75492490144881\\
26.7459898281359	9.75464236458165\\
26.7455903207588	9.75435981852375\\
26.7451907950055	9.75407726327452\\
26.7447912508743	9.75379469883337\\
26.7443916883636	9.75351212519969\\
26.7439921074716	9.75322954237289\\
26.7435925081967	9.75294695035237\\
26.7431928905371	9.75266434913752\\
26.7427932544912	9.75238173872776\\
26.7423936000573	9.75209911912248\\
26.7419939272337	9.75181649032109\\
26.7415942360186	9.75153385232298\\
26.7411945264105	9.75125120512756\\
26.7407947984076	9.75096854873423\\
26.7403950520082	9.75068588314239\\
26.7399952872106	9.75040320835145\\
26.7395955040131	9.75012052436079\\
26.7391957024141	9.74983783116983\\
26.7387958824118	9.74955512877797\\
26.7383960440046	9.7492724171846\\
26.7379961871907	9.74898969638913\\
26.7375963119684	9.74870696639096\\
26.7371964183362	9.74842422718948\\
26.7367965062922	9.74814147878411\\
26.7363965758348	9.74785872117424\\
26.7359966269622	9.74757595435927\\
26.7355966596728	9.74729317833861\\
26.735196673965	9.74701039311164\\
26.7347966698369	9.74672759867778\\
26.7343966472869	9.74644479503642\\
26.7339966063132	9.74616198218697\\
26.7335965469143	9.74587916012882\\
26.7331964690884	9.74559632886138\\
26.7327963728338	9.74531348838404\\
26.7323962581488	9.7450306386962\\
26.7319961250317	9.74474777979728\\
26.7315959734808	9.74446491168665\\
26.7311958034944	9.74418203436374\\
26.7307956150708	9.74389914782792\\
26.7303954082083	9.74361625207862\\
26.7299951829053	9.74333334711522\\
26.7295949391599	9.74305043293712\\
26.7291946769706	9.74276750954373\\
26.7287943963356	9.74248457693444\\
26.7283940972532	9.74220163510866\\
26.7279937797217	9.74191868406577\\
26.7275934437394	9.7416357238052\\
26.7271930893046	9.74135275432632\\
26.7267927164157	9.74106977562855\\
26.7263923250708	9.74078678771127\\
26.7259919152683	9.7405037905739\\
26.7255914870066	9.74022078421583\\
26.7251910402839	9.73993776863645\\
26.7247905750984	9.73965474383517\\
26.7243900914486	9.73937170981139\\
26.7239895893326	9.7390886665645\\
26.7235890687489	9.73880561409391\\
26.7231885296956	9.73852255239901\\
26.7227879721711	9.7382394814792\\
26.7223873961737	9.73795640133389\\
26.7219868017017	9.73767331196246\\
26.7215861887534	9.73739021336432\\
26.721185557327	9.73710710553887\\
26.7207849074209	9.7368239884855\\
26.7203842390333	9.73654086220361\\
26.7199835521626	9.73625772669261\\
26.7195828468071	9.73597458195188\\
26.719182122965	9.73569142798084\\
26.7187813806346	9.73540826477887\\
26.7183806198143	9.73512509234537\\
26.7179798405022	9.73484191067975\\
26.7175790426968	9.73455871978139\\
26.7171782263964	9.73427551964971\\
26.7167773915991	9.73399231028409\\
26.7163765383033	9.73370909168393\\
26.7159756665073	9.73342586384864\\
26.7155747762094	9.73314262677761\\
26.7151738674079	9.73285938047023\\
26.714772940101	9.73257612492591\\
26.7143719942871	9.73229286014405\\
26.7139710299644	9.73200958612402\\
26.7135700471313	9.73172630286525\\
26.713169045786	9.73144301036713\\
26.7127680259269	9.73115970862904\\
26.7123669875521	9.73087639765039\\
26.7119659306601	9.73059307743058\\
26.711564855249	9.73030974796901\\
26.7111637613172	9.73002640926506\\
26.710762648863	9.72974306131814\\
26.7103615178847	9.72945970412765\\
26.7099603683805	9.72917633769297\\
26.7095592003487	9.72889296201352\\
26.7091580137877	9.72860957708868\\
26.7087568086957	9.72832618291785\\
26.708355585071	9.72804277950042\\
26.7079543429118	9.72775936683581\\
26.7075530822166	9.72747594492339\\
26.7071518029835	9.72719251376257\\
26.7067505052109	9.72690907335275\\
26.706349188897	9.72662562369332\\
26.7059478540401	9.72634216478367\\
26.7055465006386	9.72605869662321\\
26.7051451286906	9.72577521921133\\
26.7047437381946	9.72549173254741\\
26.7043423291487	9.72520823663088\\
26.7039409015512	9.72492473146111\\
26.7035394554005	9.7246412170375\\
26.7031379906949	9.72435769335945\\
26.7027365074325	9.72407416042636\\
26.7023350056117	9.72379061823762\\
26.7019334852308	9.72350706679262\\
26.7015319462881	9.72322350609077\\
26.7011303887818	9.72293993613145\\
26.7007288127103	9.72265635691407\\
26.7003272180717	9.72237276843801\\
26.6999256048645	9.72208917070268\\
26.6995239730868	9.72180556370747\\
26.699122322737	9.72152194745178\\
26.6987206538134	9.72123832193499\\
26.6983189663141	9.72095468715651\\
26.6979172602376	9.72067104311573\\
26.6975155355821	9.72038738981204\\
26.6971137923458	9.72010372724484\\
26.6967120305271	9.71982005541353\\
26.6963102501242	9.7195363743175\\
26.6959084511355	9.71925268395614\\
26.6955066335591	9.71896898432886\\
26.6951047973934	9.71868527543503\\
26.6947029426367	9.71840155727407\\
26.6943010692872	9.71811782984536\\
26.6938991773432	9.7178340931483\\
26.693497266803	9.71755034718228\\
26.6930953376648	9.71726659194669\\
26.692693389927	9.71698282744094\\
26.6922914235879	9.71669905366441\\
26.6918894386456	9.7164152706165\\
26.6914874350986	9.71613147829661\\
26.691085412945	9.71584767670412\\
26.6906833721831	9.71556386583844\\
26.6902813128113	9.71528004569895\\
26.6898792348277	9.71499621628505\\
26.6894771382307	9.71471237759614\\
26.6890750230186	9.7144285296316\\
26.6886728891896	9.71414467239083\\
26.688270736742	9.71386080587323\\
26.6878685656741	9.71357693007819\\
26.6874663759841	9.71329304500509\\
26.6870641676703	9.71300915065335\\
26.6866619407311	9.71272524702234\\
26.6862596951646	9.71244133411147\\
26.6858574309692	9.71215741192012\\
26.6854551481431	9.71187348044769\\
26.6850528466846	9.71158953969357\\
26.684650526592	9.71130558965716\\
26.6842481878635	9.71102163033785\\
26.6838458304975	9.71073766173502\\
26.6834434544921	9.71045368384808\\
26.6830410598457	9.71016969667642\\
26.6826386465566	9.70988570021943\\
26.6822362146229	9.7096016944765\\
26.6818337640431	9.70931767944703\\
26.6814312948153	9.70903365513041\\
26.6810288069378	9.70874962152602\\
26.680626300409	9.70846557863328\\
26.680223775227	9.70818152645155\\
26.6798212313901	9.70789746498025\\
26.6794186688967	9.70761339421875\\
26.6790160877449	9.70732931416646\\
26.6786134879331	9.70704522482276\\
26.6782108694596	9.70676112618706\\
26.6778082323225	9.70647701825873\\
26.6774055765202	9.70619290103717\\
26.6770029020509	9.70590877452177\\
26.6766002089129	9.70562463871194\\
26.6761974971045	9.70534049360705\\
26.6757947666239	9.70505633920649\\
26.6753920174695	9.70477217550968\\
26.6749892496394	9.70448800251598\\
26.674586463132	9.7042038202248\\
26.6741836579454	9.70391962863552\\
26.6737808340781	9.70363542774754\\
26.6733779915282	9.70335121756025\\
26.672975130294	9.70306699807304\\
26.6725722503738	9.70278276928531\\
26.6721693517659	9.70249853119643\\
26.6717664344684	9.70221428380582\\
26.6713634984798	9.70193002711284\\
26.6709605437981	9.70164576111691\\
26.6705575704218	9.7013614858174\\
26.6701545783491	9.70107720121371\\
26.6697515675782	9.70079290730523\\
26.6693485381074	9.70050860409135\\
26.668945489935	9.70022429157146\\
26.6685424230592	9.69993996974495\\
26.6681393374784	9.69965563861122\\
26.6677362331906	9.69937129816965\\
26.6673331101943	9.69908694841964\\
26.6669299684877	9.69880258936057\\
26.6665268080691	9.69851822099183\\
26.6661236289366	9.69823384331282\\
26.6657204310887	9.69794945632293\\
26.6653172145235	9.69766506002154\\
26.6649139792393	9.69738065440804\\
26.6645107252343	9.69709623948184\\
26.6641074525069	9.69681181524231\\
26.6637041610552	9.69652738168885\\
26.6633008508777	9.69624293882084\\
26.6628975219724	9.69595848663769\\
26.6624941743377	9.69567402513876\\
26.6620908079718	9.69538955432347\\
26.661687422873	9.69510507419119\\
26.6612840190396	9.69482058474131\\
26.6608805964697	9.69453608597324\\
26.6604771551618	9.69425157788634\\
26.660073695114	9.69396706048002\\
26.6596702163245	9.69368253375366\\
26.6592667187917	9.69339799770666\\
26.6588632025138	9.6931134523384\\
26.6584596674891	9.69282889764827\\
26.6580561137158	9.69254433363566\\
26.6576525411922	9.69225976029996\\
26.6572489499165	9.69197517764057\\
26.656845339887	9.69169058565686\\
26.656441711102	9.69140598434822\\
26.6560380635597	9.69112137371406\\
26.6556343972583	9.69083675375374\\
26.6552307121962	9.69055212446667\\
26.6548270083715	9.69026748585224\\
26.6544232857826	9.68998283790982\\
26.6540195444276	9.68969818063881\\
26.6536157843049	9.68941351403861\\
26.6532120054127	9.68912883810859\\
26.6528082077492	9.68884415284814\\
26.6524043913128	9.68855945825666\\
26.6520005561016	9.68827475433353\\
26.651596702114	9.68799004107813\\
26.6511928293481	9.68770531848987\\
26.6507889378022	9.68742058656813\\
26.6503850274746	9.68713584531228\\
26.6499810983636	9.68685109472173\\
26.6495771504673	9.68656633479586\\
26.6491731837841	9.68628156553406\\
26.6487691983121	9.68599678693572\\
26.6483651940497	9.68571199900021\\
26.6479611709951	9.68542720172694\\
26.6475571291465	9.68514239511529\\
26.6471530685022	9.68485757916465\\
26.6467489890605	9.6845727538744\\
26.6463448908196	9.68428791924392\\
26.6459407737777	9.68400307527263\\
26.6455366379331	9.68371822195988\\
26.6451324832841	9.68343335930508\\
26.6447283098289	9.68314848730761\\
26.6443241175657	9.68286360596686\\
26.6439199064928	9.68257871528222\\
26.6435156766085	9.68229381525306\\
26.6431114279109	9.68200890587879\\
26.6427071603984	9.68172398715877\\
26.6423028740692	9.68143905909242\\
26.6418985689216	9.6811541216791\\
26.6414942449537	9.68086917491821\\
26.6410899021639	9.68058421880913\\
26.6406855405504	9.68029925335126\\
26.6402811601114	9.68001427854396\\
26.6398767608451	9.67972929438665\\
26.6394723427499	9.67944430087869\\
26.639067905824	9.67915929801947\\
26.6386634500656	9.67887428580839\\
26.638258975473	9.67858926424483\\
26.6378544820443	9.67830423332816\\
26.637449969778	9.67801919305779\\
26.6370454386721	9.6777341434331\\
26.636640888725	9.67744908445347\\
26.6362363199348	9.67716401611829\\
26.6358317323	9.67687893842694\\
26.6354271258185	9.67659385137881\\
26.6350225004889	9.67630875497329\\
26.6346178563092	9.67602364920976\\
26.6342131932777	9.67573853408761\\
26.6338085113926	9.67545340960621\\
26.6334038106523	9.67516827576497\\
26.6329990910549	9.67488313256326\\
26.6325943525986	9.67459798000047\\
26.6321895952819	9.67431281807599\\
26.6317848191027	9.67402764678919\\
26.6313800240595	9.67374246613947\\
26.6309752101505	9.67345727612621\\
26.6305703773738	9.6731720767488\\
26.6301655257278	9.67288686800662\\
26.6297606552106	9.67260164989905\\
26.6293557658206	9.67231642242548\\
26.6289508575559	9.6720311855853\\
26.6285459304149	9.67174593937788\\
26.6281409843956	9.67146068380262\\
26.6277360194965	9.6711754188589\\
26.6273310357156	9.67089014454611\\
26.6269260330513	9.67060486086362\\
26.6265210115018	9.67031956781083\\
26.6261159710654	9.67003426538711\\
26.6257109117402	9.66974895359185\\
26.6253058335245	9.66946363242445\\
26.6249007364166	9.66917830188427\\
26.6244956204146	9.6688929619707\\
26.6240904855169	9.66860761268314\\
26.6236853317217	9.66832225402096\\
26.6232801590271	9.66803688598354\\
26.6228749674315	9.66775150857027\\
26.622469756933	9.66746612178054\\
26.62206452753	9.66718072561373\\
26.6216592792206	9.66689532006922\\
26.621254012003	9.6666099051464\\
26.6208487258756	9.66632448084464\\
26.6204434208366	9.66603904716334\\
26.6200380968841	9.66575360410187\\
26.6196327540164	9.66546815165963\\
26.6192273922318	9.66518268983599\\
26.6188220115285	9.66489721863034\\
26.6184166119047	9.66461173804205\\
26.6180111933587	9.66432624807052\\
26.6176057558886	9.66404074871513\\
26.6172002994928	9.66375523997525\\
26.6167948241694	9.66346972185028\\
26.6163893299167	9.66318419433959\\
26.615983816733	9.66289865744257\\
26.6155782846164	9.6626131111586\\
26.6151727335651	9.66232755548707\\
26.6147671635775	9.66204199042736\\
26.6143615746518	9.66175641597884\\
26.6139559667861	9.66147083214091\\
26.6135503399787	9.66118523891294\\
26.6131446942279	9.66089963629432\\
26.6127390295318	9.66061402428442\\
26.6123333458888	9.66032840288264\\
26.6119276432969	9.66004277208835\\
26.6115219217546	9.65975713190095\\
26.6111161812599	9.65947148231979\\
26.6107104218112	9.65918582334428\\
26.6103046434066	9.6589001549738\\
26.6098988460443	9.65861447720772\\
26.6094930297227	9.65832879004542\\
26.60908719444	9.65804309348629\\
26.6086813401942	9.65775738752972\\
26.6082754669838	9.65747167217507\\
26.6078695748069	9.65718594742174\\
26.6074636636618	9.65690021326911\\
26.6070577335466	9.65661446971655\\
26.6066517844596	9.65632871676345\\
26.606245816399	9.6560429544092\\
26.6058398293631	9.65575718265316\\
26.6054338233501	9.65547140149473\\
26.6050277983582	9.65518561093328\\
26.6046217543856	9.65489981096819\\
26.6042156914306	9.65461400159885\\
26.6038096094913	9.65432818282465\\
26.6034035085661	9.65404235464494\\
26.602997388653	9.65375651705913\\
26.6025912497505	9.65347067006659\\
26.6021850918566	9.6531848136667\\
26.6017789149696	9.65289894785884\\
26.6013727190878	9.6526130726424\\
26.6009665042093	9.65232718801675\\
26.6005602703323	9.65204129398127\\
26.6001540174552	9.65175539053535\\
26.5997477455761	9.65146947767836\\
26.5993414546932	9.65118355540969\\
26.5989351448048	9.65089762372871\\
26.598528815909	9.65061168263481\\
26.5981224680042	9.65032573212737\\
26.5977161010885	9.65003977220577\\
26.5973097151602	9.64975380286937\\
26.5969033102174	9.64946782411758\\
26.5964968862584	9.64918183594976\\
26.5960904432814	9.6488958383653\\
26.5956839812846	9.64860983136358\\
26.5952775002663	9.64832381494397\\
26.5948710002247	9.64803778910586\\
26.5944644811579	9.64775175384862\\
26.5940579430643	9.64746570917164\\
26.593651385942	9.6471796550743\\
26.5932448097892	9.64689359155597\\
26.5928382146042	9.64660751861603\\
26.5924316003852	9.64632143625386\\
26.5920249671304	9.64603534446885\\
26.591618314838	9.64574924326037\\
26.5912116435062	9.64546313262781\\
26.5908049531333	9.64517701257052\\
26.5903982437174	9.64489088308792\\
26.5899915152569	9.64460474417935\\
26.5895847677498	9.64431859584422\\
26.5891780011944	9.64403243808189\\
26.588771215589	9.64374627089175\\
26.5883644109318	9.64346009427316\\
26.5879575872209	9.64317390822553\\
26.5875507444545	9.64288771274821\\
26.587143882631	9.64260150784059\\
26.5867370017485	9.64231529350205\\
26.5863301018052	9.64202906973197\\
26.5859231827993	9.64174283652972\\
26.5855162447291	9.64145659389468\\
26.5851092875928	9.64117034182624\\
26.5847023113885	9.64088408032376\\
26.5842953161145	9.64059780938663\\
26.583888301769	9.64031152901423\\
26.5834812683503	9.64002523920593\\
26.5830742158564	9.63973893996112\\
26.5826671442857	9.63945263127916\\
26.5822600536364	9.63916631315944\\
26.5818529439066	9.63887998560134\\
26.5814458150946	9.63859364860423\\
26.5810386671986	9.63830730216749\\
26.5806315002168	9.6380209462905\\
26.5802243141474	9.63773458097263\\
26.5798171089886	9.63744820621327\\
26.5794098847386	9.63716182201179\\
26.5790026413957	9.63687542836756\\
26.578595378958	9.63658902527997\\
26.5781880974238	9.6363026127484\\
26.5777807967912	9.63601619077221\\
26.5773734770585	9.63572975935079\\
26.5769661382239	9.63544331848351\\
26.5765587802856	9.63515686816976\\
26.5761514032418	9.6348704084089\\
26.5757440070907	9.63458393920032\\
26.5753365918305	9.63429746054338\\
26.5749291574594	9.63401097243748\\
26.5745217039757	9.63372447488198\\
26.5741142313775	9.63343796787626\\
26.573706739663	9.6331514514197\\
26.5732992288305	9.63286492551168\\
26.5728916988781	9.63257839015156\\
26.5724841498041	9.63229184533873\\
26.5720765816067	9.63200529107257\\
26.571668994284	9.63171872735245\\
26.5712613878343	9.63143215417773\\
26.5708537622558	9.63114557154782\\
26.5704461175467	9.63085897946207\\
26.5700384537052	9.63057237791986\\
26.5696307707294	9.63028576692058\\
26.5692230686177	9.62999914646359\\
26.5688153473682	9.62971251654827\\
26.568407606979	9.629425877174\\
26.5679998474485	9.62913922834015\\
26.5675920687748	9.6288525700461\\
26.567184270956	9.62856590229122\\
26.5667764539905	9.6282792250749\\
26.5663686178765	9.6279925383965\\
26.565960762612	9.6277058422554\\
26.5655528881954	9.62741913665097\\
26.5651449946247	9.6271324215826\\
26.5647370818983	9.62684569704965\\
26.5643291500143	9.6265589630515\\
26.563921198971	9.62627221958753\\
26.5635132287664	9.62598546665711\\
26.5631052393989	9.62569870425962\\
26.5626972308666	9.62541193239443\\
26.5622892031678	9.62512515106091\\
26.5618811563005	9.62483836025845\\
26.5614730902631	9.62455155998641\\
26.5610650050537	9.62426475024417\\
26.5606569006705	9.6239779310311\\
26.5602487771117	9.62369110234659\\
26.5598406343756	9.62340426419\\
26.5594324724602	9.6231174165607\\
26.5590242913639	9.62283055945808\\
26.5586160910847	9.6225436928815\\
26.558207871621	9.62225681683035\\
26.5577996329709	9.62196993130399\\
26.5573913751325	9.6216830363018\\
26.5569830981042	9.62139613182315\\
26.556574801884	9.62110921786741\\
26.5561664864702	9.62082229443397\\
26.555758151861	9.6205353615222\\
26.5553497980546	9.62024841913146\\
26.5549414250491	9.61996146726113\\
26.5545330328428	9.61967450591059\\
26.5541246214339	9.61938753507921\\
26.5537161908205	9.61910055476636\\
26.5533077410008	9.61881356497141\\
26.5528992719731	9.61852656569375\\
26.5524907837356	9.61823955693274\\
26.5520822762863	9.61795253868775\\
26.5516737496236	9.61766551095817\\
26.5512652037456	9.61737847374336\\
26.5508566386504	9.61709142704269\\
26.5504480543364	9.61680437085554\\
26.5500394508017	9.61651730518128\\
26.5496308280444	9.61623023001929\\
26.5492221860628	9.61594314536893\\
26.5488135248551	9.61565605122959\\
26.5484048444194	9.61536894760062\\
26.547996144754	9.61508183448141\\
26.547587425857	9.61479471187133\\
26.5471786877266	9.61450757976975\\
26.546769930361	9.61422043817605\\
26.5463611537585	9.61393328708958\\
26.5459523579171	9.61364612650974\\
26.5455435428351	9.61335895643588\\
26.5451347085107	9.61307177686739\\
26.544725854942	9.61278458780363\\
26.5443169821273	9.61249738924398\\
26.5439080900647	9.6122101811878\\
26.5434991787524	9.61192296363448\\
26.5430902481886	9.61163573658338\\
26.5426812983716	9.61134850003387\\
26.5422723292994	9.61106125398533\\
26.5418633409703	9.61077399843712\\
26.5414543333824	9.61048673338863\\
26.541045306534	9.61019945883921\\
26.5406362604232	9.60991217478825\\
26.5402271950482	9.60962488123511\\
26.5398181104072	9.60933757817916\\
26.5394090064984	9.60905026561978\\
26.53899988332	9.60876294355634\\
26.5385907408701	9.60847561198821\\
26.538181579147	9.60818827091476\\
26.5377723981487	9.60790092033536\\
26.5373631978736	9.60761356024937\\
26.5369539783198	9.60732619065619\\
26.5365447394855	9.60703881155517\\
26.5361354813688	9.60675142294568\\
26.5357262039679	9.6064640248271\\
26.5353169072811	9.60617661719879\\
26.5349075913064	9.60588920006013\\
26.5344982560422	9.60560177341049\\
26.5340889014865	9.60531433724923\\
26.5336795276376	9.60502689157574\\
26.5332701344936	9.60473943638937\\
26.5328607220527	9.6044519716895\\
26.5324512903131	9.60416449747551\\
26.532041839273	9.60387701374675\\
26.5316323689305	9.6035895205026\\
26.5312228792839	9.60330201774243\\
26.5308133703312	9.60301450546561\\
26.5304038420708	9.60272698367151\\
26.5299942945007	9.6024394523595\\
26.5295847276192	9.60215191152895\\
26.5291751414244	9.60186436117923\\
26.5287655359146	9.6015768013097\\
26.5283559110878	9.60128923191975\\
26.5279462669423	9.60100165300873\\
26.5275366034762	9.60071406457602\\
26.5271269206877	9.60042646662099\\
26.526717218575	9.60013885914301\\
26.5263074971363	9.59985124214144\\
26.5258977563698	9.59956361561566\\
26.5254879962736	9.59927597956503\\
26.5250782168458	9.59898833398892\\
26.5246684180848	9.59870067888671\\
26.5242585999886	9.59841301425776\\
26.5238487625554	9.59812534010144\\
26.5234389057835	9.59783765641712\\
26.5230290296709	9.59754996320417\\
26.5226191342159	9.59726226046195\\
26.5222092194166	9.59697454818984\\
26.5217992852712	9.59668682638721\\
26.5213893317779	9.59639909505341\\
26.5209793589348	9.59611135418783\\
26.5205693667402	9.59582360378983\\
26.5201593551921	9.59553584385878\\
26.5197493242888	9.59524807439404\\
26.5193392740285	9.59496029539499\\
26.5189292044093	9.594672506861\\
26.5185191154293	9.59438470879142\\
26.5181090070868	9.59409690118563\\
26.51769887938	9.593809084043\\
26.5172887323069	9.5935212573629\\
26.5168785658658	9.59323342114469\\
26.5164683800549	9.59294557538773\\
26.5160581748723	9.59265772009141\\
26.5156479503161	9.59236985525509\\
26.5152377063846	9.59208198087813\\
26.5148274430759	9.5917940969599\\
26.5144171603882	9.59150620349977\\
26.5140068583197	9.5912183004971\\
26.5135965368685	9.59093038795127\\
26.5131861960328	9.59064246586165\\
26.5127758358108	9.59035453422759\\
26.5123654562006	9.59006659304846\\
26.5119550572004	9.58977864232364\\
26.5115446388084	9.58949068205249\\
26.5111342010227	9.58920271223438\\
26.5107237438415	9.58891473286867\\
26.510313267263	9.58862674395473\\
26.5099027712853	9.58833874549193\\
26.5094922559066	9.58805073747963\\
26.5090817211251	9.5877627199172\\
26.5086711669389	9.58747469280402\\
26.5082605933463	9.58718665613943\\
26.5078500003453	9.58689860992282\\
26.5074393879341	9.58661055415354\\
26.5070287561109	9.58632248883097\\
26.5066181048739	9.58603441395447\\
26.5062074342212	9.5857463295234\\
26.505796744151	9.58545823553714\\
26.5053860346615	9.58517013199504\\
26.5049753057507	9.58488201889648\\
26.504564557417	9.58459389624082\\
26.5041537896584	9.58430576402743\\
26.5037430024731	9.58401762225567\\
26.5033321958592	9.5837294709249\\
26.502921369815	9.5834413100345\\
26.5025105243386	9.58315313958383\\
26.5020996594281	9.58286495957226\\
26.5016887750818	9.58257676999914\\
26.5012778712977	9.58228857086386\\
26.5008669480741	9.58200036216576\\
26.500456005409	9.58171214390422\\
26.5000450433008	9.58142391607861\\
26.4996340617474	9.58113567868828\\
26.4992230607471	9.5808474317326\\
26.498812040298	9.58055917521094\\
26.4984010003984	9.58027090912267\\
26.4979899410463	9.57998263346714\\
26.4975788622399	9.57969434824373\\
26.4971677639774	9.57940605345179\\
26.4967566462569	9.5791177490907\\
26.4963455090766	9.57882943515981\\
26.4959343524346	9.57854111165849\\
26.4955231763291	9.57825277858612\\
26.4951119807583	9.57796443594204\\
26.4947007657204	9.57767608372563\\
26.4942895312134	9.57738772193625\\
26.4938782772355	9.57709935057326\\
26.4934670037849	9.57681096963603\\
26.4930557108598	9.57652257912392\\
26.4926443984582	9.5762341790363\\
26.4922330665784	9.57594576937254\\
26.4918217152185	9.57565735013199\\
26.4914103443767	9.57536892131401\\
26.4909989540511	9.57508048291798\\
26.4905875442399	9.57479203494326\\
26.4901761149412	9.5745035773892\\
26.4897646661531	9.57421511025519\\
26.4893531978739	9.57392663354056\\
26.4889417101017	9.57363814724471\\
26.4885302028346	9.57334965136697\\
26.4881186760709	9.57306114590673\\
26.4877071298085	9.57277263086333\\
26.4872955640458	9.57248410623616\\
26.4868839787808	9.57219557202456\\
26.4864723740117	9.5719070282279\\
26.4860607497367	9.57161847484555\\
26.4856491059539	9.57132991187687\\
26.4852374426614	9.57104133932121\\
26.4848257598574	9.57075275717795\\
26.4844140575401	9.57046416544645\\
26.4840023357076	9.57017556412607\\
26.483590594358	9.56988695321617\\
26.4831788334895	9.56959833271612\\
26.4827670531003	9.56930970262527\\
26.4823552531885	9.569021062943\\
26.4819434337523	9.56873241366866\\
26.4815315947897	9.56844375480161\\
26.481119736299	9.56815508634122\\
26.4807078582783	9.56786640828685\\
26.4802959607258	9.56757772063787\\
26.4798840436395	9.56728902339363\\
26.4794721070176	9.56700031655349\\
26.4790601508584	9.56671160011683\\
26.4786481751599	9.56642287408299\\
26.4782361799202	9.56613413845135\\
26.4778241651376	9.56584539322126\\
26.4774121308102	9.56555663839209\\
26.4770000769361	9.5652678739632\\
26.4765880035134	9.56497909993395\\
26.4761759105403	9.56469031630369\\
26.475763798015	9.56440152307181\\
26.4753516659356	9.56411272023765\\
26.4749395143002	9.56382390780057\\
26.474527343107	9.56353508575994\\
26.4741151523542	9.56324625411512\\
26.4737029420398	9.56295741286547\\
26.473290712162	9.56266856201035\\
26.472878462719	9.56237970154912\\
26.4724661937089	9.56209083148115\\
26.4720539051298	9.56180195180579\\
26.4716415969799	9.5615130625224\\
26.4712292692573	9.56122416363035\\
26.4708169219602	9.56093525512901\\
26.4704045550867	9.56064633701771\\
26.469992168635	9.56035740929584\\
26.4695797626032	9.56006847196274\\
26.4691673369894	9.55977952501779\\
26.4687548917917	9.55949056846033\\
26.4683424270084	9.55920160228973\\
26.4679299426375	9.55891262650536\\
26.4675174386772	9.55862364110656\\
26.4671049151257	9.55833464609271\\
26.466692371981	9.55804564146316\\
26.4662798092413	9.55775662721727\\
26.4658672269048	9.5574676033544\\
26.4654546249696	9.55717856987391\\
26.4650420034337	9.55688952677516\\
26.4646293622955	9.55660047405752\\
26.4642167015529	9.55631141172034\\
26.4638040212042	9.55602233976297\\
26.4633913212475	9.55573325818479\\
26.4629786016808	9.55544416698515\\
26.4625658625024	9.5551550661634\\
26.4621531037104	9.55486595571892\\
26.4617403253029	9.55457683565105\\
26.4613275272781	9.55428770595917\\
26.460914709634	9.55399856664261\\
26.4605018723689	9.55370941770076\\
26.4600890154808	9.55342025913296\\
26.459676138968	9.55313109093857\\
26.4592632428284	9.55284191311696\\
26.4588503270603	9.55255272566748\\
26.4584373916618	9.55226352858949\\
26.458024436631	9.55197432188235\\
26.4576114619661	9.55168510554542\\
26.4571984676652	9.55139587957806\\
26.4567854537264	9.55110664397962\\
26.4563724201479	9.55081739874947\\
26.4559593669278	9.55052814388696\\
26.4555462940641	9.55023887939145\\
26.4551332015552	9.5499496052623\\
26.454720089399	9.54966032149887\\
26.4543069575937	9.54937102810052\\
26.4538938061375	9.5490817250666\\
26.4534806350285	9.54879241239647\\
26.4530674442648	9.5485030900895\\
26.4526542338445	9.54821375814503\\
26.4522410037657	9.54792441656243\\
26.4518277540267	9.54763506534105\\
26.4514144846255	9.54734570448026\\
26.4510011955603	9.54705633397941\\
26.4505878868291	9.54676695383785\\
26.4501745584302	9.54647756405495\\
26.4497612103616	9.54618816463006\\
26.4493478426214	9.54589875556255\\
26.4489344552079	9.54560933685175\\
26.4485210481191	9.54531990849705\\
26.4481076213531	9.54503047049779\\
26.4476941749081	9.54474102285332\\
26.4472807087822	9.54445156556302\\
26.4468672229735	9.54416209862623\\
26.4464537174802	9.5438726220423\\
26.4460401923004	9.54358313581061\\
26.4456266474322	9.5432936399305\\
26.4452130828737	9.54300413440133\\
26.444799498623	9.54271461922246\\
26.4443858946784	9.54242509439325\\
26.4439722710378	9.54213555991304\\
26.4435586276995	9.54184601578121\\
26.4431449646615	9.5415564619971\\
26.4427312819221	9.54126689856007\\
26.4423175794792	9.54097732546947\\
26.441903857331	9.54068774272468\\
26.4414901154757	9.54039815032503\\
26.4410763539113	9.54010854826988\\
26.4406625726361	9.5398189365586\\
26.440248771648	9.53952931519054\\
26.4398349509453	9.53923968416505\\
26.439421110526	9.53895004348149\\
26.4390072503883	9.53866039313922\\
26.4385933705304	9.5383707331376\\
26.4381794709502	9.53808106347596\\
26.4377655516459	9.53779138415369\\
26.4373516126158	9.53750169517011\\
26.4369376538578	9.53721199652461\\
26.4365236753701	9.53692228821652\\
26.4361096771508	9.53663257024521\\
26.435695659198	9.53634284261003\\
26.4352816215099	9.53605310531034\\
26.4348675640846	9.53576335834549\\
26.4344534869202	9.53547360171483\\
26.4340393900148	9.53518383541772\\
26.4336252733665	9.53489405945352\\
26.4332111369734	9.53460427382159\\
26.4327969808337	9.53431447852127\\
26.4323828049455	9.53402467355192\\
26.4319686093069	9.53373485891289\\
26.431554393916	9.53344503460355\\
26.4311401587709	9.53315520062324\\
26.4307259038697	9.53286535697132\\
26.4303116292107	9.53257550364715\\
26.4298973347917	9.53228564065007\\
26.4294830206111	9.53199576797945\\
26.4290686866669	9.53170588563464\\
26.4286543329572	9.53141599361498\\
26.4282399594801	9.53112609191984\\
26.4278255662338	9.53083618054858\\
26.4274111532163	9.53054625950053\\
26.4269967204258	9.53025632877506\\
26.4265822678604	9.52996638837153\\
26.4261677955182	9.52967643828928\\
26.4257533033973	9.52938647852767\\
26.4253387914958	9.52909650908606\\
26.4249242598118	9.52880652996379\\
26.4245097083435	9.52851654116022\\
26.424095137089	9.52822654267471\\
26.4236805460463	9.5279365345066\\
26.4232659352136	9.52764651665526\\
26.422851304589	9.52735648912003\\
26.4224366541706	9.52706645190026\\
26.4220219839564	9.52677640499532\\
26.4216072939448	9.52648634840455\\
26.4211925841336	9.52619628212731\\
26.4207778545211	9.52590620616294\\
26.4203631051053	9.52561612051081\\
26.4199483358844	9.52532602517027\\
26.4195335468565	9.52503592014067\\
26.4191187380196	9.52474580542136\\
26.4187039093719	9.52445568101169\\
26.4182890609115	9.52416554691101\\
26.4178741926365	9.52387540311869\\
26.417459304545	9.52358524963407\\
26.4170443966351	9.52329508645651\\
26.4166294689049	9.52300491358534\\
26.4162145213526	9.52271473101994\\
26.4157995539761	9.52242453875965\\
26.4153845667737	9.52213433680383\\
26.4149695597435	9.52184412515182\\
26.4145545328835	9.52155390380297\\
26.4141394861918	9.52126367275665\\
26.4137244196666	9.52097343201219\\
26.4133093333059	9.52068318156896\\
26.412894227108	9.52039292142631\\
26.4124791010708	9.52010265158358\\
26.4120639551924	9.51981237204013\\
26.411648789471	9.51952208279531\\
26.4112336039047	9.51923178384847\\
26.4108183984916	9.51894147519897\\
26.4104031732298	9.51865115684615\\
26.4099879281173	9.51836082878936\\
26.4095726631524	9.51807049102796\\
26.409157378333	9.5177801435613\\
26.4087420736573	9.51748978638873\\
26.4083267491234	9.5171994195096\\
26.4079114047294	9.51690904292326\\
26.4074960404734	9.51661865662907\\
26.4070806563534	9.51632826062636\\
26.4066652523677	9.5160378549145\\
26.4062498285143	9.51574743949284\\
26.4058343847912	9.51545701436072\\
26.4054189211966	9.51516657951749\\
26.4050034377287	9.51487613496252\\
26.4045879343854	9.51458568069513\\
26.4041724111649	9.5142952167147\\
26.4037568680653	9.51400474302057\\
26.4033413050846	9.51371425961208\\
26.4029257222211	9.51342376648859\\
26.4025101194727	9.51313326364945\\
26.4020944968376	9.51284275109401\\
26.4016788543139	9.51255222882162\\
26.4012631918996	9.51226169683162\\
26.4008475095929	9.51197115512338\\
26.4004318073918	9.51168060369623\\
26.4000160852946	9.51139004254953\\
26.3996003432991	9.51109947168263\\
26.3991845814036	9.51080889109487\\
26.3987687996062	9.51051830078561\\
26.3983529979049	9.5102277007542\\
26.3979371762978	9.50993709099999\\
26.397521334783	9.50964647152231\\
26.3971054733586	9.50935584232054\\
26.3966895920228	9.50906520339401\\
26.3962736907736	9.50877455474207\\
26.395857769609	9.50848389636407\\
26.3954418285273	9.50819322825936\\
26.3950258675264	9.5079025504273\\
26.3946098866045	9.50761186286722\\
26.3941938857597	9.50732116557848\\
26.39377786499	9.50703045856043\\
26.3933618242935	9.50673974181241\\
26.3929457636685	9.50644901533378\\
26.3925296831128	9.50615827912388\\
26.3921135826247	9.50586753318206\\
26.3916974622021	9.50557677750767\\
26.3912813218433	9.50528601210006\\
26.3908651615462	9.50499523695857\\
26.3904489813091	9.50470445208256\\
26.3900327811299	9.50441365747137\\
26.3896165610067	9.50412285312436\\
26.3892003209377	9.50383203904086\\
26.388784060921	9.50354121522023\\
26.3883677809546	9.50325038166181\\
26.3879514810365	9.50295953836496\\
26.387535161165	9.50266868532902\\
26.387118821338	9.50237782255333\\
26.3867024615538	9.50208695003726\\
26.3862860818102	9.50179606778013\\
26.3858696821056	9.50150517578131\\
26.3854532624378	9.50121427404013\\
26.3850368228051	9.50092336255595\\
26.3846203632054	9.50063244132812\\
26.384203883637	9.50034151035597\\
26.3837873840978	9.50005056963887\\
26.383370864586	9.49975961917614\\
26.3829543250996	9.49946865896715\\
26.3825377656368	9.49917768901124\\
26.3821211861955	9.49888670930776\\
26.381704586774	9.49859571985604\\
26.3812879673702	9.49830472065545\\
26.3808713279823	9.49801371170533\\
26.3804546686083	9.49772269300501\\
26.3800379892463	9.49743166455386\\
26.3796212898945	9.49714062635121\\
26.3792045705508	9.49684957839642\\
26.3787878312134	9.49655852068883\\
26.3783710718804	9.49626745322778\\
26.3779542925497	9.49597637601262\\
26.3775374932196	9.4956852890427\\
26.3771206738881	9.49539419231737\\
26.3767038345533	9.49510308583596\\
26.3762869752132	9.49481196959784\\
26.3758700958659	9.49452084360233\\
26.3754531965096	9.4942297078488\\
26.3750362771422	9.49393856233658\\
26.374619337762	9.49364740706501\\
26.3742023783668	9.49335624203346\\
26.3737853989549	9.49306506724125\\
26.3733683995243	9.49277388268774\\
26.3729513800731	9.49248268837228\\
26.3725343405993	9.4921914842942\\
26.3721172811011	9.49190027045286\\
26.3717002015765	9.49160904684759\\
26.3712831020236	9.49131781347774\\
26.3708659824404	9.49102657034267\\
26.3704488428251	9.49073531744171\\
26.3700316831757	9.49044405477421\\
26.3696145034903	9.49015278233951\\
26.369197303767	9.48986150013697\\
26.3687800840038	9.48957020816591\\
26.3683628441989	9.4892789064257\\
26.3679455843502	9.48898759491566\\
26.3675283044559	9.48869627363516\\
26.3671110045141	9.48840494258353\\
26.3666936845227	9.48811360176011\\
26.36627634448	9.48782225116426\\
26.3658589843839	9.48753089079531\\
26.3654416042325	9.48723952065261\\
26.365024204024	9.48694814073551\\
26.3646067837563	9.48665675104335\\
26.3641893434276	9.48636535157546\\
26.3637718830359	9.48607394233121\\
26.3633544025793	9.48578252330993\\
26.3629369020559	9.48549109451096\\
26.3625193814637	9.48519965593365\\
26.3621018408008	9.48490820757734\\
26.3616842800653	9.48461674944138\\
26.3612666992552	9.4843252815251\\
26.3608490983687	9.48403380382786\\
26.3604314774037	9.483742316349\\
26.3600138363584	9.48345081908786\\
26.3595961752308	9.48315931204378\\
26.359178494019	9.48286779521611\\
26.3587607927211	9.48257626860419\\
26.3583430713351	9.48228473220736\\
26.3579253298591	9.48199318602497\\
26.3575075682912	9.48170163005636\\
26.3570897866294	9.48141006430087\\
26.3566719848718	9.48111848875785\\
26.3562541630164	9.48082690342663\\
26.3558363210614	9.48053530830657\\
26.3554184590048	9.480243703397\\
26.3550005768447	9.47995208869727\\
26.3545826745791	9.47966046420671\\
26.354164752206	9.47936882992468\\
26.3537468097237	9.47907718585052\\
26.35332884713	9.47878553198356\\
26.3529108644232	9.47849386832315\\
26.3524928616012	9.47820219486863\\
26.3520748386621	9.47791051161934\\
26.351656795604	9.47761881857463\\
26.3512387324249	9.47732711573383\\
26.350820649123	9.4770354030963\\
26.3504025456962	9.47674368066137\\
26.3499844221426	9.47645194842838\\
26.3495662784604	9.47616020639668\\
26.3491481146475	9.4758684545656\\
26.348729930702	9.47557669293449\\
26.348311726622	9.47528492150269\\
26.3478935024055	9.47499314026955\\
26.3474752580506	9.4747013492344\\
26.3470569935554	9.47440954839658\\
26.3466387089179	9.47411773775544\\
26.3462204041362	9.47382591731032\\
26.3458020792084	9.47353408706055\\
26.3453837341324	9.47324224700549\\
26.3449653689064	9.47295039714446\\
26.3445469835284	9.47265853747682\\
26.3441285779965	9.4723666680019\\
26.3437101523087	9.47207478871904\\
26.3432917064631	9.47178289962759\\
26.3428732404578	9.47149100072689\\
26.3424547542908	9.47119909201626\\
26.3420362479601	9.47090717349507\\
26.3416177214639	9.47061524516264\\
26.3411991748001	9.47032330701832\\
26.3407806079669	9.47003135906145\\
26.3403620209622	9.46973940129137\\
26.3399434137842	9.46944743370741\\
26.3395247864309	9.46915545630892\\
26.3391061389004	9.46886346909525\\
26.3386874711907	9.46857147206572\\
26.3382687832998	9.46827946521968\\
26.3378500752259	9.46798744855647\\
26.3374313469669	9.46769542207542\\
26.337012598521	9.46740338577589\\
26.3365938298862	9.4671113396572\\
26.3361750410604	9.46681928371871\\
26.3357562320419	9.46652721795973\\
26.3353374028286	9.46623514237963\\
26.3349185534186	9.46594305697774\\
26.33449968381	9.46565096175339\\
26.3340807940007	9.46535885670592\\
26.3336618839889	9.46506674183468\\
26.3332429537726	9.46477461713901\\
26.3328240033499	9.46448248261823\\
26.3324050327187	9.4641903382717\\
26.3319860418772	9.46389818409875\\
26.3315670308234	9.46360602009872\\
26.3311479995554	9.46331384627095\\
26.3307289480711	9.46302166261478\\
26.3303098763687	9.46272946912954\\
26.3298907844462	9.46243726581458\\
26.3294716723017	9.46214505266924\\
26.3290525399331	9.46185282969284\\
26.3286333873386	9.46156059688473\\
26.3282142145162	9.46126835424426\\
26.3277950214639	9.46097610177075\\
26.3273758081798	9.46068383946355\\
26.326956574662	9.46039156732199\\
26.3265373209084	9.46009928534542\\
26.3261180469171	9.45980699353316\\
26.3256987526863	9.45951469188456\\
26.3252794382138	9.45922238039896\\
26.3248601034978	9.4589300590757\\
26.3244407485363	9.4586377279141\\
26.3240213733274	9.45834538691352\\
26.3236019778691	9.45805303607328\\
26.3231825621594	9.45776067539273\\
26.3227631261964	9.4574683048712\\
26.3223436699781	9.45717592450803\\
26.3219241935026	9.45688353430255\\
26.3215046967679	9.45659113425411\\
26.3210851797721	9.45629872436205\\
26.3206656425132	9.45600630462569\\
26.3202460849893	9.45571387504438\\
26.3198265071983	9.45542143561745\\
26.3194069091384	9.45512898634425\\
26.3189872908075	9.4548365272241\\
26.3185676522037	9.45454405825634\\
26.3181479933251	9.45425157944032\\
26.3177283141697	9.45395909077536\\
26.3173086147356	9.45366659226081\\
26.3168888950207	9.453374083896\\
26.3164691550231	9.45308156568026\\
26.3160493947409	9.45278903761294\\
26.3156296141721	9.45249649969337\\
26.3152098133147	9.45220395192089\\
26.3147899921668	9.45191139429483\\
26.3143701507264	9.45161882681453\\
26.3139502889915	9.45132624947933\\
26.3135304069603	9.45103366228856\\
26.3131105046306	9.45074106524155\\
26.3126905820006	9.45044845833765\\
26.3122706390684	9.45015584157618\\
26.3118506758318	9.4498632149565\\
26.3114306922891	9.44957057847792\\
26.3110106884381	9.44927793213979\\
26.310590664277	9.44898527594144\\
26.3101706198037	9.44869260988221\\
26.3097505550164	9.44839993396143\\
26.3093304699131	9.44810724817844\\
26.3089103644917	9.44781455253257\\
26.3084902387503	9.44752184702317\\
26.308070092687	9.44722913164956\\
26.3076499262998	9.44693640641107\\
26.3072297395867	9.44664367130705\\
26.3068095325458	9.44635092633684\\
26.306389305175	9.44605817149975\\
26.3059690574725	9.44576540679513\\
26.3055487894362	9.44547263222232\\
26.3051285010642	9.44517984778065\\
26.3047081923545	9.44488705346945\\
26.3042878633052	9.44459424928806\\
26.3038675139143	9.44430143523581\\
26.3034471441797	9.44400861131204\\
26.3030267540997	9.44371577751608\\
26.302606343672	9.44342293384726\\
26.3021859128949	9.44313008030493\\
26.3017654617663	9.44283721688841\\
26.3013449902843	9.44254434359704\\
26.3009244984469	9.44225146043014\\
26.3005039862521	9.44195856738707\\
26.3000834536979	9.44166566446714\\
26.2996629007825	9.4413727516697\\
26.2992423275037	9.44107982899408\\
26.2988217338597	9.44078689643961\\
26.2984011198484	9.44049395400563\\
26.2979804854679	9.44020100169146\\
26.2975598307162	9.43990803949644\\
26.2971391555914	9.43961506741991\\
26.2967184600915	9.43932208546121\\
26.2962977442144	9.43902909361965\\
26.2958770079583	9.43873609189458\\
26.2954562513211	9.43844308028533\\
26.2950354743009	9.43815005879122\\
26.2946146768957	9.43785702741161\\
26.2941938591035	9.43756398614581\\
26.2937730209223	9.43727093499316\\
26.2933521623503	9.436977873953\\
26.2929312833853	9.43668480302465\\
26.2925103840254	9.43639172220745\\
26.2920894642687	9.43609863150074\\
26.2916685241131	9.43580553090383\\
26.2912475635568	9.43551242041608\\
26.2908265825976	9.4352193000368\\
26.2904055812337	9.43492616976534\\
26.289984559463	9.43463302960102\\
26.2895635172836	9.43433987954318\\
26.2891424546935	9.43404671959114\\
26.2887213716908	9.43375354974425\\
26.2883002682733	9.43346037000182\\
26.2878791444393	9.4331671803632\\
26.2874580001866	9.43287398082772\\
26.2870368355133	9.43258077139471\\
26.2866156504174	9.43228755206349\\
26.2861944448969	9.43199432283341\\
26.28577321895	9.43170108370379\\
26.2853519725744	9.43140783467397\\
26.2849307057684	9.43111457574327\\
26.2845094185299	9.43082130691103\\
26.284088110857	9.43052802817658\\
26.2836667827475	9.43023473953924\\
26.2832454341997	9.42994144099836\\
26.2828240652114	9.42964813255327\\
26.2824026757807	9.42935481420329\\
26.2819812659057	9.42906148594775\\
26.2815598355843	9.42876814778598\\
26.2811383848145	9.42847479971732\\
26.2807169135944	9.42818144174111\\
26.2802954219219	9.42788807385665\\
26.2798739097952	9.4275946960633\\
26.2794523772121	9.42730130836038\\
26.2790308241708	9.42700791074722\\
26.2786092506692	9.42671450322314\\
26.2781876567054	9.42642108578749\\
26.2777660422774	9.42612765843959\\
26.2773444073831	9.42583422117877\\
26.2769227520206	9.42554077400436\\
26.2765010761879	9.4252473169157\\
26.276079379883	9.4249538499121\\
26.2756576631039	9.42466037299291\\
26.2752359258487	9.42436688615745\\
26.2748141681154	9.42407338940505\\
26.2743923899019	9.42377988273504\\
26.2739705912062	9.42348636614675\\
26.2735487720265	9.42319283963952\\
26.2731269323606	9.42289930321266\\
26.2727050722067	9.42260575686551\\
26.2722831915626	9.42231220059741\\
26.2718612904265	9.42201863440767\\
26.2714393687963	9.42172505829562\\
26.2710174266701	9.42143147226061\\
26.2705954640458	9.42113787630196\\
26.2701734809215	9.42084427041898\\
26.2697514772951	9.42055065461103\\
26.2693294531647	9.42025702887741\\
26.2689074085283	9.41996339321747\\
26.2684853433839	9.41966974763053\\
26.2680632577294	9.41937609211592\\
26.267641151563	9.41908242667297\\
26.2672190248826	9.41878875130101\\
26.2667968776861	9.41849506599936\\
26.2663747099718	9.41820137076735\\
26.2659525217374	9.41790766560432\\
26.265530312981	9.41761395050959\\
26.2651080837007	9.41732022548249\\
26.2646858338945	9.41702649052235\\
26.2642635635603	9.41673274562849\\
26.2638412726961	9.41643899080025\\
26.2634189613	9.41614522603695\\
26.2629966293699	9.41585145133792\\
26.2625742769039	9.41555766670249\\
26.2621519038999	9.41526387212998\\
26.261729510356	9.41497006761973\\
26.2613070962702	9.41467625317105\\
26.2608846616405	9.41438242878329\\
26.2604622064648	9.41408859445576\\
26.2600397307411	9.4137947501878\\
26.2596172344676	9.41350089597872\\
26.2591947176421	9.41320703182787\\
26.2587721802627	9.41291315773456\\
26.2583496223274	9.41261927369812\\
26.2579270438341	9.41232537971789\\
26.2575044447809	9.41203147579318\\
26.2570818251658	9.41173756192332\\
26.2566591849867	9.41144363810765\\
26.2562365242418	9.41114970434548\\
26.2558138429288	9.41085576063615\\
26.255391141046	9.41056180697898\\
26.2549684185912	9.4102678433733\\
26.2545456755625	9.40997386981843\\
26.2541229119578	9.4096798863137\\
26.2537001277752	9.40938589285845\\
26.2532773230126	9.40909188945198\\
26.2528544976681	9.40879787609364\\
26.2524316517396	9.40850385278274\\
26.2520087852252	9.40820981951861\\
26.2515858981228	9.40791577630059\\
26.2511629904304	9.40762172312798\\
26.2507400621461	9.40732766000013\\
26.2503171132677	9.40703358691636\\
26.2498941437934	9.40673950387599\\
26.2494711537211	9.40644541087834\\
26.2490481430488	9.40615130792276\\
26.2486251117745	9.40585719500854\\
26.2482020598962	9.40556307213504\\
26.2477789874118	9.40526893930157\\
26.2473558943195	9.40497479650745\\
26.2469327806171	9.40468064375201\\
26.2465096463026	9.40438648103459\\
26.2460864913741	9.40409230835449\\
26.2456633158296	9.40379812571105\\
26.2452401196669	9.40350393310359\\
26.2448169028842	9.40320973053144\\
26.2443936654794	9.40291551799392\\
26.2439704074506	9.40262129549036\\
26.2435471287956	9.40232706302008\\
26.2431238295125	9.4020328205824\\
26.2427005095992	9.40173856817666\\
26.2422771690538	9.40144430580218\\
26.2418538078743	9.40115003345827\\
26.2414304260586	9.40085575114427\\
26.2410070236048	9.4005614588595\\
26.2405836005107	9.40026715660328\\
26.2401601567745	9.39997284437494\\
26.239736692394	9.3996785221738\\
26.2393132073673	9.39938418999919\\
26.2388897016924	9.39908984785043\\
26.2384661753672	9.39879549572684\\
26.2380426283898	9.39850113362775\\
26.2376190607581	9.39820676155248\\
26.2371954724701	9.39791237950036\\
26.2367718635237	9.3976179874707\\
26.2363482339171	9.39732358546284\\
26.2359245836481	9.3970291734761\\
26.2355009127148	9.3967347515098\\
26.2350772211151	9.39644031956326\\
26.234653508847	9.3961458776358\\
26.2342297759085	9.39585142572676\\
26.2338060222976	9.39555696383545\\
26.2333822480122	9.3952624919612\\
26.2329584530504	9.39496801010333\\
26.2325346374101	9.39467351826116\\
26.2321108010894	9.39437901643401\\
26.2316869440861	9.39408450462122\\
26.2312630663983	9.39378998282209\\
26.2308391680239	9.39349545103596\\
26.230415248961	9.39320090926214\\
26.2299913092075	9.39290635749997\\
26.2295673487613	9.39261179574876\\
26.2291433676206	9.39231722400783\\
26.2287193657832	9.3920226422765\\
26.2282953432472	9.39172805055411\\
26.2278713000104	9.39143344883997\\
26.227447236071	9.3911388371334\\
26.2270231514268	9.39084421543373\\
26.2265990460759	9.39054958374028\\
26.2261749200162	9.39025494205236\\
26.2257507732457	9.38996029036931\\
26.2253266057623	9.38966562869044\\
26.2249024175642	9.38937095701507\\
26.2244782086491	9.38907627534253\\
26.2240539790152	9.38878158367214\\
26.2236297286604	9.38848688200322\\
26.2232054575826	9.3881921703351\\
26.2227811657798	9.38789744866708\\
26.2223568532501	9.3876027169985\\
26.2219325199914	9.38730797532868\\
26.2215081660016	9.38701322365693\\
26.2210837912787	9.38671846198258\\
26.2206593958208	9.38642369030495\\
26.2202349796258	9.38612890862336\\
26.2198105426916	9.38583411693713\\
26.2193860850162	9.38553931524558\\
26.2189616065977	9.38524450354804\\
26.2185371074339	9.38494968184382\\
26.2181125875228	9.38465485013224\\
26.2176880468625	9.38436000841263\\
26.2172634854509	9.3840651566843\\
26.2168389032859	9.38377029494658\\
26.2164143003656	9.38347542319879\\
26.2159896766879	9.38318054144024\\
26.2155650322507	9.38288564967026\\
26.2151403670521	9.38259074788817\\
26.21471568109	9.38229583609328\\
26.2142909743624	9.38200091428492\\
26.2138662468672	9.38170598246241\\
26.2134414986025	9.38141104062506\\
26.2130167295661	9.3811160887722\\
26.2125919397561	9.38082112690315\\
26.2121671291705	9.38052615501723\\
26.2117422978071	9.38023117311375\\
26.211317445664	9.37993618119203\\
26.2108925727391	9.3796411792514\\
26.2104676790304	9.37934616729118\\
26.2100427645359	9.37905114531068\\
26.2096178292535	9.37875611330922\\
26.2091928731812	9.37846107128613\\
26.208767896317	9.37816601924071\\
26.2083428986587	9.3778709571723\\
26.2079178802045	9.37757588508021\\
26.2074928409522	9.37728080296375\\
26.2070677808999	9.37698571082226\\
26.2066427000454	9.37669060865504\\
26.2062175983868	9.37639549646141\\
26.205792475922	9.3761003742407\\
26.205367332649	9.37580524199222\\
26.2049421685657	9.37551009971529\\
26.20451698367	9.37521494740923\\
26.2040917779601	9.37491978507335\\
26.2036665514338	9.37462461270699\\
26.2032413040891	9.37432943030944\\
26.2028160359239	9.37403423788004\\
26.2023907469362	9.3737390354181\\
26.201965437124	9.37344382292294\\
26.2015401064852	9.37314860039388\\
26.2011147550178	9.37285336783022\\
26.2006893827198	9.3725581252313\\
26.2002639895891	9.37226287259644\\
26.1998385756236	9.37196760992494\\
26.1994131408214	9.37167233721612\\
26.1989876851803	9.3713770544693\\
26.1985622086984	9.37108176168381\\
26.1981367113736	9.37078645885896\\
26.1977111932039	9.37049114599405\\
26.1972856541872	9.37019582308842\\
26.1968600943214	9.36990049014138\\
26.1964345136046	9.36960514715225\\
26.1960089120347	9.36930979412034\\
26.1955832896096	9.36901443104498\\
26.1951576463273	9.36871905792546\\
26.1947319821858	9.36842367476113\\
26.194306297183	9.36812828155128\\
26.1938805913168	9.36783287829524\\
26.1934548645853	9.36753746499233\\
26.1930291169864	9.36724204164186\\
26.192603348518	9.36694660824315\\
26.1921775591781	9.36665116479551\\
26.1917517489646	9.36635571129825\\
26.1913259178755	9.36606024775071\\
26.1909000659087	9.36576477415219\\
26.1904741930623	9.36546929050201\\
26.1900482993341	9.36517379679948\\
26.1896223847221	9.36487829304393\\
26.1891964492242	9.36458277923466\\
26.1887704928385	9.364287255371\\
26.1883445155628	9.36399172145225\\
26.1879185173951	9.36369617747774\\
26.1874924983334	9.36340062344678\\
26.1870664583756	9.36310505935868\\
26.1866403975197	9.36280948521277\\
26.1862143157635	9.36251390100835\\
26.1857882131051	9.36221830674475\\
26.1853620895425	9.36192270242127\\
26.1849359450734	9.36162708803723\\
26.184509779696	9.36133146359196\\
26.1840835934082	9.36103582908475\\
26.1836573862078	9.36074018451493\\
26.1832311580929	9.36044452988182\\
26.1828049090614	9.36014886518473\\
26.1823786391112	9.35985319042297\\
26.1819523482403	9.35955750559585\\
26.1815260364466	9.3592618107027\\
26.1810997037282	9.35896610574282\\
26.1806733500828	9.35867039071554\\
26.1802469755085	9.35837466562016\\
26.1798205800033	9.358078930456\\
26.179394163565	9.35778318522238\\
26.1789677261916	9.35748742991861\\
26.1785412678811	9.35719166454399\\
26.1781147886313	9.35689588909786\\
26.1776882884403	9.35660010357952\\
26.177261767306	9.35630430798828\\
26.1768352252263	9.35600850232346\\
26.1764086621992	9.35571268658437\\
26.1759820782226	9.35541686077034\\
26.1755554732944	9.35512102488066\\
26.1751288474127	9.35482517891465\\
26.1747022005752	9.35452932287164\\
26.1742755327801	9.35423345675092\\
26.1738488440251	9.35393758055182\\
26.1734221343084	9.35364169427365\\
26.1729954036277	9.35334579791571\\
26.1725686519811	9.35304989147734\\
26.1721418793664	9.35275397495783\\
26.1717150857817	9.3524580483565\\
26.1712882712248	9.35216211167266\\
26.1708614356937	9.35186616490564\\
26.1704345791863	9.35157020805472\\
26.1700077017007	9.35127424111925\\
26.1695808032346	9.35097826409852\\
26.1691538837861	9.35068227699184\\
26.1687269433531	9.35038627979854\\
26.1682999819334	9.35009027251792\\
26.1678729995252	9.34979425514929\\
26.1674459961262	9.34949822769197\\
26.1670189717345	9.34920219014528\\
26.166591926348	9.34890614250851\\
26.1661648599645	9.34861008478099\\
26.1657377725821	9.34831401696202\\
26.1653106641987	9.34801793905092\\
26.1648835348121	9.34772185104701\\
26.1644563844204	9.34742575294958\\
26.1640292130215	9.34712964475796\\
26.1636020206133	9.34683352647146\\
26.1631748071938	9.34653739808938\\
26.1627475727608	9.34624125961104\\
26.1623203173123	9.34594511103575\\
26.1618930408463	9.34564895236283\\
26.1614657433606	9.34535278359157\\
26.1610384248532	9.34505660472131\\
26.1606110853221	9.34476041575133\\
26.1601837247652	9.34446421668097\\
26.1597563431803	9.34416800750953\\
26.1593289405655	9.34387178823631\\
26.1589015169186	9.34357555886063\\
26.1584740722376	9.34327931938181\\
26.1580466065205	9.34298306979914\\
26.1576191197651	9.34268681011195\\
26.1571916119693	9.34239054031954\\
26.1567640831312	9.34209426042123\\
26.1563365332486	9.34179797041632\\
26.1559089623195	9.34150167030413\\
26.1554813703417	9.34120536008396\\
26.1550537573133	9.34090903975512\\
26.1546261232321	9.34061270931694\\
26.1541984680961	9.34031636876871\\
26.1537707919032	9.34002001810974\\
26.1533430946513	9.33972365733936\\
26.1529153763384	9.33942728645686\\
26.1524876369624	9.33913090546155\\
26.1520598765211	9.33883451435275\\
26.1516320950126	9.33853811312977\\
26.1512042924347	9.33824170179191\\
26.1507764687854	9.33794528033849\\
26.1503486240626	9.33764884876881\\
26.1499207582643	9.33735240708219\\
26.1494928713882	9.33705595527793\\
26.1490649634325	9.33675949335534\\
26.1486370343949	9.33646302131374\\
26.1482090842735	9.33616653915242\\
26.1477811130661	9.33587004687071\\
26.1473531207706	9.33557354446791\\
26.1469251073851	9.33527703194332\\
26.1464970729073	9.33498050929627\\
26.1460690173353	9.33468397652605\\
26.1456409406669	9.33438743363197\\
26.1452128429001	9.33409088061335\\
26.1447847240328	9.33379431746949\\
26.1443565840628	9.3334977441997\\
26.1439284229883	9.33320116080329\\
26.1435002408069	9.33290456727958\\
26.1430720375167	9.33260796362785\\
26.1426438131156	9.33231134984743\\
26.1422155676015	9.33201472593763\\
26.1417873009723	9.33171809189774\\
26.141359013226	9.33142144772708\\
26.1409307043604	9.33112479342496\\
26.1405023743734	9.33082812899069\\
26.1400740232631	9.33053145442357\\
26.1396456510273	9.33023476972291\\
26.1392172576639	9.32993807488801\\
26.1387888431708	9.3296413699182\\
26.138360407546	9.32934465481276\\
26.1379319507873	9.32904792957102\\
26.1375034728928	9.32875119419228\\
26.1370749738602	9.32845444867584\\
26.1366464536875	9.32815769302101\\
26.1362179123727	9.32786092722711\\
26.1357893499136	9.32756415129343\\
26.1353607663081	9.32726736521929\\
26.1349321615542	9.32697056900398\\
26.1345035356498	9.32667376264683\\
26.1340748885928	9.32637694614713\\
26.1336462203811	9.3260801195042\\
26.1332175310126	9.32578328271733\\
26.1327888204852	9.32548643578584\\
26.1323600887969	9.32518957870903\\
26.1319313359455	9.32489271148621\\
26.131502561929	9.32459583411668\\
26.1310737667452	9.32429894659976\\
26.1306449503921	9.32400204893474\\
26.1302161128676	9.32370514112094\\
26.1297872541697	9.32340822315765\\
26.1293583742961	9.3231112950442\\
26.1289294732448	9.32281435677987\\
26.1285005510138	9.32251740836398\\
26.1280716076008	9.32222044979584\\
26.127642643004	9.32192348107475\\
26.127213657221	9.32162650220001\\
26.12678465025	9.32132951317093\\
26.1263556220886	9.32103251398682\\
26.125926572735	9.32073550464698\\
26.1254975021869	9.32043848515072\\
26.1250684104423	9.32014145549734\\
26.1246392974991	9.31984441568615\\
26.1242101633552	9.31954736571645\\
26.1237810080085	9.31925030558756\\
26.1233518314569	9.31895323529876\\
26.1229226336983	9.31865615484938\\
26.1224934147306	9.31835906423871\\
26.1220641745517	9.31806196346605\\
26.1216349131595	9.31776485253072\\
26.121205630552	9.31746773143202\\
26.120776326727	9.31717060016925\\
26.1203470016824	9.31687345874172\\
26.1199176554161	9.31657630714873\\
26.1194882879261	9.31627914538958\\
26.1190588992101	9.31598197346359\\
26.1186294892663	9.31568479137005\\
26.1182000580923	9.31538759910827\\
26.1177706056862	9.31509039667755\\
26.1173411320458	9.3147931840772\\
26.1169116371691	9.31449596130653\\
26.1164821210539	9.31419872836482\\
26.1160525836981	9.3139014852514\\
26.1156230250996	9.31360423196556\\
26.1151934452564	9.3133069685066\\
26.1147638441664	9.31300969487384\\
26.1143342218273	9.31271241106657\\
26.1139045782372	9.3124151170841\\
26.1134749133939	9.31211781292573\\
26.1130452272953	9.31182049859076\\
26.1126155199393	9.3115231740785\\
26.1121857913239	9.31122583938825\\
26.1117560414468	9.31092849451931\\
26.1113262703061	9.31063113947099\\
26.1108964778995	9.3103337742426\\
26.1104666642251	9.31003639883342\\
26.1100368292807	9.30973901324277\\
26.1096069730641	9.30944161746995\\
26.1091770955733	9.30914421151425\\
26.1087471968062	9.308846795375\\
26.1083172767607	9.30854936905148\\
26.1078873354346	9.308251932543\\
26.1074573728259	9.30795448584886\\
26.1070273889325	9.30765702896836\\
26.1065973837522	9.30735956190081\\
26.1061673572829	9.30706208464551\\
26.1057373095225	9.30676459720176\\
26.105307240469	9.30646709956886\\
26.1048771501201	9.30616959174612\\
26.1044470384739	9.30587207373283\\
26.1040169055282	9.3055745455283\\
26.1035867512808	9.30527700713183\\
26.1031565757297	9.30497945854273\\
26.1027263788728	9.30468189976028\\
26.1022961607079	9.30438433078381\\
26.101865921233	9.3040867516126\\
26.1014356604459	9.30378916224595\\
26.1010053783445	9.30349156268318\\
26.1005750749267	9.30319395292357\\
26.1001447501904	9.30289633296644\\
26.0997144041334	9.30259870281108\\
26.0992840367538	9.3023010624568\\
26.0988536480493	9.30200341190289\\
26.0984232380178	9.30170575114865\\
26.0979928066573	9.3014080801934\\
26.0975623539656	9.30111039903642\\
26.0971318799406	9.30081270767702\\
26.0967013845801	9.3005150061145\\
26.0962708678822	9.30021729434815\\
26.0958403298445	9.29991957237729\\
26.0954097704652	9.29962184020121\\
26.0949791897419	9.29932409781921\\
26.0945485876727	9.29902634523059\\
26.0941179642553	9.29872858243465\\
26.0936873194877	9.29843080943069\\
26.0932566533678	9.29813302621802\\
26.0928259658934	9.29783523279593\\
26.0923952570624	9.29753742916371\\
26.0919645268727	9.29723961532068\\
26.0915337753223	9.29694179126613\\
26.0911030024088	9.29664395699936\\
26.0906722081304	9.29634611251968\\
26.0902413924847	9.29604825782637\\
26.0898105554698	9.29575039291874\\
26.0893796970834	9.29545251779609\\
26.0889488173235	9.29515463245772\\
26.0885179161879	9.29485673690292\\
26.0880869936746	9.294558831131\\
26.0876560497814	9.29426091514126\\
26.0872250845062	9.293962988933\\
26.0867940978468	9.2936650525055\\
26.0863630898011	9.29336710585808\\
26.0859320603671	9.29306914899003\\
26.0855010095426	9.29277118190065\\
26.0850699373254	9.29247320458925\\
26.0846388437135	9.29217521705511\\
26.0842077287047	9.29187721929753\\
26.0837765922969	9.29157921131582\\
26.0833454344879	9.29128119310927\\
26.0829142552757	9.29098316467719\\
26.0824830546582	9.29068512601886\\
26.0820518326331	9.29038707713359\\
26.0816205891984	9.29008901802068\\
26.081189324352	9.28979094867943\\
26.0807580380917	9.28949286910912\\
26.0803267304153	9.28919477930906\\
26.0798954013209	9.28889667927856\\
26.0794640508061	9.28859856901689\\
26.079032678869	9.28830044852337\\
26.0786012855074	9.2880023177973\\
26.0781698707191	9.28770417683796\\
26.077738434502	9.28740602564465\\
26.0773069768541	9.28710786421668\\
26.0768754977731	9.28680969255334\\
26.0764439972569	9.28651151065393\\
26.0760124753035	9.28621331851775\\
26.0755809319106	9.28591511614408\\
26.0751493670762	9.28561690353223\\
26.0747177807981	9.28531868068151\\
26.0742861730742	9.28502044759119\\
26.0738545439024	9.28472220426058\\
26.0734228932804	9.28442395068898\\
26.0729912212063	9.28412568687569\\
26.0725595276778	9.28382741281999\\
26.0721278126928	9.28352912852119\\
26.0716960762493	9.28323083397858\\
26.071264318345	9.28293252919146\\
26.0708325389778	9.28263421415913\\
26.0704007381456	9.28233588888088\\
26.0699689158463	9.28203755335601\\
26.0695370720777	9.28173920758381\\
26.0691052068377	9.28144085156358\\
26.0686733201241	9.28114248529462\\
26.0682414119348	9.28084410877621\\
26.0678094822678	9.28054572200767\\
26.0673775311208	9.28024732498828\\
26.0669455584916	9.27994891771734\\
26.0665135643783	9.27965050019414\\
26.0660815487786	9.27935207241799\\
26.0656495116904	9.27905363438816\\
26.0652174531115	9.27875518610398\\
26.0647853730399	9.27845672756471\\
26.0643532714734	9.27815825876967\\
26.0639211484098	9.27785977971815\\
26.063489003847	9.27756129040943\\
26.0630568377829	9.27726279084282\\
26.0626246502153	9.27696428101762\\
26.062192441142	9.27666576093311\\
26.0617602105611	9.27636723058859\\
26.0613279584702	9.27606868998336\\
26.0608956848673	9.27577013911671\\
26.0604633897502	9.27547157798793\\
26.0600310731168	9.27517300659632\\
26.059598734965	9.27487442494118\\
26.0591663752925	9.2745758330218\\
26.0587339940973	9.27427723083747\\
26.0583015913772	9.27397861838748\\
26.0578691671301	9.27367999567114\\
26.0574367213538	9.27338136268773\\
26.0570042540462	9.27308271943655\\
26.0565717652051	9.27278406591689\\
26.0561392548284	9.27248540212806\\
26.055706722914	9.27218672806933\\
26.0552741694596	9.27188804374\\
26.0548415944633	9.27158934913938\\
26.0544089979227	9.27129064426674\\
26.0539763798358	9.27099192912139\\
26.0535437402004	9.27069320370262\\
26.0531110790144	9.27039446800972\\
26.0526783962756	9.27009572204198\\
26.0522456919819	9.26979696579871\\
26.0518129661311	9.26949819927918\\
26.0513802187211	9.2691994224827\\
26.0509474497497	9.26890063540855\\
26.0505146592148	9.26860183805604\\
26.0500818471143	9.26830303042445\\
26.0496490134459	9.26800421251307\\
26.0492161582076	9.2677053843212\\
26.0487832813971	9.26740654584814\\
26.0483503830124	9.26710769709316\\
26.0479174630513	9.26680883805558\\
26.0474845215116	9.26650996873467\\
26.0470515583913	9.26621108912973\\
26.046618573688	9.26591219924005\\
26.0461855673997	9.26561329906493\\
26.0457525395242	9.26531438860366\\
26.0453194900595	9.26501546785552\\
26.0448864190032	9.26471653681982\\
26.0444533263533	9.26441759549584\\
26.0440202121076	9.26411864388287\\
26.043587076264	9.26381968198021\\
26.0431539188203	9.26352070978715\\
26.0427207397744	9.26322172730298\\
26.042287539124	9.26292273452699\\
26.0418543168671	9.26262373145847\\
26.0414210730015	9.26232471809672\\
26.040987807525	9.26202569444102\\
26.0405545204355	9.26172666049066\\
26.0401212117308	9.26142761624495\\
26.0396878814088	9.26112856170316\\
26.0392545294673	9.2608294968646\\
26.0388211559041	9.26053042172854\\
26.0383877607171	9.26023133629428\\
26.0379543439042	9.25993224056112\\
26.0375209054631	9.25963313452834\\
26.0370874453917	9.25933401819523\\
26.0366539636879	9.25903489156109\\
26.0362204603495	9.2587357546252\\
26.0357869353744	9.25843660738686\\
26.0353533887603	9.25813744984535\\
26.0349198205051	9.25783828199997\\
26.0344862306067	9.25753910385\\
26.0340526190629	9.25723991539474\\
26.0336189858715	9.25694071663347\\
26.0331853310304	9.25664150756549\\
26.0327516545374	9.25634228819009\\
26.0323179563904	9.25604305850655\\
26.0318842365871	9.25574381851417\\
26.0314504951255	9.25544456821223\\
26.0310167320033	9.25514530760002\\
26.0305829472185	9.25484603667684\\
26.0301491407687	9.25454675544198\\
26.029715312652	9.25424746389471\\
26.029281462866	9.25394816203434\\
26.0288475914087	9.25364884986015\\
26.0284136982778	9.25334952737143\\
26.0279797834713	9.25305019456747\\
26.0275458469869	9.25275085144755\\
26.0271118888225	9.25245149801098\\
26.0266779089759	9.25215213425703\\
26.0262439074449	9.251852760185\\
26.0258098842274	9.25155337579417\\
26.0253758393213	9.25125398108383\\
26.0249417727242	9.25095457605328\\
26.0245076844342	9.2506551607018\\
26.0240735744489	9.25035573502867\\
26.0236394427663	9.25005629903318\\
26.0232052893842	9.24975685271464\\
26.0227711143003	9.24945739607231\\
26.0223369175126	9.2491579291055\\
26.0219026990188	9.24885845181349\\
26.0214684588169	9.24855896419556\\
26.0210341969045	9.24825946625101\\
26.0205999132796	9.24795995797912\\
26.02016560794	9.24766043937918\\
26.0197312808835	9.24736091045048\\
26.0192969321079	9.24706137119231\\
26.0188625616111	9.24676182160395\\
26.0184281693908	9.24646226168468\\
26.017993755445	9.24616269143381\\
26.0175593197714	9.24586311085061\\
26.0171248623679	9.24556351993438\\
26.0166903832322	9.24526391868439\\
26.0162558823623	9.24496430709995\\
26.015821359756	9.24466468518032\\
26.015386815411	9.24436505292481\\
26.0149522493252	9.2440654103327\\
26.0145176614964	9.24376575740327\\
26.0140830519225	9.24346609413581\\
26.0136484206012	9.24316642052961\\
26.0132137675304	9.24286673658395\\
26.012779092708	9.24256704229813\\
26.0123443961316	9.24226733767142\\
26.0119096777993	9.24196762270311\\
26.0114749377087	9.2416678973925\\
26.0110401758577	9.24136816173886\\
26.0106053922441	9.24106841574149\\
26.0101705868658	9.24076865939966\\
26.0097357597206	9.24046889271267\\
26.0093009108062	9.2401691156798\\
26.0088660401206	9.23986932830033\\
26.0084311476614	9.23956953057356\\
26.0079962334267	9.23926972249877\\
26.0075612974141	9.23896990407523\\
26.0071263396214	9.23867007530225\\
26.0066913600466	9.2383702361791\\
26.0062563586874	9.23807038670508\\
26.0058213355416	9.23777052687945\\
26.0053862906071	9.23747065670152\\
26.0049512238817	9.23717077617056\\
26.0045161353632	9.23687088528586\\
26.0040810250493	9.23657098404671\\
26.003645892938	9.23627107245239\\
26.003210739027	9.23597115050218\\
26.0027755633142	9.23567121819538\\
26.0023403657974	9.23537127553126\\
26.0019051464743	9.23507132250911\\
26.0014699053429	9.23477135912821\\
26.0010346424008	9.23447138538785\\
26.000599357646	9.23417140128732\\
26.0001640510763	9.23387140682589\\
25.9997287226893	9.23357140200286\\
25.9992933724831	9.2332713868175\\
25.9988580004553	9.2329713612691\\
25.9984226066039	9.23267132535694\\
25.9979871909265	9.23237127908032\\
25.9975517534211	9.2320712224385\\
25.9971162940853	9.23177115543078\\
25.9966808129172	9.23147107805644\\
25.9962453099143	9.23117099031477\\
25.9958097850747	9.23087089220504\\
25.995374238396	9.23057078372654\\
25.9949386698761	9.23027066487856\\
25.9945030795127	9.22997053566037\\
25.9940674673038	9.22967039607127\\
25.9936318332471	9.22937024611052\\
25.9931961773404	9.22907008577743\\
25.9927604995815	9.22876991507126\\
25.9923247999683	9.22846973399131\\
25.9918890784985	9.22816954253686\\
25.9914533351699	9.22786934070718\\
25.9910175699804	9.22756912850157\\
25.9905817829277	9.2272689059193\\
25.9901459740098	9.22696867295966\\
25.9897101432242	9.22666842962192\\
25.989274290569	9.22636817590538\\
25.9888384160418	9.22606791180932\\
25.9884025196406	9.22576763733301\\
25.987966601363	9.22546735247574\\
25.9875306612069	9.22516705723679\\
25.9870946991701	9.22486675161545\\
25.9866587152504	9.22456643561099\\
25.9862227094457	9.2242661092227\\
25.9857866817536	9.22396577244987\\
25.985350632172	9.22366542529175\\
25.9849145606988	9.22336506774766\\
25.9844784673317	9.22306469981686\\
25.9840423520684	9.22276432149863\\
25.9836062149069	9.22246393279227\\
25.983170055845	9.22216353369704\\
25.9827338748803	9.22186312421223\\
25.9822976720107	9.22156270433713\\
25.9818614472341	9.22126227407101\\
25.9814252005482	9.22096183341316\\
25.9809889319508	9.22066138236285\\
25.9805526414397	9.22036092091936\\
25.9801163290127	9.22006044908199\\
25.9796799946677	9.21975996685\\
25.9792436384023	9.21945947422268\\
25.9788072602145	9.21915897119931\\
25.978370860102	9.21885845777918\\
25.9779344380625	9.21855793396155\\
25.977497994094	9.21825739974572\\
25.9770615281941	9.21795685513095\\
25.9766250403608	9.21765630011654\\
25.9761885305917	9.21735573470177\\
25.9757519988847	9.21705515888591\\
25.9753154452376	9.21675457266824\\
25.9748788696481	9.21645397604804\\
25.9744422721142	9.21615336902459\\
25.9740056526334	9.21585275159718\\
25.9735690112037	9.21555212376508\\
25.9731323478229	9.21525148552757\\
25.9726956624887	9.21495083688394\\
25.9722589551989	9.21465017783346\\
25.9718222259514	9.2143495083754\\
25.9713854747439	9.21404882850906\\
25.9709487015741	9.21374813823371\\
25.97051190644	9.21344743754863\\
25.9700750893392	9.21314672645309\\
25.9696382502697	9.21284600494639\\
25.969201389229	9.21254527302779\\
25.9687645062152	9.21224453069658\\
25.9683276012259	9.21194377795203\\
25.9678906742589	9.21164301479342\\
25.9674537253121	9.21134224122004\\
25.9670167543831	9.21104145723116\\
25.9665797614699	9.21074066282606\\
25.9661427465702	9.21043985800401\\
25.9657057096817	9.21013904276431\\
25.9652686508023	9.20983821710621\\
25.9648315699297	9.20953738102901\\
25.9643944670618	9.20923653453199\\
25.9639573421963	9.20893567761441\\
25.963520195331	9.20863481027556\\
25.9630830264637	9.20833393251471\\
25.9626458355922	9.20803304433115\\
25.9622086227142	9.20773214572415\\
25.9617713878276	9.207431236693\\
25.9613341309302	9.20713031723695\\
25.9608968520196	9.2068293873553\\
25.9604595510937	9.20652844704733\\
25.9600222281504	9.20622749631231\\
25.9595848831872	9.20592653514951\\
25.9591475162022	9.20562556355821\\
25.9587101271929	9.2053245815377\\
25.9582727161573	9.20502358908725\\
25.9578352830931	9.20472258620613\\
25.957397827998	9.20442157289363\\
25.9569603508699	9.20412054914902\\
25.9565228517066	9.20381951497157\\
25.9560853305057	9.20351847036057\\
25.9556477872652	9.20321741531529\\
25.9552102219827	9.202916349835\\
25.954772634656	9.20261527391899\\
25.9543350252831	9.20231418756653\\
25.9538973938615	9.2020130907769\\
25.9534597403891	9.20171198354937\\
25.9530220648636	9.20141086588322\\
25.952584367283	9.20110973777773\\
25.9521466476448	9.20080859923216\\
25.9517089059469	9.20050745024581\\
25.9512711421871	9.20020629081794\\
25.9508333563632	9.19990512094783\\
25.9503955484729	9.19960394063475\\
25.949957718514	9.19930274987799\\
25.9495198664842	9.19900154867681\\
25.9490819923814	9.1987003370305\\
25.9486440962034	9.19839911493832\\
25.9482061779478	9.19809788239956\\
25.9477682376125	9.19779663941349\\
25.9473302751953	9.19749538597938\\
25.9468922906939	9.19719412209651\\
25.946454284106	9.19689284776416\\
25.9460162554296	9.19659156298159\\
25.9455782046623	9.19629026774809\\
25.9451401318018	9.19598896206293\\
25.9447020368461	9.19568764592539\\
25.9442639197928	9.19538631933473\\
25.9438257806398	9.19508498229024\\
25.9433876193847	9.19478363479119\\
25.9429494360254	9.19448227683685\\
25.9425112305596	9.1941809084265\\
25.9420730029851	9.19387952955941\\
25.9416347532997	9.19357814023486\\
25.9411964815012	9.19327674045211\\
25.9407581875872	9.19297533021046\\
25.9403198715556	9.19267390950916\\
25.9398815334042	9.1923724783475\\
25.9394431731306	9.19207103672474\\
25.9390047907328	9.19176958464016\\
25.9385663862084	9.19146812209304\\
25.9381279595552	9.19116664908265\\
25.9376895107709	9.19086516560826\\
25.9372510398535	9.19056367166914\\
25.9368125468005	9.19026216726458\\
25.9363740316098	9.18996065239383\\
25.9359354942791	9.18965912705618\\
25.9354969348062	9.18935759125091\\
25.9350583531889	9.18905604497727\\
25.934619749425	9.18875448823455\\
25.9341811235121	9.18845292102202\\
25.933742475448	9.18815134333895\\
25.9333038052306	9.18784975518462\\
25.9328651128576	9.18754815655829\\
25.9324263983267	9.18724654745924\\
25.9319876616357	9.18694492788674\\
25.9315489027824	9.18664329784007\\
25.9311101217645	9.1863416573185\\
25.9306713185798	9.1860400063213\\
25.930232493226	9.18573834484774\\
25.9297936457009	9.1854366728971\\
25.9293547760023	9.18513499046864\\
25.9289158841279	9.18483329756164\\
25.9284769700755	9.18453159417537\\
25.9280380338429	9.18422988030911\\
25.9275990754277	9.18392815596212\\
25.9271600948278	9.18362642113368\\
25.9267210920409	9.18332467582305\\
25.9262820670648	9.18302292002952\\
25.9258430198972	9.18272115375235\\
25.9254039505359	9.18241937699081\\
25.9249648589787	9.18211758974418\\
25.9245257452232	9.18181579201173\\
25.9240866092673	9.18151398379272\\
25.9236474511087	9.18121216508643\\
25.9232082707452	9.18091033589213\\
25.9227690681744	9.18060849620909\\
25.9223298433943	9.18030664603659\\
25.9218905964025	9.18000478537389\\
25.9214513271967	9.17970291422026\\
25.9210120357748	9.17940103257497\\
25.9205727221344	9.1790991404373\\
25.9201333862734	9.17879723780652\\
25.9196940281895	9.17849532468189\\
25.9192546478805	9.17819340106269\\
25.918815245344	9.17789146694819\\
25.9183758205778	9.17758952233766\\
25.9179363735798	9.17728756723036\\
25.9174969043476	9.17698560162557\\
25.917057412879	9.17668362552256\\
25.9166178991717	9.1763816389206\\
25.9161783632235	9.17607964181895\\
25.9157388050322	9.1757776342169\\
25.9152992245954	9.1754756161137\\
25.914859621911	9.17517358750863\\
25.9144199969767	9.17487154840095\\
25.9139803497902	9.17456949878995\\
25.9135406803493	9.17426743867487\\
25.9131009886517	9.17396536805501\\
25.9126612746952	9.17366328692962\\
25.9122215384775	9.17336119529797\\
25.9117817799964	9.17305909315934\\
25.9113419992496	9.17275698051298\\
25.9109021962348	9.17245485735818\\
25.9104623709499	9.1721527236942\\
25.9100225233925	9.17185057952031\\
25.9095826535603	9.17154842483578\\
25.9091427614513	9.17124625963987\\
25.908702847063	9.17094408393186\\
25.9082629103932	9.17064189771101\\
25.9078229514396	9.17033970097659\\
25.9073829702001	9.17003749372788\\
25.9069429666724	9.16973527596414\\
25.9065029408541	9.16943304768463\\
25.906062892743	9.16913080888862\\
25.905622822337	9.1688285595754\\
25.9051827296336	9.16852629974421\\
25.9047426146307	9.16822402939433\\
25.904302477326	9.16792174852503\\
25.9038623177172	9.16761945713557\\
25.9034221358021	9.16731715522523\\
25.9029819315784	9.16701484279326\\
25.9025417050438	9.16671251983895\\
25.9021014561962	9.16641018636155\\
25.9016611850331	9.16610784236033\\
25.9012208915525	9.16580548783457\\
25.9007805757519	9.16550312278352\\
25.9003402376292	9.16520074720646\\
25.899899877182	9.16489836110264\\
25.8994594944082	9.16459596447135\\
25.8990190893054	9.16429355731184\\
25.8985786618713	9.16399113962339\\
25.8981382121038	9.16368871140525\\
25.8976977400006	9.16338627265671\\
25.8972572455593	9.16308382337701\\
25.8968167287778	9.16278136356543\\
25.8963761896537	9.16247889322124\\
25.8959356281848	9.1621764123437\\
25.8954950443689	9.16187392093208\\
25.8950544382036	9.16157141898565\\
25.8946138096866	9.16126890650367\\
25.8941731588158	9.1609663834854\\
25.8937324855889	9.16066384993012\\
25.8932917900035	9.16036130583708\\
25.8928510720574	9.16005875120557\\
25.8924103317484	9.15975618603483\\
25.8919695690742	9.15945361032414\\
25.8915287840324	9.15915102407276\\
25.8910879766209	9.15884842727996\\
25.8906471468374	9.158545819945\\
25.8902062946796	9.15824320206715\\
25.8897654201451	9.15794057364568\\
25.8893245232319	9.15763793467985\\
25.8888836039375	9.15733528516892\\
25.8884426622597	9.15703262511215\\
25.8880016981962	9.15672995450883\\
25.8875607117448	9.1564272733582\\
25.8871197029032	9.15612458165954\\
25.8866786716691	9.15582187941211\\
25.8862376180403	9.15551916661517\\
25.8857965420144	9.15521644326799\\
25.8853554435892	9.15491370936983\\
25.8849143227624	9.15461096491997\\
25.8844731795317	9.15430820991765\\
25.8840320138949	9.15400544436215\\
25.8835908258497	9.15370266825273\\
25.8831496153938	9.15339988158866\\
25.882708382525	9.1530970843692\\
25.8822671272409	9.1527942765936\\
25.8818258495392	9.15249145826115\\
25.8813845494178	9.1521886293711\\
25.8809432268743	9.15188578992271\\
25.8805018819065	9.15158293991525\\
25.880060514512	9.15128007934798\\
25.8796191246886	9.15097720822017\\
25.879177712434	9.15067432653108\\
25.8787362777459	9.15037143427997\\
25.878294820622	9.1500685314661\\
25.8778533410602	9.14976561808875\\
25.877411839058	9.14946269414717\\
25.8769703146132	9.14915975964062\\
25.8765287677236	9.14885681456837\\
25.8760871983867	9.14855385892969\\
25.8756456066005	9.14825089272383\\
25.8752039923625	9.14794791595005\\
25.8747623556705	9.14764492860763\\
25.8743206965222	9.14734193069582\\
25.8738790149153	9.14703892221388\\
25.8734373108476	9.14673590316108\\
25.8729955843167	9.14643287353668\\
25.8725538353204	9.14612983333995\\
25.8721120638564	9.14582678257014\\
25.8716702699223	9.14552372122651\\
25.871228453516	9.14522064930834\\
25.8707866146351	9.14491756681487\\
25.8703447532774	9.14461447374538\\
25.8699028694405	9.14431137009912\\
25.8694609631222	9.14400825587536\\
25.8690190343201	9.14370513107336\\
25.868577083032	9.14340199569238\\
25.8681351092557	9.14309884973168\\
25.8676931129887	9.14279569319053\\
25.8672510942289	9.14249252606817\\
25.8668090529739	9.14218934836389\\
25.8663669892215	9.14188616007693\\
25.8659249029693	9.14158296120656\\
25.8654827942151	9.14127975175204\\
25.8650406629565	9.14097653171263\\
25.8645985091913	9.14067330108758\\
25.8641563329172	9.14037005987618\\
25.863714134132	9.14006680807766\\
25.8632719128332	9.1397635456913\\
25.8628296690186	9.13946027271636\\
25.862387402686	9.13915698915209\\
25.861945113833	9.13885369499775\\
25.8615028024573	9.13855039025261\\
25.8610604685566	9.13824707491593\\
25.8606181121287	9.13794374898696\\
25.8601757331713	9.13764041246497\\
25.859733331682	9.13733706534921\\
25.8592909076586	9.13703370763896\\
25.8588484610987	9.13673033933346\\
25.8584059920001	9.13642696043197\\
25.8579635003605	9.13612357093376\\
25.8575209861776	9.13582017083809\\
25.857078449449	9.13551676014422\\
25.8566358901725	9.13521333885139\\
25.8561933083458	9.13490990695889\\
25.8557507039666	9.13460646446595\\
25.8553080770326	9.13430301137186\\
25.8548654275414	9.13399954767585\\
25.8544227554909	9.13369607337719\\
25.8539800608786	9.13339258847515\\
25.8535373437023	9.13308909296898\\
25.8530946039597	9.13278558685793\\
25.8526518416485	9.13248207014127\\
25.8522090567664	9.13217854281826\\
25.8517662493111	9.13187500488816\\
25.8513234192803	9.13157145635021\\
25.8508805666716	9.13126789720369\\
25.8504376914828	9.13096432744785\\
25.8499947937116	9.13066074708195\\
25.8495518733557	9.13035715610525\\
25.8491089304127	9.130053554517\\
25.8486659648804	9.12974994231647\\
25.8482229767565	9.12944631950291\\
25.8477799660386	9.12914268607558\\
25.8473369327245	9.12883904203374\\
25.8468938768118	9.12853538737664\\
25.8464507982983	9.12823172210354\\
25.8460076971816	9.12792804621371\\
25.8455645734595	9.1276243597064\\
25.8451214271295	9.12732066258086\\
25.8446782581895	9.12701695483636\\
25.8442350666371	9.12671323647214\\
25.84379185247	9.12640950748748\\
25.8433486156859	9.12610576788162\\
25.8429053562824	9.12580201765382\\
25.8424620742574	9.12549825680334\\
25.8420187696084	9.12519448532944\\
25.8415754423332	9.12489070323138\\
25.8411320924294	9.1245869105084\\
25.8406887198947	9.12428310715978\\
25.8402453247269	9.12397929318475\\
25.8398019069236	9.12367546858259\\
25.8393584664825	9.12337163335254\\
25.8389150034013	9.12306778749387\\
25.8384715176777	9.12276393100583\\
25.8380280093093	9.12246006388767\\
25.8375844782939	9.12215618613866\\
25.8371409246292	9.12185229775804\\
25.8366973483128	9.12154839874508\\
25.8362537493424	9.12124448909903\\
25.8358101277157	9.12094056881914\\
25.8353664834304	9.12063663790468\\
25.8349228164841	9.1203326963549\\
25.8344791268747	9.12002874416905\\
25.8340354145996	9.11972478134638\\
25.8335916796567	9.11942080788617\\
25.8331479220436	9.11911682378765\\
25.832704141758	9.11881282905009\\
25.8322603387976	9.11850882367274\\
25.83181651316	9.11820480765486\\
25.831372664843	9.1179007809957\\
25.8309287938442	9.11759674369451\\
25.8304849001613	9.11729269575055\\
25.830040983792	9.11698863716308\\
25.829597044734	9.11668456793135\\
25.8291530829849	9.11638048805462\\
25.8287090985424	9.11607639753214\\
25.8282650914043	9.11577229636316\\
25.8278210615681	9.11546818454694\\
25.8273770090317	9.11516406208273\\
25.8269329337925	9.11485992896979\\
25.8264888358484	9.11455578520738\\
25.826044715197	9.11425163079474\\
25.825600571836	9.11394746573113\\
25.825156405763	9.11364329001581\\
25.8247122169758	9.11333910364802\\
25.8242680054719	9.11303490662703\\
25.8238237712492	9.11273069895208\\
25.8233795143052	9.11242648062243\\
25.8229352346377	9.11212225163733\\
25.8224909322443	9.11181801199605\\
25.8220466071227	9.11151376169782\\
25.8216022592705	9.11120950074191\\
25.8211578886855	9.11090522912756\\
25.8207134953653	9.11060094685403\\
25.8202690793076	9.11029665392058\\
25.8198246405101	9.10999235032645\\
25.8193801789703	9.10968803607091\\
25.8189356946861	9.10938371115319\\
25.8184911876551	9.10907937557256\\
25.8180466578749	9.10877502932827\\
25.8176021053432	9.10847067241957\\
25.8171575300578	9.10816630484571\\
25.8167129320161	9.10786192660595\\
25.8162683112161	9.10755753769954\\
25.8158236676552	9.10725313812572\\
25.8153790013312	9.10694872788376\\
25.8149343122417	9.10664430697291\\
25.8144896003845	9.10633987539241\\
25.8140448657571	9.10603543314153\\
25.8136001083573	9.1057309802195\\
25.8131553281826	9.10542651662558\\
25.8127105252309	9.10512204235904\\
25.8122656994997	9.10481755741911\\
25.8118208509868	9.10451306180504\\
25.8113759796897	9.1042085555161\\
25.8109310856061	9.10390403855153\\
25.8104861687338	9.10359951091058\\
25.8100412290704	9.10329497259251\\
25.8095962666135	9.10299042359656\\
25.8091512813608	9.10268586392198\\
25.8087062733099	9.10238129356804\\
25.8082612424587	9.10207671253398\\
25.8078161888046	9.10177212081904\\
25.8073711123454	9.10146751842248\\
25.8069260130787	9.10116290534356\\
25.8064808910021	9.10085828158152\\
25.8060357461135	9.10055364713561\\
25.8055905784103	9.10024900200509\\
25.8051453878904	9.0999443461892\\
25.8047001745512	9.09963967968719\\
25.8042549383906	9.09933500249832\\
25.8038096794061	9.09903031462183\\
25.8033643975954	9.09872561605698\\
25.8029190929562	9.09842090680301\\
25.8024737654861	9.09811618685918\\
25.8020284151829	9.09781145622473\\
25.801583042044	9.09750671489892\\
25.8011376460673	9.09720196288099\\
25.8006922272504	9.0968972001702\\
25.8002467855908	9.0965924267658\\
25.7998013210864	9.09628764266702\\
25.7993558337347	9.09598284787313\\
25.7989103235334	9.09567804238337\\
25.7984647904801	9.09537322619699\\
25.7980192345725	9.09506839931324\\
25.7975736558083	9.09476356173137\\
25.7971280541851	9.09445871345063\\
25.7966824297006	9.09415385447027\\
25.7962367823524	9.09384898478954\\
25.7957911121382	9.09354410440768\\
25.7953454190556	9.09323921332394\\
25.7948997031023	9.09293431153758\\
25.794453964276	9.09262939904784\\
25.7940082025742	9.09232447585397\\
25.7935624179946	9.09201954195522\\
25.793116610535	9.09171459735084\\
25.7926707801929	9.09140964204007\\
25.792224926966	9.09110467602217\\
25.7917790508519	9.09079969929637\\
25.7913331518484	9.09049471186194\\
25.7908872299529	9.09018971371811\\
25.7904412851633	9.08988470486414\\
25.7899953174771	9.08957968529927\\
25.789549326892	9.08927465502276\\
25.7891033134057	9.08896961403384\\
25.7886572770157	9.08866456233177\\
25.7882112177198	9.08835949991579\\
25.7877651355155	9.08805442678515\\
25.7873190304006	9.0877493429391\\
25.7868729023727	9.08744424837689\\
25.7864267514294	9.08713914309775\\
25.7859805775683	9.08683402710095\\
25.7855343807872	9.08652890038573\\
25.7850881610836	9.08622376295133\\
25.7846419184552	9.085918614797\\
25.7841956528997	9.08561345592198\\
25.7837493644147	9.08530828632553\\
25.7833030529978	9.08500310600689\\
25.7828567186467	9.08469791496531\\
25.782410361359	9.08439271320003\\
25.7819639811323	9.0840875007103\\
25.7815175779644	9.08378227749537\\
25.7810711518528	9.08347704355448\\
25.7806247027952	9.08317179888687\\
25.7801782307893	9.08286654349181\\
25.7797317358326	9.08256127736852\\
25.7792852179228	9.08225600051626\\
25.7788386770576	9.08195071293427\\
25.7783921132345	9.0816454146218\\
25.7779455264513	9.08134010557809\\
25.7774989167056	9.08103478580239\\
25.777052283995	9.08072945529395\\
25.7766056283171	9.080424114052\\
25.7761589496696	9.08011876207581\\
25.7757122480501	9.0798133993646\\
25.7752655234563	9.07950802591762\\
25.7748187758857	9.07920264173413\\
25.7743720053361	9.07889724681337\\
25.7739252118051	9.07859184115457\\
25.7734783952903	9.07828642475699\\
25.7730315557893	9.07798099761987\\
25.7725846932998	9.07767555974246\\
25.7721378078194	9.07737011112399\\
25.7716908993457	9.07706465176373\\
25.7712439678764	9.07675918166089\\
25.7707970134091	9.07645370081475\\
25.7703500359415	9.07614820922453\\
25.7699030354711	9.07584270688947\\
25.7694560119957	9.07553719380884\\
25.7690089655127	9.07523166998187\\
25.76856189602	9.07492613540779\\
25.768114803515	9.07462059008587\\
25.7676676879955	9.07431503401534\\
25.767220549459	9.07400946719545\\
25.7667733879033	9.07370388962543\\
25.7663262033258	9.07339830130453\\
25.7658789957243	9.073092702232\\
25.7654317650964	9.07278709240708\\
25.7649845114397	9.07248147182902\\
25.7645372347519	9.07217584049705\\
25.7640899350305	9.07187019841041\\
25.7636426122732	9.07156454556836\\
25.7631952664776	9.07125888197014\\
25.7627478976413	9.07095320761498\\
25.7623005057621	9.07064752250214\\
25.7618530908374	9.07034182663084\\
25.761405652865	9.07003612000035\\
25.7609581918424	9.06973040260989\\
25.7605107077673	9.06942467445872\\
25.7600632006372	9.06911893554607\\
25.75961567045	9.06881318587118\\
25.759168117203	9.0685074254333\\
25.758720540894	9.06820165423168\\
25.7582729415207	9.06789587226555\\
25.7578253190805	9.06759007953415\\
25.7573776735712	9.06728427603673\\
25.7569300049904	9.06697846177253\\
25.7564823133356	9.06667263674079\\
25.7560345986045	9.06636680094075\\
25.7555868607948	9.06606095437166\\
25.755139099904	9.06575509703275\\
25.7546913159298	9.06544922892327\\
25.7542435088697	9.06514335004246\\
25.7537956787215	9.06483746038956\\
25.7533478254827	9.06453155996381\\
25.7528999491509	9.06422564876445\\
25.7524520497238	9.06391972679073\\
25.752004127199	9.06361379404188\\
25.7515561815741	9.06330785051714\\
25.7511082128467	9.06300189621576\\
25.7506602210144	9.06269593113698\\
25.7502122060749	9.06238995528004\\
25.7497641680257	9.06208396864418\\
25.7493161068646	9.06177797122863\\
25.748868022589	9.06147196303264\\
25.7484199151966	9.06116594405546\\
25.7479717846851	9.06085991429631\\
25.747523631052	9.06055387375445\\
25.7470754542949	9.0602478224291\\
25.7466272544115	9.05994176031952\\
25.7461790313994	9.05963568742493\\
25.7457307852562	9.05932960374459\\
25.7452825159795	9.05902350927772\\
25.744834223567	9.05871740402358\\
25.7443859080161	9.0584112879814\\
25.7439375693246	9.05810516115041\\
25.74348920749	9.05779902352986\\
25.74304082251	9.057492875119\\
25.7425924143822	9.05718671591704\\
25.7421439831042	9.05688054592325\\
25.7416955286735	9.05657436513685\\
25.7412470510879	9.05626817355708\\
25.7407985503448	9.05596197118319\\
25.740350026442	9.05565575801442\\
25.739901479377	9.05534953404999\\
25.7394529091475	9.05504329928916\\
25.739004315751	9.05473705373115\\
25.7385556991851	9.05443079737521\\
25.7381070594475	9.05412453022058\\
25.7376583965357	9.05381825226649\\
25.7372097104475	9.05351196351218\\
25.7367610011803	9.0532056639569\\
25.7363122687317	9.05289935359988\\
25.7358635130995	9.05259303244035\\
25.7354147342811	9.05228670047756\\
25.7349659322742	9.05198035771074\\
25.7345171070765	9.05167400413913\\
25.7340682586854	9.05136763976198\\
25.7336193870986	9.05106126457851\\
25.7331704923137	9.05075487858796\\
25.7327215743284	9.05044848178958\\
25.7322726331401	9.0501420741826\\
25.7318236687465	9.04983565576625\\
25.7313746811453	9.04952922653978\\
25.7309256703339	9.04922278650242\\
25.7304766363101	9.0489163356534\\
25.7300275790714	9.04860987399198\\
25.7295784986154	9.04830340151737\\
25.7291293949397	9.04799691822883\\
25.7286802680419	9.04769042412558\\
25.7282311179196	9.04738391920687\\
25.7277819445704	9.04707740347193\\
25.7273327479919	9.04677087691999\\
25.7268835281817	9.0464643395503\\
25.7264342851374	9.04615779136208\\
25.7259850188566	9.04585123235458\\
25.7255357293369	9.04554466252704\\
25.7250864165758	9.04523808187868\\
25.7246370805711	9.04493149040874\\
25.7241877213202	9.04462488811647\\
25.7237383388208	9.04431827500109\\
25.7232889330704	9.04401165106185\\
25.7228395040667	9.04370501629797\\
25.7223900518072	9.0433983707087\\
25.7219405762896	9.04309171429326\\
25.7214910775115	9.0427850470509\\
25.7210415554703	9.04247836898085\\
25.7205920101638	9.04217168008235\\
25.7201424415895	9.04186498035463\\
25.719692849745	9.04155826979692\\
25.7192432346279	9.04125154840846\\
25.7187935962357	9.0409448161885\\
25.7183439345662	9.04063807313625\\
25.7178942496168	9.04033131925096\\
25.7174445413851	9.04002455453186\\
25.7169948098688	9.03971777897818\\
25.7165450550655	9.03941099258917\\
25.7160952769726	9.03910419536405\\
25.7156454755879	9.03879738730206\\
25.7151956509088	9.03849056840243\\
25.7147458029331	9.0381837386644\\
25.7142959316582	9.03787689808721\\
25.7138460370818	9.03757004667008\\
25.7133961192014	9.03726318441225\\
25.7129461780146	9.03695631131296\\
25.712496213519	9.03664942737143\\
25.7120462257123	9.03634253258691\\
25.7115962145919	9.03603562695862\\
25.7111461801555	9.0357287104858\\
25.7106961224006	9.03542178316769\\
25.7102460413249	9.03511484500351\\
25.7097959369259	9.0348078959925\\
25.7093458092011	9.0345009361339\\
25.7088956581483	9.03419396542694\\
25.7084454837649	9.03388698387084\\
25.7079952860486	9.03357999146485\\
25.7075450649969	9.03327298820819\\
25.7070948206074	9.0329659741001\\
25.7066445528777	9.03265894913981\\
25.7061942618053	9.03235191332656\\
25.7057439473879	9.03204486665958\\
25.7052936096231	9.0317378091381\\
25.7048432485083	9.03143074076135\\
25.7043928640412	9.03112366152856\\
25.7039424562194	9.03081657143897\\
25.7034920250404	9.03050947049181\\
25.7030415705018	9.03020235868631\\
25.7025910926012	9.02989523602171\\
25.7021405913362	9.02958810249723\\
25.7016900667044	9.02928095811212\\
25.7012395187032	9.02897380286559\\
25.7007889473304	9.02866663675688\\
25.7003383525834	9.02835945978523\\
25.6998877344599	9.02805227194986\\
25.6994370929574	9.02774507325\\
25.6989864280735	9.0274378636849\\
25.6985357398057	9.02713064325378\\
25.6980850281517	9.02682341195586\\
25.6976342931091	9.02651616979039\\
25.6971835346753	9.02620891675659\\
25.6967327528479	9.0259016528537\\
25.6962819476246	9.02559437808094\\
25.6958311190029	9.02528709243754\\
25.6953802669804	9.02497979592275\\
25.6949293915546	9.02467248853578\\
25.6944784927231	9.02436517027587\\
25.6940275704836	9.02405784114225\\
25.6935766248334	9.02375050113415\\
25.6931256557703	9.02344315025079\\
25.6926746632918	9.02313578849142\\
25.6922236473954	9.02282841585526\\
25.6917726080788	9.02252103234154\\
25.6913215453395	9.02221363794949\\
25.690870459175	9.02190623267834\\
25.690419349583	9.02159881652732\\
25.6899682165609	9.02129138949566\\
25.6895170601064	9.02098395158259\\
25.6890658802171	9.02067650278734\\
25.6886146768904	9.02036904310915\\
25.688163450124	9.02006157254723\\
25.6877121999154	9.01975409110081\\
25.6872609262622	9.01944659876914\\
25.6868096291619	9.01913909555143\\
25.6863583086122	9.01883158144692\\
25.6859069646105	9.01852405645483\\
25.6854555971545	9.0182165205744\\
25.6850042062417	9.01790897380485\\
25.6845527918696	9.01760141614542\\
25.6841013540359	9.01729384759532\\
25.683649892738	9.0169862681538\\
25.6831984079736	9.01667867782007\\
25.6827468997402	9.01637107659337\\
25.6822953680354	9.01606346447292\\
25.6818438128567	9.01575584145796\\
25.6813922342017	9.01544820754771\\
25.6809406320679	9.0151405627414\\
25.6804890064529	9.01483290703825\\
25.6800373573543	9.01452524043751\\
25.6795856847696	9.01421756293839\\
25.6791339886964	9.01390987454011\\
25.6786822691322	9.01360217524192\\
25.6782305260746	9.01329446504304\\
25.6777787595211	9.01298674394269\\
25.6773269694694	9.01267901194011\\
25.6768751559169	9.01237126903451\\
25.6764233188612	9.01206351522513\\
25.6759714582999	9.0117557505112\\
25.6755195742305	9.01144797489194\\
25.6750676666505	9.01114018836658\\
25.6746157355576	9.01083239093435\\
25.6741637809493	9.01052458259446\\
25.6737118028231	9.01021676334616\\
25.6732598011766	9.00990893318866\\
25.6728077760074	9.0096010921212\\
25.6723557273129	9.009293240143\\
25.6719036550907	9.00898537725329\\
25.6714515593385	9.00867750345128\\
25.6709994400537	9.00836961873622\\
25.6705472972339	9.00806172310732\\
25.6700951308766	9.00775381656382\\
25.6696429409794	9.00744589910493\\
25.6691907275399	9.00713797072989\\
25.6687384905555	9.00683003143792\\
25.6682862300239	9.00652208122825\\
25.6678339459426	9.0062141201001\\
25.6673816383091	9.00590614805269\\
25.6669293071209	9.00559816508526\\
25.6664769523757	9.00529017119703\\
25.666024574071	9.00498216638722\\
25.6655721722043	9.00467415065506\\
25.6651197467731	9.00436612399978\\
25.6646672977751	9.0040580864206\\
25.6642148252077	9.00375003791674\\
25.6637623290685	9.00344197848744\\
25.663309809355	9.00313390813191\\
25.6628572660648	9.00282582684938\\
25.6624046991955	9.00251773463908\\
25.6619521087445	9.00220963150022\\
25.6614994947094	9.00190151743205\\
25.6610468570878	9.00159339243377\\
25.6605941958772	9.00128525650461\\
25.6601415110751	9.0009771096438\\
25.6596888026791	9.00066895185057\\
25.6592360706867	9.00036078312414\\
25.6587833150955	9.00005260346372\\
25.6583305359029	8.99974441286855\\
25.6578777331066	8.99943621133785\\
25.6574249067041	8.99912799887084\\
25.6569720566929	8.99881977546675\\
25.6565191830705	8.9985115411248\\
25.6560662858346	8.99820329584422\\
25.6556133649825	8.99789503962422\\
25.655160420512	8.99758677246403\\
25.6547074524204	8.99727849436288\\
25.6542544607053	8.99697020531999\\
25.6538014453644	8.99666190533458\\
25.653348406395	8.99635359440587\\
25.6528953437948	8.99604527253309\\
25.6524422575613	8.99573693971547\\
25.6519891476919	8.99542859595221\\
25.6515360141843	8.99512024124255\\
25.651082857036	8.99481187558572\\
25.6506296762446	8.99450349898092\\
25.6501764718074	8.99419511142739\\
25.6497232437222	8.99388671292435\\
25.6492699919863	8.99357830347101\\
25.6488167165974	8.99326988306661\\
25.648363417553	8.99296145171036\\
25.6479100948505	8.99265300940149\\
25.6474567484876	8.99234455613922\\
25.6470033784618	8.99203609192277\\
25.6465499847705	8.99172761675135\\
25.6460965674114	8.99141913062421\\
25.6456431263819	8.99111063354055\\
25.6451896616796	8.9908021254996\\
25.6447361733021	8.99049360650057\\
25.6442826612467	8.9901850765427\\
25.6438291255111	8.98987653562521\\
25.6433755660928	8.9895679837473\\
25.6429219829894	8.98925942090821\\
25.6424683761982	8.98895084710716\\
25.642014745717	8.98864226234336\\
25.6415610915431	8.98833366661604\\
25.6411074136741	8.98802505992442\\
25.6406537121076	8.98771644226772\\
25.6401999868411	8.98740781364516\\
25.639746237872	8.98709917405596\\
25.639292465198	8.98679052349935\\
25.6388386688165	8.98648186197454\\
25.6383848487251	8.98617318948075\\
25.6379310049212	8.9858645060172\\
25.6374771374025	8.98555581158312\\
25.6370232461663	8.98524710617772\\
25.6365693312104	8.98493838980023\\
25.6361153925321	8.98462966244986\\
25.6356614301289	8.98432092412583\\
25.6352074439985	8.98401217482737\\
25.6347534341384	8.98370341455369\\
25.6342994005459	8.98339464330402\\
25.6338453432187	8.98308586107757\\
25.6333912621544	8.98277706787356\\
25.6329371573503	8.98246826369121\\
25.632483028804	8.98215944852974\\
25.6320288765131	8.98185062238838\\
25.631574700475	8.98154178526633\\
25.6311205006873	8.98123293716282\\
25.6306662771475	8.98092407807707\\
25.6302120298531	8.9806152080083\\
25.6297577588016	8.98030632695573\\
25.6293034639906	8.97999743491856\\
25.6288491454175	8.97968853189603\\
25.6283948030798	8.97937961788736\\
25.6279404369751	8.97907069289175\\
25.6274860471009	8.97876175690843\\
25.6270316334547	8.97845280993662\\
25.6265771960341	8.97814385197553\\
25.6261227348364	8.97783488302439\\
25.6256682498593	8.97752590308241\\
25.6252137411003	8.97721691214881\\
25.6247592085568	8.97690791022281\\
25.6243046522263	8.97659889730362\\
25.6238500721065	8.97628987339047\\
25.6233954681947	8.97598083848257\\
25.6229408404886	8.97567179257913\\
25.6224861889855	8.97536273567938\\
25.6220315136831	8.97505366778254\\
25.6215768145788	8.97474458888781\\
25.6211220916701	8.97443549899443\\
25.6206673449546	8.9741263981016\\
25.6202125744297	8.97381728620854\\
25.6197577800929	8.97350816331447\\
25.6193029619419	8.97319902941861\\
25.618848119974	8.97288988452017\\
25.6183932541867	8.97258072861837\\
25.6179383645777	8.97227156171243\\
25.6174834511443	8.97196238380156\\
25.6170285138841	8.97165319488498\\
25.6165735527946	8.97134399496191\\
25.6161185678733	8.97103478403156\\
25.6156635591177	8.97072556209315\\
25.6152085265253	8.9704163291459\\
25.6147534700936	8.97010708518901\\
25.6142983898201	8.96979783022171\\
25.6138432857024	8.96948856424322\\
25.6133881577378	8.96917928725274\\
25.6129330059239	8.96886999924951\\
25.6124778302582	8.96856070023272\\
25.6120226307383	8.96825139020159\\
25.6115674073615	8.96794206915535\\
25.6111121601255	8.96763273709321\\
25.6106568890276	8.96732339401438\\
25.6102015940655	8.96701403991807\\
25.6097462752366	8.96670467480351\\
25.6092909325384	8.96639529866991\\
25.6088355659683	8.96608591151648\\
25.608380175524	8.96577651334243\\
25.6079247612028	8.96546710414699\\
25.6074693230024	8.96515768392937\\
25.6070138609201	8.96484825268878\\
25.6065583749535	8.96453881042444\\
25.6061028651001	8.96422935713556\\
25.6056473313573	8.96391989282135\\
25.6051917737227	8.96361041748103\\
25.6047361921938	8.96330093111382\\
25.604280586768	8.96299143371893\\
25.6038249574428	8.96268192529557\\
25.6033693042158	8.96237240584295\\
25.6029136270844	8.9620628753603\\
25.6024579260462	8.96175333384682\\
25.6020022010985	8.96144378130173\\
25.6015464522389	8.96113421772424\\
25.6010906794649	8.96082464311357\\
25.600634882774	8.96051505746892\\
25.6001790621637	8.96020546078952\\
25.5997232176315	8.95989585307457\\
25.5992673491747	8.95958623432329\\
25.5988114567911	8.9592766045349\\
25.5983555404779	8.9589669637086\\
25.5978996002328	8.9586573118436\\
25.5974436360531	8.95834764893913\\
25.5969876479365	8.95803797499439\\
25.5965316358803	8.9577282900086\\
25.5960755998821	8.95741859398097\\
25.5956195399393	8.95710888691071\\
25.5951634560495	8.95679916879703\\
25.59470734821	8.95648943963915\\
25.5942512164185	8.95617969943628\\
25.5937950606724	8.95586994818764\\
25.5933388809691	8.95556018589242\\
25.5928826773062	8.95525041254985\\
25.5924264496811	8.95494062815914\\
25.5919701980914	8.9546308327195\\
25.5915139225344	8.95432102623013\\
25.5910576230077	8.95401120869026\\
25.5906012995088	8.9537013800991\\
25.5901449520351	8.95339154045585\\
25.5896885805842	8.95308168975973\\
25.5892321851535	8.95277182800995\\
25.5887757657404	8.95246195520572\\
25.5883193223425	8.95215207134625\\
25.5878628549573	8.95184217643076\\
25.5874063635822	8.95153227045844\\
25.5869498482146	8.95122235342853\\
25.5864933088522	8.95091242534021\\
25.5860367454923	8.95060248619272\\
25.5855801581325	8.95029253598525\\
25.5851235467701	8.94998257471702\\
25.5846669114028	8.94967260238724\\
25.5842102520279	8.94936261899511\\
25.583753568643	8.94905262453986\\
25.5832968612454	8.94874261902069\\
25.5828401298328	8.9484326024368\\
25.5823833744025	8.94812257478742\\
25.581926594952	8.94781253607174\\
25.5814697914789	8.94750248628899\\
25.5810129639805	8.94719242543836\\
25.5805561124544	8.94688235351908\\
25.580099236898	8.94657227053034\\
25.5796423373089	8.94626217647136\\
25.5791854136844	8.94595207134135\\
25.578728466022	8.94564195513952\\
25.5782714943192	8.94533182786508\\
25.5778144985736	8.94502168951723\\
25.5773574787825	8.94471154009519\\
25.5769004349434	8.94440137959816\\
25.5764433670538	8.94409120802536\\
25.5759862751111	8.94378102537599\\
25.5755291591129	8.94347083164926\\
25.5750720190566	8.94316062684438\\
25.5746148549396	8.94285041096057\\
25.5741576667595	8.94254018399701\\
25.5737004545137	8.94222994595294\\
25.5732432181996	8.94191969682755\\
25.5727859578148	8.94160943662005\\
25.5723286733566	8.94129916532965\\
25.5718713648227	8.94098888295557\\
25.5714140322103	8.940678589497\\
25.5709566755171	8.94036828495316\\
25.5704992947404	8.94005796932325\\
25.5700418898777	8.93974764260648\\
25.5695844609264	8.93943730480206\\
25.5691270078842	8.9391269559092\\
25.5686695307483	8.9388165959271\\
25.5682120295163	8.93850622485497\\
25.5677545041857	8.93819584269202\\
25.5672969547538	8.93788544943746\\
25.5668393812181	8.9375750450905\\
25.5663817835762	8.93726462965033\\
25.5659241618255	8.93695420311617\\
25.5654665159634	8.93664376548723\\
25.5650088459873	8.93633331676271\\
25.5645511518948	8.93602285694182\\
25.5640934336834	8.93571238602377\\
25.5636356913504	8.93540190400775\\
25.5631779248933	8.93509141089299\\
25.5627201343096	8.93478090667868\\
25.5622623195967	8.93447039136403\\
25.5618044807521	8.93415986494825\\
25.5613466177732	8.93384932743054\\
25.5608887306576	8.93353877881012\\
25.5604308194026	8.93322821908618\\
25.5599728840058	8.93291764825793\\
25.5595149244645	8.93260706632459\\
25.5590569407762	8.93229647328534\\
25.5585989329384	8.9319858691394\\
25.5581409009485	8.93167525388599\\
25.557682844804	8.93136462752429\\
25.5572247645023	8.93105399005352\\
25.5567666600409	8.93074334147288\\
25.5563085314173	8.93043268178157\\
25.5558503786288	8.93012201097881\\
25.555392201673	8.9298113290638\\
25.5549340005473	8.92950063603574\\
25.5544757752491	8.92918993189383\\
25.5540175257758	8.92887921663729\\
25.5535592521251	8.92856849026532\\
25.5531009542942	8.92825775277711\\
25.5526426322806	8.92794700417189\\
25.5521842860818	8.92763624444884\\
25.5517259156953	8.92732547360718\\
25.5512675211184	8.92701469164611\\
25.5508091023486	8.92670389856483\\
25.5503506593834	8.92639309436255\\
25.5498921922202	8.92608227903848\\
25.5494337008565	8.92577145259181\\
25.5489751852897	8.92546061502174\\
25.5485166455172	8.9251497663275\\
25.5480580815365	8.92483890650827\\
25.5475994933451	8.92452803556326\\
25.5471408809403	8.92421715349168\\
25.5466822443197	8.92390626029272\\
25.5462235834806	8.9235953559656\\
25.5457648984205	8.92328444050951\\
25.5453061891369	8.92297351392367\\
25.5448474556272	8.92266257620726\\
25.5443886978888	8.92235162735949\\
25.5439299159191	8.92204066737958\\
25.5434711097157	8.92172969626671\\
25.5430122792759	8.9214187140201\\
25.5425534245973	8.92110772063895\\
25.5420945456771	8.92079671612245\\
25.5416356425129	8.92048570046981\\
25.5411767151021	8.92017467368024\\
25.5407177634422	8.91986363575293\\
25.5402587875306	8.91955258668709\\
25.5397997873646	8.91924152648193\\
25.5393407629418	8.91893045513663\\
25.5388817142596	8.91861937265041\\
25.5384226413155	8.91830827902247\\
25.5379635441068	8.917997174252\\
25.537504422631	8.91768605833822\\
25.5370452768855	8.91737493128032\\
25.5365861068678	8.9170637930775\\
25.5361269125753	8.91675264372896\\
25.5356676940054	8.91644148323392\\
25.5352084511556	8.91613031159156\\
25.5347491840232	8.91581912880109\\
25.5342898926058	8.91550793486171\\
25.5338305769008	8.91519672977262\\
25.5333712369055	8.91488551353302\\
25.5329118726175	8.91457428614212\\
25.5324524840342	8.91426304759911\\
25.5319930711529	8.9139517979032\\
25.5315336339711	8.91364053705358\\
25.5310741724863	8.91332926504946\\
25.5306146866958	8.91301798189004\\
25.5301551765971	8.91270668757451\\
25.5296956421877	8.91239538210208\\
25.5292360834649	8.91208406547195\\
25.5287765004262	8.91177273768332\\
25.528316893069	8.91146139873538\\
25.5278572613907	8.91115004862734\\
25.5273976053888	8.9108386873584\\
25.5269379250607	8.91052731492776\\
25.5264782204037	8.91021593133461\\
25.5260184914155	8.90990453657817\\
25.5255587380932	8.90959313065761\\
25.5250989604345	8.90928171357216\\
25.5246391584367	8.908970285321\\
25.5241793320972	8.90865884590333\\
25.5237194814134	8.90834739531836\\
25.5232596063828	8.90803593356528\\
25.5227997070028	8.90772446064329\\
25.5223397832709	8.9074129765516\\
25.5218798351843	8.90710148128939\\
25.5214198627407	8.90678997485587\\
25.5209598659373	8.90647845725024\\
25.5204998447716	8.90616692847169\\
25.520039799241	8.90585538851943\\
25.519579729343	8.90554383739266\\
25.5191196350749	8.90523227509056\\
25.5186595164342	8.90492070161234\\
25.5181993734182	8.9046091169572\\
25.5177392060245	8.90429752112434\\
25.5172790142504	8.90398591411295\\
25.5168187980934	8.90367429592223\\
25.5163585575508	8.90336266655138\\
25.5158982926201	8.9030510259996\\
25.5154380032986	8.90273937426608\\
25.5149776895839	8.90242771135003\\
25.5145173514733	8.90211603725063\\
25.5140569889642	8.90180435196709\\
25.5135966020541	8.90149265549861\\
25.5131361907403	8.90118094784438\\
25.5126757550203	8.9008692290036\\
25.5122152948915	8.90055749897546\\
25.5117548103512	8.90024575775917\\
25.511294301397	8.89993400535391\\
25.5108337680262	8.89962224175889\\
25.5103732102362	8.89931046697331\\
25.5099126280245	8.89899868099635\\
25.5094520213884	8.89868688382722\\
25.5089913903253	8.89837507546512\\
25.5085307348328	8.89806325590923\\
25.5080700549081	8.89775142515876\\
25.5076093505487	8.8974395832129\\
25.507148621752	8.89712773007085\\
25.5066878685153	8.8968158657318\\
25.5062270908362	8.89650399019495\\
25.505766288712	8.8961921034595\\
25.5053054621402	8.89588020552464\\
25.504844611118	8.89556829638956\\
25.504383735643	8.89525637605347\\
25.5039228357125	8.89494444451556\\
25.5034619113239	8.89463250177501\\
25.5030009624747	8.89432054783104\\
25.5025399891622	8.89400858268283\\
25.5020789913839	8.89369660632958\\
25.5016179691372	8.89338461877049\\
25.5011569224194	8.89307262000474\\
25.5006958512279	8.89276061003154\\
25.5002347555602	8.89244858885008\\
25.4997736354137	8.89213655645954\\
25.4993124907857	8.89182451285914\\
25.4988513216737	8.89151245804806\\
25.4983901280751	8.8912003920255\\
25.4979289099872	8.89088831479065\\
25.4974676674075	8.8905762263427\\
25.4970064003333	8.89026412668086\\
25.4965451087621	8.8899520158043\\
25.4960837926912	8.88963989371224\\
25.4956224521181	8.88932776040386\\
25.4951610870402	8.88901561587835\\
25.4946996974548	8.88870346013492\\
25.4942382833593	8.88839129317274\\
25.4937768447512	8.88807911499103\\
25.4933153816278	8.88776692558896\\
25.4928538939865	8.88745472496574\\
25.4923923818248	8.88714251312055\\
25.4919308451399	8.88683029005259\\
25.4914692839294	8.88651805576106\\
25.4910076981906	8.88620581024515\\
25.4905460879209	8.88589355350404\\
25.4900844531177	8.88558128553693\\
25.4896227937783	8.88526900634302\\
25.4891611099002	8.8849567159215\\
25.4886994014808	8.88464441427156\\
25.4882376685175	8.88433210139239\\
25.4877759110076	8.88401977728319\\
25.4873141289485	8.88370744194314\\
25.4868523223377	8.88339509537145\\
25.4863904911724	8.88308273756729\\
25.4859286354502	8.88277036852987\\
25.4854667551684	8.88245798825838\\
25.4850048503243	8.88214559675201\\
25.4845429209154	8.88183319400994\\
25.4840809669391	8.88152078003138\\
25.4836189883927	8.88120835481551\\
25.4831569852736	8.88089591836152\\
25.4826949575792	8.88058347066862\\
25.482232905307	8.88027101173598\\
25.4817708284542	8.87995854156279\\
25.4813087270183	8.87964606014826\\
25.4808466009966	8.87933356749157\\
25.4803844503866	8.87902106359192\\
25.4799222751856	8.87870854844849\\
25.479460075391	8.87839602206047\\
25.4789978510002	8.87808348442705\\
25.4785356020105	8.87777093554744\\
25.4780733284194	8.87745837542081\\
25.4776110302242	8.87714580404635\\
25.4771487074224	8.87683322142327\\
25.4766863600112	8.87652062755074\\
25.4762239879881	8.87620802242796\\
25.4757615913504	8.87589540605412\\
25.4752991700956	8.8755827784284\\
25.474836724221	8.87527013955001\\
25.4743742537239	8.87495748941812\\
25.4739117586018	8.87464482803193\\
25.4734492388521	8.87433215539063\\
25.4729866944721	8.87401947149341\\
25.4725241254592	8.87370677633946\\
25.4720615318107	8.87339406992796\\
25.4715989135241	8.8730813522581\\
25.4711362705967	8.87276862332909\\
25.470673603026	8.87245588314009\\
25.4702109108091	8.87214313169031\\
25.4697481939437	8.87183036897894\\
25.4692854524269	8.87151759500515\\
25.4688226862563	8.87120480976814\\
25.4683598954291	8.87089201326711\\
25.4678970799427	8.87057920550123\\
25.4674342397946	8.8702663864697\\
25.466971374982	8.8699535561717\\
25.4665084855024	8.86964071460643\\
25.4660455713531	8.86932786177307\\
25.4655826325315	8.86901499767081\\
25.465119669035	8.86870212229884\\
25.4646566808609	8.86838923565634\\
25.4641936680066	8.86807633774252\\
25.4637306304695	8.86776342855653\\
25.4632675682469	8.8674505080976\\
25.4628044813362	8.86713757636488\\
25.4623413697348	8.86682463335759\\
25.4618782334401	8.86651167907489\\
25.4614150724493	8.86619871351599\\
25.4609518867599	8.86588573668007\\
25.4604886763693	8.86557274856631\\
25.4600254412748	8.8652597491739\\
25.4595621814737	8.86494673850203\\
25.4590988969635	8.86463371654989\\
25.4586355877414	8.86432068331666\\
25.4581722538049	8.86400763880154\\
25.4577088951514	8.86369458300369\\
25.4572455117781	8.86338151592232\\
25.4567821036825	8.86306843755662\\
25.4563186708619	8.86275534790576\\
25.4558552133137	8.86244224696893\\
25.4553917310352	8.86212913474532\\
25.4549282240238	8.86181601123412\\
25.4544646922768	8.86150287643451\\
25.4540011357917	8.86118973034568\\
25.4535375545658	8.86087657296681\\
25.4530739485963	8.86056340429708\\
25.4526103178808	8.8602502243357\\
25.4521466624165	8.85993703308184\\
25.4516829822008	8.85962383053468\\
25.4512192772311	8.85931061669342\\
25.4507555475047	8.85899739155723\\
25.450291793019	8.85868415512531\\
25.4498280137714	8.85837090739684\\
25.4493642097591	8.85805764837099\\
25.4489003809795	8.85774437804697\\
25.4484365274301	8.85743109642395\\
25.4479726491081	8.85711780350112\\
25.4475087460109	8.85680449927766\\
25.4470448181359	8.85649118375277\\
25.4465808654804	8.85617785692561\\
25.4461168880417	8.85586451879538\\
25.4456528858173	8.85555116936126\\
25.4451888588044	8.85523780862244\\
25.4447248070005	8.8549244365781\\
25.4442607304029	8.85461105322743\\
25.4437966290088	8.8542976585696\\
25.4433325028157	8.85398425260381\\
25.442868351821	8.85367083532923\\
25.4424041760219	8.85335740674505\\
25.4419399754159	8.85304396685046\\
25.4414757500002	8.85273051564463\\
25.4410114997722	8.85241705312676\\
25.4405472247293	8.85210357929602\\
25.4400829248688	8.8517900941516\\
25.4396186001881	8.85147659769268\\
25.4391542506845	8.85116308991845\\
25.4386898763553	8.85084957082809\\
25.4382254771979	8.85053604042077\\
25.4377610532097	8.85022249869569\\
25.4372966043879	8.84990894565203\\
25.43683213073	8.84959538128897\\
25.4363676322332	8.84928180560569\\
25.435903108895	8.84896821860138\\
25.4354385607126	8.84865462027521\\
25.4349739876834	8.84834101062637\\
25.4345093898048	8.84802738965405\\
25.434044767074	8.84771375735742\\
25.4335801194885	8.84740011373566\\
25.4331154470455	8.84708645878797\\
25.4326507497425	8.84677279251352\\
25.4321860275766	8.84645911491149\\
25.4317212805454	8.84614542598106\\
25.4312565086461	8.84583172572142\\
25.4307917118761	8.84551801413175\\
25.4303268902326	8.84520429121123\\
25.4298620437131	8.84489055695904\\
25.4293971723149	8.84457681137436\\
25.4289322760353	8.84426305445638\\
25.4284673548716	8.84394928620427\\
25.4280024088212	8.84363550661722\\
25.4275374378815	8.8433217156944\\
25.4270724420497	8.84300791343501\\
25.4266074213232	8.84269409983821\\
25.4261423756993	8.84238027490319\\
25.4256773051754	8.84206643862913\\
25.4252122097487	8.84175259101521\\
25.4247470894167	8.84143873206061\\
25.4242819441767	8.84112486176451\\
25.4238167740259	8.8408109801261\\
25.4233515789618	8.84049708714455\\
25.4228863589816	8.84018318281904\\
25.4224211140827	8.83986926714875\\
25.4219558442624	8.83955534013286\\
25.421490549518	8.83924140177056\\
25.4210252298469	8.83892745206102\\
25.4205598852465	8.83861349100343\\
25.4200945157139	8.83829951859695\\
25.4196291212466	8.83798553484077\\
25.4191637018419	8.83767153973407\\
25.4186982574972	8.83735753327603\\
25.4182327882096	8.83704351546583\\
25.4177672939766	8.83672948630265\\
25.4173017747956	8.83641544578567\\
25.4168362306637	8.83610139391406\\
25.4163706615784	8.83578733068701\\
25.415905067537	8.83547325610369\\
25.4154394485367	8.83515917016328\\
25.414973804575	8.83484507286496\\
25.4145081356491	8.83453096420791\\
25.4140424417564	8.83421684419131\\
25.4135767228941	8.83390271281434\\
25.4131109790597	8.83358857007617\\
25.4126452102504	8.83327441597599\\
25.4121794164636	8.83296025051296\\
25.4117135976965	8.83264607368627\\
25.4112477539465	8.8323318854951\\
25.410781885211	8.83201768593862\\
25.4103159914871	8.83170347501602\\
25.4098500727724	8.83138925272647\\
25.409384129064	8.83107501906914\\
25.4089181603593	8.83076077404322\\
25.4084521666556	8.83044651764788\\
25.4079861479502	8.83013224988229\\
25.4075201042404	8.82981797074565\\
25.4070540355237	8.82950368023712\\
25.4065879417972	8.82918937835588\\
25.4061218230583	8.82887506510111\\
25.4056556793043	8.82856074047198\\
25.4051895105325	8.82824640446767\\
25.4047233167402	8.82793205708736\\
25.4042570979249	8.82761769833023\\
25.4037908540836	8.82730332819544\\
25.4033245852139	8.82698894668219\\
25.4028582913129	8.82667455378963\\
25.4023919723781	8.82636014951696\\
25.4019256284066	8.82604573386334\\
25.4014592593959	8.82573130682796\\
25.4009928653432	8.82541686840998\\
25.4005264462459	8.82510241860858\\
25.4000600021012	8.82478795742295\\
25.3995935329065	8.82447348485225\\
25.399127038659	8.82415900089566\\
25.3986605193562	8.82384450555236\\
25.3981939749952	8.82352999882152\\
25.3977274055735	8.82321548070231\\
25.3972608110882	8.82290095119392\\
25.3967941915368	8.82258641029551\\
25.3963275469165	8.82227185800627\\
25.3958608772246	8.82195729432536\\
25.3953941824584	8.82164271925197\\
25.3949274626153	8.82132813278526\\
25.3944607176926	8.82101353492442\\
25.3939939476875	8.82069892566861\\
25.3935271525973	8.82038430501701\\
25.3930603324194	8.8200696729688\\
25.3925934871511	8.81975502952314\\
25.3921266167896	8.81944037467922\\
25.3916597213323	8.81912570843621\\
25.3911928007764	8.81881103079328\\
25.3907258551193	8.8184963417496\\
25.3902588843583	8.81818164130435\\
25.3897918884907	8.81786692945671\\
25.3893248675137	8.81755220620584\\
25.3888578214247	8.81723747155093\\
25.388390750221	8.81692272549114\\
25.3879236538998	8.81660796802565\\
25.3874565324585	8.81629319915362\\
25.3869893858944	8.81597841887425\\
25.3865222142047	8.81566362718669\\
25.3860550173868	8.81534882409012\\
25.385587795438	8.81503400958371\\
25.3851205483555	8.81471918366664\\
25.3846532761366	8.81440434633808\\
25.3841859787787	8.8140894975972\\
25.383718656279	8.81377463744317\\
25.3832513086349	8.81345976587517\\
25.3827839358436	8.81314488289237\\
25.3823165379024	8.81282998849395\\
25.3818491148086	8.81251508267906\\
25.3813816665595	8.8122001654469\\
25.3809141931525	8.81188523679662\\
25.3804466945847	8.8115702967274\\
25.3799791708535	8.81125534523841\\
25.3795116219561	8.81094038232883\\
25.37904404789	8.81062540799782\\
25.3785764486523	8.81031042224456\\
25.3781088242403	8.80999542506822\\
25.3776411746513	8.80968041646797\\
25.3771734998827	8.80936539644298\\
25.3767057999317	8.80905036499242\\
25.3762380747956	8.80873532211547\\
25.3757703244717	8.80842026781129\\
25.3753025489572	8.80810520207906\\
25.3748347482495	8.80779012491795\\
25.3743669223459	8.80747503632712\\
25.3738990712436	8.80715993630576\\
25.3734311949398	8.80684482485302\\
25.372963293432	8.80652970196808\\
25.3724953667174	8.80621456765011\\
25.3720274147933	8.80589942189828\\
25.3715594376569	8.80558426471176\\
25.3710914353055	8.80526909608973\\
25.3706234077365	8.80495391603134\\
25.370155354947	8.80463872453578\\
25.3696872769344	8.8043235216022\\
25.369219173696	8.80400830722979\\
25.3687510452291	8.80369308141771\\
25.3682828915308	8.80337784416512\\
25.3678147125986	8.80306259547121\\
25.3673465084297	8.80274733533514\\
25.3668782790213	8.80243206375608\\
25.3664100243708	8.80211678073319\\
25.3659417444753	8.80180148626565\\
25.3654734393323	8.80148618035263\\
25.365005108939	8.80117086299329\\
25.3645367532926	8.8008555341868\\
25.3640683723904	8.80054019393234\\
25.3635999662298	8.80022484222907\\
25.3631315348079	8.79990947907616\\
25.3626630781221	8.79959410447278\\
25.3621945961696	8.7992787184181\\
25.3617260889477	8.79896332091128\\
25.3612575564537	8.79864791195149\\
25.3607889986848	8.79833249153791\\
25.3603204156384	8.7980170596697\\
25.3598518073116	8.79770161634603\\
25.3593831737019	8.79738616156606\\
25.3589145148064	8.79707069532897\\
25.3584458306224	8.79675521763391\\
25.3579771211471	8.79643972848007\\
25.357508386378	8.7961242278666\\
25.3570396263121	8.79580871579268\\
25.3565708409468	8.79549319225747\\
25.3561020302794	8.79517765726014\\
25.3556331943072	8.79486211079986\\
25.3551643330273	8.79454655287579\\
25.3546954464371	8.7942309834871\\
25.3542265345338	8.79391540263295\\
25.3537575973148	8.79359981031252\\
25.3532886347772	8.79328420652497\\
25.3528196469183	8.79296859126947\\
25.3523506337354	8.79265296454518\\
25.3518815952258	8.79233732635128\\
25.3514125313868	8.79202167668692\\
25.3509434422155	8.79170601555127\\
25.3504743277092	8.7913903429435\\
25.3500051878653	8.79107465886278\\
25.349536022681	8.79075896330827\\
25.3490668321535	8.79044325627914\\
25.3485976162801	8.79012753777455\\
25.3481283750581	8.78981180779367\\
25.3476591084847	8.78949606633566\\
25.3471898165572	8.7891803133997\\
25.3467204992728	8.78886454898494\\
25.3462511566289	8.78854877309055\\
25.3457817886226	8.78823298571569\\
25.3453123952513	8.78791718685954\\
25.3448429765121	8.78760137652126\\
25.3443735324024	8.7872855547\\
25.3439040629193	8.78696972139495\\
25.3434345680603	8.78665387660525\\
25.3429650478224	8.78633802033008\\
25.342495502203	8.7860221525686\\
25.3420259311994	8.78570627331998\\
25.3415563348087	8.78539038258338\\
25.3410867130283	8.78507448035796\\
25.3406170658553	8.78475856664289\\
25.3401473932871	8.78444264143733\\
25.3396776953209	8.78412670474044\\
25.3392079719539	8.7838107565514\\
25.3387382231835	8.78349479686937\\
25.3382684490068	8.7831788256935\\
25.3377986494211	8.78286284302296\\
25.3373288244236	8.78254684885692\\
25.3368589740117	8.78223084319454\\
25.3363890981825	8.78191482603499\\
25.3359191969333	8.78159879737741\\
25.3354492702614	8.78128275722099\\
25.334979318164	8.78096670556488\\
25.3345093406384	8.78065064240825\\
25.3340393376818	8.78033456775026\\
25.3335693092914	8.78001848159006\\
25.3330992554645	8.77970238392683\\
25.3326291761983	8.77938627475973\\
25.3321590714902	8.77907015408792\\
25.3316889413373	8.77875402191056\\
25.3312187857368	8.77843787822682\\
25.3307486046861	8.77812172303585\\
25.3302783981824	8.77780555633682\\
25.3298081662229	8.7774893781289\\
25.3293379088048	8.77717318841124\\
25.3288676259255	8.776856987183\\
25.3283973175821	8.77654077444335\\
25.3279269837719	8.77622455019145\\
25.3274566244921	8.77590831442647\\
25.32698623974	8.77559206714755\\
25.3265158295127	8.77527580835387\\
25.3260453938077	8.77495953804459\\
25.325574932622	8.77464325621886\\
25.325104445953	8.77432696287585\\
25.3246339337979	8.77401065801472\\
25.3241633961538	8.77369434163464\\
25.3236928330181	8.77337801373475\\
25.323222244388	8.77306167431423\\
25.3227516302607	8.77274532337223\\
25.3222809906335	8.77242896090791\\
25.3218103255036	8.77211258692044\\
25.3213396348682	8.77179620140897\\
25.3208689187246	8.77147980437267\\
25.32039817707	8.7711633958107\\
25.3199274099016	8.77084697572221\\
25.3194566172167	8.77053054410637\\
25.3189857990125	8.77021410096234\\
25.3185149552862	8.76989764628927\\
25.3180440860351	8.76958118008633\\
25.3175731912564	8.76926470235267\\
25.3171022709474	8.76894821308747\\
25.3166313251052	8.76863171228986\\
25.3161603537271	8.76831519995902\\
25.3156893568104	8.76799867609411\\
25.3152183343522	8.76768214069428\\
25.3147472863498	8.76736559375869\\
25.3142762128004	8.76704903528651\\
25.3138051137013	8.76673246527689\\
25.3133339890497	8.76641588372899\\
25.3128628388427	8.76609929064196\\
25.3123916630777	8.76578268601498\\
25.3119204617518	8.76546606984719\\
25.3114492348624	8.76514944213775\\
25.3109779824065	8.76483280288583\\
25.3105067043815	8.76451615209059\\
25.3100354007845	8.76419948975117\\
25.3095640716128	8.76388281586674\\
25.3090927168637	8.76356613043646\\
25.3086213365343	8.76324943345949\\
25.3081499306218	8.76293272493497\\
25.3076784991235	8.76261600486208\\
25.3072070420366	8.76229927323997\\
25.3067355593584	8.76198253006779\\
25.3062640510859	8.76166577534471\\
25.3057925172166	8.76134900906988\\
25.3053209577476	8.76103223124245\\
25.304849372676	8.7607154418616\\
25.3043777619992	8.76039864092646\\
25.3039061257143	8.76008182843621\\
25.3034344638186	8.75976500438999\\
25.3029627763093	8.75944816878697\\
25.3024910631836	8.7591313216263\\
25.3020193244387	8.75881446290714\\
25.3015475600719	8.75849759262864\\
25.3010757700803	8.75818071078997\\
25.3006039544612	8.75786381739027\\
25.3001321132117	8.7575469124287\\
25.2996602463292	8.75722999590443\\
25.2991883538108	8.7569130678166\\
25.2987164356537	8.75659612816438\\
25.2982444918552	8.75627917694692\\
25.2977725224124	8.75596221416336\\
25.2973005273226	8.75564523981288\\
25.296828506583	8.75532825389463\\
25.2963564601907	8.75501125640776\\
25.2958843881431	8.75469424735142\\
25.2954122904373	8.75437722672478\\
25.2949401670705	8.75406019452699\\
25.29446801804	8.7537431507572\\
25.2939958433429	8.75342609541457\\
25.2935236429765	8.75310902849825\\
25.2930514169379	8.75279195000741\\
25.2925791652245	8.75247485994118\\
25.2921068878333	8.75215775829874\\
25.2916345847615	8.75184064507923\\
25.2911622560065	8.7515235202818\\
25.2906899015654	8.75120638390562\\
25.2902175214355	8.75088923594984\\
25.2897451156138	8.7505720764136\\
25.2892726840976	8.75025490529608\\
25.2888002268842	8.74993772259641\\
25.2883277439707	8.74962052831376\\
25.2878552353543	8.74930332244727\\
25.2873827010323	8.74898610499611\\
25.2869101410018	8.74866887595942\\
25.2864375552601	8.74835163533636\\
25.2859649438043	8.74803438312609\\
25.2854923066317	8.74771711932775\\
25.2850196437394	8.7473998439405\\
25.2845469551247	8.74708255696349\\
25.2840742407847	8.74676525839589\\
25.2836015007166	8.74644794823682\\
25.2831287349177	8.74613062648547\\
25.2826559433852	8.74581329314097\\
25.2821831261162	8.74549594820247\\
25.2817102831079	8.74517859166914\\
25.2812374143576	8.74486122354012\\
25.2807645198624	8.74454384381457\\
25.2802915996196	8.74422645249164\\
25.2798186536263	8.74390904957048\\
25.2793456818797	8.74359163505024\\
25.278872684377	8.74327420893008\\
25.2783996611154	8.74295677120914\\
25.2779266120922	8.74263932188659\\
25.2774535373045	8.74232186096157\\
25.2769804367494	8.74200438843324\\
25.2765073104243	8.74168690430073\\
25.2760341583262	8.74136940856322\\
25.2755609804524	8.74105190121984\\
25.2750877768001	8.74073438226976\\
25.2746145473664	8.74041685171212\\
25.2741412921486	8.74009930954607\\
25.2736680111438	8.73978175577076\\
25.2731947043493	8.73946419038535\\
25.2727213717622	8.73914661338899\\
25.2722480133796	8.73882902478083\\
25.2717746291989	8.73851142456001\\
25.2713012192171	8.73819381272569\\
25.2708277834316	8.73787618927702\\
25.2703543218393	8.73755855421315\\
25.2698808344377	8.73724090753323\\
25.2694073212237	8.73692324923642\\
25.2689337821947	8.73660557932185\\
25.2684602173477	8.73628789778868\\
25.26798662668	8.73597020463607\\
25.2675130101889	8.73565249986316\\
25.2670393678713	8.7353347834691\\
25.2665656997246	8.73501705545304\\
25.2660920057459	8.73469931581413\\
25.2656182859324	8.73438156455152\\
25.2651445402813	8.73406380166436\\
25.2646707687898	8.7337460271518\\
25.264196971455	8.73342824101299\\
25.2637231482741	8.73311044324708\\
25.2632492992443	8.73279263385321\\
25.2627754243628	8.73247481283054\\
25.2623015236268	8.73215698017822\\
25.2618275970334	8.73183913589539\\
25.2613536445799	8.7315212799812\\
25.2608796662633	8.73120341243481\\
25.2604056620809	8.73088553325535\\
25.2599316320299	8.73056764244198\\
25.2594575761074	8.73024973999385\\
25.2589834943106	8.72993182591011\\
25.2585093866366	8.7296139001899\\
25.2580352530828	8.72929596283237\\
25.2575610936461	8.72897801383667\\
25.2570869083239	8.72866005320195\\
25.2566126971132	8.72834208092735\\
25.2561384600113	8.72802409701202\\
25.2556641970153	8.72770610145512\\
25.2551899081224	8.72738809425579\\
25.2547155933297	8.72707007541317\\
25.2542412526345	8.72675204492641\\
25.2537668860339	8.72643400279467\\
25.2532924935251	8.72611594901708\\
25.2528180751052	8.72579788359279\\
25.2523436307715	8.72547980652096\\
25.251869160521	8.72516171780073\\
25.251394664351	8.72484361743124\\
25.2509201422586	8.72452550541165\\
25.250445594241	8.72420738174109\\
25.2499710202953	8.72388924641872\\
25.2494964204188	8.72357109944368\\
25.2490217946086	8.72325294081512\\
25.2485471428618	8.72293477053218\\
25.2480724651756	8.72261658859402\\
25.2475977615473	8.72229839499977\\
25.2471230319738	8.72198018974858\\
25.2466482764525	8.7216619728396\\
25.2461734949805	8.72134374427197\\
25.2456986875549	8.72102550404485\\
25.2452238541729	8.72070725215737\\
25.2447489948316	8.72038898860868\\
25.2442741095283	8.72007071339793\\
25.2437991982601	8.71975242652426\\
25.2433242610241	8.71943412798682\\
25.2428492978176	8.71911581778474\\
25.2423743086375	8.71879749591719\\
25.2418992934813	8.71847916238329\\
25.2414242523459	8.71816081718221\\
25.2409491852285	8.71784246031307\\
25.2404740921263	8.71752409177503\\
25.2399989730365	8.71720571156724\\
25.2395238279563	8.71688731968882\\
25.2390486568826	8.71656891613894\\
25.2385734598128	8.71625050091673\\
25.238098236744	8.71593207402134\\
25.2376229876733	8.71561363545191\\
25.237147712598	8.71529518520759\\
25.236672411515	8.71497672328752\\
25.2361970844217	8.71465824969084\\
25.2357217313151	8.71433976441669\\
25.2352463521924	8.71402126746423\\
25.2347709470508	8.7137027588326\\
25.2342955158873	8.71338423852093\\
25.2338200586993	8.71306570652837\\
25.2333445754837	8.71274716285407\\
25.2328690662378	8.71242860749716\\
25.2323935309587	8.7121100404568\\
25.2319179696436	8.71179146173212\\
25.2314423822896	8.71147287132226\\
25.2309667688938	8.71115426922638\\
25.2304911294534	8.71083565544361\\
25.2300154639656	8.71051702997309\\
25.2295397724275	8.71019839281398\\
25.2290640548362	8.70987974396539\\
25.2285883111889	8.70956108342649\\
25.2281125414828	8.70924241119642\\
25.2276367457149	8.70892372727431\\
25.2271609238825	8.70860503165931\\
25.2266850759826	8.70828632435056\\
25.2262092020125	8.7079676053472\\
25.2257333019692	8.70764887464837\\
25.2252573758499	8.70733013225322\\
25.2247814236517	8.70701137816089\\
25.2243054453718	8.70669261237051\\
25.2238294410074	8.70637383488124\\
25.2233534105555	8.7060550456922\\
25.2228773540133	8.70573624480255\\
25.222401271378	8.70541743221142\\
25.2219251626467	8.70509860791796\\
25.2214490278165	8.70477977192129\\
25.2209728668845	8.70446092422058\\
25.220496679848	8.70414206481495\\
25.2200204667039	8.70382319370355\\
25.2195442274496	8.70350431088551\\
25.2190679620821	8.70318541635999\\
25.2185916705985	8.70286651012611\\
25.218115352996	8.70254759218302\\
25.2176390092717	8.70222866252985\\
25.2171626394228	8.70190972116576\\
25.2166862434464	8.70159076808988\\
25.2162098213396	8.70127180330134\\
25.2157333730995	8.70095282679929\\
25.2152568987234	8.70063383858287\\
25.2147803982082	8.70031483865121\\
25.2143038715512	8.69999582700347\\
25.2138273187495	8.69967680363877\\
25.2133507398002	8.69935776855625\\
25.2128741347005	8.69903872175506\\
25.2123975034474	8.69871966323433\\
25.2119208460382	8.69840059299321\\
25.2114441624698	8.69808151103082\\
25.2109674527396	8.69776241734632\\
25.2104907168445	8.69744331193883\\
25.2100139547818	8.6971241948075\\
25.2095371665485	8.69680506595147\\
25.2090603521417	8.69648592536987\\
25.2085835115587	8.69616677306185\\
25.2081066447965	8.69584760902654\\
25.2076297518522	8.69552843326307\\
25.207152832723	8.6952092457706\\
25.2066758874061	8.69489004654824\\
25.2061989158984	8.69457083559515\\
25.2057219181972	8.69425161291046\\
25.2052448942995	8.69393237849331\\
25.2047678442026	8.69361313234284\\
25.2042907679035	8.69329387445818\\
25.2038136653993	8.69297460483847\\
25.2033365366871	8.69265532348284\\
25.2028593817641	8.69233603039044\\
25.2023822006275	8.69201672556041\\
25.2019049932742	8.69169740899187\\
25.2014277597015	8.69137808068397\\
25.2009504999064	8.69105874063585\\
25.2004732138861	8.69073938884663\\
25.1999959016377	8.69042002531545\\
25.1995185631583	8.69010065004147\\
25.1990411984451	8.6897812630238\\
25.1985638074951	8.68946186426158\\
25.1980863903054	8.68914245375396\\
25.1976089468732	8.68882303150006\\
25.1971314771956	8.68850359749903\\
25.1966539812697	8.68818415175\\
25.1961764590926	8.68786469425211\\
25.1956989106614	8.68754522500448\\
25.1952213359733	8.68722574400626\\
25.1947437350253	8.68690625125659\\
25.1942661078146	8.68658674675459\\
25.1937884543383	8.68626723049941\\
25.1933107745934	8.68594770249017\\
25.1928330685772	8.68562816272602\\
25.1923553362867	8.68530861120609\\
25.191877577719	8.68498904792951\\
25.1913997928712	8.68466947289542\\
25.1909219817404	8.68434988610295\\
25.1904441443238	8.68403028755124\\
25.1899662806185	8.68371067723943\\
25.1894883906215	8.68339105516664\\
25.18901047433	8.68307142133202\\
25.188532531741	8.68275177573469\\
25.1880545628518	8.68243211837379\\
25.1875765676593	8.68211244924845\\
25.1870985461607	8.68179276835781\\
25.1866204983531	8.68147307570101\\
25.1861424242336	8.68115337127717\\
25.1856643237993	8.68083365508543\\
25.1851861970473	8.68051392712493\\
25.1847080439747	8.68019418739479\\
25.1842298645786	8.67987443589415\\
25.1837516588561	8.67955467262215\\
25.1832734268043	8.67923489757792\\
25.1827951684204	8.67891511076058\\
25.1823168837013	8.67859531216928\\
25.1818385726443	8.67827550180315\\
25.1813602352464	8.67795567966131\\
25.1808818715046	8.67763584574291\\
25.1804034814162	8.67731600004707\\
25.1799250649782	8.67699614257292\\
25.1794466221877	8.67667627331961\\
25.1789681530418	8.67635639228627\\
25.1784896575376	8.67603649947201\\
25.1780111356722	8.67571659487598\\
25.1775325874426	8.67539667849731\\
25.1770540128461	8.67507675033514\\
25.1765754118796	8.67475681038858\\
25.1760967845403	8.67443685865678\\
25.1756181308252	8.67411689513887\\
25.1751394507315	8.67379691983397\\
25.1746607442563	8.67347693274122\\
25.1741820113965	8.67315693385976\\
25.1737032521494	8.67283692318871\\
25.1732244665121	8.6725169007272\\
25.1727456544815	8.67219686647436\\
25.1722668160549	8.67187682042934\\
25.1717879512292	8.67155676259124\\
25.1713090600016	8.67123669295922\\
25.1708301423692	8.67091661153239\\
25.1703511983291	8.6705965183099\\
25.1698722278783	8.67027641329086\\
25.1693932310139	8.66995629647441\\
25.168914207733	8.66963616785969\\
25.1684351580327	8.66931602744581\\
25.1679560819102	8.66899587523192\\
25.1674769793624	8.66867571121714\\
25.1669978503864	8.6683555354006\\
25.1665186949794	8.66803534778142\\
25.1660395131385	8.66771514835875\\
25.1655603048606	8.66739493713172\\
25.165081070143	8.66707471409944\\
25.1646018089826	8.66675447926105\\
25.1641225213765	8.66643423261568\\
25.163643207322	8.66611397416246\\
25.1631638668159	8.66579370390051\\
25.1626844998554	8.66547342182898\\
25.1622051064376	8.66515312794698\\
25.1617256865596	8.66483282225364\\
25.1612462402184	8.6645125047481\\
25.1607667674111	8.66419217542948\\
25.1602872681348	8.66387183429691\\
25.1598077423866	8.66355148134952\\
25.1593281901635	8.66323111658644\\
25.1588486114627	8.66291074000679\\
25.1583690062812	8.66259035160971\\
25.157889374616	8.66226995139432\\
25.1574097164643	8.66194953935976\\
25.1569300318231	8.66162911550514\\
25.1564503206895	8.6613086798296\\
25.1559705830606	8.66098823233226\\
25.1554908189334	8.66066777301226\\
25.1550110283051	8.66034730186871\\
25.1545312111726	8.66002681890075\\
25.1540513675331	8.65970632410751\\
25.1535714973837	8.65938581748811\\
25.1530916007213	8.65906529904168\\
25.1526116775432	8.65874476876735\\
25.1521317278462	8.65842422666424\\
25.1516517516276	8.65810367273148\\
25.1511717488844	8.65778310696819\\
25.1506917196136	8.65746252937351\\
25.1502116638123	8.65714193994656\\
25.1497315814776	8.65682133868647\\
25.1492514726066	8.65650072559236\\
25.1487713371963	8.65618010066336\\
25.1482911752437	8.6558594638986\\
25.1478109867461	8.6555388152972\\
25.1473307717003	8.65521815485829\\
25.1468505301035	8.65489748258099\\
25.1463702619528	8.65457679846443\\
25.1458899672451	8.65425610250774\\
25.1454096459777	8.65393539471003\\
25.1449292981474	8.65361467507045\\
25.1444489237515	8.65329394358811\\
25.1439685227869	8.65297320026213\\
25.1434880952507	8.65265244509165\\
25.14300764114	8.65233167807579\\
25.1425271604518	8.65201089921367\\
25.1420466531833	8.65169010850442\\
25.1415661193314	8.65136930594716\\
25.1410855588932	8.65104849154102\\
25.1406049718657	8.65072766528512\\
25.1401243582461	8.65040682717859\\
25.1396437180314	8.65008597722055\\
25.1391630512186	8.64976511541013\\
25.1386823578049	8.64944424174644\\
25.1382016377871	8.64912335622863\\
25.1377208911625	8.6488024588558\\
25.1372401179281	8.64848154962708\\
25.1367593180808	8.6481606285416\\
25.1362784916179	8.64783969559848\\
25.1357976385362	8.64751875079685\\
25.1353167588329	8.64719779413582\\
25.1348358525051	8.64687682561452\\
25.1343549195497	8.64655584523208\\
25.1338739599639	8.64623485298762\\
25.1333929737447	8.64591384888026\\
25.132911960889	8.64559283290912\\
25.1324309213941	8.64527180507333\\
25.1319498552569	8.64495076537201\\
25.1314687624745	8.64462971380428\\
25.1309876430439	8.64430865036927\\
25.1305064969622	8.6439875750661\\
25.1300253242264	8.64366648789389\\
25.1295441248335	8.64334538885176\\
25.1290628987807	8.64302427793884\\
25.128581646065	8.64270315515425\\
25.1281003666833	8.64238202049711\\
25.1276190606328	8.64206087396654\\
25.1271377279105	8.64173971556166\\
25.1266563685135	8.6414185452816\\
25.1261749824387	8.64109736312548\\
25.1256935696832	8.64077616909242\\
25.1252121302441	8.64045496318154\\
25.1247306641185	8.64013374539196\\
25.1242491713032	8.63981251572281\\
25.1237676517955	8.63949127417321\\
25.1232861055923	8.63917002074226\\
25.1228045326907	8.63884875542911\\
25.1223229330876	8.63852747823287\\
25.1218413067803	8.63820618915265\\
25.1213596537656	8.63788488818759\\
25.1208779740406	8.6375635753368\\
25.1203962676024	8.6372422505994\\
25.119914534448	8.63692091397451\\
25.1194327745745	8.63659956546125\\
25.1189509879788	8.63627820505875\\
25.118469174658	8.63595683276612\\
25.1179873346092	8.63563544858248\\
25.1175054678293	8.63531405250696\\
25.1170235743155	8.63499264453867\\
25.1165416540647	8.63467122467673\\
25.116059707074	8.63434979292027\\
25.1155777333404	8.63402834926839\\
25.1150957328609	8.63370689372023\\
25.1146137056326	8.6333854262749\\
25.1141316516525	8.63306394693153\\
25.1136495709176	8.63274245568922\\
25.113167463425	8.6324209525471\\
25.1126853291717	8.63209943750429\\
25.1122031681548	8.63177791055991\\
25.1117209803711	8.63145637171307\\
25.1112387658179	8.6311348209629\\
25.1107565244921	8.6308132583085\\
25.1102742563907	8.63049168374902\\
25.1097919615107	8.63017009728355\\
25.1093096398493	8.62984849891122\\
25.1088272914033	8.62952688863115\\
25.1083449161699	8.62920526644245\\
25.1078625141461	8.62888363234424\\
25.1073800853288	8.62856198633565\\
25.1068976297152	8.62824032841579\\
25.1064151473022	8.62791865858377\\
25.1059326380868	8.62759697683872\\
25.1054501020661	8.62727528317975\\
};
\addplot [color=mycolor1, forget plot]
  table[row sep=crcr]{%
25.1054501020661	8.62727528317975\\
25.1049675392371	8.62695357760597\\
25.1044849495969	8.62663186011652\\
25.1040023331423	8.6263101307105\\
25.1035196898706	8.62598838938703\\
25.1030370197786	8.62566663614522\\
25.1025543228634	8.62534487098421\\
25.102071599122	8.6250230939031\\
25.1015888485515	8.624701304901\\
25.1011060711488	8.62437950397704\\
25.100623266911	8.62405769113033\\
25.1001404358351	8.62373586635999\\
25.0996575779181	8.62341402966514\\
25.099174693157	8.62309218104489\\
25.0986917815488	8.62277032049835\\
25.0982088430906	8.62244844802465\\
25.0977258777793	8.6221265636229\\
25.0972428856121	8.62180466729221\\
25.0967598665858	8.62148275903171\\
25.0962768206975	8.6211608388405\\
25.0957937479442	8.62083890671771\\
25.0953106483229	8.62051696266244\\
25.0948275218307	8.62019500667382\\
25.0943443684646	8.61987303875096\\
25.0938611882215	8.61955105889296\\
25.0933779810984	8.61922906709897\\
25.0928947470924	8.61890706336807\\
25.0924114862005	8.61858504769939\\
25.0919281984198	8.61826302009205\\
25.0914448837471	8.61794098054516\\
25.0909615421795	8.61761892905782\\
25.090478173714	8.61729686562917\\
25.0899947783477	8.61697479025831\\
25.0895113560774	8.61665270294435\\
25.0890279069003	8.61633060368642\\
25.0885444308134	8.61600849248362\\
25.0880609278135	8.61568636933507\\
25.0875773978979	8.61536423423989\\
25.0870938410633	8.61504208719718\\
25.0866102573069	8.61471992820606\\
25.0861266466257	8.61439775726564\\
25.0856430090166	8.61407557437505\\
25.0851593444767	8.61375337953338\\
25.0846756530029	8.61343117273976\\
25.0841919345923	8.61310895399329\\
25.0837081892418	8.6127867232931\\
25.0832244169485	8.61246448063829\\
25.0827406177094	8.61214222602798\\
25.0822567915213	8.61181995946127\\
25.0817729383815	8.61149768093729\\
25.0812890582868	8.61117539045515\\
25.0808051512342	8.61085308801395\\
25.0803212172207	8.61053077361282\\
25.0798372562434	8.61020844725085\\
25.0793532682992	8.60988610892717\\
25.0788692533852	8.60956375864088\\
25.0783852114982	8.60924139639111\\
25.0779011426354	8.60891902217695\\
25.0774170467937	8.60859663599753\\
25.07693292397	8.60827423785196\\
25.0764487741615	8.60795182773933\\
25.075964597365	8.60762940565878\\
25.0754803935776	8.60730697160941\\
25.0749961627962	8.60698452559032\\
25.074511905018	8.60666206760064\\
25.0740276202397	8.60633959763947\\
25.0735433084585	8.60601711570593\\
25.0730589696713	8.60569462179912\\
25.0725746038751	8.60537211591816\\
25.0720902110669	8.60504959806215\\
25.0716057912437	8.60472706823021\\
25.0711213444024	8.60440452642145\\
25.0706368705401	8.60408197263497\\
25.0701523696537	8.60375940686989\\
25.0696678417403	8.60343682912533\\
25.0691832867968	8.60311423940038\\
25.0686987048201	8.60279163769416\\
25.0682140958073	8.60246902400579\\
25.0677294597554	8.60214639833436\\
25.0672447966614	8.60182376067898\\
25.0667601065221	8.60150111103878\\
25.0662753893347	8.60117844941286\\
25.065790645096	8.60085577580032\\
25.0653058738031	8.60053309020028\\
25.064821075453	8.60021039261185\\
25.0643362500426	8.59988768303413\\
25.0638513975689	8.59956496146624\\
25.0633665180288	8.59924222790728\\
25.0628816114195	8.59891948235636\\
25.0623966777377	8.59859672481259\\
25.0619117169806	8.59827395527509\\
25.0614267291451	8.59795117374295\\
25.0609417142282	8.59762838021529\\
25.0604566722267	8.59730557469121\\
25.0599716031378	8.59698275716983\\
25.0594865069584	8.59665992765025\\
25.0590013836855	8.59633708613158\\
25.058516233316	8.59601423261293\\
25.0580310558469	8.5956913670934\\
25.0575458512752	8.59536848957211\\
25.0570606195979	8.59504560004816\\
25.0565753608118	8.59472269852066\\
25.0560900749141	8.59439978498871\\
25.0556047619017	8.59407685945142\\
25.0551194217715	8.59375392190791\\
25.0546340545205	8.59343097235728\\
25.0541486601456	8.59310801079863\\
25.053663238644	8.59278503723108\\
25.0531777900124	8.59246205165372\\
25.0526923142479	8.59213905406567\\
25.0522068113475	8.59181604446603\\
25.051721281308	8.59149302285391\\
25.0512357241266	8.59116998922842\\
25.0507501398001	8.59084694358866\\
25.0502645283255	8.59052388593374\\
25.0497788896998	8.59020081626276\\
25.0492932239199	8.58987773457483\\
25.0488075309828	8.58955464086906\\
25.0483218108855	8.58923153514455\\
25.0478360636249	8.58890841740042\\
25.047350289198	8.58858528763575\\
25.0468644876017	8.58826214584967\\
25.0463786588331	8.58793899204127\\
25.045892802889	8.58761582620966\\
25.0454069197664	8.58729264835394\\
25.0449210094624	8.58696945847323\\
25.0444350719737	8.58664625656663\\
25.0439491072975	8.58632304263323\\
25.0434631154307	8.58599981667216\\
25.0429770963701	8.5856765786825\\
25.0424910501129	8.58535332866337\\
25.0420049766559	8.58503006661387\\
25.041518875996	8.5847067925331\\
25.0410327481303	8.58438350642017\\
25.0405465930557	8.58406020827419\\
25.0400604107692	8.58373689809425\\
25.0395742012676	8.58341357587947\\
25.039087964548	8.58309024162894\\
25.0386017006074	8.58276689534177\\
25.0381154094426	8.58244353701707\\
25.0376290910506	8.58212016665393\\
25.0371427454283	8.58179678425146\\
25.0366563725728	8.58147338980877\\
25.0361699724809	8.58114998332496\\
25.0356835451497	8.58082656479912\\
25.035197090576	8.58050313423037\\
25.0347106087568	8.58017969161781\\
25.0342240996891	8.57985623696053\\
25.0337375633698	8.57953277025766\\
25.0332509997959	8.57920929150827\\
25.0327644089642	8.57888580071149\\
25.0322777908718	8.5785622978664\\
25.0317911455156	8.57823878297212\\
25.0313044728926	8.57791525602775\\
25.0308177729996	8.57759171703238\\
25.0303310458336	8.57726816598513\\
25.0298442913916	8.57694460288509\\
25.0293575096705	8.57662102773136\\
25.0288707006672	8.57629744052305\\
25.0283838643788	8.57597384125926\\
25.0278970008021	8.57565022993909\\
25.027410109934	8.57532660656164\\
25.0269231917716	8.57500297112601\\
25.0264362463118	8.57467932363131\\
25.0259492735514	8.57435566407663\\
25.0254622734875	8.57403199246108\\
25.0249752461169	8.57370830878376\\
25.0244881914366	8.57338461304377\\
25.0240011094436	8.57306090524021\\
25.0235140001348	8.57273718537218\\
25.0230268635071	8.57241345343878\\
25.0225396995574	8.57208970943911\\
25.0220525082828	8.57176595337228\\
25.02156528968	8.57144218523737\\
25.0210780437462	8.57111840503351\\
25.0205907704781	8.57079461275977\\
25.0201034698727	8.57047080841527\\
25.019616141927	8.5701469919991\\
25.0191287866379	8.56982316351037\\
25.0186414040023	8.56949932294817\\
25.0181539940171	8.5691754703116\\
25.0176665566794	8.56885160559976\\
25.0171790919859	8.56852772881176\\
25.0166915999337	8.56820383994669\\
25.0162040805196	8.56787993900365\\
25.0157165337407	8.56755602598174\\
25.0152289595938	8.56723210088005\\
25.0147413580758	8.5669081636977\\
25.0142537291837	8.56658421443377\\
25.0137660729144	8.56626025308737\\
25.0132783892649	8.56593627965759\\
25.012790678232	8.56561229414353\\
25.0123029398127	8.5652882965443\\
25.0118151740039	8.56496428685899\\
25.0113273808025	8.56464026508669\\
25.0108395602054	8.56431623122651\\
25.0103517122097	8.56399218527754\\
25.0098638368121	8.56366812723889\\
25.0093759340097	8.56334405710964\\
25.0088880037993	8.56301997488891\\
25.0084000461779	8.56269588057577\\
25.0079120611423	8.56237177416935\\
25.0074240486896	8.56204765566872\\
25.0069360088165	8.56172352507299\\
25.0064479415201	8.56139938238126\\
25.0059598467973	8.56107522759261\\
25.0054717246449	8.56075106070616\\
25.0049835750599	8.56042688172099\\
25.0044953980393	8.56010269063621\\
25.0040071935798	8.5597784874509\\
25.0035189616785	8.55945427216417\\
25.0030307023322	8.55913004477512\\
25.0025424155379	8.55880580528284\\
25.0020541012925	8.55848155368642\\
25.0015657595929	8.55815728998496\\
25.001077390436	8.55783301417756\\
25.0005889938187	8.55750872626332\\
25.0001005697379	8.55718442624132\\
24.9996121181906	8.55686011411068\\
24.9991236391736	8.55653578987047\\
24.9986351326839	8.5562114535198\\
24.9981465987184	8.55588710505777\\
24.9976580372739	8.55556274448346\\
24.9971694483474	8.55523837179598\\
24.9966808319358	8.55491398699441\\
24.9961921880361	8.55458959007786\\
24.995703516645	8.55426518104542\\
24.9952148177596	8.55394075989619\\
24.9947260913767	8.55361632662925\\
24.9942373374932	8.55329188124371\\
24.993748556106	8.55296742373865\\
24.9932597472121	8.55264295411318\\
24.9927709108083	8.55231847236638\\
24.9922820468916	8.55199397849736\\
24.9917931554589	8.55166947250521\\
24.9913042365069	8.55134495438901\\
24.9908152900328	8.55102042414786\\
24.9903263160333	8.55069588178087\\
24.9898373145053	8.55037132728712\\
24.9893482854458	8.5500467606657\\
24.9888592288516	8.54972218191571\\
24.9883701447198	8.54939759103624\\
24.987881033047	8.54907298802639\\
24.9873918938303	8.54874837288525\\
24.9869027270666	8.54842374561191\\
24.9864135327527	8.54809910620547\\
24.9859243108855	8.54777445466502\\
24.985435061462	8.54744979098964\\
24.984945784479	8.54712511517844\\
24.9844564799335	8.54680042723052\\
24.9839671478223	8.54647572714494\\
24.9834777881422	8.54615101492083\\
24.9829884008903	8.54582629055725\\
24.9824989860634	8.54550155405331\\
24.9820095436584	8.5451768054081\\
24.9815200736722	8.54485204462071\\
24.9810305761017	8.54452727169024\\
24.9805410509437	8.54420248661577\\
24.9800514981952	8.5438776893964\\
24.9795619178531	8.54355288003121\\
24.9790723099142	8.54322805851931\\
24.9785826743754	8.54290322485978\\
24.9780930112336	8.54257837905171\\
24.9776033204857	8.54225352109419\\
24.9771136021286	8.54192865098632\\
24.9766238561592	8.54160376872719\\
24.9761340825744	8.54127887431588\\
24.975644281371	8.54095396775149\\
24.9751544525459	8.54062904903311\\
24.9746645960961	8.54030411815983\\
24.9741747120184	8.53997917513074\\
24.9736848003096	8.53965421994493\\
24.9731948609667	8.5393292526015\\
24.9727048939866	8.53900427309952\\
24.9722148993661	8.5386792814381\\
24.9717248771021	8.53835427761631\\
24.9712348271915	8.53802926163326\\
24.9707447496312	8.53770423348803\\
24.970254644418	8.53737919317971\\
24.9697645115489	8.53705414070739\\
24.9692743510206	8.53672907607016\\
24.9687841628302	8.53640399926711\\
24.9682939469744	8.53607891029733\\
24.9678037034502	8.53575380915991\\
24.9673134322544	8.53542869585393\\
24.9668231333839	8.53510357037849\\
24.9663328068356	8.53477843273268\\
24.9658424526063	8.53445328291558\\
24.965352070693	8.53412812092628\\
24.9648616610924	8.53380294676388\\
24.9643712238015	8.53347776042746\\
24.9638807588171	8.5331525619161\\
24.9633902661362	8.5328273512289\\
24.9628997457556	8.53250212836494\\
24.9624091976721	8.53217689332332\\
24.9619186218826	8.53185164610312\\
24.961428018384	8.53152638670343\\
24.9609373871732	8.53120111512333\\
24.9604467282471	8.53087583136192\\
24.9599560416024	8.53055053541828\\
24.9594653272361	8.5302252272915\\
24.9589745851451	8.52989990698067\\
24.9584838153262	8.52957457448487\\
24.9579930177762	8.5292492298032\\
24.9575021924921	8.52892387293472\\
24.9570113394707	8.52859850387855\\
24.9565204587088	8.52827312263376\\
24.9560295502034	8.52794772919944\\
24.9555386139513	8.52762232357467\\
24.9550476499494	8.52729690575855\\
24.9545566581945	8.52697147575015\\
24.9540656386834	8.52664603354857\\
24.9535745914132	8.52632057915289\\
24.9530835163805	8.5259951125622\\
24.9525924135823	8.52566963377558\\
24.9521012830154	8.52534414279213\\
24.9516101246767	8.52501863961091\\
24.9511189385631	8.52469312423103\\
24.9506277246714	8.52436759665156\\
24.9501364829984	8.5240420568716\\
24.9496452135411	8.52371650489023\\
24.9491539162962	8.52339094070653\\
24.9486625912607	8.52306536431958\\
24.9481712384314	8.52273977572848\\
24.9476798578051	8.52241417493231\\
24.9471884493787	8.52208856193015\\
24.9466970131491	8.52176293672109\\
24.9462055491131	8.52143729930421\\
24.9457140572676	8.5211116496786\\
24.9452225376093	8.52078598784334\\
24.9447309901353	8.52046031379752\\
24.9442394148422	8.52013462754022\\
24.9437478117271	8.51980892907052\\
24.9432561807866	8.51948321838752\\
24.9427645220177	8.51915749549029\\
24.9422728354173	8.51883176037791\\
24.9417811209821	8.51850601304947\\
24.941289378709	8.51818025350406\\
24.9407976085949	8.51785448174076\\
24.9403058106366	8.51752869775865\\
24.939813984831	8.51720290155681\\
24.9393221311749	8.51687709313433\\
24.9388302496651	8.51655127249029\\
24.9383383402985	8.51622543962378\\
24.937846403072	8.51589959453387\\
24.9373544379823	8.51557373721965\\
24.9368624450264	8.51524786768021\\
24.936370424201	8.51492198591462\\
24.9358783755031	8.51459609192197\\
24.9353862989294	8.51427018570135\\
24.9348941944768	8.51394426725182\\
24.9344020621422	8.51361833657248\\
24.9339099019223	8.51329239366241\\
24.9334177138141	8.51296643852068\\
24.9329254978143	8.51264047114639\\
24.9324332539198	8.51231449153861\\
24.9319409821274	8.51198849969643\\
24.931448682434	8.51166249561892\\
24.9309563548364	8.51133647930517\\
24.9304639993315	8.51101045075426\\
24.929971615916	8.51068440996527\\
24.9294792045869	8.51035835693728\\
24.9289867653409	8.51003229166937\\
24.9284942981749	8.50970621416063\\
24.9280018030857	8.50938012441014\\
24.9275092800702	8.50905402241697\\
24.9270167291251	8.50872790818021\\
24.9265241502474	8.50840178169893\\
24.9260315434338	8.50807564297222\\
24.9255389086812	8.50774949199916\\
24.9250462459864	8.50742332877882\\
24.9245535553463	8.5070971533103\\
24.9240608367576	8.50677096559266\\
24.9235680902173	8.50644476562499\\
24.923075315722	8.50611855340637\\
24.9225825132687	8.50579232893588\\
24.9220896828542	8.5054660922126\\
24.9215968244753	8.5051398432356\\
24.9211039381289	8.50481358200397\\
24.9206110238117	8.50448730851678\\
24.9201180815206	8.50416102277313\\
24.9196251112524	8.50383472477207\\
24.9191321130039	8.5035084145127\\
24.918639086772	8.50318209199409\\
24.9181460325535	8.50285575721532\\
24.9176529503452	8.50252941017547\\
24.9171598401439	8.50220305087362\\
24.9166667019465	8.50187667930885\\
24.9161735357498	8.50155029548023\\
24.9156803415505	8.50122389938685\\
24.9151871193455	8.50089749102778\\
24.9146938691317	8.50057107040209\\
24.9142005909058	8.50024463750888\\
24.9137072846647	8.49991819234721\\
24.9132139504052	8.49959173491617\\
24.9127205881241	8.49926526521482\\
24.9122271978182	8.49893878324226\\
24.9117337794843	8.49861228899755\\
24.9112403331192	8.49828578247978\\
24.9107468587199	8.49795926368801\\
24.910253356283	8.49763273262134\\
24.9097598258054	8.49730618927883\\
24.9092662672839	8.49697963365956\\
24.9087726807153	8.49665306576261\\
24.9082790660965	8.49632648558706\\
24.9077854234242	8.49599989313198\\
24.9072917526953	8.49567328839645\\
24.9067980539065	8.49534667137954\\
24.9063043270547	8.49502004208034\\
24.9058105721367	8.49469340049791\\
24.9053167891492	8.49436674663134\\
24.9048229780892	8.4940400804797\\
24.9043291389534	8.49371340204206\\
24.9038352717386	8.49338671131751\\
24.9033413764416	8.49306000830511\\
24.9028474530592	8.49273329300395\\
24.9023535015883	8.49240656541309\\
24.9018595220256	8.49207982553162\\
24.901365514368	8.49175307335861\\
24.9008714786122	8.49142630889313\\
24.9003774147551	8.49109953213426\\
24.8998833227934	8.49077274308108\\
24.899389202724	8.49044594173266\\
24.8988950545436	8.49011912808806\\
24.8984008782491	8.48979230214638\\
24.8979066738372	8.48946546390669\\
24.8974124413048	8.48913861336804\\
24.8969181806487	8.48881175052954\\
24.8964238918657	8.48848487539024\\
24.8959295749524	8.48815798794922\\
24.8954352299059	8.48783108820555\\
24.8949408567228	8.48750417615831\\
24.8944464554	8.48717725180658\\
24.8939520259342	8.48685031514942\\
24.8934575683222	8.4865233661859\\
24.8929630825609	8.48619640491512\\
24.892468568647	8.48586943133613\\
24.8919740265774	8.48554244544801\\
24.8914794563487	8.48521544724983\\
24.8909848579579	8.48488843674067\\
24.8904902314017	8.4845614139196\\
24.8899955766769	8.48423437878569\\
24.8895008937802	8.48390733133801\\
24.8890061827086	8.48358027157564\\
24.8885114434587	8.48325319949765\\
24.8880166760274	8.48292611510312\\
24.8875218804114	8.48259901839111\\
24.8870270566076	8.48227190936069\\
24.8865322046127	8.48194478801095\\
24.8860373244235	8.48161765434095\\
24.8855424160369	8.48129050834975\\
24.8850474794495	8.48096335003645\\
24.8845525146582	8.4806361794001\\
24.8840575216597	8.48030899643978\\
24.8835625004509	8.47998180115456\\
24.8830674510285	8.47965459354351\\
24.8825723733893	8.47932737360571\\
24.8820772675302	8.47900014134022\\
24.8815821334478	8.47867289674611\\
24.8810869711389	8.47834563982246\\
24.8805917806004	8.47801837056833\\
24.8800965618291	8.47769108898281\\
24.8796013148216	8.47736379506495\\
24.8791060395748	8.47703648881383\\
24.8786107360855	8.47670917022852\\
24.8781154043504	8.47638183930809\\
24.8776200443664	8.4760544960516\\
24.8771246561301	8.47572714045814\\
24.8766292396384	8.47539977252677\\
24.8761337948881	8.47507239225656\\
24.8756383218759	8.47474499964658\\
24.8751428205986	8.4744175946959\\
24.874647291053	8.47409017740359\\
24.8741517332358	8.47376274776872\\
24.8736561471439	8.47343530579035\\
24.8731605327739	8.47310785146756\\
24.8726648901228	8.47278038479942\\
24.8721692191871	8.472452905785\\
24.8716735199638	8.47212541442336\\
24.8711777924496	8.47179791071357\\
24.8706820366412	8.47147039465471\\
24.8701862525355	8.47114286624583\\
24.8696904401292	8.47081532548602\\
24.869194599419	8.47048777237434\\
24.8686987304018	8.47016020690985\\
24.8682028330742	8.46983262909163\\
24.8677069074332	8.46950503891874\\
24.8672109534753	8.46917743639025\\
24.8667149711975	8.46884982150523\\
24.8662189605964	8.46852219426274\\
24.8657229216689	8.46819455466187\\
24.8652268544117	8.46786690270166\\
24.8647307588215	8.46753923838119\\
24.8642346348952	8.46721156169953\\
24.8637384826294	8.46688387265575\\
24.863242302021	8.4665561712489\\
24.8627460930667	8.46622845747807\\
24.8622498557633	8.46590073134231\\
24.8617535901075	8.46557299284069\\
24.8612572960961	8.46524524197229\\
24.8607609737259	8.46491747873616\\
24.8602646229936	8.46458970313137\\
24.8597682438959	8.46426191515699\\
24.8592718364297	8.46393411481208\\
24.8587754005916	8.46360630209572\\
24.8582789363785	8.46327847700697\\
24.8577824437871	8.46295063954489\\
24.8572859228141	8.46262278970855\\
24.8567893734564	8.46229492749701\\
24.8562927957106	8.46196705290935\\
24.8557961895735	8.46163916594462\\
24.8552995550419	8.46131126660189\\
24.8548028921126	8.46098335488024\\
24.8543062007822	8.46065543077872\\
24.8538094810475	8.46032749429639\\
24.8533127329053	8.45999954543233\\
24.8528159563523	8.4596715841856\\
24.8523191513853	8.45934361055526\\
24.851822318001	8.45901562454037\\
24.8513254561962	8.45868762614002\\
24.8508285659677	8.45835961535325\\
24.8503316473121	8.45803159217913\\
24.8498347002262	8.45770355661672\\
24.8493377247068	8.4573755086651\\
24.8488407207506	8.45704744832332\\
24.8483436883544	8.45671937559045\\
24.8478466275148	8.45639129046555\\
24.8473495382288	8.45606319294769\\
24.8468524204929	8.45573508303593\\
24.8463552743039	8.45540696072933\\
24.8458580996586	8.45507882602696\\
24.8453608965538	8.45475067892788\\
24.8448636649861	8.45442251943116\\
24.8443664049522	8.45409434753585\\
24.8438691164491	8.45376616324102\\
24.8433717994733	8.45343796654573\\
24.8428744540217	8.45310975744905\\
24.8423770800909	8.45278153595004\\
24.8418796776777	8.45245330204776\\
24.8413822467788	8.45212505574127\\
24.840884787391	8.45179679702964\\
24.840387299511	8.45146852591193\\
24.8398897831356	8.4511402423872\\
24.8393922382615	8.45081194645451\\
24.8388946648853	8.45048363811293\\
24.838397063004	8.45015531736151\\
24.8378994326141	8.44982698419932\\
24.8374017737124	8.44949863862542\\
24.8369040862957	8.44917028063888\\
24.8364063703606	8.44884191023875\\
24.835908625904	8.44851352742409\\
24.8354108529225	8.44818513219397\\
24.8349130514129	8.44785672454744\\
24.8344152213719	8.44752830448358\\
24.8339173627962	8.44719987200143\\
24.8334194756826	8.44687142710006\\
24.8329215600278	8.44654296977854\\
24.8324236158285	8.44621450003591\\
24.8319256430814	8.44588601787125\\
24.8314276417833	8.44555752328361\\
24.830929611931	8.44522901627205\\
24.830431553521	8.44490049683564\\
24.8299334665502	8.44457196497342\\
24.8294353510153	8.44424342068447\\
24.8289372069129	8.44391486396785\\
24.8284390342399	8.44358629482261\\
24.8279408329929	8.4432577132478\\
24.8274426031687	8.4429291192425\\
24.826944344764	8.44260051280576\\
24.8264460577755	8.44227189393664\\
24.8259477421999	8.44194326263421\\
24.8254493980339	8.44161461889751\\
24.8249510252743	8.4412859627256\\
24.8244526239178	8.44095729411756\\
24.8239541939611	8.44062861307243\\
24.8234557354009	8.44029991958928\\
24.822957248234	8.43997121366715\\
24.822458732457	8.43964249530512\\
24.8219601880667	8.43931376450224\\
24.8214616150597	8.43898502125757\\
24.8209630134329	8.43865626557017\\
24.8204643831829	8.43832749743908\\
24.8199657243063	8.43799871686339\\
24.8194670368001	8.43766992384213\\
24.8189683206608	8.43734111837437\\
24.8184695758851	8.43701230045916\\
24.8179708024698	8.43668347009557\\
24.8174720004116	8.43635462728265\\
24.8169731697072	8.43602577201945\\
24.8164743103533	8.43569690430505\\
24.8159754223466	8.43536802413848\\
24.8154765056839	8.43503913151881\\
24.8149775603617	8.4347102264451\\
24.8144785863769	8.4343813089164\\
24.8139795837262	8.43405237893177\\
24.8134805524061	8.43372343649027\\
24.8129814924136	8.43339448159095\\
24.8124824037452	8.43306551423287\\
24.8119832863977	8.43273653441508\\
24.8114841403677	8.43240754213665\\
24.810984965652	8.43207853739662\\
24.8104857622473	8.43174952019406\\
24.8099865301502	8.43142049052801\\
24.8094872693576	8.43109144839754\\
24.808987979866	8.4307623938017\\
24.8084886616722	8.43043332673954\\
24.8079893147728	8.43010424721013\\
24.8074899391647	8.42977515521251\\
24.8069905348444	8.42944605074575\\
24.8064911018087	8.42911693380889\\
24.8059916400543	8.42878780440099\\
24.8054921495778	8.42845866252111\\
24.804992630376	8.4281295081683\\
24.8044930824456	8.42780034134162\\
24.8039935057832	8.42747116204012\\
24.8034939003855	8.42714197026285\\
24.8029942662494	8.42681276600887\\
24.8024946033713	8.42648354927723\\
24.8019949117481	8.426154320067\\
24.8014951913764	8.42582507837721\\
24.800995442253	8.42549582420693\\
24.8004956643744	8.42516655755521\\
24.7999958577374	8.4248372784211\\
24.7994960223388	8.42450798680365\\
24.7989961581751	8.42417868270193\\
24.798496265243	8.42384936611498\\
24.7979963435394	8.42352003704186\\
24.7974963930607	8.42319069548161\\
24.7969964138038	8.4228613414333\\
24.7964964057654	8.42253197489597\\
24.795996368942	8.42220259586868\\
24.7954963033304	8.42187320435049\\
24.7949962089273	8.42154380034044\\
24.7944960857293	8.42121438383758\\
24.7939959337332	8.42088495484097\\
24.7934957529356	8.42055551334966\\
24.7929955433332	8.4202260593627\\
24.7924953049228	8.41989659287915\\
24.7919950377008	8.41956711389806\\
24.7914947416642	8.41923762241847\\
24.7909944168095	8.41890811843944\\
24.7904940631334	8.41857860196003\\
24.7899936806325	8.41824907297928\\
24.7894932693037	8.41791953149624\\
24.7889928291435	8.41758997750998\\
24.7884923601486	8.41726041101952\\
24.7879918623157	8.41693083202394\\
24.7874913356416	8.41660124052228\\
24.7869907801227	8.41627163651358\\
24.7864901957559	8.41594201999691\\
24.7859895825379	8.41561239097131\\
24.7854889404651	8.41528274943583\\
24.7849882695345	8.41495309538952\\
24.7844875697426	8.41462342883144\\
24.783986841086	8.41429374976063\\
24.7834860835616	8.41396405817614\\
24.7829852971659	8.41363435407703\\
24.7824844818956	8.41330463746234\\
24.7819836377473	8.41297490833112\\
24.7814827647179	8.41264516668242\\
24.7809818628038	8.4123154125153\\
24.7804809320019	8.4119856458288\\
24.7799799723087	8.41165586662197\\
24.7794789837209	8.41132607489386\\
24.7789779662352	8.41099627064352\\
24.7784769198483	8.41066645386999\\
24.7779758445568	8.41033662457234\\
24.7774747403574	8.4100067827496\\
24.7769736072467	8.40967692840082\\
24.7764724452215	8.40934706152506\\
24.7759712542783	8.40901718212136\\
24.7754700344139	8.40868729018877\\
24.7749687856249	8.40835738572634\\
24.7744675079079	8.40802746873312\\
24.7739662012597	8.40769753920815\\
24.7734648656769	8.40736759715048\\
24.7729635011561	8.40703764255917\\
24.772462107694	8.40670767543325\\
24.7719606852873	8.40637769577177\\
24.7714592339326	8.40604770357379\\
24.7709577536266	8.40571769883836\\
24.7704562443659	8.40538768156451\\
24.7699547061472	8.40505765175129\\
24.7694531389671	8.40472760939776\\
24.7689515428224	8.40439755450295\\
24.7684499177096	8.40406748706592\\
24.7679482636254	8.40373740708572\\
24.7674465805665	8.40340731456138\\
24.7669448685295	8.40307720949196\\
24.7664431275111	8.40274709187651\\
24.7659413575079	8.40241696171406\\
24.7654395585165	8.40208681900366\\
24.7649377305337	8.40175666374437\\
24.7644358735561	8.40142649593523\\
24.7639339875803	8.40109631557527\\
24.7634320726029	8.40076612266356\\
24.7629301286207	8.40043591719913\\
24.7624281556302	8.40010569918102\\
24.7619261536282	8.3997754686083\\
24.7614241226112	8.39944522548\\
24.7609220625759	8.39911496979516\\
24.760419973519	8.39878470155283\\
24.759917855437	8.39845442075206\\
24.7594157083267	8.39812412739188\\
24.7589135321847	8.39779382147136\\
24.7584113270076	8.39746350298952\\
24.7579090927921	8.39713317194542\\
24.7574068295348	8.3968028283381\\
24.7569045372323	8.3964724721666\\
24.7564022158813	8.39614210342997\\
24.7558998654785	8.39581172212725\\
24.7553974860204	8.39548132825749\\
24.7548950775037	8.39515092181973\\
24.7543926399251	8.39482050281301\\
24.7538901732811	8.39449007123638\\
24.7533876775685	8.39415962708888\\
24.7528851527838	8.39382917036955\\
24.7523825989238	8.39349870107745\\
24.7518800159849	8.3931682192116\\
24.751377403964	8.39283772477106\\
24.7508747628575	8.39250721775486\\
24.7503720926621	8.39217669816206\\
24.7498693933746	8.39184616599169\\
24.7493666649914	8.39151562124279\\
24.7488639075092	8.39118506391441\\
24.7483611209247	8.39085449400559\\
24.7478583052345	8.39052391151538\\
24.7473554604352	8.39019331644281\\
24.7468525865235	8.38986270878693\\
24.7463496834959	8.38953208854678\\
24.7458467513492	8.38920145572141\\
24.7453437900799	8.38887081030984\\
24.7448407996846	8.38854015231113\\
24.74433778016	8.38820948172432\\
24.7438347315028	8.38787879854844\\
24.7433316537095	8.38754810278255\\
24.7428285467767	8.38721739442567\\
24.7423254107011	8.38688667347686\\
24.7418222454794	8.38655593993515\\
24.7413190511081	8.38622519379959\\
24.7408158275838	8.38589443506921\\
24.7403125749032	8.38556366374306\\
24.739809293063	8.38523287982017\\
24.7393059820596	8.38490208329959\\
24.7388026418898	8.38457127418036\\
24.7382992725501	8.38424045246151\\
24.7377958740372	8.3839096181421\\
24.7372924463477	8.38357877122115\\
24.7367889894783	8.38324791169771\\
24.7362855034254	8.38291703957082\\
24.7357819881859	8.38258615483951\\
24.7352784437561	8.38225525750284\\
24.7347748701329	8.38192434755982\\
24.7342712673127	8.38159342500952\\
24.7337676352923	8.38126248985096\\
24.7332639740682	8.38093154208319\\
24.732760283637	8.38060058170524\\
24.7322565639954	8.38026960871615\\
24.7317528151399	8.37993862311496\\
24.7312490370672	8.37960762490072\\
24.7307452297739	8.37927661407245\\
24.7302413932565	8.3789455906292\\
24.7297375275118	8.37861455457001\\
24.7292336325363	8.37828350589391\\
24.7287297083266	8.37795244459995\\
24.7282257548793	8.37762137068715\\
24.7277217721911	8.37729028415457\\
24.7272177602585	8.37695918500123\\
24.7267137190781	8.37662807322618\\
24.7262096486466	8.37629694882844\\
24.7257055489606	8.37596581180707\\
24.7252014200166	8.37563466216109\\
24.7246972618112	8.37530349988955\\
24.7241930743412	8.37497232499148\\
24.723688857603	8.37464113746592\\
24.7231846115933	8.37430993731191\\
24.7226803363086	8.37397872452848\\
24.7221760317456	8.37364749911466\\
24.7216716979009	8.37331626106951\\
24.7211673347711	8.37298501039204\\
24.7206629423527	8.37265374708131\\
24.7201585206424	8.37232247113634\\
24.7196540696368	8.37199118255617\\
24.7191495893324	8.37165988133984\\
24.7186450797259	8.37132856748639\\
24.7181405408138	8.37099724099484\\
24.7176359725928	8.37066590186424\\
24.7171313750595	8.37033455009362\\
24.7166267482104	8.37000318568202\\
24.7161220920421	8.36967180862847\\
24.7156174065512	8.36934041893201\\
24.7151126917344	8.36900901659167\\
24.7146079475881	8.36867760160649\\
24.7141031741091	8.3683461739755\\
24.7135983712938	8.36801473369774\\
24.713093539139	8.36768328077224\\
24.7125886776411	8.36735181519804\\
24.7120837867968	8.36702033697417\\
24.7115788666026	8.36668884609967\\
24.7110739170551	8.36635734257357\\
24.710568938151	8.36602582639491\\
24.7100639298868	8.36569429756271\\
24.709558892259	8.36536275607602\\
24.7090538252644	8.36503120193386\\
24.7085487288994	8.36469963513528\\
24.7080436031606	8.3643680556793\\
24.7075384480447	8.36403646356496\\
24.7070332635482	8.36370485879129\\
24.7065280496677	8.36337324135732\\
24.7060228063998	8.3630416112621\\
24.705517533741	8.36270996850464\\
24.705012231688	8.36237831308399\\
24.7045069002373	8.36204664499918\\
24.7040015393855	8.36171496424924\\
24.7034961491292	8.3613832708332\\
24.7029907294649	8.3610515647501\\
24.7024852803892	8.36071984599896\\
24.7019798018988	8.36038811457882\\
24.7014742939901	8.36005637048872\\
24.7009687566598	8.35972461372768\\
24.7004631899045	8.35939284429474\\
24.6999575937206	8.35906106218893\\
24.6994519681048	8.35872926740927\\
24.6989463130537	8.35839745995482\\
24.6984406285638	8.35806563982458\\
24.6979349146318	8.3577338070176\\
24.6974291712541	8.3574019615329\\
24.6969233984273	8.35707010336953\\
24.696417596148	8.3567382325265\\
24.6959117644128	8.35640634900285\\
24.6954059032183	8.35607445279762\\
24.694900012561	8.35574254390982\\
24.6943940924375	8.3554106223385\\
24.6938881428444	8.35507868808268\\
24.6933821637781	8.3547467411414\\
24.6928761552354	8.35441478151368\\
24.6923701172127	8.35408280919855\\
24.6918640497066	8.35375082419505\\
24.6913579527137	8.3534188265022\\
24.6908518262306	8.35308681611904\\
24.6903456702537	8.35275479304459\\
24.6898394847798	8.35242275727789\\
24.6893332698052	8.35209070881796\\
24.6888270253267	8.35175864766383\\
24.6883207513407	8.35142657381453\\
24.6878144478438	8.35109448726909\\
24.6873081148326	8.35076238802655\\
24.6868017523037	8.35043027608593\\
24.6862953602535	8.35009815144625\\
24.6857889386787	8.34976601410655\\
24.6852824875758	8.34943386406586\\
24.6847760069413	8.3491017013232\\
24.6842694967719	8.34876952587761\\
24.683762957064	8.34843733772811\\
24.6832563878143	8.34810513687372\\
24.6827497890193	8.34777292331349\\
24.6822431606754	8.34744069704643\\
24.6817365027794	8.34710845807157\\
24.6812298153277	8.34677620638795\\
24.6807230983169	8.34644394199459\\
24.6802163517436	8.34611166489051\\
24.6797095756042	8.34577937507475\\
24.6792027698954	8.34544707254632\\
24.6786959346137	8.34511475730427\\
24.6781890697556	8.34478242934762\\
24.6776821753177	8.34445008867539\\
24.6771752512966	8.3441177352866\\
24.6766682976887	8.3437853691803\\
24.6761613144906	8.3434529903555\\
24.675654301699	8.34312059881123\\
24.6751472593102	8.34278819454651\\
24.6746401873209	8.34245577756038\\
24.6741330857276	8.34212334785186\\
24.6736259545269	8.34179090541997\\
24.6731187937153	8.34145845026375\\
24.6726116032893	8.34112598238221\\
24.6721043832454	8.34079350177438\\
24.6715971335803	8.34046100843929\\
24.6710898542905	8.34012850237597\\
24.6705825453724	8.33979598358344\\
24.6700752068227	8.33946345206072\\
24.6695678386378	8.33913090780685\\
24.6690604408144	8.33879835082084\\
24.6685530133489	8.33846578110172\\
24.6680455562379	8.33813319864851\\
24.6675380694779	8.33780060346025\\
24.6670305530655	8.33746799553595\\
24.6665230069971	8.33713537487464\\
24.6660154312694	8.33680274147534\\
24.6655078258789	8.33647009533708\\
24.665000190822	8.33613743645889\\
24.6644925260954	8.33580476483978\\
24.6639848316955	8.33547208047878\\
24.6634771076189	8.33513938337492\\
24.6629693538621	8.33480667352721\\
24.6624615704216	8.33447395093469\\
24.6619537572941	8.33414121559637\\
24.6614459144759	8.33380846751128\\
24.6609380419637	8.33347570667844\\
24.6604301397539	8.33314293309688\\
24.6599222078431	8.33281014676562\\
24.6594142462278	8.33247734768368\\
24.6589062549045	8.33214453585008\\
24.6583982338699	8.33181171126385\\
24.6578901831203	8.33147887392401\\
24.6573821026523	8.33114602382958\\
24.6568739924624	8.33081316097959\\
24.6563658525472	8.33048028537305\\
24.6558576829032	8.330147397009\\
24.6553494835268	8.32981449588645\\
24.6548412544147	8.32948158200441\\
24.6543329955634	8.32914865536193\\
24.6538247069692	8.32881571595801\\
24.6533163886289	8.32848276379168\\
24.6528080405389	8.32814979886196\\
24.6522996626956	8.32781682116787\\
24.6517912550957	8.32748383070844\\
24.6512828177357	8.32715082748268\\
24.650774350612	8.32681781148961\\
24.6502658537212	8.32648478272826\\
24.6497573270597	8.32615174119765\\
24.6492487706242	8.3258186868968\\
24.6487401844111	8.32548561982472\\
24.6482315684169	8.32515253998044\\
24.6477229226382	8.32481944736299\\
24.6472142470714	8.32448634197137\\
24.6467055417131	8.32415322380461\\
24.6461968065598	8.32382009286173\\
24.6456880416079	8.32348694914175\\
24.645179246854	8.3231537926437\\
24.6446704222947	8.32282062336658\\
24.6441615679263	8.32248744130942\\
24.6436526837455	8.32215424647124\\
24.6431437697487	8.32182103885105\\
24.6426348259324	8.32148781844788\\
24.6421258522932	8.32115458526076\\
24.6416168488274	8.32082133928868\\
24.6411078155318	8.32048808053068\\
24.6405987524027	8.32015480898577\\
24.6400896594366	8.31982152465298\\
24.6395805366301	8.31948822753132\\
24.6390713839796	8.3191549176198\\
24.6385622014817	8.31882159491745\\
24.6380529891329	8.31848825942329\\
24.6375437469296	8.31815491113633\\
24.6370344748683	8.3178215500556\\
24.6365251729456	8.3174881761801\\
24.636015841158	8.31715478950886\\
24.6355064795019	8.3168213900409\\
24.6349970879738	8.31648797777522\\
24.6344876665703	8.31615455271086\\
24.6339782152879	8.31582111484683\\
24.6334687341229	8.31548766418214\\
24.632959223072	8.31515420071582\\
24.6324496821316	8.31482072444687\\
24.6319401112982	8.31448723537432\\
24.6314305105683	8.31415373349718\\
24.6309208799384	8.31382021881447\\
24.630411219405	8.31348669132521\\
24.6299015289646	8.31315315102841\\
24.6293918086136	8.31281959792309\\
24.6288820583486	8.31248603200826\\
24.6283722781661	8.31215245328294\\
24.6278624680624	8.31181886174615\\
24.6273526280342	8.31148525739691\\
24.6268427580779	8.31115164023422\\
24.62633285819	8.31081801025711\\
24.625822928367	8.31048436746459\\
24.6253129686053	8.31015071185567\\
24.6248029789015	8.30981704342937\\
24.624292959252	8.30948336218471\\
24.6237829096534	8.3091496681207\\
24.623272830102	8.30881596123636\\
24.6227627205944	8.3084822415307\\
24.622252581127	8.30814850900273\\
24.6217424116965	8.30781476365147\\
24.6212322122991	8.30748100547594\\
24.6207219829314	8.30714723447515\\
24.6202117235899	8.30681345064811\\
24.6197014342711	8.30647965399384\\
24.6191911149714	8.30614584451135\\
24.6186807656873	8.30581202219965\\
24.6181703864153	8.30547818705777\\
24.6176599771518	8.30514433908471\\
24.6171495378935	8.30481047827948\\
24.6166390686366	8.30447660464111\\
24.6161285693777	8.3041427181686\\
24.6156180401132	8.30380881886096\\
24.6151074808397	8.30347490671722\\
24.6145968915536	8.30314098173638\\
24.6140862722514	8.30280704391745\\
24.6135756229295	8.30247309325946\\
24.6130649435844	8.3021391297614\\
24.6125542342126	8.3018051534223\\
24.6120434948106	8.30147116424117\\
24.6115327253748	8.30113716221702\\
24.6110219259017	8.30080314734886\\
24.6105110963877	8.3004691196357\\
24.6100002368294	8.30013507907655\\
24.6094893472232	8.29980102567044\\
24.6089784275655	8.29946695941636\\
24.6084674778528	8.29913288031333\\
24.6079564980816	8.29879878836037\\
24.6074454882484	8.29846468355648\\
24.6069344483495	8.29813056590067\\
24.6064233783816	8.29779643539196\\
24.6059122783409	8.29746229202936\\
24.6054011482241	8.29712813581188\\
24.6048899880275	8.29679396673853\\
24.6043787977476	8.29645978480831\\
24.6038675773809	8.29612559002025\\
24.6033563269238	8.29579138237335\\
24.6028450463728	8.29545716186662\\
24.6023337357244	8.29512292849907\\
24.6018223949749	8.29478868226972\\
24.6013110241209	8.29445442317756\\
24.6007996231588	8.29412015122162\\
24.6002881920851	8.29378586640091\\
24.5997767308962	8.29345156871442\\
24.5992652395886	8.29311725816117\\
24.5987537181587	8.29278293474018\\
24.5982421666029	8.29244859845045\\
24.5977305849178	8.29211424929098\\
24.5972189730998	8.2917798872608\\
24.5967073311453	8.2914455123589\\
24.5961956590508	8.2911111245843\\
24.5956839568127	8.290776723936\\
24.5951722244275	8.29044231041302\\
24.5946604618916	8.29010788401437\\
24.5941486692014	8.28977344473904\\
24.5936368463535	8.28943899258605\\
24.5931249933443	8.28910452755441\\
24.5926131101701	8.28877004964313\\
24.5921011968275	8.28843555885121\\
24.5915892533129	8.28810105517767\\
24.5910772796227	8.2877665386215\\
24.5905652757534	8.28743200918172\\
24.5900532417015	8.28709746685734\\
24.5895411774633	8.28676291164736\\
24.5890290830353	8.28642834355079\\
24.5885169584139	8.28609376256663\\
24.5880048035956	8.2857591686939\\
24.5874926185769	8.2854245619316\\
24.5869804033541	8.28508994227874\\
24.5864681579237	8.28475530973433\\
24.5859558822821	8.28442066429736\\
24.5854435764258	8.28408600596685\\
24.5849312403512	8.28375133474181\\
24.5844188740548	8.28341665062124\\
24.5839064775329	8.28308195360414\\
24.583394050782	8.28274724368953\\
24.5828815937986	8.28241252087641\\
24.582369106579	8.28207778516378\\
24.5818565891198	8.28174303655065\\
24.5813440414173	8.28140827503603\\
24.5808314634679	8.28107350061892\\
24.5803188552682	8.28073871329833\\
24.5798062168145	8.28040391307325\\
24.5792935481032	8.28006909994271\\
24.5787808491308	8.2797342739057\\
24.5782681198938	8.27939943496122\\
24.5777553603884	8.27906458310829\\
24.5772425706113	8.27872971834591\\
24.5767297505587	8.27839484067307\\
24.5762169002271	8.27805995008879\\
24.575704019613	8.27772504659207\\
24.5751911087127	8.27739013018192\\
24.5746781675227	8.27705520085734\\
24.5741651960395	8.27672025861732\\
24.5736521942593	8.27638530346088\\
24.5731391621787	8.27605033538703\\
24.5726260997941	8.27571535439476\\
24.5721130071018	8.27538036048307\\
24.5715998840984	8.27504535365098\\
24.5710867307802	8.27471033389747\\
24.5705735471436	8.27437530122157\\
24.5700603331851	8.27404025562227\\
24.5695470889011	8.27370519709857\\
24.569033814288	8.27337012564947\\
24.5685205093422	8.27303504127399\\
24.5680071740601	8.27269994397111\\
24.5674938084381	8.27236483373985\\
24.5669804124727	8.27202971057921\\
24.5664669861603	8.27169457448819\\
24.5659535294973	8.27135942546578\\
24.56544004248	8.271024263511\\
24.5649265251049	8.27068908862284\\
24.5644129773685	8.27035390080031\\
24.563899399267	8.27001870004241\\
24.563385790797	8.26968348634814\\
24.5628721519548	8.2693482597165\\
24.5623584827369	8.26901302014649\\
24.5618447831396	8.26867776763711\\
24.5613310531594	8.26834250218737\\
24.5608172927926	8.26800722379627\\
24.5603035020358	8.2676719324628\\
24.5597896808851	8.26733662818597\\
24.5592758293372	8.26700131096478\\
24.5587619473883	8.26666598079822\\
24.558248035035	8.26633063768531\\
24.5577340922735	8.26599528162503\\
24.5572201191003	8.2656599126164\\
24.5567061155117	8.2653245306584\\
24.5561920815043	8.26498913575005\\
24.5556780170744	8.26465372789033\\
24.5551639222183	8.26431830707826\\
24.5546497969325	8.26398287331282\\
24.5541356412134	8.26364742659302\\
24.5536214550574	8.26331196691785\\
24.5531072384609	8.26297649428633\\
24.5525929914202	8.26264100869744\\
24.5520787139318	8.26230551015018\\
24.551564405992	8.26196999864356\\
24.5510500675973	8.26163447417658\\
24.550535698744	8.26129893674822\\
24.5500212994286	8.26096338635749\\
24.5495068696474	8.2606278230034\\
24.5489924093968	8.26029224668492\\
24.5484779186733	8.25995665740108\\
24.5479633974731	8.25962105515086\\
24.5474488457927	8.25928543993325\\
24.5469342636285	8.25894981174727\\
24.5464196509769	8.2586141705919\\
24.5459050078342	8.25827851646615\\
24.5453903341969	8.25794284936901\\
24.5448756300613	8.25760716929948\\
24.5443608954237	8.25727147625656\\
24.5438461302807	8.25693577023924\\
24.5433313346286	8.25660005124652\\
24.5428165084637	8.2562643192774\\
24.5423016517825	8.25592857433087\\
24.5417867645813	8.25559281640593\\
24.5412718468565	8.25525704550158\\
24.5407568986045	8.25492126161682\\
24.5402419198217	8.25458546475063\\
24.5397269105044	8.25424965490203\\
24.5392118706491	8.25391383206999\\
24.538696800252	8.25357799625353\\
24.5381816993096	8.25324214745163\\
24.5376665678183	8.25290628566328\\
24.5371514057745	8.2525704108875\\
24.5366362131744	8.25223452312327\\
24.5361209900146	8.25189862236959\\
24.5356057362912	8.25156270862545\\
24.5350904520009	8.25122678188984\\
24.5345751371398	8.25089084216177\\
24.5340597917044	8.25055488944023\\
24.5335444156911	8.25021892372421\\
24.5330290090962	8.2498829450127\\
24.5325135719161	8.24954695330471\\
24.5319981041471	8.24921094859923\\
24.5314826057857	8.24887493089524\\
24.5309670768282	8.24853890019175\\
24.530451517271	8.24820285648775\\
24.5299359271104	8.24786679978223\\
24.5294203063428	8.24753073007419\\
24.5289046549646	8.24719464736262\\
24.5283889729721	8.24685855164652\\
24.5278732603617	8.24652244292487\\
24.5273575171297	8.24618632119668\\
24.5268417432726	8.24585018646093\\
24.5263259387867	8.24551403871661\\
24.5258101036683	8.24517787796274\\
24.5252942379139	8.24484170419828\\
24.5247783415197	8.24450551742224\\
24.5242624144821	8.24416931763361\\
24.5237464567976	8.24383310483138\\
24.5232304684624	8.24349687901455\\
24.5227144494729	8.2431606401821\\
24.5221983998255	8.24282438833304\\
24.5216823195165	8.24248812346634\\
24.5211662085423	8.24215184558101\\
24.5206500668992	8.24181555467604\\
24.5201338945836	8.24147925075041\\
24.5196176915918	8.24114293380312\\
24.5191014579203	8.24080660383316\\
24.5185851935653	8.24047026083953\\
24.5180688985232	8.2401339048212\\
24.5175525727904	8.23979753577719\\
24.5170362163631	8.23946115370646\\
24.5165198292379	8.23912475860803\\
24.5160034114109	8.23878835048087\\
24.5154869628786	8.23845192932398\\
24.5149704836373	8.23811549513635\\
24.5144539736833	8.23777904791697\\
24.5139374330131	8.23744258766483\\
24.5134208616229	8.23710611437892\\
24.5129042595091	8.23676962805823\\
24.512387626668	8.23643312870175\\
24.511870963096	8.23609661630847\\
24.5113542687894	8.23576009087738\\
24.5108375437446	8.23542355240746\\
24.5103207879579	8.23508700089772\\
24.5098040014257	8.23475043634714\\
24.5092871841442	8.23441385875471\\
24.5087703361099	8.23407726811941\\
24.5082534573191	8.23374066444024\\
24.5077365477681	8.23340404771618\\
24.5072196074532	8.23306741794624\\
24.5067026363708	8.23273077512938\\
24.5061856345173	8.2323941192646\\
24.5056686018889	8.2320574503509\\
24.505151538482	8.23172076838725\\
24.5046344442929	8.23138407337265\\
24.504117319318	8.23104736530608\\
24.5036001635537	8.23071064418654\\
24.5030829769961	8.23037391001301\\
24.5025657596418	8.23003716278448\\
24.5020485114869	8.22970040249993\\
24.5015312325278	8.22936362915836\\
24.501013922761	8.22902684275875\\
24.5004965821826	8.22869004330009\\
24.499979210789	8.22835323078136\\
24.4994618085767	8.22801640520156\\
24.4989443755417	8.22767956655967\\
24.4984269116806	8.22734271485467\\
24.4979094169896	8.22700585008556\\
24.4973918914651	8.22666897225132\\
24.4968743351034	8.22633208135094\\
24.4963567479008	8.22599517738339\\
24.4958391298536	8.22565826034768\\
24.4953214809582	8.22532133024279\\
24.4948038012109	8.22498438706769\\
24.494286090608	8.22464743082139\\
24.4937683491457	8.22431046150285\\
24.4932505768206	8.22397347911108\\
24.4927327736288	8.22363648364505\\
24.4922149395667	8.22329947510375\\
24.4916970746306	8.22296245348617\\
24.4911791788168	8.22262541879129\\
24.4906612521217	8.22228837101809\\
24.4901432945415	8.22195131016557\\
24.4896253060727	8.2216142362327\\
24.4891072867114	8.22127714921847\\
24.488589236454	8.22094004912186\\
24.4880711552968	8.22060293594187\\
24.4875530432361	8.22026580967747\\
24.4870349002683	8.21992867032765\\
24.4865167263897	8.21959151789139\\
24.4859985215965	8.21925435236767\\
24.4854802858852	8.21891717375549\\
24.4849620192519	8.21857998205382\\
24.484443721693	8.21824277726165\\
24.4839253932048	8.21790555937796\\
24.4834070337836	8.21756832840173\\
24.4828886434258	8.21723108433195\\
24.4823702221276	8.21689382716761\\
24.4818517698854	8.21655655690767\\
24.4813332866953	8.21621927355114\\
24.4808147725539	8.21588197709699\\
24.4802962274573	8.2155446675442\\
24.4797776514018	8.21520734489175\\
24.4792590443839	8.21487000913863\\
24.4787404063996	8.21453266028382\\
24.4782217374455	8.21419529832631\\
24.4777030375177	8.21385792326507\\
24.4771843066126	8.21352053509909\\
24.4766655447265	8.21318313382734\\
24.4761467518556	8.21284571944881\\
24.4756279279963	8.21250829196249\\
24.4751090731448	8.21217085136736\\
24.4745901872975	8.21183339766238\\
24.4740712704507	8.21149593084656\\
24.4735523226006	8.21115845091886\\
24.4730333437436	8.21082095787827\\
24.4725143338759	8.21048345172377\\
24.4719952929939	8.21014593245433\\
24.4714762210937	8.20980840006896\\
24.4709571181719	8.20947085456661\\
24.4704379842245	8.20913329594627\\
24.4699188192479	8.20879572420693\\
24.4693996232384	8.20845813934756\\
24.4688803961923	8.20812054136714\\
24.4683611381059	8.20778293026466\\
24.4678418489754	8.20744530603909\\
24.4673225287972	8.20710766868941\\
24.4668031775675	8.2067700182146\\
24.4662837952827	8.20643235461365\\
24.465764381939	8.20609467788552\\
24.4652449375326	8.20575698802921\\
24.46472546206	8.20541928504369\\
24.4642059555173	8.20508156892793\\
24.4636864179008	8.20474383968093\\
24.4631668492069	8.20440609730164\\
24.4626472494318	8.20406834178907\\
24.4621276185717	8.20373057314218\\
24.4616079566231	8.20339279135995\\
24.4610882635821	8.20305499644136\\
24.460568539445	8.20271718838539\\
24.4600487842082	8.20237936719101\\
24.4595289978678	8.20204153285721\\
24.4590091804202	8.20170368538296\\
24.4584893318617	8.20136582476725\\
24.4579694521884	8.20102795100904\\
24.4574495413968	8.20069006410731\\
24.456929599483	8.20035216406105\\
24.4564096264434	8.20001425086923\\
24.4558896222741	8.19967632453083\\
24.4553695869716	8.19933838504482\\
24.454849520532	8.19900043241018\\
24.4543294229517	8.19866246662589\\
24.4538092942268	8.19832448769092\\
24.4532891343537	8.19798649560426\\
24.4527689433287	8.19764849036487\\
24.452248721148	8.19731047197174\\
24.4517284678078	8.19697244042383\\
24.4512081833045	8.19663439572014\\
24.4506878676343	8.19629633785962\\
24.4501675207934	8.19595826684126\\
24.4496471427782	8.19562018266404\\
24.4491267335849	8.19528208532692\\
24.4486062932098	8.19494397482889\\
24.4480858216491	8.19460585116892\\
24.4475653188991	8.19426771434598\\
24.4470447849561	8.19392956435906\\
24.4465242198162	8.19359140120712\\
24.4460036234759	8.19325322488914\\
24.4454829959313	8.19291503540409\\
24.4449623371786	8.19257683275096\\
24.4444416472143	8.19223861692871\\
24.4439209260344	8.19190038793631\\
24.4434001736353	8.19156214577275\\
24.4428793900132	8.191223890437\\
24.4423585751644	8.19088562192802\\
24.4418377290852	8.19054734024481\\
24.4413168517717	8.19020904538631\\
24.4407959432203	8.18987073735153\\
24.4402750034272	8.18953241613942\\
24.4397540323886	8.18919408174895\\
24.4392330301008	8.18885573417911\\
24.4387119965601	8.18851737342887\\
24.4381909317627	8.18817899949719\\
24.4376698357048	8.18784061238306\\
24.4371487083827	8.18750221208544\\
24.4366275497927	8.1871637986033\\
24.436106359931	8.18682537193563\\
24.4355851387938	8.18648693208139\\
24.4350638863774	8.18614847903955\\
24.434542602678	8.18581001280909\\
24.4340212876919	8.18547153338898\\
24.4334999414154	8.18513304077819\\
24.4329785638446	8.1847945349757\\
24.4324571549758	8.18445601598046\\
24.4319357148053	8.18411748379147\\
24.4314142433292	8.18377893840768\\
24.4308927405439	8.18344037982807\\
24.4303712064456	8.18310180805161\\
24.4298496410305	8.18276322307727\\
24.4293280442948	8.18242462490403\\
24.4288064162348	8.18208601353085\\
24.4282847568468	8.1817473889567\\
24.4277630661269	8.18140875118056\\
24.4272413440714	8.18107010020139\\
24.4267195906766	8.18073143601817\\
24.4261978059386	8.18039275862986\\
24.4256759898537	8.18005406803545\\
24.4251541424182	8.17971536423388\\
24.4246322636282	8.17937664722415\\
24.4241103534801	8.17903791700521\\
24.42358841197	8.17869917357603\\
24.4230664390941	8.1783604169356\\
24.4225444348487	8.17802164708286\\
24.4220223992301	8.1776828640168\\
24.4215003322344	8.17734406773639\\
24.4209782338579	8.17700525824058\\
24.4204561040968	8.17666643552836\\
24.4199339429474	8.17632759959869\\
24.4194117504058	8.17598875045054\\
24.4188895264683	8.17564988808287\\
24.4183672711311	8.17531101249466\\
24.4178449843904	8.17497212368488\\
24.4173226662425	8.17463322165249\\
24.4168003166835	8.17429430639646\\
24.4162779357098	8.17395537791576\\
24.4157555233175	8.17361643620935\\
24.4152330795028	8.17327748127621\\
24.414710604262	8.17293851311531\\
24.4141880975913	8.1725995317256\\
24.4136655594869	8.17226053710606\\
24.413142989945	8.17192152925566\\
24.4126203889618	8.17158250817335\\
24.4120977565336	8.17124347385811\\
24.4115750926566	8.17090442630892\\
24.4110523973269	8.17056536552472\\
24.4105296705409	8.17022629150449\\
24.4100069122946	8.1698872042472\\
24.4094841225844	8.16954810375181\\
24.4089613014064	8.16920899001729\\
24.4084384487568	8.1688698630426\\
24.4079155646319	8.16853072282671\\
24.4073926490279	8.16819156936859\\
24.4068697019409	8.16785240266721\\
24.4063467233673	8.16751322272152\\
24.4058237133031	8.16717402953049\\
24.4053006717446	8.1668348230931\\
24.4047775986881	8.1664956034083\\
24.4042544941296	8.16615637047505\\
24.4037313580655	8.16581712429234\\
24.4032081904919	8.16547786485911\\
24.402684991405	8.16513859217434\\
24.4021617608011	8.16479930623699\\
24.4016384986763	8.16446000704602\\
24.4011152050268	8.1641206946004\\
24.4005918798489	8.1637813688991\\
24.4000685231387	8.16344202994107\\
24.3995451348925	8.16310267772528\\
24.3990217151063	8.1627633122507\\
24.3984982637766	8.16242393351629\\
24.3979747808993	8.16208454152101\\
24.3974512664708	8.16174513626383\\
24.3969277204872	8.16140571774371\\
24.3964041429447	8.16106628595961\\
24.3958805338396	8.1607268409105\\
24.395356893168	8.16038738259534\\
24.394833220926	8.16004791101309\\
24.39430951711	8.15970842616272\\
24.3937857817161	8.15936892804319\\
24.3932620147404	8.15902941665346\\
24.3927382161793	8.15868989199249\\
24.3922143860288	8.15835035405925\\
24.3916905242851	8.1580108028527\\
24.3911666309445	8.1576712383718\\
24.3906427060031	8.15733166061552\\
24.3901187494571	8.1569920695828\\
24.3895947613028	8.15665246527263\\
24.3890707415362	8.15631284768395\\
24.3885466901536	8.15597321681574\\
24.3880226071512	8.15563357266694\\
24.3874984925251	8.15529391523653\\
24.3869743462716	8.15495424452347\\
24.3864501683867	8.15461456052671\\
24.3859259588668	8.15427486324521\\
24.3854017177079	8.15393515267795\\
24.3848774449063	8.15359542882387\\
24.3843531404581	8.15325569168193\\
24.3838288043596	8.15291594125111\\
24.3833044366068	8.15257617753036\\
24.382780037196	8.15223640051864\\
24.3822556061234	8.15189661021491\\
24.381731143385	8.15155680661812\\
24.3812066489772	8.15121698972725\\
24.3806821228961	8.15087715954124\\
24.3801575651378	8.15053731605907\\
24.3796329756985	8.15019745927968\\
24.3791083545744	8.14985758920204\\
24.3785837017617	8.14951770582511\\
24.3780590172566	8.14917780914785\\
24.3775343010551	8.14883789916921\\
24.3770095531536	8.14849797588815\\
24.3764847735481	8.14815803930364\\
24.3759599622347	8.14781808941463\\
24.3754351192098	8.14747812622008\\
24.3749102444695	8.14713814971895\\
24.3743853380099	8.14679815991019\\
24.3738603998271	8.14645815679277\\
24.3733354299174	8.14611814036565\\
24.372810428277	8.14577811062777\\
24.3722853949019	8.1454380675781\\
24.3717603297883	8.1450980112156\\
24.3712352329325	8.14475794153922\\
24.3707101043306	8.14441785854793\\
24.3701849439786	8.14407776224067\\
24.3696597518729	8.14373765261641\\
24.3691345280095	8.1433975296741\\
24.3686092723846	8.14305739341271\\
24.3680839849944	8.14271724383118\\
24.3675586658351	8.14237708092847\\
24.3670333149027	8.14203690470354\\
24.3665079321934	8.14169671515535\\
24.3659825177035	8.14135651228285\\
24.3654570714289	8.14101629608501\\
24.364931593366	8.14067606656076\\
24.3644060835109	8.14033582370908\\
24.3638805418597	8.13999556752891\\
24.3633549684085	8.13965529801922\\
24.3628293631536	8.13931501517895\\
24.362303726091	8.13897471900708\\
24.3617780572169	8.13863440950253\\
24.3612523565275	8.13829408666428\\
24.3607266240189	8.13795375049129\\
24.3602008596873	8.13761340098249\\
24.3596750635287	8.13727303813686\\
24.3591492355395	8.13693266195334\\
24.3586233757156	8.13659227243089\\
24.3580974840533	8.13625186956846\\
24.3575715605487	8.135911453365\\
24.3570456051979	8.13557102381948\\
24.356519617997	8.13523058093085\\
24.3559935989423	8.13489012469805\\
24.3554675480299	8.13454965512005\\
24.3549414652559	8.1342091721958\\
24.3544153506164	8.13386867592425\\
24.3538892041076	8.13352816630435\\
24.3533630257256	8.13318764333506\\
24.3528368154665	8.13284710701533\\
24.3523105733266	8.13250655734411\\
24.3517842993019	8.13216599432036\\
24.3512579933886	8.13182541794304\\
24.3507316555827	8.13148482821108\\
24.3502052858805	8.13114422512345\\
24.3496788842781	8.1308036086791\\
24.3491524507716	8.13046297887698\\
24.3486259853571	8.13012233571604\\
24.3480994880308	8.12978167919524\\
24.3475729587888	8.12944100931352\\
24.3470463976273	8.12910032606984\\
24.3465198045422	8.12875962946316\\
24.3459931795299	8.12841891949241\\
24.3454665225864	8.12807819615656\\
24.3449398337079	8.12773745945456\\
24.3444131128904	8.12739670938535\\
24.3438863601301	8.12705594594789\\
24.3433595754231	8.12671516914113\\
24.3428327587656	8.12637437896401\\
24.3423059101537	8.1260335754155\\
24.3417790295834	8.12569275849454\\
24.341252117051	8.12535192820008\\
24.3407251725525	8.12501108453107\\
24.3401981960841	8.12467022748646\\
24.3396711876419	8.12432935706521\\
24.3391441472219	8.12398847326626\\
24.3386170748204	8.12364757608855\\
24.3380899704335	8.12330666553106\\
24.3375628340571	8.12296574159272\\
24.3370356656876	8.12262480427247\\
24.336508465321	8.12228385356928\\
24.3359812329533	8.12194288948209\\
24.3354539685808	8.12160191200984\\
24.3349266721995	8.1212609211515\\
24.3343993438056	8.120919916906\\
24.3338719833951	8.1205788992723\\
24.3333445909642	8.12023786824935\\
24.332817166509	8.11989682383608\\
24.3322897100255	8.11955576603146\\
24.33176222151	8.11921469483443\\
24.3312347009585	8.11887361024394\\
24.3307071483672	8.11853251225893\\
24.330179563732	8.11819140087836\\
24.3296519470492	8.11785027610117\\
24.3291242983148	8.11750913792631\\
24.328596617525	8.11716798635273\\
24.3280689046759	8.11682682137937\\
24.3275411597635	8.11648564300519\\
24.3270133827839	8.11614445122912\\
24.3264855737334	8.11580324605012\\
24.3259577326078	8.11546202746714\\
24.3254298594035	8.11512079547912\\
24.3249019541164	8.114779550085\\
24.3243740167427	8.11443829128373\\
24.3238460472785	8.11409701907427\\
24.3233180457199	8.11375573345555\\
24.3227900120629	8.11341443442652\\
24.3222619463036	8.11307312198614\\
24.3217338484383	8.11273179613334\\
24.3212057184628	8.11239045686706\\
24.3206775563735	8.11204910418627\\
24.3201493621662	8.11170773808989\\
24.3196211358373	8.11136635857689\\
24.3190928773826	8.11102496564619\\
24.3185645867984	8.11068355929676\\
24.3180362640806	8.11034213952752\\
24.3175079092255	8.11000070633744\\
24.3169795222291	8.10965925972545\\
24.3164511030874	8.10931779969049\\
24.3159226517967	8.10897632623152\\
24.3153941683529	8.10863483934747\\
24.3148656527521	8.10829333903729\\
24.3143371049905	8.10795182529993\\
24.3138085250641	8.10761029813433\\
24.313279912969	8.10726875753943\\
24.3127512687012	8.10692720351418\\
24.312222592257	8.10658563605752\\
24.3116938836323	8.10624405516839\\
24.3111651428232	8.10590246084574\\
24.3106363698259	8.10556085308851\\
24.3101075646364	8.10521923189564\\
24.3095787272507	8.10487759726609\\
24.309049857665	8.10453594919878\\
24.3085209558753	8.10419428769266\\
24.3079920218778	8.10385261274669\\
24.3074630556684	8.10351092435979\\
24.3069340572434	8.10316922253091\\
24.3064050265986	8.10282750725899\\
24.3058759637303	8.10248577854298\\
24.3053468686344	8.10214403638182\\
24.3048177413072	8.10180228077445\\
24.3042885817446	8.10146051171981\\
24.3037593899426	8.10111872921684\\
24.3032301658975	8.10077693326448\\
24.3027009096052	8.10043512386169\\
24.3021716210619	8.10009330100739\\
24.3016423002635	8.09975146470052\\
24.3011129472062	8.09940961494005\\
24.300583561886	8.09906775172488\\
24.300054144299	8.09872587505398\\
24.2995246944413	8.09838398492629\\
24.2989952123089	8.09804208134073\\
24.2984656978979	8.09770016429626\\
24.2979361512044	8.09735823379182\\
24.2974065722244	8.09701628982634\\
24.296876960954	8.09667433239876\\
24.2963473173892	8.09633236150803\\
24.2958176415261	8.09599037715308\\
24.2952879333608	8.09564837933285\\
24.2947581928893	8.09530636804629\\
24.2942284201077	8.09496434329233\\
24.293698615012	8.09462230506991\\
24.2931687775984	8.09428025337798\\
24.2926389078628	8.09393818821546\\
24.2921090058013	8.0935961095813\\
24.29157907141	8.09325401747444\\
24.2910491046849	8.09291191189382\\
24.2905191056221	8.09256979283837\\
24.2899890742176	8.09222766030704\\
24.2894590104675	8.09188551429876\\
24.2889289143679	8.09154335481247\\
24.2883987859147	8.09120118184712\\
24.2878686251041	8.09085899540162\\
24.2873384319321	8.09051679547493\\
24.2868082063947	8.09017458206599\\
24.286277948488	8.08983235517372\\
24.285747658208	8.08949011479707\\
24.2852173355508	8.08914786093497\\
24.2846869805125	8.08880559358637\\
24.284156593089	8.08846331275019\\
24.2836261732764	8.08812101842538\\
24.2830957210708	8.08777871061088\\
24.2825652364682	8.08743638930561\\
24.2820347194646	8.08709405450852\\
24.2815041700562	8.08675170621854\\
24.2809735882388	8.08640934443461\\
24.2804429740087	8.08606696915567\\
24.2799123273617	8.08572458038065\\
24.279381648294	8.08538217810848\\
24.2788509368016	8.0850397623381\\
24.2783201928805	8.08469733306846\\
24.2777894165267	8.08435489029848\\
24.2772586077364	8.0840124340271\\
24.2767277665055	8.08366996425325\\
24.27619689283	8.08332748097587\\
24.2756659867061	8.0829849841939\\
24.2751350481297	8.08264247390627\\
24.2746040770968	8.08229995011191\\
24.2740730736036	8.08195741280976\\
24.273542037646	8.08161486199876\\
24.27301096922	8.08127229767784\\
24.2724798683217	8.08092971984593\\
24.2719487349472	8.08058712850196\\
24.2714175690924	8.08024452364488\\
24.2708863707533	8.07990190527361\\
24.2703551399261	8.07955927338709\\
24.2698238766067	8.07921662798426\\
24.2692925807911	8.07887396906404\\
24.2687612524754	8.07853129662537\\
24.2682298916556	8.07818861066718\\
24.2676984983278	8.07784591118841\\
24.2671670724878	8.07750319818799\\
24.2666356141319	8.07716047166485\\
24.2661041232559	8.07681773161792\\
24.2655725998559	8.07647497804615\\
24.2650410439279	8.07613221094845\\
24.264509455468	8.07578943032377\\
24.2639778344722	8.07544663617103\\
24.2634461809364	8.07510382848916\\
24.2629144948567	8.07476100727711\\
24.2623827762291	8.0744181725338\\
24.2618510250497	8.07407532425816\\
24.2613192413144	8.07373246244912\\
24.2607874250192	8.07338958710562\\
24.2602555761602	8.07304669822659\\
24.2597236947334	8.07270379581096\\
24.2591917807347	8.07236087985765\\
24.2586598341603	8.07201795036561\\
24.258127855006	8.07167500733376\\
24.257595843268	8.07133205076103\\
24.2570637989422	8.07098908064636\\
24.2565317220247	8.07064609698867\\
24.2559996125113	8.07030309978689\\
24.2554674703983	8.06996008903996\\
24.2549352956814	8.0696170647468\\
24.2544030883568	8.06927402690635\\
24.2538708484205	8.06893097551753\\
24.2533385758685	8.06858791057928\\
24.2528062706967	8.06824483209052\\
24.2522739329011	8.06790174005018\\
24.2517415624779	8.06755863445719\\
24.2512091594229	8.06721551531049\\
24.2506767237321	8.066872382609\\
24.2501442554017	8.06652923635165\\
24.2496117544275	8.06618607653737\\
24.2490792208055	8.06584290316508\\
24.2485466545318	8.06549971623373\\
24.2480140556024	8.06515651574222\\
24.2474814240132	8.0648133016895\\
24.2469487597602	8.06447007407449\\
24.2464160628395	8.06412683289613\\
24.245883333247	8.06378357815333\\
24.2453505709788	8.06344030984502\\
24.2448177760307	8.06309702797014\\
24.2442849483988	8.06275373252761\\
24.2437520880792	8.06241042351636\\
24.2432191950677	8.06206710093531\\
24.2426862693603	8.0617237647834\\
24.2421533109532	8.06138041505954\\
24.2416203198422	8.06103705176267\\
24.2410872960233	8.06069367489172\\
24.2405542394925	8.06035028444561\\
24.2400211502458	8.06000688042326\\
24.2394880282792	8.05966346282361\\
24.2389548735887	8.05932003164558\\
24.2384216861702	8.05897658688809\\
24.2378884660197	8.05863312855008\\
24.2373552131333	8.05828965663046\\
24.2368219275068	8.05794617112817\\
24.2362886091364	8.05760267204213\\
24.2357552580178	8.05725915937126\\
24.2352218741472	8.0569156331145\\
24.2346884575205	8.05657209327076\\
24.2341550081337	8.05622853983897\\
24.2336215259828	8.05588497281806\\
24.2330880110636	8.05554139220695\\
24.2325544633723	8.05519779800456\\
24.2320208829048	8.05485419020983\\
24.231487269657	8.05451056882167\\
24.230953623625	8.05416693383901\\
24.2304199448046	8.05382328526077\\
24.2298862331919	8.05347962308588\\
24.2293524887828	8.05313594731327\\
24.2288187115734	8.05279225794185\\
24.2282849015595	8.05244855497055\\
24.2277510587372	8.05210483839829\\
24.2272171831024	8.051761108224\\
24.226683274651	8.05141736444661\\
24.2261493333791	8.05107360706502\\
24.2256153592826	8.05072983607817\\
24.2250813523574	8.05038605148498\\
24.2245473125996	8.05004225328438\\
24.2240132400051	8.04969844147528\\
24.2234791345699	8.04935461605661\\
24.2229449962898	8.04901077702729\\
24.222410825161	8.04866692438624\\
24.2218766211793	8.04832305813238\\
24.2213423843406	8.04797917826465\\
24.2208081146411	8.04763528478195\\
24.2202738120765	8.04729137768322\\
24.2197394766429	8.04694745696737\\
24.2192051083362	8.04660352263332\\
24.2186707071524	8.04625957468\\
24.2181362730875	8.04591561310632\\
24.2176018061373	8.04557163791122\\
24.2170673062979	8.0452276490936\\
24.2165327735651	8.0448836466524\\
24.215998207935	8.04453963058652\\
24.2154636094035	8.0441956008949\\
24.2149289779665	8.04385155757646\\
24.21439431362	8.0435075006301\\
24.21385961636	8.04316343005476\\
24.2133248861823	8.04281934584935\\
24.212790123083	8.0424752480128\\
24.212255327058	8.04213113654402\\
24.2117204981031	8.04178701144193\\
24.2111856362145	8.04144287270546\\
24.2106507413879	8.04109872033352\\
24.2101158136194	8.04075455432503\\
24.209580852905	8.04041037467892\\
24.2090458592404	8.04006618139409\\
24.2085108326217	8.03972197446948\\
24.2079757730449	8.03937775390399\\
24.2074406805058	8.03903351969655\\
24.2069055550004	8.03868927184608\\
24.2063703965246	8.03834501035149\\
24.2058352050745	8.03800073521171\\
24.2052999806458	8.03765644642564\\
24.2047647232345	8.03731214399222\\
24.2042294328367	8.03696782791035\\
24.2036941094481	8.03662349817896\\
24.2031587530648	8.03627915479696\\
24.2026233636827	8.03593479776328\\
24.2020879412977	8.03559042707682\\
24.2015524859057	8.0352460427365\\
24.2010169975027	8.03490164474125\\
24.2004814760847	8.03455723308998\\
24.1999459216474	8.03421280778161\\
24.1994103341869	8.03386836881505\\
24.198874713699	8.03352391618922\\
24.1983390601798	8.03317944990304\\
24.1978033736252	8.03283496995542\\
24.1972676540309	8.03249047634529\\
24.1967319013931	8.03214596907155\\
24.1961961157076	8.03180144813312\\
24.1956602969703	8.03145691352892\\
24.1951244451772	8.03111236525786\\
24.1945885603241	8.03076780331886\\
24.194052642407	8.03042322771084\\
24.1935166914219	8.03007863843271\\
24.1929807073646	8.02973403548338\\
24.192444690231	8.02938941886178\\
24.1919086400171	8.02904478856682\\
24.1913725567188	8.0287001445974\\
24.190836440332	8.02835548695245\\
24.1903002908526	8.02801081563088\\
24.1897641082766	8.02766613063161\\
24.1892278925998	8.02732143195354\\
24.1886916438181	8.0269767195956\\
24.1881553619275	8.0266319935567\\
24.1876190469239	8.02628725383575\\
24.1870826988032	8.02594250043166\\
24.1865463175613	8.02559773334336\\
24.1860099031941	8.02525295256975\\
24.1854734556975	8.02490815810974\\
24.1849369750675	8.02456334996226\\
24.1844004612999	8.02421852812621\\
24.1838639143906	8.0238736926005\\
24.1833273343355	8.02352884338406\\
24.1827907211307	8.02318398047579\\
24.1822540747718	8.0228391038746\\
24.181717395255	8.02249421357941\\
24.1811806825759	8.02214930958913\\
24.1806439367307	8.02180439190267\\
24.1801071577151	8.02145946051895\\
24.1795703455251	8.02111451543688\\
24.1790335001565	8.02076955665536\\
24.1784966216053	8.02042458417331\\
24.1779597098673	8.02007959798965\\
24.1774227649385	8.01973459810328\\
24.1768857868148	8.01938958451311\\
24.1763487754919	8.01904455721806\\
24.175811730966	8.01869951621704\\
24.1752746532327	8.01835446150896\\
24.1747375422881	8.01800939309273\\
24.174200398128	8.01766431096725\\
24.1736632207483	8.01731921513145\\
24.1731260101449	8.01697410558423\\
24.1725887663138	8.0166289823245\\
24.1720514892506	8.01628384535118\\
24.1715141789515	8.01593869466316\\
24.1709768354122	8.01559353025937\\
24.1704394586287	8.01524835213871\\
24.1699020485967	8.0149031603001\\
24.1693646053123	8.01455795474243\\
24.1688271287713	8.01421273546462\\
24.1682896189696	8.01386750246559\\
24.1677520759031	8.01352225574423\\
24.1672144995676	8.01317699529947\\
24.166676889959	8.0128317211302\\
24.1661392470733	8.01248643323534\\
24.1656015709062	8.01214113161379\\
24.1650638614537	8.01179581626447\\
24.1645261187117	8.01145048718628\\
24.1639883426759	8.01110514437812\\
24.1634505333424	8.01075978783892\\
24.162912690707	8.01041441756758\\
24.1623748147655	8.010069033563\\
24.1618369055139	8.00972363582409\\
24.161298962948	8.00937822434976\\
24.1607609870636	8.00903279913892\\
24.1602229778567	8.00868736019047\\
24.1596849353231	8.00834190750333\\
24.1591468594587	8.00799644107639\\
24.1586087502594	8.00765096090857\\
24.158070607721	8.00730546699878\\
24.1575324318394	8.00695995934591\\
24.1569942226105	8.00661443794888\\
24.1564559800302	8.0062689028066\\
24.1559177040942	8.00592335391796\\
24.1553793947985	8.00557779128188\\
24.1548410521389	8.00523221489726\\
24.1543026761113	8.00488662476301\\
24.1537642667116	8.00454102087803\\
24.1532258239356	8.00419540324123\\
24.1526873477792	8.00384977185152\\
24.1521488382383	8.00350412670779\\
24.1516102953086	8.00315846780897\\
24.1510717189861	8.00281279515394\\
24.1505331092667	8.00246710874162\\
24.1499944661461	8.0021214085709\\
24.1494557896203	8.00177569464071\\
24.148917079685	8.00142996694993\\
24.1483783363362	8.00108422549748\\
24.1478395595698	8.00073847028226\\
24.1473007493814	8.00039270130317\\
24.1467619057671	8.00004691855912\\
24.1462230287227	7.99970112204901\\
24.145684118244	7.99935531177174\\
24.1451451743268	7.99900948772622\\
24.1446061969671	7.99866364991136\\
24.1440671861606	7.99831779832605\\
24.1435281419032	7.9979719329692\\
24.1429890641909	7.99762605383972\\
24.1424499530193	7.9972801609365\\
24.1419108083844	7.99693425425845\\
24.1413716302819	7.99658833380447\\
24.1408324187079	7.99624239957346\\
24.140293173658	7.99589645156433\\
24.1397538951282	7.99555048977598\\
24.1392145831142	7.99520451420731\\
24.138675237612	7.99485852485723\\
24.1381358586173	7.99451252172463\\
24.137596446126	7.99416650480842\\
24.137057000134	7.99382047410749\\
24.136517520637	7.99347442962076\\
24.1359780076309	7.99312837134712\\
24.1354384611117	7.99278229928548\\
24.1348988810749	7.99243621343473\\
24.1343592675167	7.99209011379377\\
24.1338196204326	7.99174400036151\\
24.1332799398187	7.99139787313686\\
24.1327402256707	7.9910517321187\\
24.1322004779845	7.99070557730593\\
24.1316606967558	7.99035940869747\\
24.1311208819806	7.99001322629221\\
24.1305810336546	7.98966703008905\\
24.1300411517737	7.98932082008689\\
24.1295012363337	7.98897459628463\\
24.1289612873305	7.98862835868117\\
24.1284213047598	7.98828210727541\\
24.1278812886175	7.98793584206624\\
24.1273412388994	7.98758956305258\\
24.1268011556013	7.98724327023332\\
24.1262610387192	7.98689696360735\\
24.1257208882486	7.98655064317358\\
24.1251807041857	7.9862043089309\\
24.124640486526	7.98585796087822\\
24.1241002352655	7.98551159901442\\
24.1235599503999	7.98516522333842\\
24.1230196319252	7.98481883384911\\
24.122479279837	7.98447243054539\\
24.1219388941313	7.98412601342615\\
24.1213984748039	7.9837795824903\\
24.1208580218504	7.98343313773672\\
24.1203175352669	7.98308667916433\\
24.119777015049	7.98274020677201\\
24.1192364611927	7.98239372055867\\
24.1186958736936	7.98204722052319\\
24.1181552525477	7.98170070666449\\
24.1176145977507	7.98135417898145\\
24.1170739092985	7.98100763747297\\
24.1165331871868	7.98066108213796\\
24.1159924314115	7.98031451297529\\
24.1154516419684	7.97996792998388\\
24.1149108188532	7.97962133316262\\
24.1143699620619	7.97927472251041\\
24.1138290715901	7.97892809802613\\
24.1132881474337	7.97858145970869\\
24.1127471895886	7.97823480755699\\
24.1122061980504	7.97788814156991\\
24.1116651728151	7.97754146174636\\
24.1111241138783	7.97719476808522\\
24.110583021236	7.9768480605854\\
24.1100418948839	7.97650133924579\\
24.1095007348178	7.97615460406529\\
24.1089595410335	7.97580785504278\\
24.1084183135269	7.97546109217718\\
24.1078770522936	7.97511431546735\\
24.1073357573295	7.97476752491222\\
24.1067944286305	7.97442072051066\\
24.1062530661922	7.97407390226158\\
24.1057116700105	7.97372707016386\\
24.1051702400812	7.9733802242164\\
24.1046287764001	7.9730333644181\\
24.1040872789629	7.97268649076784\\
24.1035457477655	7.97233960326453\\
24.1030041828036	7.97199270190705\\
24.1024625840731	7.9716457866943\\
24.1019209515697	7.97129885762517\\
24.1013792852892	7.97095191469856\\
24.1008375852274	7.97060495791335\\
24.1002958513801	7.97025798726845\\
24.099754083743	7.96991100276274\\
24.099212282312	7.96956400439511\\
24.0986704470829	7.96921699216446\\
24.0981285780514	7.96886996606969\\
24.0975866752132	7.96852292610967\\
24.0970447385643	7.96817587228331\\
24.0965027681003	7.9678288045895\\
24.0959607638171	7.96748172302712\\
24.0954187257104	7.96713462759508\\
24.094876653776	7.96678751829225\\
24.0943345480097	7.96644039511754\\
24.0937924084072	7.96609325806983\\
24.0932502349644	7.96574610714802\\
24.092708027677	7.96539894235099\\
24.0921657865407	7.96505176367764\\
24.0916235115514	7.96470457112686\\
24.0910812027049	7.96435736469753\\
24.0905388599968	7.96401014438856\\
24.089996483423	7.96366291019882\\
24.0894540729793	7.96331566212721\\
24.0889116286614	7.96296840017262\\
24.088369150465	7.96262112433393\\
24.087826638386	7.96227383461005\\
24.0872840924201	7.96192653099985\\
24.0867415125631	7.96157921350223\\
24.0861988988107	7.96123188211608\\
24.0856562511588	7.96088453684028\\
24.085113569603	7.96053717767373\\
24.0845708541392	7.96018980461531\\
24.0840281047631	7.95984241766392\\
24.0834853214704	7.95949501681843\\
24.0829425042569	7.95914760207775\\
24.0823996531185	7.95880017344075\\
24.0818567680508	7.95845273090633\\
24.0813138490496	7.95810527447338\\
24.0807708961106	7.95775780414077\\
24.0802279092297	7.95741031990741\\
24.0796848884025	7.95706282177218\\
24.0791418336249	7.95671530973396\\
24.0785987448925	7.95636778379165\\
24.0780556222012	7.95602024394412\\
24.0775124655467	7.95567269019028\\
24.0769692749247	7.955325122529\\
24.076426050331	7.95497754095917\\
24.0758827917613	7.95462994547968\\
24.0753394992115	7.95428233608942\\
24.0747961726772	7.95393471278727\\
24.0742528121541	7.95358707557211\\
24.0737094176381	7.95323942444284\\
24.0731659891249	7.95289175939834\\
24.0726225266102	7.9525440804375\\
24.0720790300898	7.9521963875592\\
24.0715354995595	7.95184868076233\\
24.0709919350148	7.95150096004578\\
24.0704483364517	7.95115322540842\\
24.0699047038658	7.95080547684915\\
24.0693610372529	7.95045771436684\\
24.0688173366088	7.9501099379604\\
24.0682736019291	7.94976214762869\\
24.0677298332096	7.94941434337061\\
24.067186030446	7.94906652518503\\
24.0666421936341	7.94871869307086\\
24.0660983227697	7.94837084702695\\
24.0655544178483	7.94802298705222\\
24.0650104788659	7.94767511314552\\
24.0644665058181	7.94732722530576\\
24.0639224987006	7.94697932353182\\
24.0633784575093	7.94663140782257\\
24.0628343822397	7.9462834781769\\
24.0622902728877	7.9459355345937\\
24.061746129449	7.94558757707185\\
24.0612019519192	7.94523960561023\\
24.0606577402942	7.94489162020773\\
24.0601134945697	7.94454362086322\\
24.0595692147413	7.94419560757559\\
24.0590249008049	7.94384758034373\\
24.058480552756	7.94349953916651\\
24.0579361705906	7.94315148404282\\
24.0573917543042	7.94280341497154\\
24.0568473038926	7.94245533195156\\
24.0563028193516	7.94210723498175\\
24.0557583006768	7.94175912406099\\
24.055213747864	7.94141099918817\\
24.0546691609089	7.94106286036217\\
24.0541245398071	7.94071470758188\\
24.0535798845545	7.94036654084616\\
24.0530351951468	7.94001836015391\\
24.0524904715796	7.93967016550401\\
24.0519457138486	7.93932195689533\\
24.0514009219497	7.93897373432675\\
24.0508560958784	7.93862549779717\\
24.0503112356306	7.93827724730545\\
24.0497663412019	7.93792898285048\\
24.049221412588	7.93758070443113\\
24.0486764497847	7.93723241204629\\
24.0481314527876	7.93688410569485\\
24.0475864215925	7.93653578537567\\
24.047041356195	7.93618745108764\\
24.046496256591	7.93583910282963\\
24.045951122776	7.93549074060053\\
24.0454059547458	7.93514236439922\\
24.0448607524961	7.93479397422457\\
24.0443155160226	7.93444557007547\\
24.043770245321	7.93409715195079\\
24.0432249403871	7.93374871984941\\
24.0426796012164	7.93340027377021\\
24.0421342278047	7.93305181371207\\
24.0415888201478	7.93270333967387\\
24.0410433782412	7.93235485165448\\
24.0404979020808	7.93200634965278\\
24.0399523916621	7.93165783366766\\
24.039406846981	7.93130930369799\\
24.038861268033	7.93096075974264\\
24.0383156548139	7.9306122018005\\
24.0377700073195	7.93026362987043\\
24.0372243255452	7.92991504395133\\
24.036678609487	7.92956644404207\\
24.0361328591404	7.92921783014151\\
24.0355870745012	7.92886920224855\\
24.035041255565	7.92852056036205\\
24.0344954023276	7.9281719044809\\
24.0339495147846	7.92782323460397\\
24.0334035929316	7.92747455073013\\
24.0328576367645	7.92712585285826\\
24.0323116462789	7.92677714098725\\
24.0317656214704	7.92642841511595\\
24.0312195623347	7.92607967524326\\
24.0306734688676	7.92573092136804\\
24.0301273410648	7.92538215348918\\
24.0295811789218	7.92503337160554\\
24.0290349824344	7.924684575716\\
24.0284887515982	7.92433576581944\\
24.027942486409	7.92398694191473\\
24.0273961868624	7.92363810400074\\
24.0268498529542	7.92328925207636\\
24.0263034846798	7.92294038614045\\
24.0257570820352	7.9225915061919\\
24.0252106450158	7.92224261222957\\
24.0246641736175	7.92189370425233\\
24.0241176678358	7.92154478225907\\
24.0235711276665	7.92119584624866\\
24.0230245531052	7.92084689621996\\
24.0224779441476	7.92049793217186\\
24.0219313007893	7.92014895410323\\
24.0213846230261	7.91979996201294\\
24.0208379108536	7.91945095589986\\
24.0202911642674	7.91910193576287\\
24.0197443832633	7.91875290160084\\
24.0191975678369	7.91840385341264\\
24.0186507179839	7.91805479119715\\
24.0181038336998	7.91770571495323\\
24.0175569149805	7.91735662467976\\
24.0170099618216	7.91700752037562\\
24.0164629742186	7.91665840203967\\
24.0159159521673	7.91630926967079\\
24.0153688956634	7.91596012326784\\
24.0148218047025	7.91561096282971\\
24.0142746792803	7.91526178835526\\
24.0137275193923	7.91491259984336\\
24.0131803250344	7.91456339729289\\
24.0126330962021	7.91421418070271\\
24.012085832891	7.9138649500717\\
24.011538535097	7.91351570539872\\
24.0109912028155	7.91316644668265\\
24.0104438360423	7.91281717392236\\
24.0098964347729	7.91246788711672\\
24.0093489990032	7.9121185862646\\
24.0088015287286	7.91176927136487\\
24.0082540239449	7.9114199424164\\
24.0077064846477	7.91107059941806\\
24.0071589108327	7.91072124236872\\
24.0066113024955	7.91037187126725\\
24.0060636596317	7.91002248611252\\
24.005515982237	7.90967308690339\\
24.0049682703071	7.90932367363875\\
24.0044205238375	7.90897424631745\\
24.003872742824	7.90862480493836\\
24.0033249272621	7.90827534950036\\
24.0027770771476	7.90792588000231\\
24.002229192476	7.90757639644308\\
24.001681273243	7.90722689882155\\
24.0011333194443	7.90687738713657\\
24.0005853310754	7.90652786138702\\
24.000037308132	7.90617832157176\\
23.9994892506098	7.90582876768966\\
23.9989411585044	7.9054791997396\\
23.9983930318114	7.90512961772043\\
23.9978448705264	7.90478002163103\\
23.9972966746451	7.90443041147026\\
23.9967484441631	7.90408078723699\\
23.9962001790761	7.90373114893009\\
23.9956518793797	7.90338149654842\\
23.9951035450695	7.90303183009085\\
23.9945551761411	7.90268214955625\\
23.9940067725902	7.90233245494348\\
23.9934583344124	7.90198274625141\\
23.9929098616033	7.90163302347891\\
23.9923613541586	7.90128328662484\\
23.9918128120739	7.90093353568807\\
23.9912642353447	7.90058377066746\\
23.9907156239668	7.90023399156189\\
23.9901669779357	7.8998841983702\\
23.9896182972471	7.89953439109128\\
23.9890695818966	7.89918456972399\\
23.9885208318798	7.89883473426719\\
23.9879720471923	7.89848488471974\\
23.9874232278298	7.89813502108052\\
23.9868743737879	7.89778514334837\\
23.9863254850621	7.89743525152219\\
23.9857765616481	7.89708534560081\\
23.9852276035416	7.89673542558312\\
23.984678610738	7.89638549146796\\
23.9841295832332	7.89603554325422\\
23.9835805210225	7.89568558094075\\
23.9830314241018	7.89533560452641\\
23.9824822924665	7.89498561401007\\
23.9819331261123	7.89463560939059\\
23.9813839250348	7.89428559066684\\
23.9808346892297	7.89393555783768\\
23.9802854186924	7.89358551090197\\
23.9797361134187	7.89323544985857\\
23.9791867734041	7.89288537470636\\
23.9786373986442	7.89253528544418\\
23.9780879891346	7.89218518207091\\
23.9775385448711	7.8918350645854\\
23.976989065849	7.89148493298652\\
23.9764395520641	7.89113478727313\\
23.975890003512	7.89078462744409\\
23.9753404201882	7.89043445349826\\
23.9747908020883	7.89008426543451\\
23.974241149208	7.8897340632517\\
23.9736914615429	7.88938384694868\\
23.9731417390885	7.88903361652432\\
23.9725919818404	7.88868337197749\\
23.9720421897942	7.88833311330704\\
23.9714923629456	7.88798284051182\\
23.9709425012901	7.88763255359072\\
23.9703926048234	7.88728225254257\\
23.9698426735409	7.88693193736625\\
23.9692927074383	7.88658160806062\\
23.9687427065112	7.88623126462453\\
23.9681926707552	7.88588090705685\\
23.9676426001659	7.88553053535643\\
23.9670924947388	7.88518014952213\\
23.9665423544696	7.88482974955282\\
23.9659921793537	7.88447933544736\\
23.9654419693869	7.88412890720459\\
23.9648917245647	7.88377846482339\\
23.9643414448827	7.88342800830262\\
23.9637911303364	7.88307753764112\\
23.9632407809215	7.88272705283776\\
23.9626903966335	7.8823765538914\\
23.962139977468	7.88202604080089\\
23.9615895234206	7.8816755135651\\
23.9610390344869	7.88132497218289\\
23.9604885106624	7.8809744166531\\
23.9599379519427	7.88062384697461\\
23.9593873583234	7.88027326314626\\
23.9588367298001	7.87992266516691\\
23.9582860663684	7.87957205303543\\
23.9577353680237	7.87922142675067\\
23.9571846347618	7.87887078631149\\
23.9566338665781	7.87852013171674\\
23.9560830634682	7.87816946296528\\
23.9555322254277	7.87781878005597\\
23.9549813524522	7.87746808298767\\
23.9544304445373	7.87711737175922\\
23.9538795016785	7.8767666463695\\
23.9533285238713	7.87641590681734\\
23.9527775111114	7.87606515310162\\
23.9522264633943	7.87571438522119\\
23.9516753807155	7.8753636031749\\
23.9511242630707	7.8750128069616\\
23.9505731104554	7.87466199658017\\
23.9500219228652	7.87431117202944\\
23.9494707002955	7.87396033330827\\
23.9489194427421	7.87360948041553\\
23.9483681502004	7.87325861335006\\
23.947816822666	7.87290773211072\\
23.9472654601344	7.87255683669636\\
23.9467140626012	7.87220592710585\\
23.946162630062	7.87185500333802\\
23.9456111625124	7.87150406539175\\
23.9450596599478	7.87115311326588\\
23.9445081223638	7.87080214695926\\
23.9439565497561	7.87045116647076\\
23.94340494212	7.87010017179922\\
23.9428532994513	7.8697491629435\\
23.9423016217453	7.86939813990245\\
23.9417499089978	7.86904710267493\\
23.9411981612042	7.86869605125978\\
23.94064637836	7.86834498565586\\
23.9400945604609	7.86799390586204\\
23.9395427075024	7.86764281187714\\
23.9389908194799	7.86729170370004\\
23.9384388963892	7.86694058132958\\
23.9378869382256	7.86658944476462\\
23.9373349449848	7.866238294004\\
23.9367829166622	7.86588712904658\\
23.9362308532535	7.86553594989122\\
23.9356787547542	7.86518475653675\\
23.9351266211597	7.86483354898205\\
23.9345744524657	7.86448232722595\\
23.9340222486677	7.8641310912673\\
23.9334700097612	7.86377984110497\\
23.9329177357417	7.8634285767378\\
23.9323654266049	7.86307729816464\\
23.9318130823461	7.86272600538434\\
23.931260702961	7.86237469839575\\
23.9307082884451	7.86202337719774\\
23.9301558387939	7.86167204178913\\
23.929603354003	7.86132069216879\\
23.9290508340678	7.86096932833556\\
23.9284982789839	7.8606179502883\\
23.9279456887469	7.86026655802586\\
23.9273930633522	7.85991515154708\\
23.9268404027954	7.85956373085081\\
23.926287707072	7.85921229593591\\
23.9257349761775	7.85886084680122\\
23.9251822101075	7.8585093834456\\
23.9246294088575	7.85815790586788\\
23.924076572423	7.85780641406693\\
23.9235237007995	7.85745490804159\\
23.9229707939825	7.8571033877907\\
23.9224178519676	7.85675185331312\\
23.9218648747503	7.8564003046077\\
23.9213118623261	7.85604874167328\\
23.9207588146905	7.85569716450871\\
23.920205731839	7.85534557311284\\
23.9196526137672	7.85499396748452\\
23.9190994604705	7.85464234762259\\
23.9185462719445	7.85429071352591\\
23.9179930481847	7.85393906519331\\
23.9174397891866	7.85358740262366\\
23.9168864949456	7.85323572581578\\
23.9163331654574	7.85288403476854\\
23.9157798007175	7.85253232948078\\
23.9152264007212	7.85218060995134\\
23.9146729654642	7.85182887617907\\
23.914119494942	7.85147712816282\\
23.91356598915	7.85112536590144\\
23.9130124480837	7.85077358939376\\
23.9124588717387	7.85042179863864\\
23.9119052601105	7.85006999363493\\
23.9113516131946	7.84971817438146\\
23.9107979309864	7.84936634087709\\
23.9102442134814	7.84901449312065\\
23.9096904606753	7.848662631111\\
23.9091366725634	7.84831075484697\\
23.9085828491412	7.84795886432743\\
23.9080289904044	7.8476069595512\\
23.9074750963483	7.84725504051714\\
23.9069211669684	7.84690310722408\\
23.9063672022603	7.84655115967088\\
23.9058132022194	7.84619919785637\\
23.9052591668412	7.84584722177941\\
23.9047050961213	7.84549523143883\\
23.9041509900551	7.84514322683348\\
23.9035968486381	7.84479120796221\\
23.9030426718657	7.84443917482386\\
23.9024884597336	7.84408712741726\\
23.9019342122371	7.84373506574126\\
23.9013799293718	7.84338298979472\\
23.9008256111331	7.84303089957646\\
23.9002712575166	7.84267879508534\\
23.8997168685176	7.84232667632019\\
23.8991624441318	7.84197454327987\\
23.8986079843545	7.8416223959632\\
23.8980534891813	7.84127023436904\\
23.8974989586076	7.84091805849622\\
23.8969443926289	7.84056586834359\\
23.8963897912407	7.84021366390999\\
23.8958351544385	7.83986144519426\\
23.8952804822177	7.83950921219524\\
23.8947257745739	7.83915696491178\\
23.8941710315025	7.83880470334271\\
23.8936162529989	7.83845242748688\\
23.8930614390587	7.83810013734313\\
23.8925065896773	7.8377478329103\\
23.8919517048502	7.83739551418723\\
23.8913967845729	7.83704318117275\\
23.8908418288407	7.83669083386572\\
23.8902868376493	7.83633847226498\\
23.8897318109941	7.83598609636935\\
23.8891767488705	7.83563370617768\\
23.888621651274	7.83528130168882\\
23.8880665182001	7.8349288829016\\
23.8875113496442	7.83457644981486\\
23.8869561456019	7.83422400242744\\
23.8864009060684	7.83387154073818\\
23.8858456310395	7.83351906474592\\
23.8852903205103	7.8331665744495\\
23.8847349744766	7.83281406984776\\
23.8841795929336	7.83246155093953\\
23.8836241758769	7.83210901772366\\
23.883068723302	7.83175647019898\\
23.8825132352042	7.83140390836433\\
23.881957711579	7.83105133221856\\
23.881402152422	7.83069874176049\\
23.8808465577285	7.83034613698897\\
23.880290927494	7.82999351790283\\
23.8797352617139	7.82964088450092\\
23.8791795603837	7.82928823678206\\
23.8786238234989	7.8289355747451\\
23.8780680510549	7.82858289838888\\
23.8775122430472	7.82823020771223\\
23.8769563994711	7.82787750271399\\
23.8764005203223	7.82752478339298\\
23.875844605596	7.82717204974807\\
23.8752886552877	7.82681930177807\\
23.8747326693929	7.82646653948183\\
23.8741766479071	7.82611376285817\\
23.8736205908256	7.82576097190595\\
23.873064498144	7.82540816662399\\
23.8725083698576	7.82505534701112\\
23.8719522059619	7.8247025130662\\
23.8713960064523	7.82434966478804\\
23.8708397713243	7.82399680217549\\
23.8702835005734	7.82364392522738\\
23.8697271941949	7.82329103394254\\
23.8691708521842	7.82293812831982\\
23.8686144745369	7.82258520835804\\
23.8680580612484	7.82223227405605\\
23.867501612314	7.82187932541267\\
23.8669451277292	7.82152636242674\\
23.8663886074895	7.82117338509709\\
23.8658320515903	7.82082039342257\\
23.865275460027	7.82046738740199\\
23.8647188327951	7.8201143670342\\
23.8641621698899	7.81976133231803\\
23.8636054713069	7.81940828325232\\
23.8630487370415	7.81905521983589\\
23.8624919670892	7.81870214206758\\
23.8619351614454	7.81834904994623\\
23.8613783201055	7.81799594347066\\
23.8608214430648	7.81764282263971\\
23.860264530319	7.81728968745221\\
23.8597075818633	7.816936537907\\
23.8591505976931	7.81658337400291\\
23.858593577804	7.81623019573876\\
23.8580365221913	7.8158770031134\\
23.8574794308504	7.81552379612565\\
23.8569223037768	7.81517057477434\\
23.8563651409659	7.81481733905832\\
23.8558079424131	7.8144640889764\\
23.8552507081137	7.81411082452742\\
23.8546934380633	7.81375754571021\\
23.8541361322572	7.8134042525236\\
23.8535787906908	7.81305094496644\\
23.8530214133596	7.81269762303753\\
23.8524640002589	7.81234428673572\\
23.8519065513842	7.81199093605983\\
23.8513490667309	7.8116375710087\\
23.8507915462944	7.81128419158116\\
23.85023399007	7.81093079777604\\
23.8496763980533	7.81057738959216\\
23.8491187702395	7.81022396702836\\
23.8485611066241	7.80987053008346\\
23.8480034072026	7.80951707875631\\
23.8474456719702	7.80916361304571\\
23.8468879009225	7.80881013295051\\
23.8463300940547	7.80845663846954\\
23.8457722513624	7.80810312960162\\
23.8452143728409	7.80774960634558\\
23.8446564584855	7.80739606870025\\
23.8440985082918	7.80704251666446\\
23.843540522255	7.80668895023704\\
23.8429825003707	7.80633536941681\\
23.8424244426341	7.80598177420261\\
23.8418663490407	7.80562816459326\\
23.8413082195858	7.80527454058759\\
23.840750054265	7.80492090218443\\
23.8401918530734	7.8045672493826\\
23.8396336160066	7.80421358218094\\
23.8390753430599	7.80385990057826\\
23.8385170342287	7.80350620457341\\
23.8379586895084	7.80315249416519\\
23.8374003088944	7.80279876935245\\
23.8368418923821	7.80244503013401\\
23.8362834399668	7.80209127650869\\
23.835724951644	7.80173750847532\\
23.835166427409	7.80138372603273\\
23.8346078672571	7.80102992917974\\
23.8340492711839	7.80067611791519\\
23.8334906391846	7.80032229223788\\
23.8329319712546	7.79996845214666\\
23.8323732673894	7.79961459764035\\
23.8318145275842	7.79926072871776\\
23.8312557518345	7.79890684537774\\
23.8306969401357	7.79855294761909\\
23.830138092483	7.79819903544066\\
23.829579208872	7.79784510884125\\
23.8290202892979	7.79749116781971\\
23.8284613337562	7.79713721237484\\
23.8279023422421	7.79678324250548\\
23.8273433147511	7.79642925821045\\
23.8267842512786	7.79607525948858\\
23.8262251518199	7.79572124633868\\
23.8256660163703	7.79536721875959\\
23.8251068449253	7.79501317675012\\
23.8245476374802	7.79465912030911\\
23.8239883940304	7.79430504943536\\
23.8234291145712	7.79395096412771\\
23.822869799098	7.79359686438498\\
23.8223104476061	7.793242750206\\
23.821751060091	7.79288862158958\\
23.821191636548	7.79253447853454\\
23.8206321769724	7.79218032103972\\
23.8200726813595	7.79182614910393\\
23.8195131497049	7.791471962726\\
23.8189535820037	7.79111776190474\\
23.8183939782515	7.79076354663898\\
23.8178343384434	7.79040931692755\\
23.8172746625749	7.79005507276925\\
23.8167149506413	7.78970081416293\\
23.816155202638	7.78934654110738\\
23.8155954185604	7.78899225360145\\
23.8150355984037	7.78863795164394\\
23.8144757421633	7.78828363523368\\
23.8139158498346	7.78792930436949\\
23.813355921413	7.7875749590502\\
23.8127959568937	7.78722059927461\\
23.8122359562721	7.78686622504156\\
23.8116759195436	7.78651183634986\\
23.8111158467034	7.78615743319833\\
23.8105557377471	7.78580301558579\\
23.8099955926698	7.78544858351106\\
23.8094354114669	7.78509413697297\\
23.8088751941338	7.78473967597033\\
23.8083149406658	7.78438520050195\\
23.8077546510583	7.78403071056667\\
23.8071943253065	7.78367620616329\\
23.8066339634058	7.78332168729065\\
23.8060735653516	7.78296715394755\\
23.8055131311392	7.78261260613282\\
23.8049526607639	7.78225804384526\\
23.8043921542211	7.78190346708372\\
23.803831611506	7.78154887584699\\
23.8032710326141	7.7811942701339\\
23.8027104175406	7.78083964994326\\
23.8021497662809	7.7804850152739\\
23.8015890788303	7.78013036612464\\
23.8010283551841	7.77977570249428\\
23.8004675953377	7.77942102438165\\
23.7999067992864	7.77906633178556\\
23.7993459670254	7.77871162470484\\
23.7987850985503	7.77835690313829\\
23.7982241938561	7.77800216708474\\
23.7976632529383	7.77764741654299\\
23.7971022757923	7.77729265151188\\
23.7965412624132	7.77693787199021\\
23.7959802127965	7.7765830779768\\
23.7954191269375	7.77622826947047\\
23.7948580048314	7.77587344647003\\
23.7942968464736	7.7755186089743\\
23.7937356518594	7.77516375698209\\
23.7931744209841	7.77480889049222\\
23.7926131538431	7.77445400950351\\
23.7920518504316	7.77409911401476\\
23.791490510745	7.7737442040248\\
23.7909291347785	7.77338927953244\\
23.7903677225275	7.7730343405365\\
23.7898062739874	7.77267938703578\\
23.7892447891533	7.77232441902911\\
23.7886832680206	7.7719694365153\\
23.7881217105846	7.77161443949315\\
23.7875601168406	7.7712594279615\\
23.786998486784	7.77090440191914\\
23.7864368204099	7.7705493613649\\
23.7858751177138	7.77019430629758\\
23.785313378691	7.76983923671601\\
23.7847516033366	7.76948415261899\\
23.7841897916461	7.76912905400533\\
23.7836279436147	7.76877394087386\\
23.7830660592378	7.76841881322338\\
23.7825041385106	7.7680636710527\\
23.7819421814284	7.76770851436065\\
23.7813801879865	7.76735334314602\\
23.7808181581802	7.76699815740764\\
23.7802560920048	7.76664295714431\\
23.7796939894557	7.76628774235485\\
23.779131850528	7.76593251303807\\
23.7785696752171	7.76557726919278\\
23.7780074635182	7.76522201081779\\
23.7774452154268	7.76486673791192\\
23.7768829309379	7.76451145047397\\
23.7763206100471	7.76415614850275\\
23.7757582527494	7.76380083199709\\
23.7751958590402	7.76344550095578\\
23.7746334289149	7.76309015537764\\
23.7740709623686	7.76273479526147\\
23.7735084593966	7.7623794206061\\
23.7729459199943	7.76202403141033\\
23.772383344157	7.76166862767297\\
23.7718207318798	7.76131320939283\\
23.7712580831581	7.76095777656871\\
23.7706953979872	7.76060232919944\\
23.7701326763623	7.76024686728381\\
23.7695699182787	7.75989139082064\\
23.7690071237317	7.75953589980874\\
23.7684442927166	7.75918039424692\\
23.7678814252286	7.75882487413398\\
23.767318521263	7.75846933946873\\
23.7667555808151	7.75811379024999\\
23.7661926038801	7.75775822647656\\
23.7656295904533	7.75740264814725\\
23.7650665405301	7.75704705526086\\
23.7645034541056	7.75669144781621\\
23.7639403311751	7.75633582581211\\
23.7633771717339	7.75598018924735\\
23.7628139757773	7.75562453812076\\
23.7622507433005	7.75526887243113\\
23.7616874742988	7.75491319217727\\
23.7611241687674	7.754557497358\\
23.7605608267016	7.75420178797211\\
23.7599974480967	7.75384606401842\\
23.7594340329479	7.75349032549573\\
23.7588705812505	7.75313457240284\\
23.7583070929998	7.75277880473858\\
23.7577435681909	7.75242302250173\\
23.7571800068193	7.75206722569111\\
23.75661640888	7.75171141430552\\
23.7560527743684	7.75135558834377\\
23.7554891032797	7.75099974780467\\
23.7549253956092	7.75064389268701\\
23.7543616513521	7.75028802298961\\
23.7537978705037	7.74993213871127\\
23.7532340530592	7.7495762398508\\
23.7526701990139	7.749220326407\\
23.752106308363	7.74886439837867\\
23.7515423811017	7.74850845576462\\
23.7509784172254	7.74815249856366\\
23.7504144167293	7.74779652677459\\
23.7498503796085	7.74744054039621\\
23.7492863058584	7.74708453942733\\
23.7487221954742	7.74672852386674\\
23.7481580484511	7.74637249371327\\
23.7475938647843	7.7460164489657\\
23.7470296444692	7.74566038962284\\
23.7464653875009	7.7453043156835\\
23.7459010938748	7.74494822714648\\
23.7453367635859	7.74459212401059\\
23.7447723966296	7.74423600627461\\
23.7442079930011	7.74387987393736\\
23.7436435526956	7.74352372699765\\
23.7430790757083	7.74316756545426\\
23.7425145620346	7.74281138930601\\
23.7419500116695	7.7424551985517\\
23.7413854246085	7.74209899319013\\
23.7408208008466	7.74174277322009\\
23.7402561403791	7.74138653864041\\
23.7396914432012	7.74103028944986\\
23.7391267093082	7.74067402564726\\
23.7385619386953	7.74031774723141\\
23.7379971313578	7.7399614542011\\
23.7374322872907	7.73960514655515\\
23.7368674064894	7.73924882429234\\
23.7363024889491	7.73889248741149\\
23.735737534665	7.73853613591138\\
23.7351725436323	7.73817976979083\\
23.7346075158462	7.73782338904863\\
23.734042451302	7.73746699368358\\
23.7334773499949	7.73711058369448\\
23.7329122119201	7.73675415908014\\
23.7323470370727	7.73639771983935\\
23.7317818254481	7.7360412659709\\
23.7312165770415	7.73568479747361\\
23.7306512918479	7.73532831434627\\
23.7300859698628	7.73497181658767\\
23.7295206110812	7.73461530419663\\
23.7289552154983	7.73425877717192\\
23.7283897831095	7.73390223551237\\
23.7278243139099	7.73354567921675\\
23.7272588078946	7.73318910828388\\
23.726693265059	7.73283252271255\\
23.7261276853982	7.73247592250156\\
23.7255620689073	7.7321193076497\\
23.7249964155818	7.73176267815577\\
23.7244307254166	7.73140603401858\\
23.723864998407	7.73104937523691\\
23.7232992345483	7.73069270180957\\
23.7227334338356	7.73033601373536\\
23.7221675962641	7.72997931101306\\
23.721601721829	7.72962259364148\\
23.7210358105256	7.72926586161941\\
23.7204698623489	7.72890911494566\\
23.7199038772943	7.72855235361901\\
23.7193378553568	7.72819557763826\\
23.7187717965318	7.72783878700221\\
23.7182057008143	7.72748198170966\\
23.7176395681996	7.7271251617594\\
23.7170733986828	7.72676832715022\\
23.7165071922592	7.72641147788092\\
23.715940948924	7.7260546139503\\
23.7153746686723	7.72569773535716\\
23.7148083514992	7.72534084210028\\
23.7142419974001	7.72498393417846\\
23.7136756063701	7.7246270115905\\
23.7131091784043	7.7242700743352\\
23.712542713498	7.72391312241133\\
23.7119762116463	7.72355615581771\\
23.7114096728444	7.72319917455313\\
23.7108430970875	7.72284217861637\\
23.7102764843708	7.72248516800623\\
23.7097098346894	7.72212814272151\\
23.7091431480385	7.721771102761\\
23.7085764244133	7.7214140481235\\
23.7080096638091	7.72105697880779\\
23.7074428662208	7.72069989481267\\
23.7068760316438	7.72034279613693\\
23.7063091600731	7.71998568277937\\
23.705742251504	7.71962855473878\\
23.7051753059317	7.71927141201394\\
23.7046083233513	7.71891425460367\\
23.7040413037579	7.71855708250673\\
23.7034742471467	7.71819989572194\\
23.702907153513	7.71784269424807\\
23.7023400228518	7.71748547808392\\
23.7017728551584	7.71712824722829\\
23.7012056504279	7.71677100167996\\
23.7006384086554	7.71641374143773\\
23.7000711298361	7.71605646650038\\
23.6995038139652	7.71569917686671\\
23.6989364610379	7.71534187253551\\
23.6983690710492	7.71498455350557\\
23.6978016439944	7.71462721977569\\
23.6972341798687	7.71426987134464\\
23.6966666786671	7.71391250821122\\
23.6960991403848	7.71355513037422\\
23.695531565017	7.71319773783243\\
23.6949639525588	7.71284033058465\\
23.6943963030054	7.71248290862966\\
23.693828616352	7.71212547196624\\
23.6932608925936	7.7117680205932\\
23.6926931317255	7.71141055450931\\
23.6921253337427	7.71105307371337\\
23.6915574986405	7.71069557820418\\
23.690989626414	7.7103380679805\\
23.6904217170582	7.70998054304114\\
23.6898537705685	7.70962300338488\\
23.6892857869398	7.70926544901052\\
23.6887177661674	7.70890787991683\\
23.6881497082464	7.70855029610261\\
23.6875816131719	7.70819269756665\\
23.6870134809391	7.70783508430773\\
23.6864453115431	7.70747745632464\\
23.6858771049791	7.70711981361616\\
23.6853088612422	7.7067621561811\\
23.6847405803274	7.70640448401822\\
23.6841722622301	7.70604679712633\\
23.6836039069452	7.7056890955042\\
23.6830355144679	7.70533137915063\\
23.6824670847934	7.7049736480644\\
23.6818986179168	7.70461590224429\\
23.6813301138332	7.7042581416891\\
23.6807615725378	7.7039003663976\\
23.6801929940256	7.70354257636859\\
23.6796243782918	7.70318477160085\\
23.6790557253316	7.70282695209316\\
23.67848703514	7.70246911784432\\
23.6779183077122	7.7021112688531\\
23.6773495430433	7.7017534051183\\
23.6767807411284	7.70139552663869\\
23.6762119019627	7.70103763341307\\
23.6756430255412	7.70067972544021\\
23.6750741118591	7.7003218027189\\
23.6745051609116	7.69996386524793\\
23.6739361726936	7.69960591302609\\
23.6733671472004	7.69924794605214\\
23.672798084427	7.69888996432488\\
23.6722289843686	7.6985319678431\\
23.6716598470203	7.69817395660557\\
23.6710906723772	7.69781593061108\\
23.6705214604344	7.69745788985842\\
23.669952211187	7.69709983434636\\
23.6693829246302	7.69674176407369\\
23.668813600759	7.69638367903919\\
23.6682442395685	7.69602557924165\\
23.6676748410539	7.69566746467984\\
23.6671054052102	7.69530933535256\\
23.6665359320327	7.69495119125857\\
23.6659664215162	7.69459303239668\\
23.6653968736561	7.69423485876565\\
23.6648272884473	7.69387667036427\\
23.664257665885	7.69351846719132\\
23.6636880059643	7.69316024924558\\
23.6631183086803	7.69280201652583\\
23.662548574028	7.69244376903086\\
23.6619788020026	7.69208550675945\\
23.6614089925992	7.69172722971037\\
23.6608391458128	7.69136893788241\\
23.6602692616386	7.69101063127435\\
23.6596993400716	7.69065230988497\\
23.659129381107	7.69029397371305\\
23.6585593847398	7.68993562275737\\
23.6579893509652	7.68957725701671\\
23.6574192797782	7.68921887648984\\
23.6568491711738	7.68886048117556\\
23.6562790251473	7.68850207107264\\
23.6557088416937	7.68814364617985\\
23.655138620808	7.68778520649598\\
23.6545683624853	7.68742675201982\\
23.6539980667209	7.68706828275012\\
23.6534277335096	7.68670979868569\\
23.6528573628466	7.68635129982528\\
23.652286954727	7.68599278616769\\
23.6517165091459	7.68563425771169\\
23.6511460260983	7.68527571445606\\
23.6505755055793	7.68491715639957\\
23.650004947584	7.68455858354101\\
23.6494343521075	7.68419999587916\\
23.6488637191448	7.68384139341278\\
23.6482930486911	7.68348277614067\\
23.6477223407413	7.68312414406159\\
23.6471515952907	7.68276549717432\\
23.6465808123341	7.68240683547764\\
23.6460099918668	7.68204815897033\\
23.6454391338837	7.68168946765116\\
23.64486823838	7.68133076151892\\
23.6442973053507	7.68097204057236\\
23.6437263347909	7.68061330481029\\
23.6431553266956	7.68025455423146\\
23.64258428106	7.67989578883465\\
23.642013197879	7.67953700861865\\
23.6414420771477	7.67917821358223\\
23.6408709188613	7.67881940372416\\
23.6402997230147	7.67846057904321\\
23.639728489603	7.67810173953817\\
23.6391572186213	7.6777428852078\\
23.6385859100647	7.67738401605089\\
23.6380145639281	7.67702513206621\\
23.6374431802067	7.67666623325253\\
23.6368717588955	7.67630731960862\\
23.6363002999895	7.67594839113327\\
23.6357288034838	7.67558944782524\\
23.6351572693735	7.67523048968332\\
23.6345856976537	7.67487151670626\\
23.6340140883192	7.67451252889285\\
23.6334424413653	7.67415352624187\\
23.632870756787	7.67379450875207\\
23.6322990345792	7.67343547642225\\
23.6317272747371	7.67307642925116\\
23.6311554772557	7.67271736723759\\
23.6305836421301	7.67235829038031\\
23.6300117693552	7.67199919867808\\
23.6294398589261	7.67164009212969\\
23.628867910838	7.67128097073389\\
23.6282959250857	7.67092183448948\\
23.6277239016643	7.67056268339521\\
23.627151840569	7.67020351744987\\
23.6265797417947	7.66984433665221\\
23.6260076053364	7.66948514100102\\
23.6254354311892	7.66912593049506\\
23.6248632193482	7.66876670513311\\
23.6242909698083	7.66840746491394\\
23.6237186825646	7.66804820983631\\
23.6231463576121	7.667688939899\\
23.6225739949459	7.66732965510078\\
23.622001594561	7.66697035544043\\
23.6214291564523	7.6666110409167\\
23.6208566806151	7.66625171152837\\
23.6202841670441	7.66589236727422\\
23.6197116157346	7.66553300815301\\
23.6191390266815	7.6651736341635\\
23.6185663998798	7.66481424530448\\
23.6179937353246	7.66445484157471\\
23.6174210330109	7.66409542297295\\
23.6168482929336	7.66373598949798\\
23.6162755150879	7.66337654114857\\
23.6157026994688	7.66301707792349\\
23.6151298460712	7.6626575998215\\
23.6145569548901	7.66229810684138\\
23.6139840259207	7.66193859898188\\
23.6134110591579	7.66157907624179\\
23.6128380545966	7.66121953861986\\
23.6122650122321	7.66085998611487\\
23.6116919320591	7.66050041872559\\
23.6111188140728	7.66014083645077\\
23.6105456582682	7.65978123928919\\
23.6099724646402	7.65942162723962\\
23.609399233184	7.65906200030082\\
23.6088259638944	7.65870235847157\\
23.6082526567665	7.65834270175062\\
23.6076793117953	7.65798303013674\\
23.6071059289758	7.65762334362871\\
23.606532508303	7.65726364222528\\
23.6059590497719	7.65690392592522\\
23.6053855533775	7.6565441947273\\
23.6048120191148	7.65618444863029\\
23.6042384469788	7.65582468763295\\
23.6036648369645	7.65546491173405\\
23.603091189067	7.65510512093234\\
23.6025175032811	7.65474531522661\\
23.6019437796019	7.6543854946156\\
23.6013700180244	7.6540256590981\\
23.6007962185435	7.65366580867286\\
23.6002223811543	7.65330594333864\\
23.5996485058518	7.65294606309421\\
23.599074592631	7.65258616793835\\
23.5985006414867	7.6522262578698\\
23.5979266524141	7.65186633288733\\
23.5973526254081	7.65150639298972\\
23.5967785604637	7.65114643817571\\
23.5962044575759	7.65078646844409\\
23.5956303167397	7.6504264837936\\
23.59505613795	7.65006648422301\\
23.5944819212018	7.64970646973109\\
23.5939076664901	7.6493464403166\\
23.59333337381	7.6489863959783\\
23.5927590431563	7.64862633671496\\
23.592184674524	7.64826626252533\\
23.5916102679082	7.64790617340818\\
23.5910358233038	7.64754606936227\\
23.5904613407057	7.64718595038637\\
23.589886820109	7.64682581647923\\
23.5893122615086	7.64646566763962\\
23.5887376648995	7.6461055038663\\
23.5881630302767	7.64574532515803\\
23.5875883576351	7.64538513151357\\
23.5870136469698	7.64502492293168\\
23.5864388982755	7.64466469941113\\
23.5858641115475	7.64430446095067\\
23.5852892867805	7.64394420754907\\
23.5847144239696	7.64358393920509\\
23.5841395231098	7.64322365591748\\
23.5835645841959	7.64286335768501\\
23.582989607223	7.64250304450643\\
23.582414592186	7.64214271638052\\
23.5818395390799	7.64178237330601\\
23.5812644478996	7.64142201528169\\
23.5806893186401	7.6410616423063\\
23.5801141512964	7.64070125437861\\
23.5795389458634	7.64034085149737\\
23.578963702336	7.63998043366134\\
23.5783884207093	7.63962000086929\\
23.5778131009781	7.63925955311996\\
23.5772377431375	7.63889909041213\\
23.5766623471823	7.63853861274454\\
23.5760869131076	7.63817812011597\\
23.5755114409082	7.63781761252515\\
23.5749359305792	7.63745708997086\\
23.5743603821154	7.63709655245184\\
23.5737847955118	7.63673599996687\\
23.5732091707634	7.63637543251468\\
23.5726335078651	7.63601485009406\\
23.5720578068118	7.63565425270374\\
23.5714820675985	7.63529364034249\\
23.5709062902201	7.63493301300906\\
23.5703304746717	7.63457237070221\\
23.569754620948	7.63421171342069\\
23.569178729044	7.63385104116327\\
23.5686027989548	7.6334903539287\\
23.5680268306751	7.63312965171574\\
23.5674508242	7.63276893452313\\
23.5668747795244	7.63240820234965\\
23.5662986966432	7.63204745519403\\
23.5657225755513	7.63168669305504\\
23.5651464162438	7.63132591593144\\
23.5645702187154	7.63096512382198\\
23.5639939829612	7.63060431672541\\
23.563417708976	7.63024349464048\\
23.5628413967549	7.62988265756596\\
23.5622650462926	7.6295218055006\\
23.5616886575842	7.62916093844315\\
23.5611122306246	7.62880005639236\\
23.5605357654087	7.628439159347\\
23.5599592619313	7.62807824730581\\
23.5593827201875	7.62771732026754\\
23.5588061401722	7.62735637823096\\
23.5582295218802	7.62699542119481\\
23.5576528653065	7.62663444915786\\
23.557076170446	7.62627346211884\\
23.5564994372936	7.62591246007652\\
23.5559226658443	7.62555144302964\\
23.5553458560929	7.62519041097697\\
23.5547690080344	7.62482936391724\\
23.5541921216636	7.62446830184922\\
23.5536151969756	7.62410722477166\\
23.5530382339651	7.6237461326833\\
23.5524612326271	7.62338502558291\\
23.5518841929565	7.62302390346923\\
23.5513071149483	7.62266276634101\\
23.5507299985972	7.62230161419701\\
23.5501528438983	7.62194044703597\\
23.5495756508464	7.62157926485666\\
23.5489984194365	7.62121806765781\\
23.5484211496634	7.62085685543818\\
23.547843841522	7.62049562819652\\
23.5472664950072	7.62013438593158\\
23.546689110114	7.61977312864211\\
23.5461116868372	7.61941185632687\\
23.5455342251717	7.61905056898459\\
23.5449567251125	7.61868926661404\\
23.5443791866543	7.61832794921395\\
23.5438016097922	7.61796661678309\\
23.5432239945209	7.6176052693202\\
23.5426463408355	7.61724390682402\\
23.5420686487307	7.61688252929331\\
23.5414909182015	7.61652113672682\\
23.5409131492428	7.6161597291233\\
23.5403353418494	7.61579830648149\\
23.5397574960162	7.61543686880014\\
23.5391796117382	7.615075416078\\
23.5386016890101	7.61471394831382\\
23.5380237278269	7.61435246550635\\
23.5374457281835	7.61399096765433\\
23.5368676900748	7.61362945475652\\
23.5362896134955	7.61326792681166\\
23.5357114984407	7.61290638381849\\
23.5351333449052	7.61254482577577\\
23.5345551528838	7.61218325268224\\
23.5339769223714	7.61182166453665\\
23.533398653363	7.61146006133774\\
23.5328203458533	7.61109844308427\\
23.5322419998373	7.61073680977497\\
23.5316636153099	7.6103751614086\\
23.5310851922658	7.6100134979839\\
23.5305067307	7.60965181949962\\
23.5299282306073	7.6092901259545\\
23.5293496919827	7.60892841734728\\
23.5287711148209	7.60856669367672\\
23.5281924991168	7.60820495494156\\
23.5276138448654	7.60784320114055\\
23.5270351520614	7.60748143227242\\
23.5264564206998	7.60711964833592\\
23.5258776507753	7.60675784932981\\
23.5252988422829	7.60639603525282\\
23.5247199952174	7.60603420610369\\
23.5241411095737	7.60567236188118\\
23.5235621853466	7.60531050258402\\
23.522983222531	7.60494862821096\\
23.5224042211217	7.60458673876075\\
23.5218251811136	7.60422483423212\\
23.5212461025016	7.60386291462382\\
23.5206669852804	7.6035009799346\\
23.5200878294451	7.60313903016319\\
23.5195086349903	7.60277706530835\\
23.5189294019109	7.6024150853688\\
23.5183501302019	7.6020530903433\\
23.517770819858	7.60169108023059\\
23.5171914708741	7.60132905502941\\
23.516612083245	7.60096701473851\\
23.5160326569656	7.60060495935661\\
23.5154531920307	7.60024288888247\\
23.5148736884352	7.59988080331483\\
23.5142941461739	7.59951870265243\\
23.5137145652416	7.59915658689401\\
23.5131349456333	7.59879445603832\\
23.5125552873436	7.59843231008408\\
23.5119755903675	7.59807014903005\\
23.5113958546998	7.59770797287497\\
23.5108160803354	7.59734578161757\\
23.510236267269	7.5969835752566\\
23.5096564154954	7.59662135379079\\
23.5090765250097	7.59625911721889\\
23.5084965958064	7.59589686553964\\
23.5079166278806	7.59553459875177\\
23.507336621227	7.59517231685402\\
23.5067565758404	7.59481001984515\\
23.5061764917157	7.59444770772387\\
23.5055963688477	7.59408538048894\\
23.5050162072312	7.5937230381391\\
23.504436006861	7.59336068067308\\
23.5038557677321	7.59299830808961\\
23.5032754898391	7.59263592038745\\
23.5026951731769	7.59227351756532\\
23.5021148177403	7.59191109962197\\
23.5015344235242	7.59154866655613\\
23.5009539905233	7.59118621836655\\
23.5003735187326	7.59082375505195\\
23.4997930081467	7.59046127661108\\
23.4992124587605	7.59009878304268\\
23.4986318705689	7.58973627434547\\
23.4980512435665	7.58937375051821\\
23.4974705777483	7.58901121155962\\
23.4968898731091	7.58864865746845\\
23.4963091296436	7.58828608824342\\
23.4957283473467	7.58792350388328\\
23.4951475262132	7.58756090438676\\
23.4945666662379	7.58719828975261\\
23.4939857674155	7.58683565997954\\
23.4934048297409	7.58647301506631\\
23.492823853209	7.58611035501164\\
23.4922428378144	7.58574767981427\\
23.491661783552	7.58538498947294\\
23.4910806904166	7.58502228398639\\
23.490499558403	7.58465956335334\\
23.489918387506	7.58429682757253\\
23.4893371777204	7.5839340766427\\
23.4887559290409	7.58357131056258\\
23.4881746414624	7.58320852933091\\
23.4875933149797	7.58284573294641\\
23.4870119495876	7.58248292140784\\
23.4864305452808	7.58212009471391\\
23.4858491020542	7.58175725286336\\
23.4852676199024	7.58139439585493\\
23.4846860988204	7.58103152368735\\
23.4841045388029	7.58066863635935\\
23.4835229398447	7.58030573386966\\
23.4829413019406	7.57994281621703\\
23.4823596250853	7.57957988340018\\
23.4817779092737	7.57921693541784\\
23.4811961545005	7.57885397226875\\
23.4806143607605	7.57849099395163\\
23.4800325280485	7.57812800046523\\
23.4794506563592	7.57776499180828\\
23.4788687456875	7.5774019679795\\
23.4782867960281	7.57703892897762\\
23.4777048073758	7.57667587480139\\
23.4771227797254	7.57631280544952\\
23.4765407130716	7.57594972092076\\
23.4759586074092	7.57558662121383\\
23.475376462733	7.57522350632746\\
23.4747942790377	7.57486037626039\\
23.4742120563182	7.57449723101134\\
23.4736297945692	7.57413407057905\\
23.4730474937854	7.57377089496225\\
23.4724651539616	7.57340770415966\\
23.4718827750927	7.57304449817001\\
23.4713003571732	7.57268127699205\\
23.4707179001981	7.57231804062448\\
23.4701354041621	7.57195478906606\\
23.4695528690599	7.57159152231549\\
23.4689702948863	7.57122824037153\\
23.468387681636	7.57086494323288\\
23.4678050293039	7.57050163089828\\
23.4672223378846	7.57013830336647\\
23.466639607373	7.56977496063616\\
23.4660568377637	7.56941160270609\\
23.4654740290516	7.56904822957499\\
23.4648911812313	7.56868484124158\\
23.4643082942977	7.56832143770459\\
23.4637253682455	7.56795801896275\\
23.4631424030694	7.56759458501479\\
23.4625593987642	7.56723113585943\\
23.4619763553246	7.56686767149541\\
23.4613932727453	7.56650419192144\\
23.4608101510212	7.56614069713625\\
23.460226990147	7.56577718713858\\
23.4596437901174	7.56541366192715\\
23.4590605509271	7.56505012150069\\
23.4584772725709	7.56468656585791\\
23.4578939550435	7.56432299499756\\
23.4573105983397	7.56395940891834\\
23.4567272024542	7.563595807619\\
23.4561437673817	7.56323219109826\\
23.4555602931171	7.56286855935483\\
23.4549767796549	7.56250491238745\\
23.4543932269899	7.56214125019484\\
23.453809635117	7.56177757277573\\
23.4532260040307	7.56141388012884\\
23.4526423337259	7.5610501722529\\
23.4520586241972	7.56068644914663\\
23.4514748754394	7.56032271080875\\
23.4508910874473	7.55995895723799\\
23.4503072602154	7.55959518843308\\
23.4497233937387	7.55923140439273\\
23.4491394880117	7.55886760511568\\
23.4485555430293	7.55850379060064\\
23.4479715587861	7.55813996084633\\
23.4473875352768	7.55777611585149\\
23.4468034724963	7.55741225561484\\
23.4462193704391	7.55704838013509\\
23.4456352291	7.55668448941097\\
23.4450510484738	7.55632058344121\\
23.4444668285551	7.55595666222452\\
23.4438825693386	7.55559272575963\\
23.4432982708191	7.55522877404526\\
23.4427139329913	7.55486480708012\\
23.4421295558499	7.55450082486296\\
23.4415451393896	7.55413682739248\\
23.4409606836051	7.5537728146674\\
23.4403761884911	7.55340878668646\\
23.4397916540424	7.55304474344837\\
23.4392070802535	7.55268068495184\\
23.4386224671193	7.55231661119561\\
23.4380378146345	7.55195252217839\\
23.4374531227937	7.5515884178989\\
23.4368683915916	7.55122429835586\\
23.4362836210229	7.550860163548\\
23.4356988110824	7.55049601347403\\
23.4351139617648	7.55013184813268\\
23.4345290730646	7.54976766752265\\
23.4339441449767	7.54940347164268\\
23.4333591774957	7.54903926049149\\
23.4327741706164	7.54867503406778\\
23.4321891243333	7.54831079237028\\
23.4316040386413	7.54794653539772\\
23.4310189135349	7.5475822631488\\
23.4304337490089	7.54721797562225\\
23.429848545058	7.54685367281678\\
23.4292633016769	7.54648935473112\\
23.4286780188602	7.54612502136398\\
23.4280926966026	7.54576067271407\\
23.4275073348988	7.54539630878013\\
23.4269219337435	7.54503192956086\\
23.4263364931315	7.54466753505498\\
23.4257510130572	7.54430312526121\\
23.4251654935155	7.54393870017827\\
23.424579934501	7.54357425980488\\
23.4239943360084	7.54320980413974\\
23.4234086980324	7.54284533318159\\
23.4228230205676	7.54248084692912\\
23.4222373036088	7.54211634538107\\
23.4216515471505	7.54175182853614\\
23.4210657511875	7.54138729639306\\
23.4204799157145	7.54102274895053\\
23.419894040726	7.54065818620728\\
23.4193081262168	7.54029360816202\\
23.4187221721816	7.53992901481346\\
23.418136178615	7.53956440616032\\
23.4175501455116	7.53919978220132\\
23.4169640728662	7.53883514293517\\
23.4163779606734	7.53847048836059\\
23.4157918089278	7.53810581847628\\
23.4152056176242	7.53774113328097\\
23.4146193867572	7.53737643277337\\
23.4140331163214	7.53701171695218\\
23.4134468063115	7.53664698581614\\
23.4128604567222	7.53628223936395\\
23.4122740675481	7.53591747759431\\
23.4116876387839	7.53555270050596\\
23.4111011704242	7.5351879080976\\
23.4105146624636	7.53482310036794\\
23.409928114897	7.53445827731569\\
23.4093415277187	7.53409343893958\\
23.4087549009237	7.53372858523831\\
23.4081682345064	7.53336371621059\\
23.4075815284615	7.53299883185514\\
23.4069947827837	7.53263393217066\\
23.4064079974676	7.53226901715588\\
23.4058211725079	7.5319040868095\\
23.4052343078991	7.53153914113024\\
23.4046474036361	7.5311741801168\\
23.4040604597133	7.5308092037679\\
23.4034734761254	7.53044421208224\\
23.4028864528671	7.53007920505855\\
23.402299389933	7.52971418269552\\
23.4017122873178	7.52934914499188\\
23.401125145016	7.52898409194633\\
23.4005379630223	7.52861902355758\\
23.3999507413314	7.52825393982434\\
23.3993634799378	7.52788884074532\\
23.3987761788363	7.52752372631924\\
23.3981888380214	7.52715859654479\\
23.3976014574877	7.5267934514207\\
23.39701403723	7.52642829094567\\
23.3964265772427	7.52606311511841\\
23.3958390775206	7.52569792393763\\
23.3952515380583	7.52533271740203\\
23.3946639588504	7.52496749551033\\
23.3940763398914	7.52460225826123\\
23.3934886811761	7.52423700565345\\
23.3929009826991	7.5238717376857\\
23.3923132444549	7.52350645435666\\
23.3917254664383	7.52314115566507\\
23.3911376486437	7.52277584160962\\
23.3905497910659	7.52241051218903\\
23.3899618936994	7.522045167402\\
23.3893739565388	7.52167980724723\\
23.3887859795788	7.52131443172344\\
23.388197962814	7.52094904082933\\
23.387609906239	7.5205836345636\\
23.3870218098484	7.52021821292498\\
23.3864336736367	7.51985277591215\\
23.3858454975987	7.51948732352383\\
23.3852572817289	7.51912185575873\\
23.3846690260219	7.51875637261555\\
23.3840807304724	7.51839087409299\\
23.3834923950749	7.51802536018976\\
23.382904019824	7.51765983090457\\
23.3823156047143	7.51729428623612\\
23.3817271497405	7.51692872618312\\
23.381138654897	7.51656315074427\\
23.3805501201786	7.51619755991827\\
23.3799615455799	7.51583195370384\\
23.3793729310953	7.51546633209968\\
23.3787842767196	7.51510069510448\\
23.3781955824472	7.51473504271696\\
23.3776068482729	7.51436937493582\\
23.3770180741911	7.51400369175976\\
23.3764292601965	7.51363799318749\\
23.3758404062836	7.51327227921771\\
23.3752515124471	7.51290654984912\\
23.3746625786816	7.51254080508042\\
23.3740736049815	7.51217504491033\\
23.3734845913416	7.51180926933753\\
23.3728955377563	7.51144347836074\\
23.3723064442203	7.51107767197866\\
23.3717173107282	7.51071185018999\\
23.3711281372745	7.51034601299343\\
23.3705389238538	7.50998016038768\\
23.3699496704607	7.50961429237145\\
23.3693603770897	7.50924840894343\\
23.3687710437354	7.50888251010234\\
23.3681816703925	7.50851659584686\\
23.3675922570555	7.50815066617571\\
23.3670028037189	7.50778472108757\\
23.3664133103773	7.50741876058117\\
23.3658237770253	7.50705278465518\\
23.3652342036575	7.50668679330832\\
23.3646445902684	7.50632078653929\\
23.3640549368526	7.50595476434678\\
23.3634652434047	7.50558872672949\\
23.3628755099192	7.50522267368614\\
23.3622857363906	7.5048566052154\\
23.3616959228137	7.50449052131599\\
23.3611060691828	7.5041244219866\\
23.3605161754926	7.50375830722594\\
23.3599262417376	7.5033921770327\\
23.3593362679124	7.50302603140558\\
23.3587462540116	7.50265987034328\\
23.3581562000296	7.5022936938445\\
23.3575661059611	7.50192750190793\\
23.3569759718006	7.50156129453229\\
23.3563857975426	7.50119507171625\\
23.3557955831818	7.50082883345852\\
23.3552053287126	7.50046257975781\\
23.3546150341296	7.5000963106128\\
23.3540246994274	7.49973002602219\\
23.3534343246004	7.49936372598468\\
23.3528439096433	7.49899741049897\\
23.3522534545506	7.49863107956376\\
23.3516629593168	7.49826473317774\\
23.3510724239365	7.49789837133961\\
23.3504818484042	7.49753199404806\\
23.3498912327144	7.4971656013018\\
23.3493005768618	7.49679919309951\\
23.3487098808407	7.49643276943989\\
23.3481191446458	7.49606633032165\\
23.3475283682717	7.49569987574346\\
23.3469375517127	7.49533340570404\\
23.3463466949635	7.49496692020208\\
23.3457557980186	7.49460041923626\\
23.3451648608726	7.49423390280529\\
23.3445738835199	7.49386737090786\\
23.343982865955	7.49350082354266\\
23.3433918081726	7.4931342607084\\
23.3428007101672	7.49276768240376\\
23.3422095719331	7.49240108862743\\
23.3416183934651	7.49203447937812\\
23.3410271747576	7.49166785465451\\
23.3404359158051	7.4913012144553\\
23.3398446166022	7.49093455877918\\
23.3392532771433	7.49056788762485\\
23.3386618974231	7.490201200991\\
23.3380704774359	7.48983449887632\\
23.3374790171764	7.48946778127951\\
23.336887516639	7.48910104819925\\
23.3362959758182	7.48873429963424\\
23.3357043947087	7.48836753558318\\
23.3351127733048	7.48800075604475\\
23.334521111601	7.48763396101764\\
23.333929409592	7.48726715050056\\
23.3333376672722	7.48690032449218\\
23.3327458846361	7.4865334829912\\
23.3321540616782	7.48616662599632\\
23.331562198393	7.48579975350622\\
23.3309702947751	7.48543286551959\\
23.3303783508189	7.48506596203513\\
23.3297863665189	7.48469904305153\\
23.3291943418697	7.48433210856747\\
23.3286022768657	7.48396515858165\\
23.3280101715014	7.48359819309276\\
23.3274180257713	7.48323121209948\\
23.32682583967	7.48286421560052\\
23.3262336131918	7.48249720359454\\
23.3256413463314	7.48213017608026\\
23.3250490390832	7.48176313305634\\
23.3244566914417	7.4813960745215\\
23.3238643034013	7.4810290004744\\
23.3232718749567	7.48066191091375\\
23.3226794061022	7.48029480583823\\
23.3220868968323	7.47992768524653\\
23.3214943471416	7.47956054913733\\
23.3209017570245	7.47919339750934\\
23.3203091264755	7.47882623036122\\
23.3197164554891	7.47845904769168\\
23.3191237440597	7.4780918494994\\
23.3185309921819	7.47772463578306\\
23.3179381998501	7.47735740654136\\
23.3173453670589	7.47699016177298\\
23.3167524938026	7.47662290147661\\
23.3161595800757	7.47625562565093\\
23.3155666258728	7.47588833429463\\
23.3149736311883	7.47552102740641\\
23.3143805960166	7.47515370498493\\
23.3137875203523	7.4747863670289\\
23.3131944041899	7.474419013537\\
23.3126012475237	7.4740516445079\\
23.3120080503482	7.47368425994031\\
23.311414812658	7.4733168598329\\
23.3108215344475	7.47294944418435\\
23.3102282157111	7.47258201299337\\
23.3096348564433	7.47221456625862\\
23.3090414566386	7.47184710397879\\
23.3084480162914	7.47147962615258\\
23.3078545353962	7.47111213277865\\
23.3072610139475	7.47074462385571\\
23.3066674519397	7.47037709938242\\
23.3060738493672	7.47000955935748\\
23.3054802062246	7.46964200377958\\
23.3048865225062	7.46927443264738\\
23.3042927982066	7.46890684595958\\
23.3036990333201	7.46853924371486\\
23.3031052278413	7.4681716259119\\
23.3025113817646	7.46780399254939\\
23.3019174950844	7.46743634362601\\
23.3013235677951	7.46706867914044\\
23.3007295998913	7.46670099909136\\
23.3001355913674	7.46633330347746\\
23.2995415422177	7.46596559229742\\
23.2989474524368	7.46559786554992\\
23.2983533220191	7.46523012323364\\
23.2977591509591	7.46486236534726\\
23.2971649392511	7.46449459188947\\
23.2965706868896	7.46412680285895\\
23.2959763938691	7.46375899825438\\
23.295382060184	7.46339117807443\\
23.2947876858286	7.46302334231779\\
23.2941932707976	7.46265549098314\\
23.2935988150852	7.46228762406916\\
23.2930043186859	7.46191974157454\\
23.2924097815942	7.46155184349794\\
23.2918152038045	7.46118392983806\\
23.2912205853111	7.46081600059357\\
23.2906259261086	7.46044805576315\\
23.2900312261914	7.46008009534547\\
23.2894364855538	7.45971211933923\\
23.2888417041903	7.45934412774309\\
23.2882468820954	7.45897612055575\\
23.2876520192634	7.45860809777587\\
23.2870571156887	7.45824005940213\\
23.2864621713659	7.45787200543322\\
23.2858671862892	7.45750393586781\\
23.2852721604532	7.45713585070457\\
23.2846770938522	7.4567677499422\\
23.2840819864807	7.45639963357936\\
23.283486838333	7.45603150161473\\
23.2828916494036	7.455663354047\\
23.2822964196868	7.45529519087483\\
23.2817011491772	7.45492701209691\\
23.281105837869	7.45455881771191\\
23.2805104857568	7.45419060771851\\
23.2799150928349	7.45382238211538\\
23.2793196590977	7.4534541409012\\
23.2787241845396	7.45308588407465\\
23.278128669155	7.45271761163441\\
23.2775331129384	7.45234932357915\\
23.2769375158841	7.45198101990754\\
23.2763418779865	7.45161270061826\\
23.2757461992401	7.45124436570999\\
23.2751504796391	7.45087601518141\\
23.2745547191781	7.45050764903117\\
23.2739589178514	7.45013926725798\\
23.2733630756534	7.44977086986048\\
23.2727671925785	7.44940245683737\\
23.272171268621	7.44903402818732\\
23.2715753037755	7.44866558390899\\
23.2709792980362	7.44829712400107\\
23.2703832513976	7.44792864846223\\
23.269787163854	7.44756015729114\\
23.2691910353998	7.44719165048647\\
23.2685948660294	7.44682312804691\\
23.2679986557373	7.44645458997111\\
23.2674024045177	7.44608603625776\\
23.266806112365	7.44571746690553\\
23.2662097792737	7.44534888191309\\
23.2656134052381	7.44498028127911\\
23.2650169902526	7.44461166500227\\
23.2644205343116	7.44424303308124\\
23.2638240374094	7.44387438551468\\
23.2632274995404	7.44350572230128\\
23.262630920699	7.4431370434397\\
23.2620343008795	7.44276834892862\\
23.2614376400764	7.4423996387667\\
23.260840938284	7.44203091295262\\
23.2602441954966	7.44166217148505\\
23.2596474117087	7.44129341436266\\
23.2590505869145	7.44092464158412\\
23.2584537211085	7.44055585314809\\
23.2578568142851	7.44018704905327\\
23.2572598664385	7.4398182292983\\
23.2566628775632	7.43944939388186\\
23.2560658476535	7.43908054280263\\
23.2554687767037	7.43871167605926\\
23.2548716647083	7.43834279365044\\
23.2542745116616	7.43797389557483\\
23.2536773175578	7.43760498183109\\
23.2530800823915	7.4372360524179\\
23.2524828061569	7.43686710733393\\
23.2518854888484	7.43649814657785\\
23.2512881304604	7.43612917014832\\
23.2506907309871	7.43576017804401\\
23.250093290423	7.43539117026359\\
23.2494958087624	7.43502214680573\\
23.2488982859996	7.4346531076691\\
23.2483007221289	7.43428405285236\\
23.2477031171448	7.43391498235418\\
23.2471054710416	7.43354589617323\\
23.2465077838135	7.43317679430818\\
23.245910055455	7.43280767675768\\
23.2453122859604	7.43243854352042\\
23.244714475324	7.43206939459506\\
23.2441166235402	7.43170022998026\\
23.2435187306032	7.43133104967468\\
23.2429207965075	7.43096185367701\\
23.2423228212474	7.43059264198589\\
23.2417248048171	7.4302234146\\
23.241126747211	7.429854171518\\
23.2405286484235	7.42948491273856\\
23.2399305084489	7.42911563826034\\
23.2393323272815	7.42874634808202\\
23.2387341049156	7.42837704220224\\
23.2381358413456	7.42800772061968\\
23.2375375365657	7.42763838333301\\
23.2369391905704	7.42726903034088\\
23.2363408033538	7.42689966164196\\
23.2357423749105	7.42653027723492\\
23.2351439052345	7.42616087711842\\
23.2345453943204	7.42579146129112\\
23.2339468421624	7.42542202975169\\
23.2333482487547	7.42505258249879\\
23.2327496140918	7.42468311953108\\
23.2321509381679	7.42431364084722\\
23.2315522209774	7.42394414644589\\
23.2309534625145	7.42357463632573\\
23.2303546627737	7.42320511048543\\
23.229755821749	7.42283556892362\\
23.229156939435	7.42246601163899\\
23.2285580158259	7.42209643863019\\
23.2279590509159	7.42172684989588\\
23.2273600446995	7.42135724543472\\
23.2267609971709	7.42098762524539\\
23.2261619083243	7.42061798932652\\
23.2255627781542	7.4202483376768\\
23.2249636066548	7.41987867029488\\
23.2243643938203	7.41950898717941\\
23.2237651396452	7.41913928832907\\
23.2231658441236	7.41876957374251\\
23.22256650725	7.41839984341839\\
23.2219671290185	7.41803009735537\\
23.2213677094235	7.41766033555212\\
23.2207682484592	7.41729055800728\\
23.22016874612	7.41692076471953\\
23.2195692024002	7.41655095568752\\
23.2189696172939	7.41618113090991\\
23.2183699907956	7.41581129038535\\
23.2177703228995	7.41544143411252\\
23.2171706135998	7.41507156209006\\
23.2165708628909	7.41470167431664\\
23.2159710707671	7.41433177079091\\
23.2153712372226	7.41396185151154\\
23.2147713622516	7.41359191647717\\
23.2141714458486	7.41322196568647\\
23.2135714880077	7.4128519991381\\
23.2129714887232	7.41248201683072\\
23.2123714479895	7.41211201876297\\
23.2117713658007	7.41174200493352\\
23.2111712421512	7.41137197534103\\
23.2105710770351	7.41100192998415\\
23.2099708704469	7.41063186886154\\
23.2093706223807	7.41026179197185\\
23.2087703328309	7.40989169931375\\
23.2081700017916	7.40952159088588\\
23.2075696292572	7.40915146668691\\
23.2069692152219	7.40878132671549\\
23.20636875968	7.40841117097027\\
23.2057682626258	7.40804099944991\\
23.2051677240534	7.40767081215307\\
23.2045671439572	7.40730060907841\\
23.2039665223314	7.40693039022457\\
23.2033658591702	7.40656015559021\\
23.202765154468	7.40618990517399\\
23.202164408219	7.40581963897455\\
23.2015636204174	7.40544935699057\\
23.2009627910575	7.40507905922068\\
23.2003619201335	7.40470874566354\\
23.1997610076397	7.40433841631781\\
23.1991600535704	7.40396807118215\\
23.1985590579197	7.40359771025519\\
23.1979580206819	7.40322733353561\\
23.1973569418513	7.40285694102204\\
23.1967558214221	7.40248653271315\\
23.1961546593886	7.40211610860758\\
23.195553455745	7.40174566870399\\
23.1949522104855	7.40137521300104\\
23.1943509236043	7.40100474149736\\
23.1937495950958	7.40063425419162\\
23.1931482249541	7.40026375108247\\
23.1925468131735	7.39989323216856\\
23.1919453597482	7.39952269744854\\
23.1913438646724	7.39915214692106\\
23.1907423279404	7.39878158058477\\
23.1901407495464	7.39841099843833\\
23.1895391294847	7.39804040048038\\
23.1889374677494	7.39766978670958\\
23.1883357643348	7.39729915712458\\
23.1877340192351	7.39692851172402\\
23.1871322324445	7.39655785050656\\
23.1865304039573	7.39618717347085\\
23.1859285337677	7.39581648061553\\
23.1853266218699	7.39544577193925\\
23.1847246682581	7.39507504744068\\
23.1841226729265	7.39470430711845\\
23.1835206358694	7.39433355097121\\
23.1829185570809	7.39396277899762\\
23.1823164365554	7.39359199119632\\
23.1817142742869	7.39322118756595\\
23.1811120702697	7.39285036810518\\
23.1805098244981	7.39247953281265\\
23.1799075369662	7.392108681687\\
23.1793052076682	7.39173781472689\\
23.1787028365984	7.39136693193095\\
23.1781004237509	7.39099603329785\\
23.1774979691199	7.39062511882622\\
23.1768954726998	7.39025418851471\\
23.1762929344845	7.38988324236197\\
23.1756903544685	7.38951228036665\\
23.1750877326458	7.3891413025274\\
23.1744850690106	7.38877030884286\\
23.1738823635572	7.38839929931167\\
23.1732796162798	7.38802827393249\\
23.1726768271725	7.38765723270395\\
23.1720739962295	7.38728617562471\\
23.1714711234451	7.38691510269341\\
23.1708682088134	7.3865440139087\\
23.1702652523286	7.38617290926922\\
23.169662253985	7.38580178877361\\
23.1690592137766	7.38543065242053\\
23.1684561316976	7.38505950020861\\
23.1678530077424	7.38468833213651\\
23.167249841905	7.38431714820286\\
23.1666466341796	7.38394594840631\\
23.1660433845605	7.3835747327455\\
23.1654400930417	7.38320350121908\\
23.1648367596175	7.38283225382569\\
23.164233384282	7.38246099056398\\
23.1636299670295	7.38208971143259\\
23.1630265078541	7.38171841643015\\
23.1624230067499	7.38134710555533\\
23.1618194637112	7.38097577880675\\
23.1612158787321	7.38060443618306\\
23.1606122518068	7.38023307768291\\
23.1600085829295	7.37986170330492\\
23.1594048720943	7.37949031304776\\
23.1588011192953	7.37911890691006\\
23.1581973245269	7.37874748489046\\
23.1575934877831	7.3783760469876\\
23.156989609058	7.37800459320013\\
23.1563856883459	7.37763312352668\\
23.155781725641	7.3772616379659\\
23.1551777209373	7.37689013651643\\
23.154573674229	7.3765186191769\\
23.1539695855103	7.37614708594597\\
23.1533654547754	7.37577553682226\\
23.1527612820183	7.37540397180443\\
23.1521570672334	7.37503239089111\\
23.1515528104146	7.37466079408093\\
23.1509485115562	7.37428918137255\\
23.1503441706522	7.3739175527646\\
23.149739787697	7.37354590825571\\
23.1491353626845	7.37317424784454\\
23.148530895609	7.37280257152971\\
23.1479263864646	7.37243087930987\\
23.1473218352454	7.37205917118365\\
23.1467172419456	7.37168744714969\\
23.1461126065593	7.37131570720664\\
23.1455079290807	7.37094395135313\\
23.1449032095039	7.37057217958779\\
23.144298447823	7.37020039190928\\
23.1436936440322	7.36982858831621\\
23.1430887981256	7.36945676880724\\
23.1424839100974	7.36908493338099\\
23.1418789799416	7.36871308203611\\
23.1412740076525	7.36834121477123\\
23.140668993224	7.36796933158499\\
23.1400639366505	7.36759743247603\\
23.1394588379259	7.36722551744298\\
23.1388536970445	7.36685358648448\\
23.1382485140003	7.36648163959916\\
23.1376432887875	7.36610967678566\\
23.1370380214002	7.36573769804262\\
23.1364327118325	7.36536570336867\\
23.1358273600786	7.36499369276245\\
23.1352219661325	7.36462166622259\\
23.1346165299884	7.36424962374772\\
23.1340110516403	7.36387756533649\\
23.1334055310825	7.36350549098753\\
23.132799968309	7.36313340069946\\
23.132194363314	7.36276129447094\\
23.1315887160915	7.36238917230058\\
23.1309830266356	7.36201703418703\\
23.1303772949406	7.36164488012891\\
23.1297715210004	7.36127271012487\\
23.1291657048091	7.36090052417353\\
23.128559846361	7.36052832227352\\
23.1279539456501	7.36015610442349\\
23.1273480026704	7.35978387062207\\
23.1267420174162	7.35941162086788\\
23.1261359898815	7.35903935515956\\
23.1255299200603	7.35866707349574\\
23.1249238079469	7.35829477587505\\
23.1243176535353	7.35792246229613\\
23.1237114568195	7.35755013275761\\
23.1231052177937	7.35717778725812\\
23.1224989364521	7.35680542579629\\
23.1218926127886	7.35643304837075\\
23.1212862467973	7.35606065498014\\
23.1206798384724	7.35568824562307\\
23.120073387808	7.3553158202982\\
23.119466894798	7.35494337900414\\
23.1188603594367	7.35457092173953\\
23.1182537817181	7.35419844850299\\
23.1176471616363	7.35382595929316\\
23.1170404991854	7.35345345410867\\
23.1164337943593	7.35308093294815\\
23.1158270471524	7.35270839581022\\
23.1152202575585	7.35233584269351\\
23.1146134255718	7.35196327359666\\
23.1140065511863	7.3515906885183\\
23.1133996343962	7.35121808745704\\
23.1127926751955	7.35084547041153\\
23.1121856735783	7.35047283738039\\
23.1115786295386	7.35010018836225\\
23.1109715430706	7.34972752335573\\
23.1103644141682	7.34935484235947\\
23.1097572428256	7.34898214537209\\
23.1091500290368	7.34860943239222\\
23.108542772796	7.34823670341848\\
23.107935474097	7.34786395844951\\
23.1073281329341	7.34749119748394\\
23.1067207493012	7.34711842052038\\
23.1061133231925	7.34674562755747\\
23.105505854602	7.34637281859382\\
23.1048983435237	7.34599999362808\\
23.1042907899518	7.34562715265886\\
23.1036831938802	7.3452542956848\\
23.103075555303	7.3448814227045\\
23.1024678742142	7.34450853371661\\
23.101860150608	7.34413562871975\\
23.1012523844784	7.34376270771254\\
23.1006445758194	7.34338977069361\\
23.1000367246251	7.34301681766159\\
23.0994288308894	7.34264384861509\\
23.0988208946065	7.34227086355274\\
23.0982129157705	7.34189786247317\\
23.0976048943752	7.341524845375\\
23.0969968304149	7.34115181225686\\
23.0963887238835	7.34077876311736\\
23.095780574775	7.34040569795514\\
23.0951723830835	7.34003261676882\\
23.0945641488031	7.33965951955702\\
23.0939558719278	7.33928640631835\\
23.0933475524515	7.33891327705146\\
23.0927391903684	7.33854013175496\\
23.0921307856725	7.33816697042746\\
23.0915223383577	7.3377937930676\\
23.0909138484182	7.337420599674\\
23.0903053158479	7.33704739024528\\
23.089696740641	7.33667416478005\\
23.0890881227913	7.33630092327695\\
23.0884794622929	7.3359276657346\\
23.0878707591399	7.33555439215161\\
23.0872620133263	7.3351811025266\\
23.0866532248461	7.33480779685821\\
23.0860443936933	7.33443447514504\\
23.0854355198619	7.33406113738572\\
23.084826603346	7.33368778357888\\
23.0842176441395	7.33331441372312\\
23.0836086422365	7.33294102781707\\
23.082999597631	7.33256762585936\\
23.0823905103169	7.33219420784859\\
23.0817813802884	7.3318207737834\\
23.0811722075394	7.33144732366239\\
23.0805629920639	7.33107385748419\\
23.079953733856	7.33070037524743\\
23.0793444329096	7.33032687695071\\
23.0787350892187	7.32995336259265\\
23.0781257027774	7.32957983217188\\
23.0775162735796	7.32920628568701\\
23.0769068016193	7.32883272313666\\
23.0762972868906	7.32845914451945\\
23.0756877293874	7.328085549834\\
23.0750781291037	7.32771193907892\\
23.0744684860336	7.32733831225284\\
23.073858800171	7.32696466935437\\
23.0732490715099	7.32659101038212\\
23.0726393000443	7.32621733533471\\
23.0720294857682	7.32584364421077\\
23.0714196286756	7.3254699370089\\
23.0708097287604	7.32509621372773\\
23.0701997860168	7.32472247436586\\
23.0695898004385	7.32434871892192\\
23.0689797720197	7.32397494739453\\
23.0683697007542	7.32360115978229\\
23.0677595866362	7.32322735608382\\
23.0671494296595	7.32285353629774\\
23.0665392298181	7.32247970042267\\
23.0659289871061	7.32210584845721\\
23.0653187015174	7.32173198039999\\
23.0647083730459	7.32135809624962\\
23.0640980016857	7.32098419600471\\
23.0634875874307	7.32061027966387\\
23.0628771302749	7.32023634722573\\
23.0622666302122	7.31986239868889\\
23.0616560872366	7.31948843405197\\
23.0610455013422	7.31911445331359\\
23.0604348725228	7.31874045647234\\
23.0598242007724	7.31836644352686\\
23.059213486085	7.31799241447575\\
23.0586027284545	7.31761836931763\\
23.0579919278749	7.3172443080511\\
23.0573810843402	7.31687023067478\\
23.0567701978444	7.31649613718729\\
23.0561592683813	7.31612202758723\\
23.0555482959449	7.31574790187321\\
23.0549372805292	7.31537376004386\\
23.0543262221282	7.31499960209777\\
23.0537151207357	7.31462542803357\\
23.0531039763458	7.31425123784985\\
23.0524927889524	7.31387703154525\\
23.0518815585494	7.31350280911835\\
23.0512702851308	7.31312857056778\\
23.0506589686905	7.31275431589215\\
23.0500476092225	7.31238004509007\\
23.0494362067207	7.31200575816014\\
23.0488247611791	7.31163145510097\\
23.0482132725916	7.31125713591119\\
23.0476017409521	7.31088280058939\\
23.0469901662545	7.31050844913418\\
23.0463785484929	7.31013408154418\\
23.0457668876611	7.30975969781799\\
23.0451551837531	7.30938529795423\\
23.0445434367628	7.30901088195149\\
23.0439316466842	7.3086364498084\\
23.0433198135111	7.30826200152356\\
23.0427079372375	7.30788753709557\\
23.0420960178574	7.30751305652304\\
23.0414840553646	7.30713855980459\\
23.040872049753	7.30676404693882\\
23.0402600010167	7.30638951792434\\
23.0396479091495	7.30601497275974\\
23.0390357741453	7.30564041144366\\
23.0384235959981	7.30526583397468\\
23.0378113747018	7.30489124035142\\
23.0371991102502	7.30451663057248\\
23.0365868026374	7.30414200463646\\
23.0359744518572	7.30376736254199\\
23.0353620579036	7.30339270428765\\
23.0347496207704	7.30301802987206\\
23.0341371404515	7.30264333929382\\
23.033524616941	7.30226863255154\\
23.0329120502326	7.30189390964383\\
23.0322994403203	7.30151917056928\\
23.0316867871979	7.3011444153265\\
23.0310740908595	7.30076964391411\\
23.0304613512989	7.30039485633069\\
23.02984856851	7.30002005257487\\
23.0292357424866	7.29964523264523\\
23.0286228732228	7.29927039654039\\
23.0280099607124	7.29889554425895\\
23.0273970049492	7.29852067579951\\
23.0267840059272	7.29814579116068\\
23.0261709636404	7.29777089034106\\
23.0255578780825	7.29739597333925\\
23.0249447492474	7.29702104015385\\
23.0243315771291	7.29664609078348\\
23.0237183617215	7.29627112522673\\
23.0231051030183	7.2958961434822\\
23.0224918010136	7.29552114554849\\
23.0218784557012	7.29514613142422\\
23.021265067075	7.29477110110797\\
23.0206516351288	7.29439605459836\\
23.0200381598566	7.29402099189398\\
23.0194246412522	7.29364591299343\\
23.0188110793095	7.29327081789532\\
23.0181974740224	7.29289570659824\\
23.0175838253848	7.29252057910081\\
23.0169701333905	7.2921454354016\\
23.0163563980333	7.29177027549925\\
23.0157426193073	7.29139509939232\\
23.0151287972062	7.29101990707944\\
23.0145149317239	7.29064469855919\\
23.0139010228543	7.29026947383018\\
23.0132870705912	7.28989423289102\\
23.0126730749286	7.28951897574029\\
23.0120590358602	7.28914370237659\\
23.01144495338	7.28876841279853\\
23.0108308274817	7.2883931070047\\
23.0102166581593	7.28801778499371\\
23.0096024454067	7.28764244676415\\
23.0089881892176	7.28726709231461\\
23.008373889586	7.28689172164371\\
23.0077595465056	7.28651633475003\\
23.0071451599704	7.28614093163217\\
23.0065307299742	7.28576551228873\\
23.0059162565108	7.28539007671832\\
23.0053017395741	7.28501462491951\\
23.0046871791579	7.28463915689092\\
23.0040725752562	7.28426367263114\\
23.0034579278626	7.28388817213876\\
23.0028432369712	7.28351265541239\\
23.0022285025756	7.28313712245061\\
23.0016137246699	7.28276157325203\\
23.0009989032477	7.28238600781524\\
23.000384038303	7.28201042613883\\
22.9997691298295	7.2816348282214\\
22.9991541778212	7.28125921406155\\
22.9985391822718	7.28088358365788\\
22.9979241431752	7.28050793700897\\
22.9973090605252	7.28013227411342\\
22.9966939343157	7.27975659496983\\
22.9960787645404	7.27938089957679\\
22.9954635511933	7.27900518793289\\
22.9948482942681	7.27862946003673\\
22.9942329937586	7.27825371588691\\
22.9936176496588	7.27787795548201\\
22.9930022619623	7.27750217882063\\
22.9923868306631	7.27712638590137\\
22.9917713557549	7.27675057672281\\
22.9911558372316	7.27637475128355\\
22.990540275087	7.27599890958218\\
22.9899246693148	7.2756230516173\\
22.989309019909	7.2752471773875\\
22.9886933268634	7.27487128689137\\
22.9880775901717	7.27449538012749\\
22.9874618098277	7.27411945709447\\
22.9868459858253	7.2737435177909\\
22.9862301181583	7.27336756221536\\
22.9856142068205	7.27299159036645\\
22.9849982518056	7.27261560224277\\
22.9843822531076	7.27223959784288\\
22.9837662107202	7.27186357716541\\
22.9831501246371	7.27148754020892\\
22.9825339948523	7.27111148697201\\
22.9819178213595	7.27073541745328\\
22.9813016041524	7.27035933165131\\
22.980685343225	7.26998322956469\\
22.9800690385709	7.26960711119201\\
22.9794526901841	7.26923097653187\\
22.9788362980582	7.26885482558284\\
22.9782198621871	7.26847865834353\\
22.9776033825646	7.26810247481251\\
22.9769868591844	7.26772627498838\\
22.9763702920404	7.26735005886972\\
22.9757536811262	7.26697382645514\\
22.9751370264358	7.2665975777432\\
22.9745203279629	7.2662213127325\\
22.9739035857013	7.26584503142163\\
22.9732867996447	7.26546873380918\\
22.972669969787	7.26509241989373\\
22.9720530961219	7.26471608967388\\
22.9714361786431	7.2643397431482\\
22.9708192173446	7.26396338031528\\
22.97020221222	7.26358700117372\\
22.9695851632631	7.26321060572209\\
22.9689680704677	7.26283419395899\\
22.9683509338275	7.262457765883\\
22.9677337533364	7.26208132149271\\
22.9671165289881	7.2617048607867\\
22.9664992607763	7.26132838376356\\
22.9658819486949	7.26095189042187\\
22.9652645927376	7.26057538076022\\
22.9646471928981	7.26019885477719\\
22.9640297491702	7.25982231247138\\
22.9634122615477	7.25944575384136\\
22.9627947300244	7.25906917888571\\
22.9621771545939	7.25869258760303\\
22.9615595352501	7.25831597999189\\
22.9609418719867	7.25793935605089\\
22.9603241647975	7.2575627157786\\
22.9597064136761	7.2571860591736\\
22.9590886186164	7.25680938623449\\
22.9584707796121	7.25643269695985\\
22.957852896657	7.25605599134825\\
22.9572349697448	7.25567926939828\\
22.9566169988692	7.25530253110853\\
22.955998984024	7.25492577647757\\
22.9553809252029	7.25454900550399\\
22.9547628223998	7.25417221818637\\
22.9541446756082	7.25379541452329\\
22.953526484822	7.25341859451334\\
22.9529082500348	7.25304175815509\\
22.9522899712405	7.25266490544713\\
22.9516716484328	7.25228803638804\\
22.9510532816054	7.25191115097639\\
22.9504348707519	7.25153424921078\\
22.9498164158663	7.25115733108978\\
22.9491979169421	7.25078039661197\\
22.9485793739731	7.25040344577593\\
22.9479607869531	7.25002647858024\\
22.9473421558757	7.24964949502349\\
22.9467234807348	7.24927249510424\\
22.9461047615239	7.24889547882109\\
22.9454859982369	7.24851844617261\\
22.9448671908674	7.24814139715737\\
22.9442483394092	7.24776433177397\\
22.9436294438559	7.24738725002097\\
22.9430105042014	7.24701015189696\\
22.9423915204393	7.24663303740051\\
22.9417724925633	7.24625590653021\\
22.9411534205672	7.24587875928462\\
22.9405343044446	7.24550159566234\\
22.9399151441893	7.24512441566193\\
22.9392959397949	7.24474721928197\\
22.9386766912553	7.24437000652105\\
22.938057398564	7.24399277737773\\
22.9374380617148	7.2436155318506\\
22.9368186807013	7.24323826993823\\
22.9361992555174	7.2428609916392\\
22.9355797861567	7.24248369695208\\
22.9349602726128	7.24210638587546\\
22.9343407148795	7.2417290584079\\
22.9337211129505	7.24135171454799\\
22.9331014668195	7.24097435429429\\
22.9324817764801	7.24059697764539\\
22.9318620419261	7.24021958459986\\
22.9312422631511	7.23984217515627\\
22.9306224401489	7.23946474931321\\
22.9300025729131	7.23908730706923\\
22.9293826614374	7.23870984842293\\
22.9287627057155	7.23833237337287\\
22.9281427057411	7.23795488191762\\
22.9275226615078	7.23757737405577\\
22.9269025730094	7.23719984978588\\
22.9262824402395	7.23682230910654\\
22.9256622631917	7.2364447520163\\
22.9250420418599	7.23606717851375\\
22.9244217762376	7.23568958859746\\
22.9238014663185	7.23531198226599\\
22.9231811120964	7.23493435951794\\
22.9225607135648	7.23455672035186\\
};
\addplot [color=mycolor1, forget plot]
  table[row sep=crcr]{%
22.9225607135648	7.23455672035186\\
22.9219402707174	7.23417906476632\\
22.9213197835479	7.23380139275991\\
22.92069925205	7.2334237043312\\
22.9200786762174	7.23304599947874\\
22.9194580560436	7.23266827820113\\
22.9188373915224	7.23229054049692\\
22.9182166826475	7.23191278636469\\
22.9175959294124	7.23153501580301\\
22.9169751318109	7.23115722881045\\
22.9163542898366	7.23077942538558\\
22.9157334034831	7.23040160552697\\
22.9151124727441	7.2300237692332\\
22.9144914976134	7.22964591650283\\
22.9138704780844	7.22926804733443\\
22.9132494141509	7.22889016172657\\
22.9126283058066	7.22851225967782\\
22.912007153045	7.22813434118675\\
22.9113859558598	7.22775640625194\\
22.9107647142447	7.22737845487194\\
22.9101434281933	7.22700048704533\\
22.9095220976992	7.22662250277067\\
22.9089007227561	7.22624450204654\\
22.9082793033577	7.2258664848715\\
22.9076578394975	7.22548845124413\\
22.9070363311692	7.22511040116298\\
22.9064147783665	7.22473233462662\\
22.9057931810829	7.22435425163364\\
22.9051715393122	7.22397615218258\\
22.9045498530479	7.22359803627202\\
22.9039281222836	7.22321990390052\\
22.9033063470131	7.22284175506666\\
22.9026845272299	7.22246358976899\\
22.9020626629276	7.22208540800609\\
22.9014407541	7.22170720977652\\
22.9008188007405	7.22132899507884\\
22.9001968028428	7.22095076391163\\
22.8995747604006	7.22057251627345\\
22.8989526734075	7.22019425216286\\
22.898330541857	7.21981597157843\\
22.8977083657428	7.21943767451872\\
22.8970861450586	7.2190593609823\\
22.8964638797979	7.21868103096773\\
22.8958415699543	7.21830268447358\\
22.8952192155215	7.21792432149841\\
22.894596816493	7.21754594204079\\
22.8939743728625	7.21716754609928\\
22.8933518846236	7.21678913367244\\
22.8927293517698	7.21641070475884\\
22.8921067742949	7.21603225935704\\
22.8914841521923	7.2156537974656\\
22.8908614854558	7.21527531908309\\
22.8902387740788	7.21489682420807\\
22.889616018055	7.2145183128391\\
22.888993217378	7.21413978497474\\
22.8883703720414	7.21376124061356\\
22.8877474820387	7.21338267975412\\
22.8871245473637	7.21300410239498\\
22.8865015680098	7.2126255085347\\
22.8858785439706	7.21224689817185\\
22.8852554752398	7.21186827130498\\
22.8846323618109	7.21148962793265\\
22.8840092036776	7.21111096805343\\
22.8833860008333	7.21073229166588\\
22.8827627532718	7.21035359876856\\
22.8821394609865	7.20997488936003\\
22.881516123971	7.20959616343884\\
22.880892742219	7.20921742100356\\
22.880269315724	7.20883866205275\\
22.8796458444795	7.20845988658497\\
22.8790223284793	7.20808109459878\\
22.8783987677167	7.20770228609273\\
22.8777751621855	7.20732346106538\\
22.8771515118791	7.2069446195153\\
22.8765278167912	7.20656576144105\\
22.8759040769153	7.20618688684117\\
22.875280292245	7.20580799571424\\
22.8746564627738	7.2054290880588\\
22.8740325884954	7.20505016387342\\
22.8734086694032	7.20467122315665\\
22.8727847054909	7.20429226590705\\
22.8721606967519	7.20391329212317\\
22.8715366431799	7.20353430180359\\
22.8709125447684	7.20315529494684\\
22.870288401511	7.2027762715515\\
22.8696642134012	7.20239723161611\\
22.8690399804326	7.20201817513923\\
22.8684157025987	7.20163910211942\\
22.8677913798931	7.20126001255523\\
22.8671670123093	7.20088090644522\\
22.8665425998409	7.20050178378795\\
22.8659181424814	7.20012264458197\\
22.8652936402244	7.19974348882583\\
22.8646690930634	7.1993643165181\\
22.8640445009919	7.19898512765732\\
22.8634198640036	7.19860592224205\\
22.8627951820918	7.19822670027084\\
22.8621704552503	7.19784746174226\\
22.8615456834724	7.19746820665485\\
22.8609208667518	7.19708893500716\\
22.8602960050819	7.19670964679776\\
22.8596710984564	7.19633034202519\\
22.8590461468686	7.19595102068801\\
22.8584211503123	7.19557168278477\\
22.8577961087808	7.19519232831402\\
22.8571710222678	7.19481295727432\\
22.8565458907667	7.19443356966422\\
22.8559207142711	7.19405416548227\\
22.8552954927745	7.19367474472703\\
22.8546702262704	7.19329530739704\\
22.8540449147523	7.19291585349086\\
22.8534195582138	7.19253638300704\\
22.8527941566483	7.19215689594412\\
22.8521687100495	7.19177739230068\\
22.8515432184107	7.19139787207524\\
22.8509176817255	7.19101833526636\\
22.8502920999875	7.19063878187261\\
22.8496664731901	7.19025921189252\\
22.8490408013268	7.18987962532464\\
22.8484150843912	7.18950002216753\\
22.8477893223767	7.18912040241973\\
22.8471635152769	7.1887407660798\\
22.8465376630853	7.18836111314628\\
22.8459117657953	7.18798144361773\\
22.8452858234004	7.18760175749269\\
22.8446598358943	7.18722205476972\\
22.8440338032703	7.18684233544735\\
22.8434077255219	7.18646259952414\\
22.8427816026427	7.18608284699865\\
22.8421554346261	7.18570307786941\\
22.8415292214657	7.18532329213497\\
22.8409029631549	7.18494348979388\\
22.8402766596872	7.1845636708447\\
22.8396503110561	7.18418383528596\\
22.8390239172551	7.18380398311622\\
22.8383974782776	7.18342411433401\\
22.8377709941172	7.1830442289379\\
22.8371444647673	7.18266432692642\\
22.8365178902215	7.18228440829812\\
22.8358912704731	7.18190447305155\\
22.8352646055158	7.18152452118526\\
22.8346378953428	7.18114455269778\\
22.8340111399478	7.18076456758767\\
22.8333843393242	7.18038456585347\\
22.8327574934654	7.18000454749373\\
22.832130602365	7.17962451250699\\
22.8315036660164	7.1792444608918\\
22.8308766844131	7.17886439264669\\
22.8302496575485	7.17848430777023\\
22.8296225854161	7.17810420626094\\
22.8289954680093	7.17772408811738\\
22.8283683053218	7.17734395333809\\
22.8277410973467	7.17696380192161\\
22.8271138440778	7.17658363386648\\
22.8264865455083	7.17620344917126\\
22.8258592016319	7.17582324783448\\
22.8252318124418	7.17544302985468\\
22.8246043779316	7.17506279523041\\
22.8239768980947	7.17468254396022\\
22.8233493729246	7.17430227604263\\
22.8227218024148	7.17392199147621\\
22.8220941865586	7.17354169025948\\
22.8214665253495	7.17316137239099\\
22.820838818781	7.17278103786929\\
22.8202110668465	7.17240068669291\\
22.8195832695394	7.17202031886039\\
22.8189554268532	7.17163993437028\\
22.8183275387814	7.17125953322111\\
22.8176996053173	7.17087911541143\\
22.8170716264544	7.17049868093978\\
22.8164436021862	7.17011822980471\\
22.815815532506	7.16973776200473\\
22.8151874174074	7.16935727753841\\
22.8145592568837	7.16897677640428\\
22.8139310509283	7.16859625860087\\
22.8133027995347	7.16821572412673\\
22.8126745026964	7.1678351729804\\
22.8120461604066	7.16745460516041\\
22.811417772659	7.16707402066531\\
22.8107893394468	7.16669341949363\\
22.8101608607636	7.16631280164392\\
22.8095323366027	7.1659321671147\\
22.8089037669575	7.16555151590452\\
22.8082751518215	7.16517084801192\\
22.807646491188	7.16479016343543\\
22.8070177850506	7.16440946217359\\
22.8063890334026	7.16402874422494\\
22.8057602362373	7.16364800958802\\
22.8051313935483	7.16326725826136\\
22.804502505329	7.1628864902435\\
22.8038735715726	7.16250570553297\\
22.8032445922727	7.16212490412832\\
22.8026155674227	7.16174408602807\\
22.8019864970159	7.16136325123077\\
22.8013573810457	7.16098239973495\\
22.8007282195056	7.16060153153914\\
22.800099012389	7.16022064664189\\
22.7994697596892	7.15983974504172\\
22.7988404613996	7.15945882673717\\
22.7982111175137	7.15907789172678\\
22.7975817280247	7.15869694000907\\
22.7969522929262	7.1583159715826\\
22.7963228122115	7.15793498644588\\
22.795693285874	7.15755398459746\\
22.7950637139071	7.15717296603586\\
22.7944340963041	7.15679193075962\\
22.7938044330585	7.15641087876728\\
22.7931747241635	7.15602981005736\\
22.7925449696128	7.15564872462841\\
22.7919151693994	7.15526762247895\\
22.791285323517	7.15488650360751\\
22.7906554319588	7.15450536801263\\
22.7900254947182	7.15412421569285\\
22.7893955117886	7.15374304664668\\
22.7887654831634	7.15336186087267\\
22.7881354088358	7.15298065836935\\
22.7875052887994	7.15259943913525\\
22.7868751230475	7.15221820316889\\
22.7862449115734	7.15183695046882\\
22.7856146543705	7.15145568103355\\
22.7849843514321	7.15107439486163\\
22.7843540027517	7.15069309195158\\
22.7837236083226	7.15031177230193\\
22.7830931681381	7.14993043591122\\
22.7824626821916	7.14954908277797\\
22.7818321504765	7.14916771290072\\
22.781201572986	7.14878632627798\\
22.7805709497137	7.1484049229083\\
22.7799402806527	7.1480235027902\\
22.7793095657965	7.14764206592221\\
22.7786788051385	7.14726061230285\\
22.7780479986719	7.14687914193067\\
22.7774171463901	7.14649765480418\\
22.7767862482864	7.14611615092192\\
22.7761553043543	7.1457346302824\\
22.775524314587	7.14535309288417\\
22.7748932789778	7.14497153872575\\
22.7742621975202	7.14458996780566\\
22.7736310702075	7.14420838012243\\
22.7729998970329	7.14382677567459\\
22.7723686779898	7.14344515446067\\
22.7717374130717	7.14306351647919\\
22.7711061022717	7.14268186172868\\
22.7704747455832	7.14230019020767\\
22.7698433429995	7.14191850191468\\
22.769211894514	7.14153679684823\\
22.7685804001201	7.14115507500686\\
22.7679488598109	7.14077333638909\\
22.7673172735799	7.14039158099344\\
22.7666856414204	7.14000980881845\\
22.7660539633256	7.13962801986262\\
22.765422239289	7.1392462141245\\
22.7647904693038	7.1388643916026\\
22.7641586533633	7.13848255229544\\
22.7635267914608	7.13810069620156\\
22.7628948835897	7.13771882331948\\
22.7622629297433	7.13733693364771\\
22.7616309299149	7.13695502718479\\
22.7609988840978	7.13657310392924\\
22.7603667922853	7.13619116387957\\
22.7597346544706	7.13580920703432\\
22.7591024706472	7.135427233392\\
22.7584702408083	7.13504524295114\\
22.7578379649472	7.13466323571027\\
22.7572056430572	7.13428121166789\\
22.7565732751316	7.13389917082254\\
22.7559408611637	7.13351711317273\\
22.7553084011468	7.133135038717\\
22.7546758950743	7.13275294745385\\
22.7540433429392	7.13237083938182\\
22.7534107447351	7.13198871449942\\
22.7527781004551	7.13160657280517\\
22.7521454100926	7.13122441429759\\
22.7515126736408	7.13084223897521\\
22.7508798910931	7.13046004683654\\
22.7502470624426	7.1300778378801\\
22.7496141876827	7.12969561210442\\
22.7489812668067	7.12931336950801\\
22.7483482998078	7.1289311100894\\
22.7477152866794	7.12854883384709\\
22.7470822274147	7.12816654077962\\
22.7464491220069	7.12778423088551\\
22.7458159704494	7.12740190416325\\
22.7451827727354	7.12701956061139\\
22.7445495288581	7.12663720022843\\
22.743916238811	7.1262548230129\\
22.7432829025871	7.12587242896331\\
22.7426495201799	7.12549001807817\\
22.7420160915825	7.12510759035602\\
22.7413826167882	7.12472514579535\\
22.7407490957902	7.1243426843947\\
22.740115528582	7.12396020615258\\
22.7394819151566	7.1235777110675\\
22.7388482555073	7.12319519913799\\
22.7382145496275	7.12281267036255\\
22.7375807975103	7.12243012473971\\
22.7369469991491	7.12204756226798\\
22.736313154537	7.12166498294587\\
22.7356792636673	7.1212823867719\\
22.7350453265333	7.12089977374459\\
22.7344113431281	7.12051714386245\\
22.7337773134452	7.120134497124\\
22.7331432374776	7.11975183352775\\
22.7325091152186	7.11936915307222\\
22.7318749466616	7.11898645575592\\
22.7312407317996	7.11860374157736\\
22.730606470626	7.11822101053506\\
22.729972163134	7.11783826262753\\
22.7293378093168	7.11745549785329\\
22.7287034091677	7.11707271621085\\
22.7280689626798	7.11668991769872\\
22.7274344698465	7.11630710231542\\
22.7267999306609	7.11592427005945\\
22.7261653451163	7.11554142092934\\
22.7255307132058	7.11515855492358\\
22.7248960349228	7.11477567204071\\
22.7242613102605	7.11439277227922\\
22.723626539212	7.11400985563763\\
22.7229917217706	7.11362692211446\\
22.7223568579295	7.1132439717082\\
22.7217219476819	7.11286100441738\\
22.7210869910211	7.1124780202405\\
22.7204519879402	7.11209501917608\\
22.7198169384325	7.11171200122263\\
22.7191818424911	7.11132896637865\\
22.7185467001094	7.11094591464266\\
22.7179115112804	7.11056284601316\\
22.7172762759974	7.11017976048867\\
22.7166409942536	7.1097966580677\\
22.7160056660422	7.10941353874876\\
22.7153702913564	7.10903040253035\\
22.7147348701895	7.10864724941098\\
22.7140994025345	7.10826407938917\\
22.7134638883847	7.10788089246341\\
22.7128283277334	7.10749768863224\\
22.7121927205736	7.10711446789413\\
22.7115570668986	7.10673123024761\\
22.7109213667016	7.10634797569119\\
22.7102856199757	7.10596470422337\\
22.7096498267142	7.10558141584266\\
22.7090139869103	7.10519811054757\\
22.708378100557	7.1048147883366\\
22.7077421676477	7.10443144920826\\
22.7071061881755	7.10404809316106\\
22.7064701621335	7.1036647201935\\
22.705834089515	7.10328133030409\\
22.7051979703132	7.10289792349134\\
22.7045618045211	7.10251449975375\\
22.703925592132	7.10213105908983\\
22.7032893331391	7.10174760149808\\
22.7026530275355	7.10136412697701\\
22.7020166753144	7.10098063552513\\
22.701380276469	7.10059712714093\\
22.7007438309924	7.10021360182293\\
22.7001073388777	7.09983005956962\\
22.6994708001183	7.09944650037952\\
22.6988342147072	7.09906292425112\\
22.6981975826376	7.09867933118294\\
22.6975609039026	7.09829572117346\\
22.6969241784954	7.09791209422121\\
22.6962874064092	7.09752845032468\\
22.6956505876371	7.09714478948237\\
22.6950137221723	7.09676111169278\\
22.6943768100079	7.09637741695443\\
22.6937398511371	7.09599370526581\\
22.693102845553	7.09560997662542\\
22.6924657932488	7.09522623103177\\
22.6918286942175	7.09484246848336\\
22.6911915484525	7.09445868897869\\
22.6905543559468	7.09407489251626\\
22.6899171166935	7.09369107909457\\
22.6892798306858	7.09330724871213\\
22.6886424979168	7.09292340136743\\
22.6880051183797	7.09253953705898\\
22.6873676920676	7.09215565578527\\
22.6867302189736	7.09177175754482\\
22.6860926990909	7.09138784233611\\
22.6854551324126	7.09100391015764\\
22.6848175189318	7.09061996100792\\
22.6841798586417	7.09023599488545\\
22.6835421515353	7.08985201178873\\
22.6829043976059	7.08946801171625\\
22.6822665968465	7.08908399466651\\
22.6816287492502	7.08869996063802\\
22.6809908548103	7.08831590962926\\
22.6803529135197	7.08793184163875\\
22.6797149253716	7.08754775666497\\
22.6790768903591	7.08716365470643\\
22.6784388084754	7.08677953576162\\
22.6778006797136	7.08639539982904\\
22.6771625040667	7.08601124690719\\
22.6765242815279	7.08562707699457\\
22.6758860120903	7.08524289008967\\
22.6752476957469	7.08485868619098\\
22.674609332491	7.08447446529702\\
22.6739709223155	7.08409022740626\\
22.6733324652137	7.08370597251721\\
22.6726939611786	7.08332170062837\\
22.6720554102032	7.08293741173822\\
22.6714168122808	7.08255310584528\\
22.6707781674044	7.08216878294802\\
22.670139475567	7.08178444304495\\
22.6695007367618	7.08140008613455\\
22.668861950982	7.08101571221534\\
22.6682231182204	7.08063132128579\\
22.6675842384704	7.08024691334441\\
22.6669453117249	7.07986248838969\\
22.666306337977	7.07947804642012\\
22.6656673172198	7.07909358743419\\
22.6650282494464	7.07870911143041\\
22.6643891346499	7.07832461840726\\
22.6637499728233	7.07794010836324\\
22.6631107639598	7.07755558129683\\
22.6624715080524	7.07717103720654\\
22.6618322050942	7.07678647609085\\
22.6611928550782	7.07640189794826\\
22.6605534579975	7.07601730277726\\
22.6599140138453	7.07563269057634\\
22.6592745226145	7.07524806134399\\
22.6586349842983	7.07486341507871\\
22.6579953988897	7.07447875177898\\
22.6573557663817	7.0740940714433\\
22.6567160867674	7.07370937407016\\
22.65607636004	7.07332465965805\\
22.6554365861924	7.07293992820545\\
22.6547967652177	7.07255517971087\\
22.654156897109	7.07217041417278\\
22.6535169818592	7.07178563158969\\
22.6528770194616	7.07140083196007\\
22.6522370099091	7.07101601528243\\
22.6515969531947	7.07063118155524\\
22.6509568493116	7.070246330777\\
22.6503166982528	7.0698614629462\\
22.6496765000112	7.06947657806132\\
22.6490362545801	7.06909167612085\\
22.6483959619523	7.06870675712329\\
22.647755622121	7.06832182106712\\
22.6471152350791	7.06793686795083\\
22.6464748008198	7.0675518977729\\
22.6458343193361	7.06716691053182\\
22.6451937906209	7.06678190622609\\
22.6445532146674	7.06639688485419\\
22.6439125914685	7.0660118464146\\
22.6432719210173	7.06562679090581\\
22.6426312033068	7.06524171832631\\
22.6419904383301	7.06485662867458\\
22.6413496260801	7.06447152194912\\
22.64070876655	7.0640863981484\\
22.6400678597326	7.06370125727092\\
22.6394269056211	7.06331609931515\\
22.6387859042085	7.06293092427959\\
22.6381448554877	7.06254573216271\\
22.6375037594519	7.06216052296301\\
22.6368626160939	7.06177529667896\\
22.6362214254068	7.06139005330906\\
22.6355801873837	7.06100479285179\\
22.6349389020176	7.06061951530562\\
22.6342975693014	7.06023422066905\\
22.6336561892281	7.05984890894056\\
22.6330147617908	7.05946358011864\\
22.6323732869825	7.05907823420175\\
22.6317317647962	7.0586928711884\\
22.6310901952248	7.05830749107706\\
22.6304485782614	7.05792209386621\\
22.629806913899	7.05753667955434\\
22.6291652021305	7.05715124813993\\
22.628523442949	7.05676579962145\\
22.6278816363475	7.05638033399741\\
22.6272397823189	7.05599485126626\\
22.6265978808562	7.05560935142651\\
22.6259559319525	7.05522383447661\\
22.6253139356006	7.05483830041508\\
22.6246718917937	7.05445274924036\\
22.6240298005247	7.05406718095096\\
22.6233876617865	7.05368159554535\\
22.6227454755721	7.05329599302201\\
22.6221032418746	7.05291037337942\\
22.6214609606869	7.05252473661606\\
22.620818632002	7.05213908273042\\
22.6201762558128	7.05175341172096\\
22.6195338321123	7.05136772358617\\
22.6188913608936	7.05098201832453\\
22.6182488421495	7.05059629593452\\
22.617606275873	7.05021055641461\\
22.6169636620572	7.04982479976329\\
22.6163210006949	7.04943902597903\\
22.6156782917791	7.04905323506031\\
22.6150355353029	7.04866742700561\\
22.614392731259	7.04828160181341\\
22.6137498796406	7.04789575948218\\
22.6131069804406	7.0475099000104\\
22.6124640336519	7.04712402339655\\
22.6118210392675	7.0467381296391\\
22.6111779972803	7.04635221873653\\
22.6105349076833	7.04596629068733\\
22.6098917704694	7.04558034548995\\
22.6092485856316	7.04519438314289\\
22.6086053531628	7.04480840364461\\
22.607962073056	7.0444224069936\\
22.6073187453041	7.04403639318832\\
22.6066753699001	7.04365036222725\\
22.6060319468369	7.04326431410887\\
22.6053884761074	7.04287824883165\\
22.6047449577046	7.04249216639407\\
22.6041013916214	7.0421060667946\\
22.6034577778507	7.04171995003171\\
22.6028141163855	7.04133381610388\\
22.6021704072187	7.04094766500959\\
22.6015266503433	7.0405614967473\\
22.6008828457521	7.0401753113155\\
22.6002389934381	7.03978910871264\\
22.5995950933942	7.03940288893722\\
22.5989511456133	7.03901665198768\\
22.5983071500884	7.03863039786253\\
22.5976631068124	7.03824412656021\\
22.5970190157782	7.03785783807921\\
22.5963748769787	7.037471532418\\
22.5957306904068	7.03708520957505\\
22.5950864560555	7.03669886954883\\
22.5944421739176	7.03631251233781\\
22.593797843986	7.03592613794047\\
22.5931534662538	7.03553974635527\\
22.5925090407137	7.03515333758068\\
22.5918645673587	7.03476691161518\\
22.5912200461817	7.03438046845724\\
22.5905754771755	7.03399400810533\\
22.5899308603332	7.03360753055791\\
22.5892861956475	7.03322103581345\\
22.5886414831115	7.03283452387043\\
22.5879967227179	7.03244799472732\\
22.5873519144597	7.03206144838258\\
22.5867070583298	7.03167488483468\\
22.586062154321	7.03128830408209\\
22.5854172024263	7.03090170612328\\
22.5847722026385	7.03051509095673\\
22.5841271549506	7.03012845858088\\
22.5834820593553	7.02974180899423\\
22.5828369158457	7.02935514219522\\
22.5821917244145	7.02896845818233\\
22.5815464850547	7.02858175695403\\
22.5809011977592	7.02819503850878\\
22.5802558625207	7.02780830284505\\
22.5796104793322	7.02742154996131\\
22.5789650481866	7.02703477985602\\
22.5783195690767	7.02664799252765\\
22.5776740419955	7.02626118797467\\
22.5770284669357	7.02587436619554\\
22.5763828438902	7.02548752718873\\
22.575737172852	7.02510067095269\\
22.5750914538138	7.02471379748591\\
22.5744456867685	7.02432690678684\\
22.5737998717091	7.02393999885395\\
22.5731540086283	7.02355307368569\\
22.572508097519	7.02316613128055\\
22.5718621383741	7.02277917163697\\
22.5712161311864	7.02239219475343\\
22.5705700759488	7.02200520062839\\
22.5699239726541	7.02161818926031\\
22.5692778212952	7.02123116064766\\
22.5686316218649	7.02084411478889\\
22.5679853743561	7.02045705168247\\
22.5673390787616	7.02006997132687\\
22.5666927350743	7.01968287372055\\
22.566046343287	7.01929575886197\\
22.5653999033925	7.01890862674958\\
22.5647534153837	7.01852147738186\\
22.5641068792535	7.01813431075727\\
22.5634602949946	7.01774712687425\\
22.5628136625998	7.01735992573129\\
22.5621669820622	7.01697270732684\\
22.5615202533743	7.01658547165935\\
22.5608734765291	7.0161982187273\\
22.5602266515195	7.01581094852914\\
22.5595797783382	7.01542366106333\\
22.558932856978	7.01503635632832\\
22.5582858874319	7.01464903432259\\
22.5576388696925	7.01426169504459\\
22.5569918037528	7.01387433849278\\
22.5563446896055	7.01348696466562\\
22.5556975272434	7.01309957356157\\
22.5550503166595	7.01271216517909\\
22.5544030578464	7.01232473951663\\
22.553755750797	7.01193729657265\\
22.5531083955041	7.01154983634562\\
22.5524609919606	7.01116235883399\\
22.5518135401592	7.01077486403621\\
22.5511660400927	7.01038735195075\\
22.5505184917539	7.00999982257607\\
22.5498708951356	7.00961227591061\\
22.5492232502307	7.00922471195285\\
22.548575557032	7.00883713070122\\
22.5479278155321	7.0084495321542\\
22.547280025724	7.00806191631023\\
22.5466321876004	7.00767428316778\\
22.5459843011541	7.0072866327253\\
22.545336366378	7.00689896498124\\
22.5446883832647	7.00651127993406\\
22.5440403518071	7.00612357758222\\
22.5433922719979	7.00573585792417\\
22.54274414383	7.00534812095836\\
22.5420959672962	7.00496036668325\\
22.5414477423891	7.0045725950973\\
22.5407994691016	7.00418480619896\\
22.5401511474265	7.00379699998668\\
22.5395027773566	7.00340917645892\\
22.5388543588846	7.00302133561413\\
22.5382058920032	7.00263347745077\\
22.5375573767053	7.00224560196728\\
22.5369088129837	7.00185770916212\\
22.5362602008311	7.00146979903374\\
22.5356115402402	7.0010818715806\\
22.5349628312038	7.00069392680115\\
22.5343140737148	7.00030596469384\\
22.5336652677658	6.99991798525713\\
22.5330164133496	6.99952998848945\\
22.532367510459	6.99914197438928\\
22.5317185590868	6.99875394295505\\
22.5310695592256	6.99836589418522\\
22.5304205108682	6.99797782807823\\
22.5297714140075	6.99758974463255\\
22.5291222686361	6.99720164384662\\
22.5284730747468	6.99681352571889\\
22.5278238323323	6.99642539024782\\
22.5271745413854	6.99603723743184\\
22.5265252018988	6.99564906726942\\
22.5258758138653	6.99526087975899\\
22.5252263772776	6.99487267489902\\
22.5245768921285	6.99448445268794\\
22.5239273584106	6.99409621312421\\
22.5232777761168	6.99370795620628\\
22.5226281452397	6.99331968193259\\
22.5219784657721	6.9929313903016\\
22.5213287377067	6.99254308131175\\
22.5206789610363	6.99215475496148\\
22.5200291357535	6.99176641124926\\
22.5193792618511	6.99137805017352\\
22.5187293393219	6.9909896717327\\
22.5180793681585	6.99060127592527\\
22.5174293483536	6.99021286274967\\
22.5167792799001	6.98982443220433\\
22.5161291627905	6.98943598428771\\
22.5154789970177	6.98904751899826\\
22.5148287825743	6.98865903633442\\
22.5141785194531	6.98827053629463\\
22.5135282076467	6.98788201887735\\
22.5128778471478	6.98749348408101\\
22.5122274379493	6.98710493190407\\
22.5115769800437	6.98671636234496\\
22.5109264734238	6.98632777540213\\
22.5102759180823	6.98593917107403\\
22.5096253140119	6.98555054935911\\
22.5089746612053	6.98516191025579\\
22.5083239596551	6.98477325376254\\
22.5076732093541	6.98438457987778\\
22.507022410295	6.98399588859998\\
22.5063715624705	6.98360717992756\\
22.5057206658733	6.98321845385897\\
22.505069720496	6.98282971039266\\
22.5044187263313	6.98244094952707\\
22.5037676833719	6.98205217126064\\
22.5031165916106	6.9816633755918\\
22.50246545104	6.98127456251902\\
22.5018142616527	6.98088573204072\\
22.5011630234415	6.98049688415534\\
22.500511736399	6.98010801886134\\
22.4998604005179	6.97971913615715\\
22.4992090157909	6.9793302360412\\
22.4985575822107	6.97894131851195\\
22.4979060997698	6.97855238356784\\
22.4972545684611	6.97816343120729\\
22.4966029882772	6.97777446142876\\
22.4959513592107	6.97738547423068\\
22.4952996812542	6.97699646961149\\
22.4946479544006	6.97660744756964\\
22.4939961786424	6.97621840810355\\
22.4933443539722	6.97582935121168\\
22.4926924803828	6.97544027689245\\
22.4920405578668	6.97505118514432\\
22.4913885864168	6.97466207596571\\
22.4907365660255	6.97427294935506\\
22.4900844966856	6.97388380531082\\
22.4894323783898	6.97349464383142\\
22.4887802111305	6.9731054649153\\
22.4881279949006	6.97271626856089\\
22.4874757296926	6.97232705476663\\
22.4868234154992	6.97193782353096\\
22.486171052313	6.97154857485232\\
22.4855186401268	6.97115930872915\\
22.484866178933	6.97077002515987\\
22.4842136687243	6.97038072414292\\
22.4835611094935	6.96999140567675\\
22.4829085012331	6.96960206975979\\
22.4822558439357	6.96921271639047\\
22.481603137594	6.96882334556722\\
22.4809503822006	6.96843395728849\\
22.4802975777482	6.96804455155271\\
22.4796447242293	6.96765512835831\\
22.4789918216365	6.96726568770372\\
22.4783388699626	6.96687622958739\\
22.4776858692002	6.96648675400774\\
22.4770328193417	6.96609726096321\\
22.4763797203799	6.96570775045224\\
22.4757265723074	6.96531822247325\\
22.4750733751168	6.96492867702467\\
22.4744201288007	6.96453911410496\\
22.4737668333517	6.96414953371252\\
22.4731134887624	6.9637599358458\\
22.4724600950255	6.96337032050324\\
22.4718066521335	6.96298068768325\\
22.471153160079	6.96259103738428\\
22.4704996188546	6.96220136960475\\
22.469846028453	6.96181168434311\\
22.4691923888668	6.96142198159776\\
22.4685387000884	6.96103226136716\\
22.4678849621106	6.96064252364973\\
22.4672311749259	6.9602527684439\\
22.466577338527	6.9598629957481\\
22.4659234529063	6.95947320556077\\
22.4652695180565	6.95908339788032\\
22.4646155339702	6.9586935727052\\
22.46396150064	6.95830373003382\\
22.4633074180584	6.95791386986463\\
22.462653286218	6.95752399219605\\
22.4619991051115	6.95713409702651\\
22.4613448747313	6.95674418435443\\
22.4606905950701	6.95635425417825\\
22.4600362661205	6.95596430649639\\
22.4593818878749	6.95557434130729\\
22.4587274603261	6.95518435860936\\
22.4580729834665	6.95479435840104\\
22.4574184572887	6.95440434068076\\
22.4567638817853	6.95401430544694\\
22.4561092569489	6.95362425269802\\
22.455454582772	6.9532341824324\\
22.4547998592472	6.95284409464853\\
22.454145086367	6.95245398934483\\
22.453490264124	6.95206386651973\\
22.4528353925107	6.95167372617165\\
22.4521804715198	6.95128356829901\\
22.4515255011437	6.95089339290025\\
22.4508704813751	6.95050319997379\\
22.4502154122064	6.95011298951805\\
22.4495602936302	6.94972276153146\\
22.4489051256391	6.94933251601244\\
22.4482499082256	6.94894225295942\\
22.4475946413822	6.94855197237083\\
22.4469393251016	6.94816167424507\\
22.4462839593761	6.94777135858059\\
22.4456285441984	6.94738102537581\\
22.4449730795611	6.94699067462914\\
22.4443175654565	6.94660030633901\\
22.4436620018774	6.94620992050384\\
22.4430063888161	6.94581951712206\\
22.4423507262653	6.94542909619209\\
22.4416950142174	6.94503865771235\\
22.441039252665	6.94464820168126\\
22.4403834416006	6.94425772809725\\
22.4397275810168	6.94386723695874\\
22.439071670906	6.94347672826415\\
22.4384157112608	6.94308620201189\\
22.4377597020736	6.9426956582004\\
22.4371036433371	6.94230509682809\\
22.4364475350437	6.94191451789337\\
22.4357913771859	6.94152392139469\\
22.4351351697563	6.94113330733045\\
22.4344789127473	6.94074267569907\\
22.4338226061515	6.94035202649897\\
22.4331662499614	6.93996135972858\\
22.4325098441694	6.93957067538631\\
22.4318533887681	6.93917997347058\\
22.43119688375	6.93878925397981\\
22.4305403291076	6.93839851691242\\
22.4298837248333	6.93800776226683\\
22.4292270709197	6.93761699004146\\
22.4285703673593	6.93722620023471\\
22.4279136141445	6.93683539284502\\
22.4272568112678	6.93644456787081\\
22.4265999587218	6.93605372531047\\
22.4259430564989	6.93566286516245\\
22.4252861045916	6.93527198742514\\
22.4246291029924	6.93488109209698\\
22.4239720516938	6.93449017917636\\
22.4233149506882	6.93409924866172\\
22.4226577999682	6.93370830055147\\
22.4220005995261	6.93331733484402\\
22.4213433493546	6.93292635153779\\
22.420686049446	6.9325353506312\\
22.4200286997928	6.93214433212266\\
22.4193713003875	6.93175329601058\\
22.4187138512226	6.93136224229338\\
22.4180563522905	6.93097117096948\\
22.4173988035838	6.93058008203729\\
22.4167412050947	6.93018897549522\\
22.4160835568159	6.92979785134169\\
22.4154258587398	6.92940670957511\\
22.4147681108589	6.92901555019389\\
22.4141103131655	6.92862437319646\\
22.4134524656522	6.92823317858122\\
22.4127945683113	6.92784196634658\\
22.4121366211355	6.92745073649096\\
22.411478624117	6.92705948901277\\
22.4108205772484	6.92666822391042\\
22.4101624805221	6.92627694118233\\
22.4095043339305	6.92588564082691\\
22.4088461374661	6.92549432284256\\
22.4081878911214	6.92510298722771\\
22.4075295948887	6.92471163398075\\
22.4068712487605	6.92432026310011\\
22.4062128527292	6.92392887458419\\
22.4055544067874	6.92353746843141\\
22.4048959109273	6.92314604464017\\
22.4042373651415	6.92275460320888\\
22.4035787694223	6.92236314413596\\
22.4029201237622	6.92197166741981\\
22.4022614281537	6.92158017305884\\
22.4016026825891	6.92118866105147\\
22.4009438870608	6.9207971313961\\
22.4002850415613	6.92040558409115\\
22.3996261460831	6.92001401913501\\
22.3989672006184	6.9196224365261\\
22.3983082051598	6.91923083626282\\
22.3976491596997	6.9188392183436\\
22.3969900642304	6.91844758276682\\
22.3963309187443	6.9180559295309\\
22.395671723234	6.91766425863425\\
22.3950124776917	6.91727257007528\\
22.3943531821099	6.91688086385239\\
22.393693836481	6.91648913996398\\
22.3930344407974	6.91609739840847\\
22.3923749950515	6.91570563918426\\
22.3917154992357	6.91531386228976\\
22.3910559533423	6.91492206772337\\
22.3903963573639	6.91453025548351\\
22.3897367112927	6.91413842556856\\
22.3890770151212	6.91374657797694\\
22.3884172688417	6.91335471270707\\
22.3877574724467	6.91296282975732\\
22.3870976259285	6.91257092912613\\
22.3864377292795	6.91217901081188\\
22.3857777824922	6.91178707481298\\
22.3851177855587	6.91139512112784\\
22.3844577384717	6.91100314975486\\
22.3837976412233	6.91061116069245\\
22.3831374938061	6.910219153939\\
22.3824772962123	6.90982712949292\\
22.3818170484344	6.90943508735261\\
22.3811567504647	6.90904302751648\\
22.3804964022957	6.90865094998293\\
22.3798360039195	6.90825885475036\\
22.3791755553287	6.90786674181717\\
22.3785150565155	6.90747461118177\\
22.3778545074724	6.90708246284255\\
22.3771939081917	6.90669029679792\\
22.3765332586658	6.90629811304627\\
22.375872558887	6.90590591158602\\
22.3752118088476	6.90551369241555\\
22.3745510085401	6.90512145553328\\
22.3738901579568	6.90472920093759\\
22.3732292570899	6.9043369286269\\
22.372568305932	6.90394463859959\\
22.3719073044752	6.90355233085408\\
22.3712462527121	6.90316000538876\\
22.3705851506348	6.90276766220202\\
22.3699239982358	6.90237530129227\\
22.3692627955073	6.90198292265791\\
22.3686015424418	6.90159052629733\\
22.3679402390315	6.90119811220894\\
22.3672788852688	6.90080568039113\\
22.3666174811461	6.90041323084229\\
22.3659560266555	6.90002076356084\\
22.3652945217896	6.89962827854516\\
22.3646329665405	6.89923577579365\\
22.3639713609007	6.89884325530471\\
22.3633097048625	6.89845071707673\\
22.3626479984181	6.89805816110812\\
22.3619862415599	6.89766558739726\\
22.3613244342802	6.89727299594256\\
22.3606625765713	6.89688038674241\\
22.3600006684256	6.8964877597952\\
22.3593387098353	6.89609511509934\\
22.3586767007928	6.89570245265321\\
22.3580146412904	6.89530977245522\\
22.3573525313203	6.89491707450375\\
22.3566903708749	6.8945243587972\\
22.3560281599466	6.89413162533397\\
22.3553658985274	6.89373887411245\\
22.3547035866099	6.89334610513103\\
22.3540412241863	6.89295331838811\\
22.3533788112489	6.89256051388209\\
22.3527163477899	6.89216769161134\\
22.3520538338017	6.89177485157428\\
22.3513912692765	6.89138199376928\\
22.3507286542068	6.89098911819475\\
22.3500659885846	6.89059622484907\\
22.3494032724024	6.89020331373064\\
22.3487405056524	6.88981038483786\\
22.3480776883269	6.8894174381691\\
22.3474148204181	6.88902447372276\\
22.3467519019184	6.88863149149724\\
22.3460889328201	6.88823849149093\\
22.3454259131153	6.88784547370221\\
22.3447628427964	6.88745243812948\\
22.3440997218557	6.88705938477113\\
22.3434365502853	6.88666631362555\\
22.3427733280777	6.88627322469112\\
22.342110055225	6.88588011796624\\
22.3414467317196	6.8854869934493\\
22.3407833575536	6.88509385113869\\
22.3401199327193	6.88470069103279\\
22.3394564572091	6.88430751313\\
22.3387929310151	6.8839143174287\\
22.3381293541297	6.88352110392728\\
22.337465726545	6.88312787262413\\
22.3368020482533	6.88273462351765\\
22.3361383192469	6.88234135660621\\
22.335474539518	6.8819480718882\\
22.3348107090589	6.88155476936201\\
22.3341468278618	6.88116144902604\\
22.3334828959189	6.88076811087865\\
22.3328189132226	6.88037475491825\\
22.332154879765	6.87998138114323\\
22.3314907955383	6.87958798955195\\
22.3308266605349	6.87919458014282\\
22.3301624747469	6.87880115291421\\
22.3294982381666	6.87840770786452\\
22.3288339507862	6.87801424499212\\
22.328169612598	6.87762076429541\\
22.3275052235941	6.87722726577277\\
22.3268407837668	6.87683374942258\\
22.3261762931083	6.87644021524323\\
22.3255117516109	6.8760466632331\\
22.3248471592667	6.87565309339058\\
22.324182516068	6.87525950571405\\
22.323517822007	6.87486590020189\\
22.3228530770759	6.87447227685249\\
22.3221882812669	6.87407863566423\\
22.3215234345722	6.87368497663549\\
22.3208585369841	6.87329129976466\\
22.3201935884948	6.87289760505012\\
22.3195285890964	6.87250389249025\\
22.3188635387811	6.87211016208344\\
22.3181984375412	6.87171641382806\\
22.3175332853689	6.87132264772249\\
22.3168680822563	6.87092886376513\\
22.3162028281957	6.87053506195435\\
22.3155375231793	6.87014124228852\\
22.3148721671992	6.86974740476604\\
22.3142067602476	6.86935354938528\\
22.3135413023167	6.86895967614463\\
22.3128757933988	6.86856578504246\\
22.312210233486	6.86817187607715\\
22.3115446225705	6.86777794924708\\
22.3108789606444	6.86738400455063\\
22.3102132477	6.86699004198619\\
22.3095474837294	6.86659606155213\\
22.3088816687248	6.86620206324683\\
22.3082158026784	6.86580804706866\\
22.3075498855823	6.86541401301601\\
22.3068839174288	6.86501996108726\\
22.30621789821	6.86462589128078\\
22.305551827918	6.86423180359495\\
22.3048857065451	6.86383769802814\\
22.3042195340834	6.86344357457875\\
22.303553310525	6.86304943324513\\
22.3028870358621	6.86265527402568\\
22.302220710087	6.86226109691875\\
22.3015543331916	6.86186690192275\\
22.3008879051683	6.86147268903602\\
22.3002214260091	6.86107845825697\\
22.2995548957062	6.86068420958395\\
22.2988883142517	6.86028994301535\\
22.2982216816378	6.85989565854954\\
22.2975549978567	6.8595013561849\\
22.2968882629005	6.85910703591979\\
22.2962214767613	6.85871269775261\\
22.2955546394312	6.85831834168171\\
22.2948877509025	6.85792396770548\\
22.2942208111672	6.85752957582228\\
22.2935538202176	6.8571351660305\\
22.2928867780456	6.8567407383285\\
22.2922196846435	6.85634629271466\\
22.2915525400033	6.85595182918735\\
22.2908853441173	6.85555734774495\\
22.2902180969775	6.85516284838582\\
22.2895507985761	6.85476833110834\\
22.2888834489051	6.85437379591089\\
22.2882160479567	6.85397924279183\\
22.2875485957231	6.85358467174954\\
22.2868810921963	6.85319008278238\\
22.2862135373685	6.85279547588873\\
22.2855459312317	6.85240085106696\\
22.2848782737781	6.85200620831544\\
22.2842105649999	6.85161154763254\\
22.283542804889	6.85121686901664\\
22.2828749934376	6.85082217246609\\
22.2822071306378	6.85042745797928\\
22.2815392164818	6.85003272555457\\
22.2808712509615	6.84963797519033\\
22.2802032340692	6.84924320688493\\
22.2795351657969	6.84884842063673\\
22.2788670461367	6.84845361644412\\
22.2781988750807	6.84805879430546\\
22.277530652621	6.84766395421911\\
22.2768623787497	6.84726909618344\\
22.2761940534589	6.84687422019683\\
22.2755256767406	6.84647932625763\\
22.274857248587	6.84608441436423\\
22.2741887689901	6.84568948451497\\
22.273520237942	6.84529453670824\\
22.2728516554348	6.8448995709424\\
22.2721830214606	6.84450458721581\\
22.2715143360114	6.84410958552684\\
22.2708455990793	6.84371456587386\\
22.2701768106564	6.84331952825523\\
22.2695079707347	6.84292447266932\\
22.2688390793064	6.8425293991145\\
22.2681701363635	6.84213430758913\\
22.267501141898	6.84173919809157\\
22.2668320959021	6.84134407062019\\
22.2661629983677	6.84094892517336\\
22.265493849287	6.84055376174943\\
22.264824648652	6.84015858034678\\
22.2641553964547	6.83976338096377\\
22.2634860926872	6.83936816359876\\
22.2628167373416	6.83897292825011\\
22.2621473304099	6.83857767491619\\
22.2614778718842	6.83818240359536\\
22.2608083617564	6.83778711428598\\
22.2601388000187	6.83739180698642\\
22.2594691866631	6.83699648169504\\
22.2587995216816	6.8366011384102\\
22.2581298050663	6.83620577713027\\
22.2574600368092	6.8358103978536\\
22.2567902169023	6.83541500057855\\
22.2561203453377	6.83501958530349\\
22.2554504221074	6.83462415202678\\
22.2547804472034	6.83422870074678\\
22.2541104206179	6.83383323146185\\
22.2534403423426	6.83343774417035\\
22.2527702123698	6.83304223887064\\
22.2521000306915	6.83264671556108\\
22.2514297972996	6.83225117424004\\
22.2507595121862	6.83185561490586\\
22.2500891753432	6.83146003755691\\
22.2494187867628	6.83106444219155\\
22.2487483464369	6.83066882880813\\
22.2480778543575	6.83027319740502\\
22.2474073105166	6.82987754798058\\
22.2467367149063	6.82948188053315\\
22.2460660675186	6.82908619506111\\
22.2453953683454	6.8286904915628\\
22.2447246173788	6.82829477003659\\
22.2440538146107	6.82789903048083\\
22.2433829600332	6.82750327289388\\
22.2427120536382	6.8271074972741\\
22.2420410954178	6.82671170361984\\
22.2413700853639	6.82631589192946\\
22.2406990234686	6.82592006220131\\
22.2400279097238	6.82552421443376\\
22.2393567441215	6.82512834862515\\
22.2386855266536	6.82473246477385\\
22.2380142573123	6.8243365628782\\
22.2373429360894	6.82394064293656\\
22.2366715629769	6.82354470494729\\
22.2360001379669	6.82314874890875\\
22.2353286610512	6.82275277481928\\
22.234657132222	6.82235678267724\\
22.233985551471	6.82196077248098\\
22.2333139187904	6.82156474422886\\
22.232642234172	6.82116869791924\\
22.2319704976079	6.82077263355046\\
22.23129870909	6.82037655112087\\
22.2306268686102	6.81998045062884\\
22.2299549761606	6.81958433207271\\
22.2292830317331	6.81918819545083\\
22.2286110353196	6.81879204076156\\
22.2279389869121	6.81839586800325\\
22.2272668865026	6.81799967717425\\
22.226594734083	6.81760346827291\\
22.2259225296452	6.81720724129759\\
22.2252502731813	6.81681099624663\\
22.2245779646831	6.81641473311839\\
22.2239056041426	6.81601845191121\\
22.2232331915518	6.81562215262344\\
22.2225607269025	6.81522583525344\\
22.2218882101867	6.81482949979956\\
22.2212156413964	6.81443314626014\\
22.2205430205235	6.81403677463354\\
22.2198703475599	6.8136403849181\\
22.2191976224976	6.81324397711217\\
22.2185248453284	6.81284755121411\\
22.2178520160444	6.81245110722225\\
22.2171791346373	6.81205464513495\\
22.2165062010993	6.81165816495056\\
22.2158332154221	6.81126166666742\\
22.2151601775978	6.81086515028388\\
22.2144870876181	6.8104686157983\\
22.2138139454751	6.810072063209\\
22.2131407511606	6.80967549251435\\
22.2124675046666	6.80927890371269\\
22.211794205985	6.80888229680237\\
22.2111208551076	6.80848567178172\\
22.2104474520265	6.80808902864911\\
22.2097739967334	6.80769236740287\\
22.2091004892203	6.80729568804134\\
22.2084269294791	6.80689899056289\\
22.2077533175017	6.80650227496584\\
22.20707965328	6.80610554124854\\
22.2064059368059	6.80570878940935\\
22.2057321680713	6.8053120194466\\
22.2050583470681	6.80491523135863\\
22.2043844737881	6.8045184251438\\
22.2037105482232	6.80412160080044\\
22.2030365703654	6.8037247583269\\
22.2023625402066	6.80332789772153\\
22.2016884577385	6.80293101898266\\
22.2010143229531	6.80253412210863\\
22.2003401358422	6.8021372070978\\
22.1996658963978	6.8017402739485\\
22.1989916046118	6.80134332265907\\
22.1983172604759	6.80094635322786\\
22.197642863982	6.80054936565321\\
22.1969684151221	6.80015235993346\\
22.196293913888	6.79975533606695\\
22.1956193602715	6.79935829405202\\
22.1949447542646	6.79896123388701\\
22.194270095859	6.79856415557027\\
22.1935953850467	6.79816705910014\\
22.1929206218195	6.79776994447494\\
22.1922458061692	6.79737281169303\\
22.1915709380877	6.79697566075274\\
22.1908960175669	6.79657849165241\\
22.1902210445986	6.79618130439039\\
22.1895460191747	6.79578409896501\\
22.188870941287	6.7953868753746\\
22.1881958109273	6.79498963361751\\
22.1875206280875	6.79459237369209\\
22.1868453927595	6.79419509559665\\
22.186170104935	6.79379779932954\\
22.1854947646059	6.79340048488911\\
22.1848193717641	6.79300315227368\\
22.1841439264013	6.79260580148159\\
22.1834684285094	6.79220843251119\\
22.1827928780803	6.7918110453608\\
22.1821172751058	6.79141364002877\\
22.1814416195776	6.79101621651343\\
22.1807659114876	6.79061877481311\\
22.1800901508277	6.79022131492616\\
22.1794143375896	6.7898238368509\\
22.1787384717652	6.78942634058567\\
22.1780625533463	6.78902882612881\\
22.1773865823246	6.78863129347866\\
22.1767105586921	6.78823374263354\\
22.1760344824405	6.78783617359179\\
22.1753583535616	6.78743858635175\\
22.1746821720473	6.78704098091174\\
22.1740059378893	6.78664335727011\\
22.1733296510794	6.78624571542519\\
22.1726533116095	6.78584805537531\\
22.1719769194713	6.7854503771188\\
22.1713004746567	6.78505268065399\\
22.1706239771574	6.78465496597922\\
22.1699474269652	6.78425723309283\\
22.1692708240719	6.78385948199313\\
22.1685941684693	6.78346171267847\\
22.1679174601492	6.78306392514718\\
22.1672406991034	6.78266611939758\\
22.1665638853237	6.78226829542801\\
22.1658870188018	6.7818704532368\\
22.1652100995295	6.78147259282228\\
22.1645331274986	6.78107471418278\\
22.1638561027008	6.78067681731663\\
22.1631790251281	6.78027890222217\\
22.162501894772	6.77988096889771\\
22.1618247116244	6.77948301734159\\
22.1611474756771	6.77908504755215\\
22.1604701869218	6.7786870595277\\
22.1597928453502	6.77828905326658\\
22.1591154509543	6.77789102876711\\
22.1584380037256	6.77749298602763\\
22.157760503656	6.77709492504646\\
22.1570829507372	6.77669684582193\\
22.156405344961	6.77629874835237\\
22.1557276863191	6.77590063263611\\
22.1550499748033	6.77550249867147\\
22.1543722104053	6.77510434645678\\
22.1536943931168	6.77470617599036\\
22.1530165229297	6.77430798727055\\
22.1523385998357	6.77390978029567\\
22.1516606238264	6.77351155506405\\
22.1509825948937	6.773113311574\\
22.1503045130293	6.77271504982387\\
22.1496263782249	6.77231676981197\\
22.1489481904722	6.77191847153662\\
22.148269949763	6.77152015499616\\
22.147591656089	6.7711218201889\\
22.1469133094419	6.77072346711318\\
22.1462349098135	6.77032509576732\\
22.1455564571954	6.76992670614963\\
22.1448779515795	6.76952829825845\\
22.1441993929574	6.7691298720921\\
22.1435207813208	6.76873142764889\\
22.1428421166614	6.76833296492716\\
22.1421633989711	6.76793448392524\\
22.1414846282414	6.76753598464142\\
22.1408058044641	6.76713746707405\\
22.1401269276309	6.76673893122145\\
22.1394479977335	6.76634037708193\\
22.1387690147636	6.76594180465382\\
22.1380899787129	6.76554321393544\\
22.1374108895731	6.76514460492511\\
22.1367317473359	6.76474597762115\\
22.136052551993	6.76434733202189\\
22.1353733035361	6.76394866812564\\
22.1346940019569	6.76354998593072\\
22.1340146472471	6.76315128543545\\
22.1333352393983	6.76275256663816\\
22.1326557784023	6.76235382953716\\
22.1319762642507	6.76195507413078\\
22.1312966969353	6.76155630041733\\
22.1306170764476	6.76115750839512\\
22.1299374027794	6.76075869806249\\
22.1292576759224	6.76035986941774\\
22.1285778958682	6.7599610224592\\
22.1278980626085	6.75956215718519\\
22.127218176135	6.75916327359402\\
22.1265382364393	6.758764371684\\
22.1258582435132	6.75836545145347\\
22.1251781973481	6.75796651290072\\
22.124498097936	6.75756755602409\\
22.1238179452683	6.75716858082188\\
22.1231377393368	6.75676958729243\\
22.1224574801331	6.75637057543402\\
22.1217771676488	6.755971545245\\
22.1210968018757	6.75557249672367\\
22.1204163828054	6.75517342986834\\
22.1197359104295	6.75477434467734\\
22.1190553847396	6.75437524114897\\
22.1183748057275	6.75397611928156\\
22.1176941733847	6.75357697907341\\
22.1170134877029	6.75317782052284\\
22.1163327486737	6.75277864362817\\
22.1156519562888	6.75237944838771\\
22.1149711105399	6.75198023479977\\
22.1142902114184	6.75158100286267\\
22.1136092589162	6.75118175257472\\
22.1129282530247	6.75078248393422\\
22.1122471937357	6.7503831969395\\
22.1115660810407	6.74998389158887\\
22.1108849149315	6.74958456788064\\
22.1102036953995	6.74918522581312\\
22.1095224224364	6.74878586538462\\
22.1088410960339	6.74838648659346\\
22.1081597161835	6.74798708943795\\
22.1074782828769	6.74758767391639\\
22.1067967961057	6.74718824002709\\
22.1061152558615	6.74678878776838\\
22.1054336621359	6.74638931713856\\
22.1047520149205	6.74598982813593\\
22.1040703142068	6.74559032075881\\
22.1033885599866	6.74519079500551\\
22.1027067522514	6.74479125087434\\
22.1020248909928	6.74439168836361\\
22.1013429762025	6.74399210747162\\
22.1006610078719	6.74359250819669\\
22.0999789859927	6.74319289053711\\
22.0992969105565	6.74279325449121\\
22.0986147815548	6.7423936000573\\
22.0979325989793	6.74199392723366\\
22.0972503628215	6.74159423601863\\
22.0965680730731	6.74119452641049\\
22.0958857297256	6.74079479840756\\
22.0952033327705	6.74039505200815\\
22.0945208821995	6.73999528721056\\
22.0938383780041	6.7395955040131\\
22.0931558201759	6.73919570241408\\
22.0924732087065	6.7387958824118\\
22.0917905435874	6.73839604400456\\
22.0911078248103	6.73799618719068\\
22.0904250523666	6.73759631196845\\
22.0897422262479	6.73719641833619\\
22.0890593464458	6.73679650629219\\
22.0883764129519	6.73639657583476\\
22.0876934257577	6.73599662696221\\
22.0870103848547	6.73559665967284\\
22.0863272902346	6.73519667396496\\
22.0856441418888	6.73479666983686\\
22.0849609398089	6.73439664728685\\
22.0842776839865	6.73399660631324\\
22.083594374413	6.73359654691432\\
22.0829110110802	6.7331964690884\\
22.0822275939794	6.73279637283378\\
22.0815441231022	6.73239625814877\\
22.0808605984402	6.73199612503166\\
22.0801770199849	6.73159597348076\\
22.0794933877278	6.73119580349437\\
22.0788097016604	6.73079561507079\\
22.0781259617744	6.73039540820832\\
22.0774421680611	6.72999518290527\\
22.0767583205122	6.72959493915992\\
22.0760744191191	6.72919467697059\\
22.0753904638735	6.72879439633557\\
22.0747064547667	6.72839409725316\\
22.0740223917903	6.72799377972167\\
22.0733382749358	6.72759344373938\\
22.0726541041948	6.72719308930461\\
22.0719698795588	6.72679271641565\\
22.0712856010191	6.7263923250708\\
22.0706012685675	6.72599191526835\\
22.0699168821953	6.7255914870066\\
22.0692324418941	6.72519104028386\\
22.0685479476554	6.72479057509842\\
22.0678633994706	6.72439009144858\\
22.0671787973313	6.72398958933264\\
22.0664941412289	6.72358906874888\\
22.065809431155	6.72318852969561\\
22.065124667101	6.72278797217113\\
22.0644398490585	6.72238739617374\\
22.0637549770189	6.72198680170171\\
22.0630700509736	6.72158618875336\\
22.0623850709143	6.72118555732698\\
22.0617000368323	6.72078490742087\\
22.0610149487191	6.72038423903332\\
22.0603298065663	6.71998355216262\\
22.0596446103653	6.71958284680707\\
22.0589593601075	6.71918212296496\\
22.0582740557844	6.71878138063459\\
22.0575886973876	6.71838061981425\\
22.0569032849084	6.71797984050223\\
22.0562178183383	6.71757904269684\\
22.0555322976688	6.71717822639636\\
22.0548467228914	6.71677739159908\\
22.0541610939975	6.7163765383033\\
22.0534754109785	6.71597566650731\\
22.052789673826	6.7155747762094\\
22.0521038825313	6.71517386740787\\
22.051418037086	6.71477294010101\\
22.0507321374814	6.7143719942871\\
22.050046183709	6.71397102996444\\
22.0493601757603	6.71357004713132\\
22.0486741136267	6.71316904578603\\
22.0479879972996	6.71276802592686\\
22.0473018267706	6.71236698755211\\
22.0466156020309	6.71196593066006\\
22.045929323072	6.711564855249\\
22.0452429898855	6.71116376131722\\
22.0445566024626	6.71076264886301\\
22.0438701607949	6.71036151788467\\
22.0431836648737	6.70996036838047\\
22.0424971146905	6.70955920034872\\
22.0418105102367	6.70915801378768\\
22.0411238515038	6.70875680869567\\
22.0404371384831	6.70835558507095\\
22.039750371166	6.70795434291183\\
22.039063549544	6.70755308221659\\
22.0383766736085	6.70715180298351\\
22.0376897433508	6.70675050521088\\
22.0370027587625	6.706349188897\\
22.0363157198349	6.70594785404013\\
22.0356286265594	6.70554650063859\\
22.0349414789273	6.70514512869063\\
22.0342542769302	6.70474373819457\\
22.0335670205594	6.70434232914868\\
22.0328797098063	6.70394090155123\\
22.0321923446622	6.70353945540053\\
22.0315049251187	6.70313799069486\\
22.030817451167	6.7027365074325\\
22.0301299227986	6.70233500561173\\
22.0294423400048	6.70193348523084\\
22.028754702777	6.70153194628811\\
22.0280670111066	6.70113038878183\\
22.027379264985	6.70072881271028\\
22.0266914644036	6.70032721807174\\
22.0260036093537	6.6999256048645\\
22.0253156998267	6.69952397308684\\
22.024627735814	6.69912232273704\\
22.0239397173069	6.69872065381338\\
22.0232516442968	6.69831896631415\\
22.0225635167751	6.69791726023763\\
22.0218753347332	6.69751553558209\\
22.0211870981623	6.69711379234583\\
22.0204988070539	6.69671203052712\\
22.0198104613993	6.69631025012424\\
22.0191220611899	6.69590845113548\\
22.0184336064169	6.69550663355911\\
22.0177450970719	6.69510479739341\\
22.017056533146	6.69470294263667\\
22.0163679146307	6.69430106928716\\
22.0156792415173	6.69389917734316\\
22.0149905137972	6.69349726680296\\
22.0143017314616	6.69309533766483\\
22.013612894502	6.69269338992704\\
22.0129240029096	6.69229142358789\\
22.0122350566758	6.69188943864564\\
22.0115460557919	6.69148743509858\\
22.0108570002493	6.69108541294498\\
22.0101678900392	6.69068337218312\\
22.0094787251531	6.69028131281127\\
22.0087895055822	6.68987923482772\\
22.0081002313178	6.68947713823075\\
22.0074109023513	6.68907502301862\\
22.0067215186739	6.68867288918961\\
22.0060320802771	6.688270736742\\
22.005342587152	6.68786856567407\\
22.0046530392901	6.6874663759841\\
22.0039634366826	6.68706416767035\\
22.0032737793209	6.6866619407311\\
22.0025840671962	6.68625969516462\\
22.0018943002998	6.6858574309692\\
22.0012044786231	6.68545514814311\\
22.0005146021573	6.68505284668461\\
21.9998246708937	6.68465052659199\\
21.9991346848237	6.68424818786352\\
21.9984446439384	6.68384583049747\\
21.9977545482293	6.68344345449212\\
21.9970643976876	6.68304105984573\\
21.9963741923046	6.68263864655658\\
21.9956839320716	6.68223621462294\\
21.9949936169797	6.68183376404309\\
21.9943032470205	6.6814312948153\\
21.993612822185	6.68102880693784\\
21.9929223424646	6.68062630040897\\
21.9922318078506	6.68022377522698\\
21.9915412183342	6.67982123139013\\
21.9908505739067	6.6794186688967\\
21.9901598745593	6.67901608774494\\
21.9894691202834	6.67861348793314\\
21.9887783110701	6.67821086945957\\
21.9880874469108	6.6778082323225\\
21.9873965277968	6.67740557652018\\
21.9867055537192	6.6770029020509\\
21.9860145246693	6.67660020891292\\
21.9853234406383	6.67619749710451\\
21.9846323016176	6.67579476662395\\
21.9839411075984	6.67539201746949\\
21.9832498585719	6.67498924963941\\
21.9825585545293	6.67458646313197\\
21.981867195462	6.67418365794545\\
21.9811757813611	6.6737808340781\\
21.9804843122179	6.67337799152821\\
21.9797927880235	6.67297513029402\\
21.9791012087694	6.67257225037382\\
21.9784095744466	6.67216935176586\\
21.9777178850464	6.67176643446842\\
21.9770261405601	6.67136349847976\\
21.9763343409788	6.67096054379814\\
21.9756424862938	6.67055757042183\\
21.9749505764964	6.6701545783491\\
21.9742586115776	6.66975156757821\\
21.9735665915288	6.66934853810742\\
21.9728745163411	6.66894548993501\\
21.9721823860059	6.66854242305923\\
21.9714902005141	6.66813933747835\\
21.9707979598572	6.66773623319063\\
21.9701056640263	6.66733311019433\\
21.9694133130126	6.66692996848773\\
21.9687209068073	6.66652680806908\\
21.9680284454015	6.66612362893664\\
21.9673359287866	6.66572043108869\\
21.9666433569537	6.66531721452347\\
21.965950729894	6.66491397923926\\
21.9652580475986	6.6645107252343\\
21.9645653100588	6.66410745250688\\
21.9638725172658	6.66370416105524\\
21.9631796692107	6.66330085087766\\
21.9624867658847	6.66289752197238\\
21.961793807279	6.66249417433767\\
21.9611007933849	6.66209080797179\\
21.9604077241934	6.661687422873\\
21.9597145996957	6.66128401903957\\
21.959021419883	6.66088059646974\\
21.9583281847465	6.66047715516179\\
21.9576348942774	6.66007369511396\\
21.9569415484667	6.65967021632452\\
21.9562481473058	6.65926671879173\\
21.9555546907857	6.65886320251385\\
21.9548611788976	6.65845966748912\\
21.9541676116326	6.65805611371582\\
21.953473988982	6.6576525411922\\
21.9527803109368	6.65724894991652\\
21.9520865774882	6.65684533988703\\
21.9513927886274	6.65644171110199\\
21.9506989443456	6.65603806355966\\
21.9500050446337	6.6556343972583\\
21.9493110894831	6.65523071219615\\
21.9486170788848	6.65482700837149\\
21.9479230128301	6.65442328578255\\
21.9472288913099	6.6540195444276\\
21.9465347143155	6.6536157843049\\
21.9458404818379	6.6532120054127\\
21.9451461938684	6.65280820774925\\
21.944451850398	6.65240439131281\\
21.9437574514179	6.65200055610163\\
21.9430629969192	6.65159670211397\\
21.942368486893	6.65119282934808\\
21.9416739213304	6.65078893780221\\
21.9409793002226	6.65038502747462\\
21.9402846235607	6.64998109836357\\
21.9395898913357	6.64957715046729\\
21.9388951035388	6.64917318378405\\
21.9382002601611	6.64876919831211\\
21.9375053611938	6.6483651940497\\
21.9368104066278	6.64796117099509\\
21.9361153964543	6.64755712914651\\
21.9354203306645	6.64715306850224\\
21.9347252092494	6.64674898906051\\
21.9340300322001	6.64634489081958\\
21.9333347995076	6.64594077377771\\
21.9326395111632	6.64553663793313\\
21.9319441671579	6.6451324832841\\
21.9312487674827	6.64472830982886\\
21.9305533121288	6.64432411756568\\
21.9298578010872	6.6439199064928\\
21.9291622343491	6.64351567660847\\
21.9284666119054	6.64311142791093\\
21.9277709337473	6.64270716039844\\
21.9270751998659	6.64230287406924\\
21.9263794102522	6.64189856892159\\
21.9256835648973	6.64149424495372\\
21.9249876637922	6.6410899021639\\
21.9242917069281	6.64068554055036\\
21.923595694296	6.64028116011135\\
21.9228996258869	6.63987676084513\\
21.9222035016919	6.63947234274993\\
21.9215073217021	6.639067905824\\
21.9208110859086	6.6386634500656\\
21.9201147943023	6.63825897547296\\
21.9194184468743	6.63785448204433\\
21.9187220436157	6.63744996977796\\
21.9180255845175	6.63704543867209\\
21.9173290695709	6.63664088872497\\
21.9166324987667	6.63623631993485\\
21.915935872096	6.63583173229996\\
21.91523918955	6.63542712581855\\
21.9145424511196	6.63502250048887\\
21.9138456567958	6.63461785630915\\
21.9131488065698	6.63421319327765\\
21.9124519004324	6.63380851139261\\
21.9117549383749	6.63340381065226\\
21.911057920388	6.63299909105486\\
21.910360846463	6.63259435259865\\
21.9096637165908	6.63218959528186\\
21.9089665307624	6.63178481910274\\
21.9082692889689	6.63138002405953\\
21.9075719912013	6.63097521015047\\
21.9068746374505	6.63057037737381\\
21.9061772277076	6.63016552572779\\
21.9054797619636	6.62976065521064\\
21.9047822402096	6.62935576582061\\
21.9040846624364	6.62895085755594\\
21.9033870286351	6.62854593041486\\
21.9026893387967	6.62814098439562\\
21.9019915929123	6.62773601949646\\
21.9012937909727	6.62733103571562\\
21.900595932969	6.62692603305133\\
21.8998980188922	6.62652101150184\\
21.8992000487333	6.62611597106538\\
21.8985020224832	6.62571091174019\\
21.897803940133	6.62530583352452\\
21.8971058016736	6.62490073641659\\
21.896407607096	6.62449562041465\\
21.8957093563912	6.62409048551693\\
21.8950110495501	6.62368533172167\\
21.8943126865637	6.62328015902711\\
21.8936142674231	6.62287496743148\\
21.8929157921191	6.62246975693303\\
21.8922172606427	6.62206452752998\\
21.891518672985	6.62165927922057\\
21.8908200291368	6.62125401200305\\
21.8901213290891	6.62084872587564\\
21.8894225728329	6.62044342083657\\
21.8887237603591	6.62003809688409\\
21.8880248916587	6.61963275401643\\
21.8873259667227	6.61922739223183\\
21.8866269855419	6.61882201152851\\
21.8859279481074	6.61841661190471\\
21.88522885441	6.61801119335867\\
21.8845297044408	6.61760575588862\\
21.8838304981906	6.61720029949279\\
21.8831312356505	6.61679482416942\\
21.8824319168112	6.61638932991674\\
21.8817325416639	6.61598381673297\\
21.8810331101993	6.61557828461636\\
21.8803336224084	6.61517273356513\\
21.8796340782823	6.61476716357752\\
21.8789344778117	6.61436157465176\\
21.8782348209876	6.61395596678608\\
21.8775351078009	6.61355033997871\\
21.8768353382426	6.61314469422788\\
21.8761355123035	6.61273902953182\\
21.8754356299746	6.61233334588876\\
21.8747356912468	6.61192764329694\\
21.874035696111	6.61152192175458\\
21.8733356445581	6.61111618125991\\
21.872635536579	6.61071042181116\\
21.8719353721646	6.61030464340656\\
21.8712351513058	6.60989884604434\\
21.8705348739935	6.60949302972273\\
21.8698345402186	6.60908719443995\\
21.8691341499721	6.60868134019424\\
21.8684337032447	6.60827546698382\\
21.8677332000274	6.60786957480691\\
21.8670326403111	6.60746366366175\\
21.8663320240866	6.60705773354657\\
21.8656313513449	6.60665178445959\\
21.8649306220768	6.60624581639903\\
21.8642298362732	6.60583982936313\\
21.863528993925	6.6054338233501\\
21.862828095023	6.60502779835818\\
21.8621271395581	6.60462175438559\\
21.8614261275213	6.60421569143056\\
21.8607250589033	6.60380960949131\\
21.860023933695	6.60340350856606\\
21.8593227518873	6.60299738865304\\
21.858621513471	6.60259124975048\\
21.8579202184371	6.6021850918566\\
21.8572188667763	6.60177891496962\\
21.8565174584796	6.60137271908776\\
21.8558159935377	6.60096650420925\\
21.8551144719416	6.60056027033232\\
21.854412893682	6.60015401745518\\
21.8537112587498	6.59974774557606\\
21.8530095671359	6.59934145469318\\
21.8523078188311	6.59893514480476\\
21.8516060138262	6.59852881590903\\
21.8509041521121	6.5981224680042\\
21.8502022336796	6.5977161010885\\
21.8495002585196	6.59730971516015\\
21.8487982266228	6.59690331021737\\
21.8480961379802	6.59649688625837\\
21.8473939925824	6.59609044328139\\
21.8466917904204	6.59568398128463\\
21.845989531485	6.59527750026633\\
21.845287215767	6.59487100022469\\
21.8445848432571	6.59446448115794\\
21.8438824139463	6.59405794306431\\
21.8431799278254	6.59365138594199\\
21.842477384885	6.59324480978923\\
21.8417747851161	6.59283821460422\\
21.8410721285095	6.5924316003852\\
21.8403694150559	6.59202496713037\\
21.8396666447462	6.59161831483797\\
21.8389638175712	6.5912116435062\\
21.8382609335216	6.59080495313328\\
21.8375579925883	6.59039824371742\\
21.836854994762	6.58999151525686\\
21.8361519400335	6.58958476774979\\
21.8354488283937	6.58917800119444\\
21.8347456598332	6.58877121558902\\
21.834042434343	6.58836441093176\\
21.8333391519137	6.58795758722085\\
21.8326358125362	6.58755074445453\\
21.8319324162012	6.587143882631\\
21.8312289628994	6.58673700174847\\
21.8305254526218	6.58633010180517\\
21.829821885359	6.58592318279931\\
21.8291182611018	6.5855162447291\\
21.828414579841	6.58510928759275\\
21.8277108415674	6.58470231138848\\
21.8270070462716	6.5842953161145\\
21.8263031939445	6.58388830176902\\
21.8255992845768	6.58348126835026\\
21.8248953181592	6.58307421585643\\
21.8241912946826	6.58266714428573\\
21.8234872141376	6.58226005363639\\
21.8227830765151	6.58185294390662\\
21.8220788818057	6.58144581509461\\
21.8213746300002	6.5810386671986\\
21.8206703210894	6.58063150021678\\
21.819965955064	6.58022431414737\\
21.8192615319146	6.57981710898857\\
21.8185570516322	6.57940988473861\\
21.8178525142073	6.57900264139568\\
21.8171479196307	6.578595378958\\
21.8164432678931	6.57818809742377\\
21.8157385589854	6.57778079679121\\
21.8150337928981	6.57737347705853\\
21.814328969622	6.57696613822392\\
21.8136240891478	6.57655878028561\\
21.8129191514663	6.5761514032418\\
21.8122141565681	6.57574400709069\\
21.811509104444	6.5753365918305\\
21.8108039950847	6.57492915745943\\
21.8100988284809	6.57452170397569\\
21.8093936046232	6.57411423137748\\
21.8086883235024	6.57370673966302\\
21.8079829851093	6.5732992288305\\
21.8072775894344	6.57289169887814\\
21.8065721364684	6.57248414980414\\
21.8058666262022	6.5720765816067\\
21.8051610586263	6.57166899428404\\
21.8044554337315	6.57126138783435\\
21.8037497515084	6.57085376225584\\
21.8030440119478	6.57044611754672\\
21.8023382150402	6.57003845370518\\
21.8016323607764	6.56963077072944\\
21.8009264491471	6.5692230686177\\
21.8002204801429	6.56881534736815\\
21.7995144537546	6.56840760697901\\
21.7988083699727	6.56799984744848\\
21.798102228788	6.56759206877475\\
21.797396030191	6.56718427095604\\
21.7966897741726	6.56677645399054\\
21.7959834607233	6.56636861787646\\
21.7952770898338	6.565960762612\\
21.7945706614947	6.56555288819535\\
21.7938641756968	6.56514499462473\\
21.7931576324306	6.56473708189832\\
21.7924510316868	6.56432915001434\\
21.7917443734561	6.56392119897099\\
21.7910376577291	6.56351322876645\\
21.7903308844964	6.56310523939894\\
21.7896240537487	6.56269723086665\\
21.7889171654767	6.56228920316778\\
21.7882102196709	6.56188115630053\\
21.787503216322	6.5614730902631\\
21.7867961554206	6.56106500505369\\
21.7860890369574	6.5606569006705\\
21.785381860923	6.56024877711173\\
21.784674627308	6.55984063437556\\
21.783967336103	6.55943247246021\\
21.7832599872986	6.55902429136387\\
21.7825525808856	6.55861609108473\\
21.7818451168544	6.55820787162099\\
21.7811375951957	6.55779963297085\\
21.7804300159001	6.55739137513251\\
21.7797223789583	6.55698309810416\\
21.7790146843607	6.55657480188399\\
21.7783069320981	6.55616648647021\\
21.777599122161	6.55575815186101\\
21.7768912545401	6.55534979805458\\
21.7761833292259	6.55494142504912\\
21.775475346209	6.55453303284282\\
21.77476730548	6.55412462143388\\
21.7740592070295	6.55371619082049\\
21.7733510508481	6.55330774100084\\
21.7726428369264	6.55289927197313\\
21.7719345652549	6.55249078373556\\
21.7712262358243	6.55208227628631\\
21.7705178486251	6.55167374962358\\
21.7698094036479	6.55126520374556\\
21.7691009008833	6.55085663865044\\
21.7683923403218	6.55044805433642\\
21.7676837219541	6.55003945080168\\
21.7669750457706	6.54963082804442\\
21.7662663117619	6.54922218606283\\
21.7655575199187	6.5488135248551\\
21.7648486702314	6.54840484441943\\
21.7641397626907	6.54799614475399\\
21.763430797287	6.54758742585699\\
21.762721774011	6.54717868772661\\
21.7620126928532	6.54676993036104\\
21.7613035538041	6.54636115375847\\
21.7605943568543	6.54595235791709\\
21.7598851019943	6.54554354283509\\
21.7591757892147	6.54513470851066\\
21.7584664185059	6.54472585494199\\
21.7577569898587	6.54431698212726\\
21.7570475032634	6.54390809006467\\
21.7563379587106	6.54349917875239\\
21.7556283561908	6.54309024818863\\
21.7549186956947	6.54268129837156\\
21.7542089772126	6.54227232929937\\
21.7534992007351	6.54186334097025\\
21.7527893662528	6.54145433338239\\
21.7520794737562	6.54104530653396\\
21.7513695232357	6.54063626042317\\
21.7506595146819	6.54022719504819\\
21.7499494480853	6.53981811040721\\
21.7492393234364	6.5394090064984\\
21.7485291407257	6.53899988331997\\
21.7478188999438	6.5385907408701\\
21.747108601081	6.53818157914696\\
21.746398244128	6.53777239814874\\
21.7456878290752	6.53736319787363\\
21.7449773559132	6.5369539783198\\
21.7442668246323	6.53654473948546\\
21.7435562352231	6.53613548136876\\
21.7428455876762	6.5357262039679\\
21.7421348819819	6.53531690728107\\
21.7414241181307	6.53490759130644\\
21.7407132961132	6.53449825604219\\
21.7400024159198	6.53408890148651\\
21.739291477541	6.53367952763758\\
21.7385804809673	6.53327013449358\\
21.7378694261891	6.53286072205269\\
21.7371583131969	6.5324512903131\\
21.7364471419812	6.53204183927297\\
21.7357359125325	6.5316323689305\\
21.7350246248411	6.53122287928386\\
21.7343132788977	6.53081337033124\\
21.7336018746925	6.5304038420708\\
21.7328904122161	6.52999429450074\\
21.732178891459	6.52958472761923\\
21.7314673124115	6.52917514142444\\
21.7307556750642	6.52876553591457\\
21.7300439794074	6.52835591108778\\
21.7293322254316	6.52794626694225\\
21.7286204131274	6.52753660347616\\
21.727908542485	6.5271269206877\\
21.727196613495	6.52671721857503\\
21.7264846261477	6.52630749713633\\
21.7257725804336	6.52589775636978\\
21.7250604763432	6.52548799627356\\
21.7243483138669	6.52507821684584\\
21.723636092995	6.5246684180848\\
21.722923813718	6.52425859998861\\
21.7222114760264	6.52384876255545\\
21.7214990799105	6.52343890578349\\
21.7207866253608	6.52302902967091\\
21.7200741123676	6.52261913421589\\
21.7193615409215	6.52220921941659\\
21.7186489110127	6.5217992852712\\
21.7179362226318	6.52138933177788\\
21.717223475769	6.52097935893481\\
21.7165106704149	6.52056936674017\\
21.7157978065597	6.52015935519212\\
21.715084884194	6.51974932428883\\
21.714371903308	6.51933927402849\\
21.7136588638923	6.51892920440926\\
21.7129457659371	6.51851911542932\\
21.7122326094329	6.51810900708684\\
21.7115193943701	6.51769887937998\\
21.710806120739	6.51728873230692\\
21.71009278853	6.51687856586584\\
21.7093793977334	6.51646838005489\\
21.7086659483398	6.51605817487225\\
21.7079524403394	6.5156479503161\\
21.7072388737226	6.5152377063846\\
21.7065252484798	6.51482744307593\\
21.7058115646013	6.51441716038824\\
21.7050978220775	6.51400685831971\\
21.7043840208988	6.51359653686852\\
21.7036701610556	6.51318619603281\\
21.7029562425381	6.51277583581078\\
21.7022422653367	6.51236545620058\\
21.7015282294419	6.51195505720038\\
21.7008141348439	6.51154463880836\\
21.700099981533	6.51113420102266\\
21.6993857694997	6.51072374384148\\
21.6986714987343	6.51031326726296\\
21.6979571692271	6.50990277128528\\
21.6972427809684	6.50949225590661\\
21.6965283339487	6.5090817211251\\
21.6958138281581	6.50867116693894\\
21.6950992635871	6.50826059334627\\
21.6943846402259	6.50785000034527\\
21.693669958065	6.5074393879341\\
21.6929552170946	6.50702875611092\\
21.692240417305	6.50661810487391\\
21.6915255586866	6.50620743422122\\
21.6908106412297	6.50579674415102\\
21.6900956649246	6.50538603466148\\
21.6893806297615	6.50497530575074\\
21.6886655357309	6.50456455741699\\
21.687950382823	6.50415378965838\\
21.6872351710281	6.50374300247308\\
21.6865199003365	6.50333219585924\\
21.6858045707386	6.50292136981503\\
21.6850891822246	6.50251052433861\\
21.6843737347848	6.50209965942815\\
21.6836582284095	6.5016887750818\\
21.682942663089	6.50127787129772\\
21.6822270388135	6.50086694807408\\
21.6815113555735	6.50045600540904\\
21.6807956133591	6.50004504330075\\
21.6800798121606	6.49963406174738\\
21.6793639519683	6.49922306074709\\
21.6786480327725	6.49881204029803\\
21.6779320545635	6.49840100039837\\
21.6772160173315	6.49798994104627\\
21.6764999210667	6.49757886223987\\
21.6757837657596	6.49716776397735\\
21.6750675514002	6.49675664625686\\
21.6743512779789	6.49634550907655\\
21.673634945486	6.4959343524346\\
21.6729185539117	6.49552317632914\\
21.6722021032462	6.49511198075835\\
21.6714855934798	6.49470076572037\\
21.6707690246028	6.49428953121337\\
21.6700523966053	6.49387827723549\\
21.6693357094777	6.49346700378491\\
21.6686189632102	6.49305571085976\\
21.667902157793	6.49264439845821\\
21.6671852932164	6.49223306657842\\
21.6664683694705	6.49182171521853\\
21.6657513865457	6.49141034437671\\
21.6650343444321	6.4909989540511\\
21.6643172431199	6.49058754423986\\
21.6636000825995	6.49017611494116\\
21.662882862861	6.48976466615313\\
21.6621655838947	6.48935319787393\\
21.6614482456907	6.48894171010172\\
21.6607308482393	6.48853020283465\\
21.6600133915307	6.48811867607087\\
21.659295875555	6.48770712980854\\
21.6585783003026	6.4872955640458\\
21.6578606657636	6.48688397878081\\
21.6571429719282	6.48647237401172\\
21.6564252187866	6.48606074973669\\
21.655707406329	6.48564910595385\\
21.6549895345456	6.48523744266137\\
21.6542716034266	6.4848257598574\\
21.6535536129622	6.48441405754007\\
21.6528355631426	6.48400233570756\\
21.652117453958	6.483590594358\\
21.6513992853985	6.48317883348954\\
21.6506810574543	6.48276705310033\\
21.6499627701156	6.48235525318853\\
21.6492444233726	6.48194343375228\\
21.6485260172154	6.48153159478973\\
21.6478075516343	6.48111973629902\\
21.6470890266194	6.48070785827832\\
21.6463704421608	6.48029596072575\\
21.6456517982487	6.47988404363948\\
21.6449330948734	6.47947210701764\\
21.6442143320249	6.47906015085839\\
21.6434955096933	6.47864817515988\\
21.642776627869	6.47823617992024\\
21.6420576865419	6.47782416513762\\
21.6413386857023	6.47741213081018\\
21.6406196253402	6.47700007693605\\
21.639900505446	6.47658800351338\\
21.6391813260096	6.47617591054032\\
21.6384620870212	6.47576379801501\\
21.637742788471	6.4753516659356\\
21.6370234303491	6.47493951430022\\
21.6363040126456	6.47452734310703\\
21.6355845353507	6.47411515235417\\
21.6348649984544	6.47370294203978\\
21.634145401947	6.473290712162\\
21.6334257458185	6.47287846271898\\
21.632706030059	6.47246619370886\\
21.6319862546587	6.47205390512979\\
21.6312664196077	6.4716415969799\\
21.6305465248961	6.47122926925733\\
21.629826570514	6.47081692196023\\
21.6291065564515	6.47040455508675\\
21.6283864826987	6.46999216863501\\
21.6276663492457	6.46957976260317\\
21.6269461560826	6.46916733698935\\
21.6262259031996	6.46875489179171\\
21.6255055905867	6.46834242700839\\
21.6247852182339	6.46792994263751\\
21.6240647861315	6.46751743867723\\
21.6233442942695	6.46710491512568\\
21.6226237426379	6.466692371981\\
21.6219031312268	6.46627980924132\\
21.6211824600264	6.4658672269048\\
21.6204617290267	6.46545462496956\\
21.6197409382177	6.46504200343374\\
21.6190200875897	6.46462936229549\\
21.6182991771325	6.46421670155293\\
21.6175782068363	6.46380402120421\\
21.6168571766912	6.46339132124746\\
21.6161360866872	6.46297860168082\\
21.6154149368144	6.46256586250243\\
21.6146937270629	6.46215310371042\\
21.6139724574226	6.46174032530292\\
21.6132511278837	6.46132752727808\\
21.6125297384361	6.46091470963403\\
21.61180828907	6.4605018723689\\
21.6110867797754	6.46008901548083\\
21.6103652105424	6.45967613896795\\
21.6096435813609	6.45926324282839\\
21.608921892221	6.4588503270603\\
21.6082001431127	6.4584373916618\\
21.6074783340262	6.45802443663103\\
21.6067564649513	6.45761146196612\\
21.6060345358782	6.45719846766521\\
21.6053125467969	6.45678545372642\\
21.6045904976973	6.45637242014789\\
21.6038683885696	6.45595936692775\\
21.6031462194037	6.45554629406413\\
21.6024239901896	6.45513320155517\\
21.6017017009174	6.45472008939899\\
21.600979351577	6.45430695759373\\
21.6002569421586	6.45389380613752\\
21.599534472652	6.45348063502849\\
21.5988119430472	6.45306744426476\\
21.5980893533344	6.45265423384447\\
21.5973667035035	6.45224100376574\\
21.5966439935444	6.45182775402672\\
21.5959212234472	6.45141448462552\\
21.5951983932019	6.45100119556028\\
21.5944755027984	6.45058788682912\\
21.5937525522267	6.45017455843017\\
21.5930295414769	6.44976121036156\\
21.5923064705389	6.44934784262142\\
21.5915833394026	6.44893445520787\\
21.5908601480581	6.44852104811905\\
21.5901368964954	6.44810762135308\\
21.5894135847043	6.44769417490808\\
21.5886902126749	6.44728070878219\\
21.5879667803971	6.44686722297352\\
21.5872432878609	6.44645371748021\\
21.5865197350563	6.44604019230039\\
21.5857961219732	6.44562664743216\\
21.5850724486016	6.44521308287367\\
21.5843487149313	6.44479949862304\\
21.5836249209525	6.44438589467838\\
21.582901066655	6.44397227103784\\
21.5821771520287	6.44355862769951\\
21.5814531770637	6.44314496466155\\
21.5807291417498	6.44273128192206\\
21.580005046077	6.44231757947917\\
21.5792808900352	6.441903857331\\
21.5785566736144	6.44149011547568\\
21.5778323968045	6.44107635391132\\
21.5771080595953	6.44066257263606\\
21.576383661977	6.44024877164801\\
21.5756592039393	6.43983495094529\\
21.5749346854721	6.43942111052603\\
21.5742101065655	6.43900725038834\\
21.5734854672093	6.43859337053035\\
21.5727607673935	6.43817947095018\\
21.5720360071078	6.43776555164594\\
21.5713111863423	6.43735161261576\\
21.5705863050869	6.43693765385777\\
21.5698613633315	6.43652367537006\\
21.5691363610659	6.43610967715078\\
21.56841129828	6.43569565919803\\
21.5676861749638	6.43528162150993\\
21.5669609911071	6.43486756408461\\
21.5662357466999	6.43445348692018\\
21.565510441732	6.43403939001476\\
21.5647850761933	6.43362527336646\\
21.5640596500737	6.43321113697341\\
21.5633341633631	6.43279698083371\\
21.5626086160513	6.4323828049455\\
21.5618830081282	6.43196860930688\\
21.5611573395837	6.43155439391596\\
21.5604316104077	6.43114015877088\\
21.55970582059	6.43072590386974\\
21.5589799701206	6.43031162921065\\
21.5582540589891	6.42989733479174\\
21.5575280871856	6.42948302061111\\
21.5568020546999	6.42906868666689\\
21.5560759615218	6.42865433295719\\
21.5553498076411	6.42823995948011\\
21.5546235930478	6.42782556623378\\
21.5538973177317	6.42741115321631\\
21.5531709816826	6.42699672042581\\
21.5524445848904	6.42658226786039\\
21.5517181273449	6.42616779551817\\
21.5509916090359	6.42575330339726\\
21.5502650299533	6.42533879149577\\
21.5495383900868	6.42492425981182\\
21.5488116894265	6.42450970834351\\
21.548084927962	6.42409513708896\\
21.5473581056831	6.42368054604628\\
21.5466312225798	6.42326593521358\\
21.5459042786418	6.42285130458896\\
21.545177273859	6.42243665417055\\
21.5444502082211	6.42202198395645\\
21.5437230817179	6.42160729394476\\
21.5429958943394	6.42119258413361\\
21.5422686460752	6.42077785452109\\
21.5415413369152	6.42036310510533\\
21.5408139668492	6.41994833588442\\
21.540086535867	6.41953354685647\\
21.5393590439584	6.4191187380196\\
21.5386314911131	6.4187039093719\\
21.537903877321	6.4182890609115\\
21.5371762025719	6.41787419263649\\
21.5364484668555	6.41745930454498\\
21.5357206701616	6.41704439663509\\
21.53499281248	6.41662946890491\\
21.5342648938005	6.41621452135255\\
21.5335369141129	6.41579955397612\\
21.5328088734068	6.41538456677373\\
21.5320807716722	6.41496955974347\\
21.5313526088988	6.41455453288347\\
21.5306243850762	6.4141394861918\\
21.5298961001944	6.4137244196666\\
21.529167754243	6.41330933330595\\
21.5284393472119	6.41289422710797\\
21.5277108790907	6.41247910107075\\
21.5269823498692	6.4120639551924\\
21.5262537595372	6.41164878947103\\
21.5255251080844	6.41123360390473\\
21.5247963955005	6.41081839849161\\
21.5240676217754	6.41040317322978\\
21.5233387868987	6.40998792811733\\
21.5226098908601	6.40957266315236\\
21.5218809336495	6.40915737833298\\
21.5211519152565	6.40874207365729\\
21.5204228356709	6.40832674912339\\
21.5196936948824	6.40791140472938\\
21.5189644928807	6.40749604047336\\
21.5182352296556	6.40708065635344\\
21.5175059051967	6.4066652523677\\
21.5167765194937	6.40624982851426\\
21.5160470725365	6.4058343847912\\
21.5153175643147	6.40541892119664\\
21.514587994818	6.40500343772867\\
21.5138583640361	6.40458793438538\\
21.5131286719587	6.40417241116488\\
21.5123989185755	6.40375686806527\\
21.5116691038763	6.40334130508463\\
21.5109392278506	6.40292572222108\\
21.5102092904883	6.4025101194727\\
21.5094792917789	6.4020944968376\\
21.5087492317123	6.40167885431386\\
21.508019110278	6.4012631918996\\
21.5072889274657	6.40084750959289\\
21.5065586832652	6.40043180739185\\
21.505828377666	6.40001608529456\\
21.505098010658	6.39960034329912\\
21.5043675822307	6.39918458140363\\
21.5036370923738	6.39876879960617\\
21.502906541077	6.39835299790485\\
21.50217592833	6.39793717629776\\
21.5014452541224	6.39752133478299\\
21.5007145184438	6.39710547335864\\
21.499983721284	6.3966895920228\\
21.4992528626325	6.39627369077356\\
21.4985219424791	6.39585776960902\\
21.4977909608134	6.39544182852726\\
21.4970599176249	6.39502586752639\\
21.4963288129035	6.39460988660449\\
21.4955976466387	6.39419388575966\\
21.4948664188201	6.39377786498998\\
21.4941351294373	6.39336182429355\\
21.4934037784801	6.39294576366845\\
21.4926723659381	6.39252968311279\\
21.4919408918008	6.39211358262465\\
21.4912093560579	6.39169746220212\\
21.490477758699	6.39128132184329\\
21.4897460997138	6.39086516154624\\
21.4890143790918	6.39044898130908\\
21.4882825968227	6.39003278112988\\
21.4875507528961	6.38961656100674\\
21.4868188473015	6.38920032093775\\
21.4860868800287	6.38878406092099\\
21.4853548510672	6.38836778095455\\
21.4846227604065	6.38795148103652\\
21.4838906080364	6.38753516116499\\
21.4831583939463	6.38711882133804\\
21.482426118126	6.38670246155376\\
21.4816937805649	6.38628608181024\\
21.4809613812527	6.38586968210556\\
21.4802289201789	6.38545326243781\\
21.4794963973332	6.38503682280508\\
21.4787638127051	6.38462036320545\\
21.4780311662841	6.384203883637\\
21.47729845806	6.38378738409783\\
21.4765656880222	6.38337086458601\\
21.4758328561603	6.38295432509963\\
21.4750999624638	6.38253776563677\\
21.4743670069225	6.38212118619552\\
21.4736339895257	6.38170458677397\\
21.472900910263	6.38128796737019\\
21.4721677691241	6.38087132798226\\
21.4714345660985	6.38045466860828\\
21.4707013011757	6.38003798924632\\
21.4699679743452	6.37962128989447\\
21.4692345855967	6.3792045705508\\
21.4685011349196	6.3787878312134\\
21.4677676223036	6.37837107188035\\
21.467034047738	6.37795429254974\\
21.4663004112126	6.37753749321963\\
21.4655667127167	6.37712067388812\\
21.46483295224	6.37670383455328\\
21.4640991297719	6.37628697521319\\
21.463365245302	6.37587009586594\\
21.4626312988199	6.37545319650959\\
21.4618972903149	6.37503627714224\\
21.4611632197767	6.37461933776196\\
21.4604290871948	6.37420237836682\\
21.4596948925586	6.37378539895492\\
21.4589606358577	6.37336839952432\\
21.4582263170816	6.37295138007309\\
21.4574919362198	6.37253434059933\\
21.4567574932617	6.37211728110111\\
21.456022988197	6.3717002015765\\
21.455288421015	6.37128310202359\\
21.4545537917053	6.37086598244044\\
21.4538191002574	6.37044884282513\\
21.4530843466607	6.37003168317574\\
21.4523495309047	6.36961450349035\\
21.451614652979	6.36919730376702\\
21.450879712873	6.36878008400384\\
21.4501447105761	6.36836284419888\\
21.4494096460779	6.36794558435022\\
21.4486745193678	6.36752830445592\\
21.4479393304352	6.36711100451407\\
21.4472040792698	6.36669368452273\\
21.4464687658608	6.36627634447998\\
21.4457333901978	6.36585898438389\\
21.4449979522703	6.36544160423254\\
21.4442624520676	6.36502420402399\\
21.4435268895793	6.36460678375632\\
21.4427912647948	6.36418934342761\\
21.4420555777035	6.36377188303592\\
21.4413198282949	6.36335440257932\\
21.4405840165584	6.36293690205589\\
21.4398481424835	6.36251938146369\\
21.4391122060596	6.3621018408008\\
21.4383762072761	6.36168428006529\\
21.4376401461225	6.36126669925523\\
21.4369040225882	6.36084909836868\\
21.4361678366626	6.36043147740372\\
21.4354315883351	6.36001383635842\\
21.4346952775952	6.35959617523083\\
21.4339589044323	6.35917849401905\\
21.4332224688358	6.35876079272112\\
21.432485970795	6.35834307133512\\
21.4317494102995	6.35792532985912\\
21.4310127873386	6.35750756829119\\
21.4302761019018	6.35708978662938\\
21.4295393539783	6.35667198487178\\
21.4288025435578	6.35625416301644\\
21.4280656706294	6.35583632106143\\
21.4273287351826	6.35541845900482\\
21.4265917372068	6.35500057684467\\
21.4258546766915	6.35458267457905\\
21.4251175536259	6.35416475220603\\
21.4243803679995	6.35374680972366\\
21.4236431198016	6.35332884713002\\
21.4229058090217	6.35291086442317\\
21.422168435649	6.35249286160117\\
21.421430999673	6.35207483866209\\
21.420693501083	6.35165679560398\\
21.4199559398685	6.35123873242493\\
21.4192183160187	6.35082064912297\\
21.418480629523	6.35040254569619\\
21.4177428803708	6.34998442214263\\
21.4170050685515	6.34956627846037\\
21.4162671940543	6.34914811464747\\
21.4155292568687	6.34872993070198\\
21.414791256984	6.34831172662197\\
21.4140531943895	6.3478935024055\\
21.4133150690746	6.34747525805063\\
21.4125768810285	6.34705699355542\\
21.4118386302408	6.34663870891794\\
21.4111003167006	6.34622040413623\\
21.4103619403974	6.34580207920837\\
21.4096235013204	6.3453837341324\\
21.4088849994589	6.34496536890639\\
21.4081464348024	6.3445469835284\\
21.4074078073401	6.34412857799649\\
21.4066691170613	6.34371015230871\\
21.4059303639553	6.34329170646313\\
21.4051915480115	6.34287324045779\\
21.4044526692192	6.34245475429076\\
21.4037137275676	6.3420362479601\\
21.4029747230461	6.34161772146386\\
21.402235655644	6.3411991748001\\
21.4014965253506	6.34078060796687\\
21.4007573321551	6.34036202096223\\
21.4000180760469	6.33994341378424\\
21.3992787570153	6.33952478643095\\
21.3985393750494	6.33910613890041\\
21.3977999301388	6.33868747119069\\
21.3970604222725	6.33826878329984\\
21.3963208514399	6.3378500752259\\
21.3955812176303	6.33743134696694\\
21.3948415208329	6.33701259852101\\
21.394101761037	6.33659382988616\\
21.3933619382319	6.33617504106044\\
21.3926220524069	6.33575623204191\\
21.3918821035511	6.33533740282863\\
21.3911420916539	6.33491855341863\\
21.3904020167045	6.33449968380999\\
21.3896618786922	6.33408079400073\\
21.3889216776063	6.33366188398893\\
21.3881814134359	6.33324295377262\\
21.3874410861703	6.33282400334987\\
21.3867006957988	6.33240503271872\\
21.3859602423106	6.33198604187722\\
21.3852197256949	6.33156703082341\\
21.3844791459411	6.33114799955537\\
21.3837385030382	6.33072894807112\\
21.3829977969756	6.33030987636872\\
21.3822570277424	6.32989078444622\\
21.381516195328	6.32947167230167\\
21.3807752997214	6.32905253993312\\
21.380034340912	6.3286333873386\\
21.3792933188889	6.32821421451619\\
21.3785522336414	6.32779502146391\\
21.3778110851587	6.32737580817981\\
21.37706987343	6.32695657466196\\
21.3763285984444	6.32653732090838\\
21.3755872601912	6.32611804691713\\
21.3748458586596	6.32569875268625\\
21.3741043938388	6.32527943821379\\
21.373362865718	6.3248601034978\\
21.3726212742863	6.32444074853631\\
21.371879619533	6.32402137332738\\
21.3711379014473	6.32360197786905\\
21.3703961200182	6.32318256215936\\
21.3696542752351	6.32276312619637\\
21.368912367087	6.3223436699781\\
21.3681703955632	6.32192419350261\\
21.3674283606529	6.32150469676794\\
21.3666862623451	6.32108517977213\\
21.3659441006291	6.32066564251322\\
21.365201875494	6.32024608498926\\
21.364459586929	6.31982650719829\\
21.3637172349232	6.31940690913835\\
21.3629748194659	6.31898729080749\\
21.3622323405461	6.31856765220374\\
21.361489798153	6.31814799332514\\
21.3607471922757	6.31772831416974\\
21.3600045229034	6.31730861473557\\
21.3592617900253	6.31688889502069\\
21.3585189936304	6.31646915502311\\
21.357776133708	6.3160493947409\\
21.3570332102471	6.31562961417208\\
21.3562902232368	6.31520981331469\\
21.3555471726663	6.31478999216678\\
21.3548040585248	6.31437015072638\\
21.3540608808013	6.31395028899152\\
21.353317639485	6.31353040696026\\
21.3525743345649	6.31311050463062\\
21.3518309660302	6.31269058200064\\
21.3510875338701	6.31227063906836\\
21.3503440380735	6.31185067583182\\
21.3496004786297	6.31143069228906\\
21.3488568555276	6.3110106884381\\
21.3481131687565	6.31059066427698\\
21.3473694183054	6.31017061980375\\
21.3466256041634	6.30975055501643\\
21.3458817263195	6.30933046991307\\
21.345137784763	6.30891036449169\\
21.3443937794828	6.30849023875032\\
21.3436497104681	6.30807009268702\\
21.3429055777079	6.3076499262998\\
21.3421613811913	6.30722973958671\\
21.3414171209074	6.30680953254576\\
21.3406727968452	6.30638930517501\\
21.3399284089939	6.30596905747248\\
21.3391839573424	6.3055487894362\\
21.3384394418799	6.30512850106421\\
21.3376948625954	6.30470819235454\\
21.336950219478	6.30428786330521\\
21.3362055125167	6.30386751391427\\
21.3354607417005	6.30344714417974\\
21.3347159070186	6.30302675409965\\
21.3339710084599	6.30260634367203\\
21.3332260460135	6.30218591289492\\
21.3324810196685	6.30176546176634\\
21.3317359294139	6.30134499028433\\
21.3309907752387	6.3009244984469\\
21.3302455571319	6.3005039862521\\
21.3295002750826	6.30008345369794\\
21.3287549290798	6.29966290078247\\
21.3280095191126	6.2992423275037\\
21.3272640451699	6.29882173385967\\
21.3265185072408	6.2984011198484\\
21.3257729053142	6.29798048546791\\
21.3250272393792	6.29755983071625\\
21.3242815094249	6.29713915559143\\
21.3235357154401	6.29671846009147\\
21.322789857414	6.29629774421442\\
21.3220439353355	6.29587700795828\\
21.3212979491935	6.2954562513211\\
21.3205518989772	6.29503547430088\\
21.3198057846755	6.29461467689567\\
21.3190596062774	6.29419385910347\\
21.3183133637718	6.29377302092233\\
21.3175670571478	6.29335216235025\\
21.3168206863944	6.29293128338527\\
21.3160742515004	6.29251038402541\\
21.315327752455	6.2920894642687\\
21.314581189247	6.29166852411314\\
21.3138345618654	6.29124756355678\\
21.3130878702993	6.29082658259762\\
21.3123411145374	6.2904055812337\\
21.311594294569	6.28998455946303\\
21.3108474103828	6.28956351728364\\
21.3101004619678	6.28914245469354\\
21.309353449313	6.28872137169077\\
21.3086063724074	6.28830026827333\\
21.3078592312398	6.28787914443925\\
21.3071120257993	6.28745800018656\\
21.3063647560747	6.28703683551326\\
21.3056174220551	6.28661565041738\\
21.3048700237293	6.28619444489694\\
21.3041225610862	6.28577321894995\\
21.3033750341149	6.28535197257445\\
21.3026274428042	6.28493070576843\\
21.301879787143	6.28450941852993\\
21.3011320671204	6.28408811085697\\
21.3003842827251	6.28366678274754\\
21.2996364339461	6.28324543419969\\
21.2988885207724	6.28282406521142\\
21.2981405431927	6.28240267578074\\
21.2973925011961	6.28198126590568\\
21.2966443947715	6.28155983558425\\
21.2958962239077	6.28113838481448\\
21.2951479885936	6.28071691359436\\
21.2943996888181	6.28029542192192\\
21.2936513245702	6.27987390979518\\
21.2929028958387	6.27945237721214\\
21.2921544026125	6.27903082417083\\
21.2914058448805	6.27860925066925\\
21.2906572226316	6.27818765670542\\
21.2899085358546	6.27776604227735\\
21.2891597845384	6.27734440738307\\
21.2884109686719	6.27692275202057\\
21.2876620882439	6.27650107618787\\
21.2869131432434	6.27607937988299\\
21.2861641336592	6.27565766310394\\
21.2854150594801	6.27523592584873\\
21.2846659206951	6.27481416811536\\
21.2839167172929	6.27439238990186\\
21.2831674492624	6.27397059120623\\
21.2824181165924	6.27354877202648\\
21.2816687192719	6.27312693236063\\
21.2809192572897	6.27270507220668\\
21.2801697306345	6.27228319156264\\
21.2794201392952	6.27186129042652\\
21.2786704832608	6.27143936879634\\
21.2779207625199	6.2710174266701\\
21.2771709770614	6.2705954640458\\
21.2764211268742	6.27017348092147\\
21.2756712119471	6.2697514772951\\
21.2749212322689	6.2693294531647\\
21.2741711878283	6.26890740852828\\
21.2734210786144	6.26848534338385\\
21.2726709046157	6.26806325772942\\
21.2719206658212	6.26764115156298\\
21.2711703622196	6.26721902488256\\
21.2704199937998	6.26679687768615\\
21.2696695605506	6.26637470997176\\
21.2689190624607	6.26595252173739\\
21.2681684995189	6.26553031298105\\
21.2674178717141	6.26510808370074\\
21.266667179035	6.26468583389448\\
21.2659164214704	6.26426356356026\\
21.2651655990091	6.26384127269608\\
21.2644147116399	6.26341896129996\\
21.2636637593515	6.26299662936989\\
21.2629127421328	6.26257427690388\\
21.2621616599724	6.26215190389992\\
21.2614105128592	6.26172951035604\\
21.2606593007819	6.26130709627021\\
21.2599080237293	6.26088466164045\\
21.2591566816902	6.26046220646476\\
21.2584052746533	6.26003973074114\\
21.2576538026073	6.25961723446759\\
21.256902265541	6.25919471764211\\
21.2561506634432	6.2587721802627\\
21.2553989963026	6.25834962232737\\
21.2546472641079	6.25792704383411\\
21.2538954668479	6.25750444478092\\
21.2531436045113	6.25708182516579\\
21.2523916770868	6.25665918498674\\
21.2516396845633	6.25623652424176\\
21.2508876269293	6.25581384292884\\
21.2501355041737	6.25539114104599\\
21.2493833162851	6.2549684185912\\
21.2486310632522	6.25454567556247\\
21.2478787450639	6.2541229119578\\
21.2471263617088	6.25370012777518\\
21.2463739131755	6.25327732301262\\
21.2456213994529	6.2528544976681\\
21.2448688205296	6.25243165173962\\
21.2441161763943	6.25200878522519\\
21.2433634670357	6.25158589812278\\
21.2426106924425	6.25116299043041\\
21.2418578526034	6.25074006214606\\
21.2411049475071	6.25031711326774\\
21.2403519771422	6.24989414379342\\
21.2395989414976	6.24947115372112\\
21.2388458405617	6.24904814304881\\
21.2380926743234	6.2486251117745\\
21.2373394427713	6.24820205989618\\
21.236586145894	6.24777898741184\\
21.2358327836802	6.24735589431947\\
21.2350793561187	6.24693278061707\\
21.2343258631979	6.24650964630262\\
21.2335723049067	6.24608649137412\\
21.2328186812337	6.24566331582957\\
21.2320649921675	6.24524011966694\\
21.2313112376968	6.24481690288424\\
21.2305574178102	6.24439366547945\\
21.2298035324963	6.24397040745056\\
21.2290495817439	6.24354712879557\\
21.2282955655415	6.24312382951246\\
21.2275414838778	6.24270050959922\\
21.2267873367413	6.24227716905385\\
21.2260331241209	6.24185380787432\\
21.225278846005	6.24143042605863\\
21.2245245023823	6.24100702360476\\
21.2237700932414	6.24058360051071\\
21.223015618571	6.24016015677447\\
21.2222610783596	6.23973669239401\\
21.2215064725958	6.23931320736732\\
21.2207518012684	6.2388897016924\\
21.2199970643658	6.23846617536723\\
21.2192422618767	6.23804262838979\\
21.2184873937896	6.23761906075808\\
21.2177324600932	6.23719547247007\\
21.2169774607761	6.23677186352375\\
21.2162223958269	6.23634823391711\\
21.2154672652341	6.23592458364813\\
21.2147120689864	6.23550091271479\\
21.2139568070722	6.23507722111508\\
21.2132014794803	6.23465350884699\\
21.2124460861991	6.23422977590849\\
21.2116906272173	6.23380602229758\\
21.2109351025234	6.23338224801222\\
21.210179512106	6.23295845305041\\
21.2094238559536	6.23253463741013\\
21.2086681340548	6.23211080108935\\
21.2079123463982	6.23168694408607\\
21.2071564929723	6.23126306639826\\
21.2064005737657	6.2308391680239\\
21.2056445887669	6.23041524896097\\
21.2048885379645	6.22999130920746\\
21.204132421347	6.22956734876135\\
21.203376238903	6.22914336762061\\
21.2026199906209	6.22871936578322\\
21.2018636764894	6.22829534324717\\
21.2011072964969	6.22787130001043\\
21.200350850632	6.22744723607098\\
21.1995943388832	6.2270231514268\\
21.1988377612391	6.22659904607586\\
21.1980811176881	6.22617492001615\\
21.1973244082188	6.22575077324565\\
21.1965676328197	6.22532660576232\\
21.1958107914792	6.22490241756415\\
21.195053884186	6.22447820864912\\
21.1942969109284	6.2240539790152\\
21.1935398716951	6.22362972866036\\
21.1927827664745	6.22320545758258\\
21.1920255952551	6.22278116577984\\
21.1912683580254	6.22235685325011\\
21.1905110547738	6.22193251999137\\
21.1897536854889	6.22150816600159\\
21.1889962501592	6.22108379127875\\
21.1882387487731	6.22065939582082\\
21.1874811813192	6.22023497962576\\
21.1867235477858	6.21981054269157\\
21.1859658481614	6.21938608501621\\
21.1852080824346	6.21896160659765\\
21.1844502505938	6.21853710743386\\
21.1836923526274	6.21811258752283\\
21.182934388524	6.21768804686251\\
21.1821763582719	6.21726348545088\\
21.1814182618596	6.21683890328591\\
21.1806600992756	6.21641430036558\\
21.1799018705083	6.21598967668785\\
21.1791435755462	6.2155650322507\\
21.1783852143777	6.21514036705208\\
21.1776267869912	6.21471568108999\\
21.1768682933752	6.21429097436237\\
21.1761097335181	6.21386624686721\\
21.1753511074084	6.21344149860247\\
21.1745924150344	6.21301672956612\\
21.1738336563846	6.21259193975613\\
21.1730748314474	6.21216712917047\\
21.1723159402112	6.21174229780709\\
21.1715569826645	6.21131744566399\\
21.1707979587956	6.21089257273911\\
21.170038868593	6.21046767903042\\
21.169279712045	6.2100427645359\\
21.1685204891401	6.2096178292535\\
21.1677611998667	6.2091928731812\\
21.1670018442131	6.20876789631696\\
21.1662424221677	6.20834289865874\\
21.165482933719	6.20791788020452\\
21.1647233788553	6.20749284095224\\
21.1639637575651	6.20706778089989\\
21.1632040698366	6.20664270004542\\
21.1624443156583	6.2062175983868\\
21.1616844950185	6.20579247592199\\
21.1609246079056	6.20536733264895\\
21.1601646543081	6.20494216856565\\
21.1594046342141	6.20451698367005\\
21.1586445476122	6.20409177796011\\
21.1578843944906	6.20366655143379\\
21.1571241748378	6.20324130408906\\
21.156363888642	6.20281603592388\\
21.1556035358916	6.2023907469362\\
21.1548431165751	6.20196543712399\\
21.1540826306806	6.20154010648521\\
21.1533220781966	6.20111475501783\\
21.1525614591114	6.20068938271979\\
21.1518007734133	6.20026398958906\\
21.1510400210907	6.1998385756236\\
21.1502792021318	6.19941314082137\\
21.1495183165251	6.19898768518032\\
21.1487573642588	6.19856220869841\\
21.1479963453212	6.19813671137361\\
21.1472352597008	6.19771119320388\\
21.1464741073857	6.19728565418716\\
21.1457128883643	6.19686009432141\\
21.1449516026249	6.1964345136046\\
21.1441902501558	6.19600891203467\\
21.1434288309453	6.19558328960959\\
21.1426673449817	6.19515764632732\\
21.1419057922534	6.19473198218579\\
21.1411441727484	6.19430629718298\\
21.1403824864553	6.19388059131684\\
21.1396207333623	6.19345486458533\\
21.1388589134575	6.19302911698639\\
21.1380970267295	6.19260334851797\\
21.1373350731663	6.19217755917805\\
21.1365730527562	6.19175174896456\\
21.1358109654877	6.19132591787547\\
21.1350488113488	6.19090006590872\\
21.1342865903279	6.19047419306227\\
21.1335243024132	6.19004829933406\\
21.132761947593	6.18962238472207\\
21.1319995258556	6.18919644922422\\
21.1312370371892	6.18877049283848\\
21.130474481582	6.1883445155628\\
21.1297118590223	6.18791851739512\\
21.1289491694983	6.18749249833341\\
21.1281864129984	6.1870664583756\\
21.1274235895106	6.18664039751965\\
21.1266606990232	6.18621431576351\\
21.1258977415246	6.18578821310513\\
21.1251347170028	6.18536208954245\\
21.1243716254461	6.18493594507343\\
21.1236084668428	6.18450977969601\\
21.122845241181	6.18408359340815\\
21.1220819484491	6.18365738620779\\
21.1213185886351	6.18323115809287\\
21.1205551617272	6.18280490906135\\
21.1197916677138	6.18237863911117\\
21.119028106583	6.18195234824028\\
21.118264478323	6.18152603644663\\
21.1175007829219	6.18109970372816\\
21.1167370203681	6.18067335008281\\
21.1159731906496	6.18024697550854\\
21.1152092937547	6.17982058000328\\
21.1144453296715	6.17939416356499\\
21.1136812983883	6.1789677261916\\
21.1129171998931	6.17854126788107\\
21.1121530341742	6.17811478863133\\
21.1113888012198	6.17768828844033\\
21.110624501018	6.17726176730601\\
21.1098601335569	6.17683522522632\\
21.1090956988248	6.17640866219919\\
21.1083311968098	6.17598207822258\\
21.1075666275	6.17555547329442\\
21.1068019908836	6.17512884741266\\
21.1060372869488	6.17470220057523\\
21.1052725156837	6.17427553278007\\
21.1045076770764	6.17384884402514\\
21.1037427711151	6.17342213430837\\
21.1029777977879	6.17299540362769\\
21.1022127570829	6.17256865198105\\
21.1014476489884	6.17214187936639\\
21.1006824734923	6.17171508578166\\
21.0999172305828	6.17128827122477\\
21.0991519202481	6.17086143569369\\
21.0983865424763	6.17043457918634\\
21.0976210972554	6.17000770170066\\
21.0968555845737	6.1695808032346\\
21.0960900044191	6.16915388378608\\
21.0953243567798	6.16872694335305\\
21.094558641644	6.16829998193344\\
21.0937928589996	6.16787299952519\\
21.0930270088348	6.16744599612624\\
21.0922610911378	6.16701897173451\\
21.0914951058965	6.16659192634796\\
21.090729053099	6.16616485996451\\
21.0899629327335	6.1657377725821\\
21.0891967447881	6.16531066419866\\
21.0884304892507	6.16488353481213\\
21.0876641661095	6.16445638442044\\
21.0868977753526	6.16402921302152\\
21.086131316968	6.16360202061332\\
21.0853647909437	6.16317480719376\\
21.0845981972679	6.16274757276077\\
21.0838315359286	6.16232031731229\\
21.0830648069139	6.16189304084626\\
21.0822980102117	6.16146574336059\\
21.0815311458102	6.16103842485324\\
21.0807642136974	6.16061108532211\\
21.0799972138613	6.16018372476516\\
21.07923014629	6.15975634318031\\
21.0784630109715	6.15932894056548\\
21.0776958078938	6.15890151691861\\
21.0769285370451	6.15847407223763\\
21.0761611984132	6.15804660652047\\
21.0753937919862	6.15761911976506\\
21.0746263177521	6.15719161196933\\
21.0738587756991	6.1567640831312\\
21.0730911658149	6.15633653324861\\
21.0723234880878	6.15590896231948\\
21.0715557425057	6.15548137034174\\
21.0707879290565	6.15505375731331\\
21.0700200477283	6.15462612323214\\
21.0692520985091	6.15419846809613\\
21.0684840813869	6.15377079190323\\
21.0677159963496	6.15334309465135\\
21.0669478433853	6.15291537633842\\
21.066179622482	6.15248763696236\\
21.0654113336276	6.15205987652111\\
21.06464297681	6.15163209501259\\
21.0638745520174	6.15120429243472\\
21.0631060592376	6.15077646878542\\
21.0623374984586	6.15034862406263\\
21.0615688696684	6.14992075826426\\
21.0608001728549	6.14949287138823\\
21.0600314080062	6.14906496343248\\
21.0592625751101	6.14863703439493\\
21.0584936741546	6.14820908427348\\
21.0577247051276	6.14778111306608\\
21.0569556680172	6.14735312077064\\
21.0561865628112	6.14692510738508\\
21.0554173894976	6.14649707290733\\
21.0546481480644	6.1460690173353\\
21.0538788384994	6.14564094066692\\
21.0531094607905	6.1452128429001\\
21.0523400149258	6.14478472403277\\
21.0515705008932	6.14435658406285\\
21.0508009186804	6.14392842298825\\
21.0500312682756	6.1435002408069\\
21.0492615496666	6.14307203751671\\
21.0484917628412	6.1426438131156\\
21.0477219077875	6.1422155676015\\
21.0469519844933	6.14178730097232\\
21.0461819929465	6.14135901322597\\
21.045411933135	6.14093070436037\\
21.0446418050467	6.14050237437345\\
21.0438716086695	6.14007402326311\\
21.0431013439913	6.13964565102728\\
21.042331011	6.13921725766387\\
21.0415606096834	6.1387888431708\\
21.0407901400294	6.13836040754598\\
21.040019602026	6.13793195078732\\
21.0392489956609	6.13750347289275\\
21.0384783209221	6.13707497386017\\
21.0377075777974	6.13664645368751\\
21.0369367662747	6.13621791237267\\
21.0361658863418	6.13578934991357\\
21.0353949379866	6.13536076630812\\
21.034623921197	6.13493216155424\\
21.0338528359607	6.13450353564983\\
21.0330816822657	6.13407488859281\\
21.0323104600998	6.1336462203811\\
21.0315391694508	6.1332175310126\\
21.0307678103065	6.13278882048523\\
21.0299963826549	6.1323600887969\\
21.0292248864837	6.13193133594551\\
21.0284533217807	6.13150256192898\\
21.0276816885338	6.13107376674522\\
21.0269099867308	6.13064495039214\\
21.0261382163595	6.13021611286765\\
21.0253663774077	6.12978725416965\\
21.0245944698633	6.12935837429607\\
21.023822493714	6.12892947324479\\
21.0230504489476	6.12850055101375\\
21.022278335552	6.12807160760083\\
21.021506153515	6.12764264300396\\
21.0207339028243	6.12721365722103\\
21.0199615834677	6.12678465024996\\
21.019189195433	6.12635562208864\\
21.0184167387081	6.125926572735\\
21.0176442132806	6.12549750218693\\
21.0168716191384	6.12506841044234\\
21.0160989562692	6.12463929749914\\
21.0153262246608	6.12421016335523\\
21.0145534243011	6.12378100800851\\
21.0137805551776	6.1233518314569\\
21.0130076172783	6.12292263369829\\
21.0122346105908	6.12249341473059\\
21.011461535103	6.12206417455171\\
21.0106883908025	6.12163491315954\\
21.0099151776771	6.12120563055199\\
21.0091418957146	6.12077632672697\\
21.0083685449027	6.12034700168237\\
21.0075951252292	6.11991765541609\\
21.0068216366817	6.11948828792605\\
21.006048079248	6.11905889921014\\
21.0052744529159	6.11862948926626\\
21.0045007576731	6.11820005809231\\
21.0037269935072	6.1177706056862\\
21.0029531604061	6.11734113204582\\
21.0021792583574	6.11691163716908\\
21.0014052873488	6.11648212105386\\
21.000631247368	6.11605258369809\\
20.9998571384029	6.11562302509964\\
20.9990829604409	6.11519344525643\\
20.99830871347	6.11476384416635\\
20.9975343974776	6.11433422182729\\
20.9967600124517	6.11390457823717\\
20.9959855583797	6.11347491339386\\
20.9952110352495	6.11304522729528\\
20.9944364430487	6.11261551993931\\
20.993661781765	6.11218579132386\\
20.992887051386	6.11175604144681\\
20.9921122518995	6.11132627030608\\
20.991337383293	6.11089647789955\\
20.9905624455544	6.11046666422511\\
20.9897874386711	6.11003682928067\\
20.989012362631	6.10960697306411\\
20.9882372174216	6.10917709557334\\
20.9874620030306	6.10874719680624\\
20.9866867194457	6.10831727676071\\
20.9859113666544	6.10788733543464\\
20.9851359446445	6.10745737282593\\
20.9843604534036	6.10702738893248\\
20.9835848929193	6.10659738375216\\
20.9828092631793	6.10616735728287\\
20.9820335641711	6.10573730952251\\
20.9812577958825	6.10530724046897\\
20.980481958301	6.10487715012014\\
20.9797060514142	6.10444703847391\\
20.9789300752098	6.10401690552817\\
20.9781540296754	6.1035867512808\\
20.9773779147986	6.10315657572971\\
20.976601730567	6.10272637887278\\
20.9758254769682	6.1022961607079\\
20.9750491539898	6.10186592123296\\
20.9742727616194	6.10143566044585\\
20.9734962998446	6.10100537834446\\
20.972719768653	6.10057507492666\\
20.9719431680321	6.10014475019036\\
20.9711664979696	6.09971440413344\\
20.9703897584531	6.09928403675379\\
20.96961294947	6.09885364804929\\
20.9688360710081	6.09842323801783\\
20.9680591230548	6.0979928066573\\
20.9672821055977	6.09756235396558\\
20.9665050186244	6.09713187994056\\
20.9657278621225	6.09670138458013\\
20.9649506360795	6.09627086788216\\
20.9641733404829	6.09584032984455\\
20.9633959753204	6.09540977046517\\
20.9626185405793	6.09497918974192\\
20.9618410362474	6.09454858767267\\
20.9610634623122	6.09411796425531\\
20.9602858187611	6.09368731948772\\
20.9595081055817	6.09325665336779\\
20.9587303227616	6.09282596589339\\
20.9579524702882	6.09239525706242\\
20.9571745481491	6.09196452687275\\
20.9563965563319	6.09153377532226\\
20.9556184948239	6.09110300240883\\
20.9548403636128	6.09067220813035\\
20.9540621626861	6.0902413924847\\
20.9532838920312	6.08981055546975\\
20.9525055516357	6.08937969708339\\
20.951727141487	6.0889488173235\\
20.9509486615727	6.08851791618795\\
20.9501701118802	6.08808699367462\\
20.9493914923971	6.0876560497814\\
20.9486128031108	6.08722508450616\\
20.9478340440089	6.08679409784678\\
20.9470552150787	6.08636308980113\\
20.9462763163078	6.0859320603671\\
20.9454973476836	6.08550100954256\\
20.9447183091936	6.0850699373254\\
20.9439392008253	6.08463884371347\\
20.9431600225662	6.08420772870467\\
20.9423807744036	6.08377659229686\\
20.9416014563251	6.08334543448793\\
20.9408220683181	6.08291425527575\\
20.9400426103701	6.08248305465819\\
20.9392630824685	6.08205183263312\\
20.9384834846007	6.08162058919843\\
20.9377038167542	6.08118932435198\\
20.9369240789164	6.08075803809166\\
20.9361442710748	6.08032673041532\\
20.9353643932168	6.07989540132085\\
20.9345844453297	6.07946405080613\\
20.9338044274011	6.07903267886901\\
20.9330243394184	6.07860128550737\\
20.9322441813689	6.07816987071909\\
20.9314639532401	6.07773843450204\\
20.9306836550194	6.07730697685408\\
20.9299032866942	6.07687549777309\\
20.9291228482518	6.07644399725694\\
20.9283423396798	6.0760124753035\\
20.9275617609654	6.07558093191063\\
20.9267811120962	6.07514936707622\\
20.9260003930594	6.07471778079812\\
20.9252196038424	6.07428617307421\\
20.9244387444327	6.07385454390235\\
20.9236578148176	6.07342289328042\\
20.9228768149845	6.07299122120628\\
20.9220957449208	6.0725595276778\\
20.9213146046138	6.07212781269284\\
20.9205333940508	6.07169607624927\\
20.9197521132193	6.07126431834497\\
20.9189707621067	6.07083253897779\\
20.9181893407002	6.0704007381456\\
20.9174078489872	6.06996891584627\\
20.9166262869551	6.06953707207767\\
20.9158446545912	6.06910520683765\\
20.9150629518828	6.06867332012409\\
20.9142811788174	6.06824141193484\\
20.9134993353821	6.06780948226777\\
20.9127174215644	6.06737753112075\\
20.9119354373516	6.06694555849164\\
20.911153382731	6.0665135643783\\
20.9103712576899	6.0660815487786\\
20.9095890622156	6.06564951169039\\
20.9088067962955	6.06521745311154\\
20.9080244599168	6.06478537303992\\
20.907242053067	6.06435327147338\\
20.9064595757331	6.06392114840978\\
20.9056770279027	6.06348900384698\\
20.9048944095629	6.06305683778286\\
20.9041117207011	6.06262465021525\\
20.9033289613046	6.06219244114204\\
20.9025461313605	6.06176021056107\\
20.9017632308563	6.06132795847021\\
20.9009802597792	6.06089568486731\\
20.9001972181165	6.06046338975023\\
20.8994141058554	6.06003107311684\\
20.8986309229832	6.05959873496498\\
20.8978476694872	6.05916637529252\\
20.8970643453546	6.05873399409732\\
20.8962809505728	6.05830159137723\\
20.8954974851288	6.0578691671301\\
20.8947139490101	6.0574367213538\\
20.8939303422039	6.05700425404618\\
20.8931466646973	6.0565717652051\\
20.8923629164777	6.05613925482842\\
20.8915790975323	6.05570672291397\\
20.8907952078482	6.05527416945963\\
20.8900112474128	6.05484159446325\\
20.8892272162133	6.05440899792268\\
20.8884431142369	6.05397637983577\\
20.8876589414707	6.05354374020039\\
20.8868746979021	6.05311107901437\\
20.8860903835183	6.05267839627558\\
20.8853059983064	6.05224569198187\\
20.8845215422537	6.05181296613108\\
20.8837370153473	6.05138021872108\\
20.8829524175745	6.05094744974971\\
20.8821677489224	6.05051465921483\\
20.8813830093783	6.05008184711428\\
20.8805981989293	6.04964901344592\\
20.8798133175627	6.04921615820759\\
20.8790283652655	6.04878328139715\\
20.8782433420251	6.04835038301245\\
20.8774582478285	6.04791746305133\\
20.8766730826629	6.04748452151165\\
20.8758878465156	6.04705155839125\\
20.8751025393736	6.04661857368798\\
20.8743171612241	6.0461855673997\\
20.8735317120544	6.04575253952424\\
20.8727461918514	6.04531949005946\\
20.8719606006025	6.0448864190032\\
20.8711749382947	6.04445332635331\\
20.8703892049151	6.04402021210764\\
20.869603400451	6.04358707626402\\
20.8688175248895	6.04315391882032\\
20.8680315782176	6.04272073977437\\
20.8672455604226	6.04228753912402\\
20.8664594714915	6.04185431686711\\
20.8656733114115	6.0414210730015\\
20.8648870801697	6.04098780752501\\
20.8641007777532	6.0405545204355\\
20.8633144041491	6.04012121173081\\
20.8625279593445	6.03968788140878\\
20.8617414433266	6.03925452946726\\
20.8609548560824	6.03882115590409\\
20.860168197599	6.03838776071711\\
20.8593814678636	6.03795434390416\\
20.8585946668632	6.03752090546309\\
20.8578077945849	6.03708744539173\\
20.8570208510158	6.03665396368793\\
20.856233836143	6.03622046034953\\
20.8554467499535	6.03578693537437\\
20.8546595924344	6.03535338876029\\
20.8538723635729	6.03491982050512\\
20.8530850633559	6.03448623060671\\
20.8522976917705	6.0340526190629\\
20.8515102488038	6.03361898587152\\
20.8507227344429	6.03318533103042\\
20.8499351486747	6.03275165453743\\
20.8491474914864	6.03231795639039\\
20.848359762865	6.03188423658713\\
20.8475719627975	6.03145049512551\\
20.846784091271	6.03101673200334\\
20.8459961482725	6.03058294721846\\
20.845208133789	6.03014914076873\\
20.8444200478076	6.02971531265196\\
20.8436318903153	6.029281462866\\
20.8428436612991	6.02884759140867\\
20.8420553607461	6.02841369827783\\
20.8412669886431	6.02797978347129\\
20.8404785449773	6.02754584698689\\
20.8396900297357	6.02711188882248\\
20.8389014429053	6.02667790897587\\
20.838112784473	6.02624390744491\\
20.8373240544259	6.02580988422743\\
20.8365352527509	6.02537583932126\\
20.835746379435	6.02494177272423\\
20.8349574344653	6.02450768443417\\
20.8341684178287	6.02407357444892\\
20.8333793295122	6.02363944276631\\
20.8325901695028	6.02320528938416\\
20.8318009377873	6.02277111430032\\
20.8310116343529	6.0223369175126\\
20.8302222591865	6.02190269901885\\
20.8294328122749	6.02146845881688\\
20.8286432936053	6.02103419690453\\
20.8278537031645	6.02059991327962\\
20.8270640409395	6.02016560794\\
20.8262743069172	6.01973128088347\\
20.8254845010846	6.01929693210788\\
20.8246946234286	6.01886256161105\\
20.8239046739362	6.01842816939081\\
20.8231146525942	6.01799375544498\\
20.8223245593897	6.01755931977139\\
20.8215343943095	6.01712486236787\\
20.8207441573406	6.01669038323225\\
20.8199538484698	6.01625588236234\\
20.8191634676841	6.01582135975598\\
20.8183730149705	6.01538681541098\\
20.8175824903157	6.01495224932519\\
20.8167918937068	6.01451766149641\\
20.8160012251306	6.01408305192247\\
20.815210484574	6.0136484206012\\
20.8144196720239	6.01321376753043\\
20.8136287874672	6.01277909270796\\
20.8128378308907	6.01234439613164\\
20.8120468022815	6.01190967779927\\
20.8112557016262	6.01147493770868\\
20.8104645289119	6.0110401758577\\
20.8096732841254	6.01060539224414\\
20.8088819672535	6.01017058686583\\
20.8080905782832	6.00973575972058\\
20.8072991172012	6.00930091080622\\
20.8065075839945	6.00886604012057\\
20.8057159786498	6.00843114766144\\
20.8049243011541	6.00799623342666\\
20.8041325514942	6.00756129741405\\
20.8033407296569	6.00712633962142\\
20.8025488356291	6.0066913600466\\
20.8017568693976	6.00625635868739\\
20.8009648309492	6.00582133554163\\
20.8001727202708	6.00538629060713\\
20.7993805373492	6.0049512238817\\
20.7985882821712	6.00451613536315\\
20.7977959547236	6.00408102504932\\
20.7970035549933	6.00364589293801\\
20.796211082967	6.00321073902705\\
20.7954185386316	6.00277556331423\\
20.7946259219738	6.00234036579739\\
20.7938332329805	6.00190514647434\\
20.7930404716385	6.00146990534288\\
20.7922476379345	6.00103464240084\\
20.7914547318553	6.00059935764603\\
20.7906617533878	6.00016405107626\\
20.7898687025187	5.99972872268935\\
20.7890755792347	5.9992933724831\\
20.7882823835227	5.99885800045534\\
20.7874891153695	5.99842260660387\\
20.7866957747617	5.99798719092651\\
20.7859023616862	5.99755175342106\\
20.7851088761297	5.99711629408534\\
20.784315318079	5.99668081291717\\
20.7835216875208	5.99624530991434\\
20.7827279844419	5.99580978507467\\
20.781934208829	5.99537423839598\\
20.7811403606689	5.99493866987606\\
20.7803464399482	5.99450307951274\\
20.7795524466538	5.99406746730381\\
20.7787583807724	5.99363183324709\\
20.7779642422907	5.99319617734039\\
20.7771700311953	5.99276049958151\\
20.7763757474731	5.99232479996826\\
20.7755813911108	5.99188907849846\\
20.774786962095	5.99145333516989\\
20.7739924604125	5.99101756998039\\
20.7731978860499	5.99058178292774\\
20.7724032389941	5.99014597400975\\
20.7716085192316	5.98971014322424\\
20.7708137267491	5.989274290569\\
20.7700188615335	5.98883841604184\\
20.7692239235713	5.98840251964058\\
20.7684289128492	5.987966601363\\
20.7676338293539	5.98753066120691\\
20.7668386730721	5.98709469917013\\
20.7660434439905	5.98665871525045\\
20.7652481420957	5.98622270944567\\
20.7644527673744	5.9857866817536\\
20.7636573198132	5.98535063217204\\
20.7628617993988	5.98491456069879\\
20.762066206118	5.98447846733166\\
20.7612705399572	5.98404235206844\\
20.7604748009032	5.98360621490693\\
20.7596789889425	5.98317005584495\\
20.758883104062	5.98273387488028\\
20.7580871462481	5.98229767201074\\
20.7572911154875	5.9818614472341\\
20.7564950117668	5.98142520054819\\
20.7556988350727	5.98098893195079\\
20.7549025853918	5.98055264143971\\
20.7541062627107	5.98011632901274\\
20.753309867016	5.97967999466768\\
20.7525133982944	5.97924363840233\\
20.7517168565323	5.97880726021449\\
20.7509202417165	5.97837086010196\\
20.7501235538336	5.97793443806252\\
20.74932679287	5.97749799409398\\
20.7485299588125	5.97706152819413\\
20.7477330516475	5.97662504036078\\
20.7469360713618	5.9761885305917\\
20.7461390179418	5.97575199888471\\
20.7453418913742	5.97531544523759\\
20.7445446916455	5.97487886964814\\
20.7437474187423	5.97444227211415\\
20.7429500726511	5.97400565263343\\
20.7421526533585	5.97356901120375\\
20.7413551608512	5.97313234782291\\
20.7405575951155	5.97269566248871\\
20.7397599561381	5.97225895519894\\
20.7389622439056	5.97182222595139\\
20.7381644584044	5.97138547474385\\
20.7373665996211	5.97094870157412\\
20.7365686675423	5.97051190643998\\
20.7357706621544	5.97007508933923\\
20.734972583444	5.96963825026966\\
20.7341744313977	5.96920138922905\\
20.7333762060019	5.9687645062152\\
20.7325779072432	5.96832760122589\\
20.7317795351081	5.96789067425892\\
20.730981089583	5.96745372531207\\
20.7301825706546	5.96701675438314\\
20.7293839783092	5.9665797614699\\
20.7285853125335	5.96614274657016\\
20.7277865733138	5.96570570968169\\
20.7269877606368	5.96526865080228\\
20.7261888744888	5.96483156992972\\
20.7253899148563	5.9643944670618\\
20.7245908817259	5.9639573421963\\
20.723791775084	5.96352019533101\\
20.7229925949171	5.96308302646371\\
20.7221933412116	5.96264583559219\\
20.721394013954	5.96220862271424\\
20.7205946131309	5.96177138782763\\
20.7197951387285	5.96133413093016\\
20.7189955907335	5.9608968520196\\
20.7181959691322	5.96045955109374\\
20.7173962739111	5.96002222815036\\
20.7165965050566	5.95958488318725\\
20.7157966625551	5.95914751620218\\
20.7149967463932	5.95871012719295\\
20.7141967565572	5.95827271615732\\
20.7133966930336	5.95783528309309\\
20.7125965558088	5.95739782799804\\
20.7117963448691	5.95696035086993\\
20.7109960602011	5.95652285170656\\
20.7101957017912	5.95608533050572\\
20.7093952696256	5.95564778726516\\
20.708594763691	5.95521022198268\\
20.7077941839735	5.95477263465605\\
20.7069935304598	5.95433502528305\\
20.706192803136	5.95389739386147\\
20.7053920019887	5.95345974038907\\
20.7045911270042	5.95302206486364\\
20.703790178169	5.95258436728296\\
20.7029891554693	5.95214664764479\\
20.7021880588915	5.95170890594693\\
20.7013868884221	5.95127114218714\\
20.7005856440474	5.9508333563632\\
20.6997843257537	5.95039554847288\\
20.6989829335274	5.94995771851397\\
20.698181467355	5.94951986648423\\
20.6973799272226	5.94908199238145\\
20.6965783131167	5.94864409620338\\
20.6957766250237	5.94820617794782\\
20.6949748629298	5.94776823761254\\
20.6941730268214	5.9473302751953\\
20.6933711166848	5.94689229069388\\
20.6925691325064	5.94645428410605\\
20.6917670742725	5.94601625542959\\
20.6909649419694	5.94557820466226\\
20.6901627355834	5.94514013180185\\
20.6893604551009	5.94470203684611\\
20.6885581005081	5.94426391979283\\
20.6877556717913	5.94382578063976\\
20.686953168937	5.94338761938469\\
20.6861505919313	5.94294943602538\\
20.6853479407605	5.94251123055961\\
20.684545215411	5.94207300298513\\
20.6837424158691	5.94163475329972\\
20.6829395421209	5.94119648150116\\
20.6821365941529	5.9407581875872\\
20.6813335719512	5.94031987155561\\
20.6805304755022	5.93988153340417\\
20.6797273047921	5.93944317313063\\
20.6789240598072	5.93900479073278\\
20.6781207405338	5.93856638620837\\
20.677317346958	5.93812795955516\\
20.6765138790662	5.93768951077094\\
20.6757103368446	5.93725103985346\\
20.6749067202795	5.93681254680048\\
20.674103029357	5.93637403160977\\
20.6732992640635	5.9359354942791\\
20.6724954243852	5.93549693480623\\
20.6716915103083	5.93505835318893\\
20.670887521819	5.93461974942495\\
20.6700834589035	5.93418112351207\\
20.6692793215481	5.93374247544804\\
20.668475109739	5.93330380523062\\
20.6676708234624	5.93286511285759\\
20.6668664627045	5.93242639832669\\
20.6660620274515	5.9319876616357\\
20.6652575176896	5.93154890278237\\
20.664452933405	5.93111012176447\\
20.6636482745839	5.93067131857975\\
20.6628435412125	5.93023249322598\\
20.662038733277	5.92979364570091\\
20.6612338507635	5.9293547760023\\
20.6604288936582	5.92891588412793\\
20.6596238619474	5.92847697007553\\
20.6588187556171	5.92803803384287\\
20.6580135746535	5.92759907542771\\
20.6572083190429	5.92716009482781\\
20.6564029887713	5.92672109204093\\
20.6555975838249	5.92628206706482\\
20.6547921041899	5.92584301989723\\
20.6539865498523	5.92540395053593\\
20.6531809207985	5.92496485897867\\
20.6523752170144	5.92452574522321\\
20.6515694384862	5.9240866092673\\
20.6507635852001	5.9236474511087\\
20.6499576571422	5.92320827074516\\
20.6491516542986	5.92276906817444\\
20.6483455766554	5.92232984339428\\
20.6475394241987	5.92189059640245\\
20.6467331969147	5.9214513271967\\
20.6459268947895	5.92101203577478\\
20.6451205178091	5.92057272213445\\
20.6443140659596	5.92013338627345\\
20.6435075392273	5.91969402818953\\
20.642700937598	5.91925464788046\\
20.641894261058	5.91881524534398\\
20.6410875095934	5.91837582057784\\
20.6402806831901	5.91793637357979\\
20.6394737818343	5.91749690434758\\
20.6386668055121	5.91705741287897\\
20.6378597542095	5.9166178991717\\
20.6370526279126	5.91617836322352\\
20.6362454266074	5.91573880503219\\
20.63543815028	5.91529922459544\\
20.6346307989165	5.91485962191103\\
20.6338233725029	5.9144199969767\\
20.6330158710253	5.91398034979021\\
20.6322082944696	5.9135406803493\\
20.631400642822	5.91310098865172\\
20.6305929160685	5.91266127469521\\
20.629785114195	5.91222153847752\\
20.6289772371877	5.91178177999639\\
20.6281692850325	5.91134199924958\\
20.6273612577155	5.91090219623482\\
20.6265531552226	5.91046237094987\\
20.62574497754	5.91002252339246\\
20.6249367246535	5.90958265356034\\
20.6241283965492	5.90914276145126\\
20.6233199932132	5.90870284706295\\
20.6225115146313	5.90826291039317\\
20.6217029607896	5.90782295143964\\
20.6208943316741	5.90738297020013\\
20.6200856272707	5.90694296667236\\
20.6192768475655	5.90650294085408\\
20.6184679925444	5.90606289274303\\
20.6176590621933	5.90562282233696\\
20.6168500564983	5.90518272963359\\
20.6160409754454	5.90474261463068\\
20.6152318190203	5.90430247732597\\
20.6144225872092	5.90386231771718\\
20.613613279998	5.90342213580207\\
20.6128038973726	5.90298193157837\\
20.611994439319	5.90254170504382\\
20.6111849058231	5.90210145619616\\
20.6103752968708	5.90166118503313\\
20.6095656124482	5.90122089155246\\
20.608755852541	5.90078057575189\\
20.6079460171352	5.90034023762916\\
20.6071361062168	5.899899877182\\
20.6063261197717	5.89945949440816\\
20.6055160577858	5.89901908930536\\
20.6047059202449	5.89857866187134\\
20.603895707135	5.89813821210385\\
20.603085418442	5.8976977400006\\
20.6022750541519	5.89725724555935\\
20.6014646142504	5.89681672877781\\
20.6006540987234	5.89637618965373\\
20.5998435075569	5.89593562818484\\
20.5990328407368	5.89549504436887\\
20.5982220982489	5.89505443820356\\
20.5974112800791	5.89461380968663\\
20.5966003862132	5.89417315881582\\
20.5957894166371	5.89373248558886\\
20.5949783713367	5.89329179000349\\
20.5941672502979	5.89285107205743\\
20.5933560535065	5.89241033174841\\
20.5925447809483	5.89196956907417\\
20.5917334326091	5.89152878403244\\
20.590922008475	5.89108797662093\\
20.5901105085315	5.8906471468374\\
20.5892989327647	5.89020629467955\\
20.5884872811603	5.88976542014513\\
20.5876755537042	5.88932452323186\\
20.5868637503821	5.88888360393746\\
20.58605187118	5.88844266225967\\
20.5852399160835	5.88800169819622\\
20.5844278850786	5.88756071174482\\
20.5836157781509	5.88711970290321\\
20.5828035952864	5.88667867166911\\
20.5819913364709	5.88623761804025\\
20.58117900169	5.88579654201436\\
20.5803665909296	5.88535544358916\\
20.5795541041756	5.88491432276236\\
20.5787415414136	5.88447317953171\\
20.5779289026294	5.88403201389492\\
20.5771161878089	5.88359082584972\\
20.5763033969377	5.88314961539383\\
20.5754905300018	5.88270838252497\\
20.5746775869867	5.88226712724086\\
20.5738645678783	5.88182584953924\\
20.5730514726624	5.88138454941781\\
20.5722383013246	5.88094322687431\\
20.5714250538508	5.88050188190645\\
20.5706117302266	5.88006051451196\\
20.5697983304378	5.87961912468855\\
20.5689848544702	5.87917771243395\\
20.5681713023094	5.87873627774587\\
20.5673576739412	5.87829482062204\\
20.5665439693514	5.87785334106018\\
20.5657301885256	5.87741183905799\\
20.5649163314495	5.87697031461321\\
20.5641023981089	5.87652876772355\\
20.5632883884894	5.87608719838673\\
20.5624743025768	5.87564560660047\\
20.5616601403568	5.87520399236248\\
20.560845901815	5.87476235567048\\
20.5600315869372	5.87432069652218\\
20.5592171957089	5.87387901491531\\
20.5584027281161	5.87343731084758\\
20.5575881841441	5.8729955843167\\
20.5567735637789	5.87255383532039\\
20.555958867006	5.87211206385636\\
20.5551440938111	5.87167026992234\\
20.5543292441798	5.87122845351602\\
20.5535143180979	5.87078661463513\\
20.5526993155509	5.87034475327739\\
20.5518842365245	5.86990286944049\\
20.5510690810044	5.86946096312216\\
20.5502538489762	5.86901903432011\\
20.5494385404256	5.86857708303205\\
20.5486231553381	5.86813510925568\\
20.5478076936994	5.86769311298873\\
20.5469921554951	5.86725109422891\\
20.5461765407109	5.86680905297392\\
20.5453608493324	5.86636698922147\\
20.5445450813451	5.86592490296928\\
20.5437292367347	5.86548279421505\\
20.5429133154868	5.8650406629565\\
20.542097317587	5.86459850919132\\
20.5412812430209	5.86415633291724\\
20.5404650917741	5.86371413413196\\
20.5396488638321	5.86327191283318\\
20.5388325591806	5.86282966901862\\
20.5380161778051	5.86238740268597\\
20.5371997196912	5.86194511383296\\
20.5363831848245	5.86150280245728\\
20.5355665731906	5.86106046855664\\
20.534749884775	5.86061811212874\\
20.5339331195632	5.8601757331713\\
20.5331162775409	5.85973333168201\\
20.5322993586935	5.85929090765858\\
20.5314823630067	5.85884846109872\\
20.5306652904659	5.85840599200013\\
20.5298481410568	5.8579635003605\\
20.5290309147648	5.85752098617756\\
20.5282136115755	5.85707844944899\\
20.5273962314744	5.8566358901725\\
20.526578774447	5.8561933083458\\
20.5257612404788	5.85575070396659\\
20.5249436295555	5.85530807703256\\
20.5241259416624	5.85486542754141\\
20.523308176785	5.85442275549086\\
20.522490334909	5.85398006087859\\
20.5216724160197	5.85353734370232\\
20.5208544201028	5.85309460395973\\
20.5200363471435	5.85265184164853\\
20.5192181971276	5.85220905676642\\
20.5183999700403	5.85176624931109\\
20.5175816658673	5.85132341928025\\
20.516763284594	5.85088056667159\\
20.5159448262058	5.85043769148281\\
20.5151262906883	5.84999479371161\\
20.5143076780268	5.84955187335568\\
20.5134889882069	5.84910893041272\\
20.512670221214	5.84866596488043\\
20.5118513770335	5.8482229767565\\
20.5110324556509	5.84777996603863\\
20.5102134570516	5.84733693272451\\
20.5093943812211	5.84689387681184\\
20.5085752281448	5.84645079829831\\
20.5077559978082	5.84600769718162\\
20.5069366901965	5.84556457345946\\
20.5061173052954	5.84512142712952\\
20.5052978430902	5.8446782581895\\
20.5044783035662	5.84423506663709\\
20.503658686709	5.84379185246998\\
20.5028389925039	5.84334861568586\\
20.5020192209363	5.84290535628243\\
20.5011993719916	5.84246207425737\\
20.5003794456552	5.84201876960838\\
20.4995594419125	5.84157544233315\\
20.4987393607489	5.84113209242937\\
20.4979192021497	5.84068871989473\\
20.4970989661004	5.84024532472691\\
20.4962786525862	5.83980190692361\\
20.4954582615926	5.83935846648251\\
20.494637793105	5.83891500340131\\
20.4938172471086	5.83847151767769\\
20.4929966235889	5.83802800930934\\
20.4921759225312	5.83758447829395\\
20.4913551439208	5.8371409246292\\
20.490534287743	5.83669734831278\\
20.4897133539833	5.83625374934238\\
20.488892342627	5.83581012771568\\
20.4880712536593	5.83536648343037\\
20.4872500870656	5.83492281648413\\
20.4864288428313	5.83447912687465\\
20.4856075209416	5.83403541459962\\
20.4847861213819	5.83359167965671\\
20.4839646441374	5.83314792204361\\
20.4831430891935	5.83270414175801\\
20.4823214565354	5.83226033879759\\
20.4814997461485	5.83181651316002\\
20.4806779580181	5.831372664843\\
20.4798560921294	5.8309287938442\\
20.4790341484678	5.83048490016131\\
20.4782121270184	5.83004098379201\\
20.4773900277666	5.82959704473398\\
20.4765678506977	5.82915308298489\\
20.4757455957969	5.82870909854244\\
20.4749232630495	5.8282650914043\\
20.4741008524407	5.82782106156814\\
20.4732783639559	5.82737700903165\\
20.4724557975801	5.82693293379251\\
20.4716331532988	5.8264888358484\\
20.4708104310971	5.82604471519699\\
20.4699876309602	5.82560057183596\\
20.4691647528735	5.82515640576299\\
20.4683417968221	5.82471221697576\\
20.4675187627912	5.82426800547193\\
20.4666956507662	5.8238237712492\\
20.4658724607321	5.82337951430524\\
20.4650491926742	5.82293523463771\\
20.4642258465778	5.8224909322443\\
20.4634024224279	5.82204660712268\\
20.4625789202099	5.82160225927053\\
20.4617553399089	5.82115788868552\\
20.4609316815101	5.82071349536532\\
20.4601079449987	5.82026907930761\\
20.4592841303599	5.81982464051005\\
20.4584602375788	5.81938017897033\\
20.4576362666406	5.81893569468612\\
20.4568122175306	5.81849118765508\\
20.4559880902338	5.8180466578749\\
20.4551638847354	5.81760210534323\\
20.4543396010206	5.81715753005776\\
20.4535152390745	5.81671293201615\\
20.4526907988824	5.81626831121607\\
20.4518662804292	5.81582366765519\\
20.4510416837003	5.81537900133118\\
20.4502170086806	5.81493431224172\\
20.4493922553554	5.81448960038446\\
20.4485674237097	5.81404486575709\\
20.4477425137288	5.81360010835726\\
20.4469175253976	5.81315532818264\\
20.4460924587014	5.81271052523091\\
20.4452673136252	5.81226569949973\\
20.4444420901541	5.81182085098676\\
20.4436167882733	5.81137597968967\\
20.4427914079678	5.81093108560613\\
20.4419659492228	5.81048616873381\\
20.4411404120233	5.81004122907036\\
20.4403147963544	5.80959626661346\\
20.4394891022012	5.80915128136076\\
20.4386633295488	5.80870627330994\\
20.4378374783822	5.80826124245866\\
20.4370115486865	5.80781618880457\\
20.4361855404468	5.80737111234535\\
20.4353594536481	5.80692601307865\\
20.4345332882755	5.80648089100214\\
20.4337070443141	5.80603574611348\\
20.4328807217489	5.80559057841034\\
20.4320543205648	5.80514538789036\\
20.4312278407471	5.80470017455122\\
20.4304012822807	5.80425493839058\\
20.4295746451505	5.80380967940608\\
20.4287479293418	5.80336439759541\\
20.4279211348394	5.8029190929562\\
20.4270942616285	5.80247376548613\\
20.4262673096939	5.80202841518286\\
20.4254402790208	5.80158304204403\\
20.424613169594	5.80113764606731\\
20.4237859813988	5.80069222725036\\
20.4229587144199	5.80024678559083\\
20.4221313686424	5.79980132108638\\
20.4213039440514	5.79935583373467\\
20.4204764406317	5.79891032353336\\
20.4196488583684	5.79846479048009\\
20.4188211972464	5.79801923457253\\
20.4179934572507	5.79757365580833\\
20.4171656383663	5.79712805418514\\
20.4163377405781	5.79668242970062\\
20.4155097638712	5.79623678235243\\
20.4146817082303	5.79579111213821\\
20.4138535736406	5.79534541905563\\
20.4130253600868	5.79489970310232\\
20.4121970675541	5.79445396427596\\
20.4113686960273	5.79400820257418\\
20.4105402454913	5.79356241799465\\
20.409711715931	5.793116610535\\
20.4088831073315	5.7926707801929\\
20.4080544196776	5.79222492696599\\
20.4072256529542	5.79177905085193\\
20.4063968071462	5.79133315184837\\
20.4055678822385	5.79088722995295\\
20.4047388782162	5.79044128516332\\
20.4039097950639	5.78999531747713\\
20.4030806327667	5.78954932689204\\
20.4022513913093	5.78910331340569\\
20.4014220706768	5.78865727701572\\
20.400592670854	5.78821121771979\\
20.3997631918257	5.78776513551555\\
20.3989336335768	5.78731903040063\\
20.3981039960923	5.78687290237269\\
20.3972742793569	5.78642675142937\\
20.3964444833555	5.78598057756833\\
20.3956146080729	5.78553438078719\\
20.3947846534941	5.78508816108361\\
20.3939546196038	5.78464191845524\\
20.393124506387	5.78419565289972\\
20.3922943138283	5.78374936441468\\
20.3914640419128	5.78330305299778\\
20.3906336906251	5.78285671864666\\
20.3898032599502	5.78241036135896\\
20.3889727498727	5.78196398113233\\
20.3881421603777	5.7815175779644\\
20.3873114914498	5.78107115185282\\
20.3864807430738	5.78062470279523\\
20.3856499152346	5.78017823078926\\
20.384819007917	5.77973173583257\\
20.3839880211058	5.7792852179228\\
20.3831569547856	5.77883867705757\\
20.3823258089415	5.77839211323453\\
20.381494583558	5.77794552645132\\
20.38066327862	5.77749891670559\\
20.3798318941122	5.77705228399496\\
20.3790004300195	5.77660562831707\\
20.3781688863265	5.77615894966957\\
20.3773372630181	5.77571224805009\\
20.376505560079	5.77526552345627\\
20.3756737774939	5.77481877588574\\
20.3748419152476	5.77437200533614\\
20.3740099733249	5.77392521180512\\
20.3731779517104	5.77347839529029\\
20.3723458503888	5.7730315557893\\
20.371513669345	5.77258469329978\\
20.3706814085637	5.77213780781937\\
20.3698490680295	5.7716908993457\\
20.3690166477271	5.77124396787641\\
20.3681841476414	5.77079701340913\\
20.3673515677569	5.77035003594148\\
20.3665189080584	5.76990303547111\\
20.3656861685306	5.76945601199565\\
20.3648533491582	5.76900896551273\\
20.3640204499258	5.76856189601997\\
20.3631874708182	5.76811480351502\\
20.36235441182	5.76766768799549\\
20.3615212729159	5.76722054945904\\
20.3606880540906	5.76677338790327\\
20.3598547553287	5.76632620332583\\
20.3590213766148	5.76587899572433\\
20.3581879179338	5.76543176509642\\
20.3573543792701	5.76498451143972\\
20.3565207606085	5.76453723475186\\
20.3556870619336	5.76408993503046\\
20.35485328323	5.76364261227315\\
20.3540194244823	5.76319526647756\\
20.3531854856753	5.76274789764133\\
20.3523514667935	5.76230050576206\\
20.3515173678215	5.7618530908374\\
20.350683188744	5.76140565286496\\
20.3498489295455	5.76095819184237\\
20.3490145902107	5.76051070776726\\
20.3481801707242	5.76006320063724\\
20.3473456710706	5.75961567044995\\
20.3465110912344	5.75916811720301\\
20.3456764312004	5.75872054089405\\
20.3448416909529	5.75827294152067\\
20.3440068704767	5.75782531908052\\
20.3431719697563	5.7573776735712\\
20.3423369887763	5.75693000499035\\
20.3415019275212	5.75648231333558\\
20.3406667859756	5.75603459860452\\
20.3398315641241	5.75558686079478\\
20.3389962619512	5.75513909990399\\
20.3381608794415	5.75469131592977\\
20.3373254165795	5.75424350886974\\
20.3364898733498	5.7537956787215\\
20.3356542497368	5.7533478254827\\
20.3348185457252	5.75289994915094\\
20.3339827612994	5.75245204972384\\
20.333146896444	5.75200412719902\\
20.3323109511435	5.75155618157411\\
20.3314749253825	5.7511082128467\\
20.3306388191453	5.75066022101443\\
20.3298026324166	5.75021220607491\\
20.3289663651808	5.74976416802575\\
20.3281300174224	5.74931610686458\\
20.3272935891259	5.74886802258899\\
20.3264570802758	5.74841991519663\\
20.3256204908566	5.74797178468508\\
20.3247838208528	5.74752363105198\\
20.3239470702488	5.74707545429493\\
20.3231102390291	5.74662725441155\\
20.3222733271782	5.74617903139945\\
20.3214363346805	5.74573078525624\\
20.3205992615205	5.74528251597954\\
20.3197621076826	5.74483422356696\\
20.3189248731513	5.7443859080161\\
20.3180875579111	5.74393756932459\\
20.3172501619463	5.74348920749003\\
20.3164126852414	5.74304082251004\\
20.3155751277808	5.74259241438222\\
20.314737489549	5.74214398310417\\
20.3138997705304	5.74169552867353\\
20.3130619707093	5.74124705108788\\
20.3122240900702	5.74079855034485\\
20.3113861285976	5.74035002644203\\
20.3105480862757	5.73990147937704\\
20.3097099630891	5.73945290914748\\
20.308871759022	5.73900431575097\\
20.308033474059	5.73855569918511\\
20.3071951081843	5.7381070594475\\
20.3063566613823	5.73765839653575\\
20.3055181336375	5.73720971044747\\
20.3046795249342	5.73676100118026\\
20.3038408352567	5.73631226873173\\
20.3030020645894	5.73586351309948\\
20.3021632129167	5.73541473428112\\
20.301324280223	5.73496593227424\\
20.3004852664925	5.73451710707647\\
20.2996461717096	5.73406825868538\\
20.2988069958587	5.7336193870986\\
20.2979677389241	5.73317049231373\\
20.2971284008901	5.73272157432835\\
20.2962889817411	5.73227263314009\\
20.2954494814613	5.73182366874652\\
20.2946099000351	5.73137468114528\\
20.2937702374468	5.73092567033393\\
20.2929304936806	5.7304766363101\\
20.292090668721	5.73002757907138\\
20.2912507625522	5.72957849861536\\
20.2904107751585	5.72912939493966\\
20.2895707065241	5.72868026804186\\
20.2887305566334	5.72823111791956\\
20.2878903254707	5.72778194457037\\
20.2870500130202	5.72733274799188\\
20.2862096192661	5.72688352818169\\
20.2853691441929	5.72643428513739\\
20.2845285877846	5.72598501885658\\
20.2836879500256	5.72553572933686\\
20.2828472309001	5.72508641657582\\
20.2820064303924	5.72463708057106\\
20.2811655484867	5.72418772132017\\
20.2803245851673	5.72373833882076\\
20.2794835404183	5.7232889330704\\
20.2786424142241	5.7228395040667\\
20.2778012065688	5.72239005180725\\
20.2769599174366	5.72194057628964\\
20.2761185468119	5.72149107751147\\
20.2752770946787	5.72104155547033\\
20.2744355610213	5.7205920101638\\
20.2735939458239	5.72014244158949\\
20.2727522490706	5.71969284974498\\
20.2719104707458	5.71924323462787\\
20.2710686108335	5.71879359623573\\
20.270226669318	5.71834393456618\\
20.2693846461834	5.71789424961678\\
20.2685425414139	5.71744454138514\\
20.2677003549937	5.71699480986885\\
20.2668580869069	5.71654505506548\\
20.2660157371378	5.71609527697263\\
20.2651733056703	5.71564547558789\\
20.2643307924888	5.71519565090885\\
20.2634881975773	5.71474580293309\\
20.2626455209201	5.71429593165819\\
20.2618027625011	5.71384603708176\\
20.2609599223046	5.71339611920136\\
20.2601170003147	5.71294617801459\\
20.2592739965156	5.71249621351903\\
20.2584309108912	5.71204622571227\\
20.2575877434258	5.71159621459189\\
20.2567444941034	5.71114618015547\\
20.2559011629083	5.71069612240061\\
20.2550577498243	5.71024604132488\\
20.2542142548357	5.70979593692586\\
20.2533706779266	5.70934580920114\\
20.252527019081	5.70889565814831\\
20.251683278283	5.70844548376493\\
20.2508394555167	5.7079952860486\\
20.2499955507662	5.7075450649969\\
20.2491515640155	5.7070948206074\\
20.2483074952487	5.70664455287769\\
20.2474633444498	5.70619426180534\\
20.2466191116029	5.70574394738794\\
20.2457747966921	5.70529360962307\\
20.2449303997014	5.7048432485083\\
20.2440859206149	5.70439286404121\\
20.2432413594165	5.70394245621939\\
20.2423967160903	5.7034920250404\\
20.2415519906203	5.70304157050183\\
20.2407071829906	5.70259109260125\\
20.2398622931851	5.70214059133624\\
20.239017321188	5.70169006670437\\
20.2381722669831	5.70123951870323\\
20.2373271305545	5.70078894733039\\
20.2364819118862	5.70033835258342\\
20.2356366109621	5.69988773445989\\
20.2347912277664	5.69943709295739\\
20.2339457622828	5.69898642807348\\
20.2331002144956	5.69853573980574\\
20.2322545843885	5.69808502815174\\
20.2314088719456	5.69763429310906\\
20.2305630771508	5.69718353467526\\
20.2297171999881	5.69673275284793\\
20.2288712404415	5.69628194762462\\
20.2280251984949	5.69583111900292\\
20.2271790741322	5.69538026698039\\
20.2263328673374	5.69492939155461\\
20.2254865780945	5.69447849272314\\
20.2246402063873	5.69402757048355\\
20.2237937521998	5.69357662483342\\
20.222947215516	5.69312565577031\\
20.2221005963196	5.6926746632918\\
20.2212538945947	5.69222364739544\\
20.2204071103251	5.69177260807881\\
20.2195602434948	5.69132154533947\\
20.2187132940877	5.690870459175\\
20.2178662620876	5.69041934958295\\
20.2170191474785	5.68996821656091\\
20.2161719502441	5.68951706010642\\
20.2153246703685	5.68906588021705\\
20.2144773078355	5.68861467689039\\
20.2136298626289	5.68816345012398\\
20.2127823347326	5.68771219991539\\
20.2119347241305	5.68726092626219\\
20.2110870308064	5.68680962916193\\
20.2102392547443	5.68635830861219\\
20.2093913959278	5.68590696461053\\
20.208543454341	5.6854555971545\\
20.2076954299675	5.68500420624168\\
20.2068473227914	5.68455279186962\\
20.2059991327962	5.68410135403588\\
20.205150859966	5.68364989273803\\
20.2043025042845	5.68319840797363\\
20.2034540657356	5.68274689974023\\
20.202605544303	5.68229536803539\\
20.2017569399705	5.68184381285668\\
20.200908252722	5.68139223420166\\
20.2000594825412	5.68094063206788\\
20.199210629412	5.68048900645291\\
20.198361693318	5.68003735735429\\
20.1975126742432	5.6795856847696\\
20.1966635721713	5.67913398869637\\
20.195814387086	5.67868226913218\\
20.1949651189711	5.67823052607458\\
20.1941157678104	5.67777875952113\\
20.1932663335876	5.67732696946937\\
20.1924168162865	5.67687515591687\\
20.1915672158908	5.67642331886118\\
20.1907175323844	5.67597145829986\\
20.1898677657508	5.67551957423046\\
20.1890179159739	5.67506766665053\\
20.1881679830373	5.67461573555763\\
20.1873179669249	5.67416378094931\\
20.1864678676203	5.67371180282312\\
20.1856176851072	5.67325980117661\\
20.1847674193693	5.67280777600735\\
20.1839170703904	5.67235572731287\\
20.1830666381542	5.67190365509074\\
20.1822161226443	5.67145155933849\\
20.1813655238444	5.67099944005369\\
20.1805148417382	5.67054729723387\\
20.1796640763094	5.6700951308766\\
20.1788132275417	5.66964294097942\\
20.1779622954188	5.66919072753987\\
20.1771112799243	5.66873849055552\\
20.1762601810418	5.6682862300239\\
20.1754089987551	5.66783394594256\\
20.1745577330477	5.66738163830905\\
20.1737063839034	5.66692930712093\\
20.1728549513057	5.66647695237572\\
20.1720034352384	5.66602457407099\\
20.1711518356849	5.66557217220427\\
20.1703001526291	5.66511974677312\\
20.1694483860544	5.66466729777507\\
20.1685965359446	5.66421482520767\\
20.1677446022832	5.66376232906847\\
20.1668925850538	5.66330980935501\\
20.16604048424	5.66285726606483\\
20.1651882998255	5.66240469919548\\
20.1643360317938	5.6619521087445\\
20.1634836801285	5.66149949470943\\
20.1626312448132	5.66104685708781\\
20.1617787258315	5.66059419587719\\
20.160926123167	5.6601415110751\\
20.1600734368032	5.65968880267909\\
20.1592206667237	5.6592360706867\\
20.158367812912	5.65878331509547\\
20.1575148753517	5.65833053590293\\
20.1566618540264	5.65787773310663\\
20.1558087489196	5.65742490670411\\
20.1549555600148	5.6569720566929\\
20.1541022872956	5.65651918307054\\
20.1532489307455	5.65606628583457\\
20.1523954903481	5.65561336498253\\
20.1515419660868	5.65516042051195\\
20.1506883579452	5.65470745242037\\
20.1498346659067	5.65425446070533\\
20.148980889955	5.65380144536436\\
20.1481270300734	5.653348406395\\
20.1472730862456	5.65289534379479\\
20.1464190584549	5.65244225756125\\
20.1455649466849	5.65198914769193\\
20.144710750919	5.65153601418435\\
20.1438564711408	5.65108285703605\\
20.1430021073337	5.65062967624456\\
20.1421476594811	5.65017647180742\\
20.1412931275666	5.64972324372215\\
20.1404385115736	5.6492699919863\\
20.1395838114855	5.64881671659739\\
20.1387290272859	5.64836341755295\\
20.137874158958	5.64791009485051\\
20.1370192064854	5.64745674848761\\
20.1361641698516	5.64700337846177\\
20.1353090490399	5.64654998477052\\
20.1344538440337	5.6460965674114\\
20.1335985548165	5.64564312638193\\
20.1327431813717	5.64518966167964\\
20.1318877236827	5.64473617330206\\
20.131032181733	5.64428266124671\\
20.1301765555058	5.64382912551112\\
20.1293208449846	5.64337556609283\\
20.1284650501529	5.64292198298935\\
20.1276091709939	5.64246837619822\\
20.126753207491	5.64201474571695\\
20.1258971596277	5.64156109154308\\
20.1250410273873	5.64110741367413\\
20.1241848107531	5.64065371210762\\
20.1233285097086	5.64019998684108\\
20.1224721242371	5.63974623787202\\
20.1216156543219	5.63929246519799\\
20.1207590999463	5.63883866881649\\
20.1199024610938	5.63838484872506\\
20.1190457377477	5.6379310049212\\
20.1181889298912	5.63747713740246\\
20.1173320375077	5.63702324616633\\
20.1164750605805	5.63656933121036\\
20.115617999093	5.63611539253206\\
20.1147608530285	5.63566143012894\\
20.1139036223702	5.63520744399853\\
20.1130463071015	5.63475343413835\\
20.1121889072056	5.63429940054592\\
20.1113314226659	5.63384534321875\\
20.1104738534656	5.63339126215437\\
20.109616199588	5.63293715735029\\
20.1087584610164	5.63248302880403\\
20.1079006377341	5.6320288765131\\
20.1070427297243	5.63157470047503\\
20.1061847369703	5.63112050068734\\
20.1053266594554	5.63066627714753\\
20.1044684971627	5.63021202985312\\
20.1036102500756	5.62975775880162\\
20.1027519181773	5.62930346399057\\
20.101893501451	5.62884914541745\\
20.1010349998799	5.6283948030798\\
20.1001764134473	5.62794043697513\\
20.0993177421365	5.62748604710094\\
20.0984589859305	5.62703163345475\\
20.0976001448127	5.62657719603408\\
20.0967412187662	5.62612273483643\\
20.0958822077742	5.62566824985932\\
20.09502311182	5.62521374110025\\
20.0941639308867	5.62475920855675\\
20.0933046649575	5.62430465222632\\
20.0924453140157	5.62385007210647\\
20.0915858780442	5.62339546819471\\
20.0907263570265	5.62294084048855\\
20.0898667509455	5.6224861889855\\
20.0890070597845	5.62203151368307\\
20.0881472835266	5.62157681457876\\
20.087287422155	5.62112209167008\\
20.0864274756529	5.62066734495455\\
20.0855674440032	5.62021257442967\\
20.0847073271893	5.61975778009293\\
20.0838471251943	5.61930296194186\\
20.0829868380012	5.61884811997396\\
20.0821264655932	5.61839325418673\\
20.0812660079533	5.61793836457768\\
20.0804054650648	5.61748345114431\\
20.0795448369107	5.61702851388413\\
20.0786841234742	5.61657355279463\\
20.0778233247382	5.61611856787334\\
20.076962440686	5.61566355911773\\
20.0761014713005	5.61520852652533\\
20.0752404165649	5.61475347009363\\
20.0743792764623	5.61429838982014\\
20.0735180509757	5.61384328570235\\
20.0726567400882	5.61338815773777\\
20.0717953437828	5.61293300592389\\
20.0709338620426	5.61247783025823\\
20.0700722948507	5.61202263073827\\
20.0692106421901	5.61156740736152\\
20.0683489040439	5.61111216012548\\
20.067487080395	5.61065688902765\\
20.0666251712265	5.61020159406552\\
20.0657631765215	5.60974627523659\\
20.0649010962629	5.60929093253836\\
20.0640389304338	5.60883556596833\\
20.0631766790172	5.60838017552399\\
20.0623143419961	5.60792476120285\\
20.0614519193535	5.60746932300239\\
20.0605894110724	5.60701386092011\\
20.0597268171357	5.60655837495351\\
20.0588641375265	5.60610286510009\\
20.0580013722277	5.60564733135733\\
20.0571385212224	5.60519177372273\\
20.0562755844935	5.60473619219379\\
20.0554125620239	5.60428058676799\\
20.0545494537966	5.60382495744284\\
20.0536862597946	5.60336930421582\\
20.0528229800008	5.60291362708443\\
20.0519596143982	5.60245792604615\\
20.0510961629696	5.60200220109849\\
20.0502326256982	5.60154645223892\\
20.0493690025666	5.60109067946494\\
20.048505293558	5.60063488277405\\
20.0476414986552	5.60017906216372\\
20.0467776178411	5.59972321763145\\
20.0459136510987	5.59926734917474\\
20.0450495984108	5.59881145679106\\
20.0441854597603	5.5983555404779\\
20.0433212351302	5.59789960023276\\
20.0424569245032	5.59744363605312\\
20.0415925278624	5.59698764793647\\
20.0407280451906	5.59653163588029\\
20.0398634764706	5.59607559988207\\
20.0389988216854	5.5956195399393\\
20.0381340808177	5.59516345604946\\
20.0372692538505	5.59470734821004\\
20.0364043407666	5.59425121641851\\
20.0355393415488	5.59379506067238\\
20.03467425618	5.59333888096911\\
20.0338090846431	5.59288267730619\\
20.0329438269208	5.59242644968112\\
20.032078482996	5.59197019809136\\
20.0312130528515	5.5915139225344\\
20.0303475364701	5.59105762300772\\
20.0294819338347	5.59060129950881\\
20.028616244928	5.59014495203514\\
20.0277504697328	5.5896885805842\\
20.026884608232	5.58923218515346\\
20.0260186604084	5.58877576574041\\
20.0251526262446	5.58831932234252\\
20.0242865057236	5.58786285495728\\
20.023420298828	5.58740636358216\\
20.0225540055407	5.58694984821464\\
20.0216876258444	5.58649330885221\\
20.0208211597219	5.58603674549233\\
20.0199546071559	5.58558015813248\\
20.0190879681292	5.58512354677014\\
20.0182212426245	5.58466691140279\\
20.0173544306245	5.5842102520279\\
20.0164875321121	5.58375356864295\\
20.0156205470699	5.58329686124541\\
20.0147534754807	5.58284012983276\\
20.0138863173271	5.58238337440248\\
20.0130190725919	5.58192659495202\\
20.0121517412578	5.58146979147888\\
20.0112843233075	5.58101296398052\\
20.0104168187237	5.58055611245442\\
20.0095492274891	5.58009923689804\\
20.0086815495864	5.57964233730886\\
20.0078137849982	5.57918541368436\\
20.0069459337073	5.57872846602199\\
20.0060779956962	5.57827149431924\\
20.0052099709478	5.57781449857357\\
20.0043418594446	5.57735747878245\\
20.0034736611693	5.57690043494336\\
20.0026053761045	5.57644336705375\\
20.001737004233	5.57598627511111\\
20.0008685455373	5.57552915911289\\
20	5.57507201905658\\
};
\addplot [color=mycolor1, forget plot]
  table[row sep=crcr]{%
20	5.57507201905658\\
19.9991313676039	5.57461485493962\\
19.9982626483314	5.5741576667595\\
19.9973938421654	5.57370045451368\\
19.9965249490882	5.57324321819961\\
19.9956559690827	5.57278595781478\\
19.9947869021314	5.57232867335664\\
19.9939177482168	5.57187136482267\\
19.9930485073216	5.57141403221032\\
19.9921791794283	5.57095667551706\\
19.9913097645196	5.57049929474035\\
19.9904402625781	5.57004188987766\\
19.9895706735862	5.56958446092645\\
19.9887009975266	5.56912700788418\\
19.9878312343818	5.56866953074832\\
19.9869613841344	5.56821202951632\\
19.986091446767	5.56775450418565\\
19.985221422262	5.56729695475377\\
19.9843513106021	5.56683938121814\\
19.9834811117697	5.56638178357623\\
19.9826108257474	5.56592416182547\\
19.9817404525178	5.56546651596336\\
19.9808699920633	5.56500884598733\\
19.9799994443664	5.56455115189485\\
19.9791288094097	5.56409343368337\\
19.9782580871757	5.56363569135036\\
19.9773872776469	5.56317792489327\\
19.9765163808057	5.56272013430955\\
19.9756453966347	5.56226231959668\\
19.9747743251164	5.56180448075209\\
19.9739031662331	5.56134661777325\\
19.9730319199675	5.56088873065761\\
19.9721605863019	5.56043081940263\\
19.9712891652188	5.55997288400577\\
19.9704176567008	5.55951492446447\\
19.9695460607301	5.55905694077619\\
19.9686743772894	5.55859893293838\\
19.967802606361	5.5581409009485\\
19.9669307479273	5.557682844804\\
19.9660588019708	5.55722476450233\\
19.965186768474	5.55676666004095\\
19.9643146474192	5.5563085314173\\
19.9634424387888	5.55585037862884\\
19.9625701425653	5.55539220167301\\
19.9616977587311	5.55493400054727\\
19.9608252872686	5.55447577524907\\
19.9599527281601	5.55401752577585\\
19.9590800813881	5.55355925212507\\
19.9582073469349	5.55310095429417\\
19.9573345247829	5.55264263228061\\
19.9564616149145	5.55218428608182\\
19.9555886173121	5.55172591569526\\
19.954715531958	5.55126752111838\\
19.9538423588345	5.55080910234862\\
19.9529690979241	5.55035065938342\\
19.9520957492091	5.54989219222024\\
19.9512223126718	5.54943370085651\\
19.9503487882945	5.54897518528969\\
19.9494751760596	5.54851664551722\\
19.9486014759494	5.54805808153654\\
19.9477276879463	5.54759949334509\\
19.9468538120324	5.54714088094032\\
19.9459798481903	5.54668224431968\\
19.9451057964021	5.5462235834806\\
19.9442316566501	5.54576489842053\\
19.9433574289167	5.5453061891369\\
19.9424831131841	5.54484745562717\\
19.9416087094346	5.54438869788876\\
19.9407342176505	5.54392991591913\\
19.9398596378141	5.54347110971571\\
19.9389849699076	5.54301227927593\\
19.9381102139133	5.54255342459725\\
19.9372353698135	5.5420945456771\\
19.9363604375903	5.54163564251292\\
19.9354854172261	5.54117671510214\\
19.9346103087031	5.54071776344221\\
19.9337351120034	5.54025878753055\\
19.9328598271094	5.53979978736462\\
19.9319844540033	5.53934076294184\\
19.9311089926673	5.53888171425964\\
19.9302334430836	5.53842264131548\\
19.9293578052343	5.53796354410677\\
19.9284820791018	5.53750442263096\\
19.9276062646682	5.53704527688549\\
19.9267303619157	5.53658610686777\\
19.9258543708264	5.53612691257526\\
19.9249782913826	5.53566769400537\\
19.9241021235665	5.53520845115555\\
19.9232258673601	5.53474918402323\\
19.9223495227458	5.53428989260583\\
19.9214730897055	5.53383057690079\\
19.9205965682215	5.53337123690555\\
19.919719958276	5.53291187261752\\
19.918843259851	5.53245248403415\\
19.9179664729287	5.53199307115286\\
19.9170895974913	5.53153363397109\\
19.9162126335209	5.53107417248625\\
19.9153355809995	5.53061468669578\\
19.9144584399093	5.53015517659712\\
19.9135812102324	5.52969564218767\\
19.912703891951	5.52923608346488\\
19.9118264850471	5.52877650042617\\
19.9109489895027	5.52831689306897\\
19.9100714053001	5.5278572613907\\
19.9091937324213	5.52739760538879\\
19.9083159708483	5.52693792506067\\
19.9074381205632	5.52647822040375\\
19.9065601815482	5.52601849141547\\
19.9056821537852	5.52555873809325\\
19.9048040372563	5.52509896043451\\
19.9039258319436	5.52463915843667\\
19.9030475378291	5.52417933209716\\
19.9021691548948	5.52371948141341\\
19.9012906831228	5.52325960638283\\
19.9004121224952	5.52279970700284\\
19.8995334729938	5.52233978327087\\
19.8986547346008	5.52187983518434\\
19.8977759072982	5.52141986274067\\
19.896896991068	5.52095986593728\\
19.8960179858921	5.52049984477158\\
19.8951388917526	5.520039799241\\
19.8942597086314	5.51957972934296\\
19.8933804365106	5.51911963507487\\
19.8925010753721	5.51865951643416\\
19.8916216251979	5.51819937341823\\
19.89074208597	5.51773920602452\\
19.8898624576702	5.51727901425043\\
19.8889827402807	5.51681879809337\\
19.8881029337833	5.51635855755078\\
19.88722303816	5.51589829262006\\
19.8863430533927	5.51543800329863\\
19.8854629794634	5.5149776895839\\
19.884582816354	5.51451735147328\\
19.8837025640463	5.5140569889642\\
19.8828222225224	5.51359660205406\\
19.8819417917642	5.51313619074029\\
19.8810612717535	5.51267575502028\\
19.8801806624723	5.51221529489145\\
19.8792999639024	5.51175481035122\\
19.8784191760257	5.511294301397\\
19.8775382988242	5.51083376802619\\
19.8766573322797	5.51037321023621\\
19.8757762763741	5.50991262802447\\
19.8748951310892	5.50945202138838\\
19.874013896407	5.50899139032535\\
19.8731325723092	5.50853073483278\\
19.8722511587778	5.50807005490809\\
19.8713696557945	5.50760935054868\\
19.8704880633413	5.50714862175196\\
19.8696063813999	5.50668786851535\\
19.8687246099522	5.50622709083623\\
19.8678427489801	5.50576628871203\\
19.8669607984652	5.50530546214015\\
19.8660787583895	5.50484461111799\\
19.8651966287347	5.50438373564297\\
19.8643144094827	5.50392283571247\\
19.8634321006153	5.50346191132392\\
19.8625497021142	5.50300096247471\\
19.8616672139612	5.50253998916225\\
19.8607846361382	5.50207899138393\\
19.8599019686268	5.50161796913717\\
19.8590192114089	5.50115692241936\\
19.8581363644663	5.50069585122792\\
19.8572534277806	5.50023475556023\\
19.8563704013336	5.4997736354137\\
19.8554872851071	5.49931249078573\\
19.8546040790829	5.49885132167372\\
19.8537207832426	5.49839012807507\\
19.852837397568	5.49792890998719\\
19.8519539220408	5.49746766740746\\
19.8510703566427	5.4970064003333\\
19.8501867013555	5.49654510876208\\
19.8493029561609	5.49608379269123\\
19.8484191210405	5.49562245211813\\
19.8475351959761	5.49516108704017\\
19.8466511809493	5.49469969745476\\
19.8457670759419	5.4942382833593\\
19.8448828809354	5.49377684475117\\
19.8439985959117	5.49331538162777\\
19.8431142208523	5.49285389398651\\
19.842229755739	5.49239238182476\\
19.8413452005533	5.49193084513993\\
19.840460555277	5.49146928392942\\
19.8395758198917	5.4910076981906\\
19.838690994379	5.49054608792089\\
19.8378060787205	5.49008445311766\\
19.8369210728979	5.48962279377832\\
19.8360359768929	5.48916110990025\\
19.835150790687	5.48869940148084\\
19.8342655142618	5.48823766851748\\
19.833380147599	5.48777591100758\\
19.8324946906801	5.48731412894851\\
19.8316091434868	5.48685232233766\\
19.8307235060006	5.48639049117242\\
19.8298377782032	5.4859286354502\\
19.8289519600761	5.48546675516836\\
19.8280660516008	5.4850048503243\\
19.827180052759	5.48454292091541\\
19.8262939635322	5.48408096693907\\
19.825407783902	5.48361898839267\\
19.8245215138499	5.4831569852736\\
19.8236351533575	5.48269495757924\\
19.8227487024062	5.48223290530697\\
19.8218621609777	5.4817708284542\\
19.8209755290535	5.48130872701828\\
19.8200888066151	5.48084660099662\\
19.819201993644	5.4803844503866\\
19.8183150901217	5.47992227518559\\
19.8174280960297	5.47946007539099\\
19.8165410113496	5.47899785100017\\
19.8156538360628	5.47853560201051\\
19.8147665701507	5.4780733284194\\
19.813879213595	5.47761103022423\\
19.8129917663771	5.47714870742236\\
19.8121042284784	5.47668636001118\\
19.8112165998804	5.47622398798807\\
19.8103288805646	5.47576159135041\\
19.8094410705124	5.47529917009558\\
19.8085531697053	5.47483672422095\\
19.8076651781247	5.47437425372392\\
19.806777095752	5.47391175860184\\
19.8058889225688	5.47344923885211\\
19.8050006585563	5.47298669447209\\
19.8041123036961	5.47252412545917\\
19.8032238579696	5.47206153181072\\
19.8023353213581	5.47159891352412\\
19.8014466938431	5.47113627059674\\
19.8005579754059	5.47067360302595\\
19.799669166028	5.47021091080914\\
19.7987802656907	5.46974819394367\\
19.7978912743755	5.46928545242693\\
19.7970021920636	5.46882268625627\\
19.7961130187365	5.46835989542908\\
19.7952237543756	5.46789707994273\\
19.7943343989621	5.46743423979458\\
19.7934449524775	5.46697137498202\\
19.792555414903	5.46650848550241\\
19.7916657862201	5.46604557135313\\
19.79077606641	5.46558263253154\\
19.7898862554542	5.46511966903501\\
19.7889963533338	5.46465668086091\\
19.7881063600303	5.46419366800662\\
19.787216275525	5.4637306304695\\
19.786326099799	5.46326756824691\\
19.7854358328339	5.46280448133623\\
19.7845454746107	5.46234136973483\\
19.783655025111	5.46187823344007\\
19.7827644843158	5.46141507244931\\
19.7818738522065	5.46095188675993\\
19.7809831287644	5.46048867636929\\
19.7800923139707	5.46002544127475\\
19.7792014078068	5.45956218147368\\
19.7783104102537	5.45909889696345\\
19.7774193212929	5.45863558774141\\
19.7765281409055	5.45817225380494\\
19.7756368690728	5.45770889515139\\
19.774745505776	5.45724551177813\\
19.7738540509963	5.45678210368251\\
19.772962504715	5.45631867086191\\
19.7720708669133	5.45585521331368\\
19.7711791375723	5.45539173103519\\
19.7702873166733	5.45492822402379\\
19.7693954041975	5.45446469227684\\
19.7685034001261	5.45400113579171\\
19.7676113044402	5.45353755456576\\
19.766719117121	5.45307394859633\\
19.7658268381498	5.4526103178808\\
19.7649344675076	5.45214666241652\\
19.7640420051756	5.45168298220085\\
19.7631494511351	5.45121927723113\\
19.762256805367	5.45075554750475\\
19.7613640678527	5.45029179301904\\
19.7604712385732	5.44982801377136\\
19.7595783175096	5.44936420975907\\
19.7586853046432	5.44890038097953\\
19.7577921999549	5.44843652743008\\
19.756899003426	5.44797264910809\\
19.7560057150374	5.44750874601091\\
19.7551123347705	5.44704481813588\\
19.7542188626061	5.44658086548037\\
19.7533252985255	5.44611688804173\\
19.7524316425097	5.4456528858173\\
19.7515378945397	5.44518885880445\\
19.7506440545968	5.44472480700051\\
19.7497501226619	5.44426073040285\\
19.748856098716	5.44379662900881\\
19.7479619827403	5.44333250281575\\
19.7470677747159	5.442868351821\\
19.7461734746236	5.44240417602193\\
19.7452790824447	5.44193997541588\\
19.7443845981601	5.4414757500002\\
19.7434900217508	5.44101149977223\\
19.7425953531979	5.44054722472933\\
19.7417005924824	5.44008292486884\\
19.7408057395853	5.4396186001881\\
19.7399107944876	5.43915425068448\\
19.7390157571703	5.4386898763553\\
19.7381206276144	5.43822547719791\\
19.7372254058009	5.43776105320967\\
19.7363300917107	5.43729660438791\\
19.7354346853249	5.43683213072998\\
19.7345391866244	5.43636763223322\\
19.7336435955901	5.43590310889498\\
19.7327479122031	5.43543856071259\\
19.7318521364442	5.43497398768341\\
19.7309562682945	5.43450938980477\\
19.7300603077348	5.43404476707401\\
19.7291642547461	5.43358011948848\\
19.7282681093094	5.43311544704551\\
19.7273718714055	5.43265074974245\\
19.7264755410153	5.43218602757664\\
19.7255791181198	5.43172128054541\\
19.7246826026999	5.43125650864611\\
19.7237859947365	5.43079171187607\\
19.7228892942104	5.43032689023263\\
19.7219925011026	5.42986204371313\\
19.7210956153939	5.4293971723149\\
19.7201986370652	5.42893227603529\\
19.7193015660974	5.42846735487163\\
19.7184044024714	5.42800240882125\\
19.7175071461679	5.42753743788149\\
19.7166097971678	5.42707244204969\\
19.7157123554521	5.42660742132318\\
19.7148148210015	5.42614237569929\\
19.7139171937968	5.42567730517536\\
19.713019473819	5.42521220974873\\
19.7121216610487	5.42474708941671\\
19.7112237554669	5.42428194417666\\
19.7103257570543	5.4238167740259\\
19.7094276657918	5.42335157896176\\
19.70852948166	5.42288635898157\\
19.7076312046399	5.42242111408266\\
19.7067328347123	5.42195584426237\\
19.7058343718578	5.42149054951802\\
19.7049358160572	5.42102522984694\\
19.7040371672914	5.42055988524647\\
19.7031384255411	5.42009451571392\\
19.7022395907871	5.41962912124664\\
19.70134066301	5.41916370184194\\
19.7004416421907	5.41869825749716\\
19.6995425283099	5.41823278820962\\
19.6986433213482	5.41776729397665\\
19.6977440212866	5.41730177479557\\
19.6968446281055	5.41683623066371\\
19.6959451417859	5.4163706615784\\
19.6950455623083	5.41590506753696\\
19.6941458896535	5.41543944853671\\
19.6932461238021	5.41497380457498\\
19.692346264735	5.41450813564909\\
19.6914463124326	5.41404244175636\\
19.6905462668759	5.41357672289413\\
19.6896461280453	5.4131109790597\\
19.6887458959215	5.4126452102504\\
19.6878455704853	5.41217941646356\\
19.6869451517173	5.41171359769649\\
19.6860446395981	5.41124775394652\\
19.6851440341083	5.41078188521096\\
19.6842433352287	5.41031599148713\\
19.6833425429398	5.40985007277236\\
19.6824416572222	5.40938412906396\\
19.6815406780566	5.40891816035925\\
19.6806396054236	5.40845216665555\\
19.6797384393038	5.40798614795017\\
19.6788371796778	5.40752010424044\\
19.6779358265261	5.40705403552367\\
19.6770343798294	5.40658794179717\\
19.6761328395683	5.40612182305826\\
19.6752312057233	5.40565567930426\\
19.674329478275	5.40518951053248\\
19.6734276572039	5.40472331674024\\
19.6725257424907	5.40425709792485\\
19.6716237341158	5.40379085408362\\
19.6707216320598	5.40332458521387\\
19.6698194363033	5.40285829131291\\
19.6689171468268	5.40239197237805\\
19.6680147636108	5.40192562840661\\
19.6671122866358	5.40145925939589\\
19.6662097158823	5.4009928653432\\
19.6653070513309	5.40052644624587\\
19.6644042929621	5.40006000210119\\
19.6635014407563	5.39959353290648\\
19.662598494694	5.39912703865904\\
19.6616954547558	5.39866051935619\\
19.660792320922	5.39819397499523\\
19.6598890931733	5.39772740557347\\
19.65898577149	5.39726081108822\\
19.6580823558526	5.39679419153679\\
19.6571788462415	5.39632754691648\\
19.6562752426372	5.39586087722459\\
19.6553715450202	5.39539418245845\\
19.6544677533709	5.39492746261534\\
19.6535638676697	5.39446071769258\\
19.652659887897	5.39399394768746\\
19.6517558140333	5.3935271525973\\
19.6508516460589	5.3930603324194\\
19.6499473839542	5.39259348715106\\
19.6490430276998	5.39212661678958\\
19.6481385772759	5.39165972133226\\
19.6472340326629	5.39119280077641\\
19.6463293938413	5.39072585511934\\
19.6454246607914	5.39025888435833\\
19.6445198334935	5.38979188849069\\
19.6436149119281	5.38932486751372\\
19.6427098960754	5.38885782142472\\
19.6418047859159	5.38839075022099\\
19.6408995814298	5.38792365389983\\
19.6399942825976	5.38745653245853\\
19.6390888893995	5.3869893858944\\
19.6381834018158	5.38652221420473\\
19.637277819827	5.38605501738682\\
19.6363721434133	5.38558779543797\\
19.6354663725549	5.38512054835547\\
19.6345605072323	5.38465327613662\\
19.6336545474257	5.3841859787787\\
19.6327484931154	5.38371865627903\\
19.6318423442816	5.38325130863489\\
19.6309361009047	5.38278393584358\\
19.6300297629649	5.38231653790239\\
19.6291233304425	5.38184911480861\\
19.6282168033178	5.38138166655953\\
19.6273101815709	5.38091419315246\\
19.6264034651821	5.38044669458467\\
19.6254966541318	5.37997917085347\\
19.6245897484	5.37951162195614\\
19.6236827479671	5.37904404788997\\
19.6227756528132	5.37857644865226\\
19.6218684629186	5.37810882424029\\
19.6209611782635	5.37764117465135\\
19.620053798828	5.37717349988273\\
19.6191463245924	5.37670579993172\\
19.6182387555369	5.37623807479561\\
19.6173310916416	5.37577032447168\\
19.6164233328867	5.37530254895723\\
19.6155154792524	5.37483474824953\\
19.6146075307189	5.37436692234587\\
19.6136994872663	5.37389907124355\\
19.6127913488747	5.37343119493985\\
19.6118831155244	5.37296329343204\\
19.6109747871954	5.37249536671742\\
19.6100663638679	5.37202741479326\\
19.609157845522	5.37155943765686\\
19.6082492321378	5.3710914353055\\
19.6073405236955	5.37062340773645\\
19.6064317201751	5.370155354947\\
19.6055228215568	5.36968727693444\\
19.6046138278206	5.36921917369604\\
19.6037047389467	5.36875104522908\\
19.6027955549151	5.36828289153085\\
19.6018862757059	5.36781471259862\\
19.6009769012991	5.36734650842967\\
19.6000674316749	5.36687827902129\\
19.5991578668133	5.36641002437075\\
19.5982482066943	5.36594174447534\\
19.5973384512981	5.36547343933232\\
19.5964286006045	5.36500510893897\\
19.5955186545937	5.36453675329258\\
19.5946086132457	5.36406837239042\\
19.5936984765405	5.36359996622976\\
19.5927882444581	5.36313153480788\\
19.5918779169786	5.36266307812206\\
19.5909674940819	5.36219459616956\\
19.590056975748	5.36172608894768\\
19.589146361957	5.36125755645367\\
19.5882356526887	5.36078899868481\\
19.5873248479232	5.36032041563838\\
19.5864139476405	5.35985180731165\\
19.5855029518205	5.35938317370188\\
19.5845918604431	5.35891451480636\\
19.5836806734884	5.35844583062235\\
19.5827693909362	5.35797712114713\\
19.5818580127665	5.35750838637795\\
19.5809465389593	5.3570396263121\\
19.5800349694944	5.35657084094684\\
19.5791233043518	5.35610203027945\\
19.5782115435114	5.35563319430718\\
19.5772996869531	5.35516433302731\\
19.5763877346568	5.35469544643711\\
19.5754756866024	5.35422653453385\\
19.5745635427698	5.35375759731478\\
19.5736513031389	5.35328863477718\\
19.5727389676895	5.35281964691831\\
19.5718265364015	5.35235063373544\\
19.5709140092548	5.35188159522583\\
19.5700013862292	5.35141253138675\\
19.5690886673046	5.35094344221546\\
19.5681758524608	5.35047432770923\\
19.5672629416777	5.35000518786531\\
19.566349934935	5.34953602268098\\
19.5654368322127	5.34906683215349\\
19.5645236334905	5.3485976162801\\
19.5636103387483	5.34812837505808\\
19.5626969479658	5.34765910848469\\
19.5617834611229	5.34718981655719\\
19.5608698781993	5.34672049927283\\
19.5599561991748	5.34625115662889\\
19.5590424240292	5.34578178862261\\
19.5581285527424	5.34531239525126\\
19.5572145852939	5.34484297651209\\
19.5563005216637	5.34437353240237\\
19.5553863618315	5.34390406291934\\
19.554472105777	5.34343456806027\\
19.5535577534799	5.34296504782242\\
19.55264330492	5.34249550220304\\
19.551728760077	5.34202593119938\\
19.5508141189307	5.3415563348087\\
19.5498993814607	5.34108671302826\\
19.5489845476468	5.3406170658553\\
19.5480696174687	5.34014739328709\\
19.547154590906	5.33967769532088\\
19.5462394679385	5.33920797195392\\
19.5453242485459	5.33873822318346\\
19.5444089327077	5.33826844900676\\
19.5434935204037	5.33779864942106\\
19.5425780116136	5.33732882442362\\
19.5416624063171	5.33685897401168\\
19.5407467044936	5.33638909818251\\
19.539830906123	5.33591919693335\\
19.5389150111848	5.33544927026144\\
19.5379990196588	5.33497931816404\\
19.5370829315244	5.3345093406384\\
19.5361667467613	5.33403933768176\\
19.5352504653492	5.33356930929137\\
19.5343340872676	5.33309925546448\\
19.5334176124962	5.33262917619833\\
19.5325010410145	5.33215907149018\\
19.5315843728022	5.33168894133727\\
19.5306676078388	5.33121878573683\\
19.5297507461038	5.33074860468613\\
19.5288337875769	5.3302783981824\\
19.5279167322376	5.32980816622288\\
19.5269995800655	5.32933790880483\\
19.5260823310401	5.32886762592548\\
19.5251649851409	5.32839731758207\\
19.5242475423475	5.32792698377185\\
19.5233300026395	5.32745662449207\\
19.5224123659963	5.32698623973995\\
19.5214946323975	5.32651582951275\\
19.5205768018225	5.3260453938077\\
19.5196588742509	5.32557493262204\\
19.5187408496622	5.32510444595302\\
19.5178227280359	5.32463393379786\\
19.5169045093513	5.32416339615382\\
19.5159861935881	5.32369283301813\\
19.5150677807258	5.32322224438802\\
19.5141492707436	5.32275163026074\\
19.5132306636212	5.32228099063352\\
19.5123119593379	5.32181032550359\\
19.5113931578732	5.3213396348682\\
19.5104742592066	5.32086891872458\\
19.5095552633175	5.32039817706996\\
19.5086361701853	5.31992740990158\\
19.5077169797893	5.31945661721668\\
19.5067976921092	5.31898579901247\\
19.5058783071241	5.31851495528621\\
19.5049588248136	5.31804408603513\\
19.504039245157	5.31757319125644\\
19.5031195681338	5.3171022709474\\
19.5021997937232	5.31663132510522\\
19.5012799219047	5.31616035372714\\
19.5003599526577	5.3156893568104\\
19.4994398859614	5.31521833435221\\
19.4985197217952	5.31474728634981\\
19.4975994601386	5.31427621280044\\
19.4966791009708	5.31380511370131\\
19.4957586442712	5.31333398904965\\
19.494838090019	5.31286283884271\\
19.4939174381937	5.31239166307769\\
19.4929966887744	5.31192046175183\\
19.4920758417407	5.31144923486236\\
19.4911548970716	5.3109779824065\\
19.4902338547466	5.31050670438147\\
19.4893127147448	5.31003540078451\\
19.4883914770457	5.30956407161284\\
19.4874701416285	5.30909271686368\\
19.4865487084723	5.30862133653425\\
19.4856271775566	5.30814993062178\\
19.4847055488605	5.30767849912349\\
19.4837838223633	5.30720704203661\\
19.4828619980443	5.30673555935835\\
19.4819400758826	5.30626405108594\\
19.4810180558575	5.3057925172166\\
19.4800959379483	5.30532095774755\\
19.4791737221341	5.30484937267601\\
19.4782514083941	5.30437776199919\\
19.4773289967076	5.30390612571433\\
19.4764064870537	5.30343446381863\\
19.4754838794116	5.30296277630932\\
19.4745611737605	5.30249106318362\\
19.4736383700797	5.30201932443873\\
19.4727154683481	5.30154756007188\\
19.4717924685451	5.30107577008029\\
19.4708693706498	5.30060395446117\\
19.4699461746413	5.30013211321174\\
19.4690228804987	5.29966024632921\\
19.4680994882012	5.29918835381079\\
19.467175997728	5.29871643565371\\
19.466252409058	5.29824449185517\\
19.4653287221706	5.29777252241239\\
19.4644049370447	5.29730052732258\\
19.4634810536594	5.29682850658296\\
19.462557071994	5.29635646019073\\
19.4616329920274	5.29588438814311\\
19.4607088137387	5.29541229043731\\
19.459784537107	5.29494016707053\\
19.4588601621114	5.29446801804\\
19.4579356887309	5.29399584334291\\
19.4570111169446	5.29352364297649\\
19.4560864467316	5.29305141693793\\
19.4551616780708	5.29257916522445\\
19.4542368109413	5.29210688783325\\
19.4533118453222	5.29163458476155\\
19.4523867811925	5.29116225600654\\
19.4514616185311	5.29068990156544\\
19.4505363573171	5.29021752143545\\
19.4496109975295	5.28974511561378\\
19.4486855391473	5.28927268409762\\
19.4477599821495	5.2888002268842\\
19.446834326515	5.28832774397071\\
19.4459085722228	5.28785523535435\\
19.4449827192519	5.28738270103233\\
19.4440567675813	5.28691014100185\\
19.4431307171899	5.28643755526011\\
19.4422045680566	5.28596494380433\\
19.4412783201604	5.28549230663169\\
19.4403519734803	5.28501964373939\\
19.4394255279951	5.28454695512465\\
19.4384989836838	5.28407424078466\\
19.4375723405253	5.28360150071661\\
19.4366455984985	5.28312873491772\\
19.4357187575823	5.28265594338517\\
19.4347918177556	5.28218312611617\\
19.4338647789972	5.28171028310791\\
19.4329376412861	5.28123741435759\\
19.4320104046012	5.28076451986241\\
19.4310830689212	5.28029159961956\\
19.4301556342251	5.27981865362625\\
19.4292281004917	5.27934568187966\\
19.4283004676999	5.278872684377\\
19.4273727358285	5.27839966111545\\
19.4264449048563	5.27792661209221\\
19.4255169747621	5.27745353730447\\
19.4245889455248	5.27698043674943\\
19.4236608171232	5.27650731042429\\
19.4227325895361	5.27603415832622\\
19.4218042627423	5.27556098045243\\
19.4208758367206	5.27508777680011\\
19.4199473114497	5.27461454736645\\
19.4190186869085	5.27414129214863\\
19.4180899630757	5.27366801114385\\
19.41716113993	5.2731947043493\\
19.4162322174504	5.27272137176216\\
19.4153031956154	5.27224801337963\\
19.4143740744038	5.2717746291989\\
19.4134448537944	5.27130121921715\\
19.4125155337659	5.27082778343156\\
19.411586114297	5.27035432183933\\
19.4106565953665	5.26988083443765\\
19.409726976953	5.26940732122369\\
19.4087972590353	5.26893378219465\\
19.407867441592	5.2684602173477\\
19.4069375246019	5.26798662668004\\
19.4060075080436	5.26751301018885\\
19.4050773918957	5.26703936787131\\
19.4041471761371	5.26656569972461\\
19.4032168607463	5.26609200574592\\
19.4022864457019	5.26561828593243\\
19.4013559309827	5.26514454028132\\
19.4004253165673	5.26467076878978\\
19.3994946024343	5.26419697145498\\
19.3985637885623	5.2637231482741\\
19.39763287493	5.26324929924433\\
19.3967018615159	5.26277542436285\\
19.3957707482988	5.26230152362682\\
19.3948395352571	5.26182759703344\\
19.3939082223695	5.26135364457987\\
19.3929768096145	5.26087966626331\\
19.3920452969708	5.26040566208092\\
19.3911136844169	5.25993163202988\\
19.3901819719313	5.25945757610737\\
19.3892501594927	5.25898349431056\\
19.3883182470796	5.25850938663663\\
19.3873862346706	5.25803525308276\\
19.386454122244	5.25756109364611\\
19.3855219097786	5.25708690832386\\
19.3845895972529	5.25661269711319\\
19.3836571846452	5.25613846001127\\
19.3827246719343	5.25566419701527\\
19.3817920590984	5.25518990812237\\
19.3808593461163	5.25471559332973\\
19.3799265329662	5.25424125263453\\
19.3789936196269	5.25376688603393\\
19.3780606060766	5.25329249352511\\
19.3771274922938	5.25281807510524\\
19.3761942782572	5.25234363077148\\
19.375260963945	5.25186916052101\\
19.3743275493357	5.251394664351\\
19.3733940344078	5.2509201422586\\
19.3724604191398	5.250445594241\\
19.37152670351	5.24997102029535\\
19.3705928874968	5.24949642041883\\
19.3696589710787	5.24902179460859\\
19.3687249542341	5.24854714286181\\
19.3677908369414	5.24807246517565\\
19.3668566191789	5.24759776154727\\
19.3659223009251	5.24712303197384\\
19.3649878821583	5.24664827645252\\
19.364053362857	5.24617349498049\\
19.3631187429994	5.24569868755488\\
19.362184022564	5.24522385417288\\
19.361249201529	5.24474899483165\\
19.3603142798728	5.24427410952834\\
19.3593792575738	5.24379919826011\\
19.3584441346103	5.24332426102413\\
19.3575089109607	5.24284929781756\\
19.3565735866031	5.24237430863755\\
19.355638161516	5.24189929348126\\
19.3547026356776	5.24142425234586\\
19.3537670090663	5.2409491852285\\
19.3528312816602	5.24047409212634\\
19.3518954534378	5.23999897303654\\
19.3509595243772	5.23952382795625\\
19.3500234944568	5.23904865688263\\
19.3490873636548	5.23857345981284\\
19.3481511319495	5.23809823674402\\
19.347214799319	5.23762298767335\\
19.3462783657417	5.23714771259796\\
19.3453418311957	5.23667241151502\\
19.3444051956593	5.23619708442167\\
19.3434684591108	5.23572173131508\\
19.3425316215283	5.23524635219239\\
19.3415946828899	5.23477094705075\\
19.340657643174	5.23429551588733\\
19.3397205023588	5.23382005869926\\
19.3387832604223	5.2333445754837\\
19.3378459173427	5.2328690662378\\
19.3369084730983	5.23239353095871\\
19.3359709276672	5.23191796964358\\
19.3350332810276	5.23144238228955\\
19.3340955331575	5.23096676889378\\
19.3331576840352	5.23049112945342\\
19.3322197336387	5.2300154639656\\
19.3312816819462	5.22953977242748\\
19.3303435289358	5.22906405483621\\
19.3294052745857	5.22858831118893\\
19.3284669188739	5.22811254148279\\
19.3275284617785	5.22763674571492\\
19.3265899032776	5.22716092388249\\
19.3256512433493	5.22668507598262\\
19.3247124819716	5.22620920201246\\
19.3237736191227	5.22573330196917\\
19.3228346547807	5.22525737584987\\
19.3218955889234	5.22478142365171\\
19.3209564215291	5.22430544537184\\
19.3200171525757	5.2238294410074\\
19.3190777820413	5.22335341055551\\
19.3181383099038	5.22287735401334\\
19.3171987361414	5.22240127137801\\
19.3162590607321	5.22192516264667\\
19.3153192836537	5.22144902781645\\
19.3143794048844	5.2209728668845\\
19.3134394244021	5.22049667984795\\
19.3124993421849	5.22002046670393\\
19.3115591582106	5.2195442274496\\
19.3106188724572	5.21906796208207\\
19.3096784849028	5.21859167059849\\
19.3087379955252	5.218115352996\\
19.3077974043024	5.21763900927173\\
19.3068567112124	5.21716263942281\\
19.3059159162331	5.21668624344638\\
19.3049750193424	5.21620982133958\\
19.3040340205182	5.21573337309953\\
19.3030929197385	5.21525689872336\\
19.3021517169811	5.21478039820822\\
19.301210412224	5.21430387155123\\
19.300269005445	5.21382731874953\\
19.299327496622	5.21335073980024\\
19.2983858857329	5.2128741347005\\
19.2974441727555	5.21239750344744\\
19.2965023576678	5.21192084603818\\
19.2955604404475	5.21144416246985\\
19.2946184210726	5.21096745273959\\
19.2936762995208	5.21049071684452\\
19.29273407577	5.21001395478176\\
19.2917917497981	5.20953716654846\\
19.2908493215827	5.20906035214173\\
19.2899067911019	5.20858351155869\\
19.2889641583332	5.20810664479649\\
19.2880214232546	5.20762975185223\\
19.2870785858439	5.20715283272304\\
19.2861356460787	5.20667588740606\\
19.285192603937	5.2061989158984\\
19.2842494593964	5.20572191819718\\
19.2833062124347	5.20524489429954\\
19.2823628630297	5.20476784420259\\
19.2814194111591	5.20429076790345\\
19.2804758568007	5.20381366539925\\
19.2795321999321	5.20333653668711\\
19.2785884405312	5.20285938176414\\
19.2776445785755	5.20238220062747\\
19.276700614043	5.20190499327421\\
19.2757565469111	5.20142775970149\\
19.2748123771577	5.20095049990643\\
19.2738681047604	5.20047321388614\\
19.2729237296969	5.19999590163774\\
19.2719792519448	5.19951856315835\\
19.2710346714819	5.19904119844508\\
19.2700899882858	5.19856380749506\\
19.2691452023342	5.19808639030539\\
19.2682003136046	5.19760894687319\\
19.2672553220747	5.19713147719558\\
19.2663102277222	5.19665398126967\\
19.2653650305247	5.19617645909258\\
19.2644197304598	5.19569891066141\\
19.263474327505	5.19522133597329\\
19.2625288216381	5.19474373502532\\
19.2615832128365	5.19426610781461\\
19.260637501078	5.19378845433829\\
19.2596916863399	5.19331077459345\\
19.2587457686	5.19283306857721\\
19.2577997478358	5.19235533628668\\
19.2568536240248	5.19187757771897\\
19.2559073971447	5.19139979287119\\
19.2549610671728	5.19092198174044\\
19.2540146340868	5.19044414432384\\
19.2530680978642	5.1899662806185\\
19.2521214584825	5.18948839062151\\
19.2511747159193	5.18901047432999\\
19.2502278701519	5.18853253174104\\
19.249280921158	5.18805456285177\\
19.248333868915	5.18757656765928\\
19.2473867134004	5.18709854616069\\
19.2464394545917	5.18662049835308\\
19.2454920924663	5.18614242423357\\
19.2445446270017	5.18566432379927\\
19.2435970581754	5.18518619704726\\
19.2426493859647	5.18470804397467\\
19.2417016103472	5.18422986457858\\
19.2407537313003	5.1837516588561\\
19.2398057488013	5.18327342680433\\
19.2388576628277	5.18279516842037\\
19.237909473357	5.18231688370132\\
19.2369611803665	5.18183857264429\\
19.2360127838336	5.18136023524636\\
19.2350642837357	5.18088187150464\\
19.2341156800501	5.18040348141623\\
19.2331669727543	5.17992506497822\\
19.2322181618256	5.17944662218771\\
19.2312692472414	5.17896815304181\\
19.230320228979	5.17848965753759\\
19.2293711070157	5.17801113567216\\
19.228421881329	5.17753258744263\\
19.227472551896	5.17705401284607\\
19.2265231186942	5.17657541187958\\
19.2255735817009	5.17609678454026\\
19.2246239408933	5.17561813082521\\
19.2236741962487	5.17513945073151\\
19.2227243477445	5.17466074425625\\
19.2217743953579	5.17418201139653\\
19.2208243390662	5.17370325214945\\
19.2198741788467	5.17322446651208\\
19.2189239146766	5.17274565448153\\
19.2179735465332	5.17226681605488\\
19.2170230743937	5.17178795122921\\
19.2160724982354	5.17130906000163\\
19.2151218180355	5.17083014236922\\
19.2141710337711	5.17035119832906\\
19.2132201454196	5.16987222787825\\
19.2122691529582	5.16939323101387\\
19.211318056364	5.168914207733\\
19.2103668556142	5.16843515803274\\
19.209415550686	5.16795608191017\\
19.2084641415566	5.16747697936237\\
19.2075126282032	5.16699785038644\\
19.2065610106029	5.16651869497945\\
19.2056092887328	5.16603951313848\\
19.2046574625702	5.16556030486062\\
19.2037055320922	5.16508107014296\\
19.2027534972759	5.16460180898258\\
19.2018013580984	5.16412252137655\\
19.2008491145368	5.16364320732196\\
19.1998967665683	5.16316386681588\\
19.19894431417	5.16268449985541\\
19.1979917573189	5.16220510643762\\
19.1970390959921	5.16172568655959\\
19.1960863301668	5.16124624021839\\
19.1951334598199	5.16076676741111\\
19.1941804849286	5.16028726813483\\
19.1932274054699	5.15980774238661\\
19.1922742214209	5.15932819016355\\
19.1913209327586	5.15884861146271\\
19.19036753946	5.15836900628117\\
19.1894140415021	5.157889374616\\
19.1884604388621	5.15740971646429\\
19.1875067315169	5.15693003182311\\
19.1865529194434	5.15645032068952\\
19.1855990026188	5.15597058306061\\
19.1846449810199	5.15549081893344\\
19.1836908546238	5.1550110283051\\
19.1827366234075	5.15453121117264\\
19.1817822873478	5.15405136753315\\
19.1808278464219	5.15357149738369\\
19.1798733006065	5.15309160072133\\
19.1789186498787	5.15261167754316\\
19.1779638942154	5.15213172784622\\
19.1770090335936	5.1516517516276\\
19.17605406799	5.15117174888436\\
19.1750989973818	5.15069171961357\\
19.1741438217457	5.15021166381231\\
19.1731885410587	5.14973158147762\\
19.1722331552976	5.14925147260659\\
19.1712776644393	5.14877133719627\\
19.1703220684608	5.14829117524375\\
19.1693663673389	5.14781098674607\\
19.1684105610504	5.1473307717003\\
19.1674546495721	5.14685053010351\\
19.1664986328811	5.14637026195277\\
19.165542510954	5.14588996724513\\
19.1645862837676	5.14540964597766\\
19.163629951299	5.14492929814742\\
19.1626735135247	5.14444892375148\\
19.1617169704217	5.14396852278689\\
19.1607603219667	5.14348809525071\\
19.1598035681366	5.14300764114001\\
19.1588467089081	5.14252716045185\\
19.157889744258	5.14204665318328\\
19.156932674163	5.14156611933137\\
19.1559754986	5.14108555889317\\
19.1550182175456	5.14060497186574\\
19.1540608309766	5.14012435824613\\
19.1531033388698	5.13964371803141\\
19.1521457412019	5.13916305121864\\
19.1511880379496	5.13868235780486\\
19.1502302290896	5.13820163778713\\
19.1492723145986	5.13772089116251\\
19.1483142944534	5.13724011792806\\
19.1473561686305	5.13675931808082\\
19.1463979371068	5.13627849161785\\
19.1454395998589	5.13579763853621\\
19.1444811568633	5.13531675883294\\
19.1435226080969	5.1348358525051\\
19.1425639535363	5.13435491954974\\
19.141605193158	5.13387395996391\\
19.1406463269388	5.13339297374466\\
19.1396873548552	5.13291196088905\\
19.1387282768839	5.13243092139411\\
19.1377690930015	5.13194985525691\\
19.1368098031847	5.13146876247448\\
19.1358504074099	5.13098764304389\\
19.1348909056538	5.13050649696216\\
19.133931297893	5.13002532422636\\
19.1329715841041	5.12954412483353\\
19.1320117642635	5.12906289878072\\
19.131051838348	5.12858164606496\\
19.130091806334	5.12810036668332\\
19.129131668198	5.12761906063282\\
19.1281714239167	5.12713772791052\\
19.1272110734664	5.12665636851346\\
19.1262506168239	5.12617498243868\\
19.1252900539655	5.12569356968322\\
19.1243293848678	5.12521213024413\\
19.1233686095073	5.12473066411846\\
19.1224077278604	5.12424917130323\\
19.1214467399037	5.12376765179549\\
19.1204856456135	5.12328610559229\\
19.1195244449665	5.12280453269066\\
19.118563137939	5.12232293308764\\
19.1176017245075	5.12184130678027\\
19.1166402046484	5.12135965376558\\
19.1156785783382	5.12087797404063\\
19.1147168455533	5.12039626760243\\
19.1137550062701	5.11991453444804\\
19.112793060465	5.11943277457448\\
19.1118310081145	5.1189509879788\\
19.1108688491949	5.11846917465802\\
19.1099065836825	5.11798733460919\\
19.1089442115539	5.11750546782933\\
19.1079817327853	5.11702357431549\\
19.1070191473532	5.11654165406469\\
19.1060564552338	5.11605970707397\\
19.1050936564036	5.11557773334035\\
19.1041307508388	5.11509573286088\\
19.1031677385159	5.11461370563259\\
19.102204619411	5.11413165165249\\
19.1012413935006	5.11364957091764\\
19.100278060761	5.11316746342504\\
19.0993146211684	5.11268532917174\\
19.0983510746992	5.11220316815477\\
19.0973874213296	5.11172098037115\\
19.0964236610359	5.1112387658179\\
19.0954597937943	5.11075652449207\\
19.0944958195813	5.11027425639067\\
19.0935317383729	5.10979196151072\\
19.0925675501454	5.10930963984927\\
19.0916032548751	5.10882729140333\\
19.0906388525383	5.10834491616992\\
19.089674343111	5.10786251414608\\
19.0887097265696	5.10738008532882\\
19.0877450028903	5.10689762971518\\
19.0867801720492	5.10641514730216\\
19.0858152340225	5.1059326380868\\
19.0848501887865	5.10545010206612\\
19.0838850363172	5.10496753923714\\
19.082919776591	5.10448494959687\\
19.0819544095838	5.10400233314235\\
19.0809889352719	5.10351968987058\\
19.0800233536314	5.1030370197786\\
19.0790576646385	5.10255432286341\\
19.0780918682692	5.10207159912205\\
19.0771259644997	5.10158884855151\\
19.0761599533061	5.10110607114883\\
19.0751938346646	5.10062326691103\\
19.0742276085511	5.10014043583511\\
19.0732612749418	5.09965757791808\\
19.0722948338128	5.09917469315698\\
19.0713282851401	5.09869178154882\\
19.0703616288999	5.0982088430906\\
19.069394865068	5.09772587777934\\
19.0684279936207	5.09724288561206\\
19.0674610145339	5.09675986658577\\
19.0664939277837	5.09627682069748\\
19.0655267333461	5.0957937479442\\
19.0645594311971	5.09531064832295\\
19.0635920213127	5.09482752183073\\
19.0626245036689	5.09434436846456\\
19.0616568782417	5.09386118822145\\
19.0606891450071	5.0933779810984\\
19.0597213039411	5.09289474709243\\
19.0587533550196	5.09241148620055\\
19.0577852982186	5.09192819841975\\
19.056817133514	5.09144488374706\\
19.0558488608818	5.09096154217948\\
19.054880480298	5.090478173714\\
19.0539119917383	5.08999477834765\\
19.0529433951789	5.08951135607742\\
19.0519746905955	5.08902790690033\\
19.051005877964	5.08854443081336\\
19.0500369572605	5.08806092781354\\
19.0490679284607	5.08757739789786\\
19.0480987915405	5.08709384106333\\
19.0471295464758	5.08661025730694\\
19.0461601932425	5.08612664662571\\
19.0451907318164	5.08564300901663\\
19.0442211621734	5.0851593444767\\
19.0432514842893	5.08467565300292\\
19.0422816981399	5.0841919345923\\
19.041311803701	5.08370818924184\\
19.0403418009485	5.08322441694852\\
19.0393716898582	5.08274061770936\\
19.0384014704059	5.08225679152135\\
19.0374311425673	5.08177293838148\\
19.0364607063182	5.08128905828676\\
19.0354901616345	5.08080515123418\\
19.0345195084918	5.08032121722073\\
19.033548746866	5.07983725624342\\
19.0325778767328	5.07935326829923\\
19.0316068980678	5.07886925338517\\
19.030635810847	5.07838521149823\\
19.0296646150459	5.07790114263539\\
19.0286933106403	5.07741704679365\\
19.0277218976059	5.07693292397002\\
19.0267503759184	5.07644877416146\\
19.0257787455534	5.07596459736499\\
19.0248070064868	5.07548039357759\\
19.0238351586941	5.07499616279625\\
19.0228632021511	5.07451190501796\\
19.0218911368333	5.07402762023971\\
19.0209189627164	5.07354330845849\\
19.0199466797761	5.07305896967128\\
19.018974287988	5.07257460387508\\
19.0180017873277	5.07209021106688\\
19.0170291777709	5.07160579124366\\
19.0160564592932	5.0711213444024\\
19.0150836318701	5.0706368705401\\
19.0141106954772	5.07015236965373\\
19.0131376500902	5.06966784174029\\
19.0121644956846	5.06918328679675\\
19.011191232236	5.06869870482011\\
19.0102178597199	5.06821409580735\\
19.009244378112	5.06772945975544\\
19.0082707873876	5.06724479666137\\
19.0072970875225	5.06676010652213\\
19.006323278492	5.06627538933469\\
19.0053493602718	5.06579064509603\\
19.0043753328373	5.06530587380315\\
19.003401196164	5.064821075453\\
19.0024269502275	5.06433625004259\\
19.0014525950032	5.06385139756888\\
19.0004781304665	5.06336651802885\\
18.9995035565931	5.06288161141948\\
18.9985288733582	5.06239667773774\\
18.9975540807375	5.06191171698063\\
18.9965791787063	5.06142672914511\\
18.99560416724	5.06094171422815\\
18.9946290463141	5.06045667222673\\
18.9936538159041	5.05997160313784\\
18.9926784759852	5.05948650695843\\
18.991703026533	5.0590013836855\\
18.9907274675229	5.058516233316\\
18.9897517989301	5.05803105584691\\
18.9887760207301	5.05754585127521\\
18.9878001328983	5.05706061959786\\
18.9868241354099	5.05657536081184\\
18.9858480282405	5.05609007491413\\
18.9848718113653	5.05560476190167\\
18.9838954847596	5.05511942177146\\
18.9829190483989	5.05463405452046\\
18.9819425022583	5.05414866014564\\
18.9809658463133	5.05366323864395\\
18.9799890805391	5.05317779001239\\
18.979012204911	5.0526923142479\\
18.9780352194043	5.05220681134746\\
18.9770581239943	5.05172128130803\\
18.9760809186562	5.05123572412658\\
18.9751036033654	5.05075013980008\\
18.974126178097	5.05026452832549\\
18.9731486428264	5.04977888969977\\
18.9721709975287	5.04929322391988\\
18.9711932421792	5.0488075309828\\
18.9702153767531	5.04832181088548\\
18.9692374012257	5.04783606362488\\
18.968259315572	5.04735028919798\\
18.9672811197674	5.04686448760172\\
18.966302813787	5.04637865883307\\
18.9653243976059	5.04589280288898\\
18.9643458711994	5.04540691976643\\
18.9633672345426	5.04492100946236\\
18.9623884876107	5.04443507197374\\
18.9614096303788	5.04394910729753\\
18.9604306628221	5.04346311543068\\
18.9594515849156	5.04297709637014\\
18.9584723966345	5.04249105011288\\
18.957493097954	5.04200497665586\\
18.956513688849	5.04151887599602\\
18.9555341692948	5.04103274813032\\
18.9545545392663	5.04054659305572\\
18.9535747987387	5.04006041076917\\
18.9525949476871	5.03957420126763\\
18.9516149860864	5.03908796454804\\
18.9506349139119	5.03860170060737\\
18.9496547311384	5.03811540944255\\
18.948674437741	5.03762909105055\\
18.9476940336948	5.03714274542831\\
18.9467135189748	5.03665637257279\\
18.945732893556	5.03616997248094\\
18.9447521574133	5.03568354514969\\
18.9437713105219	5.03519709057601\\
18.9427903528566	5.03471060875685\\
18.9418092843924	5.03422409968914\\
18.9408281051044	5.03373756336984\\
18.9398468149674	5.03325099979589\\
18.9388654139565	5.03276440896424\\
18.9378839020465	5.03227779087184\\
18.9369022792125	5.03179114551563\\
18.9359205454292	5.03130447289255\\
18.9349387006717	5.03081777299955\\
18.9339567449148	5.03033104583358\\
18.9329746781335	5.02984429139158\\
18.9319925003027	5.02935750967049\\
18.9310102113971	5.02887070066724\\
18.9300278113918	5.0283838643788\\
18.9290453002615	5.02789700080208\\
18.9280626779811	5.02741010993405\\
18.9270799445255	5.02692319177163\\
18.9260970998695	5.02643624631176\\
18.9251141439879	5.02594927355139\\
18.9241310768557	5.02546227348746\\
18.9231478984474	5.02497524611689\\
18.9221646087381	5.02448819143663\\
18.9211812077025	5.02400110944363\\
18.9201976953153	5.0235140001348\\
18.9192140715514	5.02302686350709\\
18.9182303363855	5.02253969955744\\
18.9172464897923	5.02205250828277\\
18.9162625317468	5.02156528968003\\
18.9152784622235	5.02107804374615\\
18.9142942811972	5.02059077047806\\
18.9133099886427	5.0201034698727\\
18.9123255845346	5.01961614192699\\
18.9113410688478	5.01912878663787\\
18.9103564415568	5.01864140400227\\
18.9093717026364	5.01815399401712\\
18.9083868520613	5.01766655667935\\
18.9074018898061	5.01717909198589\\
18.9064168158455	5.01669159993368\\
18.9054316301542	5.01620408051963\\
18.9044463327068	5.01571653374069\\
18.903460923478	5.01522895959376\\
18.9024754024424	5.01474135807579\\
18.9014897695746	5.0142537291837\\
18.9005040248493	5.01376607291442\\
18.899518168241	5.01327838926487\\
18.8985321997243	5.01279067823198\\
18.8975461192739	5.01230293981266\\
18.8965599268643	5.01181517400385\\
18.8955736224701	5.01132738080248\\
18.8945872060659	5.01083956020545\\
18.8936006776262	5.0103517122097\\
18.8926140371256	5.00986383681214\\
18.8916272845385	5.00937593400971\\
18.8906404198396	5.00888800379931\\
18.8896534430034	5.00840004617788\\
18.8886663540043	5.00791206114232\\
18.8876791528169	5.00742404868957\\
18.8866918394156	5.00693600881653\\
18.8857044137751	5.00644794152014\\
18.8847168758696	5.00595984679729\\
18.8837292256737	5.00547172464492\\
18.882741463162	5.00498357505994\\
18.8817535883087	5.00449539803927\\
18.8807656010884	5.00400719357982\\
18.8797775014754	5.0035189616785\\
18.8787892894443	5.00303070233224\\
18.8778009649695	5.00254241553794\\
18.8768125280252	5.00205410129252\\
18.875823978586	5.0015657595929\\
18.8748353166263	5.00107739043598\\
18.8738465421203	5.00058899381868\\
18.8728576550426	5.00010056973791\\
18.8718686553674	4.99961211819059\\
18.8708795430691	4.99912363917361\\
18.8698903181221	4.9986351326839\\
18.8689009805006	4.99814659871836\\
18.8679115301791	4.99765803727389\\
18.8669219671318	4.99716944834742\\
18.8659322913331	4.99668083193585\\
18.8649425027572	4.99619218803608\\
18.8639526013785	4.99570351664502\\
18.8629625871713	4.99521481775958\\
18.8619724601097	4.99472609137667\\
18.8609822201681	4.99423733749318\\
18.8599918673208	4.99374855610603\\
18.859001401542	4.99325974721212\\
18.8580108228059	4.99277091080835\\
18.8570201310867	4.99228204689163\\
18.8560293263588	4.99179315545885\\
18.8550384085963	4.99130423650693\\
18.8540473777734	4.99081529003277\\
18.8530562338642	4.99032631603325\\
18.8520649768431	4.98983731450529\\
18.8510736066842	4.98934828544579\\
18.8500821233616	4.98885922885164\\
18.8490905268495	4.98837014471975\\
18.8480988171221	4.98788103304701\\
18.8471069941535	4.98739189383032\\
18.8461150579179	4.98690272706658\\
18.8451230083893	4.98641353275269\\
18.8441308455419	4.98592431088553\\
18.8431385693498	4.98543506146202\\
18.8421461797871	4.98494578447904\\
18.8411536768279	4.98445647993348\\
18.8401610604463	4.98396714782226\\
18.8391683306163	4.98347778814225\\
18.838175487312	4.98298840089034\\
18.8371825305075	4.98249898606345\\
18.8361894601768	4.98200954365845\\
18.8351962762939	4.98152007367223\\
18.8342029788329	4.9810305761017\\
18.8332095677678	4.98054105094373\\
18.8322160430727	4.98005149819523\\
18.8312224047214	4.97956191785307\\
18.8302286526881	4.97907230991415\\
18.8292347869466	4.97858267437536\\
18.828240807471	4.97809301123359\\
18.8272467142352	4.97760332048571\\
18.8262525072132	4.97711360212863\\
18.825258186379	4.97662385615923\\
18.8242637517064	4.97613408257439\\
18.8232692031695	4.975644281371\\
18.822274540742	4.97515445254594\\
18.821279764398	4.9746645960961\\
18.8202848741114	4.97417471201836\\
18.819289869856	4.9736848003096\\
18.8182947516057	4.97319486096671\\
18.8172995193344	4.97270489398658\\
18.816304173016	4.97221489936607\\
18.8153087126244	4.97172487710208\\
18.8143131381333	4.97123482719149\\
18.8133174495166	4.97074474963116\\
18.8123216467482	4.970254644418\\
18.8113257298018	4.96976451154886\\
18.8103296986513	4.96927435102064\\
18.8093335532706	4.96878416283021\\
18.8083372936333	4.96829394697444\\
18.8073409197133	4.96780370345022\\
18.8063444314844	4.96731343225442\\
18.8053478289202	4.96682313338392\\
18.8043511119947	4.96633280683559\\
18.8033542806815	4.96584245260631\\
18.8023573349544	4.96535207069295\\
18.8013602747871	4.96486166109239\\
18.8003631001533	4.96437122380149\\
18.7993658110267	4.96388075881714\\
18.7983684073811	4.96339026613621\\
18.7973708891902	4.96289974575556\\
18.7963732564276	4.96240919767206\\
18.795375509067	4.9619186218826\\
18.7943776470821	4.96142801838403\\
18.7933796704465	4.96093738717323\\
18.792381579134	4.96044672824707\\
18.791383373118	4.95995604160241\\
18.7903850523724	4.95946532723613\\
18.7893866168706	4.95897458514509\\
18.7883880665863	4.95848381532616\\
18.7873894014932	4.9579930177762\\
18.7863906215648	4.95750219249208\\
18.7853917267746	4.95701133947067\\
18.7843927170964	4.95652045870883\\
18.7833935925035	4.95602955020343\\
18.7823943529697	4.95553861395132\\
18.7813949984685	4.95504764994938\\
18.7803955289733	4.95455665819446\\
18.7793959444578	4.95406563868343\\
18.7783962448954	4.95357459141315\\
18.7773964302597	4.95308351638048\\
18.7763965005242	4.95259241358228\\
18.7753964556623	4.95210128301541\\
18.7743962956477	4.95161012467673\\
18.7733960204536	4.9511189385631\\
18.7723956300536	4.95062772467137\\
18.7713951244212	4.95013648299842\\
18.7703945035298	4.94964521354109\\
18.7693937673529	4.94915391629623\\
18.7683929158639	4.94866259126071\\
18.7673919490361	4.94817123843139\\
18.7663908668431	4.94767985780511\\
18.7653896692582	4.94718844937874\\
18.7643883562549	4.94669701314912\\
18.7633869278064	4.94620554911311\\
18.7623853838862	4.94571405726757\\
18.7613837244677	4.94522253760934\\
18.7603819495242	4.94473099013528\\
18.7593800590291	4.94423941484224\\
18.7583780529556	4.94374781172707\\
18.7573759312772	4.94325618078662\\
18.7563736939671	4.94276452201774\\
18.7553713409987	4.94227283541729\\
18.7543688723453	4.9417811209821\\
18.7533662879801	4.94128937870903\\
18.7523635878765	4.94079760859492\\
18.7513607720077	4.94030581063663\\
18.7503578403469	4.93981398483099\\
18.7493547928675	4.93932213117487\\
18.7483516295428	4.93883024966509\\
18.7473483503458	4.93833834029851\\
18.7463449552499	4.93784640307197\\
18.7453414442283	4.93735443798232\\
18.7443378172541	4.93686244502639\\
18.7433340743007	4.93637042420104\\
18.7423302153411	4.9358783755031\\
18.7413262403486	4.93538629892941\\
18.7403221492963	4.93489419447683\\
18.7393179421574	4.93440206214218\\
18.7383136189051	4.93390990192231\\
18.7373091795125	4.93341771381405\\
18.7363046239527	4.93292549781426\\
18.7352999521988	4.93243325391976\\
18.7342951642241	4.93194098212739\\
18.7332902600015	4.931448682434\\
18.7322852395042	4.93095635483642\\
18.7312801027053	4.93046399933147\\
18.7302748495779	4.92997161591602\\
18.7292694800949	4.92947920458688\\
18.7282639942296	4.92898676534089\\
18.7272583919549	4.92849429817488\\
18.7262526732439	4.9280018030857\\
18.7252468380696	4.92750928007017\\
18.724240886405	4.92701672912512\\
18.7232348182232	4.9265241502474\\
18.7222286334971	4.92603154343382\\
18.7212223321998	4.92553890868122\\
18.7202159143042	4.92504624598644\\
18.7192093797833	4.9245535553463\\
18.7182027286101	4.92406083675763\\
18.7171959607576	4.92356809021726\\
18.7161890761987	4.92307531572201\\
18.7151820749062	4.92258251326873\\
18.7141749568533	4.92208968285422\\
18.7131677220127	4.92159682447533\\
18.7121603703574	4.92110393812888\\
18.7111529018603	4.92061102381168\\
18.7101453164943	4.92011808152058\\
18.7091376142322	4.91962511125239\\
18.708129795047	4.91913211300394\\
18.7071218589115	4.91863908677204\\
18.7061138057985	4.91814603255353\\
18.7051056356809	4.91765295034523\\
18.7040973485316	4.91715984014395\\
18.7030889443234	4.91666670194652\\
18.702080423029	4.91617353574976\\
18.7010717846213	4.91568034155049\\
18.7000630290731	4.91518711934554\\
18.6990541563572	4.91469386913171\\
18.6980451664463	4.91420059090583\\
18.6970360593132	4.91370728466472\\
18.6960268349307	4.91321395040519\\
18.6950174932716	4.91272058812406\\
18.6940080343085	4.91222719781815\\
18.6929984580142	4.91173377948427\\
18.6919887643615	4.91124033311924\\
18.6909789533229	4.91074685871988\\
18.6899690248714	4.910253356283\\
18.6889589789794	4.9097598258054\\
18.6879488156198	4.90926626728391\\
18.6869385347651	4.90877268071535\\
18.6859281363881	4.90827906609651\\
18.6849176204614	4.90778542342421\\
18.6839069869577	4.90729175269527\\
18.6828962358495	4.90679805390649\\
18.6818853671096	4.90630432705468\\
18.6808743807105	4.90581057213666\\
18.6798632766248	4.90531678914922\\
18.6788520548252	4.90482297808919\\
18.6778407152842	4.90432913895337\\
18.6768292579744	4.90383527173856\\
18.6758176828684	4.90334137644158\\
18.6748059899387	4.90284745305923\\
18.6737941791579	4.9023535015883\\
18.6727822504985	4.90185952202562\\
18.6717702039331	4.90136551436799\\
18.6707580394341	4.9008714786122\\
18.6697457569741	4.90037741475506\\
18.6687333565256	4.89988332279338\\
18.6677208380611	4.89938920272395\\
18.666708201553	4.89889505454358\\
18.6656954469739	4.89840087824908\\
18.6646825742962	4.89790667383723\\
18.6636695834923	4.89741244130484\\
18.6626564745347	4.89691818064872\\
18.6616432473958	4.89642389186566\\
18.6606299020481	4.89592957495245\\
18.659616438464	4.8954352299059\\
18.6586028566158	4.8949408567228\\
18.657589156476	4.89444645539996\\
18.6565753380169	4.89395202593416\\
18.655561401211	4.8934575683222\\
18.6545473460306	4.89296308256089\\
18.653533172448	4.892468568647\\
18.6525188804356	4.89197402657735\\
18.6515044699658	4.89147945634871\\
18.6504899410108	4.89098485795789\\
18.6494752935431	4.89049023140167\\
18.6484605275348	4.88999557667686\\
18.6474456429583	4.88950089378023\\
18.6464306397859	4.88900618270859\\
18.6454155179898	4.88851144345871\\
18.6444002775424	4.8880166760274\\
18.6433849184158	4.88752188041144\\
18.6423694405824	4.88702705660761\\
18.6413538440144	4.88653220461272\\
18.640338128684	4.88603732442354\\
18.6393222945635	4.88554241603686\\
18.6383063416249	4.88504747944947\\
18.6372902698406	4.88455251465815\\
18.6362740791828	4.88405752165969\\
18.6352577696236	4.88356250045088\\
18.6342413411351	4.8830674510285\\
18.6332247936896	4.88257237338933\\
18.6322081272592	4.88207726753016\\
18.6311913418161	4.88158213344776\\
18.6301744373324	4.88108697113892\\
18.6291574137801	4.88059178060043\\
18.6281402711315	4.88009656182907\\
18.6271230093586	4.8796013148216\\
18.6261056284335	4.87910603957482\\
18.6250881283283	4.8786107360855\\
18.624070509015	4.87811540435043\\
18.6230527704659	4.87762004436637\\
18.6220349126528	4.87712465613011\\
18.6210169355478	4.87662923963843\\
18.619998839123	4.8761337948881\\
18.6189806233505	4.8756383218759\\
18.6179622882021	4.8751428205986\\
18.6169438336499	4.87464729105298\\
18.615925259666	4.87415173323581\\
18.6149065662223	4.87365614714387\\
18.6138877532907	4.87316053277392\\
18.6128688208433	4.87266489012275\\
18.611849768852	4.87216921918713\\
18.6108305972887	4.87167351996382\\
18.6098113061254	4.8711777924496\\
18.608791895334	4.87068203664124\\
18.6077723648864	4.8701862525355\\
18.6067527147546	4.86969044012916\\
18.6057329449104	4.86919459941899\\
18.6047130553257	4.86869873040176\\
18.6036930459724	4.86820283307422\\
18.6026729168223	4.86770690743316\\
18.6016526678474	4.86721095347533\\
18.6006322990195	4.8667149711975\\
18.5996118103103	4.86621896059645\\
18.5985912016918	4.86572292166892\\
18.5975704731357	4.86522685441169\\
18.5965496246138	4.86473075882153\\
18.595528656098	4.86423463489519\\
18.5945075675601	4.86373848262943\\
18.5934863589718	4.86324230202103\\
18.5924650303048	4.86274609306673\\
18.591443581531	4.86224985576331\\
18.5904220126221	4.86175359010753\\
18.5894003235498	4.86125729609613\\
18.5883785142859	4.86076097372589\\
18.587356584802	4.86026462299356\\
18.5863345350699	4.8597682438959\\
18.5853123650613	4.85927183642966\\
18.5842900747479	4.85877540059161\\
18.5832676641013	4.8582789363785\\
18.5822451330932	4.85778244378709\\
18.5812224816953	4.85728592281414\\
18.5801997098792	4.85678937345639\\
18.5791768176166	4.8562927957106\\
18.5781538048791	4.85579618957353\\
18.5771306716382	4.85529955504193\\
18.5761074178657	4.85480289211255\\
18.5750840435331	4.85430620078215\\
18.5740605486119	4.85380948104747\\
18.5730369330739	4.85331273290527\\
18.5720131968905	4.8528159563523\\
18.5709893400333	4.8523191513853\\
18.5699653624738	4.85182231800104\\
18.5689412641836	4.85132545619625\\
18.5679170451343	4.85082856596768\\
18.5668927052972	4.85033164731209\\
18.565868244644	4.84983470022622\\
18.5648436631462	4.84933772470681\\
18.5638189607751	4.84884072075062\\
18.5627941375024	4.84834368835438\\
18.5617691932994	4.84784662751485\\
18.5607441281377	4.84734953822876\\
18.5597189419886	4.84685242049286\\
18.5586936348236	4.8463552743039\\
18.5576682066141	4.84585809965862\\
18.5566426573316	4.84536089655375\\
18.5556169869475	4.84486366498605\\
18.5545911954331	4.84436640495225\\
18.5535652827598	4.84386911644909\\
18.5525392488991	4.84337179947331\\
18.5515130938222	4.84287445402166\\
18.5504868175006	4.84237708009086\\
18.5494604199055	4.84187967767766\\
18.5484339010084	4.8413822467788\\
18.5474072607805	4.84088478739102\\
18.5463804991931	4.84038729951104\\
18.5453536162176	4.83988978313561\\
18.5443266118253	4.83939223826146\\
18.5432994859874	4.83889466488533\\
18.5422722386752	4.83839706300395\\
18.54124486986	4.83789943261406\\
18.540217379513	4.83740177371238\\
18.5391897676055	4.83690408629565\\
18.5381620341087	4.8364063703606\\
18.5371341789938	4.83590862590397\\
18.5361062022321	4.83541085292249\\
18.5350781037947	4.83491305141288\\
18.5340498836529	4.83441522137187\\
18.5330215417778	4.8339173627962\\
18.5319930781405	4.83341947568259\\
18.5309644927124	4.83292156002778\\
18.5299357854644	4.83242361582848\\
18.5289069563678	4.83192564308143\\
18.5278780053936	4.83142764178335\\
18.5268489325131	4.83092961193097\\
18.5258197376973	4.83043155352101\\
18.5247904209172	4.8299334665502\\
18.5237609821441	4.82943535101527\\
18.522731421349	4.82893720691293\\
18.5217017385029	4.82843903423991\\
18.5206719335769	4.82794083299293\\
18.5196420065421	4.82744260316871\\
18.5186119573694	4.82694434476398\\
18.51758178603	4.82644605777546\\
18.5165514924948	4.82594774219986\\
18.5155210767349	4.82544939803391\\
18.5144905387213	4.82495102527433\\
18.5134598784248	4.82445262391782\\
18.5124290958166	4.82395419396112\\
18.5113981908675	4.82345573540094\\
18.5103671635486	4.82295724823399\\
18.5093360138308	4.822458732457\\
18.5083047416849	4.82196018806667\\
18.507273347082	4.82146161505973\\
18.506241829993	4.82096301343289\\
18.5052101903887	4.82046438318285\\
18.5041784282401	4.81996572430635\\
18.503146543518	4.81946703680008\\
18.5021145361933	4.81896832066076\\
18.5010824062369	4.8184695758851\\
18.5000501536196	4.81797080246982\\
18.4990177783122	4.81747200041162\\
18.4979852802857	4.81697316970721\\
18.4969526595107	4.81647431035331\\
18.4959199159582	4.81597542234662\\
18.494887049599	4.81547650568386\\
18.4938540604037	4.81497756036172\\
18.4928209483433	4.81447858637691\\
18.4917877133884	4.81397958372615\\
18.4907543555098	4.81348055240614\\
18.4897208746783	4.81298149241359\\
18.4886872708646	4.81248240374519\\
18.4876535440395	4.81198328639766\\
18.4866196941736	4.81148414036769\\
18.4855857212376	4.81098496565199\\
18.4845516252024	4.81048576224727\\
18.4835174060384	4.80998653015023\\
18.4824830637165	4.80948726935756\\
18.4814485982073	4.80898797986597\\
18.4804140094814	4.80848866167215\\
18.4793792975094	4.80798931477282\\
18.4783444622621	4.80748993916467\\
18.47730950371	4.80699053484439\\
18.4762744218238	4.80649110180868\\
18.475239216574	4.80599164005425\\
18.4742038879313	4.80549214957778\\
18.4731684358661	4.80499263037599\\
18.4721328603492	4.80449308244555\\
18.471097161351	4.80399350578317\\
18.4700613388421	4.80349390038554\\
18.469025392793	4.80299426624936\\
18.4679893231743	4.80249460337132\\
18.4669531299565	4.80199491174811\\
18.4659168131101	4.80149519137642\\
18.4648803726055	4.80099544225295\\
18.4638438084134	4.80049566437439\\
18.462807120504	4.79999585773743\\
18.461770308848	4.79949602233875\\
18.4607333734157	4.79899615817506\\
18.4596963141777	4.79849626524303\\
18.4586591311042	4.79799634353936\\
18.4576218241659	4.79749639306073\\
18.456584393333	4.79699641380383\\
18.455546838576	4.79649640576535\\
18.4545091598652	4.79599636894197\\
18.4534713571711	4.79549630333038\\
18.452433430464	4.79499620892727\\
18.4513953797142	4.79449608572932\\
18.4503572048922	4.79399593373321\\
18.4493189059683	4.79349575293562\\
18.4482804829127	4.79299554333325\\
18.4472419356958	4.79249530492276\\
18.4462032642879	4.79199503770085\\
18.4451644686593	4.79149474166419\\
18.4441255487803	4.79099441680946\\
18.4430865046212	4.79049406313335\\
18.4420473361521	4.78999368063253\\
18.4410080433435	4.78949326930368\\
18.4399686261654	4.78899282914348\\
18.4389290845882	4.78849236014862\\
18.437889418582	4.78799186231575\\
18.4368496281172	4.78749133564157\\
18.4358097131637	4.78699078012274\\
18.434769673692	4.78649019575594\\
18.433729509672	4.78598958253785\\
18.4326892210741	4.78548894046514\\
18.4316488078683	4.78498826953449\\
18.4306082700248	4.78448756974256\\
18.4295676075138	4.78398684108603\\
18.4285268203053	4.78348608356157\\
18.4274859083695	4.78298529716586\\
18.4264448716764	4.78248448189556\\
18.4254037101963	4.78198363774734\\
18.424362423899	4.78148276471787\\
18.4233210127548	4.78098186280383\\
18.4222794767336	4.78048093200187\\
18.4212378158056	4.77997997230868\\
18.4201960299407	4.77947898372091\\
18.419154119109	4.77897796623523\\
18.4181120832805	4.77847691984831\\
18.4170699224252	4.77797584455681\\
18.4160276365131	4.7774747403574\\
18.4149852255142	4.77697360724674\\
18.4139426893984	4.7764724452215\\
18.4129000281358	4.77597125427834\\
18.4118572416962	4.77547003441392\\
18.4108143300496	4.7749687856249\\
18.409771293166	4.77446750790795\\
18.4087281310152	4.77396620125973\\
18.4076848435672	4.77346486567689\\
18.4066414307918	4.77296350115609\\
18.405597892659	4.77246210769401\\
18.4045542291386	4.77196068528728\\
18.4035104402005	4.77145923393258\\
18.4024665258145	4.77095775362655\\
18.4014224859505	4.77045624436587\\
18.4003783205783	4.76995470614717\\
18.3993340296677	4.76945313896713\\
18.3982896131886	4.76895154282238\\
18.3972450711108	4.76844991770959\\
18.3962004034039	4.76794826362542\\
18.3951556100379	4.7674465805665\\
18.3941106909824	4.76694486852951\\
18.3930656462073	4.76644312751108\\
18.3920204756822	4.76594135750788\\
18.390975179377	4.76543955851654\\
18.3899297572612	4.76493773053373\\
18.3888842093047	4.76443587355609\\
18.3878385354772	4.76393398758028\\
18.3867927357483	4.76343207260293\\
18.3857468100877	4.7629301286207\\
18.384700758465	4.76242815563024\\
18.38365458085	4.76192615362819\\
18.3826082772123	4.76142412261119\\
18.3815618475215	4.76092206257591\\
18.3805152917472	4.76041997351897\\
18.3794686098591	4.75991785543703\\
18.3784218018267	4.75941570832672\\
18.3773748676197	4.7589135321847\\
18.3763278072076	4.7584113270076\\
18.37528062056	4.75790909279208\\
18.3742333076464	4.75740682953475\\
18.3731858684365	4.75690453723228\\
18.3721383028996	4.7564022158813\\
18.3710906110055	4.75589986547845\\
18.3700427927235	4.75539748602037\\
18.3689948480232	4.7548950775037\\
18.367946776874	4.75439263992507\\
18.3668985792455	4.75389017328113\\
18.3658502551071	4.75338767756851\\
18.3648018044283	4.75288515278385\\
18.3637532271785	4.75238259892378\\
18.3627045233272	4.75188001598494\\
18.3616556928437	4.75137740396397\\
18.3606067356976	4.75087476285749\\
18.3595576518582	4.75037209266214\\
18.3585084412948	4.74986939337456\\
18.357459103977	4.74936666499138\\
18.3564096398739	4.74886390750922\\
18.3553600489551	4.74836112092472\\
18.3543103311899	4.74785830523451\\
18.3532604865475	4.74735546043522\\
18.3522105149973	4.74685258652349\\
18.3511604165087	4.74634968349593\\
18.3501101910509	4.74584675134917\\
18.3490598385933	4.74534379007986\\
18.348009359105	4.7448407996846\\
18.3469587525554	4.74433778016003\\
18.3459080189138	4.74383473150278\\
18.3448571581493	4.74333165370946\\
18.3438061702313	4.74282854677671\\
18.3427550551289	4.74232541070114\\
18.3417038128113	4.74182224547939\\
18.3406524432479	4.74131905110808\\
18.3396009464076	4.74081582758382\\
18.3385493222599	4.74031257490323\\
18.3374975707737	4.73980929306295\\
18.3364456919182	4.73930598205959\\
18.3353936856627	4.73880264188977\\
18.3343415519763	4.73829927255011\\
18.3332892908279	4.73779587403723\\
18.3322369021869	4.73729244634775\\
18.3311843860223	4.73678898947828\\
18.3301317423031	4.73628550342544\\
18.3290789709985	4.73578198818586\\
18.3280260720775	4.73527844375614\\
18.3269730455092	4.73477487013289\\
18.3259198912626	4.73427126731274\\
18.3248666093068	4.73376763529231\\
18.3238131996108	4.73326397406819\\
18.3227596621435	4.73276028363701\\
18.3217059968741	4.73225656399538\\
18.3206522037714	4.7317528151399\\
18.3195982828045	4.7312490370672\\
18.3185442339423	4.73074522977388\\
18.3174900571538	4.73024139325655\\
18.316435752408	4.72973752751182\\
18.3153813196737	4.7292336325363\\
18.3143267589199	4.7287297083266\\
18.3132720701155	4.72822575487933\\
18.3122172532293	4.72772177219109\\
18.3111623082304	4.72721776025849\\
18.3101072350875	4.72671371907813\\
18.3090520337696	4.72620964864662\\
18.3079967042454	4.72570554896057\\
18.3069412464838	4.72520142001658\\
18.3058856604537	4.72469726181125\\
18.3048299461239	4.72419307434119\\
18.3037741034631	4.72368885760299\\
18.3027181324402	4.72318461159327\\
18.301662033024	4.72268033630862\\
18.3006058051832	4.72217603174563\\
18.2995494488866	4.72167169790093\\
18.298492964103	4.72116733477109\\
18.297436350801	4.72066294235272\\
18.2963796089495	4.72015852064242\\
18.2953227385171	4.71965406963679\\
18.2942657394725	4.71914958933241\\
18.2932086117844	4.7186450797259\\
18.2921513554216	4.71814054081385\\
18.2910939703527	4.71763597259284\\
18.2900364565463	4.71713137505948\\
18.2889788139711	4.71662674821036\\
18.2879210425957	4.71612209204207\\
18.2868631423888	4.71561740655121\\
18.285805113319	4.71511269173436\\
18.2847469553548	4.71460794758812\\
18.2836886684649	4.71410317410909\\
18.2826302526179	4.71359837129384\\
18.2815717077822	4.71309353913898\\
18.2805130339266	4.71258867764108\\
18.2794542310194	4.71208378679675\\
18.2783952990293	4.71157886660256\\
18.2773362379248	4.71107391705511\\
18.2762770476743	4.71056893815098\\
18.2752177282465	4.71006392988675\\
18.2741582796097	4.70955889225902\\
18.2730987017324	4.70905382526437\\
18.2720389945831	4.70854872889938\\
18.2709791581304	4.70804360316065\\
18.2699191923425	4.70753844804473\\
18.2688590971879	4.70703326354824\\
18.2677988726351	4.70652804966774\\
18.2667385186525	4.70602280639981\\
18.2656780352084	4.70551753374105\\
18.2646174222712	4.70501223168802\\
18.2635566798094	4.70450690023731\\
18.2624958077912	4.7040015393855\\
18.261434806185	4.70349614912916\\
18.2603736749592	4.70299072946488\\
18.2593124140821	4.70248528038923\\
18.2582510235219	4.70197980189879\\
18.2571895032471	4.70147429399013\\
18.2561278532258	4.70096875665983\\
18.2550660734265	4.70046318990447\\
18.2540041638172	4.69995759372061\\
18.2529421243663	4.69945196810485\\
18.2518799550421	4.69894631305373\\
18.2508176558127	4.69844062856385\\
18.2497552266465	4.69793491463176\\
18.2486926675115	4.69742917125405\\
18.247629978376	4.69692339842728\\
18.2465671592082	4.69641759614802\\
18.2455042099762	4.69591176441285\\
18.2444411306483	4.69540590321833\\
18.2433779211925	4.69490001256103\\
18.2423145815771	4.69439409243752\\
18.24125111177	4.69388814284436\\
18.2401875117396	4.69338216377813\\
18.2391237814537	4.69287615523538\\
18.2380599208807	4.69237011721268\\
18.2369959299884	4.69186404970661\\
18.235931808745	4.69135795271372\\
18.2348675571186	4.69085182623057\\
18.2338031750772	4.69034567025373\\
18.2327386625888	4.68983948477977\\
18.2316740196215	4.68933326980523\\
18.2306092461432	4.6888270253267\\
18.229544342122	4.68832075134071\\
18.2284793075259	4.68781444784384\\
18.2274141423228	4.68730811483265\\
18.2263488464806	4.68680175230369\\
18.2252834199674	4.68629536025352\\
18.2242178627511	4.6857889386787\\
18.2231521747995	4.68528248757579\\
18.2220863560807	4.68477600694133\\
18.2210204065625	4.6842694967719\\
18.2199543262129	4.68376295706404\\
18.2188881149996	4.68325638781431\\
18.2178217728906	4.68274978901926\\
18.2167552998537	4.68224316067545\\
18.2156886958567	4.68173650277942\\
18.2146219608676	4.68122981532773\\
18.2135550948541	4.68072309831694\\
18.212488097784	4.68021635174359\\
18.2114209696252	4.67970957560423\\
18.2103537103453	4.67920276989541\\
18.2092863199123	4.67869593461369\\
18.2082187982938	4.67818906975561\\
18.2071511454576	4.67768217531771\\
18.2060833613714	4.67717525129656\\
18.205015446003	4.67666829768868\\
18.20394739932	4.67616131449063\\
18.2028792212903	4.67565430169896\\
18.2018109118814	4.67514725931021\\
18.200742471061	4.67464018732092\\
18.1996738987969	4.67413308572764\\
18.1986051950566	4.6736259545269\\
18.1975363598079	4.67311879371527\\
18.1964673930182	4.67261160328926\\
18.1953982946554	4.67210438324543\\
18.1943290646869	4.67159713358032\\
18.1932597030804	4.67108985429047\\
18.1921902098034	4.67058254537241\\
18.1911205848235	4.67007520682269\\
18.1900508281083	4.66956783863784\\
18.1889809396253	4.6690604408144\\
18.1879109193421	4.66855301334891\\
18.1868407672262	4.6680455562379\\
18.185770483245	4.66753806947792\\
18.1847000673661	4.66703055306549\\
18.1836295195571	4.66652300699715\\
18.1825588397852	4.66601543126943\\
18.1814880280181	4.66550782587887\\
18.1804170842231	4.66500019082201\\
18.1793460083678	4.66449252609536\\
18.1782748004194	4.66398483169547\\
18.1772034603455	4.66347710761886\\
18.1761319881135	4.66296935386207\\
18.1750603836907	4.66246157042163\\
18.1739886470445	4.66195375729406\\
18.1729167781423	4.66144591447589\\
18.1718447769514	4.66093804196365\\
18.1707726434392	4.66043013975387\\
18.169700377573	4.65992220784308\\
18.1686279793201	4.65941424622779\\
18.1675554486479	4.65890625490455\\
18.1664827855236	4.65839823386986\\
18.1654099899145	4.65789018312026\\
18.1643370617879	4.65738210265227\\
18.163264001111	4.65687399246241\\
18.162190807851	4.65636585254721\\
18.1611174819753	4.65585768290318\\
18.1600440234511	4.65534948352685\\
18.1589704322454	4.65484125441474\\
18.1578967083257	4.65433299556336\\
18.1568228516589	4.65382470696925\\
18.1557488622123	4.65331638862891\\
18.1546747399531	4.65280804053886\\
18.1536004848484	4.65229966269563\\
18.1525260968653	4.65179125509573\\
18.151451575971	4.65128281773567\\
18.1503769221326	4.65077435061198\\
18.1493021353171	4.65026585372115\\
18.1482272154917	4.64975732705972\\
18.1471521626235	4.6492487706242\\
18.1460769766794	4.64874018441109\\
18.1450016576266	4.64823156841692\\
18.143926205432	4.64772292263818\\
18.1428506200628	4.6472142470714\\
18.1417749014859	4.64670554171309\\
18.1406990496683	4.64619680655975\\
18.1396230645771	4.64568804160789\\
18.1385469461791	4.64517924685403\\
18.1374706944414	4.64467042229467\\
18.1363943093309	4.64416156792632\\
18.1353177908146	4.64365268374549\\
18.1342411388593	4.64314376974868\\
18.133164353432	4.6426348259324\\
18.1320874344996	4.64212585229315\\
18.131010382029	4.64161684882745\\
18.1299331959871	4.64110781553178\\
18.1288558763406	4.64059875240267\\
18.1277784230566	4.6400896594366\\
18.1267008361018	4.63958053663008\\
18.1256231154431	4.63907138397962\\
18.1245452610472	4.63856220148172\\
18.123467272881	4.63805298913286\\
18.1223891509112	4.63754374692956\\
18.1213108951047	4.63703447486832\\
18.1202325054283	4.63652517294563\\
18.1191539818485	4.63601584115798\\
18.1180753243323	4.63550647950189\\
18.1169965328464	4.63499708797384\\
18.1159176073574	4.63448766657033\\
18.114838547832	4.63397821528786\\
18.113759354237	4.63346873412291\\
18.1126800265391	4.63295922307199\\
18.1116005647049	4.63244968213159\\
18.110520968701	4.6319401112982\\
18.1094412384941	4.63143051056831\\
18.1083613740508	4.63092087993842\\
18.1072813753378	4.63041121940501\\
18.1062012423217	4.62990152896459\\
18.105120974969	4.62939180861363\\
18.1040405732464	4.62888205834862\\
18.1029600371203	4.62837227816606\\
18.1018793665574	4.62786246806243\\
18.1007985615242	4.62735262803422\\
18.0997176219873	4.62684275807792\\
18.0986365479131	4.62633285819002\\
18.0975553392681	4.62582292836699\\
18.0964739960189	4.62531296860532\\
18.0953925181319	4.62480297890151\\
18.0943109055736	4.62429295925202\\
18.0932291583105	4.62378290965335\\
18.092147276309	4.62327283010198\\
18.0910652595355	4.62276272059438\\
18.0899831079564	4.62225258112705\\
18.0889008215382	4.62174241169645\\
18.0878184002472	4.62123221229908\\
18.0867358440499	4.62072198293141\\
18.0856531529126	4.62021172358992\\
18.0845703268016	4.61970143427108\\
18.0834873656833	4.61919111497138\\
18.082404269524	4.61868076568729\\
18.0813210382901	4.61817038641529\\
18.0802376719478	4.61765997715185\\
18.0791541704634	4.61714953789345\\
18.0780705338033	4.61663906863656\\
18.0769867619336	4.61612856937766\\
18.0759028548207	4.61561804011322\\
18.0748188124308	4.61510748083971\\
18.0737346347301	4.61459689155361\\
18.0726503216848	4.61408627225138\\
18.0715658732611	4.6135756229295\\
18.0704812894252	4.61306494358443\\
18.0693965701434	4.61255423421265\\
18.0683117153817	4.61204349481062\\
18.0672267251064	4.61153272537482\\
18.0661415992835	4.6110219259017\\
18.0650563378792	4.61051109638775\\
18.0639709408596	4.61000023682941\\
18.0628854081908	4.60948934722317\\
18.0617997398389	4.60897842756548\\
18.0607139357699	4.60846747785281\\
18.0596279959501	4.60795649808161\\
18.0585419203453	4.60744548824837\\
18.0574557089216	4.60693444834954\\
18.0563693616451	4.60642337838157\\
18.0552828784817	4.60591227834093\\
18.0541962593975	4.60540114822409\\
18.0531095043585	4.6048899880275\\
18.0520226133306	4.60437879774762\\
18.0509355862798	4.60386757738091\\
18.0498484231721	4.60335632692383\\
18.0487611239733	4.60284504637284\\
18.0476736886494	4.60233373572438\\
18.0465861171664	4.60182239497493\\
18.04549840949	4.60131102412093\\
18.0444105655863	4.60079962315885\\
18.043322585421	4.60028819208512\\
18.0422344689601	4.59977673089622\\
18.0411462161693	4.59926523958859\\
18.0400578270146	4.59875371815868\\
18.0389693014617	4.59824216660294\\
18.0378806394764	4.59773058491784\\
18.0367918410246	4.59721897309981\\
18.035702906072	4.59670733114531\\
18.0346138345844	4.59619565905078\\
18.0335246265275	4.59568395681269\\
18.0324352818671	4.59517222442746\\
18.031345800569	4.59466046189156\\
18.0302561825988	4.59414866920143\\
18.0291664279222	4.59363684635352\\
18.028076536505	4.59312499334426\\
18.0269865083128	4.59261311017011\\
18.0258963433113	4.59210119682751\\
18.0248060414662	4.5915892533129\\
18.023715602743	4.59107727962273\\
18.0226250271074	4.59056527575344\\
18.0215343145251	4.59005324170147\\
18.0204434649616	4.58954117746326\\
18.0193524783825	4.58902908303526\\
18.0182613547534	4.5885169584139\\
18.0171700940398	4.58800480359562\\
18.0160786962074	4.58749261857686\\
18.0149871612216	4.58698040335407\\
18.013895489048	4.58646815792367\\
18.012803679652	4.5859558822821\\
18.0117117329992	4.5854435764258\\
18.0106196490551	4.58493124035121\\
18.0095274277851	4.58441887405476\\
18.0084350691548	4.58390647753289\\
18.0073425731294	4.58339405078202\\
18.0062499396745	4.58288159379859\\
18.0051571687556	4.58236910657904\\
18.0040642603379	4.58185658911979\\
18.0029712143869	4.58134404141728\\
18.001878030868	4.58083146346794\\
18.0007847097465	4.5803188552682\\
17.9996912509878	4.57980621681448\\
17.9985976545573	4.57929354810321\\
17.9975039204202	4.57878084913083\\
17.9964100485419	4.57826811989376\\
17.9953160388877	4.57775536038843\\
17.9942218914229	4.57724257061126\\
17.9931276061127	4.57672975055868\\
17.9920331829224	4.57621690022711\\
17.9909386218173	4.57570401961298\\
17.9898439227626	4.57519110871271\\
17.9887490857236	4.57467816752273\\
17.9876541106653	4.57416519603945\\
17.9865589975531	4.5736521942593\\
17.9854637463521	4.5731391621787\\
17.9843683570275	4.57262609979407\\
17.9832728295444	4.57211300710183\\
17.982177163868	4.5715998840984\\
17.9810813599634	4.5710867307802\\
17.9799854177958	4.57057354714364\\
17.9788893373302	4.57006033318514\\
17.9777931185317	4.56954708890112\\
17.9766967613655	4.569033814288\\
17.9756002657965	4.56852050934218\\
17.9745036317899	4.56800717406009\\
17.9734068593106	4.56749380843814\\
17.9723099483237	4.56698041247274\\
17.9712128987942	4.56646698616031\\
17.9701157106872	4.56595352949725\\
17.9690183839675	4.56544004247999\\
17.9679209186002	4.56492652510492\\
17.9668233145502	4.56441297736846\\
17.9657255717825	4.56389939926702\\
17.9646276902619	4.56338579079701\\
17.9635296699535	4.56287215195483\\
17.9624315108222	4.56235848273691\\
17.9613332128327	4.56184478313962\\
17.96023477595	4.5613310531594\\
17.959136200139	4.56081729279264\\
17.9580374853646	4.56030350203575\\
17.9569386315914	4.55978968088514\\
17.9558396387845	4.5592758293372\\
17.9547405069085	4.55876194738834\\
17.9536412359284	4.55824803503496\\
17.9525418258088	4.55773409227347\\
17.9514422765146	4.55722011910026\\
17.9503425880105	4.55670611551174\\
17.9492427602613	4.55619208150431\\
17.9481427932316	4.55567801707437\\
17.9470426868863	4.5551639222183\\
17.9459424411899	4.55464979693252\\
17.9448420561073	4.55413564121342\\
17.9437415316031	4.5536214550574\\
17.9426408676419	4.55310723846085\\
17.9415400641884	4.55259299142017\\
17.9404391212073	4.55207871393176\\
17.9393380386631	4.551564405992\\
17.9382368165205	4.55105006759729\\
17.9371354547441	4.55053569874403\\
17.9360339532984	4.55002129942861\\
17.9349323121481	4.54950686964741\\
17.9338305312577	4.54899240939683\\
17.9327286105917	4.54847791867327\\
17.9316265501147	4.5479633974731\\
17.9305243497911	4.54744884579272\\
17.9294220095855	4.54693426362853\\
17.9283195294625	4.54641965097689\\
17.9272169093863	4.54590500783421\\
17.9261141493216	4.54539033419687\\
17.9250112492328	4.54487563006125\\
17.9239082090842	4.54436089542374\\
17.9228050288404	4.54384613028073\\
17.9217017084657	4.5433313346286\\
17.9205982479245	4.54281650846373\\
17.9194946471813	4.5423016517825\\
17.9183909062003	4.54178676458131\\
17.917287024946	4.54127184685652\\
17.9161830033826	4.54075689860452\\
17.9150788414746	4.54024191982169\\
17.9139745391861	4.5397269105044\\
17.9128700964816	4.53921187064905\\
17.9117655133253	4.538696800252\\
17.9106607896814	4.53818169930964\\
17.9095559255143	4.53766656781834\\
17.9084509207882	4.53715140577447\\
17.9073457754672	4.53663621317441\\
17.9062404895157	4.53612099001455\\
17.9051350628979	4.53560573629124\\
17.9040294955779	4.53509045200088\\
17.9029237875198	4.53457513713981\\
17.901817938688	4.53405979170443\\
17.9007119490465	4.53354441569111\\
17.8996058185594	4.5330290090962\\
17.8984995471909	4.53251357191609\\
17.8973931349051	4.53199810414715\\
17.896286581666	4.53148260578574\\
17.8951798874378	4.53096707682823\\
17.8940730521846	4.53045151727098\\
17.8929660758703	4.52993592711038\\
17.8918589584591	4.52942030634278\\
17.8907516999149	4.52890465496455\\
17.8896443002018	4.52838897297205\\
17.8885367592838	4.52787326036165\\
17.8874290771248	4.52735751712972\\
17.8863212536888	4.52684174327261\\
17.8852132889398	4.5263259387867\\
17.8841051828417	4.52581010366833\\
17.8829969353584	4.52529423791387\\
17.881888546454	4.52477834151969\\
17.8807800160922	4.52426241448215\\
17.8796713442369	4.52374645679759\\
17.8785625308521	4.52323046846239\\
17.8774535759016	4.5227144494729\\
17.8763444793493	4.52219839982548\\
17.8752352411589	4.52168231951648\\
17.8741258612943	4.52116620854226\\
17.8730163397193	4.52065006689917\\
17.8719066763977	4.52013389458358\\
17.8707968712932	4.51961769159184\\
17.8696869243697	4.51910145792029\\
17.8685768355909	4.51858519356529\\
17.8674666049205	4.5180688985232\\
17.8663562323222	4.51755257279037\\
17.8652457177598	4.51703621636314\\
17.864135061197	4.51651982923787\\
17.8630242625973	4.5160034114109\\
17.8619133219246	4.51548696287859\\
17.8608022391424	4.51497048363729\\
17.8596910142143	4.51445397368334\\
17.8585796471041	4.51393743301309\\
17.8574681377753	4.51342086162288\\
17.8563564861915	4.51290425950907\\
17.8552446923163	4.51238762666799\\
17.8541327561133	4.51187096309599\\
17.853020677546	4.51135426878942\\
17.851908456578	4.51083754374462\\
17.8507960931727	4.51032078795792\\
17.8496835872938	4.50980400142568\\
17.8485709389046	4.50928718414424\\
17.8474581479687	4.50877033610992\\
17.8463452144496	4.50825345731909\\
17.8452321383107	4.50773654776806\\
17.8441189195155	4.50721960745319\\
17.8430055580273	4.50670263637081\\
17.8418920538096	4.50618563451726\\
17.8407784068258	4.50566860188887\\
17.8396646170393	4.50515153848198\\
17.8385506844135	4.50463444429292\\
17.8374366089117	4.50411731931804\\
17.8363223904972	4.50360016355366\\
17.8352080291334	4.50308297699612\\
17.8340935247837	4.50256575964175\\
17.8329788774112	4.50204851148688\\
17.8318640869793	4.50153123252785\\
17.8307491534513	4.50101392276097\\
17.8296340767904	4.5004965821826\\
17.8285188569599	4.49997921078905\\
17.827403493923	4.49946180857665\\
17.8262879876429	4.49894437554173\\
17.8251723380828	4.49842691168063\\
17.8240565452059	4.49790941698965\\
17.8229406089754	4.49739189146514\\
17.8218245293545	4.49687433510342\\
17.8207083063062	4.4963567479008\\
17.8195919397938	4.49583912985363\\
17.8184754297803	4.49532148095821\\
17.8173587762288	4.49480380121088\\
17.8162419791025	4.49428609060795\\
17.8151250383644	4.49376834914575\\
17.8140079539775	4.49325057682059\\
17.812890725905	4.49273277362881\\
17.8117733541097	4.4922149395667\\
17.8106558385549	4.49169707463061\\
17.8095381792034	4.49117917881684\\
17.8084203760183	4.49066125212171\\
17.8073024289625	4.49014329454154\\
17.806184337999	4.48962530607265\\
17.8050661030907	4.48910728671135\\
17.8039477242006	4.48858923645395\\
17.8028292012915	4.48807115529678\\
17.8017105343265	4.48755304323613\\
17.8005917232683	4.48703490026834\\
17.7994727680799	4.4865167263897\\
17.7983536687241	4.48599852159654\\
17.7972344251638	4.48548028588515\\
17.7961150373617	4.48496201925186\\
17.7949955052808	4.48444372169297\\
17.7938758288837	4.4839253932048\\
17.7927560081334	4.48340703378364\\
17.7916360429925	4.48288864342581\\
17.7905159334238	4.48237022212761\\
17.7893956793901	4.48185176988535\\
17.7882752808542	4.48133328669534\\
17.7871547377786	4.48081477255388\\
17.7860340501262	4.48029622745728\\
17.7849132178596	4.47977765140184\\
17.7837922409415	4.47925904438386\\
17.7826711193345	4.47874040639964\\
17.7815498530013	4.47822173744549\\
17.7804284419045	4.47770303751771\\
17.7793068860067	4.4771843066126\\
17.7781851852706	4.47666554472646\\
17.7770633396587	4.47614675185558\\
17.7759413491336	4.47562792799627\\
17.7748192136579	4.47510907314482\\
17.773696933194	4.47459018729754\\
17.7725745077045	4.47407127045071\\
17.771451937152	4.47355232260063\\
17.7703292214989	4.4730333437436\\
17.7692063607077	4.47251433387592\\
17.7680833547409	4.47199529299387\\
17.766960203561	4.47147622109375\\
17.7658369071303	4.47095711817185\\
17.7647134654113	4.47043798422447\\
17.7635898783665	4.46991881924789\\
17.7624661459582	4.46939962323841\\
17.7613422681487	4.46888039619231\\
17.7602182449006	4.46836113810588\\
17.759094076176	4.46784184897543\\
17.7579697619374	4.46732252879721\\
17.7568453021471	4.46680317756754\\
17.7557206967674	4.46628379528269\\
17.7545959457606	4.46576438193896\\
17.753471049089	4.46524493753261\\
17.7523460067147	4.46472546205995\\
17.7512208186002	4.46420595551725\\
17.7500954847076	4.4636864179008\\
17.7489700049991	4.46316684920688\\
17.7478443794369	4.46264724943176\\
17.7467186079833	4.46212761857175\\
17.7455926906005	4.4616079566231\\
17.7444666272505	4.46108826358211\\
17.7433404178955	4.46056853944504\\
17.7422140624977	4.46004878420819\\
17.7410875610191	4.45952899786783\\
17.739960913422	4.45900918042023\\
17.7388341196683	4.45848933186167\\
17.7377071797202	4.45796945218843\\
17.7365800935396	4.45744954139678\\
17.7354528610888	4.456929599483\\
17.7343254823296	4.45640962644336\\
17.7331979572241	4.45588962227414\\
17.7320702857342	4.4553695869716\\
17.7309424678221	4.45484952053202\\
17.7298145034496	4.45432942295166\\
17.7286863925788	4.45380929422681\\
17.7275581351714	4.45328913435373\\
17.7264297311896	4.45276894332869\\
17.7253011805951	4.45224872114795\\
17.72417248335	4.45172846780779\\
17.7230436394159	4.45120818330447\\
17.7219146487549	4.45068786763426\\
17.7207855113288	4.45016752079342\\
17.7196562270995	4.44964714277823\\
17.7185267960286	4.44912673358493\\
17.7173972180781	4.44860629320981\\
17.7162674932098	4.44808582164911\\
17.7151376213853	4.44756531889911\\
17.7140076025666	4.44704478495606\\
17.7128774367153	4.44652421981623\\
17.7117471237931	4.44600362347588\\
17.7106166637618	4.44548299593125\\
17.7094860565831	4.44496233717863\\
17.7083553022187	4.44444164721426\\
17.7072244006302	4.4439209260344\\
17.7060933517794	4.4434001736353\\
17.7049621556277	4.44287939001324\\
17.703830812137	4.44235857516445\\
17.7026993212687	4.44183772908519\\
17.7015676829845	4.44131685177173\\
17.700435897246	4.44079594322031\\
17.6993039640147	4.44027500342719\\
17.6981718832521	4.43975403238861\\
17.69703965492	4.43923303010083\\
17.6959072789796	4.43871199656011\\
17.6947747553926	4.43819093176268\\
17.6936420841205	4.43766983570481\\
17.6925092651247	4.43714870838273\\
17.6913762983667	4.43662754979271\\
17.6902431838079	4.43610635993098\\
17.6891099214098	4.43558513879379\\
17.6879765111337	4.43506388637739\\
17.6868429529412	4.43454260267802\\
17.6857092467935	4.43402128769194\\
17.6845753926521	4.43349994141537\\
17.6834413904783	4.43297856384458\\
17.6823072402334	4.43245715497579\\
17.6811729418788	4.43193571480526\\
17.6800384953757	4.43141424332922\\
17.6789039006856	4.43089274054391\\
17.6777691577696	4.43037120644558\\
17.676634266589	4.42984964103046\\
17.6754992271052	4.4293280442948\\
17.6743640392792	4.42880641623482\\
17.6732287030724	4.42828475684677\\
17.6720932184459	4.42776306612689\\
17.6709575853609	4.42724134407141\\
17.6698218037786	4.42671959067656\\
17.6686858736602	4.42619780593859\\
17.6675497949668	4.42567598985372\\
17.6664135676595	4.42515414241819\\
17.6652771916995	4.42463226362823\\
17.6641406670478	4.42411035348008\\
17.6630039936656	4.42358841196996\\
17.6618671715138	4.4230664390941\\
17.6607302005536	4.42254443484874\\
17.659593080746	4.4220223992301\\
17.658455812052	4.42150033223442\\
17.6573183944326	4.42097823385792\\
17.6561808278488	4.42045610409683\\
17.6550431122616	4.41993394294736\\
17.6539052476319	4.41941175040577\\
17.6527672339208	4.41888952646825\\
17.651629071089	4.41836727113105\\
17.6504907590976	4.41784498439038\\
17.6493522979074	4.41732266624247\\
17.6482136874794	4.41680031668354\\
17.6470749277743	4.41627793570981\\
17.645936018753	4.41575552331751\\
17.6447969603765	4.41523307950284\\
17.6436577526054	4.41471060426204\\
17.6425183954006	4.41418809759132\\
17.641378888723	4.41366555948691\\
17.6402392325332	4.41314298994501\\
17.639099426792	4.41262038896184\\
17.6379594714602	4.41209775653363\\
17.6368193664985	4.41157509265658\\
17.6356791118677	4.41105239732692\\
17.6345387075284	4.41052967054085\\
17.6333981534412	4.4100069122946\\
17.632257449567	4.40948412258436\\
17.6311165958662	4.40896130140636\\
17.6299755922997	4.40843844875681\\
17.6288344388279	4.40791556463192\\
17.6276931354115	4.40739264902789\\
17.626551682011	4.40686970194094\\
17.6254100785872	4.40634672336728\\
17.6242683251004	4.40582371330311\\
17.6231264215113	4.40530067174464\\
17.6219843677803	4.40477759868808\\
17.6208421638681	4.40425449412964\\
17.6196998097351	4.40373135806551\\
17.6185573053417	4.40320819049191\\
17.6174146506485	4.40268499140504\\
17.6162718456158	4.4021617608011\\
17.6151288902042	4.4016384986763\\
17.613985784374	4.40111520502683\\
17.6128425280857	4.4005918798489\\
17.6116991212996	4.40006852313871\\
17.6105555639761	4.39954513489246\\
17.6094118560756	4.39902171510634\\
17.6082679975583	4.39849826377656\\
17.6071239883847	4.39797478089931\\
17.605979828515	4.3974512664708\\
17.6048355179096	4.39692772048721\\
17.6036910565287	4.39640414294475\\
17.6025464443325	4.3958805338396\\
17.6014016812813	4.39535689316797\\
17.6002567673354	4.39483322092605\\
17.599111702455	4.39430951711002\\
17.5979664866002	4.39378578171609\\
17.5968211197313	4.39326201474044\\
17.5956756018083	4.39273821617926\\
17.5945299327915	4.39221438602875\\
17.5933841126411	4.39169052428509\\
17.592238141317	4.39116663094448\\
17.5910920187795	4.3906427060031\\
17.5899457449886	4.39011874945714\\
17.5887993199043	4.38959476130278\\
17.5876527434869	4.38907074153621\\
17.5865060156961	4.38854669015363\\
17.5853591364923	4.3880226071512\\
17.5842121058352	4.38749849252512\\
17.5830649236849	4.38697434627157\\
17.5819175900015	4.38645016838673\\
17.5807701047447	4.38592595886679\\
17.5796224678747	4.38540171770792\\
17.5784746793513	4.38487744490631\\
17.5773267391345	4.38435314045814\\
17.5761786471841	4.38382880435958\\
17.57503040346	4.38330443660681\\
17.5738820079222	4.38278003719601\\
17.5727334605303	4.38225560612337\\
17.5715847612444	4.38173114338505\\
17.5704359100241	4.38120664897723\\
17.5692869068294	4.38068212289608\\
17.5681377516199	4.38015756513779\\
17.5669884443555	4.37963297569852\\
17.5658389849959	4.37910835457445\\
17.5646893735009	4.37858370176174\\
17.5635396098301	4.37805901725658\\
17.5623896939434	4.37753430105513\\
17.5612396258003	4.37700955315356\\
17.5600894053605	4.37648477354805\\
17.5589390325838	4.37595996223475\\
17.5577885074297	4.37543511920984\\
17.5566378298579	4.37491024446949\\
17.555486999828	4.37438533800986\\
17.5543360172995	4.37386039982712\\
17.5531848822322	4.37333542991744\\
17.5520335945854	4.37281042827697\\
17.5508821543189	4.37228539490189\\
17.549730561392	4.37176032978835\\
17.5485788157644	4.37123523293252\\
17.5474269173955	4.37071010433056\\
17.5462748662448	4.37018494397863\\
17.5451226622717	4.36965975187289\\
17.5439703054358	4.36913452800951\\
17.5428177956964	4.36860927238464\\
17.541665133013	4.36808398499443\\
17.540512317345	4.36755866583506\\
17.5393593486517	4.36703331490266\\
17.5382062268925	4.36650793219341\\
17.5370529520269	4.36598251770345\\
17.535899524014	4.36545707142895\\
17.5347459428133	4.36493159336605\\
17.533592208384	4.36440608351091\\
17.5324383206855	4.36388054185968\\
17.5312842796769	4.36335496840851\\
17.5301300853176	4.36282936315355\\
17.5289757375668	4.36230372609097\\
17.5278212363838	4.3617780572169\\
17.5266665817276	4.36125235652749\\
17.5255117735576	4.3607266240189\\
17.5243568118328	4.36020085968727\\
17.5232016965126	4.35967506352875\\
17.5220464275559	4.35914923553948\\
17.5208910049219	4.35862337571561\\
17.5197354285698	4.35809748405329\\
17.5185796984585	4.35757156054866\\
17.5174238145473	4.35704560519786\\
17.5162677767952	4.35651961799705\\
17.5151115851611	4.35599359894234\\
17.5139552396042	4.35546754802991\\
17.5127987400834	4.35494146525587\\
17.5116420865577	4.35441535061638\\
17.5104852789862	4.35388920410757\\
17.5093283173277	4.35336302572558\\
17.5081712015413	4.35283681546654\\
17.5070139315858	4.35231057332661\\
17.5058565074202	4.3517842993019\\
17.5046989290033	4.35125799338857\\
17.5035411962941	4.35073165558274\\
17.5023833092514	4.35020528588054\\
17.501225267834	4.34967888427812\\
17.5000670720008	4.34915245077161\\
17.4989087217106	4.34862598535714\\
17.4977502169222	4.34809948803083\\
17.4965915575944	4.34757295878883\\
17.495432743686	4.34704639762725\\
17.4942737751556	4.34651980454224\\
17.493114651962	4.34599317952992\\
17.491955374064	4.34546652258642\\
17.4907959414202	4.34493983370786\\
17.4896363539893	4.34441311289038\\
17.48847661173	4.34388636013009\\
17.487316714601	4.34335957542313\\
17.4861566625608	4.34283275876562\\
17.4849964555681	4.34230591015368\\
17.4838360935814	4.34177902958344\\
17.4826755765594	4.34125211705101\\
17.4815149044607	4.34072517255253\\
17.4803540772437	4.34019819608411\\
17.4791930948671	4.33967118764187\\
17.4780319572892	4.33914414722194\\
17.4768706644687	4.33861707482043\\
17.475709216364	4.33808997043345\\
17.4745476129336	4.33756283405714\\
17.4733858541359	4.3370356656876\\
17.4722239399293	4.33650846532095\\
17.4710618702724	4.3359812329533\\
17.4698996451234	4.33545396858078\\
17.4687372644407	4.3349266721995\\
17.4675747281828	4.33439934380556\\
17.466412036308	4.33387198339509\\
17.4652491887745	4.33334459096418\\
17.4640861855408	4.33281716650897\\
17.4629230265651	4.33228971002555\\
17.4617597118057	4.33176222151003\\
17.4605962412209	4.33123470095853\\
17.4594326147689	4.33070714836715\\
17.4582688324079	4.33017956373201\\
17.4571048940962	4.3296519470492\\
17.455940799792	4.32912429831483\\
17.4547765494534	4.32859661752502\\
17.4536121430386	4.32806890467586\\
17.4524475805058	4.32754115976346\\
17.451282861813	4.32701338278392\\
17.4501179869185	4.32648557373335\\
17.4489529557803	4.32595773260785\\
17.4477877683564	4.32542985940351\\
17.446622424605	4.32490195411645\\
17.4454569244841	4.32437401674275\\
17.4442912679517	4.32384604727852\\
17.4431254549659	4.32331804571986\\
17.4419594854845	4.32279001206287\\
17.4407933594657	4.32226194630363\\
17.4396270768674	4.32173384843826\\
17.4384606376475	4.32120571846284\\
17.4372940417639	4.32067755637347\\
17.4361272891746	4.32014936216624\\
17.4349603798374	4.31962113583725\\
17.4337933137103	4.31909287738259\\
17.4326260907511	4.31856458679836\\
17.4314587109176	4.31803626408063\\
17.4302911741676	4.31750790922551\\
17.4291234804591	4.31697952222909\\
17.4279556297497	4.31645110308744\\
17.4267876219972	4.31592265179667\\
17.4256194571595	4.31539416835286\\
17.4244511351941	4.3148656527521\\
17.423282656059	4.31433710499047\\
17.4221140197117	4.31380852506406\\
17.42094522611	4.31327991296895\\
17.4197762752115	4.31275126870123\\
17.4186071669739	4.31222259225699\\
17.4174379013549	4.3116938836323\\
17.4162684783119	4.31116514282324\\
17.4150988978028	4.3106363698259\\
17.413929159785	4.31010756463637\\
17.4127592642161	4.30957872725071\\
17.4115892110537	4.30904985766501\\
17.4104190002553	4.30852095587534\\
17.4092486317784	4.3079920218778\\
17.4080781055805	4.30746305566844\\
17.4069074216192	4.30693405724335\\
17.4057365798518	4.3064050265986\\
17.4045655802359	4.30587596373028\\
17.4033944227288	4.30534686863444\\
17.402223107288	4.30481774130718\\
17.4010516338709	4.30428858174455\\
17.3998800024348	4.30375938994263\\
17.3987082129372	4.3032301658975\\
17.3975362653353	4.30270090960522\\
17.3963641595866	4.30217162106186\\
17.3951918956482	4.3016423002635\\
17.3940194734776	4.3011129472062\\
17.3928468930319	4.30058356188602\\
17.3916741542685	4.30005414429905\\
17.3905012571445	4.29952469444133\\
17.3893282016173	4.29899521230894\\
17.388154987644	4.29846569789795\\
17.3869816151819	4.29793615120441\\
17.385808084188	4.29740657222439\\
17.3846343946195	4.29687696095396\\
17.3834605464337	4.29634731738917\\
17.3822865395875	4.29581764152609\\
17.3811123740382	4.29528793336077\\
17.3799380497427	4.29475819288928\\
17.3787635666582	4.29422842010768\\
17.3775889247418	4.29369861501202\\
17.3764141239503	4.29316877759837\\
17.375239164241	4.29263890786277\\
17.3740640455707	4.29210900580129\\
17.3728887678965	4.29157907140998\\
17.3717133311753	4.2910491046849\\
17.3705377353641	4.29051910562209\\
17.3693619804197	4.28998907421762\\
17.3681860662992	4.28945901046754\\
17.3670099929594	4.28892891436789\\
17.3658337603571	4.28839878591473\\
17.3646573684493	4.28786862510412\\
17.3634808171928	4.28733843193209\\
17.3623041065443	4.2868082063947\\
17.3611272364608	4.28627794848799\\
17.359950206899	4.28574765820802\\
17.3587730178157	4.28521733555083\\
17.3575956691676	4.28468698051247\\
17.3564181609115	4.28415659308898\\
17.355240493004	4.28362617327641\\
17.354062665402	4.2830957210708\\
17.352884678062	4.2825652364682\\
17.3517065309407	4.28203471946464\\
17.3505282239949	4.28150417005617\\
17.349349757181	4.28097358823883\\
17.3481711304558	4.28044297400866\\
17.3469923437759	4.2799123273617\\
17.3458133970977	4.27938164829399\\
17.3446342903779	4.27885093680156\\
17.343455023573	4.27832019288046\\
17.3422755966395	4.27778941652672\\
17.341096009534	4.27725860773638\\
17.339916262213	4.27672776650547\\
17.3387363546328	4.27619689283003\\
17.33755628675	4.27566598670608\\
17.336376058521	4.27513504812967\\
17.3351956699022	4.27460407709683\\
17.33401512085	4.27407307360359\\
17.3328344113208	4.27354203764597\\
17.331653541271	4.27301096922001\\
17.3304725106569	4.27247986832174\\
17.3292913194348	4.27194873494718\\
17.3281099675611	4.27141756909237\\
17.326928454992	4.27088637075334\\
17.3257467816839	4.2703551399261\\
17.3245649475929	4.26982387660668\\
17.3233829526754	4.26929258079112\\
17.3222007968875	4.26876125247542\\
17.3210184801855	4.26822989165563\\
17.3198360025255	4.26769849832776\\
17.3186533638637	4.26716707248782\\
17.3174705641563	4.26663561413185\\
17.3162876033594	4.26610412325587\\
17.315104481429	4.26557259985589\\
17.3139211983214	4.26504104392793\\
17.3127377539926	4.26450945546802\\
17.3115541483986	4.26397783447216\\
17.3103703814955	4.26344618093638\\
17.3091864532393	4.2629144948567\\
17.308002363586	4.26238277622912\\
17.3068181124917	4.26185102504967\\
17.3056336999122	4.26131924131435\\
17.3044491258036	4.26078742501919\\
17.3032643901217	4.26025557616019\\
17.3020794928226	4.25972369473336\\
17.300894433862	4.25919178073472\\
17.2997092131959	4.25865983416028\\
17.2985238307801	4.25812785500605\\
17.2973382865705	4.25759584326803\\
17.2961525805229	4.25706379894223\\
17.2949667125932	4.25653172202467\\
17.293780682737	4.25599961251134\\
17.2925944909102	4.25546747039826\\
17.2914081370686	4.25493529568142\\
17.2902216211678	4.25440308835685\\
17.2890349431637	4.25387084842053\\
17.2878481030118	4.25333857586847\\
17.2866611006679	4.25280627069667\\
17.2854739360876	4.25227393290114\\
17.2842866092266	4.25174156247787\\
17.2830991200405	4.25120915942287\\
17.2819114684849	4.25067672373214\\
17.2807236545155	4.25014425540167\\
17.2795356780877	4.24961175442746\\
17.2783475391572	4.24907922080552\\
17.2771592376795	4.24854665453182\\
17.27597077361	4.24801405560238\\
17.2747821469043	4.24748142401319\\
17.273593357518	4.24694875976023\\
17.2724044054063	4.24641606283951\\
17.2712152905249	4.24588333324702\\
17.270026012829	4.24535057097875\\
17.2688365722742	4.24481777603069\\
17.2676469688158	4.24428494839883\\
17.2664572024091	4.24375208807916\\
17.2652672730096	4.24321919506766\\
17.2640771805726	4.24268626936034\\
17.2628869250534	4.24215331095318\\
17.2616965064072	4.24162031984215\\
17.2605059245894	4.24108729602326\\
17.2593151795553	4.24055423949248\\
17.25812427126	4.2400211502458\\
17.2569331996588	4.2394880282792\\
17.2557419647069	4.23895487358867\\
17.2545505663595	4.23842168617019\\
17.2533590045718	4.23788846601973\\
17.2521672792988	4.23735521313329\\
17.2509753904959	4.23682192750684\\
17.2497833381179	4.23628860913637\\
17.2485911221202	4.23575525801784\\
17.2473987424576	4.23522187414724\\
17.2462061990854	4.23468845752054\\
17.2450134919585	4.23415500813373\\
17.2438206210319	4.23362152598277\\
17.2426275862607	4.23308801106365\\
17.2414343875999	4.23255446337233\\
17.2402410250043	4.23202088290479\\
17.239047498429	4.23148726965701\\
17.2378538078289	4.23095362362495\\
17.2366599531589	4.23041994480459\\
17.2354659343739	4.2298862331919\\
17.2342717514287	4.22935248878284\\
17.2330774042782	4.22881871157339\\
17.2318828928773	4.22828490155951\\
17.2306882171808	4.22775105873717\\
17.2294933771434	4.22721718310235\\
17.2282983727199	4.226683274651\\
17.2271032038652	4.22614933337909\\
17.2259078705339	4.22561535928258\\
17.2247123726808	4.22508135235745\\
17.2235167102606	4.22454731259964\\
17.2223208832279	4.22401324000513\\
17.2211248915375	4.22347913456988\\
17.2199287351439	4.22294499628985\\
17.2187324140019	4.22241082516099\\
17.217535928066	4.22187662117926\\
17.2163392772908	4.22134238434064\\
17.2151424616308	4.22080811464106\\
17.2139454810408	4.2202738120765\\
17.2127483354751	4.21973947664291\\
17.2115510248883	4.21920510833624\\
17.2103535492349	4.21867070715244\\
17.2091559084694	4.21813627308748\\
17.2079581025463	4.2176018061373\\
17.2067601314199	4.21706730629786\\
17.2055619950447	4.21653277356511\\
17.2043636933751	4.21599820793501\\
17.2031652263656	4.21546360940349\\
17.2019665939704	4.21492897796652\\
17.2007677961439	4.21439431362003\\
17.1995688328404	4.21385961635999\\
17.1983697040143	4.21332488618233\\
17.1971704096199	4.212790123083\\
17.1959709496113	4.21225532705795\\
17.1947713239429	4.21172049810313\\
17.193571532569	4.21118563621447\\
17.1923715754436	4.21065074138793\\
17.1911714525211	4.21011581361944\\
17.1899711637555	4.20958085290496\\
17.1887707091011	4.2090458592404\\
17.187570088512	4.20851083262173\\
17.1863693019423	4.20797577304488\\
17.1851683493461	4.20744068050579\\
17.1839672306775	4.2069055550004\\
17.1827659458906	4.20637039652464\\
17.1815644949394	4.20583520507445\\
17.1803628777779	4.20529998064577\\
17.1791610943601	4.20476472323454\\
17.1779591446401	4.20422943283668\\
17.1767570285717	4.20369410944814\\
17.175554746109	4.20315875306484\\
17.1743522972058	4.20262336368272\\
17.1731496818162	4.20208794129771\\
17.1719468998938	4.20155248590574\\
17.1707439513928	4.20101699750274\\
17.1695408362668	4.20048147608465\\
17.1683375544698	4.19994592164738\\
17.1671341059555	4.19941033418687\\
17.1659304906777	4.19887471369905\\
17.1647267085903	4.19833906017983\\
17.1635227596469	4.19780337362515\\
17.1623186438013	4.19726765403094\\
17.1611143610073	4.19673190139311\\
17.1599099112185	4.19619611570759\\
17.1587052943886	4.1956602969703\\
17.1575005104713	4.19512444517716\\
17.1562955594201	4.1945885603241\\
17.1550904411888	4.19405264240704\\
17.153885155731	4.19351669142189\\
17.1526797030002	4.19298070736457\\
17.1514740829499	4.19244469023101\\
17.1502682955338	4.19190864001712\\
17.1490623407053	4.19137255671881\\
17.147856218418	4.19083644033201\\
17.1466499286254	4.19030029085262\\
17.1454434712808	4.18976410827656\\
17.1442368463378	4.18922789259975\\
17.1430300537499	4.1886916438181\\
17.1418230934703	4.18815536192751\\
17.1406159654525	4.18761904692391\\
17.1394086696498	4.1870826988032\\
17.1382012060157	4.1865463175613\\
17.1369935745034	4.1860099031941\\
17.1357857750663	4.18547345569752\\
17.1345778076577	4.18493697506747\\
17.1333696722307	4.18440046129986\\
17.1321613687387	4.18386391439058\\
17.130952897135	4.18332733433555\\
17.1297442573726	4.18279072113067\\
17.1285354494049	4.18225407477184\\
17.127326473185	4.18171739525496\\
17.126117328666	4.18118068257595\\
17.1249080158011	4.1806439367307\\
17.1236985345434	4.1801071577151\\
17.122488884846	4.17957034552507\\
17.121279066662	4.1790335001565\\
17.1200690799444	4.17849662160528\\
17.1188589246463	4.17795970986732\\
17.1176486007207	4.17742276493852\\
17.1164381081206	4.17688578681476\\
17.115227446799	4.17634877549195\\
17.1140166167087	4.17581173096597\\
17.1128056178029	4.17527465323273\\
17.1115944500344	4.17473754228812\\
17.110383113356	4.17420039812802\\
17.1091716077207	4.17366322074834\\
17.1079599330814	4.17312601014495\\
17.1067480893908	4.17258876631376\\
17.1055360766018	4.17205148925064\\
17.1043238946673	4.1715141789515\\
17.1031115435399	4.17097683541221\\
17.1018990231724	4.17043945862866\\
17.1006863335177	4.16990204859675\\
17.0994734745283	4.16936460531235\\
17.0982604461571	4.16882712877135\\
17.0970472483567	4.16828961896963\\
17.0958338810797	4.16775207590308\\
17.0946203442788	4.16721449956758\\
17.0934066379067	4.16667688995901\\
17.0921927619159	4.16613924707326\\
17.090978716259	4.16560157090619\\
17.0897645008886	4.1650638614537\\
17.0885501157572	4.16452611871166\\
17.0873355608174	4.16398834267594\\
17.0861208360216	4.16345053334243\\
17.0849059413224	4.16291269070701\\
17.0836908766721	4.16237481476553\\
17.0824756420233	4.1618369055139\\
17.0812602373284	4.16129896294796\\
17.0800446625398	4.16076098706361\\
17.0788289176098	4.16022297785671\\
17.0776130024908	4.15968493532313\\
17.0763969171353	4.15914685945874\\
17.0751806614954	4.15860875025942\\
17.0739642355235	4.15807060772103\\
17.0727476391719	4.15753243183944\\
17.0715308723928	4.15699422261053\\
17.0703139351386	4.15645598003015\\
17.0690968273613	4.15591770409418\\
17.0678795490133	4.15537939479847\\
17.0666621000467	4.1548410521389\\
17.0654444804137	4.15430267611133\\
17.0642266900663	4.15376426671162\\
17.0630087289569	4.15322582393563\\
17.0617905970373	4.15268734777923\\
17.0605722942598	4.15214883823827\\
17.0593538205764	4.15161029530863\\
17.0581351759391	4.15107171898614\\
17.0569163602999	4.15053310926668\\
17.055697373611	4.14999446614611\\
17.0544782158241	4.14945578962027\\
17.0532588868914	4.14891707968503\\
17.0520393867647	4.14837833633624\\
17.050819715396	4.14783955956976\\
17.0495998727371	4.14730074938144\\
17.04837985874	4.14676190576713\\
17.0471596733565	4.14622302872268\\
17.0459393165385	4.14568411824396\\
17.0447187882377	4.1451451743268\\
17.0434980884061	4.14460619696707\\
17.0422772169952	4.1440671861606\\
17.041056173957	4.14352814190324\\
17.0398349592431	4.14298906419085\\
17.0386135728054	4.14244995301928\\
17.0373920145953	4.14191080838436\\
17.0361702845647	4.14137163028195\\
17.0349483826653	4.14083241870788\\
17.0337263088485	4.140293173658\\
17.0325040630662	4.13975389512816\\
17.0312816452698	4.1392145831142\\
17.0300590554109	4.13867523761196\\
17.0288362934411	4.13813585861728\\
17.0276133593119	4.137596446126\\
17.0263902529749	4.13705700013396\\
17.0251669743815	4.13651752063699\\
17.0239435234832	4.13597800763095\\
17.0227199002315	4.13543846111165\\
17.0214961045777	4.13489888107495\\
17.0202721364734	4.13435926751667\\
17.0190479958699	4.13381962043265\\
17.0178236827185	4.13327993981872\\
17.0165991969706	4.13274022567072\\
17.0153745385776	4.13220047798447\\
17.0141497074907	4.13166069675582\\
17.0129247036613	4.13112088198059\\
17.0116995270406	4.13058103365461\\
17.0104741775799	4.13004115177371\\
17.0092486552303	4.12950123633372\\
17.0080229599432	4.12896128733046\\
17.0067970916696	4.12842130475977\\
17.0055710503607	4.12788128861747\\
17.0043448359678	4.12734123889938\\
17.0031184484418	4.12680115560133\\
17.001891887734	4.12626103871915\\
17.0006651537954	4.12572088824865\\
16.999438246577	4.12518070418566\\
16.9982111660299	4.12464048652599\\
16.9969839121052	4.12410023526548\\
16.9957564847537	4.12355995039994\\
16.9945288839266	4.12301963192519\\
16.9933011095747	4.12247927983705\\
16.9920731616489	4.12193889413133\\
16.9908450401003	4.12139847480385\\
16.9896167448797	4.12085802185044\\
16.988388275938	4.12031753526689\\
16.987159633226	4.11977701504903\\
16.9859308166945	4.11923646119267\\
16.9847018262945	4.11869587369363\\
16.9834726619765	4.11815525254772\\
16.9822433236915	4.11761459775074\\
16.9810138113902	4.11707390929851\\
16.9797841250233	4.11653318718683\\
16.9785542645416	4.11599243141152\\
16.9773242298956	4.11545164196839\\
16.9760940210361	4.11491081885324\\
16.9748636379137	4.11436996206188\\
16.973633080479	4.11382907159011\\
16.9724023486827	4.11328814743375\\
16.9711714424753	4.11274718958858\\
16.9699403618073	4.11220619805042\\
16.9687091066294	4.11166517281508\\
16.967477676892	4.11112411387834\\
16.9662460725457	4.11058302123602\\
16.9650142935409	4.11004189488391\\
16.963782339828	4.10950073481782\\
16.9625502113575	4.10895954103353\\
16.9613179080798	4.10841831352686\\
16.9600854299454	4.10787705229359\\
16.9588527769045	4.10733575732952\\
16.9576199489075	4.10679442863046\\
16.9563869459048	4.10625306619218\\
16.9551537678466	4.1057116700105\\
16.9539204146833	4.10517024008119\\
16.9526868863651	4.10462877640006\\
16.9514531828422	4.10408727896289\\
16.950219304065	4.10354574776548\\
16.9489852499835	4.10300418280362\\
16.9477510205479	4.10246258407309\\
16.9465166157085	4.10192095156968\\
16.9452820354153	4.10137928528918\\
16.9440472796185	4.10083758522739\\
16.9428123482681	4.10029585138008\\
16.9415772413143	4.09975408374303\\
16.9403419587071	4.09921228231205\\
16.9391065003965	4.0986704470829\\
16.9378708663325	4.09812857805137\\
16.9366350564651	4.09758667521324\\
16.9353990707444	4.0970447385643\\
16.9341629091201	4.09650276810033\\
16.9329265715423	4.0959607638171\\
16.9316900579609	4.0954187257104\\
16.9304533683257	4.09487665377599\\
16.9292165025866	4.09433454800967\\
16.9279794606935	4.09379240840721\\
16.9267422425961	4.09325023496438\\
16.9255048482443	4.09270802767695\\
16.9242672775877	4.09216578654071\\
16.9230295305763	4.09162351155142\\
16.9217916071597	4.09108120270487\\
16.9205535072876	4.09053885999681\\
16.9193152309097	4.08999648342303\\
16.9180767779756	4.08945407297928\\
16.9168381484351	4.08891162866135\\
16.9155993422378	4.088369150465\\
16.9143603593332	4.08782663838599\\
16.9131211996709	4.0872840924201\\
16.9118818632005	4.0867415125631\\
16.9106423498715	4.08619889881074\\
16.9094026596335	4.08565625115879\\
16.9081627924359	4.08511356960302\\
16.9069227482282	4.08457085413919\\
16.9056825269598	4.08402810476306\\
16.9044421285803	4.0834853214704\\
16.9032015530389	4.08294250425695\\
16.9019608002851	4.08239965311849\\
16.9007198702683	4.08185676805078\\
16.8994787629378	4.08131384904957\\
16.8982374782428	4.08077089611062\\
16.8969960161328	4.08022790922968\\
16.8957543765569	4.07968488840252\\
16.8945125594645	4.07914183362488\\
16.8932705648048	4.07859874489253\\
16.8920283925269	4.07805562220121\\
16.8907860425802	4.07751246554668\\
16.8895435149136	4.07696927492469\\
16.8883008094765	4.076426050331\\
16.8870579262179	4.07588279176134\\
16.8858148650869	4.07533949921148\\
16.8845716260326	4.07479617267716\\
16.883328209004	4.07425281215414\\
16.8820846139503	4.07370941763814\\
16.8808408408203	4.07316598912493\\
16.8795968895632	4.07262252661025\\
16.8783527601278	4.07207903008984\\
16.8771084524632	4.07153549955945\\
16.8758639665183	4.07099193501482\\
16.8746193022418	4.0704483364517\\
16.8733744595829	4.06990470386582\\
16.8721294384902	4.06936103725292\\
16.8708842389127	4.06881733660876\\
16.8696388607992	4.06827360192905\\
16.8683933040984	4.06772983320955\\
16.8671475687591	4.067186030446\\
16.8659016547301	4.06664219363412\\
16.8646555619602	4.06609832276966\\
16.863409290398	4.06555441784834\\
16.8621628399921	4.06501047886592\\
16.8609162106914	4.06446650581811\\
16.8596694024443	4.06392249870065\\
16.8584224151996	4.06337845750927\\
16.8571752489059	4.06283438223971\\
16.8559279035116	4.0622902728877\\
16.8546803789654	4.06174612944895\\
16.8534326752158	4.06120195191922\\
16.8521847922112	4.06065774029421\\
16.8509367299003	4.06011349456966\\
16.8496884882314	4.0595692147413\\
16.848440067153	4.05902490080485\\
16.8471914666135	4.05848055275604\\
16.8459426865613	4.05793617059058\\
16.8446937269448	4.05739175430421\\
16.8434445877123	4.05684730389265\\
16.8421952688122	4.05630281935161\\
16.8409457701928	4.05575830067682\\
16.8396960918023	4.055213747864\\
16.838446233589	4.05466916090886\\
16.8371961955012	4.05412453980713\\
16.8359459774871	4.05357988455453\\
16.8346955794949	4.05303519514677\\
16.8334450014727	4.05249047157956\\
16.8321942433687	4.05194571384862\\
16.8309433051311	4.05140092194967\\
16.8296921867079	4.05085609587842\\
16.8284408880472	4.05031123563058\\
16.8271894090971	4.04976634120187\\
16.8259377498056	4.04922141258799\\
16.8246859101208	4.04867644978466\\
16.8234338899906	4.04813145278758\\
16.8221816893631	4.04758642159247\\
16.8209293081861	4.04704135619503\\
16.8196767464076	4.04649625659097\\
16.8184240039754	4.04595112277599\\
16.8171710808376	4.04540595474581\\
16.8159179769419	4.04486075249612\\
16.8146646922361	4.04431551602263\\
16.8134112266682	4.04377024532104\\
16.8121575801858	4.04322494038706\\
16.8109037527368	4.04267960121639\\
16.8096497442688	4.04213422780472\\
16.8083955547297	4.04158882014776\\
16.8071411840671	4.04104337824121\\
16.8058866322287	4.04049790208076\\
16.8046318991622	4.03995239166211\\
16.8033769848151	4.03940684698096\\
16.8021218891352	4.038861268033\\
16.8008666120699	4.03831565481394\\
16.7996111535669	4.03777000731946\\
16.7983555135736	4.03722432554525\\
16.7970996920377	4.03667860948701\\
16.7958436889066	4.03613285914044\\
16.7945875041278	4.03558707450121\\
16.7933311376487	4.03504125556503\\
16.7920745894167	4.03449540232758\\
16.7908178593794	4.03394951478455\\
16.789560947484	4.03340359293163\\
16.7883038536779	4.0328576367645\\
16.7870465779084	4.03231164627885\\
16.7857891201229	4.03176562147037\\
16.7845314802687	4.03121956233474\\
16.783273658293	4.03067346886764\\
16.7820156541431	4.03012734106476\\
16.7807574677661	4.02958117892178\\
16.7794990991094	4.02903498243437\\
16.77824054812	4.02848875159823\\
16.7769818147451	4.02794248640903\\
16.7757228989319	4.02739618686244\\
16.7744638006274	4.02684985295416\\
16.7732045197788	4.02630348467984\\
16.7719450563331	4.02575708203517\\
16.7706854102374	4.02521064501583\\
16.7694255814386	4.02466417361749\\
16.7681655698837	4.02411766783582\\
16.7669053755198	4.0235711276665\\
16.7656449982938	4.0230245531052\\
16.7643844381525	4.02247794414759\\
16.763123695043	4.02193130078934\\
16.761862768912	4.02138462302612\\
16.7606016597064	4.0208379108536\\
16.7593403673731	4.02029116426744\\
16.7580788918589	4.01974438326332\\
16.7568172331105	4.01919756783691\\
16.7555553910747	4.01865071798386\\
16.7542933656982	4.01810383369984\\
16.7530311569279	4.01755691498051\\
16.7517687647102	4.01700996182155\\
16.750506188992	4.01646297421861\\
16.7492434297199	4.01591595216734\\
16.7479804868405	4.01536889566343\\
16.7467173603003	4.01482180470252\\
16.745454050046	4.01427467928026\\
16.7441905560241	4.01372751939234\\
16.7429268781812	4.01318032503438\\
16.7416630164637	4.01263309620207\\
16.7403989708182	4.01208583289104\\
16.739134741191	4.01153853509696\\
16.7378703275287	4.01099120281548\\
16.7366057297776	4.01044383604226\\
16.7353409478841	4.00989643477294\\
16.7340759817946	4.00934899900317\\
16.7328108314555	4.00880152872862\\
16.731545496813	4.00825402394492\\
16.7302799778134	4.00770648464774\\
16.7290142744031	4.0071589108327\\
16.7277483865282	4.00661130249548\\
16.726482314135	4.0060636596317\\
16.7252160571697	4.00551598223702\\
16.7239496155785	4.00496827030708\\
16.7226829893075	4.00442052383753\\
16.7214161783028	4.003872742824\\
16.7201491825106	4.00332492726215\\
16.718882001877	4.00277707714761\\
16.7176146363479	4.00222919247603\\
16.7163470858695	4.00168127324304\\
16.7150793503877	4.00113331944429\\
16.7138114298485	4.00058533107541\\
16.712543324198	4.00003730813203\\
16.7112750333819	3.99948925060981\\
16.7100065573464	3.99894115850437\\
16.7087378960372	3.99839303181135\\
16.7074690494002	3.99784487052639\\
16.7062000173813	3.99729667464511\\
16.7049307999262	3.99674844416314\\
16.7036613969809	3.99620017907614\\
16.702391808491	3.99565187937971\\
16.7011220344023	3.99510354506949\\
16.6998520746606	3.99455517614112\\
16.6985819292115	3.99400677259022\\
16.6973115980008	3.99345833441242\\
16.696041080974	3.99290986160334\\
16.6947703780768	3.99236135415861\\
16.6934994892549	3.99181281207387\\
16.6922284144537	3.99126423534472\\
16.690957153619	3.9907156239668\\
16.6896857066961	3.99016697793572\\
16.6884140736306	3.98961829724712\\
16.6871422543681	3.9890695818966\\
16.6858702488539	3.9885208318798\\
16.6845980570335	3.98797204719233\\
16.6833256788524	3.98742322782981\\
16.6820531142559	3.98687437378785\\
16.6807803631893	3.98632548506208\\
16.6795074255981	3.98577656164811\\
16.6782343014276	3.98522760354156\\
16.676960990623	3.98467861073803\\
16.6756874931296	3.98412958323315\\
16.6744138088927	3.98358052102253\\
16.6731399378574	3.98303142410178\\
16.6718658799691	3.9824822924665\\
16.6705916351729	3.98193312611232\\
16.6693172034138	3.98138392503483\\
16.6680425846372	3.98083468922966\\
16.666767778788	3.9802854186924\\
16.6654927858113	3.97973611341866\\
16.6642176056522	3.97918677340405\\
16.6629422382557	3.97863739864418\\
16.6616666835669	3.97808798913465\\
16.6603909415306	3.97753854487106\\
16.659115012092	3.97698906584901\\
16.6578388951958	3.97643955206412\\
16.6565625907871	3.97589000351197\\
16.6552860988106	3.97534042018817\\
16.6540094192113	3.97479080208832\\
16.6527325519341	3.97424114920802\\
16.6514554969236	3.97369146154287\\
16.6501782541247	3.97314173908845\\
16.6489008234822	3.97259198184038\\
16.6476232049408	3.97204218979423\\
16.6463453984452	3.97149236294562\\
16.6450674039402	3.97094250129013\\
16.6437892213703	3.97039260482335\\
16.6425108506802	3.96984267354089\\
16.6412322918145	3.96929270743831\\
16.6399535447179	3.96874270651123\\
16.6386746093349	3.96819267075522\\
16.63739548561	3.96764260016588\\
16.6361161734878	3.9670924947388\\
16.6348366729128	3.96654235446955\\
16.6335569838293	3.96599217935374\\
16.632277106182	3.96544196938693\\
16.6309970399151	3.96489172456472\\
16.6297167849731	3.96434144488269\\
16.6284363413004	3.96379113033643\\
16.6271557088414	3.96324078092151\\
16.6258748875402	3.96269039663352\\
16.6245938773413	3.96213997746803\\
16.6233126781888	3.96158952342063\\
16.6220312900272	3.96103903448689\\
16.6207497128005	3.9604885106624\\
16.619467946453	3.95993795194272\\
16.6181859909289	3.95938735832344\\
16.6169038461722	3.95883672980013\\
16.6156215121272	3.95828606636836\\
16.614338988738	3.95773536802371\\
16.6130562759485	3.95718463476175\\
16.6117733737029	3.95663386657806\\
16.6104902819452	3.95608306346819\\
16.6092070006193	3.95553222542774\\
16.6079235296694	3.95498135245225\\
16.6066398690392	3.95443044453731\\
16.6053560186728	3.95387950167847\\
16.6040719785141	3.95332852387131\\
16.6027877485069	3.95277751111139\\
16.601503328595	3.95222646339428\\
16.6002187187224	3.95167538071554\\
16.5989339188327	3.95112426307073\\
16.5976489288699	3.95057311045542\\
16.5963637487775	3.95002192286517\\
16.5950783784995	3.94947070029554\\
16.5937928179794	3.94891944274208\\
16.592507067161	3.94836815020037\\
16.5912211259879	3.94781682266595\\
16.5899349944036	3.94726546013439\\
16.588648672352	3.94671406260123\\
16.5873621597764	3.94616263006205\\
16.5860754566205	3.94561116251238\\
16.5847885628278	3.9450596599478\\
16.5835014783418	3.94450812236384\\
16.5822142031059	3.94395654975606\\
16.5809267370637	3.94340494212001\\
16.5796390801585	3.94285329945125\\
16.5783512323337	3.94230162174533\\
16.5770631935328	3.94174990899778\\
16.5757749636991	3.94119816120416\\
16.5744865427758	3.94064637836003\\
16.5731979307064	3.94009456046091\\
16.571909127434	3.93954270750237\\
16.570620132902	3.93899081947994\\
16.5693309470536	3.93843889638917\\
16.5680415698319	3.93788693822561\\
16.5667520011801	3.93733494498478\\
16.5654622410414	3.93678291666224\\
16.5641722893589	3.93623085325353\\
16.5628821460757	3.93567875475418\\
16.5615918111349	3.93512662115974\\
16.5603012844795	3.93457445246574\\
16.5590105660526	3.93402224866771\\
16.5577196557971	3.93347000976121\\
16.556428553656	3.93291773574175\\
16.5551372595723	3.93236542660488\\
16.5538457734889	3.93181308234613\\
16.5525540953487	3.93126070296103\\
16.5512622250945	3.93070828844511\\
16.5499701626692	3.93015583879392\\
16.5486779080156	3.92960335400296\\
16.5473854610765	3.92905083406778\\
16.5460928217947	3.9284982789839\\
16.5447999901129	3.92794568874686\\
16.5435069659739	3.92739306335217\\
16.5422137493202	3.92684040279537\\
16.5409203400947	3.92628770707198\\
16.5396267382399	3.92573497617752\\
16.5383329436984	3.92518221010751\\
16.5370389564129	3.92462940885749\\
16.5357447763258	3.92407657242297\\
16.5344504033798	3.92352370079947\\
16.5331558375174	3.92297079398251\\
16.531861078681	3.92241785196761\\
16.530566126813	3.9218648747503\\
16.529270981856	3.92131186232608\\
16.5279756437524	3.92075881469047\\
16.5266801124444	3.92020573183899\\
16.5253843878745	3.91965261376717\\
16.5240884699851	3.91909946047049\\
16.5227923587183	3.91854627194449\\
16.5214960540165	3.91799304818467\\
16.520199555822	3.91743978918655\\
16.518902864077	3.91688649494564\\
16.5176059787236	3.91633316545744\\
16.5163088997041	3.91577980071746\\
16.5150116269606	3.91522640072122\\
16.5137141604352	3.91467296546422\\
16.51241650007	3.91411949494197\\
16.5111186458071	3.91356598914997\\
16.5098205975886	3.91301244808372\\
16.5085223553565	3.91245887173873\\
16.5072239190527	3.91190526011051\\
16.5059252886192	3.91135161319455\\
16.504626463998	3.91079793098636\\
16.503327445131	3.91024421348144\\
16.5020282319601	3.90969046067528\\
16.5007288244271	3.90913667256338\\
16.4994292224739	3.90858284914125\\
16.4981294260423	3.90802899040438\\
16.496829435074	3.90747509634826\\
16.4955292495109	3.90692116696839\\
16.4942288692947	3.90636720226027\\
16.492928294367	3.90581320221938\\
16.4916275246697	3.90525916684123\\
16.4903265601442	3.9047050961213\\
16.4890254007323	3.90415099005508\\
16.4877240463755	3.90359684863806\\
16.4864224970154	3.90304267186574\\
16.4851207525936	3.9024884597336\\
16.4838188130517	3.90193421223712\\
16.482516678331	3.9013799293718\\
16.4812143483731	3.90082561113312\\
16.4799118231194	3.90027125751656\\
16.4786091025113	3.89971686851762\\
16.4773061864902	3.89916244413176\\
16.4760030749976	3.89860798435448\\
16.4746997679746	3.89805348918125\\
16.4733962653627	3.89749895860756\\
16.4720925671032	3.89694439262888\\
16.4707886731372	3.8963897912407\\
16.469484583406	3.89583515443849\\
16.4681802978509	3.89528048221773\\
16.466875816413	3.89472577457389\\
16.4655711390334	3.89417103150246\\
16.4642662656534	3.89361625299889\\
16.4629611962139	3.89306143905868\\
16.4616559306561	3.89250658967729\\
16.460350468921	3.89195170485019\\
16.4590448109496	3.89139678457285\\
16.457738956683	3.89084182884075\\
16.4564329060621	3.89028683764935\\
16.4551266590278	3.88973181099411\\
16.4538202155211	3.88917674887052\\
16.4525135754828	3.88862165127404\\
16.4512067388538	3.88806651820012\\
16.449899705575	3.88751134964424\\
16.4485924755871	3.88695614560186\\
16.447285048831	3.88640090606844\\
16.4459774252473	3.88584563103945\\
16.4446696047769	3.88529032051034\\
16.4433615873604	3.88473497447658\\
16.4420533729384	3.88417959293363\\
16.4407449614517	3.88362417587694\\
16.4394363528409	3.88306872330197\\
16.4381275470465	3.88251323520419\\
16.4368185440091	3.88195771157904\\
16.4355093436693	3.88140215242198\\
16.4341999459675	3.88084655772847\\
16.4328903508443	3.88029092749396\\
16.4315805582402	3.8797352617139\\
16.4302705680955	3.87917956038374\\
16.4289603803506	3.87862382349893\\
16.427649994946	3.87806805105494\\
16.4263394118219	3.87751224304719\\
16.4250286309188	3.87695639947115\\
16.4237176521769	3.87640052032226\\
16.4224064755365	3.87584460559596\\
16.4210951009378	3.8752886552877\\
16.419783528321	3.87473266939293\\
16.4184717576263	3.8741766479071\\
16.417159788794	3.87362059082563\\
16.4158476217641	3.87306449814398\\
16.4145352564767	3.87250836985759\\
16.4132226928719	3.87195220596189\\
16.4119099308898	3.87139600645233\\
16.4105969704704	3.87083977132435\\
16.4092838115537	3.87028350057338\\
16.4079704540796	3.86972719419486\\
16.4066568979882	3.86917085218422\\
16.4053431432193	3.86861447453691\\
16.4040291897128	3.86805806124835\\
16.4027150374086	3.86750161231398\\
16.4014006862465	3.86694512772923\\
16.4000861361664	3.86638860748954\\
16.3987713871079	3.86583205159033\\
16.3974564390109	3.86527546002703\\
16.3961412918151	3.86471883279508\\
16.3948259454602	3.86416216988989\\
16.3935103998859	3.86360547130691\\
16.3921946550317	3.86304873704154\\
16.3908787108374	3.86249196708923\\
16.3895625672425	3.86193516144539\\
16.3882462241865	3.86137832010545\\
16.3869296816091	3.86082144306484\\
16.3856129394496	3.86026453031896\\
16.3842959976477	3.85970758186325\\
16.3829788561427	3.85915059769312\\
16.3816615148741	3.858593577804\\
16.3803439737812	3.8580365221913\\
16.3790262328035	3.85747943085044\\
16.3777082918802	3.85692230377683\\
16.3763901509507	3.8563651409659\\
16.3750718099543	3.85580794241305\\
16.3737532688303	3.85525070811371\\
16.3724345275178	3.85469343806328\\
16.3711155859561	3.85413613225717\\
16.3697964440843	3.85357879069081\\
16.3684771018416	3.85302141335959\\
16.3671575591671	3.85246400025893\\
16.3658378159999	3.85190655138423\\
16.3645178722791	3.85134906673091\\
16.3631977279437	3.85079154629437\\
16.3618773829327	3.85023399007002\\
16.3605568371851	3.84967639805327\\
16.3592360906399	3.8491187702395\\
16.3579151432359	3.84856110662414\\
16.3565939949121	3.84800340720259\\
16.3552726456074	3.84744567197023\\
16.3539510952605	3.84688790092249\\
16.3526293438103	3.84633009405474\\
16.3513073911956	3.84577225136241\\
16.3499852373552	3.84521437284087\\
16.3486628822277	3.84465645848553\\
16.3473403257519	3.84409850829179\\
16.3460175678664	3.84354052225504\\
16.34469460851	3.84298250037068\\
16.3433714476211	3.8424244426341\\
16.3420480851385	3.84186634904069\\
16.3407245210006	3.84130821958584\\
16.339400755146	3.84075005426495\\
16.3380767875132	3.8401918530734\\
16.3367526180407	3.83963361600659\\
16.3354282466669	3.8390753430599\\
16.3341036733303	3.83851703422872\\
16.3327788979692	3.83795868950844\\
16.3314539205221	3.83740030889444\\
16.3301287409271	3.83684189238211\\
16.3288033591228	3.83628343996683\\
16.3274777750472	3.83572495164399\\
16.3261519886388	3.83516642740896\\
16.3248259998357	3.83460786725712\\
16.323499808576	3.83404927118387\\
16.3221734147981	3.83349063918457\\
16.3208468184399	3.83293197125462\\
16.3195200194397	3.83237326738937\\
16.3181930177355	3.83181452758421\\
16.3168658132653	3.83125575183452\\
16.3155384059673	3.83069694013567\\
16.3142107957793	3.83013809248304\\
16.3128829826393	3.82957920887199\\
16.3115549664853	3.82902028929791\\
16.3102267472553	3.82846133375616\\
16.308898324887	3.82790234224211\\
16.3075696993184	3.82734331475113\\
16.3062408704872	3.82678425127859\\
16.3049118383313	3.82622515181986\\
16.3035826027884	3.82566601637031\\
16.3022531637963	3.8251068449253\\
16.3009235212926	3.8245476374802\\
16.2995936752151	3.82398839403037\\
16.2982636255015	3.82342911457117\\
16.2969333720893	3.82286979909797\\
16.2956029149161	3.82231044760613\\
16.2942722539195	3.821751060091\\
16.2929413890371	3.82119163654796\\
16.2916103202064	3.82063217697235\\
16.2902790473648	3.82007268135955\\
16.2889475704498	3.81951314970489\\
16.2876158893988	3.81895358200374\\
16.2862840041492	3.81839397825146\\
16.2849519146384	3.81783433844339\\
16.2836196208037	3.8172746625749\\
16.2822871225825	3.81671495064133\\
16.280954419912	3.81615520263804\\
16.2796215127294	3.81559541856038\\
16.2782884009721	3.81503559840369\\
16.2769550845771	3.81447574216333\\
16.2756215634816	3.81391584983464\\
16.2742878376229	3.81335592141298\\
16.2729539069379	3.81279595689368\\
16.2716197713638	3.8122359562721\\
16.2702854308377	3.81167591954357\\
16.2689508852964	3.81111584670345\\
16.2676161346771	3.81055573774707\\
16.2662811789167	3.80999559266977\\
16.2649460179521	3.8094354114669\\
16.2636106517202	3.8088751941338\\
16.262275080158	3.80831494066581\\
16.2609393032022	3.80775465105825\\
16.2596033207896	3.80719432530648\\
16.2582671328571	3.80663396340583\\
16.2569307393414	3.80607356535163\\
16.2555941401793	3.80551313113922\\
16.2542573353074	3.80495266076392\\
16.2529203246624	3.80439215422109\\
16.2515831081809	3.80383161150604\\
16.2502456857997	3.8032710326141\\
16.2489080574551	3.80271041754061\\
16.2475702230839	3.8021497662809\\
16.2462321826225	3.80158907883029\\
16.2448939360074	3.80102835518412\\
16.2435554831751	3.8004675953377\\
16.242216824062	3.79990679928637\\
16.2408779586045	3.79934596702544\\
16.2395388867391	3.79878509855025\\
16.238199608402	3.79822419385611\\
16.2368601235295	3.79766325293835\\
16.2355204320581	3.79710227579228\\
16.2341805339238	3.79654126241324\\
16.232840429063	3.79598021279653\\
16.2315001174119	3.79541912693748\\
16.2301595989065	3.79485800483139\\
16.2288188734832	3.7942968464736\\
16.2274779410779	3.79373565185941\\
16.2261368016267	3.79317442098415\\
16.2247954550658	3.79261315384311\\
16.2234539013311	3.79205185043162\\
16.2221121403586	3.79149051074499\\
16.2207701720843	3.79092913477853\\
16.2194279964441	3.79036772252755\\
16.218085613374	3.78980627398735\\
16.2167430228098	3.78924478915325\\
16.2154002246873	3.78868326802056\\
16.2140572189423	3.78812171058457\\
16.2127140055107	3.7875601168406\\
16.2113705843283	3.78699848678396\\
16.2100269553306	3.78643682040993\\
16.2086831184535	3.78587511771384\\
16.2073390736325	3.78531337869097\\
16.2059948208034	3.78475160333663\\
16.2046503599017	3.78418979164612\\
16.203305690863	3.78362794361474\\
16.2019608136229	3.78306605923779\\
16.2006157281168	3.78250413851057\\
16.1992704342803	3.78194218142836\\
16.1979249320488	3.78138018798647\\
16.1965792213577	3.7808181581802\\
16.1952333021425	3.78025609200483\\
16.1938871743385	3.77969398945565\\
16.192540837881	3.77913185052797\\
16.1911942927054	3.77856967521707\\
16.1898475387468	3.77800746351824\\
16.1885005759407	3.77744521542677\\
16.1871534042221	3.77688293093794\\
16.1858060235263	3.77632061004705\\
16.1844584337884	3.77575825274939\\
16.1831106349436	3.77519585904023\\
16.1817626269269	3.77463342891486\\
16.1804144096735	3.77407096236857\\
16.1790659831182	3.77350845939664\\
16.1777173471962	3.77294591999435\\
16.1763685018425	3.77238334415697\\
16.1750194469919	3.7718207318798\\
16.1736701825794	3.77125808315811\\
16.1723207085398	3.77069539798718\\
16.1709710248081	3.77013267636228\\
16.169621131319	3.7695699182787\\
16.1682710280074	3.7690071237317\\
16.1669207148079	3.76844429271656\\
16.1655701916554	3.76788142522856\\
16.1642194584844	3.76731852126296\\
16.1628685152298	3.76675558081505\\
16.1615173618261	3.76619260388008\\
16.160165998208	3.76562959045334\\
16.15881442431	3.76506654053008\\
16.1574626400666	3.76450345410559\\
16.1561106454125	3.76394033117511\\
16.154758440282	3.76337717173393\\
16.1534060246097	3.7628139757773\\
16.1520533983299	3.76225074330049\\
16.1507005613771	3.76168747429876\\
16.1493475136856	3.76112416876738\\
16.1479942551897	3.76056082670161\\
16.1466407858238	3.75999744809671\\
16.1452871055221	3.75943403294793\\
16.1439332142189	3.75887058125054\\
16.1425791118484	3.7583070929998\\
16.1412247983448	3.75774356819095\\
16.1398702736421	3.75718000681927\\
16.1385155376746	3.75661640887999\\
16.1371605903764	3.75605277436839\\
16.1358054316813	3.7554891032797\\
16.1344500615236	3.75492539560919\\
16.1330944798372	3.75436165135209\\
16.1317386865561	3.75379787050368\\
16.1303826816141	3.75323405305919\\
16.1290264649452	3.75267019901387\\
16.1276700364834	3.75210630836297\\
16.1263133961623	3.75154238110174\\
16.1249565439158	3.75097841722542\\
16.1235994796777	3.75041441672926\\
16.1222422033818	3.7498503796085\\
16.1208847149618	3.74928630585839\\
16.1195270143513	3.74872219547416\\
16.1181691014839	3.74815804845107\\
16.1168109762935	3.74759386478434\\
16.1154526387134	3.74702964446922\\
16.1140940886773	3.74646538750094\\
16.1127353261187	3.74590109387475\\
16.1113763509711	3.74533676358589\\
16.110017163168	3.74477239662957\\
16.1086577626428	3.74420799300105\\
16.1072981493289	3.74364355269556\\
16.1059383231597	3.74307907570832\\
16.1045782840685	3.74251456203457\\
16.1032180319887	3.74195001166954\\
16.1018575668535	3.74138542460847\\
16.1004968885961	3.74082080084657\\
16.0991359971498	3.74025614037908\\
16.0977748924478	3.73969144320123\\
16.0964135744232	3.73912670930824\\
16.0950520430092	3.73856193869534\\
16.0936902981388	3.73799713135775\\
16.0923283397451	3.7374322872907\\
16.0909661677611	3.7368674064894\\
16.0896037821199	3.73630248894909\\
16.0882411827543	3.73573753466497\\
16.0868783695973	3.73517254363228\\
16.0855153425819	3.73460751584622\\
16.0841521016408	3.73404245130202\\
16.082788646707	3.7334773499949\\
16.0814249777132	3.73291221192007\\
16.0800610945923	3.73234703707274\\
16.0786969972768	3.73178182544814\\
16.0773326856997	3.73121657704147\\
16.0759681597935	3.73065129184794\\
16.0746034194909	3.73008596986277\\
16.0732384647245	3.72952061108117\\
16.0718732954269	3.72895521549835\\
16.0705079115306	3.72838978310951\\
16.0691423129683	3.72782431390986\\
16.0677764996723	3.72725880789462\\
16.0664104715751	3.72669326505898\\
16.0650442286091	3.72612768539816\\
16.0636777707068	3.72556206890735\\
16.0623110978005	3.72499641558175\\
16.0609442098226	3.72443072541658\\
16.0595771067052	3.72386499840703\\
16.0582097883808	3.7232992345483\\
16.0568422547815	3.72273343383559\\
16.0554745058395	3.7221675962641\\
16.054106541487	3.72160172182903\\
16.0527383616562	3.72103581052557\\
16.0513699662791	3.72046986234892\\
16.0500013552879	3.71990387729428\\
16.0486325286145	3.71933785535683\\
16.0472634861909	3.71877179653177\\
16.0458942279493	3.71820570081429\\
16.0445247538214	3.71763956819958\\
16.0431550637392	3.71707339868284\\
16.0417851576347	3.71650719225924\\
16.0404150354395	3.71594094892399\\
16.0390446970856	3.71537466867226\\
16.0376741425048	3.71480835149924\\
16.0363033716287	3.71424199740012\\
16.0349323843892	3.71367560637009\\
16.0335611807179	3.71310917840431\\
16.0321897605464	3.71254271349799\\
16.0308181238064	3.71197621164629\\
16.0294462704294	3.7114096728444\\
16.0280742003471	3.7108430970875\\
16.0267019134909	3.71027648437077\\
16.0253294097924	3.70970983468938\\
16.023956689183	3.70914314803852\\
16.0225837515941	3.70857642441335\\
16.0212105969571	3.70800966380905\\
16.0198372252034	3.7074428662208\\
16.0184636362644	3.70687603164377\\
16.0170898300712	3.70630916007313\\
16.0157158065553	3.70574225150405\\
16.0143415656477	3.7051753059317\\
16.0129671072798	3.70460832335126\\
16.0115924313826	3.70404130375788\\
16.0102175378874	3.70347424714674\\
16.0088424267251	3.702907153513\\
16.007467097827	3.70234002285183\\
16.006091551124	3.70177285515839\\
16.0047157865471	3.70120565042785\\
16.0033398040273	3.70063840865537\\
16.0019636034955	3.7000711298361\\
16.0005871848827	3.69950381396522\\
15.9992105481197	3.69893646103788\\
15.9978336931373	3.69836907104923\\
15.9964566198664	3.69780164399445\\
15.9950793282377	3.69723417986867\\
15.993701818182	3.69666667866707\\
15.99232408963	3.69609914038479\\
15.9909461425123	3.69553156501699\\
15.9895679767596	3.69496395255882\\
15.9881895923025	3.69439630300543\\
15.9868109890716	3.69382861635198\\
15.9854321669975	3.69326089259361\\
15.9840531260105	3.69269313172547\\
15.9826738660413	3.69212533374272\\
15.9812943870202	3.6915574986405\\
15.9799146888776	3.69098962641395\\
15.978534771544	3.69042171705822\\
15.9771546349497	3.68985377056846\\
15.975774279025	3.68928578693981\\
15.9743937037001	3.68871776616741\\
15.9730129089054	3.6881497082464\\
15.971631894571	3.68758161317193\\
15.970250660627	3.68701348093913\\
15.9688692070037	3.68644531154314\\
15.9674875336312	3.6858771049791\\
15.9661056404395	3.68530886124215\\
15.9647235273587	3.68474058032742\\
15.9633411943188	3.68417226223005\\
15.9619586412497	3.68360390694517\\
15.9605758680815	3.68303551446792\\
15.9591928747439	3.68246708479342\\
15.957809661167	3.68189861791681\\
15.9564262272805	3.68133011383322\\
15.9550425730142	3.68076157253777\\
15.953658698298	3.6801929940256\\
15.9522746030615	3.67962437829184\\
15.9508902872345	3.6790557253316\\
15.9495057507467	3.67848703514001\\
15.9481209935276	3.67791830771221\\
15.946736015507	3.6773495430433\\
15.9453508166143	3.67678074112842\\
15.9439653967792	3.67621190196269\\
15.942579755931	3.67564302554122\\
15.9411938939994	3.67507411185914\\
15.9398078109137	3.67450516091156\\
15.9384215066034	3.67393617269361\\
15.9370349809977	3.6733671472004\\
15.9356482340262	3.67279808442704\\
15.9342612656179	3.67222898436865\\
15.9328740757023	3.67165984702034\\
15.9314866642086	3.67109067237723\\
15.9300990310659	3.67052146043443\\
15.9287111762035	3.66995221118705\\
15.9273230995504	3.6693829246302\\
15.9259348010358	3.66881360075898\\
15.9245462805888	3.66824423956852\\
15.9231575381383	3.6676748410539\\
15.9217685736134	3.66710540521025\\
15.920379386943	3.66653593203266\\
15.9189899780561	3.66596642151624\\
15.9176003468815	3.66539687365609\\
15.9162104933482	3.66482728844732\\
15.9148204173849	3.66425766588502\\
15.9134301189204	3.6636880059643\\
15.9120395978836	3.66311830868026\\
15.9106488542031	3.66254857402799\\
15.9092578878076	3.66197880200259\\
15.9078666986258	3.66140899259916\\
15.9064752865863	3.6608391458128\\
15.9050836516177	3.66026926163859\\
15.9036917936485	3.65969934007163\\
15.9022997126073	3.65912938110702\\
15.9009074084225	3.65855938473985\\
15.8995148810226	3.6579893509652\\
15.8981221303361	3.65741927977817\\
15.8967291562912	3.65684917117385\\
15.8953359588164	3.65627902514731\\
15.89394253784	3.65570884169367\\
15.8925488932902	3.65513862080798\\
15.8911550250952	3.65456836248535\\
15.8897609331834	3.65399806672085\\
15.8883666174828	3.65342773350957\\
15.8869720779217	3.65285736284659\\
15.885577314428	3.65228695472699\\
15.8841823269299	3.65171650914585\\
15.8827871153555	3.65114602609826\\
15.8813916796326	3.65057550557928\\
15.8799960196894	3.650004947584\\
15.8786001354537	3.64943435210749\\
15.8772040268534	3.64886371914482\\
15.8758076938164	3.64829304869108\\
15.8744111362705	3.64772234074133\\
15.8730143541435	3.64715159529065\\
15.8716173473631	3.64658081233411\\
15.8702201158572	3.64600999186677\\
15.8688226595533	3.64543913388372\\
15.8674249783791	3.64486823838001\\
15.8660270722623	3.64429730535071\\
15.8646289411304	3.6437263347909\\
15.863230584911	3.64315532669563\\
15.8618320035316	3.64258428105997\\
15.8604331969197	3.64201319787898\\
15.8590341650026	3.64144207714773\\
15.8576349077079	3.64087091886128\\
15.8562354249629	3.64029972301469\\
15.854835716695	3.63972848960302\\
15.8534357828313	3.63915721862132\\
15.8520356232993	3.63858591006466\\
15.8506352380262	3.63801456392809\\
15.849234626939	3.63744318020667\\
15.8478337899651	3.63687175889545\\
15.8464327270315	3.63630029998949\\
15.8450314380653	3.63572880348383\\
15.8436299229936	3.63515726937354\\
15.8422281817434	3.63458569765365\\
15.8408262142416	3.63401408831923\\
15.8394240204154	3.63344244136532\\
15.8380216001914	3.63287075678697\\
15.8366189534967	3.63229903457922\\
15.8352160802581	3.63172727473713\\
15.8338129804024	3.63115547725572\\
15.8324096538563	3.63058364213007\\
15.8310061005466	3.63001176935519\\
15.8296023204	3.62943985892614\\
15.8281983133432	3.62886791083796\\
15.8267940793028	3.62829592508568\\
15.8253896182054	3.62772390166435\\
15.8239849299775	3.627151840569\\
15.8225800145457	3.62657974179467\\
15.8211748718365	3.62600760533641\\
15.8197695017763	3.62543543118923\\
15.8183639042916	3.62486321934818\\
15.8169580793086	3.62429096980829\\
15.8155520267539	3.6237186825646\\
15.8141457465536	3.62314635761212\\
15.8127392386341	3.6225739949459\\
15.8113325029215	3.62200159456097\\
15.8099255393422	3.62142915645234\\
15.8085183478222	3.62085668061506\\
15.8071109282877	3.62028416704414\\
15.8057032806648	3.61971161573461\\
15.8042954048796	3.61913902668149\\
15.802887300858	3.61856639987982\\
15.8014789685261	3.6179937353246\\
15.8000704078098	3.61742103301087\\
15.798661618635	3.61684829293364\\
15.7972526009276	3.61627551508794\\
15.7958433546135	3.61570269946877\\
15.7944338796184	3.61512984607117\\
15.7930241758682	3.61455695489013\\
15.7916142432885	3.61398402592069\\
15.7902040818051	3.61341105915786\\
15.7887936913436	3.61283805459664\\
15.7873830718296	3.61226501223205\\
15.7859722231888	3.61169193205911\\
15.7845611453467	3.61111881407282\\
15.7831498382288	3.61054565826819\\
15.7817383017606	3.60997246464024\\
15.7803265358675	3.60939923318396\\
15.778914540475	3.60882596389437\\
15.7775023155083	3.60825265676647\\
15.776089860893	3.60767931179527\\
15.7746771765541	3.60710592897577\\
15.7732642624171	3.60653250830297\\
15.7718511184072	3.60595904977188\\
15.7704377444495	3.60538555337749\\
15.7690241404691	3.60481201911481\\
15.7676103063913	3.60423844697882\\
15.766196242141	3.60366483696455\\
15.7647819476434	3.60309118906696\\
15.7633674228234	3.60251750328108\\
15.7619526676059	3.60194377960188\\
15.7605376819161	3.60137001802436\\
15.7591224656786	3.60079621854352\\
15.7577070188185	3.60022238115435\\
15.7562913412605	3.59964850585183\\
15.7548754329293	3.59907459263097\\
15.7534592937499	3.59850064148673\\
15.7520429236468	3.59792665241413\\
15.7506263225447	3.59735262540813\\
15.7492094903683	3.59677856046373\\
15.7477924270422	3.59620445757592\\
15.746375132491	3.59563031673967\\
15.7449576066391	3.59505613794996\\
15.7435398494111	3.59448192120179\\
15.7421218607314	3.59390766649013\\
15.7407036405245	3.59333337380996\\
15.7392851887147	3.59275904315626\\
15.7378665052263	3.59218467452401\\
15.7364475899837	3.59161026790818\\
15.7350284429112	3.59103582330375\\
15.733609063933	3.5904613407057\\
15.7321894529732	3.589886820109\\
15.730769609956	3.58931226150862\\
15.7293495348057	3.58873766489954\\
15.7279292274461	3.58816303027672\\
15.7265086878014	3.58758835763513\\
15.7250879157956	3.58701364696975\\
15.7236669113527	3.58643889827554\\
15.7222456743965	3.58586411154747\\
15.7208242048511	3.58528928678051\\
15.7194025026402	3.58471442396961\\
15.7179805676877	3.58413952310975\\
15.7165583999173	3.58356458419588\\
15.7151359992529	3.58298960722297\\
15.713713365618	3.58241459218598\\
15.7122904989365	3.58183953907987\\
15.7108673991319	3.5812644478996\\
15.7094440661278	3.58068931864012\\
15.7080204998478	3.5801141512964\\
15.7065967002153	3.57953894586339\\
15.705172667154	3.57896370233604\\
15.7037484005872	3.5783884207093\\
15.7023239004384	3.57781310097814\\
15.7008991666309	3.5772377431375\\
15.699474199088	3.57666234718233\\
15.6980489977331	3.57608691310758\\
15.6966235624894	3.57551144090821\\
15.6951978932801	3.57493593057916\\
15.6937719900284	3.57436038211537\\
15.6923458526575	3.5737847955118\\
15.6909194810905	3.57320917076338\\
15.6894928752503	3.57263350786506\\
15.6880660350602	3.57205780681179\\
15.686638960443	3.5714820675985\\
15.6852116513217	3.57090629022014\\
15.6837841076192	3.57033047467165\\
15.6823563292585	3.56975462094796\\
15.6809283161623	3.56917872904402\\
15.6795000682534	3.56860279895475\\
15.6780715854547	3.5680268306751\\
15.6766428676888	3.5674508242\\
15.6752139148785	3.56687477952439\\
15.6737847269463	3.56629869664319\\
15.672355303815	3.56572257555135\\
15.670925645407	3.56514641624378\\
15.6694957516449	3.56457021871542\\
15.6680656224513	3.5639939829612\\
15.6666352577485	3.56341770897604\\
15.665204657459	3.56284139675488\\
15.6637738215052	3.56226504629263\\
15.6623427498093	3.56168865758423\\
15.6609114422939	3.5611122306246\\
15.659479898881	3.56053576540865\\
15.6580481194929	3.55995926193132\\
15.6566161040518	3.55938272018752\\
15.6551838524799	3.55880614017217\\
15.6537513646993	3.55822952188019\\
15.6523186406321	3.55765286530649\\
15.6508856802002	3.557076170446\\
15.6494524833257	3.55649943729363\\
15.6480190499306	3.5559226658443\\
15.6465853799367	3.55534585609291\\
15.645151473266	3.55476900803439\\
15.6437173298403	3.55419212166363\\
15.6422829495814	3.55361519697556\\
15.6408483324111	3.55303823396508\\
15.639413478251	3.5524612326271\\
15.637978387023	3.55188419295653\\
15.6365430586486	3.55130711494828\\
15.6351074930494	3.55072999859724\\
15.633671690147	3.55015284389833\\
15.632235649863	3.54957565084645\\
15.6307993721188	3.54899841943649\\
15.629362856836	3.54842114966337\\
15.6279261039358	3.54784384152199\\
15.6264891133398	3.54726649500723\\
15.6250518849691	3.546689110114\\
15.6236144187452	3.5461116868372\\
15.6221767145893	3.54553422517172\\
15.6207387724226	3.54495672511247\\
15.6193005921663	3.54437918665432\\
15.6178621737416	3.54380160979218\\
15.6164235170695	3.54322399452095\\
15.614984622071	3.54264634083549\\
15.6135454886674	3.54206864873073\\
15.6121061167794	3.54149091820152\\
15.6106665063281	3.54091314924278\\
15.6092266572344	3.54033534184938\\
15.6077865694191	3.53975749601621\\
15.606346242803	3.53917961173816\\
15.6049056773071	3.53860168901011\\
15.6034648728519	3.53802372782694\\
15.6020238293582	3.53744572818353\\
15.6005825467468	3.53686769007477\\
15.5991410249381	3.53628961349554\\
15.5976992638529	3.53571149844071\\
15.5962572634116	3.53513334490517\\
15.5948150235348	3.53455515288378\\
15.593372544143	3.53397692237143\\
15.5919298251565	3.53339865336299\\
15.5904868664958	3.53282034585333\\
15.5890436680812	3.53224199983734\\
15.5876002298331	3.53166361530987\\
15.5861565516717	3.5310851922658\\
15.5847126335173	3.5305067307\\
15.58326847529	3.52992823060733\\
15.58182407691	3.52934969198268\\
15.5803794382974	3.52877111482089\\
15.5789345593723	3.52819249911685\\
15.5774894400548	3.52761384486541\\
15.5760440802648	3.52703515206143\\
15.5745984799222	3.52645642069978\\
15.5731526389471	3.52587765077533\\
15.5717065572592	3.52529884228292\\
15.5702602347785	3.52471999521743\\
15.5688136714247	3.5241411095737\\
15.5673668671175	3.5235621853466\\
15.5659198217767	3.52298322253098\\
15.5644725353219	3.52240422112171\\
15.5630250076729	3.52182518111362\\
};
\addplot [color=mycolor1, forget plot]
  table[row sep=crcr]{%
15.5630250076729	3.52182518111362\\
15.5615772387491	3.52124610250158\\
15.5601292284702	3.52066698528044\\
15.5586809767556	3.52008782944505\\
15.5572324835248	3.51950863499026\\
15.5557837486973	3.51892940191092\\
15.5543347721925	3.51835013020187\\
15.5528855539297	3.51777081985797\\
15.5514360938282	3.51719147087406\\
15.5499863918073	3.51661208324498\\
15.5485364477862	3.51603265696559\\
15.5470862616842	3.51545319203071\\
15.5456358334203	3.5148736884352\\
15.5441851629137	3.5142941461739\\
15.5427342500835	3.51371456524164\\
15.5412830948486	3.51313494563327\\
15.5398316971281	3.51255528734362\\
15.5383800568409	3.51197559036752\\
15.536928173906	3.51139585469983\\
15.5354760482421	3.51081608033536\\
15.5340236797682	3.51023626726895\\
15.532571068403	3.50965641549544\\
15.5311182140652	3.50907652500966\\
15.5296651166736	3.50849659580643\\
15.5282117761468	3.5079166278806\\
15.5267581924035	3.50733662122698\\
15.5253043653622	3.5067565758404\\
15.5238502949414	3.50617649171569\\
15.5223959810598	3.50559636884768\\
15.5209414236356	3.50501620723119\\
15.5194866225874	3.50443600686104\\
15.5180315778335	3.50385576773206\\
15.5165762892922	3.50327548983907\\
15.515120756882	3.50269517317688\\
15.5136649805209	3.50211481774032\\
15.5122089601272	3.5015344235242\\
15.5107526956191	3.50095399052335\\
15.5092961869148	3.50037351873257\\
15.5078394339322	3.49979300814668\\
15.5063824365896	3.49921245876051\\
15.5049251948047	3.49863187056885\\
15.5034677084957	3.49805124356652\\
15.5020099775805	3.49747057774834\\
15.5005520019769	3.49688987310911\\
15.4990937816028	3.49630912964364\\
15.4976353163759	3.49572834734674\\
15.4961766062141	3.49514752621321\\
15.4947176510351	3.49456666623787\\
15.4932584507564	3.49398576741551\\
15.4917990052959	3.49340482974094\\
15.490339314571	3.49282385320897\\
15.4888793784993	3.49224283781438\\
15.4874191969983	3.491661783552\\
15.4859587699855	3.4910806904166\\
15.4844980973784	3.490499558403\\
15.4830371790942	3.48991838750599\\
15.4815760150504	3.48933717772036\\
15.4801146051642	3.48875592904091\\
15.4786529493529	3.48817464146244\\
15.4771910475337	3.48759331497974\\
15.4757288996239	3.4870119495876\\
15.4742665055404	3.48643054528081\\
15.4728038652005	3.48584910205416\\
15.4713409785212	3.48526761990244\\
15.4698778454194	3.48468609882044\\
15.4684144658122	3.48410453880294\\
15.4669508396165	3.48352293984473\\
15.4654869667491	3.48294130194061\\
15.4640228471269	3.48235962508534\\
15.4625584806667	3.48177790927371\\
15.4610938672853	3.48119615450051\\
15.4596290068993	3.48061436076051\\
15.4581638994254	3.48003252804849\\
15.4566985447804	3.47945065635924\\
15.4552329428806	3.47886874568753\\
15.4537670936428	3.47828679602813\\
15.4523009969834	3.47770480737583\\
15.4508346528189	3.47712277972539\\
15.4493680610656	3.47654071307159\\
15.44790122164	3.4759586074092\\
15.4464341344584	3.475376462733\\
15.4449667994371	3.47479427903775\\
15.4434992164923	3.47421205631821\\
15.4420313855402	3.47362979456917\\
15.4405633064971	3.47304749378539\\
15.439094979279	3.47246515396163\\
15.437626403802	3.47188277509265\\
15.4361575799821	3.47130035717323\\
15.4346885077354	3.47071790019812\\
15.4332191869777	3.47013540416209\\
15.4317496176251	3.46955286905989\\
15.4302797995933	3.46897029488629\\
15.4288097327982	3.46838768163604\\
15.4273394171556	3.4678050293039\\
15.4258688525812	3.46722233788463\\
15.4243980389907	3.46663960737299\\
15.4229269762997	3.46605683776372\\
15.4214556644239	3.46547402905159\\
15.4199841032788	3.46489118123133\\
15.41851229278	3.46430829429771\\
15.4170402328429	3.46372536824548\\
15.415567923383	3.46314240306938\\
15.4140953643156	3.46255939876415\\
15.4126225555561	3.46197635532456\\
15.4111494970199	3.46139327274534\\
15.4096761886222	3.46081015102125\\
15.4082026302781	3.46022699014701\\
15.406728821903	3.45964379011737\\
15.4052547634119	3.45906055092709\\
15.4037804547199	3.45847727257089\\
15.402305895742	3.45789395504352\\
15.4008310863934	3.45731059833971\\
15.3993560265888	3.45672720245421\\
15.3978807162434	3.45614376738174\\
15.3964051552718	3.45556029311705\\
15.3949293435891	3.45497677965487\\
15.3934532811098	3.45439322698993\\
15.3919769677489	3.45380963511696\\
15.390500403421	3.45322600403069\\
15.3890235880408	3.45264233372586\\
15.3875465215228	3.4520586241972\\
15.3860692037816	3.45147487543942\\
15.3845916347319	3.45089108744726\\
15.383113814288	3.45030726021544\\
15.3816357423644	3.4497233937387\\
15.3801574188755	3.44913948801174\\
15.3786788437356	3.4485555430293\\
15.3772000168591	3.44797155878609\\
15.3757209381603	3.44738753527684\\
15.3742416075533	3.44680347249626\\
15.3727620249523	3.44621937043908\\
15.3712821902715	3.44563522910001\\
15.3698021034249	3.44505104847376\\
15.3683217643266	3.44446682855506\\
15.3668411728907	3.44388256933861\\
15.365360329031	3.44329827081913\\
15.3638792326614	3.44271393299133\\
15.3623978836959	3.44212955584992\\
15.3609162820483	3.44154513938961\\
15.3594344276324	3.44096068360511\\
15.3579523203618	3.44037618849113\\
15.3564699601503	3.43979165404237\\
15.3549873469116	3.43920708025354\\
15.3535044805592	3.43862246711934\\
15.3520213610067	3.43803781463448\\
15.3505379881676	3.43745312279365\\
15.3490543619555	3.43686839159157\\
15.3475704822836	3.43628362102293\\
15.3460863490655	3.43569881108242\\
15.3446019622144	3.43511396176475\\
15.3431173216436	3.43452907306462\\
15.3416324272664	3.43394414497671\\
15.3401472789961	3.43335917749573\\
15.3386618767457	3.43277417061637\\
15.3371762204284	3.43218912433332\\
15.3356903099572	3.43160403864126\\
15.3342041452452	3.43101891353491\\
15.3327177262054	3.43043374900893\\
15.3312310527506	3.42984854505802\\
15.3297441247939	3.42926330167687\\
15.328256942248	3.42867801886016\\
15.3267695050257	3.42809269660258\\
15.3252818130399	3.42750733489881\\
15.3237938662032	3.42692193374354\\
15.3223056644283	3.42633649313145\\
15.3208172076278	3.42575101305721\\
15.3193284957143	3.42516549351551\\
15.3178395286003	3.42457993450102\\
15.3163503061984	3.42399433600843\\
15.3148608284209	3.42340869803241\\
15.3133710951803	3.42282302056764\\
15.3118811063889	3.42223730360878\\
15.3103908619591	3.42165154715052\\
15.308900361803	3.42106575118752\\
15.307409605833	3.42047991571446\\
15.3059185939611	3.419894040726\\
15.3044273260996	3.41930812621682\\
15.3029358021605	3.41872217218158\\
15.3014440220558	3.41813617861495\\
15.2999519856976	3.4175501455116\\
15.2984596929978	3.41696407286618\\
15.2969671438682	3.41637796067337\\
15.2954743382208	3.41579180892783\\
15.2939812759674	3.41520561762421\\
15.2924879570196	3.41461938675718\\
15.2909943812893	3.41403311632139\\
15.2895005486882	3.41344680631151\\
15.2880064591278	3.4128604567222\\
15.2865121125197	3.4122740675481\\
15.2850175087755	3.41168763878387\\
15.2835226478066	3.41110117042417\\
15.2820275295246	3.41051466246364\\
15.2805321538407	3.40992811489695\\
15.2790365206665	3.40934152771874\\
15.2775406299131	3.40875490092366\\
15.2760444814919	3.40816823450636\\
15.274548075314	3.40758152846148\\
15.2730514112906	3.40699478278368\\
15.2715544893329	3.40640799746759\\
15.2700573093519	3.40582117250786\\
15.2685598712587	3.40523430789914\\
15.2670621749643	3.40464740363607\\
15.2655642203796	3.40406045971328\\
15.2640660074154	3.40347347612542\\
15.2625675359827	3.40288645286712\\
15.2610688059923	3.40229938993303\\
15.2595698173549	3.40171228731778\\
15.2580705699811	3.401125145016\\
15.2565710637818	3.40053796302233\\
15.2550712986675	3.3999507413314\\
15.2535712745487	3.39936347993784\\
15.2520709913361	3.39877617883629\\
15.25057044894	3.39818883802138\\
15.2490696472709	3.39760145748773\\
15.2475685862393	3.39701403722997\\
15.2460672657554	3.39642657724272\\
15.2445656857295	3.39583907752062\\
15.2430638460719	3.39525153805829\\
15.2415617466928	3.39466395885035\\
15.2400593875023	3.39407633989143\\
15.2385567684106	3.39348868117613\\
15.2370538893277	3.3929009826991\\
15.2355507501636	3.39231324445493\\
15.2340473508283	3.39172546643826\\
15.2325436912317	3.39113764864369\\
15.2310397712836	3.39054979106585\\
15.229535590894	3.38996189369935\\
15.2280311499726	3.3893739565388\\
15.2265264484291	3.38878597957882\\
15.2250214861733	3.38819796281401\\
15.2235162631146	3.38760990623899\\
15.2220107791628	3.38702180984836\\
15.2205050342275	3.38643367363673\\
15.2189990282179	3.38584549759872\\
15.2174927610438	3.38525728172892\\
15.2159862326144	3.38466902602194\\
15.214479442839	3.38408073047239\\
15.2129723916271	3.38349239507486\\
15.2114650788879	3.38290401982396\\
15.2099575045305	3.38231560471429\\
15.2084496684642	3.38172714974045\\
15.2069415705981	3.38113865489703\\
15.2054332108413	3.38055012017864\\
15.2039245891027	3.37996154557987\\
15.2024157052914	3.37937293109531\\
15.2009065593162	3.37878427671956\\
15.1993971510861	3.37819558244721\\
15.19788748051	3.37760684827285\\
15.1963775474965	3.37701807419108\\
15.1948673519545	3.37642926019647\\
15.1933568937926	3.37584040628363\\
15.1918461729195	3.37525151244713\\
15.1903351892438	3.37466257868157\\
15.188823942674	3.37407360498153\\
15.1873124331186	3.37348459134159\\
15.1858006604861	3.37289553775633\\
15.1842886246849	3.37230644422034\\
15.1827763256233	3.3717173107282\\
15.1812637632097	3.37112813727448\\
15.1797509373524	3.37053892385378\\
15.1782378479595	3.36994967046065\\
15.1767244949392	3.36936037708968\\
15.1752108781996	3.36877104373544\\
15.1736969976488	3.36818167039251\\
15.1721828531949	3.36759225705547\\
15.1706684447457	3.36700280371887\\
15.1691537722093	3.3664133103773\\
15.1676388354935	3.36582377702532\\
15.1661236345061	3.3652342036575\\
15.164608169155	3.36464459026841\\
15.1630924393478	3.36405493685261\\
15.1615764449923	3.36346524340467\\
15.160060185996	3.36287550991916\\
15.1585436622666	3.36228573639063\\
15.1570268737116	3.36169592281365\\
15.1555098202385	3.36110606918278\\
15.1539925017548	3.36051617549257\\
15.1524749181678	3.3599262417376\\
15.1509570693849	3.3593362679124\\
15.1494389553134	3.35874625401155\\
15.1479205758605	3.35815620002959\\
15.1464019309334	3.35756610596109\\
15.1448830204394	3.35697597180058\\
15.1433638442855	3.35638579754263\\
15.1418444023787	3.35579558318178\\
15.140324694626	3.35520532871259\\
15.1388047209345	3.3546150341296\\
15.137284481211	3.35402469942736\\
15.1357639753624	3.35343432460041\\
15.1342432032954	3.35284390964331\\
15.132722164917	3.35225345455059\\
15.1312008601337	3.35166295931681\\
15.1296792888522	3.35107242393649\\
15.1281574509792	3.35048184840418\\
15.1266353464212	3.34989123271442\\
15.1251129750847	3.34930057686175\\
15.1235903368762	3.34870988084071\\
15.1220674317021	3.34811914464583\\
15.1205442594688	3.34752836827165\\
15.1190208200826	3.3469375517127\\
15.1174971134498	3.34634669496351\\
15.1159731394766	3.34575579801862\\
15.1144488980692	3.34516486087256\\
15.1129243891336	3.34457388351985\\
15.111399612576	3.34398286595503\\
15.1098745683024	3.34339180817262\\
15.1083492562187	3.34280071016715\\
15.106823676231	3.34220957193314\\
15.1052978282449	3.34161839346511\\
15.1037717121665	3.3410271747576\\
15.1022453279014	3.34043591580512\\
15.1007186753554	3.33984461660219\\
15.0991917544342	3.33925327714334\\
15.0976645650434	3.33866189742307\\
15.0961371070885	3.33807047743591\\
15.0946093804751	3.33747901717637\\
15.0930813851086	3.33688751663897\\
15.0915531208946	3.33629597581823\\
15.0900245877383	3.33570439470865\\
15.0884957855452	3.33511277330474\\
15.0869667142204	3.33452111160102\\
15.0854373736692	3.333929409592\\
15.0839077637968	3.33333766727219\\
15.0823778845083	3.33274588463608\\
15.0808477357087	3.3321540616782\\
15.0793173173032	3.33156219839303\\
15.0777866291967	3.33097029477509\\
15.076255671294	3.33037835081889\\
15.0747244435002	3.32978636651891\\
15.07319294572	3.32919434186966\\
15.0716611778581	3.32860227686564\\
15.0701291398194	3.32801017150136\\
15.0685968315085	3.32741802577129\\
15.0670642528299	3.32682583966995\\
15.0655314036884	3.32623361319182\\
15.0639982839883	3.32564134633141\\
15.0624648936343	3.3250490390832\\
15.0609312325306	3.32445669144168\\
15.0593973005817	3.32386430340134\\
15.0578630976919	3.32327187495668\\
15.0563286237654	3.32267940610217\\
15.0547938787066	3.32208689683232\\
15.0532588624194	3.3214943471416\\
15.0517235748082	3.32090175702449\\
15.0501880157769	3.32030912647549\\
15.0486521852295	3.31971645548908\\
15.04711608307	3.31912374405972\\
15.0455797092024	3.31853099218192\\
15.0440430635304	3.31793819985014\\
15.042506145958	3.31734536705886\\
15.0409689563888	3.31675249380255\\
15.0394314947265	3.31615958007571\\
15.0378937608749	3.31556662587279\\
15.0363557547376	3.31497363118828\\
15.034817476218	3.31438059601663\\
15.0332789252197	3.31378752035234\\
15.0317401016462	3.31319440418986\\
15.0302010054008	3.31260124752366\\
15.0286616363869	3.31200805034822\\
15.0271219945079	3.31141481265799\\
15.0255820796668	3.31082153444745\\
15.0240418917671	3.31022821571106\\
15.0225014307117	3.30963485644327\\
15.0209606964038	3.30904145663856\\
15.0194196887464	3.30844801629139\\
15.0178784076425	3.3078545353962\\
15.0163368529951	3.30726101394747\\
15.014795024707	3.30666745193965\\
15.0132529226811	3.3060738493672\\
15.0117105468202	3.30548020622456\\
15.0101678970269	3.3048865225062\\
15.008624973204	3.30429279820657\\
15.0070817752542	3.30369903332012\\
15.0055383030799	3.3031052278413\\
15.0039945565836	3.30251138176456\\
15.002450535668	3.30191749508435\\
15.0009062402354	3.30132356779511\\
14.9993616701881	3.3007295998913\\
14.9978168254284	3.30013559136735\\
14.9962717058588	3.29954154221771\\
14.9947263113812	3.29894745243682\\
14.993180641898	3.29835332201913\\
14.9916346973112	3.29775915095907\\
14.9900884775228	3.29716493925109\\
14.988541982435	3.29657068688961\\
14.9869952119495	3.29597639386909\\
14.9854481659684	3.29538206018395\\
14.9839008443934	3.29478768582863\\
14.9823532471264	3.29419327079757\\
14.9808053740691	3.29359881508519\\
14.9792572251232	3.29300431868592\\
14.9777088001903	3.2924097815942\\
14.976160099172	3.29181520380446\\
14.9746111219699	3.29122058531112\\
14.9730618684853	3.29062592610861\\
14.9715123386198	3.29003122619136\\
14.9699625322747	3.28943648555378\\
14.9684124493514	3.28884170419031\\
14.966862089751	3.28824688209536\\
14.9653114533748	3.28765201926336\\
14.963760540124	3.28705711568873\\
14.9622093498997	3.28646217136588\\
14.9606578826029	3.28586718628923\\
14.9591061381346	3.2852721604532\\
14.9575541163958	3.2846770938522\\
14.9560018172874	3.28408198648065\\
14.9544492407102	3.28348683833296\\
14.952896386565	3.28289164940354\\
14.9513432547525	3.2822964196868\\
14.9497898451735	3.28170114917716\\
14.9482361577285	3.28110583786902\\
14.9466821923181	3.28051048575678\\
14.9451279488429	3.27991509283486\\
14.9435734272033	3.27931965909765\\
14.9420186272997	3.27872418453957\\
14.9404635490326	3.27812866915501\\
14.9389081923021	3.27753311293838\\
14.9373525570086	3.27693751588408\\
14.9357966430522	3.2763418779865\\
14.9342404503332	3.27574619924004\\
14.9326839787516	3.2751504796391\\
14.9311272282074	3.27455471917808\\
14.9295701986006	3.27395891785137\\
14.9280128898312	3.27336307565336\\
14.9264553017991	3.27276719257844\\
14.924897434404	3.27217126862102\\
14.9233392875457	3.27157530377547\\
14.921780861124	3.27097929803618\\
14.9202221550385	3.27038325139755\\
14.9186631691889	3.26978716385395\\
14.9171039034746	3.26919103539979\\
14.9155443577952	3.26859486602942\\
14.9139845320501	3.26799865573726\\
14.9124244261388	3.26740240451766\\
14.9108640399605	3.26680611236502\\
14.9093033734145	3.2662097792737\\
14.9077424264002	3.26561340523811\\
14.9061811988166	3.2650169902526\\
14.9046196905628	3.26442053431155\\
14.903057901538	3.26382403740934\\
14.9014958316411	3.26322749954035\\
14.8999334807712	3.26263092069894\\
14.8983708488271	3.26203430087949\\
14.8968079357076	3.26143764007637\\
14.8952447413116	3.26084093828394\\
14.8936812655377	3.26024419549657\\
14.8921175082848	3.25964741170864\\
14.8905534694513	3.2590505869145\\
14.888989148936	3.25845372110852\\
14.8874245466372	3.25785681428507\\
14.8858596624535	3.2572598664385\\
14.8842944962833	3.25666287756318\\
14.8827290480249	3.25606584765347\\
14.8811633175767	3.25546877670372\\
14.8795973048369	3.25487166470829\\
14.8780310097036	3.25427451166154\\
14.876464432075	3.25367731755782\\
14.8748975718492	3.2530800823915\\
14.8733304289243	3.25248280615691\\
14.8717630031981	3.25188548884841\\
14.8701952945686	3.25128813046036\\
14.8686273029337	3.2506907309871\\
14.8670590281911	3.25009329042298\\
14.8654904702387	3.24949580876234\\
14.863921628974	3.24889828599954\\
14.8623525042948	3.24830072212891\\
14.8607830960987	3.2477031171448\\
14.8592134042831	3.24710547104156\\
14.8576434287455	3.24650778381352\\
14.8560731693833	3.24591005545502\\
14.854502626094	3.24531228596041\\
14.8529317987747	3.24471447532401\\
14.8513606873228	3.24411662354018\\
14.8497892916355	3.24351873060323\\
14.8482176116099	3.24292079650751\\
14.846645647143	3.24232282124734\\
14.8450733981319	3.24172480481707\\
14.8435008644735	3.24112674721102\\
14.8419280460649	3.24052864842352\\
14.8403549428028	3.23993050844889\\
14.838781554584	3.23933232728148\\
14.8372078813053	3.23873410491559\\
14.8356339228633	3.23813584134556\\
14.8340596791548	3.2375375365657\\
14.8324851500762	3.23693919057035\\
14.8309103355242	3.23634080335383\\
14.8293352353951	3.23574237491044\\
14.8277598495854	3.23514390523452\\
14.8261841779914	3.23454539432038\\
14.8246082205094	3.23394684216233\\
14.8230319770357	3.23334824875469\\
14.8214554474664	3.23274961409178\\
14.8198786316977	3.23215093816791\\
14.8183015296257	3.23155222097739\\
14.8167241411462	3.23095346251453\\
14.8151464661554	3.23035466277363\\
14.8135685045491	3.22975582174902\\
14.8119902562231	3.22915693943499\\
14.8104117210733	3.22855801582585\\
14.8088328989953	3.22795905091591\\
14.8072537898849	3.22736004469947\\
14.8056743936376	3.22676099717083\\
14.804094710149	3.2261619083243\\
14.8025147393146	3.22556277815416\\
14.8009344810299	3.22496360665473\\
14.7993539351902	3.2243643938203\\
14.7977731016909	3.22376513964516\\
14.7961919804272	3.22316584412362\\
14.7946105712944	3.22256650724996\\
14.7930288741875	3.22196712901848\\
14.7914468890018	3.22136770942346\\
14.7898646156323	3.22076824845921\\
14.7882820539739	3.22016874612001\\
14.7866992039216	3.21956920240015\\
14.7851160653702	3.21896961729391\\
14.7835326382146	3.21836999079558\\
14.7819489223495	3.21777032289945\\
14.7803649176696	3.2171706135998\\
14.7787806240696	3.21657086289091\\
14.777196041444	3.21597107076706\\
14.7756111696874	3.21537123722253\\
14.7740260086942	3.21477136225161\\
14.7724405583589	3.21417144584856\\
14.7708548185757	3.21357148800767\\
14.769268789239	3.21297148872321\\
14.7676824702431	3.21237144798945\\
14.7660958614821	3.21177136580067\\
14.7645089628501	3.21117124215113\\
14.7629217742412	3.21057107703512\\
14.7613342955494	3.20997087044689\\
14.7597465266686	3.20937062238072\\
14.7581584674928	3.20877033283087\\
14.7565701179157	3.20817000179161\\
14.7549814778311	3.20756962925721\\
14.7533925471328	3.20696921522191\\
14.7518033257144	3.20636875968\\
14.7502138134695	3.20576826262573\\
14.7486240102916	3.20516772405336\\
14.7470339160743	3.20456714395714\\
14.7454435307109	3.20396652233134\\
14.7438528540947	3.20336585917021\\
14.7422618861192	3.202765154468\\
14.7406706266776	3.20216440821898\\
14.7390790756629	3.20156362041738\\
14.7374872329685	3.20096279105748\\
14.7358950984872	3.2003619201335\\
14.7343026721122	3.19976100763971\\
14.7327099537364	3.19916005357034\\
14.7311169432527	3.19855905791966\\
14.7295236405539	3.1979580206819\\
14.7279300455328	3.1973569418513\\
14.7263361580822	3.19675582142211\\
14.7247419780946	3.19615465938858\\
14.7231475054626	3.19555345574493\\
14.7215527400789	3.19495221048542\\
14.7199576818359	3.19435092360428\\
14.7183623306259	3.19374959509574\\
14.7167666863415	3.19314822495405\\
14.7151707488748	3.19254681317344\\
14.7135745181181	3.19194535974813\\
14.7119779939636	3.19134386467237\\
14.7103811763034	3.19074232794038\\
14.7087840650296	3.1901407495464\\
14.7071866600342	3.18953912948465\\
14.7055889612091	3.18893746774935\\
14.7039909684463	3.18833576433475\\
14.7023926816374	3.18773401923505\\
14.7007941006744	3.18713223244449\\
14.6991952254489	3.18653040395729\\
14.6975960558526	3.18592853376767\\
14.6959965917769	3.18532662186984\\
14.6943968331136	3.18472466825804\\
14.692796779754	3.18412267292647\\
14.6911964315895	3.18352063586935\\
14.6895957885115	3.18291855708091\\
14.6879948504113	3.18231643655534\\
14.6863936171801	3.18171427428687\\
14.6847920887091	3.18111207026971\\
14.6831902648893	3.18050982449806\\
14.6815881456119	3.17990753696615\\
14.6799857307677	3.17930520766816\\
14.6783830202478	3.17870283659833\\
14.676780013943	3.17810042375083\\
14.6751767117441	3.17749796911989\\
14.6735731135418	3.17689547269971\\
14.6719692192268	3.17629293448449\\
14.6703650286898	3.17569035446843\\
14.6687605418212	3.17508773264572\\
14.6671557585117	3.17448506901057\\
14.6655506786516	3.17388236355718\\
14.6639453021314	3.17327961627974\\
14.6623396288413	3.17267682717245\\
14.6607336586716	3.17207399622949\\
14.6591273915125	3.17147112344507\\
14.6575208272541	3.17086820881337\\
14.6559139657866	3.17026525232859\\
14.6543068069999	3.16966225398491\\
14.6526993507839	3.16905921377652\\
14.6510915970286	3.1684561316976\\
14.6494835456239	3.16785300774235\\
14.6478751964594	3.16724984190495\\
14.6462665494249	3.16664663417957\\
14.64465760441	3.16604338456041\\
14.6430483613043	3.16544009304163\\
14.6414388199973	3.16483675961742\\
14.6398289803786	3.16423338428197\\
14.6382188423374	3.16362996702944\\
14.6366084057632	3.163026507854\\
14.6349976705453	3.16242300674985\\
14.6333866365727	3.16181946371114\\
14.6317753037348	3.16121587873205\\
14.6301636719205	3.16061225180676\\
14.628551741019	3.16000858292942\\
14.6269395109191	3.15940487209421\\
14.6253269815098	3.1588011192953\\
14.62371415268	3.15819732452685\\
14.6221010243184	3.15759348778302\\
14.6204875963138	3.15698960905798\\
14.6188738685547	3.1563856883459\\
14.6172598409299	3.15578172564093\\
14.6156455133278	3.15517772093722\\
14.6140308856369	3.15457367422895\\
14.6124159577457	3.15396958551027\\
14.6108007295424	3.15336545477533\\
14.6091852009154	3.1527612820183\\
14.6075693717529	3.15215706723331\\
14.605953241943	3.15155281041453\\
14.6043368113739	3.1509485115561\\
14.6027200799336	3.15034417065218\\
14.60110304751	3.14973978769692\\
14.5994857139911	3.14913536268446\\
14.5978680792648	3.14853089560894\\
14.5962501432187	3.14792638646453\\
14.5946319057407	3.14732183524535\\
14.5930133667184	3.14671724194555\\
14.5913945260394	3.14611260655927\\
14.5897753835912	3.14550792908065\\
14.5881559392614	3.14490320950384\\
14.5865361929372	3.14429844782296\\
14.5849161445061	3.14369364403216\\
14.5832957938554	3.14308879812557\\
14.5816751408722	3.14248391009733\\
14.5800541854438	3.14187897994156\\
14.5784329274572	3.1412740076524\\
14.5768113667994	3.14066899322398\\
14.5751895033575	3.14006393665044\\
14.5735673370183	3.13945883792588\\
14.5719448676686	3.13885369704445\\
14.5703220951953	3.13824851400028\\
14.5686990194851	3.13764328878747\\
14.5670756404246	3.13703802140016\\
14.5654519579003	3.13643271183247\\
14.5638279717989	3.13582736007852\\
14.5622036820068	3.13522196613242\\
14.5605790884104	3.1346165299883\\
14.5589541908959	3.13401105164028\\
14.5573289893498	3.13340553108246\\
14.5557034836581	3.13279996830897\\
14.5540776737071	3.13219436331392\\
14.5524515593827	3.13158871609141\\
14.5508251405711	3.13098302663556\\
14.5491984171582	3.13037729494048\\
14.5475713890298	3.12977152100028\\
14.5459440560717	3.12916570480906\\
14.5443164181699	3.12855984636093\\
14.5426884752098	3.12795394565\\
14.5410602270771	3.12734800267036\\
14.5394316736575	3.12674201741612\\
14.5378028148364	3.12613598988139\\
14.5361736504993	3.12552992006026\\
14.5345441805314	3.12492380794682\\
14.5329144048182	3.12431765353518\\
14.5312843232449	3.12371145681943\\
14.5296539356966	3.12310521779367\\
14.5280232420584	3.12249893645199\\
14.5263922422155	3.12189261278848\\
14.5247609360528	3.12128624679723\\
14.5231293234551	3.12067983847234\\
14.5214974043074	3.12007338780789\\
14.5198651784944	3.11946689479797\\
14.518232645901	3.11886035943666\\
14.5165998064116	3.11825378171805\\
14.514966659911	3.11764716163623\\
14.5133332062836	3.11704049918527\\
14.5116994454139	3.11643379435926\\
14.5100653771863	3.11582704715227\\
14.5084310014852	3.11522025755839\\
14.5067963181947	3.11461342557169\\
14.5051613271992	3.11400655118625\\
14.5035260283827	3.11339963439615\\
14.5018904216294	3.11279267519545\\
14.5002545068231	3.11218567357823\\
14.498618283848	3.11157862953856\\
14.4969817525877	3.11097154307051\\
14.4953449129262	3.11036441416815\\
14.4937077647472	3.10975724282555\\
14.4920703079343	3.10915002903677\\
14.4904325423713	3.10854277279587\\
14.4887944679415	3.10793547409693\\
14.4871560845285	3.107328132934\\
14.4855173920158	3.10672074930115\\
14.4838783902866	3.10611332319243\\
14.4822390792242	3.10550585460191\\
14.480599458712	3.10489834352364\\
14.4789595286329	3.10429078995167\\
14.4773192888701	3.10368319388006\\
14.4756787393066	3.10307555530288\\
14.4740378798254	3.10246787421416\\
14.4723967103093	3.10186015060796\\
14.4707552306411	3.10125238447833\\
14.4691134407037	3.10064457581931\\
14.4674713403797	3.10003672462497\\
14.4658289295517	3.09942883088933\\
14.4641862081022	3.09882089460645\\
14.4625431759138	3.09821291577038\\
14.4608998328689	3.09760489437514\\
14.4592561788498	3.09699683041479\\
14.4576122137388	3.09638872388337\\
14.4559679374181	3.0957805747749\\
14.4543233497699	3.09517238308344\\
14.4526784506762	3.09456414880302\\
14.4510332400192	3.09395587192766\\
14.4493877176806	3.09334755245141\\
14.4477418835425	3.0927391903683\\
14.4460957374866	3.09213078567236\\
14.4444492793946	3.09152233835761\\
14.4428025091483	3.0909138484181\\
14.4411554266293	3.09030531584784\\
14.4395080317191	3.08969674064086\\
14.4378603242991	3.08908812279118\\
14.4362123042509	3.08847946229284\\
14.4345639714558	3.08787075913984\\
14.4329153257949	3.08726201332622\\
14.4312663671496	3.086653224846\\
14.429617095401	3.08604439369318\\
14.4279675104301	3.0854355198618\\
14.426317612118	3.08482660334586\\
14.4246674003455	3.08421764413938\\
14.4230168749937	3.08360864223637\\
14.4213660359432	3.08299959763085\\
14.4197148830748	3.08239051031683\\
14.4180634162691	3.08178138028831\\
14.4164116354069	3.08117220753931\\
14.4147595403685	3.08056299206383\\
14.4131071310345	3.07995373385588\\
14.4114544072852	3.07934443290946\\
14.409801369001	3.07873508921858\\
14.4081480160622	3.07812570277724\\
14.4064943483488	3.07751627357945\\
14.4048403657411	3.07690680161919\\
14.4031860681191	3.07629728689047\\
14.4015314553628	3.07568772938728\\
14.3998765273521	3.07507812910363\\
14.3982212839668	3.0744684860335\\
14.3965657250867	3.07385880017089\\
14.3949098505915	3.0732490715098\\
14.3932536603609	3.0726393000442\\
14.3915971542745	3.0720294857681\\
14.3899403322116	3.07141962867548\\
14.3882831940519	3.07080972876033\\
14.3866257396745	3.07019978601663\\
14.3849679689589	3.06958980043837\\
14.3833098817843	3.06897977201953\\
14.3816514780297	3.06836970075409\\
14.3799927575744	3.06775958663604\\
14.3783337202972	3.06714942965935\\
14.3766743660772	3.06653922981801\\
14.3750146947933	3.06592898710598\\
14.3733547063242	3.06531870151724\\
14.3716944005487	3.06470837304578\\
14.3700337773455	3.06409800168556\\
14.3683728365931	3.06348758743055\\
14.3667115781701	3.06287713027472\\
14.365050001955	3.06226663021205\\
14.3633881078261	3.0616560872365\\
14.3617258956618	3.06104550134204\\
14.3600633653404	3.06043487252263\\
14.3584005167399	3.05982420077225\\
14.3567373497385	3.05921348608484\\
14.3550738642143	3.05860272845437\\
14.3534100600452	3.05799192787481\\
14.3517459371092	3.05738108434011\\
14.350081495284	3.05677019784423\\
14.3484167344475	3.05615926838113\\
14.3467516544773	3.05554829594476\\
14.345086255251	3.05493728052908\\
14.3434205366462	3.05432622212803\\
14.3417544985404	3.05371512073558\\
14.3400881408109	3.05310397634566\\
14.3384214633352	3.05249278895224\\
14.3367544659905	3.05188155854925\\
14.3350871486539	3.05127028513064\\
14.3334195112027	3.05065896869037\\
14.3317515535138	3.05004760922236\\
14.3300832754643	3.04943620672057\\
14.3284146769311	3.04882476117894\\
14.326745757791	3.0482132725914\\
14.3250765179207	3.0476017409519\\
14.3234069571971	3.04699016625437\\
14.3217370754966	3.04637854849275\\
14.320066872696	3.04576688766097\\
14.3183963486716	3.04515518375297\\
14.3167255032999	3.04454343676268\\
14.3150543364572	3.04393164668403\\
14.3133828480198	3.04331981351094\\
14.3117110378639	3.04270793723736\\
14.3100389058657	3.0420960178572\\
14.3083664519011	3.04148405536439\\
14.3066936758463	3.04087204975286\\
14.305020577577	3.04026000101653\\
14.3033471569692	3.03964790914931\\
14.3016734138985	3.03903577414514\\
14.2999993482408	3.03842359599793\\
14.2983249598717	3.03781137470161\\
14.2966502486667	3.03719911025007\\
14.2949752145012	3.03658680263725\\
14.2932998572507	3.03597445185706\\
14.2916241767906	3.0353620579034\\
14.2899481729961	3.0347496207702\\
14.2882718457424	3.03413714045135\\
14.2865951949047	3.03352461694078\\
14.2849182203579	3.03291205023238\\
14.2832409219771	3.03229944032007\\
14.2815632996371	3.03168678719776\\
14.2798853532129	3.03107409085933\\
14.2782070825791	3.0304613512987\\
14.2765284876105	3.02984856850977\\
14.2748495681817	3.02923574248644\\
14.2731703241671	3.02862287322261\\
14.2714907554414	3.02800996071217\\
14.2698108618788	3.02739700494902\\
14.2681306433538	3.02678400592706\\
14.2664500997406	3.02617096364018\\
14.2647692309132	3.02555787808227\\
14.263088036746	3.02494474924722\\
14.2614065171128	3.02433157712893\\
14.2597246718877	3.02371836172127\\
14.2580425009445	3.02310510301815\\
14.256360004157	3.02249180101344\\
14.254677181399	3.02187845570103\\
14.2529940325442	3.02126506707481\\
14.2513105574662	3.02065163512864\\
14.2496267560384	3.02003815985643\\
14.2479426281343	3.01942464125203\\
14.2462581736273	3.01881107930934\\
14.2445733923907	3.01819747402223\\
14.2428882842977	3.01758382538457\\
14.2412028492215	3.01697013339025\\
14.2395170870351	3.01635639803313\\
14.2378309976116	3.01574261930708\\
14.2361445808238	3.01512879720597\\
14.2344578365447	3.01451493172368\\
14.232770764647	3.01390102285407\\
14.2310833650034	3.013287070591\\
14.2293956374866	3.01267307492835\\
14.227707581969	3.01205903585998\\
14.2260191983233	3.01144495337974\\
14.2243304864218	3.0108308274815\\
14.2226414461368	3.01021665815913\\
14.2209520773407	3.00960244540647\\
14.2192623799055	3.00898818921739\\
14.2175723537035	3.00837388958574\\
14.2158819986065	3.00775954650538\\
14.2141913144867	3.00714515997016\\
14.2125003012159	3.00653072997392\\
14.2108089586659	3.00591625651054\\
14.2091172867085	3.00530173957384\\
14.2074252852153	3.00468717915769\\
14.2057329540578	3.00407257525592\\
14.2040402931077	3.00345792786239\\
14.2023473022363	3.00284323697094\\
14.2006539813151	3.0022285025754\\
14.1989603302152	3.00161372466963\\
14.197266348808	3.00099890324746\\
14.1955720369645	3.00038403830273\\
14.1938773945558	2.99976912982928\\
14.192182421453	2.99915417782094\\
14.1904871175268	2.99853918227156\\
14.1887914826482	2.99792414317496\\
14.1870955166879	2.99730906052498\\
14.1853992195166	2.99669393431545\\
14.1837025910049	2.99607876454019\\
14.1820056310233	2.99546355119304\\
14.1803083394423	2.99484829426783\\
14.1786107161323	2.99423299375837\\
14.1769127609636	2.9936176496585\\
14.1752144738063	2.99300226196204\\
14.1735158545307	2.99238683066281\\
14.1718169030069	2.99177135575462\\
14.1701176191047	2.99115583723131\\
14.1684180026943	2.99054027508669\\
14.1667180536453	2.98992466931457\\
14.1650177718276	2.98930901990877\\
14.1633171571108	2.9886933268631\\
14.1616162093646	2.98807759017138\\
14.1599149284586	2.98746180982742\\
14.1582133142621	2.98684598582503\\
14.1565113666446	2.98623011815802\\
14.1548090854754	2.98561420682019\\
14.1531064706237	2.98499825180535\\
14.1514035219587	2.98438225310731\\
14.1497002393495	2.98376621071987\\
14.147996622665	2.98315012463683\\
14.1462926717742	2.982533994852\\
14.1445883865459	2.98191782135916\\
14.1428837668489	2.98130160415213\\
14.1411788125519	2.98068534322469\\
14.1394735235236	2.98006903857065\\
14.1377678996324	2.9794526901838\\
14.1360619407468	2.97883629805793\\
14.1343556467352	2.97821986218683\\
14.1326490174659	2.97760338256429\\
14.1309420528071	2.9769868591841\\
14.1292347526271	2.97637029204006\\
14.1275271167938	2.97575368112594\\
14.1258191451753	2.97513702643553\\
14.1241108376394	2.97452032796262\\
14.1224021940541	2.97390358570098\\
14.120693214287	2.97328679964441\\
14.1189838982059	2.97266996978667\\
14.1172742456784	2.97205309612155\\
14.115564256572	2.97143617864282\\
14.1138539307541	2.97081921734426\\
14.1121432680921	2.97020221221965\\
14.1104322684533	2.96958516326275\\
14.108720931705	2.96896807046735\\
14.1070092577142	2.9683509338272\\
14.1052972463481	2.96773375333608\\
14.1035848974735	2.96711652898776\\
14.1018722109575	2.966499260776\\
14.1001591866667	2.96588194869457\\
14.098445824468	2.96526459273724\\
14.0967321242281	2.96464719289775\\
14.0950180858134	2.96402974916989\\
14.0933037090906	2.9634122615474\\
14.091588993926	2.96279473002404\\
14.089873940186	2.96217715459358\\
14.0881585477368	2.96155953524976\\
14.0864428164447	2.96094187198635\\
14.0847267461758	2.96032416479709\\
14.083010336796	2.95970641367574\\
14.0812935881713	2.95908861861605\\
14.0795765001677	2.95847077961176\\
14.0778590726509	2.95785289665663\\
14.0761413054866	2.9572349697444\\
14.0744231985404	2.95661699886882\\
14.0727047516779	2.95599898402362\\
14.0709859647646	2.95538092520256\\
14.0692668376659	2.95476282239938\\
14.067547370247	2.95414467560781\\
14.0658275623732	2.95352648482159\\
14.0641074139097	2.95290825003446\\
14.0623869247216	2.95228997124016\\
14.0606660946737	2.95167164843241\\
14.0589449236311	2.95105328160496\\
14.0572234114586	2.95043487075154\\
14.0555015580209	2.94981641586587\\
14.0537793631827	2.94919791694169\\
14.0520568268085	2.94857937397272\\
14.050333948763	2.94796078695269\\
14.0486107289105	2.94734215587532\\
14.0468871671154	2.94672348073434\\
14.0451632632419	2.94610476152347\\
14.0434390171542	2.94548599823643\\
14.0417144287165	2.94486719086695\\
14.0399894977927	2.94424833940874\\
14.0382642242469	2.94362944385551\\
14.0365386079428	2.94301050420098\\
14.0348126487442	2.94239152043887\\
14.033086346515	2.9417724925629\\
14.0313597011185	2.94115342056676\\
14.0296327124185	2.94053430444418\\
14.0279053802784	2.93991514418886\\
14.0261777045615	2.9392959397945\\
14.0244496851311	2.93867669125483\\
14.0227213218505	2.93805739856353\\
14.0209926145828	2.93743806171432\\
14.019263563191	2.93681868070089\\
14.0175341675381	2.93619925551695\\
14.0158044274869	2.9355797861562\\
14.0140743429004	2.93496027261233\\
14.0123439136411	2.93434071487905\\
14.0106131395718	2.93372111295004\\
14.008882020555	2.93310146681901\\
14.0071505564532	2.93248177647963\\
14.0054187471287	2.93186204192562\\
14.003686592444	2.93124226315065\\
14.0019540922611	2.93062244014842\\
14.0002212464422	2.93000257291261\\
13.9984880548495	2.92938266143691\\
13.9967545173449	2.928762705715\\
13.9950206337903	2.92814270574057\\
13.9932864040475	2.9275226615073\\
13.9915518279782	2.92690257300886\\
13.9898169054441	2.92628244023895\\
13.9880816363067	2.92566226319123\\
13.9863460204276	2.92504204185939\\
13.9846100576682	2.92442177623709\\
13.9828737478897	2.92380146631802\\
13.9811370909534	2.92318111209585\\
13.9794000867204	2.92256071356424\\
13.9776627350518	2.92194027071687\\
13.9759250358086	2.9213197835474\\
13.9741869888517	2.9206992520495\\
13.972448594042	2.92007867621684\\
13.97070985124	2.91945805604308\\
13.9689707603066	2.91883739152189\\
13.9672313211022	2.91821668264693\\
13.9654915334874	2.91759592941185\\
13.9637513973224	2.91697513181032\\
13.9620109124678	2.916354289836\\
13.9602700787836	2.91573340348253\\
13.9585288961301	2.91511247274358\\
13.9567873643673	2.9144914976128\\
13.9550454833551	2.91387047808384\\
13.9533032529535	2.91324941415035\\
13.9515606730223	2.91262830580598\\
13.9498177434211	2.91200715304438\\
13.9480744640098	2.9113859558592\\
13.9463308346477	2.91076471424408\\
13.9445868551944	2.91014342819266\\
13.9428425255092	2.90952209769859\\
13.9410978454515	2.90890072275552\\
13.9393528148805	2.90827930335707\\
13.9376074336552	2.90765783949689\\
13.9358617016349	2.90703633116861\\
13.9341156186783	2.90641477836587\\
13.9323691846445	2.90579318108231\\
13.9306223993921	2.90517153931156\\
13.92887526278	2.90454985304725\\
13.9271277746666	2.90392812228301\\
13.9253799349107	2.90330634701247\\
13.9236317433705	2.90268452722926\\
13.9218831999045	2.90206266292699\\
13.9201343043709	2.90144075409931\\
13.918385056628	2.90081880073983\\
13.9166354565338	2.90019680284217\\
13.9148855039465	2.89957476039996\\
13.9131351987238	2.89895267340681\\
13.9113845407237	2.89833054185635\\
13.9096335298039	2.89770836574218\\
13.9078821658222	2.89708614505793\\
13.9061304486361	2.8964638797972\\
13.904378378103	2.89584156995361\\
13.9026259540805	2.89521921552078\\
13.9008731764259	2.89459681649231\\
13.8991200449964	2.89397437286181\\
13.8973665596491	2.89335188462288\\
13.8956127202412	2.89272935176914\\
13.8938585266297	2.89210677429419\\
13.8921039786714	2.89148415219162\\
13.8903490762231	2.89086148545504\\
13.8885938191417	2.89023877407806\\
13.8868382072836	2.88961601805427\\
13.8850822405056	2.88899321737726\\
13.883325918664	2.88837037204064\\
13.8815692416152	2.887747482038\\
13.8798122092155	2.88712454736293\\
13.8780548213212	2.88650156800902\\
13.8762970777883	2.88587854396987\\
13.8745389784729	2.88525547523906\\
13.872780523231	2.88463236181018\\
13.8710217119183	2.88400920367682\\
13.8692625443906	2.88338600083256\\
13.8675030205037	2.88276275327099\\
13.8657431401131	2.88213946098568\\
13.8639829030743	2.88151612397023\\
13.8622223092428	2.8808927422182\\
13.8604613584739	2.88026931572318\\
13.8587000506228	2.87964584447875\\
13.8569383855446	2.87902232847847\\
13.8551763630945	2.87839876771592\\
13.8534139831274	2.87777516218469\\
13.8516512454982	2.87715151187832\\
13.8498881500617	2.87652781679041\\
13.8481246966726	2.87590407691451\\
13.8463608851856	2.87528029224419\\
13.8445967154551	2.87465646277301\\
13.8428321873357	2.87403258849455\\
13.8410673006816	2.87340866940237\\
13.8393020553472	2.87278470549002\\
13.8375364511866	2.87216069675106\\
13.835770488054	2.87153664317906\\
13.8340041658032	2.87091254476758\\
13.8322374842883	2.87028840151016\\
13.8304704433631	2.86966421340036\\
13.8287030428812	2.86903998043174\\
13.8269352826964	2.86841570259784\\
13.8251671626622	2.86779137989223\\
13.8233986826321	2.86716701230844\\
13.8216298424594	2.86654259984003\\
13.8198606419974	2.86591814248053\\
13.8180910810993	2.8652936402235\\
13.8163211596183	2.86466909306248\\
13.8145508774073	2.86404450099101\\
13.8127802343193	2.86341986400264\\
13.8110092302072	2.86279518209089\\
13.8092378649236	2.86217045524931\\
13.8074661383212	2.86154568347144\\
13.8056940502526	2.8609208667508\\
13.8039216005703	2.86029600508094\\
13.8021487891266	2.85967109845539\\
13.8003756157739	2.85904614686767\\
13.7986020803644	2.85842115031132\\
13.7968281827501	2.85779610877986\\
13.7950539227831	2.85717102226682\\
13.7932793003154	2.85654589076573\\
13.7915043151988	2.85592071427011\\
13.789728967285	2.85529549277349\\
13.7879532564256	2.85467022626937\\
13.7861771824724	2.85404491475129\\
13.7844007452767	2.85341955821276\\
13.78262394469	2.8527941566473\\
13.7808467805634	2.85216871004842\\
13.7790692527484	2.85154321840965\\
13.7772913610958	2.85091768172448\\
13.7755131054569	2.85029209998643\\
13.7737344856825	2.84966647318902\\
13.7719555016234	2.84904080132574\\
13.7701761531304	2.84841508439011\\
13.7683964400542	2.84778932237564\\
13.7666163622453	2.84716351527582\\
13.7648359195542	2.84653766308416\\
13.7630551118313	2.84591176579417\\
13.7612739389269	2.84528582339933\\
13.7594924006911	2.84465983589316\\
13.7577104969741	2.84403380326914\\
13.7559282276259	2.84340772552078\\
13.7541455924964	2.84278160264156\\
13.7523625914354	2.84215543462499\\
13.7505792242927	2.84152922146455\\
13.7487954909179	2.84090296315373\\
13.7470113911605	2.84027665968602\\
13.7452269248701	2.83965031105491\\
13.743442091896	2.83902391725388\\
13.7416568920874	2.83839747827642\\
13.7398713252936	2.83777099411602\\
13.7380853913635	2.83714446476615\\
13.7362990901463	2.83651789022029\\
13.7345124214908	2.83589127047193\\
13.7327253852459	2.83526460551454\\
13.7309379812601	2.83463789534159\\
13.7291502093822	2.83401113994657\\
13.7273620694607	2.83338433932294\\
13.725573561344	2.83275749346418\\
13.7237846848805	2.83213060236375\\
13.7219954399183	2.83150366601513\\
13.7202058263057	2.83087668441179\\
13.7184158438907	2.83024965754718\\
13.7166254925213	2.82962258541478\\
13.7148347720453	2.82899546800805\\
13.7130436823105	2.82836830532045\\
13.7112522231646	2.82774109734543\\
13.7094603944552	2.82711384407647\\
13.7076681960297	2.82648654550702\\
13.7058756277357	2.82585920163053\\
13.7040826894203	2.82523181244045\\
13.7022893809308	2.82460437793025\\
13.7004957021143	2.82397689809338\\
13.6987016528178	2.82334937292328\\
13.6969072328882	2.8227218024134\\
13.6951124421725	2.82209418655719\\
13.6933172805172	2.82146652534811\\
13.6915217477691	2.82083881877958\\
13.6897258437747	2.82021106684506\\
13.6879295683804	2.81958326953799\\
13.6861329214326	2.81895542685181\\
13.6843359027776	2.81832753877996\\
13.6825385122615	2.81769960531587\\
13.6807407497304	2.81707162645299\\
13.6789426150302	2.81644360218474\\
13.6771441080069	2.81581553250457\\
13.6753452285062	2.8151874174059\\
13.6735459763738	2.81455925688216\\
13.6717463514553	2.81393105092679\\
13.6699463535962	2.8133027995332\\
13.6681459826419	2.81267450269484\\
13.6663452384377	2.81204616040512\\
13.6645441208287	2.81141777265746\\
13.6627426296601	2.81078933944529\\
13.660940764777	2.81016086076203\\
13.6591385260242	2.8095323366011\\
13.6573359132465	2.80890376695592\\
13.6555329262887	2.80827515181989\\
13.6537295649954	2.80764649118645\\
13.6519258292111	2.807017785049\\
13.6501217187802	2.80638903340095\\
13.6483172335472	2.80576023623571\\
13.6465123733561	2.8051313935467\\
13.6447071380513	2.80450250532731\\
13.6429015274766	2.80387357157097\\
13.6410955414761	2.80324459227107\\
13.6392891798937	2.80261556742101\\
13.6374824425729	2.8019864970142\\
13.6356753293576	2.80135738104404\\
13.6338678400913	2.80072821950393\\
13.6320599746174	2.80009901238727\\
13.6302517327792	2.79946975968744\\
13.6284431144202	2.79884046139786\\
13.6266341193833	2.7982111175119\\
13.6248247475117	2.79758172802297\\
13.6230149986484	2.79695229292445\\
13.6212048726362	2.79632281220973\\
13.6193943693179	2.79569328587221\\
13.6175834885362	2.79506371390526\\
13.6157722301336	2.79443409630227\\
13.6139605939527	2.79380443305663\\
13.6121485798358	2.79317472416171\\
13.6103361876251	2.7925449696109\\
13.6085234171629	2.79191516939758\\
13.6067102682913	2.79128532351512\\
13.6048967408522	2.79065543195691\\
13.6030828346875	2.79002549471631\\
13.601268549639	2.78939551178669\\
13.5994538855484	2.78876548316144\\
13.5976388422572	2.78813540883392\\
13.5958234196071	2.7875052887975\\
13.5940076174393	2.78687512304555\\
13.5921914355951	2.78624491157143\\
13.5903748739158	2.78561465436851\\
13.5885579322424	2.78498435143015\\
13.5867406104159	2.78435400274972\\
13.5849229082772	2.78372360832057\\
13.5831048256671	2.78309316813606\\
13.5812863624263	2.78246268218955\\
13.5794675183953	2.78183215047441\\
13.5776482934147	2.78120157298396\\
13.5758286873249	2.78057094971159\\
13.5740086999661	2.77994028065063\\
13.5721883311785	2.77930956579444\\
13.5703675808023	2.77867880513636\\
13.5685464486773	2.77804799866975\\
13.5667249346436	2.77741714638794\\
13.5649030385408	2.77678624828428\\
13.5630807602087	2.77615530435213\\
13.5612580994869	2.7755243145848\\
13.5594350562148	2.77489327897566\\
13.5576116302318	2.77426219751803\\
13.5557878213772	2.77363107020526\\
13.5539636294902	2.77299989703067\\
13.5521390544099	2.77236867798761\\
13.5503140959752	2.7717374130694\\
13.5484887540249	2.77110610226938\\
13.546663028398	2.77047474558087\\
13.5448369189331	2.76984334299722\\
13.5430104254687	2.76921189451173\\
13.5411835478432	2.76858040011775\\
13.5393562858952	2.76794885980858\\
13.5375286394627	2.76731727357756\\
13.5357006083841	2.76668564141801\\
13.5338721924973	2.76605396332324\\
13.5320433916404	2.76542223928658\\
13.5302142056511	2.76479046930134\\
13.5283846343672	2.76415865336083\\
13.5265546776264	2.76352679145838\\
13.5247243352662	2.76289488358729\\
13.5228936071241	2.76226292974087\\
13.5210624930374	2.76163092991243\\
13.5192309928434	2.76099888409529\\
13.5173991063791	2.76036679228274\\
13.5155668334817	2.7597346544681\\
13.513734173988	2.75910247064466\\
13.5119011277349	2.75847024080573\\
13.5100676945591	2.75783796494461\\
13.5082338742973	2.75720564305461\\
13.5063996667858	2.756573275129\\
13.5045650718613	2.7559408611611\\
13.5027300893599	2.7553084011442\\
13.5008947191179	2.75467589507159\\
13.4990589609713	2.75404334293656\\
13.4972228147562	2.75341074473241\\
13.4953862803085	2.75277810045242\\
13.493549357464	2.75214541008988\\
13.4917120460583	2.75151267363808\\
13.489874345927	2.75087989109029\\
13.4880362569056	2.75024706243982\\
13.4861977788295	2.74961418767992\\
13.484358911534	2.7489812668039\\
13.4825196548542	2.74834829980502\\
13.4806800086251	2.74771528667656\\
13.4788399726818	2.7470822274118\\
13.476999546859	2.74644912200401\\
13.4751587309915	2.74581597044646\\
13.473317524914	2.74518277273244\\
13.471475928461	2.7445495288552\\
13.4696339414669	2.74391623880802\\
13.4677915637661	2.74328290258415\\
13.4659487951927	2.74264952017688\\
13.4641056355809	2.74201609157945\\
13.4622620847647	2.74138261678515\\
13.4604181425779	2.74074909578721\\
13.4585738088544	2.74011552857891\\
13.4567290834279	2.73948191515351\\
13.454883966132	2.73884825550425\\
13.4530384568001	2.7382145496244\\
13.4511925552655	2.73758079750721\\
13.4493462613616	2.73694699914593\\
13.4474995749216	2.73631315453381\\
13.4456524957784	2.7356792636641\\
13.4438050237651	2.73504532653005\\
13.4419571587144	2.73441134312491\\
13.440108900459	2.73377731344191\\
13.4382602488317	2.73314323747431\\
13.436411203665	2.73250911521534\\
13.4345617647912	2.73187494665825\\
13.4327119320426	2.73124073179627\\
13.4308617052515	2.73060647062265\\
13.4290110842499	2.72997216313061\\
13.4271600688698	2.7293378093134\\
13.4253086589432	2.72870340916424\\
13.4234568543017	2.72806896267637\\
13.421604654777	2.72743446984301\\
13.4197520602007	2.7267999306574\\
13.4178990704042	2.72616534511277\\
13.4160456852189	2.72553071320233\\
13.4141919044759	2.72489603491931\\
13.4123377280065	2.72426131025694\\
13.4104831556416	2.72362653920843\\
13.4086281872121	2.72299172176701\\
13.4067728225488	2.7223568579259\\
13.4049170614825	2.7217219476783\\
13.4030609038436	2.72108699101744\\
13.4012043494627	2.72045198793653\\
13.3993473981701	2.71981693842878\\
13.3974900497961	2.7191818424874\\
13.3956323041707	2.71854670010561\\
13.3937741611242	2.7179115112766\\
13.3919156204863	2.71727627599359\\
13.3900566820869	2.71664099424978\\
13.3881973457557	2.71600566603837\\
13.3863376113222	2.71537029135257\\
13.3844774786161	2.71473487018557\\
13.3826169474667	2.71409940253058\\
13.3807560177031	2.71346388838079\\
13.3788946891547	2.7128283277294\\
13.3770329616504	2.71219272056959\\
13.3751708350192	2.71155706689458\\
13.3733083090898	2.71092136669754\\
13.3714453836912	2.71028561997166\\
13.3695820586517	2.70964982671014\\
13.3677183338	2.70901398690616\\
13.3658542089644	2.7083781005529\\
13.3639896839732	2.70774216764356\\
13.3621247586546	2.7071061881713\\
13.3602594328366	2.70647016212932\\
13.3583937063472	2.7058340895108\\
13.3565275790142	2.7051979703089\\
13.3546610506653	2.70456180451681\\
13.3527941211282	2.7039255921277\\
13.3509267902303	2.70328933313475\\
13.3490590577991	2.70265302753112\\
13.3471909236617	2.70201667530999\\
13.3453223876455	2.70138027646453\\
13.3434534495774	2.7007438309879\\
13.3415841092843	2.70010733887326\\
13.3397143665932	2.6994708001138\\
13.3378442213307	2.69883421470265\\
13.3359736733235	2.698197582633\\
13.334102722398	2.69756090389799\\
13.3322313683806	2.69692417849079\\
13.3303596110976	2.69628740640455\\
13.3284874503752	2.69565058763243\\
13.3266148860394	2.69501372216758\\
13.3247419179161	2.69437681000316\\
13.3228685458312	2.69373985113231\\
13.3209947696103	2.69310284554818\\
13.3191205890791	2.69246579324393\\
13.3172460040631	2.69182869421269\\
13.3153710143876	2.69119154844762\\
13.3134956198779	2.69055435594185\\
13.3116198203591	2.68991711668853\\
13.3097436156562	2.6892798306808\\
13.3078670055942	2.68864249791179\\
13.3059899899979	2.68800511837465\\
13.304112568692	2.68736769206251\\
13.302234741501	2.68673021896851\\
13.3003565082495	2.68609269908577\\
13.2984778687616	2.68545513240742\\
13.2965988228618	2.68481751892661\\
13.2947193703741	2.68417985863645\\
13.2928395111225	2.68354215153007\\
13.2909592449309	2.68290439760061\\
13.2890785716232	2.68226659684117\\
13.2871974910228	2.68162874924488\\
13.2853160029535	2.68099085480487\\
13.2834341072386	2.68035291351425\\
13.2815518037015	2.67971492536615\\
13.2796690921653	2.67907689035366\\
13.2777859724532	2.67843880846992\\
13.2759024443881	2.67780067970804\\
13.274018507793	2.67716250406111\\
13.2721341624904	2.67652428152227\\
13.2702494083031	2.6758860120846\\
13.2683642450536	2.67524769574123\\
13.2664786725642	2.67460933248525\\
13.2645926906574	2.67397092230978\\
13.2627062991551	2.6733324652079\\
13.2608194978795	2.67269396117273\\
13.2589322866525	2.67205541019736\\
13.2570446652959	2.67141681227489\\
13.2551566336315	2.67077816739842\\
13.2532681914808	2.67013947556104\\
13.2513793386652	2.66950073675584\\
13.2494900750062	2.66886195097592\\
13.247600400325	2.66822311821437\\
13.2457103144426	2.66758423846426\\
13.2438198171801	2.66694531171871\\
13.2419289083585	2.66630633797078\\
13.2400375877983	2.66566731721356\\
13.2381458553204	2.66502824944013\\
13.2362537107452	2.66438913464359\\
13.2343611538932	2.66374997281699\\
13.2324681845846	2.66311076395343\\
13.2305748026396	2.66247150804598\\
13.2286810078784	2.66183220508771\\
13.2267868001208	2.6611928550717\\
13.2248921791867	2.66055345799102\\
13.2229971448957	2.65991401383873\\
13.2211016970676	2.65927452260792\\
13.2192058355217	2.65863498429164\\
13.2173095600774	2.65799539888296\\
13.2154128705539	2.65735576637495\\
13.2135157667705	2.65671608676067\\
13.211618248546	2.65607636003317\\
13.2097203156994	2.65543658618553\\
13.2078219680493	2.65479676521079\\
13.2059232054146	2.65415689710201\\
13.2040240276136	2.65351698185226\\
13.2021244344649	2.65287701945457\\
13.2002244257866	2.65223700990201\\
13.198324001397	2.65159695318763\\
13.1964231611141	2.65095684930446\\
13.1945219047558	2.65031669824557\\
13.19262023214	2.64967650000399\\
13.1907181430843	2.64903625457278\\
13.1888156374064	2.64839596194496\\
13.1869127149235	2.6477556221136\\
13.1850093754532	2.64711523507171\\
13.1831056188126	2.64647480081235\\
13.1812014448188	2.64583431932854\\
13.1792968532887	2.64519379061333\\
13.1773918440392	2.64455321465975\\
13.1754864168871	2.64391259146082\\
13.173580571649	2.64327192100959\\
13.1716743081413	2.64263120329907\\
13.1697676261803	2.64199043832229\\
13.1678605255825	2.64134962607229\\
13.1659530061638	2.64070876654209\\
13.1640450677403	2.6400678597247\\
13.1621367101279	2.63942690561315\\
13.1602279331422	2.63878590420046\\
13.1583187365991	2.63814485547964\\
13.1564091203139	2.63750375944372\\
13.1544990841022	2.63686261608571\\
13.152588627779	2.63622142539862\\
13.1506777511597	2.63558018737546\\
13.1487664540593	2.63493890200925\\
13.1468547362925	2.63429756929298\\
13.1449425976743	2.63365618921968\\
13.1430300380193	2.63301476178234\\
13.1411170571421	2.63237328697397\\
13.139203654857	2.63173176478758\\
13.1372898309783	2.63109019521615\\
13.1353755853203	2.6304485782527\\
13.133460917697	2.62980691389022\\
13.1315458279223	2.6291652021217\\
13.12963031581	2.62852344294015\\
13.1277143811738	2.62788163633855\\
13.1257980238272	2.6272397823099\\
13.1238812435837	2.62659788084718\\
13.1219640402566	2.62595593194339\\
13.1200464136591	2.62531393559151\\
13.1181283636043	2.62467189178453\\
13.116209889905	2.62402980051542\\
13.1142909923742	2.62338766177718\\
13.1123716708244	2.62274547556279\\
13.1104519250684	2.62210324186522\\
13.1085317549184	2.62146096067745\\
13.1066111601868	2.62081863199246\\
13.1046901406859	2.62017625580322\\
13.1027686962276	2.6195338321027\\
13.100846826624	2.61889136088389\\
13.0989245316869	2.61824884213973\\
13.0970018112279	2.61760627586322\\
13.0950786650586	2.6169636620473\\
13.0931550929905	2.61632100068495\\
13.0912310948349	2.61567829176913\\
13.0893066704029	2.6150355352928\\
13.0873818195057	2.61439273124893\\
13.0854565419542	2.61374987963046\\
13.0835308375592	2.61310698043037\\
13.0816047061314	2.6124640336416\\
13.0796781474814	2.61182103925711\\
13.0777511614196	2.61117799726985\\
13.0758237477562	2.61053490767278\\
13.0738959063016	2.60989177045884\\
13.0719676368658	2.60924858562097\\
13.0700389392587	2.60860535315214\\
13.06810981329	2.60796207304528\\
13.0661802587696	2.60731874529333\\
13.0642502755069	2.60667536988924\\
13.0623198633113	2.60603194682594\\
13.0603890219923	2.60538847609639\\
13.0584577513588	2.6047449576935\\
13.0565260512201	2.60410139161022\\
13.054593921385	2.60345777783949\\
13.0526613616622	2.60281411637423\\
13.0507283718605	2.60217040720737\\
13.0487949517884	2.60152665033185\\
13.0468611012543	2.60088284574059\\
13.0449268200665	2.60023899342652\\
13.0429921080331	2.59959509338256\\
13.0410569649621	2.59895114560164\\
13.0391213906615	2.59830715007667\\
13.037185384939	2.59766310680057\\
13.0352489476022	2.59701901576628\\
13.0333120784587	2.59637487696669\\
13.0313747773158	2.59573069039473\\
13.0294370439808	2.59508645604331\\
13.0274988782609	2.59444217390534\\
13.0255602799629	2.59379784397373\\
13.0236212488938	2.59315346624139\\
13.0216817848602	2.59250904070123\\
13.0197418876689	2.59186456734615\\
13.0178015571262	2.59122004616906\\
13.0158607930386	2.59057547716286\\
13.0139195952122	2.58993086032044\\
13.0119779634531	2.58928619563472\\
13.0100358975673	2.58864148309858\\
13.0080933973606	2.58799672270493\\
13.0061504626387	2.58735191444665\\
13.0042070932072	2.58670705831665\\
13.0022632888714	2.5860621543078\\
13.0003190494368	2.58541720241301\\
12.9983743747084	2.58477220262515\\
12.9964292644913	2.58412715493712\\
12.9944837185904	2.58348205934181\\
12.9925377368106	2.58283691583208\\
12.9905913189564	2.58219172440083\\
12.9886444648323	2.58154648504094\\
12.9866971742428	2.58090119774528\\
12.9847494469921	2.58025586250674\\
12.9828012828844	2.57961047931818\\
12.9808526817235	2.57896504817248\\
12.9789036433135	2.57831956906252\\
12.9769541674579	2.57767404198116\\
12.9750042539604	2.57702846692128\\
12.9730539026245	2.57638284387573\\
12.9711031132534	2.57573717283739\\
12.9691518856505	2.57509145379912\\
12.9672002196186	2.57444568675379\\
12.9652481149609	2.57379987169425\\
12.96329557148	2.57315400861336\\
12.9613425889787	2.57250809750399\\
12.9593891672594	2.57186213835899\\
12.9574353061246	2.57121613117121\\
12.9554810053766	2.5705700759335\\
12.9535262648174	2.56992397263873\\
12.9515710842491	2.56927782127973\\
12.9496154634735	2.56863162184936\\
12.9476594022924	2.56798537434046\\
12.9457029005073	2.56733907874588\\
12.9437459579198	2.56669273505846\\
12.9417885743311	2.56604634327105\\
12.9398307495424	2.56539990337648\\
12.9378724833549	2.56475341536759\\
12.9359137755694	2.56410687923723\\
12.9339546259867	2.56346029497822\\
12.9319950344075	2.5628136625834\\
12.9300350006322	2.5621669820456\\
12.9280745244614	2.56152025335766\\
12.9261136056952	2.56087347651239\\
12.9241522441337	2.56022665150264\\
12.922190439577	2.55957977832123\\
12.9202281918248	2.55893285696097\\
12.9182655006769	2.5582858874147\\
12.9163023659328	2.55763886967523\\
12.9143387873921	2.55699180373539\\
12.9123747648539	2.55634468958799\\
12.9104102981175	2.55569752722584\\
12.9084453869818	2.55505031664177\\
12.9064800312459	2.55440305782859\\
12.9045142307083	2.5537557507791\\
12.9025479851678	2.55310839548612\\
12.9005812944228	2.55246099194246\\
12.8986141582717	2.55181354014092\\
12.8966465765127	2.5511660400743\\
12.8946785489439	2.55051849173542\\
12.8927100753631	2.54987089511707\\
12.8907411555681	2.54922325021206\\
12.8887717893568	2.54857555701318\\
12.8868019765265	2.54792781551322\\
12.8848317168746	2.54728002570499\\
12.8828610101984	2.54663218758128\\
12.880889856295	2.54598430113488\\
12.8789182549613	2.54533636635859\\
12.8769462059943	2.54468838324518\\
12.8749737091905	2.54404035178745\\
12.8730007643466	2.54339227197818\\
12.8710273712589	2.54274414381016\\
12.8690535297237	2.54209596727617\\
12.8670792395373	2.54144774236899\\
12.8651045004954	2.54079946908141\\
12.8631293123941	2.54015114740619\\
12.8611536750291	2.53950277733611\\
12.8591775881958	2.53885435886396\\
12.8572010516898	2.53820589198249\\
12.8552240653064	2.53755737668449\\
12.8532466288407	2.53690881296272\\
12.8512687420878	2.53626020080995\\
12.8492904048424	2.53561154021895\\
12.8473116168995	2.53496283118248\\
12.8453323780535	2.5343140736933\\
12.8433526880989	2.53366526774418\\
12.8413725468301	2.53301641332787\\
12.8393919540412	2.53236751043714\\
12.8374109095263	2.53171855906474\\
12.8354294130792	2.53106955920343\\
12.8334474644937	2.53042051084595\\
12.8314650635635	2.52977141398507\\
12.829482210082	2.52912226861353\\
12.8274989038425	2.52847307472407\\
12.8255151446383	2.52782383230945\\
12.8235309322623	2.52717454136242\\
12.8215462665075	2.52652520187571\\
12.8195611471666	2.52587581384206\\
12.8175755740324	2.52522637725423\\
12.8155895468971	2.52457689210494\\
12.8136030655533	2.52392735838693\\
12.8116161297931	2.52327777609295\\
12.8096287394084	2.52262814521571\\
12.8076408941914	2.52197846574796\\
12.8056525939336	2.52132873768243\\
12.8036638384268	2.52067896101184\\
12.8016746274624	2.52002913572893\\
12.7996849608318	2.51937926182641\\
12.7976948383261	2.51872933929701\\
12.7957042597364	2.51807936813346\\
12.7937132248536	2.51742934832847\\
12.7917217334685	2.51677927987476\\
12.7897297853717	2.51612916276506\\
12.7877373803537	2.51547899699208\\
12.7857445182047	2.51482878254852\\
12.7837511987151	2.51417851942711\\
12.7817574216747	2.51352820762056\\
12.7797631868736	2.51287784712157\\
12.7777684941015	2.51222743792285\\
12.775773343148	2.51157698001712\\
12.7737777338025	2.51092647339706\\
12.7717816658543	2.51027591805539\\
12.7697851390927	2.50962531398481\\
12.7677881533067	2.50897466117801\\
12.7657907082851	2.5083239596277\\
12.7637928038167	2.50767320932657\\
12.7617944396901	2.50702241026731\\
12.7597956156937	2.50637156244262\\
12.7577963316158	2.50572066584519\\
12.7557965872446	2.50506972046771\\
12.753796382368	2.50441872630287\\
12.751795716774	2.50376768334335\\
12.7497945902502	2.50311659158184\\
12.7477930025842	2.50246545101102\\
12.7457909535635	2.50181426162357\\
12.7437884429752	2.50116302341217\\
12.7417854706066	2.50051173636951\\
12.7397820362446	2.49986040048825\\
12.737778139676	2.49920901576107\\
12.7357737806875	2.49855758218065\\
12.7337689590657	2.49790609973965\\
12.7317636745968	2.49725456843075\\
12.7297579270673	2.49660298824662\\
12.7277517162631	2.49595135917991\\
12.7257450419703	2.49529968122331\\
12.7237379039745	2.49464795436946\\
12.7217303020615	2.49399617861103\\
12.7197222360167	2.49334435394069\\
12.7177137056255	2.49269248035108\\
12.715704710673	2.49204055783487\\
12.7136952509445	2.49138858638471\\
12.7116853262246	2.49073656599326\\
12.7096749362982	2.49008449665316\\
12.70766408095	2.48943237835707\\
12.7056527599642	2.48878021109764\\
12.7036409731254	2.4881279948675\\
12.7016287202175	2.48747572965931\\
12.6996160010246	2.48682341546572\\
12.6976028153305	2.48617105227935\\
12.695589162919	2.48551864009286\\
12.6935750435736	2.48486617889888\\
12.6915604570778	2.48421366869004\\
12.6895454032146	2.48356110945899\\
12.6875298817674	2.48290850119835\\
12.6855138925189	2.48225584390076\\
12.683497435252	2.48160313755885\\
12.6814805097494	2.48095038216524\\
12.6794631157935	2.48029757771257\\
12.6774452531667	2.47964472419346\\
12.6754269216511	2.47899182160053\\
12.6734081210289	2.4783388699264\\
12.6713888510818	2.4776858691637\\
12.6693691115917	2.47703281930504\\
12.6673489023401	2.47637972034304\\
12.6653282231085	2.47572657227031\\
12.6633070736781	2.47507337507947\\
12.66128545383	2.47442012876313\\
12.6592633633452	2.47376683331391\\
12.6572408020046	2.4731134887244\\
12.6552177695888	2.47246009498722\\
12.6531942658783	2.47180665209496\\
12.6511702906534	2.47115316004025\\
12.6491458436945	2.47049961881566\\
12.6471209247815	2.46984602841382\\
12.6450955336943	2.46919238882731\\
12.6430696702127	2.46853870004873\\
12.6410433341162	2.46788496207068\\
12.6390165251843	2.46723117488575\\
12.6369892431964	2.46657733848653\\
12.6349614879314	2.46592345286561\\
12.6329332591684	2.46526951801559\\
12.6309045566862	2.46461553392904\\
12.6288753802634	2.46396150059856\\
12.6268457296787	2.46330741801672\\
12.6248156047102	2.46265328617611\\
12.6227850051362	2.46199910506931\\
12.6207539307348	2.46134487468889\\
12.6187223812838	2.46069059502744\\
12.616690356561	2.46003626607753\\
12.6146578563439	2.45938188783172\\
12.61262488041	2.45872746028261\\
12.6105914285365	2.45807298342274\\
12.6085575005005	2.4574184572447\\
12.6065230960789	2.45676388174105\\
12.6044882150486	2.45610925690434\\
12.6024528571862	2.45545458272716\\
12.6004170222682	2.45479985920206\\
12.5983807100708	2.45414508632159\\
12.5963439203703	2.45349026407832\\
12.5943066529426	2.45283539246481\\
12.5922689075637	2.4521804714736\\
12.5902306840091	2.45152550109725\\
12.5881919820544	2.45087048132832\\
12.586152801475	2.45021541215935\\
12.5841131420461	2.44956029358289\\
12.5820730035427	2.44890512559149\\
12.5800323857398	2.44824990817769\\
12.5779912884121	2.44759464133403\\
12.5759497113342	2.44693932505307\\
12.5739076542805	2.44628395932733\\
12.5718651170252	2.44562854414935\\
12.5698220993425	2.44497307951168\\
12.5677786010062	2.44431756540683\\
12.5657346217903	2.44366200182736\\
12.5636901614683	2.44300638876579\\
12.5616452198136	2.44235072621465\\
12.5595997965996	2.44169501416647\\
12.5575538915994	2.44103925261377\\
12.5555075045861	2.44038344154907\\
12.5534606353323	2.43972758096491\\
12.5514132836109	2.4390716708538\\
12.5493654491942	2.43841571120826\\
12.5473171318547	2.43775970202081\\
12.5452683313644	2.43710364328397\\
12.5432190474955	2.43644753499024\\
12.5411692800198	2.43579137713215\\
12.5391190287089	2.4351351697022\\
12.5370682933345	2.4344789126929\\
12.5350170736679	2.43382260609676\\
12.5329653694802	2.43316624990629\\
12.5309131805426	2.43250984411399\\
12.5288605066259	2.43185338871236\\
12.5268073475008	2.4311968836939\\
12.524753702938	2.43054032905112\\
12.5226995727078	2.4298837247765\\
12.5206449565804	2.42922707086255\\
12.5185898543259	2.42857036730176\\
12.5165342657142	2.42791361408662\\
12.5144781905151	2.42725681120962\\
12.5124216284982	2.42659995866325\\
12.5103645794328	2.42594305644\\
12.5083070430882	2.42528610453235\\
12.5062490192335	2.42462910293279\\
12.5041905076376	2.4239720516338\\
12.5021315080694	2.42331495062786\\
12.5000720202973	2.42265779990745\\
12.4980120440898	2.42200059946504\\
12.4959515792152	2.42134334929311\\
12.4938906254416	2.42068604938413\\
12.491829182537	2.42002869973058\\
12.489767250269	2.41937130032493\\
12.4877048284053	2.41871385115964\\
12.4856419167134	2.41805635222718\\
12.4835785149604	2.41739880352001\\
12.4815146229137	2.4167412050306\\
12.4794502403399	2.41608355675141\\
12.477385367006	2.4154258586749\\
12.4753200026786	2.41476811079353\\
12.4732541471241	2.41411031309976\\
12.4711878001087	2.41345246558603\\
12.4691209613987	2.4127945682448\\
12.4670536307598	2.41213662106853\\
12.464985807958	2.41147862404966\\
12.4629174927588	2.41082057718064\\
12.4608486849276	2.41016248045391\\
12.4587793842298	2.40950433386193\\
12.4567095904304	2.40884613739713\\
12.4546393032944	2.40818789105195\\
12.4525685225865	2.40752959481884\\
12.4504972480714	2.40687124869023\\
12.4484254795134	2.40621285265855\\
12.4463532166769	2.40555440671625\\
12.4442804593259	2.40489591085575\\
12.4422072072244	2.40423736506948\\
12.4401334601361	2.40357876934988\\
12.4380592178246	2.40292012368936\\
12.4359844800534	2.40226142808036\\
12.4339092465856	2.4016026825153\\
12.4318335171844	2.4009438869866\\
12.4297572916126	2.40028504148668\\
12.4276805696331	2.39962614600796\\
12.4256033510083	2.39896720054285\\
12.4235256355007	2.39830820508378\\
12.4214474228725	2.39764915962315\\
12.4193687128858	2.39699006415338\\
12.4172895053024	2.39633091866688\\
12.4152097998841	2.39567172315605\\
12.4131295963924	2.3950124776133\\
12.4110488945887	2.39435318203104\\
12.4089676942342	2.39369383640166\\
12.4068859950899	2.39303444071758\\
12.4048037969166	2.39237499497118\\
12.4027210994752	2.39171549915487\\
12.4006379025259	2.39105595326105\\
12.3985542058294	2.3903963572821\\
12.3964700091456	2.38973671121042\\
12.3943853122345	2.38907701503841\\
12.3923001148561	2.38841726875844\\
12.39021441677	2.38775747236292\\
12.3881282177355	2.38709762584421\\
12.3860415175122	2.38643772919471\\
12.3839543158589	2.38577778240681\\
12.3818666125349	2.38511778547287\\
12.3797784072987	2.38445773838528\\
12.377689699909	2.38379764113642\\
12.3756004901243	2.38313749371865\\
12.3735107777028	2.38247729612436\\
12.3714205624026	2.38181704834592\\
12.3693298439816	2.3811567503757\\
12.3672386221976	2.38049640220605\\
12.365146896808	2.37983600382936\\
12.3630546675704	2.37917555523799\\
12.3609619342419	2.37851505642429\\
12.3588686965795	2.37785450738063\\
12.3567749543401	2.37719390809938\\
12.3546807072804	2.37653325857288\\
12.3525859551568	2.37587255879349\\
12.3504906977258	2.37521180875357\\
12.3483949347435	2.37455100844548\\
12.3462986659659	2.37389015786155\\
12.3442018911487	2.37322925699414\\
12.3421046100476	2.37256830583561\\
12.340006822418	2.37190730437828\\
12.3379085280152	2.37124625261451\\
12.3358097265943	2.37058515053665\\
12.3337104179102	2.36992399813702\\
12.3316106017177	2.36926279540797\\
12.3295102777713	2.36860154234183\\
12.3274094458254	2.36794023893095\\
12.3253081056342	2.36727888516765\\
12.3232062569517	2.36661748104426\\
12.3211038995317	2.36595602655312\\
12.319001033128	2.36529452168655\\
12.316897657494	2.36463296643689\\
12.3147937723831	2.36397136079644\\
12.3126893775483	2.36330970475755\\
12.3105844727427	2.36264799831252\\
12.3084790577189	2.36198624145368\\
12.3063731322296	2.36132443417334\\
12.3042666960272	2.36066257646383\\
12.3021597488639	2.36000066831745\\
12.3000522904918	2.35933870972652\\
12.2979443206627	2.35867670068334\\
12.2958358391284	2.35801464118023\\
12.2937268456403	2.3573525312095\\
12.2916173399497	2.35669037076344\\
12.2895073218079	2.35602815983437\\
12.2873967909658	2.35536589841458\\
12.2852857471741	2.35470358649638\\
12.2831741901835	2.35404122407206\\
12.2810621197444	2.35337881113391\\
12.278949535607	2.35271634767424\\
12.2768364375214	2.35205383368534\\
12.2747228252374	2.35139126915949\\
12.2726086985048	2.35072865408899\\
12.270494057073	2.35006598846613\\
12.2683789006915	2.34940327228319\\
12.2662632291092	2.34874050553245\\
12.2641470420752	2.3480776882062\\
12.2620303393383	2.34741482029671\\
12.2599131206469	2.34675190179628\\
12.2577953857497	2.34608893269716\\
12.2556771343947	2.34542591299165\\
12.25355836633	2.34476284267201\\
12.2514390813035	2.34409972173051\\
12.2493192790628	2.34343655015942\\
12.2471989593555	2.34277332795102\\
12.2450781219287	2.34211005509757\\
12.2429567665297	2.34144673159133\\
12.2408348929054	2.34078335742457\\
12.2387125008025	2.34011993258954\\
12.2365895899675	2.33945645707851\\
12.2344661601468	2.33879293088374\\
12.2323422110867	2.33812935399747\\
12.2302177425331	2.33746572641197\\
12.2280927542319	2.33680204811948\\
12.2259672459286	2.33613831911227\\
12.2238412173687	2.33547453938256\\
12.2217146682975	2.33481070892262\\
12.2195875984599	2.33414682772469\\
12.217460007601	2.333482895781\\
12.2153318954654	2.33281891308382\\
12.2132032617976	2.33215487962536\\
12.2110741063419	2.33149079539787\\
12.2089444288424	2.33082666039359\\
12.2068142290431	2.33016247460475\\
12.2046835066878	2.32949823802358\\
12.2025522615199	2.32883395064232\\
12.2004204932829	2.32816961245319\\
12.19828820172	2.32750522344842\\
12.196155386574	2.32684078362024\\
12.194022047588	2.32617629296087\\
12.1918881845044	2.32551175146253\\
12.1897537970657	2.32484715911744\\
12.1876188850141	2.32418251591782\\
12.1854834480918	2.3235178218559\\
12.1833474860404	2.32285307692387\\
12.1812109986017	2.32218828111395\\
12.1790739855173	2.32152343441837\\
12.1769364465282	2.32085853682932\\
12.1747983813758	2.32019358833901\\
12.1726597898007	2.31952858893965\\
12.1705206715439	2.31886353862345\\
12.1683810263457	2.3181984373826\\
12.1662408539465	2.31753328520931\\
12.1641001540865	2.31686808209577\\
12.1619589265056	2.31620282803419\\
12.1598171709435	2.31553752301675\\
12.1576748871398	2.31487216703565\\
12.1555320748339	2.31420676008308\\
12.1533887337649	2.31354130215124\\
12.1512448636718	2.31287579323231\\
12.1491004642934	2.31221023331846\\
12.1469555353683	2.31154462240191\\
12.1448100766349	2.31087896047481\\
12.1426640878313	2.31021324752936\\
12.1405175686957	2.30954748355773\\
12.1383705189658	2.30888166855209\\
12.1362229383793	2.30821580250464\\
12.1340748266735	2.30754988540753\\
12.1319261835857	2.30688391725294\\
12.129777008853	2.30621789803304\\
12.1276273022121	2.30555182774\\
12.1254770633998	2.30488570636599\\
12.1233262921524	2.30421953390316\\
12.1211749882063	2.30355331034369\\
12.1190231512975	2.30288703567973\\
12.1168707811618	2.30222070990344\\
12.1147178775349	2.30155433300698\\
12.1125644401524	2.3008879049825\\
12.1104104687494	2.30022142582217\\
12.108255963061	2.29955489551812\\
12.1061009228222	2.29888831406251\\
12.1039453477676	2.2982216814475\\
12.1017892376316	2.29755499766522\\
12.0996325921486	2.29688826270782\\
12.0974754110527	2.29622147656744\\
12.0953176940777	2.29555463923623\\
12.0931594409574	2.29488775070632\\
12.0910006514252	2.29422081096985\\
12.0888413252145	2.29355382001895\\
12.0866814620582	2.29288677784577\\
12.0845210616894	2.29221968444243\\
12.0823601238407	2.29155253980106\\
12.0801986482446	2.2908853439138\\
12.0780366346334	2.29021809677276\\
12.0758740827393	2.28955079837007\\
12.073710992294	2.28888344869786\\
12.0715473630293	2.28821604774824\\
12.0693831946768	2.28754859551334\\
12.0672184869676	2.28688109198527\\
12.0650532396329	2.28621353715615\\
12.0628874524036	2.2855459310181\\
12.0607211250104	2.28487827356322\\
12.0585542571838	2.28421056478362\\
12.056386848654	2.28354280467142\\
12.0542188991512	2.28287499321871\\
12.0520504084051	2.28220713041761\\
12.0498813761456	2.28153921626022\\
12.0477118021021	2.28087125073864\\
12.0455416860039	2.28020323384496\\
12.0433710275799	2.27953516557129\\
12.0411998265592	2.27886704590972\\
12.0390280826704	2.27819887485235\\
12.036855795642	2.27753065239126\\
12.0346829652021	2.27686237851855\\
12.0325095910789	2.27619405322631\\
12.0303356730002	2.27552567650661\\
12.0281612106937	2.27485724835156\\
12.0259862038868	2.27418876875322\\
12.0238106523067	2.27352023770368\\
12.0216345556805	2.27285165519502\\
12.019457913735	2.27218302121932\\
12.0172807261968	2.27151433576866\\
12.0151029927924	2.27084559883509\\
12.0129247132479	2.27017681041071\\
12.0107458872894	2.26950797048757\\
12.0085665146426	2.26883907905775\\
12.0063865950332	2.26817013611332\\
12.0042061281866	2.26750114164633\\
12.0020251138278	2.26683209564885\\
11.999843551682	2.26616299811294\\
11.9976614414738	2.26549384903066\\
11.9954787829278	2.26482464839407\\
11.9932955757683	2.26415539619522\\
11.9911118197196	2.26348609242616\\
11.9889275145055	2.26281673707896\\
11.9867426598498	2.26214733014565\\
11.9845572554759	2.26147787161829\\
11.9823713011073	2.26080836148892\\
11.9801847964669	2.26013879974958\\
11.9779977412777	2.25946918639233\\
11.9758101352623	2.25879952140919\\
11.9736219781433	2.25812980479221\\
11.9714332696428	2.25746003653343\\
11.969244009483	2.25679021662488\\
11.9670541973857	2.25612034505859\\
11.9648638330725	2.25545042182659\\
11.9626729162648	2.25478044692092\\
11.9604814466838	2.2541104203336\\
11.9582894240506	2.25344034205666\\
11.9560968480859	2.25277021208213\\
11.9539037185102	2.25210003040201\\
11.9517100350441	2.25142979700835\\
11.9495157974075	2.25075951189314\\
11.9473210053206	2.25008917504842\\
11.9451256585028	2.24941878646618\\
11.9429297566739	2.24874834613846\\
11.9407332995531	2.24807785405726\\
11.9385362868594	2.24740731021459\\
11.9363387183118	2.24673671460245\\
11.9341405936289	2.24606606721285\\
11.9319419125292	2.2453953680378\\
11.9297426747309	2.24472461706929\\
11.927542879952	2.24405381429933\\
11.9253425279103	2.24338295971992\\
11.9231416183234	2.24271205332305\\
11.9209401509088	2.24204109510071\\
11.9187381253835	2.2413700850449\\
11.9165355414645	2.24069902314761\\
11.9143323988685	2.24002790940083\\
11.9121286973121	2.23935674379655\\
11.9099244365115	2.23868552632674\\
11.9077196161828	2.2380142569834\\
11.905514236042	2.2373429357585\\
11.9033082958046	2.23667156264403\\
11.9011017951861	2.23600013763196\\
11.8988947339017	2.23532866071427\\
11.8966871116664	2.23465713188293\\
11.8944789281949	2.23398555112991\\
11.892270183202	2.23331391844719\\
11.8900608764018	2.23264223382674\\
11.8878510075085	2.23197049726051\\
11.8856405762361	2.23129870874047\\
11.8834295822982	2.2306268682586\\
11.8812180254084	2.22995497580684\\
11.8790059052797	2.22928303137716\\
11.8767932216254	2.22861103496151\\
11.8745799741582	2.22793898655186\\
11.8723661625907	2.22726688614014\\
11.8701517866353	2.22659473371832\\
11.8679368460041	2.22592252927835\\
11.8657213404091	2.22525027281217\\
11.8635052695621	2.22457796431173\\
11.8612886331744	2.22390560376896\\
11.8590714309573	2.22323319117583\\
11.856853662622	2.22256072652426\\
11.8546353278792	2.22188820980619\\
11.8524164264396	2.22121564101357\\
11.8501969580136	2.22054302013832\\
11.8479769223113	2.21987034717238\\
11.8457563190426	2.21919762210767\\
11.8435351479173	2.21852484493614\\
11.8413134086449	2.2178520156497\\
11.8390911009347	2.21717913424028\\
11.8368682244957	2.21650620069981\\
11.8346447790367	2.2158332150202\\
11.8324207642664	2.21516017719338\\
11.8301961798931	2.21448708721127\\
11.827971025625	2.21381394506577\\
11.82574530117	2.21314075074881\\
11.8235190062358	2.21246750425229\\
11.82129214053	2.21179420556813\\
11.8190647037597	2.21112085468824\\
11.8168366956321	2.21044745160451\\
11.8146081158538	2.20977399630887\\
11.8123789641316	2.2091004887932\\
11.8101492401717	2.20842692904942\\
11.8079189436803	2.20775331706941\\
11.8056880743632	2.20707965284508\\
11.8034566319263	2.20640593636832\\
11.8012246160749	2.20573216763103\\
11.7989920265142	2.20505834662509\\
11.7967588629492	2.2043844733424\\
11.7945251250847	2.20371054777485\\
11.7922908126253	2.20303656991431\\
11.7900559252753	2.20236253975268\\
11.7878204627387	2.20168845728184\\
11.7855844247193	2.20101432249366\\
11.783347810921	2.20034013538002\\
11.7811106210469	2.19966589593281\\
11.7788728548003	2.19899160414389\\
11.7766345118841	2.19831726000514\\
11.7743955920011	2.19764286350843\\
11.7721560948537	2.19696841464563\\
11.7699160201442	2.1962939134086\\
11.7676753675746	2.19561935978921\\
11.7654341368466	2.19494475377932\\
11.7631923276618	2.1942700953708\\
11.7609499397217	2.1935953845555\\
11.7587069727271	2.19292062132529\\
11.7564634263791	2.19224580567201\\
11.7542193003782	2.19157093758751\\
11.7519745944249	2.19089601706366\\
11.7497293082193	2.19022104409231\\
11.7474834414613	2.18954601866529\\
11.7452369938507	2.18887094077446\\
11.7429899650869	2.18819581041165\\
11.7407423548691	2.18752062756872\\
11.7384941628964	2.18684539223751\\
11.7362453888675	2.18617010440984\\
11.733996032481	2.18549476407756\\
11.7317460934351	2.1848193712325\\
11.7294955714279	2.1841439258665\\
11.7272444661573	2.18346842797138\\
11.7249927773208	2.18279287753898\\
11.7227405046159	2.18211727456112\\
11.7204876477395	2.18144161902962\\
11.7182342063887	2.18076591093632\\
11.71598018026	2.18009015027302\\
11.7137255690499	2.17941433703156\\
11.7114703724546	2.17873847120374\\
11.70921459017	2.17806255278139\\
11.7069582218918	2.17738658175632\\
11.7047012673155	2.17671055812033\\
11.7024437261363	2.17603448186525\\
11.7001855980492	2.17535835298286\\
11.697926882749	2.174682171465\\
11.6956675799302	2.17400593730344\\
11.693407689287	2.17332965049001\\
11.6911472105135	2.1726533110165\\
11.6888861433035	2.1719769188747\\
11.6866244873506	2.17130047405642\\
11.6843622423481	2.17062397655344\\
11.6820994079891	2.16994742635757\\
11.6798359839663	2.16927082346059\\
11.6775719699725	2.16859416785429\\
11.6753073657	2.16791745953045\\
11.6730421708409	2.16724069848087\\
11.670776385087	2.16656388469732\\
11.6685100081301	2.16588701817159\\
11.6662430396616	2.16521009889545\\
11.6639754793725	2.16453312686068\\
11.6617073269538	2.16385610205907\\
11.6594385820961	2.16317902448237\\
11.65716924449	2.16250189412236\\
11.6548993138255	2.16182471097082\\
11.6526287897927	2.16114747501951\\
11.6503576720812	2.16047018626019\\
11.6480859603806	2.15979284468463\\
11.6458136543799	2.15911545028459\\
11.6435407537682	2.15843800305183\\
11.6412672582342	2.15776050297811\\
11.6389931674664	2.15708295005519\\
11.636718481153	2.15640534427481\\
11.634443198982	2.15572768562873\\
11.6321673206411	2.15504997410871\\
11.629890845818	2.15437220970648\\
11.6276137741997	2.1536943924138\\
11.6253361054734	2.15301652222241\\
11.6230578393258	2.15233859912405\\
11.6207789754433	2.15166062311046\\
11.6184995135124	2.15098259417339\\
11.6162194532189	2.15030451230456\\
11.6139387942487	2.14962637749572\\
11.6116575362873	2.14894818973859\\
11.60937567902	2.14826994902491\\
11.6070932221318	2.14759165534641\\
11.6048101653075	2.1469133086948\\
11.6025265082316	2.14623490906183\\
11.6002422505885	2.1455564564392\\
11.5979573920621	2.14487795081865\\
11.5956719323362	2.14419939219189\\
11.5933858710944	2.14352078055064\\
11.59109920802	2.14284211588661\\
11.5888119427959	2.14216339819152\\
11.5865240751051	2.14148462745707\\
11.58423560463	2.14080580367499\\
11.5819465310529	2.14012692683697\\
11.5796568540558	2.13944799693472\\
11.5773665733206	2.13876901395995\\
11.5750756885287	2.13808997790435\\
11.5727841993614	2.13741088875963\\
11.5704921054999	2.13673174651749\\
11.5681994066247	2.13605255116961\\
11.5659061024165	2.1353733027077\\
11.5636121925556	2.13469400112344\\
11.5613176767218	2.13401464640854\\
11.5590225545951	2.13333523855466\\
11.5567268258549	2.13265577755351\\
11.5544304901804	2.13197626339676\\
11.5521335472507	2.1312966960761\\
11.5498359967445	2.1306170755832\\
11.5475378383403	2.12993740190975\\
11.5452390717163	2.12925767504743\\
11.5429396965505	2.12857789498789\\
11.5406397125206	2.12789806172283\\
11.5383391193041	2.12721817524391\\
11.5360379165783	2.12653823554279\\
11.5337361040199	2.12585824261115\\
11.5314336813058	2.12517819644065\\
11.5291306481124	2.12449809702295\\
11.5268270041159	2.12381794434971\\
11.5245227489921	2.12313773841259\\
11.5222178824168	2.12245747920325\\
11.5199124040654	2.12177716671334\\
11.5176063136129	2.12109680093452\\
11.5152996107344	2.12041638185844\\
11.5129922951044	2.11973590947674\\
11.5106843663973	2.11905538378107\\
11.5083758242872	2.11837480476308\\
11.506066668448	2.11769417241441\\
11.5037568985532	2.11701348672671\\
11.5014465142762	2.1163327476916\\
11.4991355152901	2.11565195530073\\
11.4968239012677	2.11497110954573\\
11.4945116718815	2.11429021041824\\
11.4921988268037	2.11360925790989\\
11.4898853657066	2.1129282520123\\
11.4875712882616	2.11224719271711\\
11.4852565941405	2.11156608001594\\
11.4829412830145	2.11088491390041\\
11.4806253545544	2.11020369436214\\
11.478308808431	2.10952242139276\\
11.4759916443148	2.10884109498387\\
11.473673861876	2.10815971512711\\
11.4713554607844	2.10747828181407\\
11.4690364407097	2.10679679503638\\
11.4667168013214	2.10611525478563\\
11.4643965422884	2.10543366105344\\
11.4620756632798	2.10475201383142\\
11.4597541639641	2.10407031311117\\
11.4574320440096	2.10338855888428\\
11.4551093030844	2.10270675114237\\
11.4527859408563	2.10202488987702\\
11.4504619569928	2.10134297507983\\
11.4481373511611	2.1006610067424\\
11.4458121230284	2.09997898485631\\
11.4434862722612	2.09929690941316\\
11.4411597985261	2.09861478040454\\
11.4388327014892	2.09793259782203\\
11.4365049808166	2.09725036165721\\
11.4341766361738	2.09656807190167\\
11.4318476672261	2.09588572854699\\
11.4295180736389	2.09520333158474\\
11.4271878550768	2.0945208810065\\
11.4248570112045	2.09383837680384\\
11.4225255416862	2.09315581896834\\
11.4201934461861	2.09247320749157\\
11.4178607243678	2.09179054236509\\
11.415527375895	2.09110782358046\\
11.4131934004307	2.09042505112927\\
11.4108587976379	2.08974222500305\\
11.4085235671795	2.08905934519338\\
11.4061877087176	2.08837641169182\\
11.4038512219145	2.08769342448991\\
11.4015141064321	2.08701038357922\\
11.3991763619319	2.08632728895129\\
11.3968379880752	2.08564414059767\\
11.3944989845232	2.08496093850992\\
11.3921593509365	2.08427768267958\\
11.3898190869757	2.08359437309819\\
11.3874781923009	2.0829110097573\\
11.3851366665722	2.08222759264844\\
11.3827945094492	2.08154412176315\\
11.3804517205913	2.08086059709297\\
11.3781082996576	2.08017701862943\\
11.3757642463069	2.07949338636407\\
11.3734195601979	2.07880970028841\\
11.3710742409889	2.07812596039399\\
11.3687282883377	2.07744216667232\\
11.3663817019022	2.07675831911494\\
11.3640344813399	2.07607441771336\\
11.3616866263079	2.07539046245911\\
11.3593381364631	2.0747064533437\\
11.3569890114621	2.07402239035864\\
11.3546392509614	2.07333827349546\\
11.3522888546169	2.07265410274566\\
11.3499378220845	2.07196987810076\\
11.3475861530196	2.07128559955226\\
11.3452338470775	2.07060126709167\\
11.3428809039131	2.06991688071048\\
11.3405273231812	2.06923244040022\\
11.3381731045361	2.06854794615236\\
11.3358182476318	2.06786339795842\\
11.3334627521223	2.06717879580989\\
11.3311066176611	2.06649413969825\\
11.3287498439014	2.06580942961501\\
11.3263924304962	2.06512466555165\\
11.3240343770983	2.06443984749966\\
11.3216756833599	2.06375497545053\\
11.3193163489333	2.06307004939574\\
11.3169563734704	2.06238506932677\\
11.3145957566225	2.0617000352351\\
11.3122344980412	2.06101494711222\\
11.3098725973772	2.06032980494958\\
11.3075100542815	2.05964460873868\\
11.3051468684043	2.05895935847097\\
11.3027830393958	2.05827405413794\\
11.3004185669059	2.05758869573104\\
11.2980534505841	2.05690328324175\\
11.2956876900797	2.05621781666152\\
11.2933212850418	2.05553229598182\\
11.290954235119	2.05484672119411\\
11.2885865399597	2.05416109228985\\
11.2862181992121	2.05347540926048\\
11.283849212524	2.05278967209747\\
11.2814795795429	2.05210388079227\\
11.2791092999163	2.05141803533632\\
11.2767383732909	2.05073213572107\\
11.2743667993136	2.05004618193797\\
11.2719945776306	2.04936017397847\\
11.2696217078882	2.04867411183399\\
11.2672481897321	2.04798799549599\\
11.264874022808	2.04730182495589\\
11.2624992067609	2.04661560020514\\
11.2601237412359	2.04592932123516\\
11.2577476258776	2.0452429880374\\
11.2553708603304	2.04455660060327\\
11.2529934442383	2.0438701589242\\
11.2506153772452	2.04318366299163\\
11.2482366589945	2.04249711279697\\
11.2458572891295	2.04181050833164\\
11.243477267293	2.04112384958706\\
11.2410965931276	2.04043713655465\\
11.2387152662756	2.03975036922583\\
11.2363332863791	2.039063547592\\
11.2339506530799	2.03837667164458\\
11.2315673660192	2.03768974137497\\
11.2291834248383	2.03700275677459\\
11.226798829178	2.03631571783484\\
11.2244135786789	2.03562862454711\\
11.2220276729811	2.03494147690282\\
11.2196411117247	2.03425427489336\\
11.2172538945493	2.03356701851012\\
11.2148660210942	2.0328797077445\\
11.2124774909986	2.03219234258789\\
11.2100883039011	2.03150492303169\\
11.2076984594403	2.03081744906727\\
11.2053079572543	2.03012992068604\\
11.2029167969809	2.02944233787936\\
11.2005249782578	2.02875470063863\\
11.1981325007222	2.02806700895522\\
11.1957393640111	2.02737926282052\\
11.1933455677612	2.02669146222589\\
11.1909511116087	2.02600360716271\\
11.1885559951898	2.02531569762235\\
11.1861602181403	2.02462773359619\\
11.1837637800955	2.02393971507559\\
11.1813666806907	2.02325164205191\\
11.1789689195608	2.02256351451653\\
11.1765704963402	2.0218753324608\\
11.1741714106633	2.02118709587607\\
11.171771662164	2.02049880475372\\
11.1693712504759	2.01981045908509\\
11.1669701752324	2.01912205886154\\
11.1645684360665	2.01843360407441\\
11.162166032611	2.01774509471507\\
11.1597629644983	2.01705653077485\\
11.1573592313604	2.0163679122451\\
11.1549548328294	2.01567923911717\\
11.1525497685365	2.01499051138238\\
11.1501440381132	2.0143017290321\\
11.1477376411901	2.01361289205764\\
11.1453305773981	2.01292400045035\\
11.1429228463673	2.01223505420155\\
11.1405144477277	2.01154605330259\\
11.138105381109	2.01085699774478\\
11.1356956461405	2.01016788751945\\
11.1332852424514	2.00947872261793\\
11.1308741696703	2.00878950303154\\
11.1284624274257	2.00810022875161\\
11.1260500153457	2.00741089976944\\
11.1236369330582	2.00672151607636\\
11.1212231801907	2.00603207766368\\
11.1188087563702	2.00534258452271\\
11.1163936612238	2.00465303664476\\
11.113977894378	2.00396343402113\\
11.1115614554591	2.00327377664315\\
11.109144344093	2.0025840645021\\
11.1067265599053	2.0018942975893\\
11.1043081025215	2.00120447589603\\
11.1018889715664	2.00051459941361\\
11.0994691666648	1.99982466813331\\
11.0970486874411	1.99913468204645\\
11.0946275335193	1.99844464114429\\
11.0922057045233	1.99775454541815\\
11.0897832000764	1.99706439485931\\
11.0873600198018	1.99637418945904\\
11.0849361633222	1.99568392920863\\
11.0825116302603	1.99499361409937\\
11.0800864202381	1.99430324412253\\
11.0776605328775	1.99361281926938\\
11.0752339678001	1.99292233953121\\
11.0728067246271	1.99223180489929\\
11.0703788029794	1.99154121536489\\
11.0679502024776	1.99085057091926\\
11.065520922742	1.99015987155369\\
11.0630909633925	1.98946911725944\\
11.0606603240488	1.98877830802776\\
11.0582290043302	1.98808744384992\\
11.0557970038556	1.98739652471718\\
11.0533643222439	1.98670555062078\\
11.0509309591132	1.98601452155199\\
11.0484969140817	1.98532343750206\\
11.0460621867671	1.98463229846224\\
11.0436267767867	1.98394110442376\\
11.0411906837577	1.98324985537789\\
11.0387539072967	1.98255855131586\\
11.0363164470203	1.98186719222892\\
11.0338783025445	1.98117577810829\\
11.0314394734851	1.98048430894523\\
11.0289999594575	1.97979278473096\\
11.0265597600769	1.97910120545672\\
11.0241188749581	1.97840957111374\\
11.0216773037156	1.97771788169324\\
11.0192350459635	1.97702613718646\\
11.0167921013157	1.97633433758461\\
11.0143484693857	1.97564248287893\\
11.0119041497866	1.97495057306062\\
11.0094591421313	1.97425860812092\\
11.0070134460323	1.97356658805103\\
11.0045670611019	1.97287451284216\\
11.0021199869519	1.97218238248554\\
10.9996722231938	1.97149019697237\\
10.9972237694389	1.97079795629385\\
10.994774625298	1.9701056604412\\
10.9923247903817	1.96941330940562\\
10.9898742643003	1.9687209031783\\
10.9874230466635	1.96802844175045\\
10.9849711370811	1.96733592511326\\
10.9825185351622	1.96664335325793\\
10.9800652405158	1.96595072617565\\
10.9776112527503	1.96525804385762\\
10.9751565714741	1.96456530629501\\
10.972701196295	1.96387251347902\\
10.9702451268207	1.96317966540084\\
10.9677883626584	1.96248676205164\\
10.9653309034149	1.9617938034226\\
10.9628727486969	1.96110078950491\\
10.9604138981106	1.96040772028973\\
10.9579543512619	1.95971459576824\\
10.9554941077565	1.95902141593162\\
10.9530331671994	1.95832818077103\\
10.9505715291956	1.95763489027764\\
10.9481091933498	1.95694154444262\\
10.9456461592661	1.95624814325713\\
10.9431824265484	1.95555468671233\\
10.9407179948002	1.95486117479937\\
10.9382528636248	1.95416760750942\\
10.9357870326252	1.95347398483363\\
10.9333205014037	1.95278030676316\\
10.9308532695626	1.95208657328915\\
10.9283853367038	1.95139278440275\\
10.9259167024288	1.95069894009511\\
10.9234473663389	1.95000504035737\\
10.9209773280347	1.94931108518067\\
10.9185065871169	1.94861707455616\\
10.9160351431855	1.94792300847497\\
10.9135629958405	1.94722888692824\\
10.9110901446813	1.9465347099071\\
10.908616589307	1.94584047740268\\
10.9061423293165	1.94514618940611\\
10.9036673643081	1.94445184590852\\
10.9011916938801	1.94375744690103\\
10.89871531763	1.94306299237477\\
10.8962382351555	1.94236848232085\\
10.8937604460535	1.9416739167304\\
10.8912819499209	1.94097929559453\\
10.8888027463538	1.94028461890436\\
10.8863228349485	1.93958988665099\\
10.8838422153007	1.93889509882554\\
10.8813608870055	1.93820025541911\\
10.8788788496581	1.93750535642282\\
10.8763961028532	1.93681040182776\\
10.8739126461849	1.93611539162504\\
10.8714284792473	1.93542032580575\\
10.868943601634	1.93472520436099\\
10.8664580129382	1.93403002728186\\
10.8639717127529	1.93333479455945\\
10.8614847006706	1.93263950618485\\
10.8589969762836	1.93194416214915\\
10.8565085391836	1.93124876244344\\
10.8540193889622	1.93055330705879\\
10.8515295252106	1.92985779598629\\
10.8490389475195	1.92916222921703\\
10.8465476554795	1.92846660674208\\
10.8440556486806	1.92777092855251\\
10.8415629267125	1.9270751946394\\
10.8390694891647	1.92637940499382\\
10.8365753356262	1.92568355960683\\
10.8340804656858	1.92498765846952\\
10.8315848789316	1.92429170157293\\
10.8290885749518	1.92359568890814\\
10.8265915533339	1.9228996204662\\
10.8240938136652	1.92220349623817\\
10.8215953555326	1.92150731621511\\
10.8190961785227	1.92081108038808\\
10.8165962822216	1.92011478874812\\
10.8140956662152	1.91941844128628\\
10.8115943300891	1.91872203799361\\
10.8090922734282	1.91802557886116\\
10.8065894958175	1.91732906387997\\
10.8040859968412	1.91663249304109\\
10.8015817760835	1.91593586633554\\
10.7990768331279	1.91523918375437\\
10.796571167558	1.91454244528862\\
10.7940647789565	1.9138456509293\\
10.7915576669062	1.91314880066746\\
10.7890498309892	1.91245189449413\\
10.7865412707875	1.91175493240032\\
10.7840319858826	1.91105791437706\\
10.7815219758555	1.91036084041538\\
10.7790112402872	1.9096637105063\\
10.7764997787581	1.90896652464082\\
10.7739875908481	1.90826928280997\\
10.7714746761371	1.90757198500476\\
10.7689610342043	1.90687463121619\\
10.7664466646288	1.90617722143529\\
10.7639315669891	1.90547975565305\\
10.7614157408634	1.90478223386048\\
10.7588991858297	1.90408465604858\\
10.7563819014655	1.90338702220835\\
10.7538638873478	1.90268933233079\\
10.7513451430535	1.90199158640689\\
10.748825668159	1.90129378442765\\
10.7463054622402	1.90059592638406\\
10.7437845248729	1.8998980122671\\
10.7412628556323	1.89920004206777\\
10.7387404540935	1.89850201577705\\
10.7362173198308	1.89780393338592\\
10.7336934524186	1.89710579488536\\
10.7311688514306	1.89640760026635\\
10.7286435164403	1.89570934951987\\
10.7261174470207	1.89501104263688\\
10.7235906427445	1.89431267960837\\
10.7210631031841	1.89361426042529\\
10.7185348279114	1.89291578507862\\
10.716005816498	1.89221725355933\\
10.713476068515	1.89151866585837\\
10.7109455835334	1.8908200219667\\
10.7084143611235	1.89012132187529\\
10.7058824008554	1.88942256557509\\
10.7033497022989	1.88872375305705\\
10.7008162650232	1.88802488431213\\
10.6982820885973	1.88732595933128\\
10.6957471725898	1.88662697810543\\
10.6932115165689	1.88592794062555\\
10.6906751201023	1.88522884688256\\
10.6881379827575	1.88452969686742\\
10.6856001041016	1.88383049057106\\
10.6830614837013	1.88313122798441\\
10.6805221211227	1.88243190909842\\
10.6779820159319	1.88173253390401\\
10.6754411676944	1.88103310239211\\
10.6728995759753	1.88033361455366\\
10.6703572403393	1.87963407037957\\
10.667814160351	1.87893446986076\\
10.6652703355743	1.87823481298817\\
10.6627257655728	1.87753509975271\\
10.6601804499097	1.87683533014529\\
10.6576343881479	1.87613550415684\\
10.65508757985	1.87543562177826\\
10.6525400245778	1.87473568300046\\
10.6499917218933	1.87403568781434\\
10.6474426713575	1.87333563621083\\
10.6448928725316	1.87263552818081\\
10.6423423249761	1.8719353637152\\
10.639791028251	1.87123514280488\\
10.6372389819162	1.87053486544076\\
10.634686185531	1.86983453161373\\
10.6321326386544	1.86913414131468\\
10.6295783408451	1.86843369453451\\
10.6270232916612	1.8677331912641\\
10.6244674906605	1.86703263149433\\
10.6219109374006	1.86633201521609\\
10.6193536314383	1.86563134242026\\
10.6167955723304	1.86493061309773\\
10.6142367596331	1.86422982723936\\
10.6116771929024	1.86352898483603\\
10.6091168716935	1.86282808587861\\
10.6065557955617	1.86212713035798\\
10.6039939640616	1.861426118265\\
10.6014313767476	1.86072504959054\\
10.5988680331734	1.86002392432546\\
10.5963039328926	1.85932274246062\\
10.5937390754583	1.85862150398689\\
10.5911734604233	1.8579202088951\\
10.5886070873397	1.85721885717614\\
10.5860399557596	1.85651744882083\\
10.5834720652345	1.85581598382005\\
10.5809034153154	1.85511446216462\\
10.5783340055531	1.8544128838454\\
10.5757638354979	1.85371124885324\\
10.5731929046998	1.85300955717897\\
10.5706212127082	1.85230780881342\\
10.5680487590724	1.85160600374745\\
10.5654755433409	1.85090414197188\\
10.5629015650621	1.85020222347755\\
10.560326823784	1.84950024825528\\
10.5577513190541	1.8487982162959\\
10.5551750504194	1.84809612759025\\
10.5525980174268	1.84739398212913\\
10.5500202196224	1.84669177990338\\
10.5474416565522	1.84598952090381\\
10.5448623277618	1.84528720512123\\
10.5422822327961	1.84458483254647\\
10.5397013711999	1.84388240317033\\
10.5371197425175	1.84317991698363\\
10.5345373462927	1.84247737397717\\
10.531954182069	1.84177477414176\\
10.5293702493895	1.84107211746819\\
10.5267855477969	1.84036940394728\\
10.5242000768333	1.83966663356982\\
10.5216138360406	1.8389638063266\\
10.5190268249602	1.83826092220843\\
10.5164390431333	1.83755798120609\\
10.5138504901002	1.83685498331037\\
10.5112611654013	1.83615192851205\\
10.5086710685764	1.83544881680193\\
10.5060801991648	1.83474564817079\\
10.5034885567054	1.83404242260941\\
10.5008961407369	1.83333914010856\\
10.4983029507973	1.83263580065902\\
10.4957089864244	1.83193240425157\\
10.4931142471555	1.83122895087698\\
10.4905187325275	1.83052544052602\\
10.4879224420768	1.82982187318944\\
10.4853253753396	1.82911824885803\\
10.4827275318514	1.82841456752254\\
10.4801289111474	1.82771082917373\\
10.4775295127626	1.82700703380236\\
10.4749293362313	1.82630318139918\\
10.4723283810874	1.82559927195495\\
10.4697266468646	1.82489530546042\\
10.4671241330959	1.82419128190635\\
10.464520839314	1.82348720128346\\
10.4619167650514	1.82278306358252\\
10.4593119098397	1.82207886879426\\
10.4567062732106	1.82137461690942\\
10.454099854695	1.82067030791874\\
10.4514926538235	1.81996594181295\\
10.4488846701264	1.81926151858279\\
10.4462759031333	1.818557038219\\
10.4436663523737	1.81785250071229\\
10.4410560173764	1.81714790605339\\
10.43844489767	1.81644325423303\\
10.4358329927825	1.81573854524193\\
10.4332203022415	1.8150337790708\\
10.4306068255742	1.81432895571038\\
10.4279925623075	1.81362407515136\\
10.4253775119677	1.81291913738447\\
10.4227616740807	1.81221414240041\\
10.4201450481721	1.81150909018989\\
10.4175276337668	1.81080398074362\\
10.4149094303896	1.8100988140523\\
10.4122904375647	1.80939359010663\\
10.4096706548158	1.80868830889732\\
10.4070500816664	1.80798297041505\\
10.4044287176392	1.80727757465052\\
10.4018065622569	1.80657212159443\\
10.3991836150414	1.80586661123746\\
10.3965598755144	1.8051610435703\\
10.3939353431971	1.80445541858364\\
10.3913100176102	1.80374973626815\\
10.3886838982741	1.80304399661453\\
10.3860569847086	1.80233819961344\\
10.3834292764332	1.80163234525557\\
10.3808007729669	1.80092643353159\\
10.3781714738283	1.80022046443216\\
10.3755413785355	1.79951443794797\\
10.3729104866062	1.79880835406966\\
10.3702787975577	1.79810221278792\\
10.3676463109069	1.7973960140934\\
10.36501302617	1.79668975797676\\
10.3623789428631	1.79598344442867\\
10.3597440605016	1.79527707343976\\
10.3571083786006	1.7945706450007\\
10.3544718966747	1.79386415910215\\
10.3518346142382	1.79315761573473\\
10.3491965308046	1.79245101488912\\
10.3465576458875	1.79174435655594\\
10.3439179589995	1.79103764072583\\
10.3412774696531	1.79033086738945\\
10.3386361773603	1.78962403653742\\
10.3359940816325	1.78891714816038\\
10.3333511819809	1.78821020224896\\
10.330707477916	1.78750319879379\\
10.3280629689481	1.7867961377855\\
10.3254176545868	1.78608901921472\\
10.3227715343415	1.78538184307206\\
10.320124607721	1.78467460934815\\
10.3174768742336	1.78396731803361\\
10.3148283333873	1.78325996911904\\
10.3121789846896	1.78255256259508\\
10.3095288276475	1.78184509845233\\
10.3068778617676	1.78113757668139\\
10.304226086556	1.78042999727287\\
10.3015735015185	1.77972236021738\\
10.2989201061601	1.77901466550553\\
10.2962658999857	1.7783069131279\\
10.2936108824996	1.7775991030751\\
10.2909550532057	1.77689123533773\\
10.2882984116074	1.77618330990637\\
10.2856409572076	1.77547532677162\\
10.2829826895087	1.77476728592406\\
10.280323608013	1.77405918735428\\
10.2776637122219	1.77335103105287\\
10.2750030016365	1.7726428170104\\
10.2723414757575	1.77193454521746\\
10.2696791340851	1.77122621566462\\
10.2670159761191	1.77051782834245\\
10.2643520013588	1.76980938324153\\
10.2616872093029	1.76910088035243\\
10.2590215994498	1.76839231966571\\
10.2563551712975	1.76768370117194\\
10.2536879243433	1.76697502486168\\
10.2510198580842	1.76626629072549\\
10.2483509720168	1.76555749875394\\
10.2456812656371	1.76484864893756\\
10.2430107384406	1.76413974126693\\
10.2403393899225	1.76343077573259\\
10.2376672195775	1.76272175232508\\
10.2349942268997	1.76201267103496\\
10.2323204113828	1.76130353185277\\
10.22964577252	1.76059433476904\\
10.2269703098043	1.75988507977432\\
10.2242940227278	1.75917576685915\\
10.2216169107824	1.75846639601406\\
10.2189389734595	1.75775696722958\\
10.2162602102499	1.75704748049625\\
10.2135806206442	1.75633793580458\\
10.2109002041322	1.75562833314511\\
10.2082189602035	1.75491867250835\\
10.2055368883471	1.75420895388483\\
10.2028539880515	1.75349917726507\\
10.2001702588047	1.75278934263957\\
10.1974857000944	1.75207944999887\\
10.1948003114076	1.75136949933345\\
10.1921140922311	1.75065949063384\\
10.189427042051	1.74994942389055\\
10.1867391603529	1.74923929909406\\
10.1840504466221	1.74852911623489\\
10.1813609003432	1.74781887530354\\
10.1786705210007	1.74710857629049\\
10.1759793080781	1.74639821918625\\
10.1732872610589	1.74568780398131\\
10.1705943794257	1.74497733066616\\
10.1679006626611	1.74426679923128\\
10.1652061102467	1.74355620966716\\
10.162510721664	1.74284556196429\\
10.1598144963938	1.74213485611313\\
10.1571174339166	1.74142409210418\\
10.1544195337123	1.7407132699279\\
10.1517207952602	1.74000238957477\\
10.1490212180394	1.73929145103527\\
10.1463208015283	1.73858045429985\\
10.1436195452048	1.73786939935898\\
10.1409174485465	1.73715828620313\\
10.1382145110304	1.73644711482276\\
10.1355107321329	1.73573588520833\\
10.1328061113301	1.73502459735029\\
10.1301006480974	1.7343132512391\\
10.1273943419101	1.73360184686521\\
10.1246871922425	1.73289038421907\\
10.1219791985688	1.73217886329112\\
10.1192703603625	1.73146728407182\\
10.1165606770967	1.73075564655159\\
10.113850148244	1.73004395072089\\
10.1111387732764	1.72933219657015\\
10.1084265516656	1.72862038408981\\
10.1057134828826	1.72790851327029\\
10.1029995663981	1.72719658410203\\
10.1002848016821	1.72648459657546\\
10.0975691882043	1.72577255068099\\
10.0948527254338	1.72506044640906\\
10.0921354128391	1.72434828375009\\
10.0894172498884	1.72363606269449\\
10.0866982360493	1.72292378323267\\
10.0839783707889	1.72221144535506\\
10.0812576535738	1.72149904905207\\
10.0785360838702	1.72078659431409\\
10.0758136611436	1.72007408113154\\
10.0730903848592	1.71936150949483\\
10.0703662544815	1.71864887939435\\
10.0676412694747	1.7179361908205\\
10.0649154293023	1.71722344376368\\
10.0621887334274	1.71651063821428\\
10.0594611813126	1.7157977741627\\
10.0567327724201	1.71508485159932\\
10.0540035062113	1.71437187051453\\
10.0512733821473	1.71365883089872\\
10.0485423996887	1.71294573274227\\
10.0458105582955	1.71223257603556\\
10.0430778574272	1.71151936076896\\
10.040344296543	1.71080608693285\\
10.0376098751012	1.71009275451761\\
10.0348745925599	1.7093793635136\\
10.0321384483766	1.70866591391119\\
10.0294014420082	1.70795240570074\\
10.0266635729113	1.70723883887263\\
10.0239248405418	1.7065252134172\\
10.021185244355	1.70581152932482\\
10.018444783806	1.70509778658585\\
10.0157034583491	1.70438398519063\\
10.0129612674382	1.70367012512951\\
10.0102182105267	1.70295620639286\\
10.0074742870675	1.702242228971\\
10.0047294965128	1.70152819285429\\
10.0019838383145	1.70081409803306\\
9.9992373119238	1.70009994449766\\
9.9964899167916	1.69938573223842\\
9.99374165236808	1.69867146124567\\
9.99099251810298	1.69795713150974\\
9.98824251344551	1.69724274302097\\
9.98549163784435	1.69652829576969\\
9.98273989074765	1.6958137897462\\
9.97998727160306	1.69509922494084\\
9.97723377985768	1.69438460134393\\
9.9744794149581	1.69366991894578\\
9.97172417635035	1.69295517773671\\
9.96896806347999	1.69224037770702\\
9.96621107579201	1.69152551884703\\
9.96345321273088	1.69081060114705\\
9.96069447374054	1.69009562459737\\
9.95793485826441	1.68938058918831\\
9.95517436574536	1.68866549491015\\
9.95241299562575	1.6879503417532\\
9.94965074734741	1.68723512970776\\
9.9468876203516	1.6865198587641\\
9.94412361407909	1.68580452891254\\
9.9413587279701	1.68508914014334\\
9.9385929614643	1.6843736924468\\
9.93582631400085	1.6836581858132\\
9.93305878501836	1.68294262023281\\
9.9302903739549	1.68222699569593\\
9.92752108024802	1.68151131219281\\
9.9247509033347	1.68079556971374\\
9.92197984265143	1.68007976824898\\
9.9192078976341	1.6793639077888\\
9.91643506771812	1.67864798832346\\
9.91366135233831	1.67793200984324\\
9.91088675092897	1.67721597233839\\
9.90811126292387	1.67649987579916\\
9.90533488775621	1.67578372021582\\
9.90255762485866	1.67506750557861\\
9.89977947366336	1.67435123187778\\
9.89700043360188	1.67363489910359\\
9.89422050410525	1.67291850724628\\
9.89143968460397	1.67220205629609\\
9.88865797452797	1.67148554624326\\
9.88587537330666	1.67076897707803\\
9.88309188036886	1.67005234879064\\
9.88030749514287	1.66933566137132\\
9.87752221705646	1.66861891481029\\
9.8747360455368	1.6679021090978\\
9.87194898001054	1.66718524422405\\
9.86916101990377	1.66646832017928\\
9.86637216464203	1.66575133695371\\
9.86358241365031	1.66503429453755\\
9.86079176635303	1.66431719292103\\
9.85800022217407	1.66360003209435\\
9.85520778053675	1.66288281204772\\
9.85241444086384	1.66216553277135\\
9.84962020257753	1.66144819425545\\
9.84682506509949	1.66073079649022\\
9.84402902785079	1.66001333946587\\
9.84123209025198	1.65929582317258\\
9.83843425172302	1.65857824760056\\
9.83563551168331	1.65786061274\\
9.83283586955172	1.65714291858109\\
9.83003532474653	1.65642516511401\\
9.82723387668545	1.65570735232896\\
9.82443152478566	1.65498948021612\\
9.82162826846374	1.65427154876566\\
9.81882410713574	1.65355355796777\\
9.8160190402171	1.65283550781262\\
9.81321306712274	1.65211739829038\\
9.81040618726698	1.65139922939123\\
9.80759840006358	1.65068100110533\\
9.80478970492575	1.64996271342285\\
9.8019801012661	1.64924436633395\\
9.79916958849669	1.64852595982879\\
9.79635816602901	1.64780749389754\\
9.79354583327397	1.64708896853034\\
9.79073258964191	1.64637038371735\\
9.78791843454259	1.64565173944872\\
9.78510336738521	1.64493303571459\\
9.78228738757839	1.64421427250512\\
9.77947049453016	1.64349544981044\\
9.77665268764801	1.6427765676207\\
9.77383396633881	1.64205762592604\\
9.77101433000889	1.64133862471658\\
9.76819377806396	1.64061956398247\\
9.7653723099092	1.63990044371383\\
9.76254992494917	1.63918126390079\\
9.75972662258787	1.63846202453347\\
9.75690240222871	1.637742725602\\
9.75407726327452	1.63702336709651\\
9.75125120512756	1.63630394900709\\
9.74842422718949	1.63558447132388\\
9.74559632886138	1.63486493403698\\
9.74276750954373	1.63414533713651\\
9.73993776863645	1.63342568061257\\
9.73710710553887	1.63270596445526\\
9.73427551964971	1.63198618865469\\
9.73144301036713	1.63126635320096\\
9.72860957708868	1.63054645808417\\
9.72577521921133	1.62982650329441\\
9.72293993613145	1.62910648882176\\
9.72010372724484	1.62838641465633\\
9.71726659194669	1.62766628078821\\
9.7144285296316	1.62694608720746\\
9.71158953969357	1.62622583390418\\
9.70874962152602	1.62550552086844\\
9.70590877452178	1.62478514809033\\
9.70306699807305	1.62406471555991\\
9.70022429157146	1.62334422326726\\
9.69738065440805	1.62262367120245\\
9.69453608597324	1.62190305935554\\
9.69169058565686	1.62118238771659\\
9.68884415284814	1.62046165627568\\
9.68599678693572	1.61974086502285\\
9.68314848730761	1.61902001394817\\
9.68029925335126	1.61829910304169\\
9.67744908445347	1.61757813229346\\
9.67459798000048	1.61685710169352\\
9.67174593937788	1.61613601123193\\
9.66889296197071	1.61541486089873\\
9.66603904716334	1.61469365068395\\
9.66318419433959	1.61397238057764\\
9.66032840288264	1.61325105056983\\
9.65747167217508	1.61252966065055\\
9.65461400159885	1.61180821080985\\
9.65175539053535	1.61108670103773\\
9.6488958383653	1.61036513132424\\
9.64603534446886	1.60964350165938\\
9.64317390822553	1.60892181203319\\
9.64031152901423	1.60820006243568\\
9.63744820621327	1.60747825285686\\
9.63458393920032	1.60675638328675\\
9.63171872735245	1.60603445371536\\
9.6288525700461	1.60531246413269\\
9.62598546665711	1.60459041452874\\
9.6231174165607	1.60386830489353\\
9.62024841913146	1.60314613521705\\
9.61737847374336	1.6024239054893\\
9.61450757976976	1.60170161570026\\
9.61163573658338	1.60097926583994\\
9.60876294355634	1.60025685589832\\
9.60588920006013	1.59953438586538\\
9.60301450546561	1.59881185573112\\
9.60013885914301	1.59808926548551\\
9.59726226046195	1.59736661511853\\
9.59438470879142	1.59664390462016\\
9.59150620349977	1.59592113398037\\
9.58862674395473	1.59519830318913\\
9.5857463295234	1.59447541223641\\
9.58286495957226	1.59375246111217\\
9.57998263346714	1.59302944980638\\
9.57709935057326	1.59230637830899\\
9.57421511025519	1.59158324660997\\
9.57132991187687	1.59086005469928\\
9.56844375480161	1.59013680256685\\
9.56555663839209	1.58941349020264\\
9.56266856201035	1.5886901175966\\
9.55977952501779	1.58796668473868\\
9.55688952677517	1.58724319161881\\
9.55399856664262	1.58651963822693\\
9.55110664397962	1.58579602455299\\
9.54821375814503	1.58507235058691\\
9.54531990849705	1.58434861631862\\
9.54242509439325	1.58362482173805\\
9.53952931519054	1.58290096683514\\
9.53663257024521	1.5821770515998\\
9.53373485891289	1.58145307602195\\
9.53083618054858	1.58072904009151\\
9.5279365345066	1.5800049437984\\
9.52503592014067	1.57928078713253\\
9.52213433680383	1.5785565700838\\
9.51923178384847	1.57783229264214\\
9.51632826062636	1.57710795479743\\
9.51342376648859	1.57638355653959\\
9.51051830078561	1.5756590978585\\
9.50761186286722	1.57493457874408\\
9.50470445208257	1.57420999918621\\
9.50179606778013	1.57348535917479\\
9.49888670930776	1.57276065869969\\
9.49597637601262	1.57203589775082\\
9.49306506724125	1.57131107631804\\
9.49015278233952	1.57058619439125\\
9.48723952065261	1.56986125196032\\
9.4843252815251	1.56913624901513\\
9.48141006430087	1.56841118554554\\
9.47849386832315	1.56768606154143\\
9.4755766929345	1.56696087699267\\
9.47265853747682	1.56623563188911\\
9.46973940129137	1.56551032622063\\
9.46681928371871	1.56478495997708\\
9.46389818409876	1.56405953314831\\
9.46097610177075	1.56333404572419\\
9.45805303607328	1.56260849769455\\
9.45512898634425	1.56188288904926\\
9.45220395192089	1.56115721977816\\
9.44927793213979	1.56043148987108\\
9.44635092633684	1.55970569931787\\
9.44342293384727	1.55897984810836\\
9.44049395400563	1.5582539362324\\
9.43756398614581	1.55752796367981\\
9.43463302960102	1.55680193044043\\
9.43170108370379	1.55607583650407\\
9.42876814778598	1.55534968186057\\
9.42583422117877	1.55462346649974\\
9.42289930321266	1.5538971904114\\
9.41996339321747	1.55317085358537\\
9.41702649052235	1.55244445601147\\
9.41408859445576	1.5517179976795\\
9.41114970434548	1.55099147857926\\
9.40820981951861	1.55026489870058\\
9.40526893930157	1.54953825803324\\
9.40232706302008	1.54881155656704\\
9.39938418999919	1.54808479429179\\
9.39644031956326	1.54735797119728\\
9.39349545103596	1.5466310872733\\
9.39054958374028	1.54590414250963\\
9.3876027169985	1.54517713689607\\
9.38465485013224	1.54445007042239\\
9.38170598246241	1.54372294307837\\
9.37875611330922	1.5429957548538\\
9.37580524199222	1.54226850573845\\
9.37285336783023	1.54154119572209\\
9.36990049014139	1.54081382479449\\
9.36694660824315	1.54008639294542\\
9.36399172145225	1.53935890016463\\
9.36103582908475	1.5386313464419\\
9.35807893045601	1.53790373176697\\
9.35512102488066	1.53717605612961\\
9.35216211167267	1.53644831951957\\
9.34920219014528	1.53572052192659\\
9.34624125961104	1.53499266334043\\
9.34327931938181	1.53426474375083\\
9.34031636876871	1.53353676314753\\
9.33735240708219	1.53280872152026\\
9.33438743363197	1.53208061885877\\
9.33142144772708	1.5313524551528\\
9.32845444867584	1.53062423039206\\
9.32548643578584	1.52989594456628\\
9.32251740836399	1.5291675976652\\
9.31954736571646	1.52843918967853\\
9.31657630714873	1.52771072059599\\
9.31360423196556	1.52698219040731\\
9.310631139471	1.52625359910218\\
9.30765702896837	1.52552494667034\\
9.30468189976029	1.52479623310147\\
9.30170575114866	1.52406745838529\\
9.29872858243465	1.5233386225115\\
9.29575039291874	1.52260972546981\\
9.29277118190066	1.52188076724989\\
9.28979094867943	1.52115174784147\\
9.28680969255335	1.52042266723421\\
9.28382741281999	1.51969352541781\\
9.28084410877622	1.51896432238196\\
9.27785977971815	1.51823505811634\\
9.27487442494118	1.51750573261063\\
9.27188804374	1.5167763458545\\
9.26890063540855	1.51604689783764\\
9.26591219924006	1.51531738854971\\
9.26292273452699	1.51458781798038\\
9.25993224056112	1.51385818611931\\
9.25694071663347	1.51312849295618\\
9.25394816203434	1.51239873848064\\
9.25095457605328	1.51166892268234\\
9.24795995797912	1.51093904555095\\
9.24496430709995	1.51020910707611\\
9.24196762270311	1.50947910724747\\
9.23896990407524	1.50874904605468\\
9.23597115050218	1.50801892348738\\
9.2329713612691	1.50728873953522\\
9.22997053566037	1.50655849418782\\
9.22696867295966	1.50582818743483\\
9.22396577244987	1.50509781926588\\
9.22096183341316	1.50436738967058\\
9.21795685513096	1.50363689863858\\
9.21495083688394	1.50290634615949\\
9.21194377795203	1.50217573222294\\
9.20893567761441	1.50144505681853\\
9.20592653514951	1.50071431993589\\
9.20291634983501	1.49998352156463\\
9.19990512094783	1.49925266169435\\
9.19689284776416	1.49852174031466\\
9.19387952955941	1.49779075741517\\
9.19086516560826	1.49705971298547\\
9.18784975518462	1.49632860701516\\
9.18483329756164	1.49559743949384\\
9.18181579201173	1.4948662104111\\
9.17879723780652	1.49413491975651\\
9.1757776342169	1.49340356751969\\
9.17275698051299	1.49267215369019\\
9.16973527596414	1.49194067825761\\
9.16671251983895	1.49120914121151\\
9.16368871140526	1.49047754254148\\
9.16066384993012	1.48974588223709\\
9.15763793467985	1.48901416028789\\
9.15461096491997	1.48828237668346\\
9.15158293991525	1.48755053141337\\
9.14855385892969	1.48681862446716\\
9.14552372122651	1.4860866558344\\
9.14249252606817	1.48535462550464\\
9.13946027271636	1.48462253346742\\
9.13642696043197	1.48389037971231\\
9.13339258847515	1.48315816422883\\
9.13035715610525	1.48242588700654\\
9.12732066258086	1.48169354803497\\
9.12428310715978	1.48096114730365\\
9.12124448909903	1.48022868480213\\
9.11820480765486	1.47949616051992\\
9.11516406208273	1.47876357444657\\
9.11212225163734	1.47803092657158\\
9.10907937557256	1.47729821688448\\
9.10603543314153	1.4765654453748\\
9.10299042359656	1.47583261203204\\
9.0999443461892	1.47509971684571\\
9.0968972001702	1.47436675980533\\
9.09384898478954	1.47363374090041\\
9.09079969929637	1.47290066012044\\
9.0877493429391	1.47216751745492\\
9.08469791496531	1.47143431289335\\
9.0816454146218	1.47070104642523\\
9.07859184115457	1.46996771804005\\
9.07553719380884	1.46923432772729\\
9.07248147182902	1.46850087547645\\
9.06942467445872	1.46776736127699\\
9.06636680094075	1.46703378511841\\
9.06330785051714	1.46630014699018\\
9.0602478224291	1.46556644688176\\
9.05718671591705	1.46483268478264\\
9.05412453022058	1.46409886068228\\
9.05106126457851	1.46336497457014\\
9.04799691822883	1.46263102643568\\
9.04493149040874	1.46189701626836\\
9.04186498035463	1.46116294405764\\
9.03879738730206	1.46042880979296\\
9.0357287104858	1.45969461346379\\
9.03265894913982	1.45896035505955\\
9.02958810249724	1.45822603456971\\
9.02651616979039	1.45749165198368\\
9.0234431502508	1.45675720729092\\
9.02036904310915	1.45602270048086\\
9.01729384759532	1.45528813154292\\
9.01421756293839	1.45455350046653\\
9.01114018836658	1.45381880724112\\
9.00806172310732	1.45308405185612\\
9.00498216638722	1.45234923430093\\
9.00190151743205	1.45161435456497\\
8.99881977546675	1.45087941263766\\
8.99573693971547	1.45014440850841\\
8.99265300940149	1.44940934216661\\
8.9895679837473	1.44867421360168\\
8.98648186197454	1.44793902280301\\
8.98339464330402	1.44720376976\\
8.98030632695573	1.44646845446205\\
8.97721691214881	1.44573307689854\\
8.9741263981016	1.44499763705886\\
8.97103478403156	1.4442621349324\\
8.96794206915535	1.44352657050854\\
8.96484825268878	1.44279094377665\\
8.96175333384682	1.44205525472611\\
8.95865731184361	1.44131950334629\\
8.95556018589242	1.44058368962657\\
8.95246195520572	1.4398478135563\\
8.94936261899512	1.43911187512485\\
8.94626217647136	1.43837587432158\\
8.94316062684439	1.43763981113584\\
8.94005796932325	1.43690368555699\\
8.93695420311618	1.43616749757438\\
8.93384932743055	1.43543124717735\\
8.93074334147288	1.43469493435526\\
8.92763624444884	1.43395855909743\\
8.92452803556326	1.4332221213932\\
8.9214187140201	1.43248562123192\\
8.91830827902247	1.43174905860291\\
8.91519672977262	1.43101243349549\\
8.91208406547195	1.430275745899\\
8.908970285321	1.42953899580275\\
8.90585538851944	1.42880218319607\\
8.90273937426608	1.42806530806826\\
8.89962224175889	1.42732837040865\\
8.89650399019495	1.42659137020653\\
8.89338461877049	1.42585430745121\\
8.89026412668086	1.42511718213201\\
8.88714251312055	1.4243799942382\\
8.88401977728319	1.4236427437591\\
8.88089591836153	1.42290543068399\\
8.87777093554744	1.42216805500217\\
8.87464482803194	1.42143061670291\\
8.87151759500515	1.42069311577551\\
8.86838923565635	1.41995555220924\\
8.8652597491739	1.41921792599337\\
8.86212913474532	1.41848023711719\\
8.85899739155723	1.41774248556995\\
8.85586451879538	1.41700467134094\\
8.85273051564463	1.41626679441941\\
8.84959538128897	1.41552885479462\\
8.84645911491149	1.41479085245582\\
8.8433217156944	1.41405278739228\\
8.84018318281904	1.41331465959325\\
8.83704351546583	1.41257646904796\\
8.83390271281434	1.41183821574566\\
8.83076077404322	1.4110998996756\\
8.82761769833023	1.41036152082701\\
8.82447348485225	1.40962307918913\\
8.82132813278526	1.40888457475118\\
8.81818164130435	1.4081460075024\\
8.81503400958371	1.407407377432\\
8.81188523679662	1.40666868452922\\
8.80873532211547	1.40592992878325\\
8.80558426471177	1.40519111018333\\
8.80243206375608	1.40445222871866\\
8.7992787184181	1.40371328437845\\
8.7961242278666	1.4029742771519\\
8.79296859126947	1.40223520702821\\
8.78981180779367	1.40149607399658\\
8.78665387660525	1.40075687804621\\
8.78349479686937	1.40001761916628\\
8.78033456775026	1.39927829734598\\
8.77717318841124	1.3985389125745\\
8.77401065801473	1.39779946484102\\
8.77084697572221	1.39705995413471\\
8.76768214069428	1.39632038044476\\
8.76451615209059	1.39558074376032\\
8.76134900906988	1.39484104407056\\
8.75818071078997	1.39410128136466\\
8.75501125640776	1.39336145563177\\
8.75184064507923	1.39262156686104\\
8.74866887595942	1.39188161504164\\
8.74549594820247	1.3911416001627\\
8.74232186096157	1.39040152221339\\
8.73914661338899	1.38966138118283\\
8.73597020463607	1.38892117706018\\
8.73279263385321	1.38818090983456\\
8.7296139001899	1.38744057949511\\
8.72643400279467	1.38670018603097\\
8.72325294081512	1.38595972943125\\
8.72007071339793	1.38521920968507\\
8.71688731968883	1.38447862678157\\
8.7137027588326	1.38373798070985\\
8.7105170299731	1.38299727145902\\
8.70733013225323	1.3822564990182\\
8.70414206481495	1.3815156633765\\
8.70095282679929	1.380774764523\\
8.69776241734632	1.38003380244681\\
8.69457083559515	1.37929277713703\\
8.69137808068398	1.37855168858274\\
8.68818415175	1.37781053677304\\
8.68498904792951	1.377069321697\\
8.68179276835782	1.37632804334372\\
8.67859531216928	1.37558670170226\\
8.67539667849731	1.37484529676171\\
8.67219686647436	1.37410382851112\\
8.66899587523192	1.37336229693957\\
8.66579370390052	1.37262070203613\\
8.66259035160971	1.37187904378984\\
8.65938581748811	1.37113732218977\\
8.65618010066336	1.37039553722497\\
8.65297320026214	1.36965368888449\\
8.64976511541013	1.36891177715736\\
8.64655584523208	1.36816980203265\\
8.64334538885177	1.36742776349938\\
8.64013374539197	1.36668566154658\\
8.63692091397451	1.3659434961633\\
8.63370689372024	1.36520126733855\\
8.63049168374902	1.36445897506136\\
8.62727528317975	1.36371661932075\\
8.62405769113033	1.36297420010574\\
8.62083890671771	1.36223171740533\\
8.61761892905782	1.36148917120855\\
8.61439775726564	1.36074656150439\\
8.61117539045515	1.36000388828186\\
8.60795182773933	1.35926115152995\\
8.60472706823021	1.35851835123767\\
8.60150111103879	1.35777548739399\\
8.59827395527509	1.35703255998791\\
8.59504560004816	1.35628956900842\\
8.59181604446603	1.35554651444448\\
8.58858528763575	1.35480339628509\\
8.58535332866337	1.3540602145192\\
8.58212016665393	1.3533169691358\\
8.57888580071149	1.35257366012384\\
8.57565022993909	1.35183028747228\\
8.57241345343878	1.3510868511701\\
8.5691754703116	1.35034335120622\\
8.56593627965759	1.34959978756962\\
8.56269588057578	1.34885616024924\\
8.55945427216418	1.34811246923401\\
8.5562114535198	1.34736871451288\\
8.55296742373865	1.34662489607478\\
8.54972218191571	1.34588101390865\\
8.54647572714495	1.34513706800341\\
8.54322805851931	1.34439305834798\\
8.53997917513075	1.34364898493129\\
8.53672907607016	1.34290484774226\\
8.53347776042746	1.34216064676979\\
8.5302252272915	1.34141638200279\\
8.52697147575015	1.34067205343017\\
8.52371650489023	1.33992766104083\\
8.52046031379752	1.33918320482367\\
8.51720290155681	1.33843868476758\\
8.51394426725182	1.33769410086145\\
8.51068440996527	1.33694945309417\\
8.50742332877883	1.33620474145462\\
8.50416102277313	1.33545996593167\\
8.50089749102778	1.3347151265142\\
8.49763273262134	1.33397022319108\\
8.49436674663134	1.33322525595118\\
8.49109953213426	1.33248022478336\\
8.48783108820555	1.33173512967648\\
8.4845614139196	1.3309899706194\\
8.48129050834976	1.33024474760095\\
8.47801837056833	1.32949946061\\
8.47474499964658	1.32875410963538\\
8.47147039465471	1.32800869466593\\
8.46819455466187	1.32726321569049\\
8.46491747873616	1.32651767269789\\
8.46163916594462	1.32577206567695\\
8.45835961535325	1.32502639461651\\
8.45507882602696	1.32428065950537\\
8.45179679702964	1.32353486033235\\
8.44851352742409	1.32278899708627\\
8.44522901627205	1.32204306975593\\
8.44194326263421	1.32129707833013\\
8.43865626557017	1.32055102279768\\
8.43536802413848	1.31980490314736\\
8.43207853739662	1.31905871936798\\
8.42878780440099	1.31831247144831\\
8.42549582420693	1.31756615937714\\
8.42220259586869	1.31681978314325\\
8.41890811843944	1.31607334273541\\
8.41561239097131	1.3153268381424\\
8.4123154125153	1.31458026935298\\
8.40901718212136	1.31383363635591\\
8.40571769883836	1.31308693913994\\
8.40241696171406	1.31234017769385\\
8.39911496979516	1.31159335200637\\
8.39581172212725	1.31084646206626\\
8.39250721775486	1.31009950786224\\
8.38920145572141	1.30935248938307\\
8.38589443506921	1.30860540661747\\
8.38258615483951	1.30785825955418\\
8.37927661407245	1.30711104818192\\
8.37596581180707	1.3063637724894\\
8.37265374708131	1.30561643246535\\
8.36934041893201	1.30486902809849\\
8.36602582639491	1.30412155937751\\
8.36270996850464	1.30337402629112\\
8.35939284429474	1.30262642882803\\
8.35607445279762	1.30187876697692\\
8.35275479304459	1.3011310407265\\
8.34943386406586	1.30038325006544\\
8.34611166489051	1.29963539498243\\
8.34278819454651	1.29888747546614\\
8.33946345206073	1.29813949150526\\
8.33613743645889	1.29739144308846\\
8.33281014676562	1.29664333020439\\
8.32948158200442	1.29589515284172\\
8.32615174119765	1.2951469109891\\
8.32282062336658	1.2943986046352\\
8.31948822753132	1.29365023376865\\
8.31615455271087	1.29290179837811\\
8.31281959792309	1.2921532984522\\
8.30948336218471	1.29140473397958\\
8.30614584451135	1.29065610494886\\
8.30280704391745	1.28990741134867\\
8.29946695941636	1.28915865316764\\
8.29612559002025	1.28840983039439\\
8.29278293474018	1.28766094301752\\
8.28943899258605	1.28691199102564\\
8.28609376256663	1.28616297440737\\
8.28274724368953	1.2854138931513\\
8.27939943496123	1.28466474724602\\
8.27605033538703	1.28391553668013\\
8.27269994397112	1.28316626144221\\
8.2693482597165	1.28241692152084\\
8.26599528162504	1.28166751690461\\
8.26264100869744	1.28091804758208\\
8.25928543993326	1.28016851354182\\
8.25592857433087	1.2794189147724\\
8.2525704108875	1.27866925126237\\
8.24921094859923	1.2779195230003\\
8.24585018646093	1.27716972997472\\
8.24248812346634	1.27641987217419\\
8.23912475860803	1.27566994958725\\
8.23576009087738	1.27491996220243\\
8.2323941192646	1.27416991000827\\
8.22902684275875	1.27341979299328\\
8.22565826034769	1.272669611146\\
8.2222883710181	1.27191936445494\\
8.21891717375549	1.27116905290861\\
8.2155446675442	1.27041867649553\\
8.21217085136736	1.26966823520419\\
8.20879572420693	1.2689177290231\\
8.20541928504369	1.26816715794074\\
8.20204153285721	1.26741652194562\\
8.19866246662589	1.2666658210262\\
8.19528208532692	1.26591505517098\\
8.19190038793631	1.26516422436843\\
8.18851737342887	1.26441332860701\\
8.1851330407782	1.2636623678752\\
8.1817473889567	1.26291134216145\\
8.1783604169356	1.26216025145422\\
8.17497212368488	1.26140909574196\\
8.17158250817335	1.26065787501312\\
8.16819156936859	1.25990658925613\\
8.16479930623699	1.25915523845944\\
8.16140571774371	1.25840382261148\\
8.1580108028527	1.25765234170067\\
8.15461456052671	1.25690079571544\\
8.15121698972725	1.2561491846442\\
8.14781808941463	1.25539750847535\\
8.14441785854793	1.25464576719733\\
8.14101629608501	1.25389396079851\\
8.13761340098249	1.2531420892673\\
8.1342091721958	1.25239015259209\\
8.13080360867911	1.25163815076127\\
8.12739670938535	1.25088608376322\\
8.12398847326626	1.25013395158632\\
8.1205788992723	1.24938175421893\\
8.11716798635273	1.24862949164943\\
8.11375573345555	1.24787716386618\\
8.11034213952752	1.24712477085753\\
8.10692720351418	1.24637231261184\\
8.10351092435979	1.24561978911745\\
8.10009330100739	1.2448672003627\\
8.09667433239876	1.24411454633593\\
8.09325401747444	1.24336182702547\\
8.08983235517372	1.24260904241964\\
8.08640934443461	1.24185619250677\\
8.0829849841939	1.24110327727518\\
8.0795592733871	1.24035029671316\\
8.07613221094845	1.23959725080903\\
8.07270379581096	1.23884413955109\\
8.06927402690635	1.23809096292762\\
8.06584290316509	1.23733772092693\\
8.06241042351636	1.23658441353729\\
8.05897658688809	1.23583104074698\\
8.05554139220695	1.23507760254428\\
8.05210483839829	1.23432409891745\\
8.04866692438624	1.23357052985476\\
8.0452276490936	1.23281689534445\\
8.04178701144193	1.23206319537479\\
8.03834501035149	1.23130942993402\\
8.03490164474126	1.23055559901039\\
8.03145691352892	1.22980170259211\\
8.02801081563088	1.22904774066744\\
8.02456334996226	1.22829371322458\\
8.02111451543688	1.22753962025177\\
8.01766431096725	1.22678546173721\\
8.01421273546463	1.22603123766911\\
8.01075978783892	1.22527694803567\\
8.00730546699878	1.22452259282509\\
8.00384977185152	1.22376817202557\\
8.00039270130317	1.22301368562528\\
7.99693425425845	1.22225913361241\\
7.99347442962076	1.22150451597513\\
7.99001322629221	1.22074983270162\\
7.98655064317358	1.21999508378002\\
7.98308667916433	1.21924026919851\\
7.97962133316263	1.21848538894523\\
7.97615460406529	1.21773044300833\\
7.97268649076784	1.21697543137595\\
7.96921699216447	1.21622035403622\\
7.96574610714802	1.21546521097728\\
7.96227383461005	1.21471000218724\\
7.95880017344075	1.21395472765423\\
7.955325122529	1.21319938736635\\
7.95184868076233	1.21244398131171\\
7.94837084702696	1.21168850947841\\
7.94489162020773	1.21093297185455\\
7.94141099918817	1.21017736842821\\
7.93792898285048	1.20942169918747\\
7.93444557007547	1.20866596412041\\
7.93096075974264	1.2079101632151\\
7.92747455073013	1.20715429645961\\
7.92398694191473	1.20639836384198\\
7.92049793217186	1.20564236535029\\
7.91700752037562	1.20488630097256\\
7.91351570539872	1.20413017069685\\
7.91002248611252	1.20337397451118\\
7.90652786138702	1.20261771240358\\
7.90303183009085	1.20186138436208\\
7.89953439109129	1.20110499037469\\
7.89603554325422	1.20034853042942\\
7.89253528544418	1.19959200451427\\
7.88903361652433	1.19883541261725\\
7.88553053535643	1.19807875472633\\
7.88202604080089	1.19732203082952\\
7.87852013171674	1.19656524091478\\
7.87501280696161	1.19580838497008\\
7.87150406539175	1.19505146298341\\
7.86799390586204	1.1942944749427\\
7.86448232722595	1.19353742083592\\
7.86096932833556	1.19278030065101\\
7.85745490804159	1.19202311437592\\
7.85393906519332	1.19126586199858\\
7.85042179863864	1.19050854350692\\
7.84690310722408	1.18975115888885\\
7.84338298979472	1.18899370813229\\
7.83986144519426	1.18823619122516\\
7.83633847226498	1.18747860815534\\
7.83281406984776	1.18672095891075\\
7.82928823678207	1.18596324347927\\
7.82576097190595	1.18520546184877\\
7.82223227405605	1.18444761400713\\
7.81870214206758	1.18368969994224\\
7.81517057477434	1.18293171964193\\
7.81163757100871	1.18217367309408\\
7.80810312960162	1.18141556028653\\
7.8045672493826	1.18065738120713\\
7.80102992917975	1.1798991358437\\
7.79749116781971	1.17914082418408\\
7.79395096412772	1.17838244621609\\
7.79040931692755	1.17762400192754\\
7.78686622504156	1.17686549130625\\
7.78332168729065	1.17610691434002\\
7.77977570249428	1.17534827101663\\
7.77622826947047	1.17458956132387\\
7.77267938703578	1.17383078524954\\
7.76912905400533	1.1730719427814\\
7.76557726919278	1.17231303390721\\
7.76202403141033	1.17155405861475\\
7.75846933946873	1.17079501689175\\
7.75491319217727	1.17003590872598\\
7.75135558834377	1.16927673410515\\
7.74779652677459	1.16851749301702\\
7.74423600627461	1.16775818544929\\
7.74067402564726	1.16699881138969\\
7.73711058369449	1.16623937082594\\
7.73354567921675	1.16547986374572\\
7.72997931101306	1.16472029013674\\
7.72641147788092	1.16396064998669\\
7.72284217861637	1.16320094328324\\
7.71927141201395	1.16244117001407\\
7.71569917686671	1.16168133016685\\
7.71212547196624	1.16092142372923\\
7.70855029610261	1.16016145068887\\
7.7049736480644	1.15940141103342\\
7.7013955266387	1.15864130475049\\
7.69781593061108	1.15788113182774\\
7.69423485876565	1.15712089225278\\
7.69065230988497	1.15636058601321\\
7.68706828275012	1.15560021309666\\
7.68348277614067	1.15483977349072\\
7.67989578883466	1.15407926718297\\
7.67630731960863	1.15331869416101\\
7.67271736723759	1.15255805441241\\
7.66912593049506	1.15179734792473\\
7.66553300815301	1.15103657468555\\
7.66193859898188	1.1502757346824\\
7.65834270175062	1.14951482790284\\
7.65474531522661	1.1487538543344\\
7.65114643817572	1.14799281396462\\
7.64754606936227	1.14723170678101\\
7.64394420754907	1.14647053277108\\
7.64034085149737	1.14570929192235\\
7.63673599996687	1.14494798422231\\
7.63312965171574	1.14418660965845\\
7.6295218055006	1.14342516821825\\
7.62591246007652	1.14266365988918\\
7.62230161419701	1.14190208465871\\
7.61868926661404	1.14114044251429\\
7.615075416078	1.14037873344339\\
7.61146006133774	1.13961695743342\\
7.60784320114055	1.13885511447184\\
7.60422483423212	1.13809320454606\\
7.60060495935661	1.1373312276435\\
7.5969835752566	1.13656918375156\\
7.59336068067308	1.13580707285764\\
7.58973627434548	1.13504489494914\\
7.58611035501164	1.13428265001344\\
7.58248292140784	1.13352033803792\\
7.57885397226875	1.13275795900993\\
7.57522350632746	1.13199551291683\\
7.5715915223155	1.13123299974597\\
7.56795801896275	1.13047041948469\\
7.56432299499756	1.12970777212033\\
7.56068644914663	1.1289450576402\\
7.55704838013509	1.12818227603162\\
7.55340878668646	1.12741942728189\\
7.54976766752266	1.12665651137831\\
7.54612502136398	1.12589352830816\\
7.54248084692912	1.12513047805873\\
7.53883514293517	1.12436736061728\\
7.5351879080976	1.12360417597108\\
7.53153914113024	1.12284092410737\\
7.52788884074532	1.1220776050134\\
7.52423700565346	1.1213142186764\\
7.52058363456361	1.12055076508359\\
7.51692872618312	1.11978724422219\\
7.51327227921771	1.11902365607941\\
7.50961429237145	1.11826000064245\\
7.50595476434678	1.11749627789848\\
7.5022936938445	1.1167324878347\\
7.49863107956376	1.11596863043826\\
7.49496692020208	1.11520470569633\\
7.4913012144553	1.11444071359606\\
7.48763396101764	1.11367665412459\\
7.48396515858165	1.11291252726905\\
7.48029480583823	1.11214833301656\\
7.47662290147661	1.11138407135424\\
7.47294944418435	1.11061974226918\\
7.46927443264738	1.10985534574848\\
7.46559786554992	1.10909088177923\\
7.46191974157454	1.1083263503485\\
7.45824005940213	1.10756175144334\\
7.45455881771191	1.10679708505083\\
7.45087601518141	1.10603235115799\\
7.44719165048648	1.10526754975187\\
7.44350572230128	1.10450268081949\\
7.4398182292983	1.10373774434786\\
7.43612917014832	1.10297274032399\\
7.43243854352043	1.10220766873488\\
7.42874634808202	1.1014425295675\\
7.42505258249879	1.10067732280883\\
7.42135724543473	1.09991204844584\\
7.41766033555212	1.09914670646548\\
7.41396185151154	1.09838129685469\\
7.41026179197185	1.09761581960041\\
7.40656015559021	1.09685027468956\\
7.40285694102204	1.09608466210905\\
7.39915214692106	1.09531898184578\\
7.39544577193926	1.09455323388665\\
7.39173781472689	1.09378741821854\\
7.38802827393249	1.09302153482832\\
7.38431714820286	1.09225558370284\\
7.38060443618306	1.09148956482896\\
7.37689013651643	1.09072347819351\\
7.37317424784454	1.08995732378333\\
7.36945676880724	1.08919110158523\\
7.36573769804262	1.08842481158601\\
7.36201703418703	1.08765845377247\\
7.35829477587505	1.0868920281314\\
7.35457092173953	1.08612553464956\\
7.35084547041153	1.08535897331373\\
7.34711842052038	1.08459234411064\\
7.34338977069361	1.08382564702705\\
7.33965951955702	1.08305888204967\\
7.3359276657346	1.08229204916523\\
7.33219420784859	1.08152514836043\\
7.32845914451945	1.08075817962197\\
7.32472247436586	1.07999114293653\\
7.32098419600471	1.07922403829078\\
7.3172443080511	1.07845686567138\\
7.31350280911836	1.07768962506498\\
7.309759697818	1.07692231645822\\
7.30601497275975	1.07615493983772\\
7.30226863255154	1.07538749519009\\
7.29852067579951	1.07461998250194\\
7.29477110110797	1.07385240175986\\
7.29101990707944	1.07308475295043\\
7.28726709231462	1.07231703606021\\
7.28351265541239	1.07154925107575\\
7.27975659496983	1.0707813979836\\
7.27599890958219	1.07001347677029\\
7.27223959784289	1.06924548742233\\
7.26847865834353	1.06847742992624\\
7.26471608967388	1.06770930426849\\
7.26095189042187	1.06694111043558\\
7.2571860591736	1.06617284841398\\
7.25341859451334	1.06540451819013\\
7.24964949502349	1.06463611975049\\
7.24587875928462	1.06386765308148\\
7.24210638587546	1.06309911816953\\
7.23833237337287	1.06233051500103\\
7.23455672035186	1.06156184356238\\
7.23077942538558	1.06079310383996\\
7.22700048704533	1.06002429582015\\
7.22321990390052	1.05925541948928\\
7.21943767451872	1.05848647483371\\
7.2156537974656	1.05771746183975\\
7.21186827130498	1.05694838049374\\
7.20808109459878	1.05617923078197\\
7.20429226590705	1.05541001269072\\
7.20050178378795	1.05464072620627\\
7.19670964679776	1.0538713713149\\
7.19291585349086	1.05310194800283\\
7.18912040241973	1.05233245625632\\
7.18532329213497	1.05156289606157\\
7.18152452118526	1.05079326740481\\
7.17772408811738	1.05002357027222\\
7.17392199147621	1.04925380464998\\
7.17011822980471	1.04848397052426\\
7.16631280164392	1.04771406788122\\
7.16250570553297	1.04694409670699\\
7.15869694000908	1.0461740569877\\
7.15488650360751	1.04540394870946\\
7.15107439486163	1.04463377185836\\
7.14726061230286	1.0438635264205\\
7.14344515446067	1.04309321238192\\
7.13962801986262	1.0423228297287\\
7.13580920703432	1.04155237844687\\
7.13198871449942	1.04078185852246\\
7.12816654077963	1.04001126994147\\
7.1243426843947	1.0392406126899\\
7.12051714386245	1.03846988675374\\
7.11668991769872	1.03769909211894\\
7.11286100441738	1.03692822877147\\
7.10903040253035	1.03615729669725\\
7.10519811054757	1.03538629588222\\
7.10136412697701	1.03461522631227\\
7.09752845032468	1.03384408797329\\
7.09369107909457	1.03307288085117\\
7.08985201178873	1.03230160493177\\
7.08601124690719	1.03153026020093\\
7.08216878294802	1.03075884664448\\
7.07832461840726	1.02998736424823\\
7.07447875177898	1.02921581299799\\
7.07063118155524	1.02844419287954\\
7.06678190622609	1.02767250387864\\
7.06293092427959	1.02690074598105\\
7.05907823420175	1.02612891917251\\
7.05522383447662	1.02535702343873\\
7.05136772358617	1.02458505876542\\
7.0475099000104	1.02381302513827\\
7.04365036222725	1.02304092254294\\
7.03978910871264	1.0222687509651\\
7.03592613794047	1.02149651039039\\
7.03206144838258	1.02072420080441\\
7.02819503850878	1.01995182219279\\
7.02432690678684	1.01917937454112\\
7.02045705168248	1.01840685783495\\
7.01658547165935	1.01763427205986\\
7.01271216517909	1.01686161720137\\
7.00883713070122	1.01608889324502\\
7.00496036668326	1.01531610017631\\
7.0010818715806	1.01454323798072\\
6.99720164384662	1.01377030664372\\
6.99331968193259	1.01299730615078\\
6.98943598428771	1.01222423648732\\
6.98555054935911	1.01145109763877\\
6.9816633755918	1.01067788959052\\
6.97777446142876	1.00990461232797\\
6.97388380531082	1.00913126583646\\
6.96999140567675	1.00835785010137\\
6.96609726096321	1.007584365108\\
6.96220136960475	1.00681081084168\\
6.95830373003383	1.00603718728771\\
6.95440434068076	1.00526349443134\\
6.95050319997379	1.00448973225786\\
6.94660030633901	1.00371590075248\\
6.9426956582004	1.00294199990044\\
6.93878925397981	1.00216802968693\\
6.93488109209698	1.00139399009715\\
6.93097117096948	1.00061988111625\\
6.92705948901277	0.999845702729376\\
6.92314604464017	0.999071454921662\\
6.91923083626282	0.998297137678213\\
6.91531386228976	0.99752275098412\\
6.91139512112784	0.996748294824451\\
6.90747461118177	0.995973769184255\\
6.90355233085408	0.995199174048567\\
6.89962827854516	0.994424509402395\\
6.89570245265321	0.993649775230733\\
6.89177485157428	0.992874971518549\\
6.88784547370221	0.9921000982508\\
6.8839143174287	0.991325155412417\\
6.87998138114323	0.990550142988311\\
6.8760466632331	0.989775060963375\\
6.87211016208344	0.988999909322482\\
6.86817187607715	0.988224688050482\\
6.86423180359495	0.987449397132207\\
6.86028994301535	0.986674036552466\\
6.85634629271466	0.985898606296051\\
6.85240085106696	0.98512310634773\\
6.84845361644413	0.984347536692252\\
6.84450458721581	0.983571897314344\\
6.84055376174943	0.982796188198711\\
6.83660113841021	0.982020409330039\\
6.83264671556109	0.981244560692993\\
6.8286904915628	0.980468642272214\\
6.82473246477385	0.979692654052323\\
6.82077263355046	0.978916596017918\\
6.81681099624663	0.978140468153575\\
6.81284755121411	0.977364270443855\\
6.80888229680237	0.976588002873285\\
6.80491523135863	0.975811665426382\\
6.80094635322786	0.97503525808763\\
6.79697566075274	0.9742587808415\\
6.79300315227368	0.973482233672432\\
6.78902882612881	0.972705616564853\\
6.78505268065399	0.971928929503158\\
6.78107471418278	0.971152172471722\\
6.77709492504646	0.970375345454903\\
6.77311331157401	0.969598448437028\\
6.7691298720921	0.968821481402406\\
6.76514460492511	0.968044444335319\\
6.76115750839512	0.967267337220027\\
6.75716858082189	0.966490160040767\\
6.75317782052285	0.965712912781754\\
6.74918522581312	0.964935595427177\\
6.74519079500551	0.964158207961199\\
6.74119452641049	0.963380750367962\\
6.73719641833619	0.962603222631586\\
6.7331964690884	0.961825624736162\\
6.72919467697059	0.961047956665756\\
6.72519104028387	0.960270218404413\\
6.72118555732699	0.959492409936156\\
6.71717822639636	0.958714531244975\\
6.71316904578603	0.957936582314841\\
6.70915801378769	0.957158563129699\\
6.70514512869064	0.956380473673465\\
6.70113038878183	0.955602313930035\\
6.69711379234583	0.954824083883276\\
6.69309533766483	0.954045783517035\\
6.68907502301862	0.953267412815121\\
6.68505284668462	0.952488971761331\\
6.68102880693784	0.951710460339425\\
6.6770029020509	0.950931878533147\\
6.67297513029402	0.950153226326205\\
6.66894548993501	0.949374503702286\\
6.66491397923926	0.948595710645051\\
6.66088059646974	0.947816847138129\\
6.65684533988703	0.947037913165128\\
6.65280820774925	0.946258908709626\\
6.64876919831211	0.945479833755173\\
6.64472830982887	0.944700688285294\\
6.64068554055036	0.943921472283486\\
6.63664088872498	0.943142185733217\\
6.63259435259865	0.942362828617926\\
6.62854593041486	0.94158340092103\\
6.62449562041465	0.940803902625911\\
6.62044342083657	0.940024333715927\\
6.61638932991674	0.939244694174406\\
6.61233334588877	0.938464983984649\\
6.60827546698382	0.937685203129924\\
6.60421569143056	0.936905351593478\\
6.60015401745518	0.936125429358517\\
6.59609044328139	0.935345436408232\\
6.59202496713038	0.934565372725775\\
6.58795758722085	0.93378523829427\\
6.58388830176902	0.933005033096815\\
6.57981710898857	0.932224757116473\\
6.57574400709069	0.931444410336281\\
6.57166899428404	0.930663992739244\\
6.56759206877475	0.929883504308338\\
6.56351322876645	0.929102945026509\\
6.55943247246021	0.928322314876669\\
6.55534979805458	0.927541613841702\\
6.55126520374556	0.926760841904461\\
6.54717868772661	0.925979999047768\\
6.54309024818863	0.92519908525441\\
6.53899988331998	0.924418100507148\\
6.53490759130644	0.923637044788706\\
6.53081337033124	0.922855918081782\\
6.52671721857503	0.922074720369038\\
6.52261913421589	0.921293451633103\\
6.51851911542932	0.920512111856579\\
6.51441716038824	0.919730701022028\\
6.51031326726296	0.918949219111985\\
6.50620743422122	0.918167666108949\\
6.50209965942815	0.917386041995387\\
6.49798994104627	0.916604346753735\\
6.49387827723549	0.915822580366391\\
6.48976466615313	0.915040742815724\\
6.48564910595385	0.914258834084063\\
6.48153159478973	0.913476854153712\\
6.47741213081018	0.912694803006933\\
6.473290712162	0.911912680625955\\
6.46916733698935	0.911130486992976\\
6.46504200343374	0.910348222090157\\
6.46091470963403	0.909565885899622\\
6.45678545372642	0.908783478403463\\
6.45265423384447	0.908000999583735\\
6.44852104811905	0.907218449422459\\
6.44438589467839	0.906435827901619\\
6.44024877164801	0.905653135003162\\
6.43610967715078	0.904870370709001\\
6.43196860930688	0.904087535001012\\
6.42782556623378	0.903304627861032\\
6.42368054604628	0.902521649270869\\
6.41953354685647	0.901738599212282\\
6.41538456677373	0.900955477667003\\
6.41123360390473	0.900172284616721\\
6.40708065635344	0.899389020043093\\
6.40292572222108	0.898605683927727\\
6.39876879960617	0.897822276252211\\
6.39460988660449	0.897038796998076\\
6.39044898130908	0.896255246146826\\
6.38628608181024	0.895471623679925\\
6.38212118619553	0.894687929578794\\
6.37795429254974	0.893904163824818\\
6.37378539895492	0.893120326399341\\
6.36961450349035	0.89233641728367\\
6.36544160423254	0.89155243645907\\
6.36126669925523	0.890768383906769\\
6.35708978662938	0.88998425960795\\
6.35291086442317	0.889200063543759\\
6.34872993070198	0.888415795695302\\
6.3445469835284	0.88763145604364\\
6.34036202096223	0.886847044569798\\
6.33617504106044	0.886062561254756\\
6.33198604187722	0.885278006079454\\
6.32779502146391	0.884493379024792\\
6.32360197786905	0.883708680071622\\
6.31940690913836	0.882923909200763\\
6.31520981331469	0.882139066392979\\
6.3110106884381	0.881354151629005\\
6.30680953254577	0.880569164889522\\
6.30260634367204	0.879784106155175\\
6.2984011198484	0.878998975406562\\
6.29419385910348	0.878213772624238\\
6.28998455946303	0.877428497788715\\
6.28577321894996	0.876643150880456\\
6.28155983558426	0.875857731879886\\
6.27734440738307	0.875072240767383\\
6.27312693236063	0.87428667752328\\
6.26890740852828	0.873501042127862\\
6.26468583389448	0.872715334561374\\
6.26046220646476	0.87192955480401\\
6.25623652424176	0.871143702835921\\
6.25200878522519	0.870357778637211\\
6.24777898741184	0.869571782187937\\
6.24354712879557	0.86878571346811\\
6.23931320736733	0.867999572457692\\
6.23507722111509	0.867213359136601\\
6.2308391680239	0.866427073484706\\
6.22659904607586	0.865640715481825\\
6.22235685325011	0.864854285107735\\
6.21811258752283	0.864067782342157\\
6.21386624686721	0.863281207164766\\
6.2096178292535	0.86249455955519\\
6.20536733264895	0.861707839493009\\
6.20111475501783	0.860921046957746\\
6.19686009432141	0.86013418192888\\
6.19260334851798	0.85934724438584\\
6.1883445155628	0.858560234308003\\
6.18408359340815	0.857773151674697\\
6.17982058000328	0.856985996465197\\
6.17555547329442	0.856198768658727\\
6.17128827122478	0.855411468234461\\
6.16701897173452	0.85462409517152\\
6.16274757276077	0.853836649448975\\
6.15847407223763	0.853049131045838\\
6.15419846809614	0.852261539941077\\
6.14992075826426	0.851473876113601\\
6.14564094066692	0.850686139542268\\
6.14135901322597	0.849898330205884\\
6.13707497386017	0.849110448083196\\
6.13278882048523	0.8483224931529\\
6.12850055101375	0.847534465393638\\
6.12421016335523	0.846746364783999\\
6.11991765541609	0.845958191302511\\
6.11562302509965	0.845169944927648\\
6.11132627030608	0.844381625637832\\
6.10702738893248	0.843593233411427\\
6.10272637887279	0.842804768226739\\
6.09842323801784	0.842016230062019\\
6.09411796425531	0.841227618895459\\
6.08981055546975	0.840438934705196\\
6.08550100954257	0.839650177469306\\
6.08118932435199	0.838861347165812\\
6.07687549777309	0.838072443772671\\
6.0725595276778	0.837283467267789\\
6.06824141193484	0.836494417629009\\
6.06392114840978	0.835705294834113\\
6.05959873496498	0.834916098860825\\
6.05527416945963	0.83412682968681\\
6.05094744974971	0.833337487289671\\
6.04661857368799	0.832548071646953\\
6.04228753912402	0.831758582736133\\
6.03795434390416	0.830969020534633\\
6.03361898587153	0.830179385019809\\
6.029281462866	0.829389676168959\\
6.02494177272423	0.828599893959314\\
6.02059991327962	0.827810038368045\\
};
\addplot [color=mycolor1, forget plot]
  table[row sep=crcr]{%
6.02059991327962	0.827810038368045\\
6.01625588236234	0.827020109372259\\
6.01190967779927	0.826230106948995\\
6.00756129741405	0.825440031075234\\
6.00321073902705	0.824649881727893\\
5.99885800045534	0.823859658883816\\
5.99450307951274	0.823069362519792\\
5.99014597400975	0.822278992612535\\
5.9857866817536	0.8214885491387\\
5.98142520054819	0.820698032074871\\
5.97706152819413	0.81990744139757\\
5.97269566248871	0.819116777083248\\
5.96832760122589	0.818326039108288\\
5.9639573421963	0.817535227449009\\
5.95958488318725	0.816744342081659\\
5.95521022198268	0.815953382982416\\
5.9508333563632	0.81516235012739\\
5.94645428410605	0.814371243492625\\
5.94207300298513	0.813580063054091\\
5.93768951077094	0.812788808787686\\
5.93330380523062	0.811997480669244\\
5.92891588412793	0.811206078674521\\
5.92452574522321	0.810414602779204\\
5.92013338627345	0.80962305295891\\
5.91573880503219	0.80883142918918\\
5.91134199924958	0.808039731445487\\
5.90694296667236	0.807247959703224\\
5.90254170504382	0.806456113937718\\
5.89813821210385	0.805664194124214\\
5.89373248558886	0.804872200237888\\
5.88932452323186	0.80408013225384\\
5.88491432276236	0.803287990147092\\
5.88050188190645	0.802495773892595\\
5.87608719838673	0.801703483465219\\
5.87167026992234	0.800911118839761\\
5.86725109422891	0.800118679990936\\
5.86282966901862	0.799326166893386\\
5.85840599200013	0.798533579521672\\
5.8539800608786	0.797740917850278\\
5.84955187335568	0.796948181853608\\
5.84512142712952	0.79615537150599\\
5.84068871989473	0.795362486781665\\
5.83625374934238	0.794569527654799\\
5.83181651316002	0.793776494099479\\
5.82737700903165	0.792983386089705\\
5.82293523463771	0.792190203599395\\
5.81849118765509	0.79139694660239\\
5.81404486575709	0.790603615072448\\
5.80959626661346	0.789810208983236\\
5.80514538789036	0.789016728308344\\
5.80069222725036	0.788223173021277\\
5.79623678235243	0.787429543095455\\
5.79177905085193	0.786635838504213\\
5.78731903040063	0.785842059220796\\
5.78285671864666	0.785048205218368\\
5.77839211323453	0.784254276470006\\
5.77392521180512	0.783460272948694\\
5.76945601199565	0.782666194627335\\
5.76498451143972	0.781872041478743\\
5.76051070776726	0.781077813475637\\
5.75603459860452	0.780283510590652\\
5.75155618157411	0.779489132796333\\
5.74707545429493	0.778694680065134\\
5.74259241438222	0.777900152369414\\
5.7381070594475	0.777105549681448\\
5.7336193870986	0.776310871973414\\
5.72912939493966	0.775516119217398\\
5.72463708057106	0.774721291385393\\
5.72014244158949	0.7739263884493\\
5.71564547558789	0.773131410380925\\
5.71114618015548	0.772336357151976\\
5.70664455287769	0.771541228734072\\
5.70214059133624	0.77074602509873\\
5.69763429310906	0.769950746217375\\
5.69312565577031	0.769155392061333\\
5.68861467689039	0.768359962601831\\
5.68410135403588	0.767564457810002\\
5.6795856847696	0.766768877656876\\
5.67506766665053	0.765973222113385\\
5.67054729723387	0.765177491150363\\
5.66602457407099	0.76438168473854\\
5.66149949470943	0.763585802848552\\
5.6569720566929	0.762789845450922\\
5.65244225756125	0.761993812516081\\
5.64791009485051	0.76119770401435\\
5.64337556609283	0.760401519915953\\
5.6388386688165	0.759605260191003\\
5.63429940054592	0.758808924809514\\
5.62975775880162	0.758012513741392\\
5.62521374110026	0.75721602695644\\
5.62066734495455	0.756419464424346\\
5.61611856787334	0.755622826114701\\
5.61156740736152	0.754826111996981\\
5.60701386092011	0.754029322040561\\
5.60245792604615	0.753232456214696\\
5.59789960023276	0.752435514488542\\
5.59333888096911	0.751638496831138\\
5.58877576574041	0.750841403211413\\
5.5842102520279	0.750044233598189\\
5.57964233730886	0.749246987960167\\
5.57507201905658	0.74844966626594\\
5.57049929474035	0.74765226848399\\
5.56592416182548	0.746854794582678\\
5.56134661777325	0.746057244530254\\
5.55676666004095	0.745259618294855\\
5.55218428608182	0.744461915844492\\
5.54759949334509	0.74366413714707\\
5.54301227927594	0.74286628217037\\
5.53842264131548	0.742068350882054\\
5.53383057690079	0.741270343249665\\
5.52923608346488	0.74047225924063\\
5.52463915843667	0.739674098822252\\
5.520039799241	0.738875861961711\\
5.51543800329863	0.738077548626069\\
5.51083376802619	0.737279158782265\\
5.50622709083623	0.736480692397106\\
5.50161796913717	0.735682149437289\\
5.4970064003333	0.734883529869373\\
5.49239238182476	0.734084833659797\\
5.48777591100758	0.733286060774876\\
5.4831569852736	0.732487211180792\\
5.47853560201051	0.731688284843602\\
5.47391175860184	0.730889281729233\\
5.46928545242693	0.730090201803484\\
5.46465668086091	0.729291045032021\\
5.46002544127475	0.728491811380385\\
5.45539173103519	0.727692500813976\\
5.45075554750475	0.726893113298069\\
5.44611688804173	0.726093648797802\\
5.4414757500002	0.725294107278176\\
5.43683213072998	0.724494488704065\\
5.43218602757664	0.723694793040201\\
5.42753743788149	0.722895020251176\\
5.42288635898157	0.722095170301455\\
5.41823278820962	0.721295243155358\\
5.41357672289413	0.720495238777063\\
5.40891816035925	0.719695157130615\\
5.40425709792485	0.718894998179914\\
5.39959353290648	0.718094761888718\\
5.39492746261534	0.717294448220645\\
5.39025888435833	0.716494057139168\\
5.38558779543797	0.715693588607614\\
5.38091419315246	0.714893042589169\\
5.37623807479561	0.714092419046873\\
5.37155943765686	0.713291717943613\\
5.36687827902129	0.712490939242135\\
5.36219459616957	0.711690082905033\\
5.35750838637795	0.710889148894754\\
5.35281964691831	0.710088137173592\\
5.34812837505808	0.709287047703693\\
5.34343456806028	0.708485880447046\\
5.33873822318346	0.707684635365492\\
5.33403933768176	0.706883312420713\\
5.32933790880483	0.706081911574241\\
5.32463393379787	0.70528043278745\\
5.31992740990158	0.704478876021556\\
5.31521833435221	0.703677241237623\\
5.31050670438148	0.702875528396546\\
5.3057925172166	0.70207373745907\\
5.30107577008029	0.701271868385774\\
5.29635646019073	0.70046992113708\\
5.29163458476155	0.699667895673242\\
5.28691014100185	0.698865791954359\\
5.28218312611617	0.698063609940352\\
5.27745353730447	0.69726134959099\\
5.27272137176216	0.696459010865869\\
5.26798662668005	0.69565659372442\\
5.26324929924433	0.694854098125904\\
5.25850938663663	0.694051524029413\\
5.25376688603393	0.693248871393868\\
5.24902179460859	0.692446140178021\\
5.24427410952834	0.691643330340448\\
5.23952382795625	0.690840441839553\\
5.23477094705076	0.690037474633571\\
5.2300154639656	0.68923442868055\\
5.22525737584987	0.688431303938372\\
5.22049667984795	0.687628100364734\\
5.21573337309953	0.686824817917162\\
5.21096745273959	0.686021456552993\\
5.2061989158984	0.685218016229388\\
5.20142775970149	0.684414496903327\\
5.19665398126967	0.683610898531603\\
5.19187757771897	0.682807221070831\\
5.18709854616069	0.682003464477437\\
5.18231688370132	0.681199628707656\\
5.17753258744263	0.680395713717544\\
5.17274565448153	0.679591719462965\\
5.16795608191017	0.678787645899591\\
5.16316386681589	0.677983492982906\\
5.15836900628117	0.677179260668201\\
5.15357149738369	0.676374948910572\\
5.14877133719628	0.675570557664921\\
5.14396852278689	0.674766086885962\\
5.13916305121864	0.673961536528199\\
5.13435491954974	0.67315690654595\\
5.12954412483353	0.672352196893327\\
5.12473066411846	0.671547407524243\\
5.11991453444804	0.670742538392414\\
5.11509573286088	0.669937589451344\\
5.11027425639067	0.669132560654345\\
5.10545010206612	0.668327451954513\\
5.10062326691103	0.667522263304744\\
5.0957937479442	0.666716994657724\\
5.09096154217948	0.665911645965933\\
5.08612664662571	0.665106217181635\\
5.08128905828676	0.664300708256889\\
5.07644877416147	0.66349511914354\\
5.07160579124366	0.662689449793215\\
5.06676010652213	0.661883700157328\\
5.06191171698063	0.661077870187082\\
5.05706061959786	0.660271959833451\\
5.05220681134746	0.659465969047202\\
5.04735028919798	0.658659897778872\\
5.04249105011288	0.657853745978779\\
5.03762909105055	0.657047513597023\\
5.03276440896424	0.656241200583474\\
5.02789700080209	0.655434806887777\\
5.02302686350709	0.654628332459351\\
5.01815399401712	0.653821777247386\\
5.01327838926487	0.653015141200842\\
5.00840004617788	0.652208424268447\\
5.0035189616785	0.6514016263987\\
4.9986351326839	0.650594747539861\\
4.99374855610603	0.649787787639956\\
4.98885922885164	0.648980746646775\\
4.98396714782226	0.648173624507873\\
4.97907230991415	0.647366421170559\\
4.97417471201836	0.646559136581901\\
4.96927435102064	0.645751770688731\\
4.96437122380149	0.644944323437632\\
4.95946532723613	0.644136794774941\\
4.95455665819446	0.643329184646752\\
4.94964521354109	0.642521492998902\\
4.94473099013528	0.641713719776987\\
4.939813984831	0.640905864926348\\
4.93489419447683	0.640097928392074\\
4.92997161591602	0.639289910118995\\
4.92504624598644	0.63848181005169\\
4.92011808152058	0.637673628134479\\
4.91518711934554	0.636865364311421\\
4.910253356283	0.636057018526316\\
4.90531678914922	0.635248590722698\\
4.90037741475506	0.634440080843846\\
4.8954352299059	0.633631488832763\\
4.89049023140168	0.632822814632191\\
4.88554241603686	0.632014058184599\\
4.88059178060043	0.631205219432193\\
4.8756383218759	0.630396298316898\\
4.87068203664124	0.62958729478037\\
4.86572292166892	0.62877820876399\\
4.86076097372589	0.627969040208863\\
4.85579618957353	0.627159789055809\\
4.85082856596769	0.626350455245375\\
4.84585809965862	0.625541038717823\\
4.84088478739102	0.624731539413133\\
4.83590862590397	0.623921957270995\\
4.83092961193097	0.623112292230816\\
4.82594774219986	0.622302544231713\\
4.82096301343289	0.62149271321251\\
4.81597542234662	0.620682799111747\\
4.810984965652	0.619872801867657\\
4.80599164005425	0.619062721418186\\
4.80099544225295	0.618252557700982\\
4.79599636894197	0.617442310653387\\
4.79099441680946	0.616631980212452\\
4.78598958253785	0.615821566314913\\
4.78098186280383	0.615011068897211\\
4.77597125427834	0.614200487895475\\
4.77095775362656	0.613389823245526\\
4.76594135750788	0.612579074882875\\
4.76092206257591	0.611768242742719\\
4.75589986547845	0.610957326759945\\
4.75087476285749	0.610146326869119\\
4.74584675134917	0.609335243004488\\
4.74081582758382	0.608524075099987\\
4.73578198818586	0.607712823089219\\
4.73074522977388	0.606901486905471\\
4.72570554896057	0.6060900664817\\
4.72066294235272	0.605278561750532\\
4.71561740655121	0.604466972644273\\
4.71056893815098	0.603655299094887\\
4.70551753374105	0.60284354103401\\
4.70046318990447	0.602031698392938\\
4.69540590321833	0.601219771102634\\
4.69034567025373	0.600407759093716\\
4.68528248757579	0.599595662296466\\
4.68021635174359	0.598783480640814\\
4.67514725931021	0.597971214056351\\
4.67007520682269	0.597158862472315\\
4.66500019082201	0.596346425817596\\
4.65992220784308	0.595533904020733\\
4.65484125441474	0.594721297009906\\
4.64975732705973	0.593908604712941\\
4.64467042229467	0.593095827057304\\
4.63958053663008	0.592282963970102\\
4.63448766657033	0.591470015378074\\
4.62939180861363	0.590656981207601\\
4.62429295925202	0.589843861384689\\
4.61919111497138	0.589030655834977\\
4.61408627225138	0.588217364483732\\
4.60897842756548	0.587403987255846\\
4.60386757738091	0.586590524075835\\
4.59875371815868	0.585776974867837\\
4.59363684635352	0.584963339555604\\
4.5885169584139	0.584149618062511\\
4.58339405078202	0.58333581031154\\
4.57826811989376	0.58252191622529\\
4.57313916217871	0.581707935725971\\
4.56800717406009	0.580893868735392\\
4.56287215195483	0.580079715174974\\
4.55773409227347	0.579265474965737\\
4.55259299142017	0.578451148028302\\
4.54744884579273	0.577636734282887\\
4.54230165178251	0.576822233649304\\
4.53715140577447	0.576007646046957\\
4.53199810414715	0.575192971394843\\
4.52684174327262	0.574378209611545\\
4.52168231951648	0.573563360615232\\
4.51651982923787	0.572748424323653\\
4.51135426878942	0.571933400654139\\
4.50618563451726	0.571118289523596\\
4.50101392276098	0.570303090848508\\
4.49583912985363	0.569487804544931\\
4.49066125212172	0.568672430528488\\
4.48548028588515	0.567856968714372\\
4.48029622745728	0.567041419017339\\
4.47510907314482	0.566225781351707\\
4.46991881924789	0.565410055631353\\
4.46472546205995	0.564594241769712\\
4.45952899786783	0.563778339679769\\
4.45432942295166	0.562962349274065\\
4.44912673358493	0.562146270464688\\
4.4439209260344	0.561330103163268\\
4.43871199656011	0.56051384728098\\
4.43349994141537	0.559697502728541\\
4.42828475684677	0.558881069416206\\
4.4230664390941	0.558064547253758\\
4.41784498439039	0.557247936150521\\
4.41262038896184	0.556431236015339\\
4.40739264902789	0.555614446756587\\
4.4021617608011	0.554797568282162\\
4.39692772048722	0.553980600499482\\
4.39169052428509	0.55316354331548\\
4.38645016838673	0.552346396636605\\
4.38120664897723	0.551529160368817\\
4.37595996223475	0.550711834417586\\
4.37071010433056	0.549894418687884\\
4.36545707142895	0.549076913084191\\
4.36020085968727	0.548259317510479\\
4.35494146525587	0.547441631870222\\
4.34967888427812	0.546623856066384\\
4.34441311289038	0.545805990001422\\
4.33914414722194	0.544988033577281\\
4.33387198339509	0.544169986695384\\
4.32859661752502	0.54335184925664\\
4.32331804571986	0.542533621161437\\
4.31803626408063	0.541715302309633\\
4.31275126870124	0.54089689260056\\
4.30746305566844	0.540078391933018\\
4.30217162106186	0.539259800205272\\
4.29687696095396	0.538441117315045\\
4.29157907140998	0.537622343159525\\
4.28627794848799	0.536803477635352\\
4.28097358823883	0.535984520638618\\
4.27566598670608	0.535165472064861\\
4.2703551399261	0.534346331809065\\
4.26504104392793	0.533527099765662\\
4.25972369473336	0.532707775828512\\
4.25440308835685	0.531888359890917\\
4.24907922080552	0.531068851845611\\
4.24375208807916	0.530249251584751\\
4.23842168617019	0.529429558999922\\
4.23308801106365	0.528609773982129\\
4.22775105873718	0.527789896421795\\
4.22241082516099	0.526969926208758\\
4.21706730629786	0.526149863232266\\
4.21172049810313	0.525329707380971\\
4.20637039652464	0.524509458542931\\
4.20101699750274	0.523689116605602\\
4.1956602969703	0.522868681455843\\
4.19030029085262	0.522048152979891\\
4.18493697506748	0.521227531063384\\
4.17957034552507	0.520406815591343\\
4.17420039812802	0.519586006448164\\
4.16882712877135	0.518765103517627\\
4.16345053334244	0.51794410668288\\
4.15807060772103	0.517123015826447\\
4.15268734777923	0.516301830830213\\
4.14730074938144	0.515480551575426\\
4.14191080838436	0.514659177942695\\
4.13651752063699	0.51383770981198\\
4.13112088198059	0.513016147062592\\
4.12572088824865	0.512194489573189\\
4.12031753526689	0.511372737221772\\
4.11491081885324	0.510550889885678\\
4.10950073481782	0.509728947441585\\
4.10408727896289	0.50890690976549\\
4.0986704470829	0.508084776732729\\
4.09325023496438	0.507262548217948\\
4.08782663838599	0.506440224095123\\
4.0823996531185	0.505617804237535\\
4.07696927492469	0.504795288517777\\
4.07153549955945	0.503972676807751\\
4.06609832276966	0.503149968978655\\
4.06065774029421	0.502327164900987\\
4.055213747864	0.501504264444537\\
4.04976634120187	0.500681267478384\\
4.04431551602263	0.499858173870892\\
4.038861268033	0.499034983489704\\
4.03340359293163	0.498211696201735\\
4.02794248640903	0.497388311873178\\
4.02247794414759	0.496564830369486\\
4.01700996182155	0.495741251555375\\
4.01153853509696	0.494917575294824\\
4.0060636596317	0.494093801451059\\
4.00058533107541	0.493269929886555\\
3.99510354506949	0.492445960463038\\
3.98961829724712	0.491621893041464\\
3.98412958323315	0.490797727482025\\
3.97863739864418	0.489973463644151\\
3.97314173908845	0.489149101386489\\
3.96764260016589	0.48832464056691\\
3.96213997746803	0.487500081042501\\
3.95663386657806	0.486675422669559\\
3.95112426307073	0.485850665303587\\
3.94561116251239	0.485025808799291\\
3.94009456046091	0.484200853010572\\
3.93457445246574	0.483375797790526\\
3.92905083406778	0.482550642991431\\
3.92352370079947	0.481725388464748\\
3.91799304818467	0.480900034061117\\
3.91245887173873	0.48007457963035\\
3.90692116696839	0.479249025021423\\
3.9013799293718	0.478423370082474\\
3.89583515443849	0.4775976146608\\
3.89028683764935	0.476771758602846\\
3.88473497447658	0.475945801754209\\
3.87917956038374	0.475119743959621\\
3.87362059082563	0.474293585062951\\
3.86805806124835	0.473467324907203\\
3.86249196708923	0.4726409633345\\
3.85692230377683	0.471814500186088\\
3.85134906673091	0.470987935302328\\
3.84577225136241	0.47016126852269\\
3.8401918530734	0.469334499685748\\
3.83460786725713	0.468507628629171\\
3.82902028929791	0.467680655189724\\
3.82342911457117	0.46685357920326\\
3.81783433844339	0.46602640050471\\
3.8122359562721	0.465199118928087\\
3.80663396340583	0.464371734306466\\
3.80102835518412	0.463544246471994\\
3.79541912693748	0.462716655255875\\
3.78980627398735	0.461888960488363\\
3.78418979164612	0.461061161998765\\
3.77856967521707	0.460233259615426\\
3.77294591999435	0.459405253165727\\
3.76731852126297	0.458577142476082\\
3.76168747429876	0.457748927371927\\
3.75605277436839	0.456920607677715\\
3.75041441672926	0.456092183216911\\
3.74477239662957	0.455263653811991\\
3.73912670930825	0.454435019284427\\
3.7334773499949	0.453606279454681\\
3.72782431390986	0.452777434142214\\
3.72216759626411	0.451948483165458\\
3.71650719225924	0.451119426341826\\
3.7108430970875	0.450290263487698\\
3.7051753059317	0.449460994418417\\
3.69950381396522	0.448631618948287\\
3.69382861635198	0.447802136890553\\
3.6881497082464	0.446972548057413\\
3.68246708479342	0.446142852259998\\
3.67678074112842	0.445313049308369\\
3.67109067237723	0.444483139011516\\
3.66539687365609	0.443653121177341\\
3.65969934007163	0.44282299561266\\
3.65399806672085	0.441992762123195\\
3.64829304869108	0.441162420513564\\
3.64258428105997	0.440331970587274\\
3.63687175889545	0.439501412146724\\
3.63115547725573	0.438670744993181\\
3.62543543118923	0.43783996892679\\
3.61971161573461	0.437009083746554\\
3.61398402592069	0.436178089250337\\
3.60825265676648	0.435346985234855\\
3.60251750328108	0.434515771495658\\
3.59677856046373	0.433684447827144\\
3.59103582330375	0.432853014022528\\
3.58528928678051	0.432021469873854\\
3.57953894586339	0.431189815171978\\
3.5737847955118	0.430358049706566\\
3.5680268306751	0.429526173266077\\
3.56226504629264	0.428694185637769\\
3.55649943729364	0.427862086607684\\
3.55072999859724	0.427029875960638\\
3.54495672511247	0.426197553480222\\
3.53917961173816	0.425365118948786\\
3.53339865336299	0.424532572147442\\
3.52761384486541	0.423699912856039\\
3.52182518111362	0.422867140853175\\
3.51603265696559	0.422034255916177\\
3.51023626726895	0.421201257821093\\
3.50443600686105	0.420368146342697\\
3.49863187056885	0.419534921254462\\
3.49282385320897	0.418701582328569\\
3.4870119495876	0.417868129335885\\
3.48119615450051	0.417034562045972\\
3.475376462733	0.416200880227062\\
3.46955286905989	0.415367083646058\\
3.46372536824548	0.414533172068521\\
3.45789395504352	0.413699145258669\\
3.4520586241972	0.412865002979364\\
3.44621937043908	0.4120307449921\\
3.44037618849113	0.411196371057007\\
3.43452907306462	0.410361880932825\\
3.42867801886017	0.409527274376911\\
3.42282302056764	0.408692551145221\\
3.41696407286619	0.407857710992313\\
3.41110117042417	0.407022753671319\\
3.40523430789915	0.406187678933955\\
3.39936347993785	0.405352486530503\\
3.39348868117614	0.404517176209806\\
3.38760990623899	0.403681747719258\\
3.38172714974045	0.40284620080479\\
3.37584040628364	0.402010535210872\\
3.36994967046065	0.401174750680493\\
3.36405493685262	0.400338846955158\\
3.3581562000296	0.399502823774881\\
3.3522534545506	0.398666680878168\\
3.34634669496352	0.397830418002016\\
3.34043591580513	0.3969940348819\\
3.33452111160103	0.39615753125176\\
3.32860227686565	0.395320906844\\
3.32267940610218	0.394484161389471\\
3.31675249380256	0.393647294617468\\
3.31082153444746	0.392810306255713\\
3.30488652250622	0.391973196030355\\
3.29894745243683	0.391135963665949\\
3.29300431868593	0.390298608885453\\
3.28705711568874	0.389461131410224\\
3.28110583786903	0.388623530959994\\
3.27515047963912	0.387785807252874\\
3.2691910353998	0.386947960005327\\
3.26322749954037	0.386109988932185\\
3.25725986643852	0.385271893746609\\
3.25128813046038	0.384433674160098\\
3.24531228596043	0.383595329882473\\
3.2393323272815	0.382756860621869\\
3.23334824875472	0.381918266084716\\
3.2273600446995	0.381079545975746\\
3.22136770942349	0.380240699997963\\
3.21537123722256	0.379401727852646\\
3.20937062238075	0.37856262923933\\
3.20336585917024	0.377723403855805\\
3.19735694185133	0.376884051398093\\
3.19134386467241	0.376044571560452\\
3.18532662186988	0.375204964035349\\
3.17930520766821	0.374365228513462\\
3.17327961627978	0.373525364683659\\
3.16724984190499	0.372685372232998\\
3.1612158787321	0.371845250846707\\
3.15517772093728	0.371005000208178\\
3.14913536268451	0.370164619998947\\
3.14308879812563	0.369324109898701\\
3.13703802140022	0.36848346958524\\
3.13098302663563	0.367642698734493\\
3.12492380794689	0.366801797020486\\
3.11886035943673	0.365960764115345\\
3.11279267519553	0.365119599689271\\
3.10672074930124	0.364278303410539\\
3.10064457581941	0.363436874945482\\
3.09456414880311	0.362595313958476\\
3.08847946229294	0.361753620111937\\
3.08239051031694	0.360911793066296\\
3.07629728689058	0.360069832480001\\
3.07019978601675	0.359227738009491\\
3.06409800168569	0.358385509309199\\
3.05799192787495	0.357543146031525\\
3.0518815585494	0.356700647826833\\
3.04576688766113	0.355858014343432\\
3.03964790914948	0.355015245227573\\
3.03352461694096	0.354172340123427\\
3.02739700494921	0.353329298673074\\
3.021265067075	0.352486120516493\\
3.01512879720618	0.351642805291551\\
3.00898818921761	0.350799352633984\\
3.00284323697117	0.349955762177387\\
2.9966939343157	0.349112033553203\\
2.99054027508696	0.348268166390706\\
2.9843822531076	0.347424160316991\\
2.97821986218713	0.346580014956956\\
2.97205309612187	0.345735729933298\\
2.96588194869491	0.344891304866487\\
2.9597064136761	0.34404673937476\\
2.95352648482198	0.34320203307411\\
2.94734215587573	0.342357185578264\\
2.9411534205672	0.341512196498677\\
2.9349602726128	0.340667065444509\\
2.92876270571549	0.339821792022621\\
2.92256071356476	0.338976375837556\\
2.91635428983655	0.338130816491525\\
2.91014342819325	0.337285113584392\\
2.90392812228364	0.336439266713659\\
2.89770836574284	0.335593275474458\\
2.89148415219233	0.334747139459526\\
2.88525547523981	0.333900858259203\\
2.87902232847927	0.333054431461402\\
2.87278470549087	0.332207858651608\\
2.86654259984093	0.331361139412854\\
2.8602960050819	0.330514273325714\\
2.85404491475231	0.329667259968283\\
2.84778932237672	0.328820098916156\\
2.8415292214657	0.327972789742425\\
2.83526460551576	0.327125332017659\\
2.82899546800935	0.326277725309879\\
2.82272180241478	0.325429969184562\\
2.81644360218621	0.324582063204604\\
2.81016086076359	0.323734006930321\\
2.80387357157263	0.322885799919423\\
2.79758172802473	0.322037441727003\\
2.79128532351699	0.321188931905518\\
2.78498435143214	0.320340270004779\\
2.77867880513848	0.319491455571924\\
2.77236867798985	0.31864248815141\\
2.76605396332563	0.317793367284999\\
2.75973465447063	0.316944092511727\\
2.7534107447351	0.316094663367908\\
2.74708222741466	0.315245079387098\\
2.74074909579025	0.314395340100096\\
2.73441134312814	0.313545445034907\\
2.7280689626798	0.312695393716741\\
2.72172194768195	0.311845185667992\\
2.71537029135645	0.31099482040822\\
2.70901398691028	0.310144297454126\\
2.7026530275355	0.309293616319549\\
2.6962874064092	0.308442776515437\\
2.68991711669347	0.307591777549835\\
2.68354215153533	0.306740618927864\\
2.67716250406669	0.305889300151707\\
2.67077816740435	0.305037820720583\\
2.66438913464989	0.304186180130741\\
2.65799539888966	0.303334377875434\\
2.65159695319474	0.302482413444896\\
2.64519379062089	0.301630286326336\\
2.63878590420849	0.300777996003912\\
2.63237328698251	0.299925541958708\\
2.62595593195246	0.299072923668725\\
2.61953383211234	0.298220140608856\\
2.61310698044061	0.29736719225087\\
2.60667536990012	0.296514078063389\\
2.60023899343809	0.295660797511872\\
2.59379784398602	0.294807350058594\\
2.58735191445971	0.293953735162628\\
2.58090119775916	0.293099952279822\\
2.57444568676854	0.292246000862785\\
2.56798537435613	0.291391880360863\\
2.56152025337431	0.290537590220119\\
2.55505031665947	0.289683129883312\\
2.54857555703198	0.288828498789885\\
2.54209596729615	0.287973696375929\\
2.53561154024018	0.287118722074179\\
2.52912226863609	0.286263575313984\\
2.52262814523969	0.285408255521283\\
2.51612916279054	0.284552762118602\\
2.50962531401188	0.283697094525007\\
2.5031165916106	0.282841252156102\\
2.49660298827719	0.281985234424006\\
2.49008449668564	0.281129040737318\\
2.4835611094935	0.280272670501117\\
2.47703281934172	0.27941612311692\\
2.47049961885464	0.278559397982672\\
2.46396150063997	0.277702494492719\\
2.45741845728871	0.27684541203779\\
2.45087048137509	0.27598815000497\\
2.44431756545653	0.275130707777684\\
2.43775970207362	0.274273084735668\\
2.43119688375002	0.273415280254953\\
2.42462910299243	0.272557293707831\\
2.41805635229055	0.271699124462846\\
2.411478624117	0.270840771884763\\
2.40489591092731	0.269982235334547\\
2.39830820515982	0.269123514169339\\
2.39171549923568	0.26826460774243\\
2.38511778555873	0.267405515403246\\
2.37851505651553	0.266546236497311\\
2.37190730447524	0.265686770366235\\
2.36529452178959	0.264827116347689\\
2.35867670079283	0.263967273775367\\
2.35205383380169	0.263107241978983\\
2.34542591311529	0.262247020284232\\
2.33879293101512	0.261386608012761\\
2.33215487976497	0.260526004482167\\
2.32551175161089	0.259665209005942\\
2.3188635387811	0.258804220893476\\
2.312210233486	0.257943039450009\\
2.30555182791803	0.257081663976622\\
2.29888831425169	0.256220093770202\\
2.29221968464346	0.255358328123417\\
2.28554593123173	0.254496366324701\\
2.27886704613674	0.253634207658211\\
2.27218302146056	0.252771851403812\\
2.26549384928701	0.251909296837047\\
2.2587995216816	0.251046543229113\\
2.25210003069149	0.250183589846833\\
2.24539536834541	0.249320435952626\\
2.23868552665363	0.248457080804484\\
2.23197049760788	0.247593523655947\\
2.22525027318131	0.246729763756067\\
2.21852484532841	0.24586580034939\\
2.21179420598498	0.245001632675922\\
2.20505834706806	0.244137259971104\\
2.19831726047587	0.243272681465785\\
2.19157093808773	0.242407896386188\\
2.18481937176407	0.241542903953891\\
2.17806255334627	0.240677703385791\\
2.17130047465669	0.239812293894073\\
2.16453312749857	0.238946674686196\\
2.15776050365597	0.238080844964841\\
2.15098259489373	0.237214803927907\\
2.14419939295737	0.23634855076846\\
2.13741088957308	0.235482084674716\\
2.13061707644763	0.234615404830006\\
2.12381794526831	0.23374851041275\\
2.11701348770287	0.232881400596423\\
2.11020369539948	0.232014074549524\\
2.10338855998663	0.231146531435552\\
2.09656807307311	0.230278770412965\\
2.0897422262479	0.229410790635164\\
2.08291101108016	0.228542591250441\\
2.07607441911914	0.22767417140197\\
2.0692324418941	0.226805530227763\\
2.06238507091428	0.225936666860633\\
2.05553229766883	0.225067580428181\\
2.04867411362673	0.224198270052742\\
2.04181051023673	0.223328734851369\\
2.03494147892733	0.222458973935794\\
2.02806701110661	0.221588986412397\\
2.02118709816231	0.22071877138217\\
2.01430173146163	0.219848327940686\\
2.00741090235126	0.218977655178067\\
2.00051460215725	0.218106752178952\\
1.993612822185	0.217235618022458\\
1.98670555371916	0.216364251782148\\
1.97979278802355	0.215492652525996\\
1.97287451634114	0.214620819316364\\
1.96595072989395	0.213748751209946\\
1.959021419883	0.212876447257753\\
1.95208657748822	0.212003906505067\\
1.9451461938684	0.21113112799141\\
1.93820026016113	0.210258110750508\\
1.93124876748271	0.209384853810256\\
1.92429170692811	0.208511356192678\\
1.91732906957085	0.207637616913896\\
1.91036084646302	0.206763634984095\\
1.9033870286351	0.205889409407479\\
1.896407607096	0.20501493918224\\
1.88942257283289	0.204140223300522\\
1.88243191681123	0.20326526074838\\
1.8754356299746	0.202390050505741\\
1.8684337032447	0.20151459154638\\
1.86142612752127	0.200638882837861\\
1.85441289368198	0.199762923341515\\
1.84739399258241	0.198886712012393\\
1.84036941505594	0.198010247799236\\
1.83333915191369	0.197133529644426\\
1.82630319394446	0.196256556483953\\
1.81926153191463	0.19537932724738\\
1.81221415656813	0.194501840857789\\
1.80516105862633	0.193624096231757\\
1.79810222878796	0.19274609227931\\
1.79103765772908	0.191867827903878\\
1.78396733610298	0.190989302002259\\
1.77689125454008	0.190110513464585\\
1.76980940364792	0.189231461174261\\
1.76272177401103	0.188352144007949\\
1.75562835619085	0.187472560835507\\
1.74852914072571	0.186592710519953\\
1.74142411813071	0.185712591917436\\
1.73431327889765	0.184832203877165\\
1.72719661349497	0.183951545241395\\
1.72007411236764	0.183070614845368\\
1.71294576593713	0.18218941151728\\
1.7058115646013	0.181307934078223\\
1.69867149873432	0.180426181342158\\
1.69152555868662	0.179544152115862\\
1.68437373478478	0.178661845198881\\
1.67721601733146	0.177779259383496\\
1.67005239660535	0.176896393454672\\
1.66288286286104	0.176013246190007\\
1.655707406329	0.1751298163597\\
1.64852601721544	0.174246102726495\\
1.64133868570226	0.173362104045641\\
1.63414540194698	0.17247781906484\\
1.62694615608265	0.171593246524209\\
1.61974093821774	0.170708385156223\\
1.61252973843612	0.169823233685677\\
1.60531254679689	0.168937790829638\\
1.59808935333441	0.168052055297388\\
1.59086014805812	0.16716602579039\\
1.5836249209525	0.166279701002224\\
1.57638366197697	0.165393079618557\\
1.56913636106585	0.164506160317079\\
1.56188300812821	0.163618941767459\\
1.55462359304784	0.162731422631299\\
1.54735810568313	0.161843601562081\\
1.54008653586701	0.160955477205115\\
1.53280887340684	0.160067048197496\\
1.52552510808435	0.159178313168043\\
1.51823522965555	0.158289270737258\\
1.51093922785061	0.157399919517269\\
1.50363709237383	0.156510258111781\\
1.4963288129035	0.155620285116026\\
1.48901437909182	0.154729999116698\\
1.48169378056488	0.153839398691923\\
1.47436700692245	0.152948482411187\\
1.46703404773802	0.15205724883529\\
1.45969489255861	0.151165696516292\\
1.45234953090473	0.150273823997465\\
1.4449979522703	0.149381629813224\\
1.43764014612251	0.148489112489088\\
1.43027610190178	0.147596270541619\\
1.42290580902166	0.146703102478364\\
1.41552925686869	0.145809606797805\\
1.40814643480239	0.144915781989298\\
1.4007573321551	0.144021626533021\\
1.39336193823191	0.143127138899915\\
1.38596024231059	0.142232317551629\\
1.37855223364144	0.141337160940461\\
1.37113790144726	0.140441667509301\\
1.36371723492323	0.139545835691573\\
1.3562902232368	0.138649663911177\\
1.34885685552761	0.137753150582432\\
1.34141712090741	0.13685629411001\\
1.33397100845991	0.13595909288889\\
1.32651850724076	0.135061545304284\\
1.31905960627739	0.134163649731582\\
1.31159429456897	0.133265404536298\\
1.30412256108624	0.132366808073997\\
1.29664439477148	0.131467858690244\\
1.28915978453837	0.130568554720542\\
1.28166871927192	0.129668894490259\\
1.27417118782835	0.128768876314578\\
1.26666717903499	0.127868498498429\\
1.25915668169021	0.126967759336424\\
1.25163968456326	0.126066657112799\\
1.24411617639425	0.125165190101341\\
1.23658614589398	0.124263356565339\\
1.22904958174387	0.123361154757499\\
1.22150647259584	0.122458582919899\\
1.21395680707223	0.121555639283902\\
1.2064005737657	0.120652322070115\\
1.1988377612391	0.119748629488302\\
1.19126835802536	0.118844559737322\\
1.18369235262742	0.117940111005072\\
1.17610973351814	0.117035281468407\\
1.16852048914011	0.116130069293077\\
1.16092460790564	0.11522447263366\\
1.15332207819658	0.114318489633489\\
1.14571288836429	0.113412118424588\\
1.13809702672945	0.112505357127594\\
1.13047448158201	0.111598203851702\\
1.12284524118105	0.110690656694571\\
1.1152092937547	0.109782713742275\\
1.1075666275	0.108874373069224\\
1.09991723058283	0.107965632738087\\
1.09226109113775	0.107056490799722\\
1.08459819726795	0.106146945293107\\
1.07692853704505	0.105236994245269\\
1.06925209850911	0.104326635671196\\
1.06156886966839	0.103415867573776\\
1.05387883849936	0.102504687943724\\
1.04618199294647	0.101593094759497\\
1.03847832092213	0.100681085987219\\
1.03076781030655	0.0997686595806126\\
1.02305044894763	0.098855813480922\\
1.01532622466085	0.0979425456168284\\
1.00759512522916	0.0970288539043761\\
0.999857138402851	0.0961147362468964\\
0.992112251899463	0.0952001905349253\\
0.984360453403633	0.0942852146461338\\
0.976601730567003	0.0933698064452317\\
0.968836071008089	0.0924539637839053\\
0.961063462312181	0.0915376845007257\\
0.9532838920312	0.0906209664210738\\
0.945497347683591	0.0897038073570528\\
0.937703816754201	0.0887862051074115\\
0.929903286694166	0.0878681574574579\\
0.922095744920774	0.086949662178983\\
0.914281178817354	0.0860307170301683\\
0.906459575733148	0.0851113197555091\\
0.898630922983201	0.0841914680857225\\
0.89079520784822	0.0832711597376742\\
0.882952417574458	0.0823503924142763\\
0.875102539373588	0.0814291638044145\\
0.86724556042259	0.0805074715828611\\
0.859381467863603	0.0795853134101769\\
0.851510248803814	0.0786626869326319\\
0.843631890315324	0.0777395897821203\\
0.835746379435035	0.076816019576062\\
0.827853703164501	0.0758919739173156\\
0.819953848469813	0.074967450394099\\
0.812046802281461	0.0740424465798809\\
0.804132551494222	0.0731169600333071\\
0.796211082967008	0.0721909882980957\\
0.788282383522744	0.0712645289029539\\
0.780346439948238	0.0703375793614767\\
0.772403238994056	0.0694101371720652\\
0.764452767374369	0.0684821998178181\\
0.756495011766839	0.0675537647664473\\
0.748529958812472	0.066624829470183\\
0.740557595115499	0.0656953913656739\\
0.732577907243223	0.064765447873886\\
0.724590881725892	0.0638349964000184\\
0.716596505056562	0.062904034333396\\
0.708594763690966	0.0619725590473736\\
0.700585644047363	0.06104056789924\\
0.692569132506409	0.0601080582301123\\
0.684545215411011	0.0591750273648438\\
0.676513879066207	0.0582414726119182\\
0.668475109738995	0.0573073912633453\\
0.660428893658215	0.0563727805945672\\
0.652375217014397	0.0554376378643516\\
0.644314065959631	0.0545019603146881\\
0.636245426607408	0.0535657451706836\\
0.628169285032484	0.0526289896404601\\
0.620085627270735	0.0516916909150463\\
0.611994439319021	0.0507538461682785\\
0.603895707135025	0.049815452556687\\
0.595789416637114	0.0488765072193888\\
0.587675553704192	0.0479370072779853\\
0.57955410417556	0.046996949836453\\
0.571425053850753	0.0460563319810274\\
0.563288388489399	0.0451151507801015\\
0.555144093811068	0.044173403284111\\
0.54699215549513	0.0432310865254235\\
0.538832559180588	0.0422881975182272\\
0.530665290465937	0.0413447332584157\\
0.522490334909004	0.0404006907234789\\
0.514307678026813	0.0394560668723874\\
0.506117305295405	0.0385108586454774\\
0.497919202149701	0.037565062964335\\
0.489713353983338	0.0366186767316751\\
0.481499746148525	0.0356716968312331\\
0.473278363955869	0.0347241201276477\\
0.465049192674231	0.0337759434663291\\
0.456812217530555	0.0328271636733569\\
0.44856742370973	0.0318777775553482\\
0.440314796354406	0.0309277818993421\\
0.432054320564846	0.029977173472673\\
0.42378598139876	0.0290259490228605\\
0.415509763871157	0.0280741052774716\\
0.407225652954157	0.0271216389440059\\
0.398933633576848	0.0261685467097713\\
0.390633690625111	0.0252148252417519\\
0.382325808941456	0.0242604711864887\\
0.374009973324867	0.0233054811699495\\
0.365686168530619	0.022349851797404\\
0.357354379270116	0.0213935796532946\\
0.349014590210723	0.0204366613010982\\
0.340666785975607	0.0194790932832144\\
0.33231095114355	0.018520872120813\\
0.323947070248783	0.0175619943137243\\
0.315575127780819	0.0166024563402882\\
0.307195108184285	0.0156422546572286\\
0.298806995858732	0.0146813856995201\\
0.290410775158476	0.0137198458802538\\
0.282006430392411	0.0127576315904948\\
0.273593945823851	0.0117947391991493\\
0.265173305670332	0.0108311650528313\\
0.256744494103446	0.00986690547571646\\
0.248307495248658	0.00890195676940623\\
0.239862293185139	0.0079363152127913\\
0.231408871945565	0.00696997706190369\\
0.222947215515951	0.00600293854977632\\
0.214477307835461	0.00503519588630241\\
0.205999132796239	0.00406674525809288\\
0.197512674243204	0.00309758282832587\\
0.189017915973881	0.00212770473660576\\
0.180514841738204	0.00115710709881435\\
0.172003435238351	0.000185786006969649\\
0.163483680128529	-0.000786262470935811\\
0.154955560014801	-0.0017590422910726\\
0.146419058454891	-0.00273255743392957\\
0.137874158958009	-0.00370681190446241\\
0.129320844984636	-0.00468180973225106\\
0.120759099946345	-0.00565755497164755\\
0.112188907205608	-0.0066340517019337\\
0.103610250075606	-0.00761130402747125\\
0.0950231118200223	-0.00858931607786557\\
0.0864274756528535	-0.0095680920081171\\
0.0778233247382095	-0.0105476359987825\\
0.0692106421901298	-0.0115279522561301\\
0.0605894110723611	-0.0125090450123046\\
0.0519596143981738	-0.0134909185254843\\
0.0433212351301526	-0.0144735770800482\\
0.0346742561800106	-0.0154570249867312\\
0.0260186604083634	-0.0164412665827954\\
0.0173544306245402	-0.017426306232193\\
0.00868154958637186	-0.018412148325734\\
0	-0.0193987972812512\\
-0.00869023548035287	-0.0203862575437684\\
-0.0173891742525778	-0.0213745335856722\\
-0.0260968337668847	-0.022363629906882\\
-0.0348132315260254	-0.0233535510350209\\
-0.04353838508549	-0.0243443015255875\\
-0.0522723120537338	-0.0253358859621337\\
-0.0610150300923756	-0.0263283089564331\\
-0.0697665569164269	-0.0273215751486638\\
-0.0785269102944925	-0.0283156892075822\\
-0.0872961080490018	-0.0293106558306999\\
-0.096074168056411	-0.0303064797444674\\
-0.104861108247438	-0.0313031657044518\\
-0.113656946607264	-0.0323007184955163\\
-0.122461701175776	-0.0332991429320097\\
-0.131275390047764	-0.0342984438579438\\
-0.14009803137317	-0.0352986261471835\\
-0.148929643357287	-0.0362996947036256\\
-0.157770244261007	-0.0373016544613985\\
-0.166619852401029	-0.0383045103850392\\
-0.175478486150103	-0.0393082674696931\\
-0.184346163937243	-0.0403129307412948\\
-0.193222904247971	-0.04131850525677\\
-0.202108725624538	-0.0423249961042241\\
-0.211003646666164	-0.0433324084031375\\
-0.219907686029263	-0.044340747304563\\
-0.22882086242769	-0.0453500179913218\\
-0.237743194632961	-0.0463602256782036\\
-0.246674701474509	-0.047371375612165\\
-0.255615401839902	-0.0483834730725317\\
-0.264565314675103	-0.0493965233711986\\
-0.273524458984693	-0.0504105318528409\\
-0.282492853832127	-0.0514255038951063\\
-0.291470518339966	-0.0524414449088339\\
-0.300457471690133	-0.0534583603382553\\
-0.309453733124148	-0.0544762556612041\\
-0.318459321943385	-0.0554951363893289\\
-0.327474257509309	-0.0565150080683005\\
-0.336498559243741	-0.0575358762780292\\
-0.345532246629092	-0.058557746632877\\
-0.354575339208632	-0.059580624781875\\
-0.363627856586727	-0.0606045164089379\\
-0.372689818429112	-0.0616294272330841\\
-0.381761244463128	-0.0626553630086573\\
-0.390842154477999	-0.0636823295255439\\
-0.399932568325073	-0.0647103326093986\\
-0.409032505918098	-0.0657393781218681\\
-0.418141987233471	-0.0667694719608165\\
-0.427261032310513	-0.0678006200605498\\
-0.436389661251721	-0.0688328283920532\\
-0.445527894223045	-0.0698661029632083\\
-0.454675751454146	-0.0709004498190342\\
-0.463833253238675	-0.0719358750419157\\
-0.473000419934531	-0.0729723847518424\\
-0.482177271964145	-0.0740099851066367\\
-0.49136382981474	-0.0750486823022014\\
-0.500560114038621	-0.0760884825727497\\
-0.509766145253432	-0.07712939219105\\
-0.518981944142453	-0.0781714174686687\\
-0.528207531454861	-0.0792145647562099\\
-0.537442928006027	-0.0802588404435675\\
-0.546688154677781	-0.0813042509601625\\
-0.555943232418711	-0.0823508027752007\\
-0.565208182244434	-0.0833985023979161\\
-0.574483025237896	-0.0844473563778246\\
-0.583767782549644	-0.0854973713049796\\
-0.593062475398133	-0.0865485538102224\\
-0.602367125070001	-0.0876009105654413\\
-0.611681752920373	-0.0886544482838283\\
-0.621006380373147	-0.0897091737201421\\
-0.630341028921298	-0.0907650936709663\\
-0.639685720127164	-0.0918222149749726\\
-0.649040475622759	-0.0928805445131876\\
-0.658405317110058	-0.0939400892092572\\
-0.667780266361314	-0.0950008560297155\\
-0.677165345219347	-0.0960628519842575\\
-0.686560575597867	-0.0971260841260079\\
-0.695965979481758	-0.0981905595517926\\
-0.705381578927414	-0.0992562854024197\\
-0.714807396063021	-0.100323268862952\\
-0.724243453088895	-0.101391517162987\\
-0.733689772277774	-0.102461037576938\\
-0.74314637597515	-0.103531837424319\\
-0.752613286599578	-0.104603924070022\\
-0.762090526642991	-0.105677304924611\\
-0.771578118671034	-0.106751987444605\\
-0.78107608532337	-0.107827979132769\\
-0.790584449314021	-0.108905287538408\\
-0.800103233431675	-0.109983920257655\\
-0.809632460540035	-0.111063884933776\\
-0.819172153578127	-0.112145189257458\\
-0.828722335560651	-0.11322784096711\\
-0.838283029578296	-0.114311847849173\\
-0.847854258798095	-0.115397217738409\\
-0.857436046463737	-0.11648395851822\\
-0.867028415895933	-0.117572078120946\\
-0.876631390492733	-0.118661584528181\\
-0.886244993729884	-0.119752485771076\\
-0.895869249161164	-0.120844789930659\\
-0.90550418041874	-0.121938505138151\\
-0.915149811213501	-0.123033639575276\\
-0.924806165335424	-0.124130201474591\\
-0.934473266653911	-0.125228199119795\\
-0.944151139118158	-0.126327640846064\\
-0.953839806757495	-0.127428535040373\\
-0.96353929368176	-0.128530890141817\\
-0.973249624081646	-0.129634714641954\\
-0.982970822229071	-0.130740017085126\\
-0.992702912477538	-0.131846806068798\\
-1.0024459192625	-0.132955090243893\\
-1.01219986710174	-0.134064878315135\\
-1.02196478059573	-0.135176179041384\\
-1.03174068442798	-0.136289001235984\\
-1.04152760336547	-0.137403353767101\\
-1.05132556225898	-0.138519245558083\\
-1.06113458604349	-0.139636685587797\\
-1.07095469973854	-0.140755682890991\\
-1.08078592844863	-0.141876246558644\\
-1.0906282973636	-0.14299838573832\\
-1.10048183175904	-0.144122109634534\\
-1.11034655699663	-0.145247427509109\\
-1.12022249852456	-0.146374348681543\\
-1.13010968187795	-0.147502882529372\\
-1.14000813267919	-0.14863303848854\\
-1.14991787663838	-0.149764826053773\\
-1.15983893955373	-0.150898254778946\\
-1.16977134731194	-0.152033334277468\\
-1.17971512588861	-0.153170074222652\\
-1.18967030134865	-0.154308484348102\\
-1.19963689984674	-0.155448574448093\\
-1.20961494762763	-0.156590354377958\\
-1.21960447102667	-0.15773383405448\\
-1.22960549647016	-0.158879023456277\\
-1.23961805047579	-0.160025932624202\\
-1.24964215965307	-0.161174571661732\\
-1.25967785070372	-0.162324950735377\\
-1.26972515042213	-0.163477080075073\\
-1.27978408569581	-0.16463096997459\\
-1.28985468350574	-0.165786630791934\\
-1.2999369709269	-0.16694407294977\\
-1.31003097512864	-0.168103306935817\\
-1.32013672337515	-0.169264343303275\\
-1.33025424302589	-0.17042719267124\\
-1.34038356153604	-0.171591865725122\\
-1.35052470645693	-0.17275837321707\\
-1.36067770543655	-0.1739267259664\\
-1.3708425862199	-0.175096934860022\\
-1.38101937664954	-0.176269010852874\\
-1.391208104666	-0.177442964968348\\
-1.40140879830824	-0.178618808298746\\
-1.41162148571414	-0.179796552005699\\
-1.42184619512095	-0.180976207320629\\
-1.43208295486572	-0.182157785545181\\
-1.44233179338586	-0.183341298051682\\
-1.45259273921953	-0.18452675628359\\
-1.46286582100615	-0.185714171755949\\
-1.4731510674869	-0.186903556055845\\
-1.48344850750515	-0.188094920842878\\
-1.49375817000701	-0.18928827784961\\
-1.50408008404176	-0.190483638882048\\
-1.51441427876237	-0.191681015820107\\
-1.52476078342599	-0.192880420618086\\
-1.53511962739447	-0.194081865305144\\
-1.5454908401348	-0.195285361985779\\
-1.55587445121967	-0.196490922840319\\
-1.56627049032796	-0.197698560125397\\
-1.57667898724522	-0.198908286174453\\
-1.58709997186425	-0.20012011339822\\
-1.59753347418552	-0.20133405428522\\
-1.60797952431778	-0.202550121402268\\
-1.61843815247852	-0.203768327394976\\
-1.62890938899453	-0.204988684988255\\
-1.6393932643024	-0.206211206986828\\
-1.64988980894907	-0.207435906275742\\
-1.66039905359235	-0.208662795820889\\
-1.6709210290015	-0.209891888669523\\
-1.68145576605768	-0.211123197950782\\
-1.6920032957546	-0.212356736876222\\
-1.70256364919899	-0.21359251874034\\
-1.71313685761119	-0.214830556921118\\
-1.72372295232567	-0.216070864880551\\
-1.73432196479163	-0.217313456165198\\
-1.74493392657354	-0.218558344406721\\
-1.75555886935169	-0.219805543322441\\
-1.76619682492278	-0.221055066715883\\
-1.77684782520047	-0.222306928477339\\
-1.78751190221597	-0.223561142584423\\
-1.79818908811864	-0.224817723102644\\
-1.80887941517649	-0.226076684185964\\
-1.81958291577688	-0.227338040077376\\
-1.830299622427	-0.228601805109478\\
-1.84102956775455	-0.229867993705056\\
-1.85177278450828	-0.231136620377661\\
-1.86252930555859	-0.232407699732203\\
-1.87329916389819	-0.233681246465542\\
-1.88408239264263	-0.234957275367075\\
-1.89487902503097	-0.236235801319353\\
-1.90568909442638	-0.237516839298667\\
-1.91651263431673	-0.238800404375663\\
-1.92734967831525	-0.240086511715963\\
-1.93820026016113	-0.241375176580765\\
-1.94906441372017	-0.242666414327475\\
-1.95994217298541	-0.243960240410327\\
-1.97083357207775	-0.245256670381014\\
-1.98173864524662	-0.246555719889323\\
-1.99265742687059	-0.247857404683766\\
-2.00358995145807	-0.249161740612229\\
-2.01453625364792	-0.250468743622618\\
-2.02549636821013	-0.2517784297635\\
-2.03647033004647	-0.253090815184777\\
-2.04745817419117	-0.254405916138327\\
-2.05845993581159	-0.255723748978678\\
-2.06947565020889	-0.257044330163676\\
-2.08050535281871	-0.258367676255158\\
-2.09154907921184	-0.259693803919632\\
-2.10260686509495	-0.261022729928954\\
-2.11367874631123	-0.262354471161022\\
-2.12476475884113	-0.263689044600466\\
-2.13586493880304	-0.265026467339345\\
-2.14697932245399	-0.266366756577847\\
-2.15810794619039	-0.267709929625002\\
-2.16925084654871	-0.269056003899384\\
-2.18040806020622	-0.270404996929836\\
-2.19157962398171	-0.271756926356191\\
-2.20276557483623	-0.273111809929992\\
-2.21396594987379	-0.274469665515229\\
-2.22518078634215	-0.275830511089077\\
-2.2364101216335	-0.277194364742633\\
-2.24765399328528	-0.278561244681666\\
-2.25891243898086	-0.279931169227369\\
-2.27018549655036	-0.281304156817114\\
-2.28147320397138	-0.282680226005217\\
-2.29277559936976	-0.284059395463711\\
-2.30409272102038	-0.285441683983106\\
-2.31542460734792	-0.286827110473182\\
-2.32677129692765	-0.288215693963764\\
-2.3381328284862	-0.289607453605515\\
-2.34950924090239	-0.291002408670732\\
-2.36090057320799	-0.292400578554144\\
-2.37230686458854	-0.293801982773723\\
-2.38372815438417	-0.29520664097149\\
-2.39516448209039	-0.296614572914338\\
-2.40661588735893	-0.298025798494851\\
-2.41808240999854	-0.299440337732139\\
-2.42956408997587	-0.300858210772669\\
-2.44106096741623	-0.302279437891106\\
-2.45257308260452	-0.30370403949116\\
-2.46410047598599	-0.305132036106443\\
-2.47564318816716	-0.306563448401321\\
-2.48720125991663	-0.307998297171782\\
-2.498774732166	-0.309436603346311\\
-2.51036364601067	-0.310878387986763\\
-2.52196804271077	-0.312323672289247\\
-2.53358796369202	-0.313772477585015\\
-2.54522345054662	-0.31522482534136\\
-2.55687454503414	-0.316680737162518\\
-2.56854128908243	-0.318140234790575\\
-2.58022372478849	-0.319603340106384\\
-2.59192189441946	-0.321070075130487\\
-2.60363584041344	-0.322540462024041\\
-2.61536560538048	-0.324014523089757\\
-2.62711123210348	-0.325492280772836\\
-2.63887276353917	-0.326973757661921\\
-2.65065024281897	-0.328458976490053\\
-2.66244371325002	-0.32994796013563\\
-2.6742532183161	-0.331440731623378\\
-2.68607880167859	-0.332937314125323\\
-2.69792050717744	-0.33443773096178\\
-2.70977837883216	-0.335942005602338\\
-2.72165246084279	-0.337450161666857\\
-2.73354279759088	-0.338962222926474\\
-2.74544943364051	-0.340478213304609\\
-2.75737241373926	-0.341998156877996\\
-2.76931178281924	-0.343522077877694\\
-2.78126758599813	-0.345050000690129\\
-2.79323986858012	-0.346581949858131\\
-2.80522867605706	-0.348117950081984\\
-2.81723405410938	-0.349658026220482\\
-2.82925604860722	-0.351202203291994\\
-2.84129470561142	-0.352750506475525\\
-2.85335007137463	-0.354302961111808\\
-2.86542219234235	-0.355859592704387\\
-2.87751111515399	-0.357420426920705\\
-2.889616886644	-0.358985489593214\\
-2.90173955384289	-0.360554806720482\\
-2.91387916397839	-0.36212840446831\\
-2.92603576447651	-0.363706309170866\\
-2.93820940296269	-0.365288547331809\\
-2.95040012726287	-0.366875145625443\\
-2.96260798540467	-0.368466130897859\\
-2.97483302561849	-0.370061530168104\\
-2.98707529633867	-0.371661370629342\\
-2.99933484620462	-0.373265679650033\\
-3.01161172406201	-0.374874484775124\\
-3.02390597896393	-0.376487813727237\\
-3.03621766017203	-0.378105694407874\\
-3.04854681715776	-0.379728154898626\\
-3.06089349960352	-0.381355223462402\\
-3.0732577574039	-0.382986928544653\\
-3.08563964066683	-0.384623298774606\\
-3.09803919971486	-0.386264362966526\\
-3.11045648508637	-0.387910150120959\\
-3.12289154753678	-0.389560689426007\\
-3.13534443803981	-0.391216010258603\\
-3.14781520778876	-0.392876142185791\\
-3.16030390819772	-0.394541114966029\\
-3.1728105909029	-0.396210958550486\\
-3.18533530776386	-0.397885703084363\\
-3.19787811086484	-0.399565378908213\\
-3.21043905251603	-0.40125001655927\\
-3.22301818525489	-0.402939646772806\\
-3.23561556184748	-0.404634300483475\\
-3.24823123528977	-0.406334008826676\\
-3.26086525880899	-0.408038803139933\\
-3.27351768586497	-0.409748714964277\\
-3.28618857015149	-0.411463776045637\\
-3.29887796559768	-0.413184018336254\\
-3.31158592636935	-0.414909473996086\\
-3.32431250687042	-0.416640175394242\\
-3.3370577617443	-0.41837615511042\\
-3.34982174587527	-0.420117445936345\\
-3.36260451438997	-0.421864080877241\\
-3.37540612265873	-0.423616093153293\\
-3.38822662629711	-0.425373516201127\\
-3.40106608116728	-0.427136383675308\\
-3.4139245433795	-0.428904729449839\\
-3.4268020692936	-0.430678587619678\\
-3.43969871552046	-0.432457992502261\\
-3.4526145389235	-0.434242978639048\\
-3.46554959662016	-0.436033580797058\\
-3.47850394598347	-0.437829833970448\\
-3.49147764464354	-0.439631773382075\\
-3.50447075048909	-0.441439434485082\\
-3.51748332166902	-0.4432528529645\\
-3.53051541659398	-0.445072064738856\\
-3.54356709393791	-0.446897105961791\\
-3.55663841263965	-0.448728013023701\\
-3.56972943190454	-0.450564822553378\\
-3.582840211206	-0.452407571419671\\
-3.5959708102872	-0.454256296733164\\
-3.60912128916262	-0.456111035847853\\
-3.6222917081198	-0.457971826362847\\
-3.63548212772089	-0.459838706124083\\
-3.64869260880438	-0.461711713226045\\
-3.66192321248679	-0.463590886013506\\
-3.67517400016434	-0.465476263083275\\
-3.68844503351465	-0.467367883285969\\
-3.70173637449852	-0.469265785727781\\
-3.71504808536159	-0.471170009772283\\
-3.72838022863616	-0.473080595042228\\
-3.74173286714289	-0.474997581421362\\
-3.75510606399261	-0.47692100905628\\
-3.76849988258813	-0.478850918358247\\
-3.78191438662599	-0.480787350005085\\
-3.79534964009832	-0.482730344943039\\
-3.80880570729464	-0.484679944388671\\
-3.82228265280376	-0.486636189830772\\
-3.83578054151556	-0.488599123032277\\
-3.84929943862293	-0.490568786032207\\
-3.86283940962365	-0.492545221147626\\
-3.87640052032225	-0.494528470975596\\
-3.889982836832	-0.496518578395179\\
-3.90358642557675	-0.498515586569417\\
-3.91721135329299	-0.500519538947364\\
-3.93085768703172	-0.502530479266108\\
-3.94452549416049	-0.504548451552819\\
-3.95821484236534	-0.506573500126814\\
-3.9719257996529	-0.508605669601638\\
-3.9856584343523	-0.510645004887156\\
-3.99941281511731	-0.512691551191674\\
-4.01318901092836	-0.514745354024062\\
-4.02698709109462	-0.516806459195902\\
-4.04080712525608	-0.518874912823656\\
-4.05464918338567	-0.520950761330846\\
-4.0685133357914	-0.523034051450249\\
-4.0823996531185	-0.525124830226122\\
-4.09630820635152	-0.527223145016428\\
-4.11023906681661	-0.529329043495096\\
-4.12419230618362	-0.531442573654284\\
-4.1381679964684	-0.533563783806679\\
-4.15216621003492	-0.535692722587792\\
-4.16618701959764	-0.537829438958295\\
-4.18023049822368	-0.539973982206356\\
-4.19429671933517	-0.542126401950016\\
-4.20838575671149	-0.544286748139559\\
-4.22249768449166	-0.546455071059926\\
-4.23663257717665	-0.54863142133313\\
-4.2507905096317	-0.550815849920704\\
-4.26497155708877	-0.553008408126156\\
-4.27917579514892	-0.555209147597459\\
-4.29340329978466	-0.557418120329546\\
-4.30765414734249	-0.559635378666845\\
-4.3219284145453	-0.561860975305807\\
-4.33622617849485	-0.564094963297484\\
-4.35054751667427	-0.566337396050111\\
-4.36489250695062	-0.568588327331711\\
-4.37926122757736	-0.570847811272728\\
-4.39365375719697	-0.573115902368674\\
-4.40807017484351	-0.575392655482804\\
-4.42251055994521	-0.577678125848811\\
-4.43697499232713	-0.579972369073541\\
-4.45146355221377	-0.582275441139739\\
-4.46597632023178	-0.584587398408801\\
-4.48051337741262	-0.586908297623573\\
-4.49507480519527	-0.589238195911149\\
-4.50966068542901	-0.591577150785712\\
-4.52427110037613	-0.593925220151383\\
-4.53890613271475	-0.596282462305109\\
-4.5535658655416	-0.598648935939564\\
-4.56825038237489	-0.601024700146074\\
-4.58295976715711	-0.603409814417578\\
-4.59769410425797	-0.605804338651606\\
-4.61245347847723	-0.608208333153278\\
-4.62723797504771	-0.610621858638336\\
-4.64204767963819	-0.613044976236203\\
-4.65688267835639	-0.615477747493059\\
-4.67174305775201	-0.617920234374951\\
-4.68662890481972	-0.620372499270926\\
-4.70154030700223	-0.622834604996193\\
-4.71647735219339	-0.625306614795311\\
-4.73144012874125	-0.6277885923454\\
-4.74642872545128	-0.630280601759386\\
-4.76144323158942	-0.632782707589277\\
-4.77648373688537	-0.635294974829453\\
-4.79155033153576	-0.637817468919999\\
-4.80664310620739	-0.640350255750058\\
-4.82176215204053	-0.642893401661215\\
-4.8369075606522	-0.645446973450915\\
-4.85207942413951	-0.648011038375901\\
-4.86727783508304	-0.650585664155698\\
-4.88250288655017	-0.653170918976101\\
-4.89775467209858	-0.655766871492728\\
-4.91303328577962	-0.658373590834566\\
-4.92833882214187	-0.660991146607587\\
-4.94367137623457	-0.663619608898358\\
-4.95903104361123	-0.666259048277715\\
-4.97441792033315	-0.668909535804446\\
-4.98983210297308	-0.67157114302902\\
-5.00527368861878	-0.674243941997343\\
-5.02074277487677	-0.676928005254548\\
-5.03623945987599	-0.679623405848817\\
-5.05176384227154	-0.682330217335245\\
-5.06731602124842	-0.685048513779726\\
-5.08289609652542	-0.687778369762884\\
-5.09850416835885	-0.690519860384029\\
-5.11414033754648	-0.693273061265163\\
-5.1298047054314	-0.696038048555003\\
-5.14549737390603	-0.698814898933056\\
-5.16121844541602	-0.701603689613721\\
-5.1769680229643	-0.704404498350436\\
-5.19274621011512	-0.70721740343985\\
-5.20855311099816	-0.710042483726049\\
-5.22438883031262	-0.712879818604798\\
-5.24025347333139	-0.71572948802785\\
-5.25614714590525	-0.718591572507268\\
-5.27206995446715	-0.721466153119801\\
-5.2880220060364	-0.724353311511294\\
-5.30400340822306	-0.727253129901138\\
-5.32001426923226	-0.730165691086766\\
-5.33605469786861	-0.733091078448185\\
-5.35212480354063	-0.736029375952549\\
-5.36822469626523	-0.738980668158773\\
-5.38435448667222	-0.741945040222205\\
-5.40051428600889	-0.74492257789931\\
-5.4167042061446	-0.747913367552428\\
-5.43292435957543	-0.750917496154558\\
-5.44917485942887	-0.753935051294193\\
-5.46545581946855	-0.756966121180198\\
-5.48176735409904	-0.760010794646732\\
-5.49810957837062	-0.763069161158219\\
-5.51448260798422	-0.76614131081436\\
-5.53088655929628	-0.769227334355203\\
-5.54732154932375	-0.772327323166242\\
-5.56378769574907	-0.775441369283584\\
-5.58028511692522	-0.77856956539915\\
-5.59681393188086	-0.781712004865929\\
-5.61337426032547	-0.784868781703289\\
-5.62996622265451	-0.788039990602319\\
-5.64658993995476	-0.791225726931248\\
-5.66324553400951	-0.794426086740891\\
-5.67993312730402	-0.797641166770163\\
-5.69665284303084	-0.800871064451634\\
-5.71340480509534	-0.804115877917147\\
-5.73018913812115	-0.807375706003481\\
-5.74700596745577	-0.810650648258077\\
-5.76385541917618	-0.813940804944808\\
-5.78073762009448	-0.817246277049815\\
-5.79765269776367	-0.820567166287389\\
-5.81460078048338	-0.823903575105921\\
-5.83158199730574	-0.827255606693898\\
-5.84859647804127	-0.830623364985968\\
-5.86564435326482	-0.83400695466905\\
-5.88272575432161	-0.837406481188523\\
-5.89984081333328	-0.840822050754448\\
-5.91698966320402	-0.844253770347884\\
-5.93417243762677	-0.847701747727235\\
-5.95138927108949	-0.851166091434674\\
-5.96864029888145	-0.854646910802632\\
-5.98592565709961	-0.858144315960349\\
-6.00324548265508	-0.861658417840473\\
-6.02059991327962	-0.865189328185756\\
-6.0379890875322	-0.868737159555787\\
-6.05541314480565	-0.87230202533381\\
-6.07287222533336	-0.875884039733592\\
-6.09036647019605	-0.879483317806383\\
-6.10789602132862	-0.883099975447924\\
-6.12546102152706	-0.886734129405536\\
-6.1430616144554	-0.890385897285274\\
-6.16069794465279	-0.894055397559162\\
-6.17837015754063	-0.897742749572483\\
-6.19607839942973	-0.901448073551171\\
-6.21382281752759	-0.90517149060924\\
-6.23160355994579	-0.908913122756327\\
-6.24942077570731	-0.912673092905275\\
-6.26727461475413	-0.916451524879825\\
-6.28516522795473	-0.920248543422363\\
-6.30309276711175	-0.924064274201761\\
-6.32105738496976	-0.927898843821282\\
-6.33905923522301	-0.931752379826587\\
-6.35709847252336	-0.935625010713799\\
-6.37517525248826	-0.93951686593768\\
-6.39328973170873	-0.943428075919857\\
-6.41144206775762	-0.947358772057167\\
-6.42963241919772	-0.951309086730058\\
-6.44786094559014	-0.955279153311108\\
-6.46612780750267	-0.959269106173598\\
-6.4844331665183	-0.963279080700205\\
-6.50277718524377	-0.967309213291767\\
-6.52116002731824	-0.971359641376146\\
-6.53958185742208	-0.975430503417185\\
-6.55804284128565	-0.97952193892375\\
-6.57654314569833	-0.983634088458881\\
-6.59508293851752	-0.98776709364903\\
-6.61366238867776	-0.991921097193395\\
-6.6322816662	-0.996096242873363\\
-6.65094094220092	-1.00029267556204\\
-6.66964038890238	-1.0045105412339\\
-6.68838017964094	-1.0087499869745\\
-6.70716048887749	-1.01301116099036\\
-6.72598149220704	-1.01729421261888\\
-6.74484336636852	-1.02159929233842\\
-6.76374628925477	-1.02592655177846\\
-6.78269043992262	-1.03027614372984\\
-6.80167599860299	-1.03464822215521\\
-6.8207031467113	-1.03904294219947\\
-6.83977206685775	-1.04346046020039\\
-6.85888294285792	-1.04790093369938\\
-6.87803595974336	-1.05236452145226\\
-6.89723130377236	-1.05685138344029\\
-6.91646916244079	-1.0613616808812\\
-6.93574972449313	-1.06589557624043\\
-6.95507317993354	-1.07045323324243\\
-6.97443972003712	-1.0750348168821\\
-6.99384953736127	-1.07964049343643\\
-7.01330282575716	-1.08427043047611\\
-7.03279978038137	-1.08892479687743\\
-7.05234059770761	-1.09360376283426\\
-7.07192547553861	-1.09830749987005\\
-7.09155461301816	-1.1030361808502\\
-7.11122821064323	-1.10778997999431\\
-7.13094647027625	-1.11256907288878\\
-7.15070959515757	-1.11737363649938\\
-7.17051778991801	-1.12220384918412\\
-7.19037126059156	-1.12705989070614\\
-7.21027021462828	-1.13194194224678\\
-7.23021486090725	-1.13685018641886\\
-7.25020540974979	-1.14178480728004\\
-7.27024207293269	-1.14674599034634\\
-7.29032506370176	-1.15173392260583\\
-7.31045459678537	-1.15674879253253\\
-7.33063088840827	-1.16179079010032\\
-7.35085415630551	-1.1668601067972\\
-7.37112461973656	-1.17195693563956\\
-7.39144249949952	-1.17708147118669\\
-7.41180801794562	-1.18223390955547\\
-7.43222139899377	-1.18741444843514\\
-7.45268286814535	-1.1926232871024\\
-7.47319265249915	-1.1978606264365\\
-7.49375098076652	-1.20312666893464\\
-7.51435808328663	-1.20842161872753\\
-7.53501419204199	-1.21374568159507\\
-7.55571954067409	-1.21909906498229\\
-7.5764743644993	-1.22448197801542\\
-7.59727890052485	-1.2298946315182\\
-7.61813338746515	-1.23533723802834\\
-7.63903806575815	-1.2408100118142\\
-7.65999317758202	-1.24631316889168\\
-7.68099896687198	-1.25184692704128\\
-7.70205567933731	-1.25741150582538\\
-7.72316356247862	-1.26300712660573\\
-7.74432286560529	-1.26863401256119\\
-7.76553383985316	-1.27429238870555\\
-7.7867967382024	-1.27998248190577\\
-7.8081118154956	-1.28570452090026\\
-7.82947932845611	-1.29145873631746\\
-7.85089953570663	-1.29724536069468\\
-7.8723726977879	-1.30306462849709\\
-7.89389907717781	-1.308916776137\\
-7.9154789383106	-1.31480204199337\\
-7.93711254759635	-1.32072066643154\\
-7.95880017344075	-1.32667289182322\\
-7.98054208626503	-1.33265896256672\\
-8.00233855852624	-1.33867912510742\\
-8.0241898647377	-1.34473362795857\\
-8.04609628148975	-1.35082272172221\\
-8.0680580874708	-1.35694665911048\\
-8.09007556348852	-1.36310569496713\\
-8.11214899249147	-1.36930008628935\\
-8.13427865959085	-1.3755300922498\\
-8.15646485208266	-1.381795974219\\
-8.17870785947001	-1.38809799578794\\
-8.20100797348585	-1.39443642279104\\
-8.22336548811585	-1.40081152332934\\
-8.24578069962177	-1.40722356779402\\
-8.2682539065649	-1.41367282889022\\
-8.29078540982999	-1.42015958166118\\
-8.31337551264938	-1.42668410351264\\
-8.33602452062754	-1.43324667423763\\
-8.35873274176582	-1.43984757604149\\
-8.38150048648761	-1.44648709356731\\
-8.40432806766379	-1.45316551392163\\
-8.42721580063855	-1.45988312670054\\
-8.45016400325549	-1.46664022401603\\
-8.47317299588414	-1.47343710052278\\
-8.49624310144678	-1.48027405344526\\
-8.51937464544562	-1.48715138260518\\
-8.54256795599039	-1.4940693904493\\
-8.56582336382625	-1.50102838207768\\
-8.58914120236205	-1.5080286652722\\
-8.61252180769908	-1.51507055052553\\
-8.6359655186601	-1.52215435107044\\
-8.65947267681879	-1.52928038290961\\
-8.68304362652965	-1.53644896484565\\
-8.70667871495821	-1.54366041851177\\
-8.73037829211178	-1.55091506840262\\
-8.75414271087051	-1.55821324190573\\
-8.77797232701888	-1.56555526933329\\
-8.80186749927775	-1.57294148395438\\
-8.82582858933668	-1.58037222202766\\
-8.84985596188684	-1.58784782283452\\
-8.87394998465425	-1.59536862871257\\
-8.89811102843362	-1.60293498508982\\
-8.92233946712251	-1.61054724051907\\
-8.94663567775614	-1.618205746713\\
-8.97100004054249	-1.62591085857958\\
-8.99543293889812	-1.63366293425812\\
-9.01993475948424	-1.6414623351557\\
-9.04450589224355	-1.64930942598419\\
-9.06914673043738	-1.65720457479776\\
-9.09385767068352	-1.66514815303094\\
-9.11863911299448	-1.6731405355372\\
-9.1434914608164	-1.68118210062806\\
-9.16841512106838	-1.68927323011283\\
-9.19341050418253	-1.69741430933884\\
-9.21847802414447	-1.70560572723228\\
-9.24361809853452	-1.71384787633965\\
-9.2688311485694	-1.72214115286974\\
-9.29411759914459	-1.7304859567363\\
-9.3194778788773	-1.73888269160128\\
-9.34491242015005	-1.74733176491869\\
-9.37042165915489	-1.75583358797915\\
-9.39600603593836	-1.76438857595503\\
-9.4216659944469	-1.77299714794632\\
-9.44740198257323	-1.78165972702712\\
-9.47321445220312	-1.79037674029283\\
-9.4991038592631	-1.79914861890806\\
-9.52507066376871	-1.80797579815523\\
-9.5511153298736	-1.81685871748394\\
-9.57723832591927	-1.82579782056097\\
-9.60344012448563	-1.83479355532118\\
-9.62972120244225	-1.84384637401905\\
-9.65608204100052	-1.85295673328105\\
-9.68252312576642	-1.86212509415883\\
-9.70904494679428	-1.87135192218317\\
-9.73564799864122	-1.88063768741874\\
-9.76233278042251	-1.88998286451978\\
-9.78909979586776	-1.89938793278655\\
-9.81594955337794	-1.90885337622268\\
-9.84288256608338	-1.91837968359341\\
-9.86989935190256	-1.92796734848472\\
-9.89700043360188	-1.93761686936338\\
-9.92418633885638	-1.94732874963798\\
-9.95145760031134	-1.9571034977208\\
-9.97881475564497	-1.96694162709077\\
-10.0062583476319	-1.97684365635738\\
-10.033788924208	-1.98681010932557\\
-10.0614070385357	-1.99684151506165\\
-10.089113249071	-2.00693840796035\\
-10.1169081196311	-2.0171013278128\\
-10.1447922194633	-2.02733081987574\\
-10.1727661233145	-2.03762743494169\\
-10.2008304115033	-2.04799172941042\\
-10.2289856699911	-2.0584242653614\\
-10.2572324904563	-2.06892561062759\\
-10.2855714703684	-2.07949633887024\\
-10.3140032130643	-2.09013702965514\\
-10.3425283278249	-2.10084826852989\\
-10.3711474299539	-2.1116306471026\\
-10.399861140857	-2.1224847631218\\
-10.4286700881231	-2.1334112205577\\
-10.4575749056067	-2.14441062968481\\
-10.4865762335114	-2.15548360716591\\
-10.5156747184749	-2.16663077613744\\
-10.5448710136558	-2.17785276629628\\
-10.5741657788212	-2.18915021398807\\
-10.6035596804367	-2.20052376229693\\
-10.6330533917569	-2.21197406113675\\
-10.6626475929178	-2.22350176734401\\
-10.6923429710316	-2.23510754477223\\
-10.7221402202819	-2.24679206438793\\
-10.7520400420209	-2.25855600436836\\
-10.782043144869	-2.2704000502008\\
-10.8121502448154	-2.28232489478368\\
-10.8423620653202	-2.29433123852939\\
-10.8726793374191	-2.30641978946886\\
-10.9031027998298	-2.31859126335804\\
-10.9336331990592	-2.33084638378619\\
-10.9642712895142	-2.34318588228608\\
-10.9950178336128	-2.35561049844616\\
-11.0258736018984	-2.3681209800247\\
-11.0568393731556	-2.38071808306594\\
-11.087915934528	-2.39340257201837\\
-11.1191040816385	-2.40617521985506\\
-11.150404618711	-2.41903680819619\\
-11.1818183586956	-2.43198812743381\\
-11.2133461233947	-2.44502997685881\\
-11.2449887435922	-2.45816316479021\\
-11.2767470591849	-2.47138850870685\\
-11.308621919316	-2.48470683538144\\
-11.3406141825119	-2.49811898101711\\
-11.3727247168203	-2.51162579138642\\
-11.4049543999518	-2.52522812197303\\
-11.4373041194242	-2.53892683811595\\
-11.4697747727085	-2.55272281515646\\
-11.5023672673787	-2.56661693858784\\
-11.5350825212638	-2.58061010420786\\
-11.5679214626034	-2.59470321827419\\
-11.6008850302048	-2.6088971976627\\
-11.6339741736051	-2.62319297002884\\
-11.6671898532344	-2.63759147397205\\
-11.7005330405836	-2.65209365920332\\
-11.734004718375	-2.66670048671597\\
-11.7676058807354	-2.68141292895981\\
-11.8013375333741	-2.6962319700185\\
-11.835200693763	-2.71115860579059\\
-11.8691963913209	-2.72619384417386\\
-11.9033256676012	-2.74133870525348\\
-11.9375895764836	-2.75659422149374\\
-11.9719891843691	-2.77196143793363\\
-12.0065255703792	-2.78744141238627\\
-12.0411998265592	-2.80303521564239\\
-12.0760130580853	-2.81874393167773\\
-12.1109663834757	-2.83456865786482\\
-12.1460609348067	-2.8505105051888\\
-12.1812978579324	-2.8665705984678\\
-12.2166783127093	-2.8827500765777\\
-12.2522034732254	-2.89905009268152\\
-12.2878745280338	-2.91547181446347\\
-12.3236926803914	-2.9320164243679\\
-12.3596591485026	-2.9486851198431\\
-12.3957751657679	-2.96547911359024\\
-12.4320419810372	-2.9823996338174\\
-12.4684608588698	-2.99944792449909\\
-12.5050330797979	-3.01662524564099\\
-12.5417599405979	-3.03393287355055\\
-12.5786427545653	-3.05137210111305\\
-12.6156828517971	-3.06894423807381\\
-12.6528815794796	-3.08665061132619\\
-12.690240302182	-3.10449256520599\\
-12.7277604021571	-3.12247146179204\\
-12.7654432796481	-3.14058868121339\\
-12.8032903532022	-3.15884562196319\\
-12.8413030599909	-3.17724370121933\\
-12.8794828561375	-3.19578435517229\\
-12.917831217052	-3.21446903936006\\
-12.9563496377727	-3.23329922901058\\
-12.9950396333167	-3.25227641939175\\
-13.0339027390368	-3.27140212616927\\
-13.0729405109872	-3.29067788577256\\
-13.1121545262978	-3.31010525576884\\
-13.1515463835559	-3.32968581524581\\
-13.1911177031976	-3.34942116520291\\
-13.2308701279079	-3.36931292895163\\
-13.2708053230294	-3.38936275252495\\
-13.3109249769814	-3.40957230509628\\
-13.3512308016879	-3.42994327940804\\
-13.3917245330162	-3.45047739221032\\
-13.4324079312252	-3.47117638470973\\
-13.473282781425	-3.49204202302882\\
-13.5143508940461	-3.51307609867639\\
-13.5556141053216	-3.53428042902891\\
-13.5970742777789	-3.55565685782345\\
-13.6387333007448	-3.57720725566247\\
-13.6805930908616	-3.59893352053065\\
-13.7226555926169	-3.62083757832435\\
-13.7649227788849	-3.64292138339384\\
-13.807396651482	-3.66518691909891\\
-13.8500792417357	-3.68763619837796\\
-13.8929726110675	-3.71027126433127\\
-13.9360788515902	-3.73309419081866\\
-13.9794000867204	-3.7561070830721\\
-14.0229384718059	-3.77931207832362\\
-14.0666961947694	-3.80271134644905\\
-14.1106754767681	-3.82630709062803\\
-14.1548785728705	-3.85010154802086\\
-14.1993077727496	-3.87409699046258\\
-14.2439654013955	-3.89829572517495\\
-14.2888538198445	-3.92270009549678\\
-14.333975425929	-3.94731248163324\\
-14.3793326550454	-3.97213530142469\\
-14.4249279809434	-3.99717101113576\\
-14.4707639165351	-4.02242210626516\\
-14.5168430147264	-4.04789112237702\\
-14.56316786927	-4.07358063595434\\
-14.6097411156417	-4.09949326527539\\
-14.6565654319397	-4.1256316713136\\
-14.7036435398093	-4.15199855866198\\
-14.7509782053914	-4.17859667648262\\
-14.7985722402985	-4.20542881948222\\
-14.8464285026163	-4.23249782891448\\
-14.8945498979339	-4.25980659361029\\
-14.9429393804021	-4.28735805103652\\
-14.9915999538221	-4.31515518838461\\
-15.0405346727639	-4.34320104368961\\
-15.089746643717	-4.37149870698114\\
-15.1392390262741	-4.40005132146691\\
-15.189015034348	-4.42886208475036\\
-15.2390779374241	-4.45793425008319\\
-15.289431061849	-4.48727112765435\\
-15.3400777921569	-4.51687608591644\\
-15.3910215724345	-4.54675255295117\\
-15.4422659077265	-4.57690401787496\\
-15.4938143654827	-4.60733403228631\\
-15.5456705770483	-4.63804621175631\\
-15.5978382391989	-4.66904423736399\\
-15.6503211157219	-4.70033185727796\\
-15.703123039046	-4.73191288838616\\
-15.7562479119208	-4.76379121797545\\
-15.8096997091474	-4.79597080546284\\
-15.863482479363	-4.82845568418032\\
-15.9176003468815	-4.8612499632153\\
-15.972057513591	-4.89435782930875\\
-16.0268582609116	-4.92778354881315\\
-16.0820069518153	-4.96153146971264\\
-16.1375080329108	-4.99560602370774\\
-16.1933660365942	-5.03001172836694\\
-16.2495855832707	-5.06475318934817\\
-16.306171383648	-5.0998351026923\\
-16.3631282411045	-5.13526225719216\\
-16.4204610541366	-5.17103953683951\\
-16.4781748188864	-5.20717192335333\\
-16.5362746317545	-5.24366449879282\\
-16.5947656921008	-5.28052244825811\\
-16.6536533050365	-5.31775106268276\\
-16.7129428843112	-5.35535574172141\\
-16.7726399553005	-5.39334199673674\\
-16.832750158095	-5.43171545388966\\
-16.8932792506988	-5.47048185733733\\
-16.9542331123389	-5.50964707254327\\
-17.0156177468924	-5.54921708970444\\
-17.0774392864352	-5.5891980273003\\
-17.1397039949181	-5.62959613576911\\
-17.2024182719753	-5.6704178013168\\
-17.2655886568719	-5.71166954986454\\
-17.3292218325956	-5.7533580511407\\
-17.3933246300999	-5.79549012292401\\
-17.4579040327038	-5.83807273544421\\
-17.5229671806583	-5.88111301594775\\
-17.588521375883	-5.92461825343561\\
-17.6545740868847	-5.96859590358139\\
-17.7211329538633	-6.01305359383765\\
-17.788205794015	-6.05799912873961\\
-17.8558006070426	-6.1034404954151\\
-17.9239255808809	-6.14938586931055\\
-17.9925890976487	-6.19584362014342\\
-18.0617997398389	-6.24282231809166\\
-18.1315662967553	-6.29033074023182\\
-18.201897771212	-6.33837787723779\\
-18.272803386505	-6.38697294035288\\
-18.344292593671	-6.43612536864898\\
-18.4163750790475	-6.48584483658678\\
-18.4890607721494	-6.5361412618924\\
-18.5623598538775	-6.58702481376629\\
-18.6362827650768	-6.6385059214416\\
-18.7108402154616	-6.69059528310964\\
-18.7860431929278	-6.74330387523198\\
-18.8619029732705	-6.79664296225898\\
-18.9384311303316	-6.8506241067766\\
-19.0156395465964	-6.90525918010409\\
-19.0935404242669	-6.96056037336703\\
-19.1721462968355	-7.01654020907127\\
-19.2514700411875	-7.07321155320556\\
-19.331524890261	-7.13058762790171\\
-19.4123244462958	-7.18868202468385\\
-19.4938826947046	-7.24750871833944\\
-19.5762140186012	-7.30708208144805\\
-19.6593332140244	-7.36741689960493\\
-19.7432555058966	-7.42852838738041\\
-19.8279965647616	-7.49043220505753\\
-19.9135725243471	-7.55314447619446\\
-20	-7.61668180606046\\
-20.087296108049	-7.68106130099856\\
-20.1754784861501	-7.74630058877092\\
-20.2645653146751	-7.81241783994782\\
-20.3545753392086	-7.87943179040466\\
-20.445527894223	-7.94736176499717\\
-20.537442928006	-8.01622770248905\\
-20.6303410289213	-8.08605018181295\\
-20.7242434530889	-8.15685044975079\\
-20.8191721535781	-8.22865045012704\\
-20.9151498112135	-8.30147285461463\\
-21.0121998671017	-8.37534109526225\\
-21.1103465569966	-8.45027939885926\\
-21.2096149476276	-8.52631282326459\\
-21.3100309751286	-8.60346729583547\\
-21.4116214857141	-8.6817696541039\\
-21.5144142787624	-8.76124768885984\\
-21.6184381524785	-8.8419301898147\\
-21.7237229523257	-8.92384699403223\\
-21.830299622427	-9.00702903733108\\
-21.9382002601611	-9.09150840888008\\
-22.0474581741912	-9.17731840922766\\
-22.1581079461904	-9.26449361202742\\
-22.2701854965504	-9.35306992974662\\
-22.3837281543842	-9.44308468366938\\
-22.498774732166	-9.53457667853648\\
-22.6153656053805	-9.62758628219481\\
-22.7335427975909	-9.72215551066574\\
-22.8533500713746	-9.81832811908065\\
-22.9748330256185	-9.91614969897643\\
-23.0980391997149	-10.0156677824917\\
-23.2230181852549	-10.1169319540602\\
-23.3498217458753	-10.2199939702576\\
-23.4785039459835	-10.3249078885271\\
-23.6091212891626	-10.431730205584\\
-23.7417328671429	-10.5405200063875\\
-23.8764005203222	-10.6513391246617\\
-24.0131890109284	-10.7642523160572\\
-24.1521662100349	-10.8793274451679\\
-24.2934032997847	-10.9966356877541\\
-24.4369749923271	-11.1162517496802\\
-24.5829597671571	-11.2382541042532\\
-24.7314401287412	-11.3627252498497\\
-24.8825028865502	-11.4897519899479\\
-25.036239459876	-11.6194257379455\\
-25.1927462101151	-11.7518428494422\\
-25.3521248035406	-11.8871049850124\\
-25.5144826079842	-12.0253195068873\\
-25.679933127304	-12.1665999134226\\
-25.8485964780413	-12.3110663157557\\
-26.0205999132796	-12.4588459616645\\
-26.1960783994297	-12.6100738123525\\
-26.3751752524882	-12.7648931787079\\
-26.5580428412857	-12.9234564245512\\
-26.7448433663685	-13.0859257455179\\
-26.9357497244931	-13.252474033552\\
-27.1309464702762	-13.423285838557\\
-27.3306308884083	-13.5985584406089\\
-27.535014192042	-13.7785030483452\\
-27.7443228656053	-13.9633461417778\\
-27.9588001734407	-14.1533309809323\\
-28.17870785947	-14.3487193055131\\
-28.4043280676638	-14.5497932553726\\
-28.6359655186601	-14.7568575471303\\
-28.8739499846542	-14.9702419490624\\
-29.1186391129945	-15.1903041046977\\
-29.3704216591549	-15.4174327657904\\
-29.6297212024423	-15.6520515080272\\
-29.8970004336019	-15.8946230186333\\
-30.1727661233146	-16.1456540648691\\
-30.4575749056067	-16.4057012774404\\
-30.7520400420209	-16.6753779146694\\
-31.0568393731556	-16.9553618140341\\
-31.3727247168203	-17.2464047903237\\
-31.7005330405836	-17.5493438081866\\
-32.0411998265593	-17.8651143469165\\
-32.3957751657679	-18.1947664948142\\
-32.7654432796481	-18.5394844706992\\
-33.1515463835559	-18.900610487408\\
-33.5556141053216	-19.2796741703525\\
-33.9794000867204	-19.6784291589894\\
-34.4249279809434	-20.0988991042921\\
-34.8945498979339	-20.5434361140866\\
-35.3910215724345	-21.0147959210987\\
-35.9176003468815	-21.5162358658485\\
-36.4781748188863	-22.0516445441318\\
-37.0774392864352	-22.6257162522201\\
-37.7211329538632	-23.2441901928339\\
-38.4163750790475	-23.914185622337\\
-39.1721462968354	-24.644683174164\\
-40	-25.4472362397757\\
-40.9151498112134	-26.337058440633\\
-41.9382002601611	-27.334754305311\\
-43.0980391997147	-28.4692116856714\\
-44.4369749923271	-29.7827389432584\\
-46.0205999132794	-31.3409283525451\\
-47.9588001734407	-33.2536661236935\\
-50.4575749056064	-35.7269513888926\\
-53.9794000867204	-39.2232601245809\\
-59.999999999999	-45.2183166135704\\
-inf	-inf\\
-60	-45.2183166135714\\
-53.9794000867204	-39.2232601245809\\
-50.4575749056067	-35.7269513888928\\
-47.9588001734407	-33.2536661236935\\
-46.0205999132796	-31.3409283525453\\
-44.4369749923271	-29.7827389432584\\
-43.0980391997149	-28.4692116856715\\
-41.9382002601611	-27.334754305311\\
-40.9151498112135	-26.3370584406331\\
-40	-25.4472362397757\\
-39.1721462968355	-24.6446831741641\\
-38.4163750790475	-23.914185622337\\
-37.7211329538633	-23.244190192834\\
-37.0774392864352	-22.6257162522201\\
-36.4781748188864	-22.0516445441319\\
-35.9176003468815	-21.5162358658485\\
-35.3910215724345	-21.0147959210987\\
-34.8945498979339	-20.5434361140866\\
-34.4249279809434	-20.0988991042922\\
-33.9794000867204	-19.6784291589894\\
-33.5556141053216	-19.2796741703525\\
-33.1515463835559	-18.900610487408\\
-32.7654432796481	-18.5394844706992\\
-32.3957751657679	-18.1947664948142\\
-32.0411998265592	-17.8651143469164\\
-31.7005330405836	-17.5493438081866\\
-31.3727247168202	-17.2464047903236\\
-31.0568393731556	-16.9553618140341\\
-30.7520400420209	-16.6753779146694\\
-30.4575749056067	-16.4057012774404\\
-30.1727661233145	-16.145654064869\\
-29.8970004336019	-15.8946230186333\\
-29.6297212024422	-15.6520515080272\\
-29.3704216591549	-15.4174327657904\\
-29.1186391129945	-15.1903041046976\\
-28.8739499846542	-14.9702419490624\\
-28.6359655186601	-14.7568575471303\\
-28.4043280676638	-14.5497932553726\\
-28.17870785947	-14.3487193055131\\
-27.9588001734407	-14.1533309809323\\
-27.7443228656053	-13.9633461417778\\
-27.535014192042	-13.7785030483452\\
-27.3306308884083	-13.5985584406089\\
-27.1309464702763	-13.423285838557\\
-26.9357497244931	-13.252474033552\\
-26.7448433663685	-13.0859257455179\\
-26.5580428412857	-12.9234564245512\\
-26.3751752524883	-12.7648931787079\\
-26.1960783994297	-12.6100738123525\\
-26.0205999132796	-12.4588459616645\\
-25.8485964780413	-12.3110663157557\\
-25.679933127304	-12.1665999134226\\
-25.5144826079842	-12.0253195068873\\
-25.3521248035406	-11.8871049850124\\
-25.1927462101151	-11.7518428494422\\
-25.036239459876	-11.6194257379455\\
-24.8825028865502	-11.4897519899479\\
-24.7314401287413	-11.3627252498497\\
-24.5829597671571	-11.2382541042532\\
-24.4369749923271	-11.1162517496802\\
-24.2934032997847	-10.9966356877541\\
-24.1521662100349	-10.879327445168\\
-24.0131890109284	-10.7642523160572\\
-23.8764005203223	-10.6513391246617\\
-23.7417328671429	-10.5405200063875\\
-23.6091212891626	-10.431730205584\\
-23.4785039459835	-10.3249078885271\\
-23.3498217458753	-10.2199939702576\\
-23.2230181852549	-10.1169319540602\\
-23.0980391997149	-10.0156677824917\\
-22.9748330256185	-9.91614969897643\\
-22.8533500713746	-9.81832811908066\\
-22.7335427975909	-9.72215551066574\\
-22.6153656053805	-9.62758628219482\\
-22.498774732166	-9.53457667853648\\
-22.3837281543842	-9.44308468366939\\
-22.2701854965504	-9.35306992974662\\
-22.1581079461904	-9.26449361202743\\
-22.0474581741912	-9.17731840922766\\
-21.9382002601611	-9.09150840888008\\
-21.830299622427	-9.00702903733108\\
-21.7237229523257	-8.92384699403224\\
-21.6184381524785	-8.8419301898147\\
-21.5144142787624	-8.76124768885985\\
-21.4116214857141	-8.6817696541039\\
-21.3100309751287	-8.60346729583548\\
-21.2096149476276	-8.52631282326459\\
-21.1103465569966	-8.45027939885927\\
-21.0121998671017	-8.37534109526225\\
-20.9151498112135	-8.30147285461464\\
-20.8191721535781	-8.22865045012704\\
-20.7242434530889	-8.1568504497508\\
-20.6303410289213	-8.08605018181295\\
-20.537442928006	-8.01622770248906\\
-20.445527894223	-7.94736176499717\\
-20.3545753392086	-7.87943179040467\\
-20.2645653146751	-7.81241783994782\\
-20.1754784861501	-7.74630058877093\\
-20.087296108049	-7.68106130099856\\
-20	-7.61668180606046\\
-19.9135725243471	-7.55314447619446\\
-19.8279965647616	-7.49043220505754\\
-19.7432555058966	-7.42852838738041\\
-19.6593332140244	-7.36741689960493\\
-19.5762140186012	-7.30708208144805\\
-19.4938826947046	-7.24750871833945\\
-19.4123244462958	-7.18868202468385\\
-19.331524890261	-7.13058762790172\\
-19.2514700411875	-7.07321155320556\\
-19.1721462968355	-7.01654020907127\\
-19.0935404242669	-6.96056037336703\\
-19.0156395465964	-6.9052591801041\\
-18.9384311303316	-6.8506241067766\\
-18.8619029732705	-6.79664296225898\\
-18.7860431929278	-6.74330387523198\\
-18.7108402154616	-6.69059528310965\\
-18.6362827650768	-6.6385059214416\\
-18.5623598538775	-6.5870248137663\\
-18.4890607721494	-6.5361412618924\\
-18.4163750790475	-6.48584483658679\\
-18.344292593671	-6.43612536864898\\
-18.272803386505	-6.38697294035288\\
-18.201897771212	-6.33837787723779\\
-18.1315662967553	-6.29033074023183\\
-18.0617997398389	-6.24282231809166\\
-17.9925890976487	-6.19584362014342\\
-17.9239255808809	-6.14938586931055\\
-17.8558006070426	-6.1034404954151\\
-17.788205794015	-6.05799912873961\\
-17.7211329538633	-6.01305359383765\\
-17.6545740868847	-5.96859590358139\\
-17.588521375883	-5.92461825343562\\
-17.5229671806583	-5.88111301594775\\
-17.4579040327038	-5.83807273544421\\
-17.3933246300999	-5.79549012292401\\
-17.3292218325956	-5.75335805114071\\
-17.2655886568719	-5.71166954986454\\
-17.2024182719753	-5.6704178013168\\
-17.1397039949181	-5.62959613576911\\
-17.0774392864352	-5.5891980273003\\
-17.0156177468924	-5.54921708970444\\
-16.9542331123389	-5.50964707254327\\
-16.8932792506988	-5.47048185733733\\
-16.832750158095	-5.43171545388966\\
-16.7726399553005	-5.39334199673674\\
-16.7129428843113	-5.35535574172142\\
-16.6536533050365	-5.31775106268276\\
-16.5947656921009	-5.28052244825811\\
-16.5362746317545	-5.24366449879282\\
-16.4781748188864	-5.20717192335334\\
-16.4204610541366	-5.17103953683951\\
-16.3631282411045	-5.13526225719216\\
-16.306171383648	-5.0998351026923\\
-16.2495855832707	-5.06475318934816\\
-16.1933660365942	-5.03001172836694\\
-16.1375080329108	-4.99560602370773\\
-16.0820069518153	-4.96153146971264\\
-16.0268582609115	-4.92778354881314\\
-15.972057513591	-4.89435782930875\\
-15.9176003468815	-4.8612499632153\\
-15.863482479363	-4.82845568418032\\
-15.8096997091474	-4.79597080546284\\
-15.7562479119208	-4.76379121797545\\
-15.703123039046	-4.73191288838616\\
-15.6503211157219	-4.70033185727796\\
-15.5978382391989	-4.66904423736398\\
-15.5456705770483	-4.63804621175631\\
-15.4938143654827	-4.60733403228631\\
-15.4422659077265	-4.57690401787496\\
-15.3910215724345	-4.54675255295117\\
-15.3400777921569	-4.51687608591644\\
-15.289431061849	-4.48727112765435\\
-15.2390779374241	-4.45793425008319\\
-15.189015034348	-4.42886208475036\\
-15.1392390262741	-4.40005132146691\\
-15.089746643717	-4.37149870698114\\
-15.0405346727639	-4.34320104368962\\
-14.9915999538221	-4.31515518838461\\
-14.9429393804021	-4.28735805103653\\
-14.8945498979339	-4.25980659361029\\
-14.8464285026163	-4.23249782891449\\
-14.7985722402985	-4.20542881948222\\
-14.7509782053914	-4.17859667648263\\
-14.7036435398093	-4.15199855866198\\
-14.6565654319397	-4.1256316713136\\
-14.6097411156417	-4.09949326527539\\
-14.56316786927	-4.07358063595435\\
-14.5168430147264	-4.04789112237702\\
-14.4707639165351	-4.02242210626516\\
-14.4249279809434	-3.99717101113576\\
-14.3793326550455	-3.97213530142469\\
-14.333975425929	-3.94731248163324\\
-14.2888538198445	-3.92270009549679\\
-14.2439654013955	-3.89829572517495\\
-14.1993077727496	-3.87409699046258\\
-14.1548785728705	-3.85010154802086\\
-14.1106754767681	-3.82630709062803\\
-14.0666961947694	-3.80271134644905\\
-14.0229384718059	-3.77931207832363\\
-13.9794000867204	-3.7561070830721\\
-13.9360788515902	-3.73309419081866\\
-13.8929726110675	-3.71027126433127\\
-13.8500792417357	-3.68763619837796\\
-13.807396651482	-3.66518691909891\\
-13.7649227788849	-3.64292138339385\\
-13.7226555926169	-3.62083757832435\\
-13.6805930908616	-3.59893352053065\\
-13.6387333007448	-3.57720725566247\\
-13.5970742777789	-3.55565685782346\\
-13.5556141053216	-3.53428042902891\\
-13.5143508940461	-3.51307609867639\\
-13.473282781425	-3.49204202302882\\
-13.4324079312252	-3.47117638470973\\
-13.3917245330162	-3.45047739221032\\
-13.3512308016879	-3.42994327940804\\
-13.3109249769814	-3.40957230509628\\
-13.2708053230294	-3.38936275252495\\
-13.2308701279079	-3.36931292895163\\
-13.1911177031976	-3.34942116520291\\
-13.1515463835559	-3.32968581524581\\
-13.1121545262978	-3.31010525576884\\
-13.0729405109872	-3.29067788577256\\
-13.0339027390368	-3.27140212616928\\
-12.9950396333167	-3.25227641939175\\
-12.9563496377727	-3.23329922901059\\
-12.917831217052	-3.21446903936006\\
-12.8794828561375	-3.19578435517229\\
-12.8413030599909	-3.17724370121933\\
-12.8032903532022	-3.15884562196319\\
-12.7654432796481	-3.14058868121339\\
-12.7277604021571	-3.12247146179204\\
-12.690240302182	-3.10449256520599\\
-12.6528815794796	-3.08665061132619\\
-12.6156828517971	-3.06894423807381\\
-12.5786427545653	-3.05137210111305\\
-12.5417599405979	-3.03393287355055\\
-12.5050330797979	-3.016625245641\\
-12.4684608588698	-2.99944792449909\\
-12.4320419810372	-2.98239963381741\\
-12.3957751657679	-2.96547911359024\\
-12.3596591485026	-2.9486851198431\\
-12.3236926803914	-2.9320164243679\\
-12.2878745280338	-2.91547181446347\\
-12.2522034732254	-2.89905009268152\\
-12.2166783127093	-2.8827500765777\\
-12.1812978579324	-2.8665705984678\\
-12.1460609348067	-2.8505105051888\\
-12.1109663834757	-2.83456865786482\\
-12.0760130580853	-2.81874393167773\\
-12.0411998265592	-2.80303521564239\\
-12.0065255703792	-2.78744141238627\\
-11.9719891843691	-2.77196143793363\\
-11.9375895764836	-2.75659422149374\\
-11.9033256676012	-2.74133870525348\\
-11.8691963913209	-2.72619384417386\\
-11.835200693763	-2.71115860579059\\
-11.8013375333741	-2.6962319700185\\
-11.7676058807354	-2.68141292895981\\
-11.734004718375	-2.66670048671597\\
-11.7005330405836	-2.65209365920332\\
-11.6671898532344	-2.63759147397205\\
-11.6339741736051	-2.62319297002884\\
-11.6008850302048	-2.6088971976627\\
-11.5679214626034	-2.59470321827419\\
-11.5350825212638	-2.58061010420786\\
-11.5023672673787	-2.56661693858784\\
-11.4697747727085	-2.55272281515646\\
-11.4373041194242	-2.53892683811595\\
-11.4049543999518	-2.52522812197303\\
-11.3727247168203	-2.51162579138642\\
-11.3406141825119	-2.49811898101711\\
-11.308621919316	-2.48470683538144\\
-11.2767470591849	-2.47138850870685\\
-11.2449887435922	-2.45816316479021\\
-11.2133461233947	-2.44502997685881\\
-11.1818183586956	-2.43198812743381\\
-11.150404618711	-2.41903680819619\\
-11.1191040816385	-2.40617521985506\\
-11.087915934528	-2.39340257201837\\
-11.0568393731556	-2.38071808306594\\
-11.0258736018984	-2.3681209800247\\
-10.9950178336128	-2.35561049844616\\
-10.9642712895142	-2.34318588228608\\
-10.9336331990592	-2.33084638378619\\
-10.9031027998298	-2.31859126335804\\
-10.8726793374191	-2.30641978946886\\
-10.8423620653202	-2.29433123852939\\
-10.8121502448154	-2.28232489478368\\
-10.782043144869	-2.2704000502008\\
-10.7520400420209	-2.25855600436836\\
-10.7221402202819	-2.24679206438793\\
-10.6923429710316	-2.23510754477223\\
-10.6626475929178	-2.22350176734401\\
-10.6330533917569	-2.21197406113675\\
-10.6035596804367	-2.20052376229693\\
-10.5741657788212	-2.18915021398807\\
-10.5448710136558	-2.17785276629628\\
-10.5156747184749	-2.16663077613744\\
-10.4865762335114	-2.15548360716591\\
-10.4575749056068	-2.14441062968481\\
-10.4286700881231	-2.1334112205577\\
-10.399861140857	-2.1224847631218\\
-10.3711474299539	-2.1116306471026\\
-10.3425283278249	-2.1008482685299\\
-10.3140032130643	-2.09013702965514\\
-10.2855714703684	-2.07949633887024\\
-10.2572324904563	-2.06892561062759\\
-10.2289856699911	-2.0584242653614\\
-10.2008304115033	-2.04799172941042\\
-10.1727661233145	-2.03762743494169\\
-10.1447922194633	-2.02733081987574\\
-10.1169081196311	-2.0171013278128\\
-10.089113249071	-2.00693840796035\\
-10.0614070385357	-1.99684151506165\\
-10.033788924208	-1.98681010932557\\
-10.0062583476319	-1.97684365635738\\
-9.97881475564497	-1.96694162709077\\
-9.95145760031135	-1.9571034977208\\
-9.92418633885638	-1.94732874963798\\
-9.89700043360188	-1.93761686936339\\
-9.86989935190256	-1.92796734848472\\
-9.84288256608338	-1.91837968359341\\
-9.81594955337794	-1.90885337622268\\
-9.78909979586776	-1.89938793278655\\
-9.76233278042251	-1.88998286451978\\
-9.73564799864122	-1.88063768741874\\
-9.70904494679428	-1.87135192218317\\
-9.68252312576642	-1.86212509415884\\
-9.65608204100052	-1.85295673328105\\
-9.62972120244225	-1.84384637401905\\
-9.60344012448563	-1.83479355532118\\
-9.57723832591928	-1.82579782056097\\
-9.5511153298736	-1.81685871748394\\
-9.52507066376871	-1.80797579815524\\
-9.4991038592631	-1.79914861890806\\
-9.47321445220312	-1.79037674029283\\
-9.44740198257323	-1.78165972702712\\
-9.42166599444691	-1.77299714794633\\
-9.39600603593836	-1.76438857595503\\
-9.3704216591549	-1.75583358797915\\
-9.34491242015005	-1.74733176491869\\
-9.3194778788773	-1.73888269160128\\
-9.29411759914459	-1.7304859567363\\
-9.2688311485694	-1.72214115286974\\
-9.24361809853452	-1.71384787633965\\
-9.21847802414447	-1.70560572723228\\
-9.19341050418253	-1.69741430933884\\
-9.16841512106838	-1.68927323011283\\
-9.1434914608164	-1.68118210062806\\
-9.11863911299449	-1.6731405355372\\
-9.09385767068352	-1.66514815303094\\
-9.06914673043738	-1.65720457479776\\
-9.04450589224355	-1.64930942598419\\
-9.01993475948424	-1.6414623351557\\
-8.99543293889812	-1.63366293425812\\
-8.9710000405425	-1.62591085857958\\
-8.94663567775614	-1.618205746713\\
-8.92233946712251	-1.61054724051907\\
-8.89811102843362	-1.60293498508982\\
-8.87394998465425	-1.59536862871257\\
-8.84985596188684	-1.58784782283452\\
-8.82582858933669	-1.58037222202766\\
-8.80186749927775	-1.57294148395438\\
-8.77797232701888	-1.56555526933329\\
-8.75414271087051	-1.55821324190573\\
-8.73037829211179	-1.55091506840262\\
-8.70667871495821	-1.54366041851177\\
-8.68304362652965	-1.53644896484566\\
-8.65947267681879	-1.52928038290961\\
-8.6359655186601	-1.52215435107044\\
-8.61252180769908	-1.51507055052553\\
-8.58914120236205	-1.5080286652722\\
-8.56582336382625	-1.50102838207768\\
-8.5425679559904	-1.4940693904493\\
-8.51937464544562	-1.48715138260518\\
-8.49624310144678	-1.48027405344526\\
-8.47317299588414	-1.47343710052278\\
-8.45016400325549	-1.46664022401603\\
-8.42721580063855	-1.45988312670054\\
-8.4043280676638	-1.45316551392163\\
-8.38150048648761	-1.44648709356731\\
-8.35873274176583	-1.43984757604149\\
-8.33602452062754	-1.43324667423763\\
-8.31337551264938	-1.42668410351264\\
-8.29078540982999	-1.42015958166118\\
-8.2682539065649	-1.41367282889022\\
-8.24578069962177	-1.40722356779402\\
-8.22336548811585	-1.40081152332934\\
-8.20100797348585	-1.39443642279104\\
-8.17870785947002	-1.38809799578794\\
-8.15646485208266	-1.381795974219\\
-8.13427865959085	-1.3755300922498\\
-8.11214899249147	-1.36930008628935\\
-8.09007556348852	-1.36310569496713\\
-8.0680580874708	-1.35694665911048\\
-8.04609628148975	-1.35082272172221\\
-8.0241898647377	-1.34473362795857\\
-8.00233855852624	-1.33867912510742\\
-7.98054208626503	-1.33265896256672\\
-7.95880017344075	-1.32667289182322\\
-7.93711254759635	-1.32072066643154\\
-7.9154789383106	-1.31480204199337\\
-7.89389907717781	-1.308916776137\\
-7.8723726977879	-1.30306462849709\\
-7.85089953570663	-1.29724536069468\\
-7.82947932845612	-1.29145873631746\\
-7.8081118154956	-1.28570452090026\\
-7.7867967382024	-1.27998248190577\\
-7.76553383985316	-1.27429238870555\\
-7.74432286560529	-1.26863401256119\\
-7.72316356247862	-1.26300712660573\\
-7.70205567933731	-1.25741150582538\\
-7.68099896687198	-1.25184692704128\\
-7.65999317758202	-1.24631316889168\\
-7.63903806575815	-1.2408100118142\\
-7.61813338746514	-1.23533723802834\\
-7.59727890052485	-1.2298946315182\\
-7.57647436449929	-1.22448197801542\\
-7.55571954067409	-1.21909906498229\\
-7.53501419204199	-1.21374568159507\\
-7.51435808328663	-1.20842161872753\\
-7.49375098076652	-1.20312666893464\\
-7.47319265249915	-1.1978606264365\\
-7.45268286814535	-1.1926232871024\\
-7.43222139899377	-1.18741444843514\\
-7.41180801794562	-1.18223390955547\\
-7.39144249949952	-1.17708147118669\\
-7.37112461973656	-1.17195693563956\\
-7.35085415630552	-1.1668601067972\\
-7.33063088840827	-1.16179079010032\\
-7.31045459678537	-1.15674879253253\\
-7.29032506370176	-1.15173392260583\\
-7.27024207293269	-1.14674599034634\\
-7.25020540974979	-1.14178480728004\\
-7.23021486090725	-1.13685018641886\\
-7.21027021462828	-1.13194194224678\\
-7.19037126059157	-1.12705989070614\\
-7.17051778991801	-1.12220384918412\\
-7.15070959515757	-1.11737363649938\\
-7.13094647027625	-1.11256907288878\\
-7.11122821064323	-1.10778997999431\\
-7.09155461301816	-1.1030361808502\\
-7.07192547553861	-1.09830749987005\\
-7.05234059770761	-1.09360376283426\\
-7.03279978038137	-1.08892479687743\\
-7.01330282575716	-1.08427043047611\\
-6.99384953736127	-1.07964049343643\\
-6.97443972003712	-1.0750348168821\\
-6.95507317993354	-1.07045323324243\\
-6.93574972449313	-1.06589557624043\\
-6.91646916244079	-1.0613616808812\\
-6.89723130377236	-1.05685138344029\\
-6.87803595974336	-1.05236452145226\\
-6.85888294285792	-1.04790093369938\\
-6.83977206685775	-1.04346046020039\\
-6.8207031467113	-1.03904294219947\\
-6.801675998603	-1.03464822215521\\
-6.78269043992262	-1.03027614372984\\
-6.76374628925478	-1.02592655177846\\
-6.74484336636852	-1.02159929233842\\
-6.72598149220704	-1.01729421261888\\
-6.70716048887749	-1.01301116099036\\
-6.68838017964094	-1.0087499869745\\
-6.66964038890238	-1.0045105412339\\
-6.65094094220092	-1.00029267556204\\
-6.6322816662	-0.996096242873363\\
-6.61366238867776	-0.991921097193395\\
-6.59508293851752	-0.98776709364903\\
-6.57654314569833	-0.983634088458881\\
-6.55804284128565	-0.97952193892375\\
-6.53958185742208	-0.975430503417185\\
-6.52116002731824	-0.971359641376146\\
-6.50277718524377	-0.967309213291767\\
-6.4844331665183	-0.963279080700205\\
-6.46612780750267	-0.959269106173598\\
-6.44786094559014	-0.955279153311108\\
-6.42963241919772	-0.951309086730058\\
-6.41144206775762	-0.947358772057167\\
-6.39328973170874	-0.943428075919858\\
-6.37517525248826	-0.93951686593768\\
-6.35709847252337	-0.9356250107138\\
-6.33905923522301	-0.931752379826587\\
-6.32105738496976	-0.927898843821283\\
-6.30309276711175	-0.924064274201761\\
-6.28516522795473	-0.920248543422363\\
-6.26727461475413	-0.916451524879825\\
-6.24942077570731	-0.912673092905275\\
-6.23160355994579	-0.908913122756327\\
-6.2138228175276	-0.905171490609241\\
-6.19607839942973	-0.901448073551171\\
-6.17837015754063	-0.897742749572483\\
-6.16069794465279	-0.894055397559162\\
-6.1430616144554	-0.890385897285274\\
-6.12546102152706	-0.886734129405536\\
-6.10789602132863	-0.883099975447925\\
-6.09036647019605	-0.879483317806383\\
-6.07287222533336	-0.875884039733592\\
-6.05541314480565	-0.87230202533381\\
-6.0379890875322	-0.868737159555788\\
-6.02059991327962	-0.865189328185756\\
-6.00324548265509	-0.861658417840473\\
-5.98592565709961	-0.858144315960349\\
-5.96864029888145	-0.854646910802633\\
-5.95138927108949	-0.851166091434674\\
-5.93417243762677	-0.847701747727235\\
-5.91698966320402	-0.844253770347884\\
-5.89984081333328	-0.840822050754448\\
-5.88272575432161	-0.837406481188523\\
-5.86564435326483	-0.834006954669051\\
-5.84859647804127	-0.830623364985968\\
-5.83158199730575	-0.827255606693898\\
-5.81460078048338	-0.823903575105921\\
-5.79765269776367	-0.820567166287389\\
-5.78073762009448	-0.817246277049815\\
-5.76385541917618	-0.813940804944808\\
-5.74700596745577	-0.810650648258077\\
-5.73018913812115	-0.807375706003481\\
-5.71340480509534	-0.804115877917147\\
-5.69665284303084	-0.800871064451634\\
-5.67993312730402	-0.797641166770163\\
-5.66324553400951	-0.794426086740892\\
-5.64658993995476	-0.791225726931248\\
-5.62996622265452	-0.78803999060232\\
-5.61337426032547	-0.784868781703289\\
-5.59681393188086	-0.78171200486593\\
-5.58028511692522	-0.77856956539915\\
-5.56378769574907	-0.775441369283585\\
-5.54732154932375	-0.772327323166242\\
-5.53088655929629	-0.769227334355204\\
-5.51448260798422	-0.76614131081436\\
-5.49810957837062	-0.763069161158219\\
-5.48176735409904	-0.760010794646732\\
-5.46545581946856	-0.756966121180199\\
-5.44917485942887	-0.753935051294193\\
-5.43292435957543	-0.750917496154559\\
-5.4167042061446	-0.747913367552428\\
-5.40051428600889	-0.74492257789931\\
-5.38435448667222	-0.741945040222205\\
-5.36822469626523	-0.738980668158773\\
-5.35212480354063	-0.736029375952549\\
-5.33605469786861	-0.733091078448185\\
-5.32001426923226	-0.730165691086766\\
-5.30400340822306	-0.727253129901138\\
-5.2880220060364	-0.724353311511294\\
-5.27206995446715	-0.721466153119801\\
-5.25614714590525	-0.718591572507268\\
-5.24025347333139	-0.71572948802785\\
-5.22438883031262	-0.712879818604798\\
-5.20855311099816	-0.710042483726049\\
-5.19274621011512	-0.70721740343985\\
-5.1769680229643	-0.704404498350436\\
-5.16121844541602	-0.701603689613721\\
-5.14549737390603	-0.698814898933056\\
-5.1298047054314	-0.696038048555003\\
-5.11414033754648	-0.693273061265163\\
-5.09850416835885	-0.690519860384029\\
-5.08289609652542	-0.687778369762884\\
-5.06731602124842	-0.685048513779726\\
-5.05176384227154	-0.682330217335245\\
-5.03623945987599	-0.679623405848817\\
-5.02074277487677	-0.676928005254548\\
-5.00527368861878	-0.674243941997343\\
-4.98983210297308	-0.67157114302902\\
-4.97441792033315	-0.668909535804446\\
-4.95903104361123	-0.666259048277715\\
-4.94367137623457	-0.663619608898358\\
-4.92833882214187	-0.660991146607587\\
-4.91303328577962	-0.658373590834566\\
-4.89775467209858	-0.655766871492728\\
-4.88250288655017	-0.653170918976101\\
-4.86727783508304	-0.650585664155698\\
-4.85207942413951	-0.648011038375901\\
-4.8369075606522	-0.645446973450915\\
-4.82176215204053	-0.642893401661215\\
-4.80664310620739	-0.640350255750058\\
-4.79155033153576	-0.637817468919999\\
-4.77648373688537	-0.635294974829453\\
-4.76144323158942	-0.632782707589277\\
-4.74642872545128	-0.630280601759386\\
-4.73144012874125	-0.6277885923454\\
-4.71647735219339	-0.625306614795311\\
-4.70154030700223	-0.622834604996193\\
-4.68662890481972	-0.620372499270926\\
-4.67174305775201	-0.617920234374951\\
-4.65688267835639	-0.615477747493059\\
-4.64204767963819	-0.613044976236203\\
-4.62723797504771	-0.610621858638336\\
-4.61245347847723	-0.608208333153278\\
-4.59769410425797	-0.605804338651606\\
-4.58295976715712	-0.603409814417579\\
-4.56825038237489	-0.601024700146074\\
-4.55356586554161	-0.598648935939564\\
-4.53890613271475	-0.596282462305109\\
-4.52427110037613	-0.593925220151383\\
-4.50966068542901	-0.591577150785712\\
-4.49507480519527	-0.589238195911149\\
-4.48051337741262	-0.586908297623573\\
-4.46597632023178	-0.584587398408801\\
-4.45146355221377	-0.582275441139739\\
-4.43697499232713	-0.579972369073542\\
-4.42251055994521	-0.577678125848811\\
-4.40807017484351	-0.575392655482804\\
-4.39365375719697	-0.573115902368674\\
-4.37926122757736	-0.570847811272728\\
-4.36489250695062	-0.568588327331711\\
-4.35054751667428	-0.566337396050111\\
-4.33622617849485	-0.564094963297484\\
-4.3219284145453	-0.561860975305807\\
-4.30765414734249	-0.559635378666845\\
-4.29340329978466	-0.557418120329547\\
-4.27917579514892	-0.555209147597459\\
-4.26497155708878	-0.553008408126156\\
-4.2507905096317	-0.550815849920704\\
-4.23663257717665	-0.54863142133313\\
-4.22249768449166	-0.546455071059926\\
-4.20838575671149	-0.544286748139559\\
-4.19429671933517	-0.542126401950016\\
-4.18023049822368	-0.539973982206356\\
-4.16618701959764	-0.537829438958295\\
-4.15216621003492	-0.535692722587793\\
-4.1381679964684	-0.533563783806679\\
-4.12419230618363	-0.531442573654285\\
-4.11023906681661	-0.529329043495096\\
-4.09630820635152	-0.527223145016428\\
-4.0823996531185	-0.525124830226122\\
-4.06851333579141	-0.523034051450249\\
-4.05464918338567	-0.520950761330846\\
-4.04080712525608	-0.518874912823656\\
-4.02698709109462	-0.516806459195902\\
-4.01318901092837	-0.514745354024062\\
-3.99941281511731	-0.512691551191674\\
-3.9856584343523	-0.510645004887156\\
-3.9719257996529	-0.508605669601638\\
-3.95821484236535	-0.506573500126814\\
-3.94452549416049	-0.504548451552819\\
-3.93085768703172	-0.502530479266108\\
-3.91721135329299	-0.500519538947364\\
-3.90358642557675	-0.498515586569417\\
-3.889982836832	-0.496518578395179\\
-3.87640052032226	-0.494528470975597\\
-3.86283940962365	-0.492545221147626\\
-3.84929943862293	-0.490568786032207\\
-3.83578054151556	-0.488599123032277\\
-3.82228265280376	-0.486636189830773\\
-3.80880570729464	-0.484679944388671\\
-3.79534964009832	-0.482730344943039\\
-3.78191438662599	-0.480787350005085\\
-3.76849988258813	-0.478850918358247\\
-3.75510606399261	-0.47692100905628\\
-3.74173286714289	-0.474997581421364\\
-3.72838022863616	-0.473080595042228\\
-3.7150480853616	-0.471170009772283\\
-3.70173637449852	-0.469265785727781\\
-3.68844503351466	-0.467367883285969\\
-3.67517400016434	-0.465476263083275\\
-3.6619232124868	-0.463590886013506\\
-3.64869260880438	-0.461711713226045\\
-3.63548212772089	-0.459838706124083\\
-3.6222917081198	-0.457971826362847\\
-3.60912128916263	-0.456111035847853\\
-3.5959708102872	-0.454256296733164\\
-3.582840211206	-0.452407571419672\\
-3.56972943190454	-0.450564822553378\\
-3.55663841263965	-0.448728013023701\\
-3.54356709393791	-0.446897105961791\\
-3.53051541659398	-0.445072064738856\\
-3.51748332166902	-0.4432528529645\\
-3.50447075048909	-0.441439434485082\\
-3.49147764464354	-0.439631773382075\\
-3.47850394598347	-0.437829833970448\\
-3.46554959662016	-0.436033580797058\\
-3.4526145389235	-0.434242978639048\\
-3.43969871552046	-0.432457992502261\\
-3.4268020692936	-0.430678587619678\\
-3.4139245433795	-0.428904729449839\\
-3.40106608116728	-0.427136383675308\\
-3.38822662629711	-0.425373516201127\\
-3.37540612265873	-0.423616093153293\\
-3.36260451438997	-0.421864080877241\\
-3.34982174587527	-0.420117445936345\\
-3.3370577617443	-0.41837615511042\\
-3.32431250687042	-0.416640175394242\\
-3.31158592636935	-0.414909473996086\\
-3.29887796559768	-0.413184018336254\\
-3.28618857015149	-0.411463776045637\\
-3.27351768586497	-0.409748714964277\\
-3.26086525880899	-0.408038803139933\\
-3.24823123528977	-0.406334008826676\\
-3.23561556184748	-0.404634300483475\\
-3.22301818525489	-0.402939646772806\\
-3.21043905251603	-0.40125001655927\\
-3.19787811086484	-0.399565378908213\\
-3.18533530776386	-0.397885703084363\\
-3.1728105909029	-0.396210958550486\\
-3.16030390819772	-0.394541114966029\\
-3.14781520778876	-0.392876142185791\\
-3.13534443803981	-0.391216010258603\\
-3.12289154753678	-0.389560689426007\\
-3.11045648508637	-0.387910150120959\\
-3.09803919971486	-0.386264362966526\\
-3.08563964066683	-0.384623298774606\\
-3.0732577574039	-0.382986928544653\\
-3.06089349960352	-0.381355223462402\\
-3.04854681715776	-0.379728154898626\\
-3.03621766017203	-0.378105694407874\\
-3.02390597896393	-0.376487813727237\\
-3.01161172406201	-0.374874484775124\\
-2.99933484620462	-0.373265679650033\\
-2.98707529633867	-0.371661370629342\\
-2.97483302561849	-0.370061530168104\\
-2.96260798540467	-0.368466130897859\\
-2.95040012726287	-0.366875145625443\\
-2.93820940296269	-0.365288547331809\\
-2.92603576447651	-0.363706309170866\\
-2.91387916397839	-0.36212840446831\\
-2.90173955384289	-0.360554806720482\\
-2.889616886644	-0.358985489593214\\
-2.87751111515399	-0.357420426920705\\
-2.86542219234235	-0.355859592704387\\
-2.85335007137463	-0.354302961111808\\
-2.84129470561142	-0.352750506475525\\
-2.82925604860722	-0.351202203291994\\
-2.81723405410938	-0.349658026220483\\
-2.80522867605706	-0.348117950081984\\
-2.79323986858013	-0.346581949858131\\
-2.78126758599813	-0.345050000690129\\
-2.76931178281924	-0.343522077877694\\
-2.75737241373926	-0.341998156877996\\
-2.74544943364051	-0.34047821330461\\
-2.73354279759088	-0.338962222926474\\
-2.72165246084279	-0.337450161666857\\
-2.70977837883216	-0.335942005602338\\
-2.69792050717744	-0.334437730961781\\
-2.68607880167859	-0.332937314125323\\
-2.6742532183161	-0.331440731623378\\
-2.66244371325002	-0.32994796013563\\
-2.65065024281897	-0.328458976490053\\
-2.63887276353917	-0.326973757661921\\
-2.62711123210349	-0.325492280772836\\
-2.61536560538048	-0.324014523089757\\
-2.60363584041344	-0.322540462024042\\
-2.59192189441946	-0.321070075130487\\
-2.58022372478849	-0.319603340106384\\
-2.56854128908243	-0.318140234790575\\
-2.55687454503414	-0.316680737162518\\
-2.54522345054662	-0.31522482534136\\
-2.53358796369202	-0.313772477585015\\
-2.52196804271077	-0.312323672289247\\
-2.51036364601067	-0.310878387986763\\
-2.498774732166	-0.309436603346311\\
-2.48720125991663	-0.307998297171782\\
-2.47564318816716	-0.306563448401321\\
-2.46410047598599	-0.305132036106443\\
-2.45257308260452	-0.30370403949116\\
-2.44106096741623	-0.302279437891106\\
-2.42956408997587	-0.300858210772669\\
-2.41808240999854	-0.299440337732139\\
-2.40661588735893	-0.298025798494851\\
-2.39516448209039	-0.296614572914338\\
-2.38372815438417	-0.29520664097149\\
-2.37230686458854	-0.293801982773723\\
-2.36090057320799	-0.292400578554144\\
-2.34950924090239	-0.291002408670732\\
-2.3381328284862	-0.289607453605515\\
-2.32677129692765	-0.288215693963764\\
-2.31542460734792	-0.286827110473182\\
-2.30409272102038	-0.285441683983106\\
-2.29277559936976	-0.284059395463711\\
-2.28147320397138	-0.282680226005217\\
-2.27018549655036	-0.281304156817114\\
-2.25891243898086	-0.279931169227369\\
-2.24765399328528	-0.278561244681666\\
-2.2364101216335	-0.277194364742633\\
-2.22518078634215	-0.275830511089077\\
-2.21396594987379	-0.274469665515229\\
-2.20276557483623	-0.273111809929992\\
-2.19157962398171	-0.271756926356191\\
-2.18040806020622	-0.270404996929836\\
-2.16925084654871	-0.269056003899384\\
-2.15810794619039	-0.267709929625002\\
-2.14697932245399	-0.266366756577848\\
-2.13586493880304	-0.265026467339345\\
-2.12476475884113	-0.263689044600467\\
-2.11367874631123	-0.262354471161022\\
-2.10260686509495	-0.261022729928954\\
-2.09154907921184	-0.259693803919632\\
-2.08050535281871	-0.258367676255158\\
-2.06947565020889	-0.257044330163676\\
-2.05845993581159	-0.255723748978678\\
-2.04745817419117	-0.254405916138327\\
-2.03647033004647	-0.253090815184777\\
-2.02549636821013	-0.2517784297635\\
-2.01453625364792	-0.250468743622618\\
-2.00358995145807	-0.249161740612229\\
-1.99265742687059	-0.247857404683766\\
-1.98173864524662	-0.246555719889323\\
-1.97083357207775	-0.245256670381014\\
-1.95994217298541	-0.243960240410327\\
-1.94906441372017	-0.242666414327475\\
-1.93820026016113	-0.241375176580765\\
-1.92734967831525	-0.240086511715963\\
-1.91651263431673	-0.238800404375663\\
-1.90568909442638	-0.237516839298667\\
-1.89487902503097	-0.236235801319353\\
-1.88408239264263	-0.234957275367075\\
-1.87329916389819	-0.233681246465542\\
-1.86252930555859	-0.232407699732203\\
-1.85177278450828	-0.231136620377661\\
-1.84102956775455	-0.229867993705056\\
-1.830299622427	-0.228601805109478\\
-1.81958291577688	-0.227338040077376\\
-1.80887941517649	-0.226076684185964\\
-1.79818908811864	-0.224817723102644\\
-1.78751190221597	-0.223561142584423\\
-1.77684782520047	-0.222306928477339\\
-1.76619682492278	-0.221055066715883\\
-1.75555886935169	-0.219805543322441\\
-1.74493392657354	-0.218558344406721\\
-1.73432196479163	-0.217313456165198\\
-1.72372295232567	-0.216070864880551\\
-1.71313685761119	-0.214830556921118\\
-1.70256364919899	-0.21359251874034\\
-1.6920032957546	-0.212356736876222\\
-1.68145576605768	-0.211123197950782\\
-1.6709210290015	-0.209891888669523\\
-1.66039905359236	-0.208662795820889\\
-1.64988980894907	-0.207435906275742\\
-1.6393932643024	-0.206211206986828\\
-1.62890938899453	-0.204988684988255\\
-1.61843815247852	-0.203768327394976\\
-1.60797952431778	-0.202550121402268\\
-1.59753347418552	-0.20133405428522\\
-1.58709997186425	-0.20012011339822\\
-1.57667898724523	-0.198908286174454\\
-1.56627049032796	-0.197698560125397\\
-1.55587445121967	-0.196490922840319\\
-1.5454908401348	-0.195285361985779\\
-1.53511962739447	-0.194081865305144\\
-1.52476078342599	-0.192880420618086\\
-1.51441427876237	-0.191681015820107\\
-1.50408008404176	-0.190483638882048\\
-1.49375817000701	-0.18928827784961\\
-1.48344850750515	-0.188094920842878\\
-1.4731510674869	-0.186903556055846\\
-1.46286582100615	-0.185714171755949\\
-1.45259273921953	-0.18452675628359\\
-1.44233179338586	-0.183341298051682\\
-1.43208295486572	-0.182157785545181\\
-1.42184619512095	-0.180976207320629\\
-1.41162148571415	-0.179796552005699\\
-1.40140879830824	-0.178618808298746\\
-1.391208104666	-0.177442964968348\\
-1.38101937664954	-0.176269010852874\\
-1.3708425862199	-0.175096934860022\\
-1.36067770543655	-0.1739267259664\\
-1.35052470645694	-0.17275837321707\\
-1.34038356153604	-0.171591865725122\\
-1.33025424302589	-0.17042719267124\\
-1.32013672337515	-0.169264343303275\\
-1.31003097512865	-0.168103306935817\\
-1.2999369709269	-0.16694407294977\\
-1.28985468350574	-0.165786630791934\\
-1.27978408569581	-0.16463096997459\\
-1.26972515042213	-0.163477080075074\\
-1.25967785070372	-0.162324950735377\\
-1.24964215965307	-0.161174571661733\\
-1.23961805047579	-0.160025932624202\\
-1.22960549647016	-0.158879023456277\\
-1.21960447102667	-0.15773383405448\\
-1.20961494762763	-0.156590354377958\\
-1.19963689984674	-0.155448574448093\\
-1.18967030134866	-0.154308484348102\\
-1.17971512588861	-0.153170074222652\\
-1.16977134731194	-0.152033334277469\\
-1.15983893955373	-0.150898254778946\\
-1.14991787663839	-0.149764826053773\\
-1.14000813267919	-0.14863303848854\\
-1.13010968187795	-0.147502882529372\\
-1.12022249852456	-0.146374348681543\\
-1.11034655699663	-0.145247427509109\\
-1.10048183175904	-0.144122109634534\\
-1.09062829736361	-0.14299838573832\\
-1.08078592844863	-0.141876246558644\\
-1.07095469973854	-0.140755682890991\\
-1.06113458604349	-0.139636685587797\\
-1.05132556225898	-0.138519245558083\\
-1.04152760336547	-0.137403353767101\\
-1.03174068442798	-0.136289001235984\\
-1.02196478059573	-0.135176179041384\\
-1.01219986710174	-0.134064878315135\\
-1.0024459192625	-0.132955090243893\\
-0.992702912477539	-0.131846806068798\\
-0.982970822229071	-0.130740017085126\\
-0.973249624081646	-0.129634714641955\\
-0.96353929368176	-0.128530890141817\\
-0.953839806757496	-0.127428535040373\\
-0.944151139118158	-0.126327640846064\\
-0.934473266653912	-0.125228199119796\\
-0.924806165335424	-0.124130201474591\\
-0.915149811213502	-0.123033639575276\\
-0.90550418041874	-0.121938505138151\\
-0.895869249161165	-0.120844789930659\\
-0.886244993729884	-0.119752485771076\\
-0.876631390492734	-0.118661584528182\\
-0.867028415895933	-0.117572078120946\\
-0.857436046463738	-0.11648395851822\\
-0.847854258798095	-0.115397217738409\\
-0.838283029578297	-0.114311847849173\\
-0.828722335560651	-0.11322784096711\\
-0.819172153578128	-0.112145189257458\\
-0.809632460540035	-0.111063884933776\\
-0.800103233431676	-0.109983920257655\\
-0.790584449314021	-0.108905287538408\\
-0.781076085323371	-0.10782797913277\\
-0.771578118671034	-0.106751987444605\\
-0.762090526642992	-0.105677304924611\\
-0.752613286599578	-0.104603924070022\\
-0.743146375975151	-0.103531837424319\\
-0.733689772277774	-0.102461037576938\\
-0.724243453088894	-0.101391517162987\\
-0.714807396063021	-0.100323268862952\\
-0.705381578927413	-0.0992562854024197\\
-0.695965979481758	-0.0981905595517926\\
-0.686560575597866	-0.0971260841260069\\
-0.677165345219347	-0.0960628519842575\\
-0.667780266361313	-0.0950008560297155\\
-0.658405317110058	-0.0939400892092572\\
-0.649040475622759	-0.0928805445131876\\
-0.639685720127164	-0.0918222149749726\\
-0.630341028921298	-0.0907650936709663\\
-0.621006380373147	-0.0897091737201421\\
-0.611681752920373	-0.0886544482838283\\
-0.602367125070001	-0.0876009105654413\\
-0.593062475398133	-0.0865485538102224\\
-0.583767782549644	-0.0854973713049796\\
-0.574483025237896	-0.0844473563778246\\
-0.565208182244434	-0.0833985023979161\\
-0.555943232418711	-0.0823508027752007\\
-0.546688154677781	-0.0813042509601625\\
-0.537442928006027	-0.0802588404435675\\
-0.528207531454862	-0.0792145647562099\\
-0.518981944142453	-0.0781714174686687\\
-0.509766145253433	-0.07712939219105\\
-0.500560114038621	-0.0760884825727497\\
-0.491363829814742	-0.0750486823022014\\
-0.482177271964145	-0.0740099851066367\\
-0.473000419934532	-0.0729723847518424\\
-0.463833253238675	-0.0719358750419157\\
-0.454675751454147	-0.0709004498190342\\
-0.445527894223045	-0.0698661029632083\\
-0.436389661251722	-0.0688328283920532\\
-0.427261032310513	-0.0678006200605498\\
-0.418141987233472	-0.0667694719608165\\
-0.409032505918098	-0.0657393781218681\\
-0.399932568325074	-0.0647103326093986\\
-0.390842154477999	-0.0636823295255439\\
-0.381761244463129	-0.0626553630086573\\
-0.372689818429112	-0.0616294272330841\\
-0.363627856586728	-0.0606045164089379\\
-0.354575339208632	-0.059580624781875\\
-0.345532246629093	-0.058557746632877\\
-0.336498559243741	-0.0575358762780292\\
-0.32747425750931	-0.0565150080683005\\
-0.318459321943385	-0.0554951363893289\\
-0.309453733124149	-0.0544762556612051\\
-0.300457471690133	-0.0534583603382553\\
-0.291470518339967	-0.0524414449088339\\
-0.282492853832127	-0.0514255038951063\\
-0.273524458984694	-0.0504105318528409\\
-0.264565314675103	-0.0493965233711986\\
-0.255615401839903	-0.0483834730725317\\
-0.246674701474509	-0.047371375612165\\
-0.237743194632962	-0.0463602256782036\\
-0.22882086242769	-0.0453500179913218\\
-0.219907686029264	-0.044340747304563\\
-0.211003646666164	-0.0433324084031375\\
-0.202108725624539	-0.0423249961042241\\
-0.193222904247971	-0.04131850525677\\
-0.184346163937244	-0.0403129307412948\\
-0.175478486150103	-0.0393082674696931\\
-0.16661985240103	-0.0383045103850392\\
-0.157770244261007	-0.0373016544613985\\
-0.148929643357288	-0.0362996947036256\\
-0.14009803137317	-0.0352986261471835\\
-0.131275390047765	-0.0342984438579448\\
-0.122461701175776	-0.0332991429320097\\
-0.113656946607265	-0.0323007184955163\\
-0.104861108247438	-0.0313031657044518\\
-0.096074168056412	-0.0303064797444674\\
-0.0872961080490018	-0.0293106558306999\\
-0.0785269102944935	-0.0283156892075822\\
-0.0697665569164269	-0.0273215751486638\\
-0.0610150300923765	-0.0263283089564331\\
-0.0522723120537338	-0.0253358859621337\\
-0.0435383850854909	-0.0243443015255875\\
-0.0348132315260254	-0.0233535510350209\\
-0.0260968337668856	-0.022363629906882\\
-0.0173891742525778	-0.0213745335856722\\
-0.00869023548035383	-0.0203862575437684\\
0	-0.0193987972812512\\
};
\end{axis}
\end{tikzpicture}%
The {\bfseries Matlab} code for performing above equation is shown next.
\begin{lstlisting}
function y=expdist(audio, gain, mix)
% Distortion based on an exponential function
% audio: input
% gainn: amount of distortion, >0
% mix: mix of original and distorted sound, 1=only distorted

q = audio * gain;
z = sign(q) .* (1 - exp(-abs(q)));
y = mix * z + (1 - mix) * x;

\end{lstlisting}


\end{document}
