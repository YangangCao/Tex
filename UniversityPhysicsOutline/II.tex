\documentclass[10pt,a4paper,oneside]{article}
\usepackage[utf8]{inputenc}
\usepackage{amsmath}
\usepackage{amsfonts}
\usepackage{amssymb}
\usepackage{graphicx}
\usepackage{geometry}
\usepackage{breqn}
\date{}
%\usepackage{draftwatermark} % 设置水印
%\SetWatermarkText{DNV Group} % 水印内容
\usepackage{tikz} % system block diagram
\usepackage{textcomp}
\usetikzlibrary{datavisualization}
\usetikzlibrary{shapes,arrows} % system block diagram
\usepackage{booktabs}
\geometry{right=3.0cm,left=3.0cm,bottom=3.0cm, top=3.0cm}
\begin{document}

\title{\vspace{-3em}Syllabus of University Physics II\vspace{-2em}}
\maketitle 
\section{Course information}
Course name: University Physics II\\
Course object: Undergraduates majoring in science and engineering\\
Course code: SCI4B3B002\\
Course category: Professional basic courses, compulsory\\
Prerequisite course: University Physics I, Advanced mathematics.\\
Total  period: 51+17 (class :51, continue :17)\\
Syllabus author: Xiaobai Chen\\
Syllabus reviewer: Denghui Xu, Aicong Geng\\


Physics is the natural science about basic structure, basic motion form and interaction. Its basic theory permeates every field of natural science and is applied in many domains of production technology, which is the foundation of other natural science and engineering technology.

In the process of pursuit of truth and exploration of the unknown world, physics has presented a series of scientific world views and methodology, which profoundly affects human's basic understanding of the material world, thinking mode and social life. It is the cornerstone of human civilization development and plays an important role in the cultivation of scientific quality of talents.

University physics II consists of the basic concepts and rules of mechanical, thermal, mechanical vibration, mechanical wave and wave optics, and it's an important general professional basic course for students in major machinery, automation, electrical, computer, software, information, mathematics, chemistry, food, environment, biology, materials science and engineering in our school. The basic concepts, theories and methods taught in this course are important parts of students' scientific literacy, which are necessary for scientists and engineers.

University physics II plays an important role that can not be replaced by other courses in setting up the necessary physical foundation for students systematically, cultivating students' scientific world view, enhancing students' ability to analyze and solve problems, and cultivating students' exploration spirit and innovation consciousness.

Through this course, students should have a systematic and correct understanding of the basic concepts, theories and research methods of electricity, magnetism, special relativity and quantum physics in physics, so as to lay a solid foundation for further study. In the course of each section, not only knowledge should be taught, at the same time, but also pay attention to the cultivation of students' ability to analyze and solve problems, the cultivation of students' exploration spirit and innovative consciousness, and strive to achieve the coordinated development of students' knowledge, ability and quality.
\section{Textbooks and teaching resources}
\subsection{Textbooks}
General physics (6th edition), volume 1 and volume 2, edited by Shouzhu Cheng and Zhiyong Jiang, Higher Education Press, 2006.
\subsection{Teaching resources}
\begin{itemize}
	\item General physics (6th edition), problem analysis and solution,  Naijiang Sui, Panxin Hu, Higher Education Press, 2006.
	\item General physics (6th edition), analysis and development of thought problems, Panxin Hu, et al, Higher Education Press, 2006.
	\item Case base of university physics peer teaching(1st edition), Xiaobai Chen, Baohe Li, et al, China Machine Press, June 2014.
	\item University physics problem seminar guide(1st edition), Huijun Shen, Huzhu Wang, Tsinghua University Press,  June 1991.
\end{itemize}
\section{Course assessment}
The grades are assessed by 30\% and 70\% respectively according to the normal time and the exam, including attendance, discussion, unit test, etc., and the final exam is a closed-book exam with 100 minutes.
\section{Teaching content and teaching hours allocation }
\subsection{Total teaching hours}
The total teaching hours of this course are 68, including 51 credit hours and 17 continuing hours.
\subsection{Content and teaching hours allocation}
{\bfseries Chapter 1 Electromagnetism (32 credit hours + 9 continuing hours)}\\

\noindent{\bfseries1 Electrostatic field in a vacuum}\\

Teaching content: Charge, the law of conservation of charge, Coulomb's law, electrostatic field, electric field intensity, power lines, electric field intensity superposition principle, the calculation of electric field intensity, electric dipole, electric flux, Gauss theorem in vacuum, work of static electric field force, electrostatic field loop theorem, electric potential energy, electric potential, electric potential superposition principle, electric potential difference, equipotential surface.\\

Teaching requirements:
\begin{itemize}
\item Mastering Coulomb's law, the concept of electric field intensity, the superposition principle of electric field intensity. Mastering the concept and superposition principle of electric potential. Using integration to calculate electric field intensity and electric potential according to superposition principle in some typical and simple problems.
\item Understanding the properties of electrostatic field, Gauss theorem and loop theorem of electrostatic field, understand their status in electromagnetism, knowing the conditions and methods of calculating electric field intensity with Gauss theorem.
\item Knowing the concept of electric dipole moment, the force and motion of electric dipole in uniform electric field can be analyzed.
\end{itemize}
\noindent{\bfseries2 Electrostatic fields in conductors and dielectrics}\\

Teaching content: Conductor electrostatic balance, electrostatic shielding, dielectric polarization, electric displacement, the relationship between electric displacement vector $D$ and electric field intensity $E$ in isotropic dielectric, Gauss theorem in dielectric, capacitance, electric field energy and electric field energy density.\\

Teaching requirements:
\begin{itemize}
	\item The charge distribution, electric field intensity distribution and electric potential distribution of a conductor or group of conductors of simple shape at electrostatic equilibrium can be analyzed and calculated, understanding electrostatic equilibrium conditions and charge distribution characteristics of conductors, knowing electrostatic shielding.
	\item Knowing the relationship and difference between electric displacement vector $D$ and electric field intensity $E$ in isotropic dielectric. Gauss theorem can be used to calculate the electric displacement and field intensity in isotropic media with simple symmetric distribution.
	\item Understanding the concept of capacitance and the characteristics of series and parallel capacitors.
	\item Knowing the material nature of the electric field. Knowing the electric field energy of the simple case.
\end{itemize}
\noindent{\bfseries3 Magnetic field with constant current}\\

Teaching content: Magnetic induction intensity, magnetic field line, superposition principle of magnetic induction intensity, magnetic flux. Bisard's law, magnetic Gauss theorem, Ampere loop theorem. Lorentz force, the effect of a magnetic field on a current-carrying wire, and the effect of a magnetic field on a planar current-carrying coil. Magnetic moment.\\

Teaching requirements:
\begin{itemize}
	\item Mastering the concept of magnetic induction intensity, understanding Bisard's law, the magnetic induction intensity of some typical and simple current distribution can be calculated with the principle of magnetic field superposition.
	\item Understanding Gauss theorem and Ampere loop theorem of magnetic field. Understanding the method and condition of calculating magnetic induction intensity by Ampere loop theorem.
	\item Understanding Ampere's law and Lorentz force, calculating the magnetic field force and moment of the current-carrying straight wire and simple planar current-carrying coil in the magnetic field, analyzing and calculating the force and motion of the moving point charge in the uniform magnetic field.
\end{itemize}
\noindent{\bfseries4 Magnetic field in a magnetic medium}\\

Teaching content: Classification of magnetic media, magnetization of magnetic media, characteristics of magnetic media. Magnetic field intensity, relationship between magnetic field intensity $H$ and magnetic induction intensity $B$ in isotropic magnetic medium. Ampere loop theorem in medium.\\

Teaching requirements:
\begin{itemize}
	\item Knowing the microscopic mechanism of magnetization of magnetic media and the characteristics of ferromagnetic materials.
	\item Knowing the definition of magnetic field intensity in homogeneous isotropic magnetic medium, knowing the Ampere loop theorem in magnetic medium, and using the Ampere loop theorem in magnetic medium to calculate simple problems.
\end{itemize}
\noindent{\bfseries5 Electromagnetic induction and electromagnetic field}\\

Teaching content: Faraday's law of electromagnetic induction, Lenz's law, electromagnetic induction phenomena, dynamic electromotive force and induced electromotive force, eddy electric field, self-inductance and mutual inductance, magnetic field energy and energy density, displacement current, integral form of Maxwell's equations, generation and properties of electromagnetic waves.\\

Teaching requirements:
\begin{itemize}
	\item Understanding the concept of electrodynamic potential.
	\item Mastering Faraday's law of electromagnetic induction and Lenz's law, knowing the relationship between electromagnetic induction and energy conservation, understanding the dynamic electromotive force, knowing the induced electromotive force, calculating the dynamic and induced electromotive force in simple situation.
	\item  Knowing self-inductance, mutual inductance phenomenon and simple calculation. Knowing the energy of magnetic field.
	\item Knowing the concepts of eddy electric field and displacement current, knowing the integral form of Maxwell's equations and its physical significance, knowing the basic properties of electromagnetic waves and the physical properties of electromagnetic fields.
\end{itemize}
{\bfseries Chapter 2 Basis of modern physics (19 credit hours + 5 continuing hours)}\\

\noindent{\bfseries6 Special relativity}\\

Teaching content: The two basic principles of special relativity, Lorentz transformation, special relativity in space and time, relativity in time, shortening of length, expansion of time, relativistic relations of mass and velocity, and relations of mass and energy.\\

Teaching requirements:
\begin{itemize}
	\item Knowing two basic postulates of Einstein's special theory of relativity and the Lorentz transformation. Understanding the space-time view of special relativity and its difference from classical mechanics.
	\item Understanding the relationship between mass and velocity, the relationship between mass and energy in special relativity.
	\item Analyzing and calculating the relatively simple questions about length shortening and time dilation.  Using mass-velocity and mass-energy formulas to calculate simple problems.
\end{itemize}
\noindent{\bfseries7 Fundamentals of quantum physics}\\

Teaching content: Thermal radiation, absolute blackbody, Schevan-Boltzmann's law, Wien's law of displacement, Planck hypothesis. Experimental law of photoelectric effect, photon hypothesis, Einstein equation of photoelectric effect, wave particle duality of light. Experimental laws of hydrogen spectra, Bohr's hydrogen theory and its significance and limitations. DE Broglie material wave hypothesis, wave-particle duality of microscopic particles, wave function and its statistical interpretation, uncertain relation. Schrodinger equation, one dimensional infinite potential well, quantum mechanics theory of hydrogen atoms, electron spin, Pauli exclusion principle, energy minimum principle, atomic shell structure.\\

Teaching requirements:
\begin{itemize}
	\item Knowing  the laws of thermal radiation.
	\item Understanding the experimental law of photoelectric effect, using Einstein equation to do simple calculation. Understanding the wave-particle duality of light.
	\item Understanding the status and significance of Planck hypothesis, Einstein photon hypothesis and Bohr quantization hypothesis in the development of modern physics
	\item Knowing DE Broglie's material wave hypothesis and electron diffraction experiments, knowing the wave-particle duality of real particles.
	\item Understanding the relationship between the physical quantities describing the volatility of particles and their properties.
	\item Knowing the uncertainty relationship between one-dimensional coordinates and momentum,  using the uncertainty relationship for simple estimation calculation.
	\item Understanding the experimental laws of hydrogen spectrum, Bohr's hydrogen theory, and its significance and limitations.
	\item Knowing wave functions and their statistical interpretation. Knowing the one-dimensional stationary Schrodinger equation and its application to one-dimensional infinite potential well.
	\item Knowing angular momentum quantization and spatial quantization, understand Stern-Gerlach experiment and electron spin.
	\item Knowing the four quantum numbers that describe the motion of electrons in an atom, the Pauli exclusion principle, and the atomic shell structure.
\end{itemize}
\noindent{\bfseries8 Elective content}\\
Reference topics:
\begin{itemize}
	\item Laser formation, laser characteristics and its important applications.
	\item Band theory of solids, properties of semiconductors.
	\item Infrared radiation and its application in industry.
	\item Nuclear decay and isotopes and their applications in industry.
\end{itemize}
Teachers may determine the content of lectures according to the development of physics and the latest achievements.
\section{Continuing teaching arrangement}
Teaching hours :51 credit hours +17 continuing hours\\
Teaching hours allocation:\\
\begin{center}
\begin{tabular}{|l|c|c|} %l(left)居左显示 r(right)居右显示 c居中显示
	\hline 
	Content&Credit hours&Continuing hours\\
	\hline  
	1 Electrostatic field in a vacuum&10&2\\
	\hline 
	2 Electrostatic fields in conductors and dielectrics&3&2\\
	\hline 
	3 Magnetic field with constant current&8&2\\
	\hline 
	4 Magnetic field in a magnetic medium&2&0\\
	\hline 
	5 Electromagnetic induction, electromagnetic field&9&3\\
	\hline 
	6 Basis of special relativity&6&1\\
	\hline 
	7 Fundamentals of quantum physics&13&2\\
	\hline 
	8 Elective content&&2\\
	\hline 
	Review&&3\\
	\hline 
	Sum&51&17\\
	\hline 
\end{tabular}
\end{center}
\end{document}
