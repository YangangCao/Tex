\documentclass[10pt,a4paper,oneside]{article}
\usepackage[utf8]{inputenc}
\usepackage{amsmath}
\usepackage{amsfonts}
\usepackage{amssymb}
\usepackage{graphicx}
\usepackage{geometry}
\usepackage{breqn}
\date{}
%\usepackage{draftwatermark} % 设置水印
%\SetWatermarkText{DNV Group} % 水印内容
\usepackage{tikz} % system block diagram
\usepackage{textcomp}
\usetikzlibrary{datavisualization}
\usetikzlibrary{shapes,arrows} % system block diagram
\usepackage{booktabs}
\geometry{right=3.0cm,left=3.0cm,bottom=3.0cm, top=3.0cm}
\begin{document}

\title{\vspace{-3em}Syllabus of University Physics I\vspace{-2em}}
\maketitle 
\section{Course information}
Course name: University Physics I \\
Course object: Undergraduates majoring in science and engineering\\
Course code: SCI4B3B001\\
Course category: Professional basic courses, compulsory\\
Prerequisite course: Advanced mathematics.\\
Total  period: 51+17 (class :51, continue :17)\\
Syllabus author: Xiaobai Chen\\
Syllabus reviewer: Denghui Xu, Aicong Geng\\


Physics is the natural science about the basic structure, basic motion form and interaction. Its basic theory permeates every field of natural science and is applied in many domains of production technology, which is the foundation of other natural science and engineering technology.

In the process of pursuit of truth and exploration of the unknown world, physics has presented a series of scientific world views and methodology, which profoundly affects human's basic understanding of the material world, thinking mode and social life. It is the cornerstone of human civilization development and plays an important role in the cultivation of scientific quality of talents.

University physics I consists of the basic concepts and rules of mechanical, thermal, mechanical vibration, mechanical wave and wave optics, and it's an important general professional basic course for students in major machinery, automation, electrical, computer, software, information, mathematics, chemistry, food, environment, biology, materials science and engineering in our school. The basic concepts, theories and methods taught in this course are important parts of students' scientific literacy, which are necessary for scientists and engineers.

University physics I plays an important role that can not be replaced by other courses in setting up the necessary physical foundation for students systematically, cultivating students' scientific world view, enhancing students' ability to analyze and solve problems, and cultivating students' exploration spirit and innovation consciousness.

Through this course, students should have a systematic and correct understanding of the basic concepts, theories and research methods of mechanics, heat, mechanical vibration, mechanical wave and wave optics in physics, so as to lay a solid foundation for further study. In the course of each section, not only knowledge should be taught, at the same time, but also pay attention to the cultivation of students' ability to analyze and solve problems, the cultivation of students' exploration spirit and innovative consciousness, and strive to achieve the coordinated development of students' knowledge, ability and quality.
\section{Textbooks and teaching resources}
\subsection{Textbooks}
General physics (6th edition), volume 1 and volume 2, edited by Shouzhu Cheng and Zhiyong Jiang, Higher Education Press, 2006.
\subsection{Teaching resources}
\begin{itemize}
\item General physics (6th edition), problem analysis and solution,  Naijiang Sui, Panxin Hu, Higher Education Press, 2006.
\item General physics (6th edition), analysis and development of thought problems, Panxin Hu, et al, Higher Education Press, 2006.
\item Case base of university physics peer teaching(1st edition), Xiaobai Chen, Baohe Li, et al, China Machine Press, June 2014.
\item University physics problem seminar guide(1st edition), Huijun Shen, Huzhu Wang, Tsinghua University Press,  June 1991.
\end{itemize}
\section{Course assessment}
The grades are assessed by 30\% and 70\% respectively according to the normal time and the exam, including attendance, discussion, unit test, etc., and the final exam is a closed-book exam with 100 minutes.
\section{Teaching content and teaching hours allocation }
\subsection{Total teaching hours}
The total teaching hours of this course are 68, including 51 class hours of classroom teaching and 17 class hours of continuous teaching.
\subsection{Content and teaching hours allocation}
{\bfseries Chapter 1 Mechanics (15 credit hours + 6 continuing hours)}\\

\noindent{\bfseries1 Force and motion}\\

Teaching content: Particle, reference system, position vector, displacement, velocity, acceleration, tangential acceleration, normal acceleration, motion equation, angular displacement, angular velocity, angular acceleration, relationship between angular quantity and linear quantity, relative motion. Newton's laws of motion, inertia, mass, force and inertial system.\\

Teaching requirements:
\begin{itemize}
\item Understanding the importance of scientific research methods for establishing particle models.
\item Mastering the position vector, displacement, velocity, acceleration, angular velocity, angular acceleration and other physical quantities describing the motion of particles. The proper reference system and coordinate system can be selected to calculate the velocity and acceleration of the one-dimensional and two-dimensional motion of the particle in rectangular coordinate system
\item  Knowing Galileo's relativity principle, Galileo's coordinate, velocity transformation and relative motion.
\item Mastering the content and application conditions of Newton's three laws. Understanding the concepts of inertia, mass, force and inertial systems. Mastering the method of force analysis by selecting isolator. Understanding the laws of common forces and mastering the basic ideas and methods of solving problems with Newton's laws. Using calculus method to solve the simple dynamics problem of particle under variable force.
\end{itemize}
\noindent{\bfseries2 Conserved quantities of motion and conservation laws}\\

Teaching content: Work of constant force and variable force, power, kinetic energy, kinetic energy theorem, conservative force, non-conservative force, potential energy, principle of work and power, conservation law of mechanical energy. Impulse, momentum, momentum theorem, momentum conservation law, collision. Center-of-mass, the motion theorem of center-of-mass. Angular momentum of particle, theorem of angular momentum, conservation law of angular momentum.\\

Teaching requirements:
\begin{itemize}
	\item Mastering the concepts of momentum and impulse, the momentum theorem and the law of conservation of momentum and its applicable conditions.
	\item Mastering the concepts of work, kinetic energy and potential energy, understanding the characteristics of work of conservative force, calculating the work of variable force in the case of linear motion, skillfully calculating gravitational potential energy, elastic potential energy, calculating attractive potential energy. Mastering the law of kinetic energy, principle of work and power, the law of conservation of mechanical energy and its applicable conditions.
	\item Knowing the concept of the center-of-mass, understanding the motion theorem of the center-of-mass.
	\item Mastering one-dimensional and two-dimensional problems of particle motion. Collisions require only the understanding of perfectly elastic and perfectly nonelastic collisions.
\end{itemize}
\noindent{\bfseries3 Fixed axis rotation of rigid body}\\

Teaching content: Rigid body, translation of rigid body, fixed axis rotation of rigid body, angular momentum, rotational kinetic energy, rotational inertia of rigid body, moment of force, law of rotation, angular momentum theorem and conservation law of angular momentum, precession.\\

Teaching requirements:
\begin{itemize}
	\item Understanding the rigid body model, knowing the concept of rotational inertia.
	\item Mastering the concept of moment of force, understanding the law of fixed axis rotation of rigid body and its applicable conditions.
	\item Understanding the concept of angular momentum, the angular momentum theorem and the conservation law of angular momentum and its applicable conditions.
	\item Knowing the concept of precession
\end{itemize}
{\bfseries Chapter 2 Thermology (12 credit hours + 4 continuing hours)}\\

\noindent{\bfseries4 Kinetic theory of gas}\\

Teaching content: The state parameters of the gas, the equilibrium state and equilibrium process, the state equation of ideal gas , the formula of pressure and temperature of ideal gas, theorem of energy according to the degrees of freedom equipartition, the internal energy of ideal gas, Maxwell speed distribution, the average rate, square root rate, most probable rate, average collision frequency and the mean free path of gas.\\

Teaching requirements:
\begin{itemize}
	\item Understanding the concept of equilibrium state.
	\item Expounding the statistical method, explaining the difference and connection between macroscopic quantity and microcosmic quantity, understanding the macroscopic property of thermodynamic system is the statistical expression of molecular thermal motion. Mastering the essence and statistical significance of concepts such as pressure, temperature and internal energy. Understanding the pressure and temperature formula.
	\item  Understanding the molecular model of an ideal gas, the average energy of gas molecules according to the equalization theorem of degrees of freedom, mastering the internal energy of ideal gas.
	\item Knowing the physical significance of Maxwell rate distribution rate, distribution function and rate distribution curve. Understanding the meaning and application of arithmetic mean rate, square root rate, most probable rate of molecular motion.
	\item Knowing the concept of average collision frequency and average free path of gas molecules.
\end{itemize}
\noindent{\bfseries5 Thermodynamic basis}\\

Teaching content: Internal energy, work, heat, the first law of thermodynamics and its application in four quasi-static processes of equal volume, equal pressure, equal temperature and adiabatic of ideal gas, molar heat capacity of a gas, cyclic process, Carnot cycle, Carnot theorem, thermal efficiency, the second law of thermodynamics, reversible and irreversible processes, entropy and entropy increasing principle.\\

Teaching requirements:
\begin{itemize}
	\item Mastering the concepts of internal energy, work and heat, understanding the quasi-static process, mastering the first law of thermodynamics and its application. mastering the calculation of work, heat and internal energy changes in equal volume, equal pressure, equal temperature and adiabatic of ideal gas. Aanalyzing and calculating the simple cycle efficiency of ideal gas.
	\item Knowing two expressions of the second law of thermodynamics and their equivalence, knowing reversible and irreversible processes.
	\item  Knowing the statistical significance of the second law of thermodynamics, the Boltzmann expressions of entropy and entropy and the principle of entropy increase.
\end{itemize}
{\bfseries Chapter 3 Mechanical vibrations and mechanical waves (12 credit hours + 3 continuing hours)}\\

\noindent{\bfseries6 Mechanical vibration}\\

Teaching content: Equation of simple harmonic vibration, amplitude, period, frequency, circular frequency, phase, rotation vector method, energy of simple harmonic vibration, damped vibration, forced vibration, resonance, synthesis of harmonic vibration in same and different frequencies in the same direction.\\

Teaching requirements:
\begin{itemize}
	\item Mastering the basic characteristics of simple harmonic vibration and the concepts of amplitude, period, frequency, circular frequency and phase. One dimensional differential equation of simple harmonic vibration can be established. The motion equation of one-dimensional harmonic vibration can be obtained according to the initial conditions.
	\item Understanding the rotation vector method, using the rotation vector to analyze the simple harmonic vibration problem.
	\item Knowing the phenomena of damped vibration, forced vibration and resonance.
	\item Understanding the law of simple harmonic vibration synthesis in the same direction and frequency, and knowing the beat phenomenon.
\end{itemize}
\noindent{\bfseries7 Mechanical wave}\\

Teaching content: Production and propagation of mechanical wave in elastic medium, transverse  and longitudinal wave, velocity, frequency, wavelength and their relationship, planar simple harmonic wave equation, energy of  mechanical wave, energy flow and energy flow density, Huygens' principle, the diffraction of the wave, principle of wave superposition, phase difference, half wave loss, wave interference, standing wave, the Doppler effect.\\

Teaching requirements:
\begin{itemize}
	\item Understanding the conditions under mechanical waves arise and propagate in an elastic medium, mastering the concept of waves, understanding the transverse  and longitudinal wave, mastering the relationship among wave speed, frequency and wavelength, understanding the wave function, establishing the wave function of planar simple harmonic according to the known conditions.
	\item Understanding the physical meaning and waveform curves of wave functions.
	\item Understanding the laws of energy propagation and change in mechanical waves, knowing the concept of energy flow and energy density.
	\item Knowing Huygens' principle.
	\item Understanding the principle of superposition and the coherent conditions of wave. The enhancement and attenuation conditions of coherent superposition can be determined by phase difference and wave path difference analysis.
	\item Knowing the formation conditions of standing waves and the characteristics of standing waves, the positions of nodal and ventral of waves can be determined, understanding half wave loss.
	\item Knowing the Doppler effect and its causes, Doppler frequency movement formula can be used to calculate when the wave source or observer moves separately with respect to the medium and the direction of motion is along the line between them.
\end{itemize}
{\bfseries Chapter 4 Wave optics (12 credit hours + 2 continuing hours)}\\

\noindent{\bfseries8 Interference of light}\\

Teaching content: Light source, monochromatism and coherence of light, acquisition of coherent light, Young's double slit interference, optical path, optical path difference, equal thickness interference, equal dip interference, Michelson interferometer.\\

Teaching requirements:
\begin{itemize}
	\item Knowing the properties of light source, understanding the monochromatism and coherence of light, and how to obtain coherent light.
	\item Mastering the coherence conditions, understanding half wave loss, mastering the concept of optical path and the relationship between optical path difference and phase difference.
	\item Analyzing Young's double slit interference and equal thickness interference, knowing equal inclination interference and the principle of Michelson interferometer.
\end{itemize}
\noindent{\bfseries9 Diffraction of light}\\

Teaching content: Huygens-Fresnel principle, single slit diffraction, grating diffraction, resolution of optical instruments, X-ray diffraction.\\

Teaching requirements:
\begin{itemize}
	\item Knowing Huygens-Fresnel principle, understanding the law of single slit Fraunhofer diffraction, knowing the method of half-wave band method to analyze the light and dark of the stripe, analyzing the influence of the width and wavelength of the slit on the diffraction fringe.
	\item Understanding the law of grating diffraction and the condition of missing order, the position of grating diffraction spectral line can be determined, and the influence of grating constant and wavelength on the distribution of grating diffraction spectral line can be analyzed..
	\item Knowing the resolution of optical instruments and X-ray diffraction.
\end{itemize}
\noindent{\bfseries10 Polarization of the light}\\

Teaching content: Natural and polarized light, acquisition and examination of linear polarized light, polarization of reflected and refracted light, Brewster's law, Marius' law\\

Teaching requirements:
\begin{itemize}
	\item Understanding the concepts and basic characteristics of natural and linearly polarized light, understanding the generation and examination of linearly polarized light.
	\item Understanding Brewster's law and Marius' law.
\end{itemize}
\section{Continuing teaching arrangement}
Teaching hours :51 credit hours +17 continuing hours\\
Teaching hours allocation:\\
\begin{center}
\begin{tabular}{|l|c|c|} %l(left)居左显示 r(right)居右显示 c居中显示
	\hline 
	Content&Credit hours&Continuing hours\\
	\hline  
	1 Force and motion&4&2\\
	\hline 
	2 Conserved quantities of motion and conservation laws&5&2\\
	\hline 
	3 Fixed axis rotation of rigid body&6&2\\
	\hline 
	4 Kinetic theory of gas&5&2\\
	\hline 
	5 Thermodynamic basis&7&2\\
	\hline 
	6 Mechanical vibration&5&1\\
	\hline 
	7 Mechanical waven&7&2\\
	\hline 
	8 Interference of light&12&2\\
	\hline 
	Review&&2\\
	\hline 
	Sum&51&17\\
	\hline 
\end{tabular}
\end{center}

\end{document}
