\documentclass[10pt,a4paper,oneside]{article}
\usepackage[utf8]{inputenc}
\usepackage{draftwatermark} % 设置水印
\SetWatermarkText{DNV Group} % 水印内容
\usepackage{amsmath}
\usepackage{amsfonts}
\usepackage{amssymb}
\usepackage{graphicx}
\usepackage{breqn}
\usepackage{tikz} % system block diagram
\usepackage{textcomp}
\usetikzlibrary{shapes,arrows} % system block diagram
\usepackage{booktabs}
\usepackage[framed,numbered,autolinebreaks,useliterate]{mcode} % matlab code block
\author{Yangang Cao}
\date{February 20, 2019}
\newcommand{\degree}{^\circ}
\tikzset{
	delay/.style    = {draw, thick, rectangle, minimum height = 3em,
		minimum width = 3em},
	sum/.style      = {draw, circle, node distance = 2cm}, 
	prod/.style     = {draw, circle, node distance = 2cm},
	input/.style    = {coordinate}, % Input
	output/.style  = {coordinate} % Output
}
% Defining string as labels of certain blocks.
\newcommand{\product}{$\displaystyle \times$}
\newcommand{\delay}{\large$z^{-1}$}
\begin{document}

\title{Wah-Wah Filter Design}
\maketitle 
The parametric filters discussed in the previous documents allow the time-varying control of the filter parameters gain, cut-off frequency, and bandwidth or Q factor. Special applications of time-varying audio filters play important roles in music signal processing, one of them is wah-wah filter.

The wah-wah effect is produced mostly by foot-controlled signal processors containing a bandpass filter with variable center frequency and a small bandwidth. Moving the pedal back and forth changes the bandpass center frequency. The “wah-wah” effect is then mixed with the direct signal. This effect leads to a spectrum shaping similar to speech and produces a speech-like “wah-wah” sound. 

Instead of manually changing the center frequency, it is also possible to let a low-frequency oscillator control the center frequency, which in turn is controlled based on parameters derived from the input signal. Such an effect is called an auto-wah filter. If the effect is combined with a low-frequency amplitude variation, which produces a tremolo, the effect is denoted a tremolo-wah filter. Replacing the unit delay in the bandpass filter by an M tap delay leads to the M-fold wah-wah filter. M bandpass filters are spread over the entire spectrum and simultaneously change their center frequency. Effects with M-fold wah-wah filter are shown in following table.
\begin{center}
	\begin{tabular}{cccc}
		\toprule  %添加表格头部粗线
		& {$M$}&{$Q^{-1}/f_m$}&{$\Delta f$}\\
		\midrule  %添加表格中横线
		Wah-Wah&1& --/3kHz & 200Hz\\
		$M$-fold Wah-Wah& 5-20 & 0.5/-- & 200-500Hz\\
		Bell effect&100&0.5/--&100Hz\\
		\bottomrule %添加表格底部粗线
	\end{tabular}
\end{center}
Based on the measured shape of the amplitude response (a bandpass-resonator characteristic), and knowledge (from circuit schematics), the transfer function of the second-order bandpass can be presumed to be of the form
\[
H(s) = g\frac{s-\xi}{\left(\frac{s}{\omega_r}\right)^2+\frac{2}{Q}\left(\frac{s}{\omega_r}\right)+1},
\]
where $g$ is an overall gain factor, $\xi$ is a real zero at or near dc (the other being at infinity), $\omega_r$ is the pole resonance frequency, and $Q$ is the so-called “quality factor” of the resonator. The measurements reveal that $\omega_r$, $Q$, and $g$ all vary significantly with pedal angle $\theta$.Good choices for these functions are as shown in following code block, where the controlling wah variable is the pedal-angle θ normalized to a [0, 1] range.

\end{document}
